\documentclass[a4paper,
fontsize=11pt,
%headings=small,
oneside,
numbers=noperiodatend,
parskip=half-,
bibliography=totoc,
final
]{scrartcl}

\usepackage[babel]{csquotes}
\usepackage{synttree}
\usepackage{graphicx}
\setkeys{Gin}{width=.4\textwidth} %default pics size

\graphicspath{{./plots/}}
\usepackage[ngerman]{babel}
\usepackage[T1]{fontenc}
%\usepackage{amsmath}
\usepackage[utf8x]{inputenc}
\usepackage [hyphens]{url}
\usepackage{booktabs} 
\usepackage[left=2.4cm,right=2.4cm,top=2.3cm,bottom=2cm,includeheadfoot]{geometry}
\usepackage{eurosym}
\usepackage{multirow}
\usepackage[ngerman]{varioref}
\setcapindent{1em}
\renewcommand{\labelitemi}{--}
\usepackage{paralist}
\usepackage{pdfpages}
\usepackage{lscape}
\usepackage{float}
\usepackage{acronym}
\usepackage{eurosym}
\usepackage{longtable,lscape}
\usepackage{mathpazo}
\usepackage[normalem]{ulem} %emphasize weiterhin kursiv
\usepackage[flushmargin,ragged]{footmisc} % left align footnote
\usepackage{ccicons} 
\setcapindent{0pt} % no indentation in captions

\usepackage{enumitem}

%%%% fancy LIBREAS URL color 
\usepackage{xcolor}
\definecolor{libreas}{RGB}{112,0,0}

\usepackage{listings}

\urlstyle{same}  % don't use monospace font for urls

\usepackage[fleqn]{amsmath}

%adjust fontsize for part

\usepackage{sectsty}
\partfont{\large}

%Das BibTeX-Zeichen mit \BibTeX setzen:
\def\symbol#1{\char #1\relax}
\def\bsl{{\tt\symbol{'134}}}
\def\BibTeX{{\rm B\kern-.05em{\sc i\kern-.025em b}\kern-.08em
    T\kern-.1667em\lower.7ex\hbox{E}\kern-.125emX}}

\usepackage{fancyhdr}
\fancyhf{}
\pagestyle{fancyplain}
\fancyhead[R]{\thepage}

% make sure bookmarks are created eventough sections are not numbered!
% uncommend if sections are numbered (bookmarks created by default)
\makeatletter
\renewcommand\@seccntformat[1]{}
\makeatother

% typo setup
\clubpenalty = 10000
\widowpenalty = 10000
\displaywidowpenalty = 10000

\usepackage{hyperxmp}
\usepackage[colorlinks, linkcolor=black,citecolor=black, urlcolor=libreas,
breaklinks= true,bookmarks=true,bookmarksopen=true]{hyperref}
\usepackage{breakurl}

%meta
%meta

\fancyhead[L]{T. Schumann\\ %author
LIBREAS. Library Ideas, 38 (2020). % journal, issue, volume.
\href{https://doi.org/10.18452/23474}{\color{black}https://doi.org/10.18452/23474}
{}} % doi 
\fancyhead[R]{\thepage} %page number
\fancyfoot[L] {\ccLogo \ccAttribution\ \href{https://creativecommons.org/licenses/by/4.0/}{\color{black}Creative Commons BY 4.0}}  %licence
\fancyfoot[R] {ISSN: 1860-7950}

\title{\LARGE{\enquote{It‘s the end of the world as we know it}}}% title
\subtitle{\Large{Ein Essay über Öffentliche Bibliotheken als zentrale lokale Einrichtung, dem Klimakollaps zu begegnen und ein Werkstattbericht aus der Heinrich-Böll-Bibliothek / Stadtbibliothek Pankow}}
\author{Tim Schumann} % author

\setcounter{page}{1}

\hypersetup{%
      pdftitle={"It‘s the end of the World as we know it". Ein Essay über Öffentliche Bibliotheken als zentrale lokale Einrichtung, dem Klimakollaps zu begegnen und ein Werkstattbericht aus der Heinrich-Böll-Bibliothek / Stadtbibliothek Pankow},
      pdfauthor={Tim Schumann},
      pdfcopyright={CC BY 4.0 International},
      pdfsubject={LIBREAS. Library Ideas, 38 (2020).},
      pdfkeywords={Öffentliche Bibliothek, Klimawandel, Klimakrise, Libraries4Future},
      pdflicenseurl={https://creativecommons.org/licenses/by/4.0/},
      pdfcontacturl={http://libreas.eu},
      baseurl={http://libreas.eu},
      pdflang={de},
      pdfmetalang={de}
     }



\date{}
\begin{document}

\maketitle
\thispagestyle{fancyplain} 

%abstracts

%body
\hypertarget{eine-etwas-luxe4ngere-einleitung}{%
\section{1. (Eine etwas längere)
Einleitung}\label{eine-etwas-luxe4ngere-einleitung}}

Der Klimawandel prägt inzwischen den medialen Alltag. Spätestens mit der
extremen Dürre in Mitteleuropa ist das Thema auch in der Gesellschaft in
Deutschland angekommen. So stellen Bilder von verdorrten Feldern,
ausgetrockneten Flüssen oder sterbenden Wäldern inzwischen eine neue
Normalität dar. Die heißesten und trockensten Jahre liegen seit Beginn
der Wetteraufzeichnung, mit wenigen Ausnahmen, fast alle in den 2000er
Jahren.\footnote{\url{https://www.wetter.de/cms/klimawandel-die-10-heissesten-sommer-in-deutschland-4571689.html},
  (letzter Zugriff: 11.10.2020). sowie
  \url{https://de.statista.com/statistik/daten/studie/157755/umfrage/klimawandel---die-weltweit-waermsten-jahre-seit-1880/},
  (letzter Zugriff: 11.10.2020).} Zudem werden die von
Klimaforscher*innen entworfene Worst-Case-Szenarien bereits jetzt schon
erreicht oder sogar übertroffen.\footnote{\url{https://www.theguardian.com/environment/2020/jun/13/climate-worst-case-scenarios-clouds-scientists-global-heating?CMP=twt_a-environment_b-gdneco},
  (letzter Zugriff: 11.10.2020). Zur Dürre:
  \url{https://www.ufz.de/index.php?de=37937}, (letzter Zugriff:
  11.10.2020). Zu den \enquote*{Tipping Points}:
  \url{https://www.wetterdienst.de/Deutschlandwetter/Thema_des_Tages/3799/tipping-points-die-kipp-elemente-im-klimasystem},
  (letzter Zugriff: 11.10.2020).}

\hypertarget{zielsetzung-des-artikels}{%
\subsection{Zielsetzung des Artikels}\label{zielsetzung-des-artikels}}

Der vorliegende Beitrag versucht, sich mit der Rolle von Öffentlichen
Bibliotheken in diesen Entwicklungen auseinanderzusetzen. Dabei wird
nicht nur hinterfragt, warum sich Öffentliche Bibliotheken mit diesem
Thema auseinandersetzen sollten. Es werden auch die Potentiale
aufgezeigt, die sich für Öffentliche Bibliotheken in Kombination mit
ihrer Rolle als \enquote*{3. Ort} und einer Ausrichtung auf soziale und
ökologische Nachhaltigkeit ergeben. Daraus wird dann eine Utopie
entwickelt, die etwas spielerisch aufzeigt, was im Jahr 2030 möglich
sein könnte.

Das bedeutet, dass der Beitrag eher die Form ein Essays aufweist. Zudem
wird auch ein polemischer Stil benutzt, um aus der Sicht des Autors die
Rolle von Öffentlichen Bibliotheken unter dem Blickpunkt des
Klimawandels neu/anders zu denken.

Der Verfasser schreibt dabei aus der Position als Leitung einer größeren
Öffentlichen Bibliothek in Berlin-Pankow, als Mitglied des Netzwerks
Grüne Bibliothek und als Unterstützer der Libraries4Future-Idee. Daher
folgt der Artikel auch dem Schema, die gesellschaftlichen und
politischen Entwicklungen in Bezug auf den Klimawandel im Allgemeinen
kurz auszuführen, anschließend auf die Idee der Grünen Bibliothek
einzugehen, um zum Abschluss eine Art \enquote*{Werkstattbericht} aus
der Heinrich-Böll-Bibliothek wiederzugeben. Dort befinden sich derzeit
mehrere Projekte in der Planungs- oder Umsetzungsphase, die die
\enquote*{Böll} in eine \enquote*{Grüne Bibliothek} umwandeln sollen.

\hypertarget{klimawandel-klimakrise-klimakollaps}{%
\subsection{Klimawandel? Klimakrise?
Klimakollaps?}\label{klimawandel-klimakrise-klimakollaps}}

Eine weitere Positionierung vorweg: Der Verfasser geht von einer
Entwicklung aus, die mit dem Wort Klimakrise nicht mehr ausreichend
beschrieben werden kann. Vielmehr droht ein Klimakollaps(!) mit
gravierenden gesellschaftlichen Folgen, wie zum Beispiel der Gefährdung
der bisherigen Lebensweise in Deutschland, der Gefährdung der Demokratie
oder Gefährdung der Zivilisation, wie wir sie bisher kennen und leben,
im Allgemeinen.\footnote{So beschreibt Andri Snaer Magnason in seinem
  aktuellen Buch \enquote{Wasser und Zeit -- eine Geschichte unserer
  Zukunft} die dramatischen und gefährlichen Auswirkungen der
  Gletscherschmelze auf die Wasserversorgung. (Insel, 2020.) Zudem lehnt
  sich der vorliegende Beitrag auch an die Beschreibungen des Buches von
  David Wallace-Wells: Die unbewohnbare Erde -- Leben nach der
  Erderwärmung, Ludwig, 2019 an. Eine der zentralen Thesen von
  Wallace-Wells lautet, dass mit dem 2-Grad-Ziel bereits massive
  Veränderungen eintreten werden und dass wir derzeit weit davon
  entfernt sind, dieses Ziel überhaupt zu erreichen: \enquote{Two
  degrees of warming used to be considered the threshold of catastrophe:
  tens of millions of climate refugees unleashed upon an unprepared
  world. Now two degrees is our goal, per the Paris climate accords, and
  experts give us only slim odds of hitting it.}
  \url{https://nymag.com/intelligencer/2017/07/climate-change-earth-too-hot-for-humans.html?abcid=intel-test-4-16\&abv=1},
  (letzter Zugriff: 11.10.2020).}

Auch weniger pessimistische Vertreter*innen sprechen von den
2020er-Jahren als der \enquote{Dekade der Entscheidungen} oder dem
\enquote{Jahrzehnt der Entscheidungen}, in dem zentrale
gesellschaftliche und politische Veränderungen bevorstünden.\footnote{\url{http://www.eco-world.de/scripts/shop.prg?img=/eco-world-buecher/doc/images/FNW_2020_01_Inhalt.pdf},
  (letzter Zugriff: 11.10.2020).
  \url{https://www.blaetter.de/ausgabe/2019/dezember/hellsicht-in-zeiten-des-umbruchs},
  (letzter Zugriff: 11.10.2020).}

Beide Sichtweisen, die sehr pessimistische sowie die nicht so
pessimistische, lassen dennoch nur einen gesamt-gesellschaftlichen
Schluss zu: Eine massive Umstellung des individuellen und
gesamt-gesellschaftlichen Lebenswandels ist nötig. So ruft zum Beispiel
der Journalist Bernd Ulrich das \enquote{Zeitalter der Ökologie} aus, an
dem wir als Gesellschaft gar nicht vorbeikommen können.\footnote{Vergleiche
  Bernd Ulrich: Alles wird anders : das Zeitalter der Ökologie, Köln :
  Kiepenheuer \& Witsch, 2019.} Zudem stellt die Sichtweise von Bruno
Latour eine wichtige Grundlage war, kommende gesellschaftliche und
politische Prozesse richtig zu deuten: \enquote{Alles, was uns
gegenwärtig beunruhige -- sei es Migration, wachsende Ungleichheit oder
Populismus -- habe eine gemeinsame Wurzel in der unheimlichen Erfahrung,
dass die Erde in Form des Klimawandels plötzlich auf unsere Handlungen
reagiert.}\footnote{\url{https://www.deutschlandfunkkultur.de/bruno-latour-das-terrestrische-manifest-die-menschheit-hat.1270.de.html?dram:article_id=423856},
  (letzter Zugriff: 11.10.2020). Vergleiche Bruno Latour: Das
  terrestrische Manifest, Berlin : Suhrkamp, 2018.} Eine Lösung für
diese Probleme sieht Latour in der Abwendung vom Paradigma der
Globalisierung hin zu einer neuen Zuwendung zum Lokalen.

Inzwischen gibt es eine Vielzahl an gesellschaftlichen und politischen
Entwicklungen, die versuchen, auf diese negativen Entwicklungen
einzugehen. Die bekannteste ist sicherlich die aktuelle
Fridays4Future-Bewegung, die inzwischen viele lokale oder berufsbezogene
Ableger hat. Zudem gibt es zahlreiche politische Entwicklungen und
Programme auf globaler, nationaler oder regionaler Ebene, die ebenso
versuchen, sich dem Thema des Klimawandels beziehungsweise des
Klimakollaps anzunehmen. So existiert auf globaler Ebene gegenwärtig die
\enquote*{UN-Agenda 2030}, die 17 Nachhaltigkeitsziele
formuliert.\footnote{\url{https://www.un.org/sustainabledevelopment/development-agenda/},
  (letzter Zugriff: 11.10.2020) sowie
  \url{https://de.wikipedia.org/wiki/Ziele_f\%C3\%BCr_nachhaltige_Entwicklung},
  (letzter Zugriff: 11.10.2020).} Die Europäische Union hat im Rahmen
eines \enquote*{Green New Deal} das Ziel ausgegeben, bis ins Jahr 2050
klimaneutral zu sein,\footnote{\url{https://ec.europa.eu/info/strategy/priorities-2019-2024/european-green-deal_en},
  (letzter Zugriff: 11.10.2020) sowie
  \url{https://en.wikipedia.org/wiki/Green_New_Deal}, (letzter Zugriff:
  11.10.2020).} während in anderen Staaten ein ähnliches Programm
diskutiert wird. Auf nationaler und lokaler Ebene in Deutschland kommen
eine Vielzahl von Maßnahmen zur Verkehrswende oder zur ökologischen
Stadtentwicklung dazu. Auch beim Deutschen Städtebund ist das Thema
Klimawandel ein zentrales.\footnote{Vergleiche
  \url{https://www.dstgb.de/dstgb/Homepage/Schwerpunkte/Klimaschutz/},
  (letzter Zugriff: 11.10.2020).}

\hypertarget{die-muxf6gliche-rolle-von-uxf6ffentlichen-bibliotheken-zwei-thesen}{%
\subsection{Die mögliche Rolle von Öffentlichen Bibliotheken -- zwei
Thesen}\label{die-muxf6gliche-rolle-von-uxf6ffentlichen-bibliotheken-zwei-thesen}}

Für den Verfasser dieses Beitrags stellen sich aus diesen Entwicklungen
heraus die Fragen, ob und wie Öffentliche Bibliotheken darauf reagieren
können oder sollten. Die Beantwortung dieser Fragen steht dabei im
Zusammenhang mit dem derzeitig diskutierten Wandel Öffentlicher
Bibliotheken hin zu \enquote*{Dritten Orten} oder hin zu Plattformen für
die lokale Zivilgesellschaft.

Einer Kritik daran, dass Öffentliche Bibliotheken hier ihr eigentliches
Aufgabengebiet verlassen, wird in diesem Beitrag auch deutlich
widersprochen. Richtet man den Blick auf gesellschaftliche Mega-Themen
wie Migration, das Altern der Gesellschaft oder die Digitalisierung,
fällt schnell auf, dass Öffentliche Bibliotheken auf diesen Gebieten
(zurecht) sehr aktiv sind. Da der Klimawandel beziehungsweise
Klimakollaps ein weiteres solches Mega-Thema darstellt, dürfen sich
Öffentliche Bibliotheken diesem Thema nicht verschließen! Daraus ergeben
sich zwei Thesen:

\textbf{These 1:}

Es reicht nicht aus, dass Bibliotheken sich als bereits nachhaltig
agierende Institutionen bezeichnen, auch wenn sie bereits einige
Merkmale (im Sinne der UN-2030-Agenda und der 17 Ziele) aufweisen, da
Raum, Ressourcen und Infrastruktur von vielen Menschen geteilt werden.
So müssen zum Beispiel auch Bibliotheken ihren eigenen Fußabdruck
überprüfen und deutlich verringern. Dabei müssen interne und externe
Prozesse hinterfragt werden um Ressourcen einzusparen.\footnote{Dem
  Verfasser ist es bewusst, dass das keine leichte Aufgabe im Rahmen des
  öffentlichen Dienstes ist. Umstände des öffentlichen Dienstes
  erschweren Veränderungen auf diesem Gebiet sehr stark.}

\textbf{These 2:}

Der Fokus auf Bücher/Medien reicht nicht mehr aus, um Wissen und
Informationen zu transportieren. Spätestens mit dem Blick auf die
Aufgaben \enquote*{Grüner Bibliotheken} muss der Fokus stark auf den
Menschen im lokalen Umfeld und deren Bedürfnissen im Bereich der lokalen
klimatischen Veränderungen liegen.

Eine \enquote*{Grüne Bibliothek} muss hier ganz im Sinne des
4-Räume-Modells agieren und als Ort der Inspiration und Innovation (die
Menschen begeistern), als Treffpunkt und Ort der Begegnung (mitmachen
und Beteiligung), als Lernraum (Raum für Entdeckungen) und als
performativer Raum (Raum für gemeinsame Aktion und des
zivilgesellschaftlichen Engagements sowie der Kreation) weiter
entwickelt werden.\footnote{Jochumsen, Henrik (et al.): Erlebnis,
  Empowerment, Beteiligung und Innovation: die neue Öffentliche
  Bibliothek. In: Eigenbrodt, Olaf: Formierung von Wissensräumen :
  Optionen des Zugangs zu Information und Bildung, S. 67--80.}

Da es keine vergleichbaren Bildungs- und Kultureinrichtung gibt, die
eine ähnliche lokale Breitenwirkung entwickeln können wie Öffentliche
Bibliotheken, nehmen diese eine zentrale Rolle bei der Bewältigung des
Klimawandels/des Klimakollaps ein!

\hypertarget{gruxfcne-bibliotheken-und-die-17-ziele-der-un-agenda-2030}{%
\section{2. Grüne Bibliotheken und die 17 Ziele der UN-Agenda
2030}\label{gruxfcne-bibliotheken-und-die-17-ziele-der-un-agenda-2030}}

Da in den letzten Jahren einige deutschsprachige Artikel und
Veröffentlichung erschienen sind, die die Idee der \enquote*{Grünen
Bibliothek} und sozial und ökologisch nachhaltigen Bibliotheksarbeit
näher darstellen, wird hier auf eine tiefergehende Beschreibung
verzichtet.\footnote{Petra Hauke (Hrsg.) (et al.): The Green Library -
  Die grüne Bibliothek: the challenge of environmental sustainability -
  Ökologische Nachhaltigkeit in der Praxis,
  (\href{https://www.degruyter.com/view/serial/IFLA-B}{IFLA Publications
  : 161}), Berlin : De Gruyter Saur, 2013, online unter:
  \url{https://edoc.hu-berlin.de/handle/18452/60}, (letzter Zugriff:
  04.10.2020). \textbar{} Petra Hauke (Hrsg.) (et al.): Going Green:
  Implementing Sustainable Strategies in Libraries Around the World ;
  Buildings, Management, Programmes and Services, IFLA Publications :
  177), Berlin : De Gruyter Saur, 2018. \textbar{} Petra Hauke (Hrsg.)
  (et al.): Öffentliche Bibliothek 2030: Herausforderungen -- Konzepte
  -- Visionen, Bad Honnef : Bock \& Herchen, 2019. Online unter:
  \url{https://edoc.hu-berlin.de/handle/18452/20799}, (letzter Zugriff:
  04.10.2020). \textbar{} Siehe auch den Abschnitt \enquote{Bibliothek
  als \enquote*{Grüner} Ort in der Festschrift für Petra Hauke. Umlauf,
  Konrad (et al.) (Hrsg.): Strategien für die Bibliothek als Ort.
  Festschrift für PetraHauke, Berlin: DeGruyter Saur,2017. \textbar{}
  Siehe auch die Ausgabe 12/2018 der Zeitschrift}Buch und Bibliothek",
  das den thematischen Schwerpunkt Nachhaltige Entwicklung hat.
  Berufsverband Information Bibliothek (Hrsg.): Buch und Bibliothek :
  Forum Bibliothek und Information, Reutlingen, Heft 12/2018, online
  unter: \url{https://www.b-u-b.de/wp-content/uploads/2018-12.pdf},
  (letzter Zugriff: 04.10.2020).}

Wichtig für diesen Beitrag ist jedoch die Ansicht des Verfassers, dass
es bisher noch nicht gelungen ist, eine genaue Definition auszuarbeiten,
was eine \enquote*{Grüne Bibliothek} eigentlich genau ausmacht. Vielmehr
gibt es mehrere Vorschläge, wobei sich der vorliegende Beitrag an die
Definition von Harri Sahavirta anlehnt. Sahavirta erweitert unter
anderem den Begriff einer \enquote*{green library} hin zu einer
\enquote*{sustainable library}. In dieser Erweiterung schlägt er vor,
\enquote{{[}that{]} we should define sustainable libraries as
responsible, respective and reactive.}\footnote{Sahavirta, Harri (2017):
  From green to sustainable libraries - widening the concept of green
  library. In: Konrad Umlauf (et al.) (Hrsg.): Strategien für die
  Bibliothek als Ort. Festschrift für Petra Hauke zum 70. Geburtstag.
  Unter Mitarbeit von Petra Hauke. Berlin: De Gruyter Saur, S. 127--137.
  Hier S. 129--130. Nach Ansicht des Verfassers weist der Vorschlag von
  Sahavirta viele Parallelen zum 4-Räume-Modell auf. Nur Bibliotheken,
  die sich selbst in Rahmen des 4-Räume-Modells denken, sind in der
  Lage, sich zu einer \enquote*{sustainable library} weiter zu
  entwickeln und Anforderungen entgegen zu kommen, die in Kapitel 1
  beschrieben werden.}

In den letzten Jahren entstanden mehrere Initiativen, die diese
Entwicklungen aufgriffen und verstärkten. Dazu zählen zum Beispiel das
\enquote*{Netzwerk Grüne Bibliothek}, das im Januar 2018 gegründet wurde
sowie die Initiative \enquote*{Libraries4Future}, die sich im Sommer
2019 zusammenfand.\footnote{\url{https://www.netzwerk-gruene-bibliothek.de/},
  \url{https://www.facebook.com/NetzwerkGrueneBibliothek}, sowie
  \url{https://libraries4future.org/}, (letzte Zugriffe: 04.10.2020).}
Auf Verbandsebene hat sich auch der Deutsche Bibliotheksverband des
Themas im Rahmen der \enquote*{Agenda 2030} angenommen und ermutigt alle
Bibliotheken, auf dem Gebiet im Rahmen der \enquote*{17 Ziele} aktiv zu
werden.\footnote{\url{https://www.bibliotheksverband.de/dbv/themen/agenda-2030.html},
  (letzter Zugriff: 04.10.2020). Die Internetseite \enquote*{Biblio2030}
  sammelt Beispiele aus Bibliotheken im deutsch-sprachigen Raum, die
  Projekte und Programme im Rahmen der \enquote*{17 Ziele} unternehmen.
  \url{https://www.biblio2030.de/}, (letzter Zugriff: 04.10.2020).}

Eine der aktuellsten regionalen Entwicklungen stellt die
Bibliotheksentwicklungsplanung für Berlin dar. Dort wurde im
gegenwärtigen Entwurf das Thema Nachhaltigkeit als eins von fünf
Mega-Themen festgelegt, in dem die Bibliotheken in Berlin aktiv werden
sollen.\footnote{\url{https://mein.berlin.de/text/chapters/6460/},
  (letzter Zugriff: 04.10.2020).}

Auf politischer Ebene können \enquote*{Grüne Bibliotheken} ihre
Aktivitäten im Bereich der sozialen und ökologischen Nachhaltigkeit
inzwischen sehr gut mit den \enquote*{17 Zielen} der UN-Agenda 2030
sichtbar machen.\footnote{Die UN-Agenda 2030 denkt soziale, ökologische
  und wirtschaftliche Nachhaltigkeit immer in Kombination. Vergleiche
  \url{https://www.un.org/sustainabledevelopment/development-agenda/},
  (letzter Zugriff: 04.10.2020). Auch die Bundesregierung hat sich den
  Zielen der \enquote*{Agenda 2030} in ihrer Nachhaltigkeitsstrategie
  verpflichtet. Vergleiche
  \url{https://www.bundesregierung.de/breg-de/themen/nachhaltigkeitspolitik/agenda-2030-die-17-ziele},
  (letzter Zugriff: 04.10.2020). Für konkrete Beispiele siehe auch:
  \url{https://17ziele.de/}, (letzter Zugriff: 04.10.2020).} Die Idee
der \enquote*{Grünen Bibliothek} kann also mit gegenwärtigen politischen
und gesellschaftlichen Entwicklungen verknüpft werden. Mit etwas Mut
kann vielleicht sogar behauptet werden, dass die 2020er Jahre das
\enquote*{Jahrzehnt der Grünen Bibliotheken} werden kann, wenn die
UN-Agenda 2030 weiter an Wirkmächtigkeit zunimmt.

\hypertarget{die-buxf6ll-auf-dem-weg-zu-einer-gruxfcnen-bibliothek}{%
\section{\texorpdfstring{3. Die \enquote*{Böll} auf dem Weg zu
einer Grünen
Bibliothek}{3. Die `Böll' auf dem Weg zu einer Grünen Bibliothek}}\label{die-buxf6ll-auf-dem-weg-zu-einer-gruxfcnen-bibliothek}}

Die Heinrich-Böll-Bibliothek ist einer von acht Standorten der
Stadtbibliothek Berlin-Pankow. Wie wahrscheinlich jede Öffentliche
Bibliothek in Deutschland ist die \enquote*{Böll} eine Bibliothek, die
sich im Umbruch befindet. Umbruch bedeutet, dass die Bestandspolitik neu
gedacht wurde (für die gesamte Stadtbibliothek Pankow) und der Fokus
stärker auf den Raum und die Menschen gelegt wird. Zudem können durch
die SIWANA-Mittel des Landes Berlin (Sondervermögen Infrastruktur der
Wachsenden Stadt und Nachhaltigkeitsfonds) die beiden Etagen der
Heinrich-Böll-Bibliothek neu gedacht und umgestaltet werden.\footnote{\url{https://www.berlin.de/sen/finanzen/haushalt/siwana/sondervermoegen-infrastruktur-der-wachsenden-stadt-und-nachhaltigkeitsfonds-siwana-673149.php},
  (letzter Zugriff: 11.10.2020) sowie
  \url{https://www.zlb.de/de/ueber-uns/presse/pressemitteilung-detail/news/gute-entscheidung-fuer-berlins-oeffentliche-bibliotheken.html?sw=0\&cHash=3feaa80eff4d5aa65faf5c4d4a84db3d},
  (letzter Zugriff: 11.10.2020).}

Die Stadtbibliothek Pankow ist Mitglied des Netzwerks Grüne Bibliothek
und hat die Grundsatzerklärung von \enquote*{Libraries4Future}
unterschrieben und damit ein eindeutiges Bekenntnis geliefert. Zudem
soll im Leitbildprozess, der im Jahr 2022 abgeschlossen sein soll, das
Thema Nachhaltigkeit und Ökologie verankert werden. Die Stadtbibliothek
kann an Zielsetzungen des Bezirks Pankow und Programme wie die
\enquote*{Fair-Trade-Town}-Pankow\footnote{\url{https://www.berlin.de/ba-pankow/politik-und-verwaltung/beauftragte/lokale-agenda-21/artikel.436832.php},
  (letzter Zugriff: 04.10.2020).}, die \enquote*{Lokale Agenda
21}\footnote{\url{https://www.berlin.de/ba-pankow/politik-und-verwaltung/beauftragte/lokale-agenda-21/},
  (letzter Zugriff: 04.10.2020).} sowie an die 17 Ziele der UN-Agenda
2030 anknüpfen.\footnote{\url{https://www.berlin.de/ba-pankow/politik-und-verwaltung/beauftragte/entwicklungspolitik/},
  (letzter Zugriff: 04.10.2020).}

Eine exakte Zielsetzung für die \enquote*{Böll} als Grüne Bibliothek
existiert bisher nicht. Der Fokus liegt derzeit darauf, die eigenen
Prozesse mit einem Blick auf ökologischere Potentiale zu prüfen. Daher
richtet sich die Strategie eher darauf zu schauen, was im Rahmen der
eigenen Möglichkeiten umsetzbar ist und welche Mittel dafür benötigt
würden.

Zur besseren Darstellung der Strategie wurde eine Trennung in eine
\emph{interne} und \emph{externe} Wirkung unternommen. Dabei sind die
nach innen gerichteten Projekte oft an Technik, Architektur und
Ressourceneinsparung orientiert, während die externen Projekte häufig an
die Nutzer*innen der Bibliothek adressiert sind. Zudem soll die
Beschreibung der folgenden Projekte in vier verschiedenen Projektstadien
unterteilt werden und damit deutlich machen, in welcher Phase sie sich
befinden. So sind die Einzelprojekte unterschieden in:

\begin{itemize}[itemsep=-5pt]
\item erste Prüfungs- beziehungsweise Planungsphase
\item konkrete Planung
\item in der Umsetzungsphase
\item konkrete Utopie\footnote{Mit \enquote*{konkreter Utopie} wird auf die
    Theorie von Ernst Bloch angespielt, der den Begriff der Utopie aus
    einem negativen Verständnis in ein positives dreht. Utopien sind für
    ihn Zustände, die durch ein positives und experimentelles voran
    tasten erreicht werden können. Vergleiche
    \url{https://de.wikipedia.org/wiki/Konkrete_Utopie}, (letzter
    Zugriff: 11.10.2020).}
\end{itemize}

\hypertarget{interne-prozesse}{%
\subsection{Interne Prozesse}\label{interne-prozesse}}

\textbf{1. Einsparung bei der Foliierung von Medien (in der
Umsetzungsphase)}

Maßnahme:

\begin{itemize}[itemsep=-5pt]

\item
  In einigen Sachgruppen wurde für 2020 beschlossen, Bücher ohne
  Schutzumschlag nicht mehr zu foliieren.
\item
  Evaluation in 2021 über die Auswirkungen (zum Beispiel Zustand der
  nicht foliierten Medien)
\end{itemize}

Ziel:

\begin{itemize}[itemsep=-5pt]

\item
  Umstellung von Plastikfolien zu ökologisch weniger schädlichen Folien
\item
  Einsparung von Ressourcen und Geld
\end{itemize}

\textbf{2. Umstellung auf Ökobons (konkrete Planung)}

Maßnahme:

\begin{itemize}[itemsep=-5pt]

\item
  Umstellung der Quittungsrollen auf \enquote{Ökobons} für die gesamte
  Stadtbibliothek Pankow\footnote{Zum Beispiel
    \href{about:blank}{https://www.ökobon.de/}, (letzter Zugriff:
    11.10.2020).}
\item
  Umsetzung für 2021 anvisiert
\end{itemize}

Ziel:

\begin{itemize}[itemsep=-5pt]

\item
  Verbesserung des ökologischen Fußabdrucks durch weniger
  umweltschädliche Quittungsrollen
\end{itemize}

\textbf{3. Wasser sparen (erste Prüfungs- beziehungsweise
Planungsphase)}

Maßnahme:

\begin{itemize}[itemsep=-5pt]

\item
  Einführung von Wassersparmaßnahmen durch Sparaufsätze an den
  Wasserhähnen im internen Bereich (zum Beispiel Küche und Personal-WCs)
  und externen Bereich auf den öffentlichen WCs\footnote{Zum Beispiel
    \url{https://www.obi.de/mischduesen/wasserspar-verschraubung-perlator-eco-12-verchromt-1-2-/p/7225352},
    (letzter Zugriff: 11.10.2020).}
\end{itemize}

Ziel:

\begin{itemize}[itemsep=-5pt]

\item
  Größtmögliche Einsparung beim Wasserverbrauch
\item
  Sensibilisierung von Kolleg*innen sowie Nutzer*innen für das Thema
  Wasserverbrauch
\end{itemize}

\textbf{4. Umstellung der Beleuchtung auf LED-Leuchtmittel (erste
Prüfungs- beziehungsweise Planungsphase)}

Maßnahme:

\begin{itemize}[itemsep=-5pt]

\item
  Austausch der gegenwärtigen Leuchtmittel-Röhren durch LED-Leuchtmittel
\item
  Nutzung des Förderprogrammes des Landes Berlin (BENE-Mittel) für die
  nachhaltige Entwicklung zur Finanzierung\footnote{\url{https://www.berlin.de/senuvk/umwelt/foerderprogramme/bene/},
    (letzter Zugriff: 11.10.2020).}
\end{itemize}

Ziel:

\begin{itemize}[itemsep=-5pt]

\item
  Deutliche Reduzierung des Stromverbrauchs und damit von Ressourcen und
  Geld
\end{itemize}

\textbf{5. Einbau von Mooswänden zur Klimatisierung (erste Prüfungs-
beziehungsweise Planungsphase)}

Maßnahme:

\begin{itemize}[itemsep=-5pt]

\item
  Einbau von Mooswänden\footnote{\url{https://www.hydroflora.de/},
    (letzter Zugriff: 11.10.2020).}
\end{itemize}

Ziel:

\begin{itemize}[itemsep=-5pt]

\item
  Verbesserung der Luftfeuchtigkeit und der Raumerfrischung ohne Strom
  (zum Beispiel durch eine Klimaanlage)
\end{itemize}

\textbf{6. Einbau einer Photovoltaik-Anlage auf dem Flachdach der
Bibliothek (erste Prüfungs- beziehungsweise Planungsphase)}

Maßnahme:

\begin{itemize}[itemsep=-5pt]

\item
  Es wird geprüft, ob auf dem Flachdach eine Photovoltaik-Anlage
  installiert werden kann.
\item
  Unterstützung durch das Land Berlin im Rahmen des \enquote{Masterplan
  Solarcity} wird geprüft.\footnote{\url{https://www.berlin.de/special/energie-und-umwelt/nachrichten/6104242-5436174-masterplan-solarcity-mehr-solaranlagen-f.html},
    (letzter Zugriff: 11.10.2020). Diese Maßnahme muss jedoch mit dem
    Vermieter abgesprochen und geplant werden.}
\end{itemize}

Ziel:

\begin{itemize}[itemsep=-5pt]

\item
  Deutliche Einsparung des Stromverbrauchs durch eigene
  Stromproduktion\footnote{Eine erste Beratung und Berechnungen durch
    \enquote{Solarwende Berlin} ergab, dass auf dem Dach der Bibliothek
    durchschnittlich 75\,\% des eigenen Stromverbrauchs möglich wäre.
    \url{https://www.solarwende-berlin.de/startseite}, (letzter Zugriff:
    11.10.2020).}
\end{itemize}

\textbf{7. Einbau eines Gründachs auf dem Flachdach der Bibliothek
(erste Prüfungs- beziehungsweise Planungsphase)}

Maßnahme:

\begin{itemize}[itemsep=-5pt]

\item
  Es wird geprüft, ob auf das Flachdach zu einem Gründach umgebaut
  werden kann.
\item
  Unterstützung durch das Land Berlin im Rahmen des \enquote{1.000 Grüne
  Dächer}-Programms wird geprüft.\footnote{\url{https://www.berlin.de/senuvk/umwelt/stadtgruen/gruendaecher/},
    (letzter Zugriff: 11.10.2020). Vielleicht lässt sich auch ein
    Projekt im Rahmen des \enquote{Green Roof Lab} umsetzen.
    \url{https://www.gruendachplus.de/green-roof-lab/}, (letzter
    Zugriff: 11.10.2020). Auch diese Maßnahme muss mit dem Vermieter
    abgesprochen beziehungsweise gemeinsam entwickelt werden.}
\end{itemize}

Ziel:

\begin{itemize}[itemsep=-5pt]

\item
  Die Bibliothek leistet einen Beitrag zum lokalen Klima.
\item
  Im Optimalfall kann ein Gründach mit einer Photovoltaik-Anlage
  kombiniert werden.
\end{itemize}

\textbf{8. Umfassende Ermittlung des CO2-Abdruck der Bibliothek (in der
Umsetzungsphase)}

Maßnahme:

\begin{itemize}[itemsep=-5pt]

\item
  Im Rahmen eines Projektes der Kulturstiftung des Bundes wird die
  \enquote*{Böll} den eigenen Co2-Fußabruck ermitteln; es erfolgt eine
  möglichst umfassende Analyse, die vom Stromverbrauch und der Frage der
  Verstromung bis hin zur Frage reicht, mit welchen Verkehrsmitteln das
  Personal zur Arbeit kommt.\footnote{(Projekt der Kulturstiftung des
    Bundes)}
\end{itemize}

Ziel:

\begin{itemize}[itemsep=-5pt]

\item
  Den eigenen Co2-Fußabdruck möglichst genau feststellen, um weitere
  Einsparpotentiale zu entdecken
\end{itemize}

\hypertarget{externe-prozesse}{%
\subsection{Externe Prozesse}\label{externe-prozesse}}

\textbf{1. Projekt \enquote{Essbare Bibliothek} (in der
Umsetzungsphase)}

Maßnahme:

\begin{itemize}[itemsep=-5pt]

\item
  Anbau von Obst- und Gemüsepflanzen an den großen Fensterflächen der
  \enquote*{Böll}
\item
  Unterstützung durch FSJ-Kraft
\end{itemize}

\pagebreak
Ziel:

\begin{itemize}[itemsep=-5pt]

\item
  Bibliothek als Lernraum, der Wissen über das Wachstum von Obst und
  Gemüse vermittelt
\item
  Vertikaler Schulgarten durch direkte Einbindung einer Schulklasse, die
  sich regelmäßig um die Beete kümmert
\end{itemize}

\textbf{2. Wurmkomposter in der Bibliothek (in der Umsetzungsphase)}

Maßnahme:

\begin{itemize}[itemsep=-5pt]

\item
  Bau und Aufstellung einer \enquote*{Wurmkiste} für den
  Öffentlichkeitsbereich der Bibliothek
\item
  Kompostierung von Lebensmittelabfällen (zum Beispiel Tee, Kaffeesatz,
  Obst- und Gemüsereste) und Pflanzenresten\footnote{Beispielhaft dazu:
    \url{https://wurmkiste.at/}, (letzter Zugriff: 11.10.2020).}
\end{itemize}

Ziel:

\begin{itemize}[itemsep=-5pt]

\item
  Wurmkiste als Lernraum für Prozesse der Kompostierung und
  Wissensvermittlung zur Bedeutung von Humus
\item
  Wurmkiste als Inspiration für Nutzer*innen, mit dem Kompostieren zu
  Hause zu beginnen
\item
  Produzierter Kompost soll für Projekt \enquote*{Essbare Bibliothek}
  genutzt werden.
\end{itemize}

\textbf{3. Saatgutbibliothek (konkrete Planung)}

Maßnahme:

\begin{itemize}[itemsep=-5pt]

\item
  Aufbau einer Saatgutbibliothek aus dem gewonnen Saatgut der
  \enquote*{essbaren Bibliothek}
\item
  Angebot des Saatguts für die Nutzer*innen der Bibliothek
\item
  Fokus auf alten und samenfesten Sorten\footnote{\href{https://de.wikipedia.org/wiki/Nachbau_(Saatgut)}{https://de.wikipedia.org/wiki/Nachbau\_(Saatgut},
    (letzter Zugriff: 11.10.2020).}
\end{itemize}

Ziel:

\begin{itemize}[itemsep=-5pt]

\item
  Saatgutbibliothek als Lernraum, Wissensvermittlung über das Thema
  Saatgut
\item
  Inspiration der Nutzer*innen
\item
  Nachahmung der Nutzer*innen, alte und samenfeste Sorten zu Hause
  anzubauen
\end{itemize}

\textbf{4. Foodsharing-Kühlschrank (konkrete Planung)}

Maßnahme:

\begin{itemize}[itemsep=-5pt]

\item
  Installation eines Foodsharing-Kühlschrank
\item
  In Kooperation mit der Verbraucherschutzzentrale und dem
  Lebensmittelaufsichtsamt
\end{itemize}

Ziel:

\begin{itemize}[itemsep=-5pt]

\item
  Nutzer*innen erfahren mehr über die Ausmaße der
  Lebensmittelverschwendung und verändern ihr Verhalten
\item
  Nutzer*innen können sich Lebensmittel aus dem Kühlschrank kostenfrei
  mitnehmen
\end{itemize}

\textbf{5. Müllsammel-Boxen (erste Prüfungs- beziehungsweise
Planungsphase)}

Maßnahme:

\begin{itemize}[itemsep=-5pt]

\item
  Angebot einer Sammelbox für speziellen Müll (zum Beispiel alte Handys,
  Smartphones et cetera)\footnote{\url{https://www.nabu-shop.de/handysammelbox},
    (letzter Zugriff: 11.10.2020).}
\end{itemize}

Ziel:

\begin{itemize}[itemsep=-5pt]

\item
  Unterstützung von Recycling-Maßnahmen, in dem Nutzer*innen ihre alten
  Geräte in die Bibliothek bringen können
\item
  Müllsammel-Boxen als Lernraum durch die Vermittlung von Informationen
  zu Thema
\end{itemize}

\textbf{6. Trinkwasserspender in der Bibliothek (erste Prüfungs-
beziehungsweise Planungsphase)}

Maßnahme:

\begin{itemize}[itemsep=-5pt]

\item
  Im Öffentlichkeitsbereich soll ein Wasserspender für kostenfreies
  Trinkwasser angeboten werden.\footnote{Beispielhaft hierfür das
    Angebot der Berliner Wasserbetriebe:
    \url{https://www.bwb.de/de/1679.php}, (letzter Zugriff: 11.10.2020).}
\item
  Teilnahme an der \enquote*{Refill-Initiative}\footnote{\url{https://refill-deutschland.de/},
    (letzter Zugriff: 11.10.2020).}
\item
  Kein Angebot von Plastikbechern
\end{itemize}

Ziel:

\begin{itemize}[itemsep=-5pt]

\item
  Menschen für die Bedeutung von Trinkwasser sensibilisieren
\item
  Durch das Fehlen von Plastikbechern beim Wasserspender sollen Menschen
  zum Gebrauch von Mehrwegbechern angehalten werden
\item
  Menschen zum Einsparen von Plastik animieren
\end{itemize}

\textbf{7. Gieß den Kiez (erste Prüfungs- beziehungsweise
Planungsphase)}

Maßnahme:

\begin{itemize}[itemsep=-5pt]

\item
  Die Bibliothek bietet ihre Infrastruktur (Gießkannen, Schläuche et
  cetera) für die Menschen in der unmittelbaren Umgebung an, um die
  Initiative \enquote*{Gieß den Kiez} zu unterstützen.\footnote{\url{https://www.giessdenkiez.de/},
    (letzter Zugriff: 11.10.2020).}
\end{itemize}

Ziel:

\begin{itemize}[itemsep=-5pt]

\item
  Die Menschen in der unmittelbaren Umgebung der Bibliothek animieren,
  sich bei langer Trockenheit um die Grünflächen zu kümmern
\item
  Menschen für das Thema Trockenheit und Klimawandel sensibilisieren
\end{itemize}

\pagebreak
\textbf{8. Aquaponik (erste Prüfungs- beziehungsweise Planungsphase)}

Maßnahme:

\begin{itemize}[itemsep=-5pt]

\item
  Eine beispielhafte Aquaponik-Anlage im Öffentlichkeitsbereich der
  Bibliothek\footnote{\url{https://de.wikipedia.org/wiki/Aquaponik},
    (letzter Zugriff: 11.10.2020).}
\end{itemize}

Ziel:

\begin{itemize}[itemsep=-5pt]

\item
  Lernraum für die gegenwärtigen Möglichkeiten urbaner Landwirtschaft
  und ökologischen Kreisläufen anbieten
\end{itemize}

\textbf{9. Kooperation mit Imkerei bei einem möglichen Gründach
(konkrete Utopie)}

Maßnahme:

\begin{itemize}[itemsep=-5pt]

\item
  Im Falle eines Gründachs für die Bibliothek soll eine Kooperation mit
  einer lokalen Imkerei gesucht werden.
\end{itemize}

Ziel:

\begin{itemize}[itemsep=-5pt]

\item
  Gründach als Lernraum zum Thema Bienen und Imkerei
\item
  Makerspace Imkerei denkbar
\end{itemize}

\textbf{10. Gemeinschaftsgarten um die Böll herum (konkrete Utopie)}

Maßnahme:

\begin{itemize}[itemsep=-5pt]

\item
  Aufbau eines Gemeinschaftsgartens für die lokalen Anwohner*innen an
  den Außenflächen der Bibliothek
\item
  Unterstützung des Gartens durch die Nutzung der Infrastruktur der
  Bibliothek und durch eine Bibliothek der Dinge
\end{itemize}

Ziel:

\begin{itemize}[itemsep=-5pt]

\item
  Community Building und Gemeinschaftsgarten als Lernraum
\item
  Unterstützung und Einbindung lokaler Initiativen
\end{itemize}

\textbf{11. RepairCafé mit Fahrradwerkstatt (konkrete Utopie)}

Maßnahme:

\begin{itemize}[itemsep=-5pt]

\item
  In den beiden Kellerräumen der Bibliothek kann ein Raum für ein
  RepairCafé mit Fahrradwerkstatt eingerichtet werden.
\item
  Raum wird lokalen Repair-Initiativen dauerhaft zur Verfügung gestellt.
\end{itemize}

Ziel:

\begin{itemize}[itemsep=-5pt]

\item
  Unterstützung und Einbindung lokaler Initiativen
\item
  RepairCafé als Lernraum für die Einsparung von Ressourcen durch das
  Reparieren von Gegenständen
\end{itemize}

\hypertarget{its-the-end-of-the-buxf6ll-as-we-know-it-and-i-feel-fine-die-buxf6ll-2030-eine-konkrete-utopie}{%
\section{\texorpdfstring{4. It's the end of the \enquote*{Böll} as
we know it, and I feel fine -- die \enquote*{Böll} 2030, eine konkrete
Utopie}{4. It's the end of the `Böll' as we know it, and I feel fine -- die `Böll' 2030, eine konkrete Utopie}}\label{its-the-end-of-the-buxf6ll-as-we-know-it-and-i-feel-fine-die-buxf6ll-2030-eine-konkrete-utopie}}

Dieser Abschnitt soll eine eher spielerisch gedachte Zusammenfassung
darstellen und ein Bild der \enquote*{Böll} aus dem Blick des Jahres
2030 zeigen. Wie würde die Bibliothek aussehen, wenn alle beschriebenen
Maßnahmen erfolgreich umgesetzt würden? Was wäre, wenn aus den
\enquote*{konkreten Utopien} Stück für Stück Realität würde?\footnote{Diese
  teilweise spielerische Art der Darstellung orientiert sich an
  ähnlichen Prinzipien im Buch \enquote{Stadt der Zukunft}, in dem eine
  Stadt aus dem Jahr 2070 heraus beschrieben wird. Vergleiche von
  Borries (et al.): Stadt der Zukunft : Wege in die Globalopolis,
  Frankfurt am Main: 2019, (Forum für Verantwortung) sowie dem aktuellen
  Buch von James Lawrence Powell: 2084 : eine Zeitreise durch den
  Klimawandel, Köln : Quadriga, 2020. Powell benutzt in seinem Buch die
  Form von Zeitzeug*innen-Berichten, die im Jahr 2070 auf unsere
  Gegenwart zurück blicken.}

Die \enquote*{Böll} ist im Jahr 2030 Teil eines größeren lokalen
Netzwerks geworden und kann nur noch schwer als einzelne Institution
betrachtet werden. Sie ist ein grüner, lebendiger
zivilgesellschaftlicher Raum. Im Öffentlichkeitsbereich der Bibliothek
können sich die Nutzer*innen durch eine Vielzahl von kleinen
ökologischen Projekten inspirieren lassen und beim Obst und Gemüse, das
in den Räumen wächst, zugreifen. Durch den Aufbau des
Gemeinschaftsgartens, des RepairCafés und der Fahrrad-Werkstatt
gründeten sich lokale Initiativen, die diese Räume auch außerhalb der
Öffnungszeiten der Bibliothek nutzen, ganz im Sinne einer Open Library.
Die Bibliothek unterstützt diese Gruppen durch ihre Medienbestände,
digitalen Angebote und auch durch eine entsprechende Bibliothek der
Dinge. Zudem fanden sich durch durch \enquote*{Gieß den Kiez} viele
Menschen aus der Nachbar*innenschaft zusammen und kümmern sich um die
Pflanzen und Bäume im Umfeld.

Die \enquote*{Böll} selbst hat ihren Verbrauch an Ressourcen (Strom,
Wasser, Materialien) stark reduzieren können. Die Photovoltaik-Anlage
auf dem Flachdach wurde als Mieter*innen-Strom-Projekt gemeinsam mit der
Wohnungsbaugesellschaft und den Mieter*innen im Haus umgesetzt und kann
durch den eigenen stark gesenkten Stromverbrauch in Zeiten starker
Sonneneinstrahlung sogar kostenlos Strom in das Haus
einspeisen.\footnote{\url{https://de.wikipedia.org/wiki/Mieterstrom},
  (letzter Zugriff: 11.10.2020).}

Zusammen mit der Saatgutbibliothek und in Kooperation mit dem
Grünflächenamt konnte die \enquote*{Essbare Bibliothek} dazu beitragen,
dass im Kiez um die \enquote*{Böll} herum inzwischen viele essbare
Pflanzen wachsen, die vor allem Kinder durch Saatgutbomben (die in
speziellen Workshops in der Bibliothek gemacht wurden) gepflanzt haben.
Zusätzlich hat die \enquote*{Essbare Bibliothek}, zusammen mit dem
Foodsharing-Kühlschrank, bei vielen Menschen zu einem Umdenken geführt
und die Wertschätzung für Lebensmittel deutlich gesteigert -- und damit
das Wegwerfen von eigentlich noch guten Lebensmitteln reduziert.

Durch eine Vielzahl von Wurmkompostern in der Bibliothek konnte nicht
nur der eigene organische Abfall wiederverwertet werden. Gleichzeitig
haben sich Kolleg*innen und Nutzer*innen Wissen über den Prozess der
Humusbildung angeeignet und können ihn wertschätzen -- dieser Humus wird
wieder für die \enquote*{Essbare Bibliothek} sowie den
Gemeinschaftsgarten rund um die Bibliothek verwendet.

Die kleine Aquaponik-Anlage dient vor allem als Beispiel für die
Menschen für zu Hause. Die \enquote*{Böll} kooperiert in diesem Fall
jedoch mit Naturschutzorganisationen. Das gemeinsame Ziel ist es,
heimische Fischarten darin wachsen zu lassen, die anschließend in die
lokalen Gewässer freigelassen werden sollen. Das ist selbstverständlich
ein großes lokales Event, das regelmäßig gefeiert wird. Das Wissen über
die lokale Umwelt wird dabei fast automatisch weitergegeben.

Das Gründach sowie die Maßnahmen, Wasser zu sparen, konnten zumindest
einen kleinen Beitrag leisten, die lokale Umwelt zu entlasten. Das
gesparte Wasser hilft den Menschen, die sich bei \enquote*{Gieß den
Kiez} engagieren, da die Bibliothek die Menschen direkt mit Wasser und
auch Gerätschaften unterstützt, den Kiez grün zu halten und grüner zu
machen.

Die Mooswände in der Bibliothek helfen, das Klima ohne den Einsatz von
umweltschädlichen Ressourcen zu kontrollieren und stabil zu halten und
sollen als Inspiration für das eigene Zuhause dienen.

Durch die Sammelboxen für speziellen Müll (zum Beispiel alte Handys und
Smartphones) und die Kooperationen mit dem Recyclinghof können wirksame
Informationsveranstaltungen zum Thema Müllvermeidung angeboten werden.
Außerdem hat sich eine Art Wettbewerb im RepairCafé entwickelt, alte
Geräte doch noch einmal flott zu bekommen, um sie dann an Menschen mit
wenig Geld kostenfrei weitergeben zu können.

Der Medienbestand der \enquote*{Böll} wird alle diese Projekte aktiv
unterstützen. So stehen zum Beispiel Medien zum Thema Müllvermeidung
direkt bei den Sammelboxen und nicht mehr in einer Sachgruppe im Regal.

\hypertarget{fazit-bibliotheken-als-akteurinnen-im-uxf6kologischen-zeitalter}{%
\section{5. Fazit -- Bibliotheken als Akteurinnen im ökologischen
Zeitalter?}\label{fazit-bibliotheken-als-akteurinnen-im-uxf6kologischen-zeitalter}}

Sicherlich wird es alles im Detail in der \enquote*{Böll} so nicht
passieren, dennoch ist dieses utopische Bild vielleicht gar nicht so
weit von einer zukünftigen Realität entfernt. Vor allem, weil sehr viele
der hier beschriebenen einzelnen Projekte bereits in anderen
Öffentlichen Bibliotheken umgesetzt werden! Daher vertritt dieser
Beitrag die These, dass das Bild der konkreten Utopie sehr gut auf die
Prozesse einer Grünen Bibliothek anwendbar ist und das Potential
\enquote*{Grüner} Öffentlicher Bibliotheken aufzeigt.

Zudem wird die Bibliothek ihrer zentralen Aufgabe der Wissensweitergabe
und -vermittlung in einer deutlich erweiterten Rolle gerecht, da hinter
so gut wie allen Prozessen und Projekten die Idee steht,
Wissensweitergabe und -vermittlung durch direkte Aktivität, durch die
Begegnung der Menschen miteinander sowie durch die Inspiration von
Menschen zu ermöglichen.

\hypertarget{bibliotheken-als-akteurinnen-im-uxf6kologischen-zeitalter}{%
\subsection{Bibliotheken als Akteurinnen im ökologischen
Zeitalter?}\label{bibliotheken-als-akteurinnen-im-uxf6kologischen-zeitalter}}

Aus Sicht des Verfassers steht das Thema \enquote*{Grüne Bibliothek}
beziehungsweise sozial und ökologisch nachhaltige Bibliotheksarbeit in
Deutschland noch am Anfang. Doch gibt es inzwischen mehrere Initiativen
und immer mehr Beispiele von Einzelprojekten, ganz im Sinne sozialer und
ökologischer Nachhaltigkeit, die auf lokaler Ebene vorangebracht werden.
Gerade die Wahrnehmung der Rolle einer Bibliothek als lokale Akteurin
auf diesem Gebiet, stellt aus Sicht des Verfassers eine riesige Chance
für Öffentliche Bibliotheken dar.\footnote{Vergleiche Fußnote 6. Damit
  würden Bibliotheken die Rolle einnehmen und Aufgabe wahrnehmen, die
  Latour als einzigen Ausweg beschreibt, um dem drohenden
  gesellschaftlichen und klimatischen Kollaps zu entgehen.}

Zudem kommt, wenn auch immer seltener, der Vorwurf, dass Grüne
Bibliotheken ihren Auftrag verlassen neutral zu agieren. Aus Sicht des
Verfassers gibt es zwei Antworten auf diese Sichtweise:


\begin{enumerate}
\item  Nein! Bibliotheken helfen den Menschen, in einer sich stark und
rasant verändernden Welt Orientierung und Wissen zu erlangen, damit sie
diesen Veränderungen positiv begegnen können, und nehmen ihnen so
Ängste. Die bevorstehenden dramatischen Veränderungen stellen eine
gewaltige Herausforderung im alltäglichen Leben dar, für die die
Menschen gestärkt werden müssen.

\item Egal! Die prognostizierten Veränderungen durch den drohenden
Klimakollaps bedrohen den gesellschaftlichen Zusammenhalt und die
Demokratie, so dass keine Zeit mehr für eine vermeintliche Neutralität
bleibt. Vielmehr kann es Bibliotheken in ihrer Rolle stärken, hier eine
klare Position zu beziehen!
\end{enumerate}

Daraus stellen sich aus Sicht des Verfassers zwei zentrale Fragen für
die Zukunft:

\begin{enumerate}
\item Welche Rollen wollen/müssen Öffentliche Bibliotheken in der
\enquote{Dekade der Entscheidungen} spielen?\footnote{So ermöglicht zum
  Beispiel der Blick auf gesellschaftliche Kipppunkte die Rolle von
  Öffentlichen Bibliotheken anders zu denken. \enquote{Wir könnten einem
  gesellschaftlichen Kipppunkt näher sein, als wir denken, ab dem
  Klimaschutz endlich mit dem nötigen Ernst angepackt wird. Denn am Ende
  zählen nicht schöne Worte und hehre Ziele, sondern nur Taten. Und die
  globale Fieberkurve.}
  \url{https://www.spiegel.de/wissenschaft/mensch/eu-kommission-was-taugt-das-neue-klimaziel-a-f9578265-ac73-4993-abc0-748d4051b510},
  (letzter Zugriff: 11.10.2020).}

\item Wie schaffen wir es, Bibliotheken global zu vernetzen, um gemeinsam
etwas zu bewirken? Frei nach dem Motto \enquote*{think global -- act
local} könnten Bibliotheken für den Bereich sozialer und ökologischer
Nachhaltigkeit eine Schlüsselstellung einnehmen.\footnote{Ein Beispiel
  für dieses Vorgehen kann zum Beispiel der \enquote*{Parking Day} sein
  (an dem zum Beispiel die Stadtbibliothek Frankfurt/Main teilnimmt.)
  Wenn viele Öffentliche Bibliotheken sich am Parking Day beteiligen,
  können sie aufgrund ihrer Breitenwirkung einen großen Beitrag leisten,
  dem Ziel einer lebenswerten und menschengerechten Stadt näher zu
  kommen.
  \url{https://www.strasse-zurueckerobern.de/anleitungen/parking-day/},
  (letzter Zugriff: 11.20.2020).}
\end{enumerate}

%autor
\begin{center}\rule{0.5\linewidth}{0.5pt}\end{center}

\textbf{Tim Schumann} ist Mit-Initiator von Libraries4Future, aktiv im
Netzwerk Grüne Bibliothek und leitet die Heinrich-Böll-Bibliothek in
Berlin.

\end{document}
