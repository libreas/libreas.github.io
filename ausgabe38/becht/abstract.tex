\textbf{Kurzfassung}: Dieser Artikel unternimmt den Versuch, das Thema
Pergament um einige Blickwinkel zu erweitern, die hauptsächlich die
Tiere betreffen, aus deren Häuten dieser Beschreibstoff hergestellt wird
bzw. wurde. Ein kurzer Abriss zum Herstellungsprozess dient dabei als
Grundlage, die um eine ebenso knappe historische Entwicklungs- und
Nutzungsgeschichte ergänzt wird. Anschließend werden einige nicht allzu
ernst gemeinte quantitative Zahlenspielereien präsentiert, um die
schiere Zahl der geschlachteten Tiere zu zeigen, die für die in den
Altbeständen befindlichen Pergamentcodices notwendig gewesen sein
müssten. Der zweite Hauptpunkt des Artikels ergänzt einige
archäozoologische und agrarhistorische Aspekte zu den wichtigsten
pergamentliefernden Tieren, also zu Schafen, Ziegen und Rindern, im
deutschen Mittelalter.

\begin{center}\rule{0.5\linewidth}{0.5pt}\end{center}

\textbf{Abstract}: This article tries to offer a new perspective on
parchment, mainly with regard to animals whose skins were, and sometimes
still are, used to manufacture this writing material. A short outline of
the manufacturing process serves as a basis and is complemented by a
similarly short history of development and usage. Subsequently, some
calculations, which should not be taken too seriously, may illustrate
the huge number of animals that had to be slaughtered to produce enough
parchment for all the codices in the older collections of our libraries.
The second main topic is the animals who served as the main source for
parchment during the German Middle Ages, namely sheep, goats and cattle.
Therefore aspects of zooarchaeology and agricultural history are brought
up here as well.
