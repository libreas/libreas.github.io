\documentclass[a4paper,
fontsize=11pt,
%headings=small,
oneside,
numbers=noperiodatend,
parskip=half-,
bibliography=totoc,
final
]{scrartcl}

\usepackage[babel]{csquotes}
\usepackage{synttree}
\usepackage{graphicx}
\setkeys{Gin}{width=.6\textwidth} %default pics size

\graphicspath{{./plots/}}
\usepackage[ngerman]{babel}
\usepackage[T1]{fontenc}
%\usepackage{amsmath}
\usepackage[utf8x]{inputenc}
\usepackage [hyphens]{url}
\usepackage{booktabs} 
\usepackage[left=2.4cm,right=2.4cm,top=2.3cm,bottom=2cm,includeheadfoot]{geometry}
\usepackage{eurosym}
\usepackage{multirow}
\usepackage[ngerman]{varioref}
\setcapindent{1em}
\renewcommand{\labelitemi}{--}
\usepackage{paralist}
\usepackage{pdfpages}
\usepackage{lscape}
\usepackage{float}
\usepackage{acronym}
\usepackage{eurosym}
\usepackage{longtable,lscape}
\usepackage{mathpazo}
\usepackage[normalem]{ulem} %emphasize weiterhin kursiv
\usepackage[flushmargin,ragged]{footmisc} % left align footnote
\usepackage{ccicons} 
\setcapindent{0pt} % no indentation in captions

%%%% fancy LIBREAS URL color 
\usepackage{xcolor}
\definecolor{libreas}{RGB}{112,0,0}

\usepackage{listings}

\urlstyle{same}  % don't use monospace font for urls

\usepackage[fleqn]{amsmath}

%adjust fontsize for part

\usepackage{sectsty}
\partfont{\large}

%Das BibTeX-Zeichen mit \BibTeX setzen:
\def\symbol#1{\char #1\relax}
\def\bsl{{\tt\symbol{'134}}}
\def\BibTeX{{\rm B\kern-.05em{\sc i\kern-.025em b}\kern-.08em
    T\kern-.1667em\lower.7ex\hbox{E}\kern-.125emX}}

\usepackage{fancyhdr}
\fancyhf{}
\pagestyle{fancyplain}
\fancyhead[R]{\thepage}

% make sure bookmarks are created eventough sections are not numbered!
% uncommend if sections are numbered (bookmarks created by default)
\makeatletter
\renewcommand\@seccntformat[1]{}
\makeatother

% typo setup
\clubpenalty = 10000
\widowpenalty = 10000
\displaywidowpenalty = 10000

\usepackage{hyperxmp}
\usepackage[colorlinks, linkcolor=black,citecolor=black, urlcolor=libreas,
breaklinks= true,bookmarks=true,bookmarksopen=true]{hyperref}
\usepackage{breakurl}

%meta
%meta

\fancyhead[L]{N. Schuck, M. Rißberger, Red. LIBREAS\\ %author
LIBREAS. Library Ideas, 38 (2020). % journal, issue, volume.
\href{http://nbn-resolving.de/}
{}} % urn 
% recommended use
%\href{http://nbn-resolving.de/}{\color{black}{urn:nbn:de...}}
\fancyhead[R]{\thepage} %page number
\fancyfoot[L] {\ccLogo \ccAttribution\ \href{https://creativecommons.org/licenses/by/4.0/}{\color{black}Creative Commons BY 4.0}}  %licence
\fancyfoot[R] {ISSN: 1860-7950}

\title{\LARGE{Interview mit Nicole Schuck und Martina Rißberger: Zwischen Naturwissenschaft und Bildender Kunst – eine künstlerisch Perspektive auf naturhistorische Medien}}% title
\author{Nicole Schuck, Martina Rißberger, Redaktion LIBREAS} % author

\setcounter{page}{1}

\hypersetup{%
      pdftitle={Interview mit Nicole Schuck und Martina Rißberger: Zwischen Naturwissenschaft und Bildender Kunst – eine künstlerisch Perspektive auf naturhistorische Medien},
      pdfauthor={Nicole Schuck, Martina Rißberger, Redaktion LIBREAS},
      pdfcopyright={CC BY 4.0 International},
      pdfsubject={LIBREAS. Library Ideas, 38 (2020).},
      pdfkeywords={Interview, Bibliothek, Kunst, Kommunikation, Illustration},
      pdflicenseurl={https://creativecommons.org/licenses/by/4.0/},
      pdfcontacturl={http://libreas.eu},
      baseurl={http://libreas.eu},
      pdflang={de},
      pdfmetalang={de}
     }



\date{}
\begin{document}

\maketitle
\thispagestyle{fancyplain} 

%abstracts

%body
(Die Fragen stellte Jana Rumler)

\emph{LIBREAS: Vielen Dank, dass Sie sich bereit erklärt haben, unsere
Fragen zu beantworten. In der LIBREAS-Redaktion haben wir gemerkt, dass
eine gewisse Anzahl von Künstler*innen und Kreativen, wenn sie mit und
für Bibliotheken und andere Bildungseinrichtungen arbeiten, gerne als
Symbol auf Tiere oder Pflanzen zurückgreifen. Uns interessiert, warum
das so ist.\footnote{\url{https://libreas.wordpress.com/2020/03/03/call-for-papers-38-tiere-und-gewaechse/}}
Gleichzeitig wollen wir die Möglichkeit auch nutzen, um von Ihrer Seite
zu hören, wie eine solche Zusammenarbeit mit Bibliotheken ablaufen
kann.}

\hypertarget{nicole-schuck}{%
\subsubsection{Nicole Schuck}\label{nicole-schuck}}

\emph{LIBREAS: Wann kamen Sie das erste Mal in Ihrer Arbeit mit der
Bibliothek des Museums für Naturkunde in Kontakt? Hatte das Herantreten
an die Bibliothek bei Ihnen \enquote{Vorläufer}, beispielsweise eigene
Erfahrungen aus der Kindheit und Jugend und später im künstlerischen
Schaffen?}

NK: Im Zentrum meiner künstlerischen Auseinandersetzung stehen
Wildtiere, natürliche und urbane Lebensräume, Fragen des Naturschutzes,
Klimawandels und der Werteökonomie. Meine Arbeitsweise basiert auf dem
Wechselspiel von Kunst und Wissenschaft, indem ich deren Schnittstellen
und Überschneidungsflächen auslote und aus ihren unterschiedlichen
Forschungs- und Erkenntnismöglichkeiten schöpfe.

Der erste persönliche Besuch der Bibliothek fand 2007 im Rahmen des
Projektes «HUM» im Museum für Naturkunde (MfN) statt, zu dem ich als
\emph{Performende} eingeladen wurde. Die Fülle an naturbezüglichen
historischen Büchern beeindruckte mich sehr und die Bibliothek in ihrem
gesamten Erscheinungsbild bietet eine Zeitreise durch die Geschichte.
Fortan dachte ich immer wieder an diesen spannenden Ort der
Wissenssammlung mit den zahlreichen Regalen und Schränken.

Für die Ausstellung «On the Edge» im Tieranatomischen Theater in Berlin
2015 wurde ich eingeladen, eine Arbeit im Dialog mit Objekten aus den
wissenschaftlichen Sammlungen der Humboldt-Universität zu Berlin zu
konzipieren. Meine langjährige Auseinandersetzung mit Wildtieren führte
mich derzeit zu dem Thema «Naturkapital». Auch wenn das Museum für
Naturkunde heute ein Forschungsmuseum der Leibniz-Gemeinschaft ist und
nicht zu der Sammlung der Humboldt-Universität gehört, kam der Wunsch
auf, eine Arbeit mit Folianten von Conrad Gessner aus der Bibliothek zu
realisieren. Seine Studien gewähren einen Einblick in eine Zeit, in der
Mythologie und Wissenschaft, Imagination und Beobachtung als
gleichwertige Erkenntnisquellen behandelt wurden. Die Überlappung dieser
unterschiedlichen Bereiche beschäftigt mich seit Jahren.

Bücher mit Tierdarstellungen und Geschichten über ihre Lebensweise
faszinieren mich seit meiner Kindheit. In meiner Verwandtschaft wurden
Bücher leidenschaftlich gesammelt und verschenkt, was mir zugute kam.
Bibliotheken waren und sind wichtige Orte der Recherche und Inspiration
für meine künstlerische Arbeit und werden es auch in Zukunft sein. Etwas
Bedeutendes zu finden, wonach ich nicht gesucht habe und es unter
anderem haptisch zu «begreifen», ist das Glück eines
Bibliotheksbesuches.

Mit meinem aktuellen Buch «Geschätzte Tiere» zum Wert von Wildtieren,
realisiert sich mein Anliegen, selbst Bücher zu Wildtieren und unserem
gemeinsamen Lebensraum herauszugeben.

\begin{figure}
\centering
\includegraphics{img/image1.jpg}
\caption{Buchcover Nicole Schuck \enquote{Geschätzte Tiere - Valued
Animals} (2020, Foto von Nicole Schuck)}
\end{figure}

\emph{LIBREAS: Unterschied sich die Arbeit mit einer Spezialbibliothek
von jener mit anderen Bibliotheken?}

NK: Ja, für mich schon. Die Zugriffsmöglichkeiten am MfN sind andere,
der Bestand ist unkompliziert zugänglich und nutzbar. Der Unterschied
ist ein direkter Kontakt zu den Medien und den Personen, die seit vielen
Jahren mit dem Bestand vertraut sind. Martina Rißberger öffnete Türen
und Schränke, im wahrsten Sinne des Wortes, und ermöglichte mir auf
diese Weise den Zugang zu speziellem, wertvollem historischem Wissen.
Sie unterstützte mein Anliegen und stellte Bezüge zu Werken anderer
Autoren her, die mir vorher unbekannt waren. Als interessierte,
engagierte und feinsinnige Vermittlerin mit umfassendem Wissen, stellte
die Kommunikation mit ihr einen wichtigen Aspekt für die Entwicklung
meiner Arbeit dar.

\emph{LIBREAS: Die fachliche Leitung der Bibliothek hat Sie bei den
kreativen Prozessen begleitet. Wie sah die Zusammenarbeit konkret aus?}

NK: Martina Rißberger ermöglichte mir Zugang zu allen Folianten Conrad
Gessners, die in der Bibliothek vorhanden sind und machte mich auf
weitere Autor*innen aufmerksam, die thematisch interessant waren. Wir
tauschten uns zu den Inhalten und Tierdarstellungen aus und diskutierten
die unterschiedlichen Visualisierungen.

Mein Animationsvideo «Von Thieren auff der erde und wasseren ir wonung
habend» stellt 13 Kreaturen aus sechs verschiedenen Folianten Gessners
vor. Die Holzschnitte in den verschiedenen Ausgaben weichen voneinander
ab, die Veränderungen werden durch Überlagerungen der Bilder desselben
Tieres in dem Video visualisiert. Das wirft die Frage auf, inwiefern die
Übertragung von Wissen durch visuelle Mittel zu unserem, sich wandelnden
Verständnis der Welt beiträgt.

\emph{LIBREAS: Wie sind Sie vorgegangen, um zu den Motiven für Ihre
Arbeiten, speziell für Ihr 2020 erschienenes Buch «Geschätzte Tiere --
Valued Animals»}\footnote{\url{https://www.hatjecantz.de/nicole-schuck-7744-0.html?article_id=7744\&clang=0}}
\emph{zu gelangen?}

NK: Die Grundlage aller meiner Projekte der vergangenen Dekade, die in
dem Buch vorgestellt werden, ist eine ausführliche Recherche und
Feldforschung: Ich trage vielfältigste Informationen zum jeweiligen
Lebensraum von Wildtier und Mensch zusammen, durch Archivrecherche
(unter anderem Bibliotheken, Internet), Gespräche, Ortsbegehungen und
Kooperationen mit Spezialist*innen wie Biolog*innen, Zoolog*innen,
Philosoph*innen, Naturschützer*innen und Einheimischen. Diese wichtigen
Bestandteile meiner Arbeit lassen ein erweitertes Verständnis für die
Fauna und den gemeinsamen Lebensraum entstehen. In meinen Zeichnungen
und Installationen transformiert sich dies zu eigenen Formen, aus denen
wiederum neue Erfahrungsebenen erwachsen.

\emph{LIBREAS: Stach das Vorgehen zum Projekt für Sie heraus aus Ihrem
sonstigen Schaffen oder würden Sie dieses eher als typisch für Ihre
Projekte ansehen?}

NK: In meinen Projekten spielt der Austausch mit Expert*innen eine
wichtige Rolle.

Das Besondere mit Martina Rißberger war und ist bis heute ihr großes
persönliches Interesse an meinen Projekten und ihre diesbezügliche
Unterstützung. Sich so auf meine Arbeiten einzulassen, eröffnet eine
tiefere Auseinandersetzung über die Inhalte.

\emph{LIBREAS: Ihre gezeichneten Tiere (Auster, Eule, Mauswiesel,
Moorfrosch, Rebhuhn, Uhu, Fledermaus, Stachelschwein und andere)
zeichnen sich durch die «Bewegungsvielfalt» mit «zeichnerischer
Kontextualisierung» aus, wie es ein Kunstkritiker in Ihrem Buch
beschreibt.}\footnote{Erich Franz: Übersetzen in Zeichnungen: Nicole
  Schucks Annäherung an das Tier. In: Schuck, Nicole (2020): Geschätzte
  Tiere -- Valued Animals. Berlin, Hatje Cantz Verlag. S. 89-93.}
\emph{Was ist damit gemeint?}

NK: Hiermit ist die Linienführung gemeint mit der ich das Tier mit dem
Stift abtaste und bewandere. Der zeichnerische Impuls folgt jedem Tier
individuell, dem was seine Haut, seine Haare, seine Schale oder Federn
mir vorgeben. Die Linien verdichten sich, tauchen aus dem Weiss des
Papieres auf und wieder ab. Indem ich mich mit Bleistiften und ihrer
ganzen Palette von Grauwerten einem Tier und seinen Strukturen und
Beziehungen nähere -- oft über mehrere Wochen und Monate hinweg --,
entstehen profunde Beziehungen und Kenntnisse, was wiederum zu einer
großen Wertschätzung des Wildtieres als Individuum / Subjekt und für den
gemeinsamen Lebensraum führt. Die Zeichnung spürt ihnen nach, geht
individuell auf sie ein und hat keine systematischen Voraussetzungen --
außer dem linear-prozessualen Sich-Einlassen auf jedes einzelne Tier und
seine Bezüge. Wichtig ist mir hier, diese bedingungslose, nachdrücklich
unwissenschaftliche Wertschätzung des einzelnen Lebewesens und unserer
Umwelt sichtbar zu machen. Die Zeichnung erlaubt es damit, das an sich
nicht Sichtbare zur Darstellung zu bringen.

Papier als Bildträger ist für meine Zeichnungen sehr bedeutsam. Es
stellt eine Allegorie zu unserem gemeinsamen Lebensraum dar. Das Papier,
das von dem gezeichneten Tier «bewohnt» wird, reagiert auf kleinste
klimatische Veränderungen. Je höher die Luftfeuchtigkeit steigt, desto
welliger wird das Papier, was besonders bei den großen Formaten sichtbar
wird. Je trockener die Luft, desto mehr spannt sich die Oberfläche.
Papier ist in vielerlei Hinsicht empfindlich. Es reißt schnell, auch
Spuren vom Radieren und Ablegen der Hände prägen sich ins Papier ein.
Die Linie lässt sich an diesen Stellen nicht mehr so präzise und klar
ziehen, sie sieht verschwommener aus. Die Oberfläche wird durch die
Anwendung von Radierern offener und rauer. Auch direktes Tageslicht
verändert über einen langen Zeitraum ebenso das Erscheinungsbild des
Papiers.

Einen Folianten aus dem 17. Jahrhundert in den Händen zu halten, mit all
seinen in das Papier «eingeschriebenen» Lebensspuren visuell und
haptisch zu erleben, ist ein bewegender Moment, der Respekt hervorruft.
Diese Spuren selbst zu berühren, das alte Papier rascheln zu hören und
seinen Geruch wahrzunehmen, setzen die Informationen auf ihm in
bestimmte historische Zusammenhänge. Dieses direkte Mitvollziehen von
Geschichte und Geschichten verfängt sich anders in der Rezeption als
internetbasierte Recherche, der etwas Steriles anhaftet.

\emph{LIBREAS: Welches Potential steckt, Ihrer Meinung nach, in der
Kommunikation mit der und über die Bibliothek? Gab es besondere
Anknüpfungspunkt zu den Mitarbeitenden in der Bibliothek sowie mit den
Wissenschaftler*innen im Kontext des Museums, die Sie für Ihre Arbeit
ebenfalls einbezogen haben?}

NK: Das Potential der Bibliothek stellt für mich der umfassende Bestand
und das persönliche Engagement und Interesse an meiner künstlerischen
Arbeit von Martina Rißberger dar, welches sich aus dem Projekt «On the
Edge» 2015 bis heute weiterentwickelte. Martina Rißberger denkt mit und
knüpft nicht nur inhaltliche Verbindungen ihr Feld betreffend, sondern
vermittelt Kontakte zu Wissenschaftler*innen und Mitarbeiter*innen, die
ihr inhaltlich sowie für die Verbreitung des Buches «Geschätzte Tiere»
sinnvoll erschienen. Beispielsweise basiert der Kontakt zu Jörg Freyhof,
Biologe und Bioökonom am Museum für Naturkunde und einer der Autor*innen
meines Buches, auf ihrer Unterstützung.

Die erste Buchvorstellung überhaupt von «Geschätzte Tiere» kam durch
Martina Rißbergers Vermittlung zustanden. Wir planten gemeinsam die
Buchveröffentlichung im Mai live mit Gästen im Museum zu feiern.
Covid-19-bedingt realisierten wir stattdessen einen einstündigen
Instagram-Livestream-Spaziergang im Museum\footnote{\url{https://www.museumfuernaturkunde.berlin/de/museum/veranstaltungen/instagram-live-dem-wert-von-wildtieren-auf-der-spur}},
bei dem ich im Gespräch mit zwei der Autor*innen -- Jörg Freyhof und
Georg Toepfer, Philosoph und Biologe -- über den Wert von Wildtieren und
Natur spreche. Elisa Herrmann, Wissenschaftliche Leiterin der
Bibliothekssammlung, führte in bedeutende historische Tierdarstellungen
aus wissenschaftlicher Sicht ein.

Allgemein von Bedeutung ist, dass durch die Vermittlung des Bestandes
vielfältigste neue Publikationen entstehen und diese wiederum der
Öffentlichkeit weltweit zur Verfügung gestellt werden. Hier wird
wissenschaftlicher, und in meinem Fall transdisziplinärer Output
unterstützt.

\emph{LIBREAS: Welche Erfahrungen nehmen Sie aus der Zusammenarbeit mit
dieser Bibliothek mit?}

NK: Die Zusammenarbeit mit Martina Rißberger ist eine herausragend gute
und unterstützende. Wie gut eine Zusammenarbeit funktioniert, hängt von
den jeweiligen Menschen ab, auf die man trifft und lässt sich demzufolge
nicht verallgemeinern. Je offener die Beteiligten sind und sich
interessiert einlassen, umso mehr kann sich daraus entwickeln.

\emph{LIBREAS: Sprechen Sie mit anderen Kreativen darüber, je nach
Ausrichtung ihres Schaffensprozesses, Bildungseinrichtungen wie
Bibliotheken, Museen und Archive einzubeziehen?}

NK: Ja klar, der Austausch über Erlebtes mit Einrichtungen und
Expert*innen ist Gesprächsstoff unter Gleichgesinnten. Positive Berichte
können sich inspirierend auswirken und zu neuen Projekten anregen. Die
Bibliothek des Museum für Naturkunde mit ihren unterstützenden
Mitarbeitenden ist ein wundervoller Ort der Wissenssammlung, -vermehrung
und -verbreitung, deren Besuch und Nutzung ich Interessierten nur ans
Herz legen kann.

\hypertarget{martina-riuxdfberger}{%
\subsubsection{Martina Rißberger}\label{martina-riuxdfberger}}

\emph{LIBREAS: Wie kam der Kontakt zur Künstlerin zustande? Gab es in
der Bibliothek vorher bereits ähnliche Projekte, die es Menschen
erlaubte, die Bestände und Räumlichkeit so zu nutzen, dass kreatives
Schaffen ermöglicht werden könnte?}

MR: Der Kontakt zu der Künstlerin Nicole Schuck entstand ursprünglich
durch das Kunstprojekt der Humboldt-Universität «On the Edge», das im
Jahr 2015 Künstler*innen die Möglichkeit gab, in Beziehung zu Objekten
aus den wissenschaftlichen Sammlungen der Universität neue Kunstwerke zu
schaffen. Begründet auf ihrem naturwissenschaftlichen Interesse
entschied sich Nicole Schuck für die Zoologische Sammlung des Museums
für Naturkunde Berlin, die bis 2009 zur Humboldt-Universität gehörte.
Der Kontakt zur Bibliothek des Museums entstand durch ihre Auswahl des
in der Bibliothek befindlichen Bandes von Conrad Gessner: Das Thierbuch,
1669.

\begin{figure}
\centering
\includegraphics{img/image2.jpg}
\caption{Conrad Gessner \enquote{Das Thierbuch} (1669, Foto von Carola
Radke vor der Restaurierung des Werkes)}
\end{figure}

\emph{LIBREAS: Gab es Unterschiede in der Kommunikation im Vergleich zu
den typischen Nutzenden der wissenschaftlichen Spezialbibliothek (eines
integrierten Forschungsmuseums). Wenn ja, wie sahen diese aus? Gab es
besondere Anliegen, die bei Ihnen, den Mitarbeitenden und der
Künstlerin, geweckt wurden?}

MR: Typische Nutzer der wissenschaftlichen Bibliothek am Museum für
Naturkunde Berlin sind deren Forscher*innen und Mitarbeiter*innen,
außerdem externe Wissenschaftler*innen, Studierende, Schüler*innen und
naturkundlich interessierte Bürger*innen, einschließlich Künstler*innen.
Das Anliegen der Künstlerin Nicole Schuck an die Bibliothek bestand
darin, verschiedene Auflagen des «Thierbuchs» von Conrad Gessner aus dem
Bestand der Bibliothek hinsichtlich der wissenschaftlichen Illustration
zu vergleichen. Dasselbe Motiv sollte in den verschiedenen Ausgaben
miteinander verglichen werden, woraus sich die Fragestellung
entwickelte, wie die Art der Darstellung im Kontext steht mit dem
Wissenschaftsstand der Zeit. Durch den breiten historischen Bestand der
Bibliothek war es möglich, dieses Anliegen zu erfüllen.

\emph{LIBREAS: Wie haben Sie den Umgang mit den Rara-Beständen in der
Zusammenarbeit erlebt? Hat sich der fachliche Blick auf die Bestände
verändert?}

MR: Bei der Bereitstellung historischer Bibliotheksmedien für
Künstler*innen steht das Interesse an der wissenschaftlichen
Illustration zumeist im Vordergrund. Ich setze beim Umgang mit
historischen Medien meine Kenntnisse der Buchkunst und auch der
Kunstgeschichte ein. In der Zusammenarbeit mit Nicole Schuck wurde meine
Aufmerksamkeit nachhaltig auf die Interpretationen der Illustrationen
hinsichtlich ihres Kontextes zu gesellschaftlichen und ökologischen
Aspekten der Zeitepoche gelenkt; ich könnte auch sagen hinsichtlich der
Wertschätzung der Tiere im jeweiligen Jahrhundert. Ich habe
diesbezüglich einen Wissenszuwachs erfahren, dass ich bei der
Betrachtung historischer Tierillustrationen den Blick stärker auf die in
ihnen ausgedrückte gesellschaftliche Wertschätzung richte.

\emph{LIBREAS: Haben Sie sich mit Fachkolleg*innen, Mitarbeitenden und
Wissenschaftler*innen dazu ausgetauscht wie man den kreativen
Schaffensprozess unterstützen kann?}

MR: Ja, ich stand im Kontakt mit einem Zoologen und einer
Kunstwissenschaftlerin an unserem Institut.

\emph{LIBREAS: Die Künstlerin hat in diesem Jahr ein Buch
veröffentlicht, das unter anderem im Kontext der Arbeit mit der
Bibliothek zu sehen ist. Dieses ist dann auch, bedingt durch die
Pandemie im digitalen Raum, über die Öffentlichkeitsarbeit des Museums
vorgestellt worden. Wie fruchtbringend empfanden Sie die Kommunikation
mit der Künstlerin?}

MR: Die Kommunikation mit der Künstlerin Nicole Schuck war sehr
gewinnbringend. Die Inhalte ihres Buches «Geschätzte Tiere» greifen mit
den Inhalten unseres Bibliotheksbestandes ineinander. Das Wissen über
die Lebensräume der Wildtiere in ihrem Buch ist mit dem Wissen, das die
Bibliothek bewahrt, eng verwoben. Die erste öffentliche Präsentation
ihres Buches sollte deshalb auch am Museum für Naturkunde Berlin
stattfinden.

In der Zeit der Covid-19-Pandemie, die umso mehr den Zusammenhang
zwischen den Lebensräumen der Wildtiere und den menschlichen Eingriffen
in diese verdeutlicht und deren Erforschung sich das Museum auch
verstärkt widmet, ist auch das Thema des Buches «Geschätzte Tiere» von
besonderer Aktualität. Eine alternative Vorstellung des Buches konnte
schließlich virtuell stattfinden.

\emph{LIBREAS: Hat sich bei Ihnen durch die Zusammenarbeit der Wunsch
nach aktivem Einbeziehen von Kunstschaffenden und anderen Kreativen als
Bibliotheksnutzende eingestellt? Welche Anknüpfungspunkte an die
bisherige Arbeit der Bibliothek ergeben sich hier?}

MR: Die Verbindung zwischen Kunst und Naturwissenschaft wird am Museum
für Naturkunde rege gelebt, unter anderem durch Projekte zur
künstlerischen Interpretation von naturkundlichen Objekten und mit
Ausstellungen an der Grenze von Naturwissenschaft und bildender Kunst.
In diesem Kontext wird auch die Bibliothek von Kunst- und
Kulturschaffenden genutzt und die Bibliothek trägt mit ihrem Bestand
dazu bei, neben den wissenschaftlichen Publikationen der eigenen
Institution auch Publikationen von Kunstschaffenden zu unterstützen. Die
Medien der Bibliothek dienen diesen künstlerischen Fragestellungen, die
neu entstanden Publikationen der Kunstschaffenden bereichern wiederum
unseren Bibliotheksbestand.

%autor
\begin{center}\rule{0.5\linewidth}{0.5pt}\end{center}

\textbf{Nicole Schuck} wurde in Herford / Westfalen geboren und lebt in
Berlin. Sie studierte Visuelle Kommunikation mit dem Schwerpunkt
Zeichnung an der Fachhochschule Bielefeld sowie Freie Kunst an der
Universität der Künste Berlin und der Hochschule für Bildende Künste
Braunschweig, wo sie als Meisterschülerin bei Professor John Armleder
2004 ihren Abschluss machte. Seitdem ist sie freiberuflich als
Künstlerin tätig, Schwerpunkte ihrer Projekte sind die Themen Wildtiere,
natürliche und urbane Lebensräume, ökologie und Naturschutz. Der
transdisziplinäre Austausch und die Zusammenarbeit mit Interessierten
ist ein wichtiger Bestandteil ihrer Projekte. Seit 2017 befasst sie sich
speziell mit ökosystemleistungen und dem Wert von Meeres- und Alpenfauna
für den Menschen. Nicole Schuck erhielt zahlreiche Stipendien,
Förderungen und Preise von den folgenden Institutionen. Ihre Arbeiten,
Projekte, Ausstellungen, Kunst am Bau, Vorträge, Publikationen und
Beiträge zu Theaterinszenierungen werden im In- und Ausland gezeigt.

\textbf{Martina Rißberger}, geboren und aufgewachsen in Berlin (DDR). Ab
1979 Buchhändlerlehre mit anschließender 2-jähriger Tätigkeit im
staatlichen Buchhandel der DDR, ab 1984 Studium an der Fachschule für
wissenschaftliche Information und wissenschaftliches Bibliothekswesen
Berlin (DDR). Ab 1987 Tätigkeit als Dipl.-Bibliothekarin an der
Humboldt-Universität zu Berlin und seit 2001 Fachliche Leiterin der
Bibliothek am Museum für Naturkunde Berlin. Ihr besonderes Interesse
gilt der Kunstgeschichte und der Bildenden Kunst, welches durch ein
jahrelanges autodidaktisches Studium der Kunstgeschichte gefestigt ist.

\end{document}
