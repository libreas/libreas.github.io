\begin{center}\rule{0.5\linewidth}{0.5pt}\end{center}

\textbf{Nicole Schuck} wurde in Herford / Westfalen geboren und lebt in
Berlin. Sie studierte Visuelle Kommunikation mit dem Schwerpunkt
Zeichnung an der Fachhochschule Bielefeld sowie Freie Kunst an der
Universität der Künste Berlin und der Hochschule für Bildende Künste
Braunschweig, wo sie als Meisterschülerin bei Professor John Armleder
2004 ihren Abschluss machte. Seitdem ist sie freiberuflich als
Künstlerin tätig, Schwerpunkte ihrer Projekte sind die Themen Wildtiere,
natürliche und urbane Lebensräume, ökologie und Naturschutz. Der
transdisziplinäre Austausch und die Zusammenarbeit mit Interessierten
ist ein wichtiger Bestandteil ihrer Projekte. Seit 2017 befasst sie sich
speziell mit ökosystemleistungen und dem Wert von Meeres- und Alpenfauna
für den Menschen. Nicole Schuck erhielt zahlreiche Stipendien,
Förderungen und Preise von den folgenden Institutionen. Ihre Arbeiten,
Projekte, Ausstellungen, Kunst am Bau, Vorträge, Publikationen und
Beiträge zu Theaterinszenierungen werden im In- und Ausland gezeigt.

\textbf{Martina Rißberger}, geboren und aufgewachsen in Berlin (DDR). Ab
1979 Buchhändlerlehre mit anschließender 2-jähriger Tätigkeit im
staatlichen Buchhandel der DDR, ab 1984 Studium an der Fachschule für
wissenschaftliche Information und wissenschaftliches Bibliothekswesen
Berlin (DDR). Ab 1987 Tätigkeit als Dipl.-Bibliothekarin an der
Humboldt-Universität zu Berlin und seit 2001 Fachliche Leiterin der
Bibliothek am Museum für Naturkunde Berlin. Ihr besonderes Interesse
gilt der Kunstgeschichte und der Bildenden Kunst, welches durch ein
jahrelanges autodidaktisches Studium der Kunstgeschichte gefestigt ist.
