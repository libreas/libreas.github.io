\begin{center}\rule{0.5\linewidth}{0.5pt}\end{center}

\textbf{Nicole Schuck} wurde in Herford / Westfalen geboren und lebt in Berlin. Sie
studierte Visuelle Kommunikation mit dem Schwerpunkt Zeichnung an der
Fachhochschule Bielefeld sowie Freie Kunst an der Universität der Künste
Berlin und der Hochschule für Bildende Künste Braunschweig, wo sie als
Meisterschülerin bei Professor John Armleder 2004 ihren Abschluss
machte. Seitdem ist sie freiberuflich als Künstlerin tätig, Schwerpunkte
ihrer Projekte sind die Themen Wildtiere, natürliche und urbane
Lebensräume, Ökologie und Naturschutz. Der transdisziplinäre Austausch
und die Zusammenarbeit mit Interessierten ist ein wichtiger Bestandteil
ihrer Projekte. Seit 2017 befasst sie sich speziell mit
Ökosystemleistungen und dem Wert von Meeres- und Alpenfauna für den
Menschen. Nicole Schuck erhielt zahlreiche Stipendien, Förderungen und
Preise von den folgenden Institutionen. Ihre Arbeiten, Projekte,
Ausstellungen, Kunst am Bau, Vorträge, Publikationen und Beiträge zu
Theaterinszenierungen werden im In- und Ausland gezeigt.

\textbf{Martina Rißberger}, geboren und aufgewachsen in Berlin (DDR). Ab 1979
Buchhändlerlehre mit anschließender 2-jähriger Tätigkeit im staatlichen
Buchhandel der DDR, ab 1984 Studium an der Fachschule für
wissenschaftliche Information und wissenschaftliches Bibliothekswesen
Berlin (DDR). Ab 1987 Tätigkeit als Dipl.-Bibliothekarin an der
Humboldt-Universität zu Berlin und seit 2001 Fachliche Leiterin der
Bibliothek am Museum für Naturkunde Berlin. Ihr besonderes Interesse
gilt der Kunstgeschichte und der Bildenden Kunst, welches durch ein
jahrelanges autodidaktisches Studium der Kunstgeschichte gefestigt ist.
