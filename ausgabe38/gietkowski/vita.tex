\begin{center}\rule{0.5\linewidth}{0.5pt}\end{center}

\textbf{Katharina Therese Gietkowski} studierte Kunstgeschichte,
Anglistik, Buchwissenschaft und Textforschung sowie Europäische
Ethnologie an der Westfälischen Wilhelms-Universität Münster und der
Università Suor Orsola Benincasa in Neapel. Es folgte ein Masterstudium
der Bibliotheks- und Informationswissenschaft an der Hochschule für
Technik, Wirtschaft und Kultur (HTWK) in Leipzig. Von 2014 bis 2015 war
sie an der Herzog August Bibliothek Wolfenbüttel als wissenschaftliche
Bibliothekarin in der Abteilung Alte Drucke tätig. Seit 2015 promoviert
Katharina Gietkowski im Graduiertenkolleg "Wissensspeicher und
Argumentationsarsenal. Funktionen der Bibliothek in den kulturellen
Zentren im Europa der Frühen Neuzeit" am IKFN der Universität
Osnabrück. Als wissenschaftliche Mitarbeiterin arbeitet sie seit 2019 im
Handschriftenzentrum der Universitätsbibliothek Leipzig im
BMBF-Teilprojekt "Mikroben als Sonden der
Buchbiographie. Kulturwissenschaftliche Objektstudien zu
Spätmittelalterlichen Sammelbänden im Bestand der Universitätsbibliothek
Leipzig."
