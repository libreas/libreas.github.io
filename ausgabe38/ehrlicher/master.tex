\documentclass[a4paper,
fontsize=11pt,
%headings=small,
oneside,
numbers=noperiodatend,
parskip=half-,
bibliography=totoc,
final
]{scrartcl}

\usepackage[babel]{csquotes}
\usepackage{synttree}
\usepackage{graphicx}
\setkeys{Gin}{width=.4\textwidth} %default pics size

\graphicspath{{./plots/}}
\usepackage[ngerman]{babel}
\usepackage[T1]{fontenc}
%\usepackage{amsmath}
\usepackage[utf8x]{inputenc}
\usepackage [hyphens]{url}
\usepackage{booktabs} 
\usepackage[left=2.4cm,right=2.4cm,top=2.3cm,bottom=2cm,includeheadfoot]{geometry}
\usepackage{eurosym}
\usepackage{multirow}
\usepackage[ngerman]{varioref}
\setcapindent{1em}
\renewcommand{\labelitemi}{--}
\usepackage{paralist}
\usepackage{pdfpages}
\usepackage{lscape}
\usepackage{float}
\usepackage{acronym}
\usepackage{eurosym}
\usepackage{longtable,lscape}
\usepackage{mathpazo}
\usepackage[normalem]{ulem} %emphasize weiterhin kursiv
\usepackage[flushmargin,ragged]{footmisc} % left align footnote
\usepackage{ccicons} 
\setcapindent{0pt} % no indentation in captions

%%%% fancy LIBREAS URL color 
\usepackage{xcolor}
\definecolor{libreas}{RGB}{112,0,0}

\usepackage{listings}

\urlstyle{same}  % don't use monospace font for urls

\usepackage[fleqn]{amsmath}

%adjust fontsize for part

\usepackage{sectsty}
\partfont{\large}

%Das BibTeX-Zeichen mit \BibTeX setzen:
\def\symbol#1{\char #1\relax}
\def\bsl{{\tt\symbol{'134}}}
\def\BibTeX{{\rm B\kern-.05em{\sc i\kern-.025em b}\kern-.08em
    T\kern-.1667em\lower.7ex\hbox{E}\kern-.125emX}}

\usepackage{fancyhdr}
\fancyhf{}
\pagestyle{fancyplain}
\fancyhead[R]{\thepage}

% make sure bookmarks are created eventough sections are not numbered!
% uncommend if sections are numbered (bookmarks created by default)
\makeatletter
\renewcommand\@seccntformat[1]{}
\makeatother

% typo setup
\clubpenalty = 10000
\widowpenalty = 10000
\displaywidowpenalty = 10000

\usepackage{hyperxmp}
\usepackage[colorlinks, linkcolor=black,citecolor=black, urlcolor=libreas,
breaklinks= true,bookmarks=true,bookmarksopen=true]{hyperref}
\usepackage{breakurl}

%meta
%meta

\fancyhead[L]{H. Ehrlicher, F. Baetcke\\ %author
LIBREAS. Library Ideas, 38 (2020). % journal, issue, volume.
\href{https://doi.org/10.18452/23475}{\color{black}https://doi.org/10.18452/23475}
{}} % doi 
\fancyhead[R]{\thepage} %page number
\fancyfoot[L] {\ccLogo \ccAttribution\ \href{https://creativecommons.org/licenses/by/4.0/}{\color{black}Creative Commons BY 4.0}}  %licence
\fancyfoot[R] {ISSN: 1860-7950}

\title{\LARGE{Was bisher geschah und was damit erreicht wurde: Ein Zwischenbericht zur Kampagne Biblio2030 des Schweizer
Bibliotheksverbands Bibliosuisse}}% title
\author{Heike Ehrlicher \& Franziska Baetcke} % author

\setcounter{page}{1}

\hypersetup{%
      pdftitle={Was bisher geschah und was damit erreicht wurde: Ein Zwischenbericht zur Kampagne Biblio2030 des Schweizer
Bibliotheksverbands Bibliosuisse},
      pdfauthor={Heike Ehrlicher, Franziska Baetcke},
      pdfcopyright={CC BY 4.0 International},
      pdfsubject={LIBREAS. Library Ideas, 38 (2020).},
      pdfkeywords={Bibliothek, Bibliothekskampagne, Bibliotheksverbund, Schweiz, Projektbericht},
      pdflicenseurl={https://creativecommons.org/licenses/by/4.0/},
      pdfcontacturl={http://libreas.eu},
      baseurl={http://libreas.eu},
      pdflang={de},
      pdfmetalang={de}
     }



\date{}
\begin{document}

\maketitle
\thispagestyle{fancyplain} 

%abstracts

%body
\hypertarget{i}{%
\section{I}\label{i}}

Angefangen hat alles an einem trüben Oktobermorgen im Jahr 2017. Wien,
Museumsquartier, Geschäftsstelle des Büchereiverbands Österreich. Für
den Advocacy-Workshop zum Thema «Rolle der Bibliotheken bei der
Zielerfüllung der UNO Agenda 2030» haben sich 22 Personen aus
Deutschland, Österreich, der Schweiz und Südtirol im Sitzungszimmer
versammelt. Ganz unterschiedliche Bibliothekswelten treffen an diesem
Vormittag aufeinander, und nicht für alle Beteiligten mag der Anlass
eine Initiation gewesen sein. Für uns Vertreterinnen aus der Schweiz
aber schon. Im Zentrum des Workshops steht eine von der IFLA entworfene
Präsentation, in der die UNO Agenda 2030 und das Set der 17
Nachhaltigkeitsziele (Sustainable Development Goals, SDGs) vorgestellt
sowie das Thema Nachhaltigkeit und Bibliotheken entwickelt wird. Wir
sitzen da und sind hellwach: Aha, so könnte das gehen. Wenn die
Nachhaltigkeit als Teil der DNA von Bibliotheken verstanden wird, dann
können die Bibliotheken bei der nachhaltigen Entwicklung gar nicht
abseits stehen. Dann sind sie immer schon mittendrin. Sie sind
Akteurinnen, selbst wenn sie sich dessen nicht bewusst sind. Sie sind
Teil der Bewegung, Teil der Lösung, immer schon auf dem Weg zum Ziel.

Wien ist ein Aha-Erlebnis. Und nach dem Workshop sind wir
zuversichtlich, einen brauchbaren Ansatz für die Arbeit mit den
Bibliotheken in der Schweiz erhalten zu haben und bei einem
vielversprechenden Auftakt für eine internationale Zusammenarbeit dabei
gewesen zu sein.

\hypertarget{ii}{%
\section{II}\label{ii}}

2017 hat die Arbeitsgruppe, die sich da noch IAP Suisse nennt, bereits
einmal getagt. In ihrer Zusammensetzung zeigt sich von Anfang an, was
auch die Kommission Biblio2030, die sich relativ rasch daraus
entwickelt, auszeichnet: Am Thema Nachhaltigkeit sind ganz
unterschiedliche Menschen interessiert. Vertreterinnen und Vertreter
aller Bibliothekstypen sitzen mit am Tisch, der Rahmen ist weit
gesteckt, auch die Stiftung SDSN Switzerland ist vertreten, sowie die
Stiftung biblio-suisse, die sich im Raum Luzern um Umweltkommunikation
bemüht. Von Anfang an zeichnet die Kommissionsarbeit aus, dass sie
multiperspektivisch ist, offen, neugierig, kritisch.

Zu Beginn dieser Kommissionsarbeit steht der interne Findungsprozess. Es
geht um die Klärung, was wir hier eigentlich tun: Wird erwartet, dass
die Bibliotheken, die im Kontext der Agenda 2030 von der UNO zu
Partnerorganisationen geadelt wurden, die Kommunikationsarbeit für die
UNO in Sachen Nachhaltigkeitsziele übernehmen? Und, wenn dieses
Verständnis existierte, liesse sich das mit der Unabhängigkeit von
Bibliotheken überhaupt vereinbaren? Müssen Bibliotheken nicht gerade
auch gegenüber einer mächtigen, weltumspannenden Organisation wie der
UNO unabhängig, frei und kritisch bleiben können? Oder andersherum
gedacht: Steckt in dem vorliegenden Setting nicht vielmehr ein grosses
Versprechen für die Bibliotheken? Dass sich nämlich ihr Engagement im
Rahmen der Agenda 2030 für sie auch marketingmässig auszahlen könnte,
weil sie über das brisante und emotionale Thema der nachhaltigen
Entwicklung (endlich) auch in der Politik und der breiteren
Öffentlichkeit als relevante gesellschaftliche Player wahrgenommen
würden?

Die Kommissionsmitglieder Biblio2030 entscheiden sich für die Lesart, in
der die Bibliotheken den Ball inhaltlich aufgreifen und sich vom
Engagement auch Rückenwind für ihre Positionierung versprechen. Vor den
sinnbildlichen Karren von UNO oder nationaler Politik lassen sie sich
nicht spannen.

\hypertarget{iii}{%
\section{III}\label{iii}}

Die Bandbreite des Themas Nachhaltigkeit ist motivierend -- aber auch
tückisch. Man kann sich darin verlieren. Nachhaltigkeit ist ein bisschen
alles und nichts. Und was genau könnte die Aufgabe der Bibliotheken
sein? Die Kommission Biblio2030 setzt den Fokus ihrer Arbeit auf die
Sensibilisierung. Die Kolleg*innen in den Bibliotheken sollen mit dem
Thema Nachhaltigkeit in Berührung kommen -- in ihrem professionellen
Kontext. Denn spätestens seit 2018 kann sich keine*r dem Thema in den
Medien und auf den Strassen mehr entziehen.

Das ist die zentrale Achse: Kommunikation nach innen mit
Weiterbildungsangeboten im regulären Programm des Verbands, mit einem
kleinen Sortiment an -- aus Kommissionssicht -- sinnvoller Merchandise,
mit einer Microsite, einem Online-Werkzeugkasten, der Literatur, Tipps,
Vorlagen, et cetera enthält und der in der Zwischenzeit zu einem
anwendungsfreundlichen Online-Tool weiterentwickelt wurde:
\url{https://padlet.com/heikeehrlicher/Biblio2030}

Daneben werden die Achsen Advocacy und Partnerschaften bewirtschaftet.
Der Bund, insbesondere das zuständige Amt für Raumentwicklung, das vom
Bundesrat mit der Umsetzung der Agendaziele beauftragt wurde, soll
Kenntnis von den Bibliotheken nehmen, und verschiedene Stiftungen werden
um Drittmittel angefragt, um die Aktionen von Biblio2030 -- mittlerweile
wird von einer Kampagne gesprochen -- zu finanzieren.

Drittens sollen zivilgesellschaftliche Akteure, viele davon NGO aus dem
Entwicklungsbereich, als Partner angesprochen werden. Biblio2030
schliesst sich konkret der Plattform Agenda 2030 an und trägt aktiv zu
deren Ausgestaltung bei: \url{https://www.plattformagenda2030.ch/}

Im Fokus der Kommissionsarbeit steht die Sichtbarmachung. Die
Sichtbarmachung der SDGs in den Bibliotheken, und die Sichtbarmachung
der Bibliotheken, ohne die die SDGs nicht zu den Leuten finden. Diese
Doppeldeutigkeit ist dem Thema Nachhaltigkeit und Bibliotheken inhärent.
Ob das Versprechen sich einlösen lässt, ob die Bibliotheken wirklich
davon profitieren werden, dass sie sich mit dem Thema Nachhaltigkeit zu
positionieren versuchen, wird sich langfristig zeigen.

\hypertarget{iv}{%
\section{IV}\label{iv}}

2020 sind die meisten Weiterbildungstermine wegen Corona abgesagt
worden. Das hat zur Folge, dass die grössten Auftritte der Kommission
Biblio2030 nach wie vor das international besetzte Podium zum Thema am
Bibliothekskongress in Montreux im August 2018 und die Keynote von
Barbara Lison mit anschliessendem Workshop an der
Bibliosuisse-Generalversammlung im Mai 2019 sind. Seither hätte
natürlich viel geschehen sollen! Geplant war insbesondere ein für
Bibliosuisse neues Veranstaltungsformat im Weiterbildungsprogramm: die
«Roadshow».

Angedacht als Weiterbildung to go soll die Roadshow in drei Stunden
handfeste Informationen zur Agenda und den SDGs vermitteln, und vor
allem auch die Frage «Was hat das alles mit mir zu tun?» aufwerfen. Das
Ziel der Roadshows ist, bei den Teilnehmenden einen unmittelbaren Bezug
zur Fragestellung zu erzeugen. Dabei die Aufmerksamkeit auf die
Bibliothek als Vermittlerin des Themas gegenüber der Öffentlichkeit
ebenso zu richten wie auf die Frage nach der Verantwortung der
Bibliothek, als Organisation selbst nachhaltig zu sein, das heisst
nachhaltig zu konsumieren und zu wirtschaften oder auch nachhaltige
Personalpolitik zu betreiben. Diese Perspektive ist insofern
interessant, als Bibliotheken als öffentliche Institutionen ja auch
vorbildhaft wirken können.

Die geplanten Roadshows werden nun erst im Herbst 2020 und Frühjahr 2021
stattfinden. Dafür hat die Corona-Pause Zeit für Reflexion und Lektüre
gebracht. Fragen, auf die im Alltag keine schnellen Antworten gefunden
wurden, konnten vertieft werden. Wieso sind zum Beispiel
Wissenschaftliche Bibliotheken weniger aufgeschlossen gegenüber
Nachhaltigkeitsprojekten? Weil diese Bibliotheken grösseren
Verwaltungseinheiten angeschlossen sind und sich nach zentralen, in der
Regel kantonalen Standards richten müssen. Veränderungen müssen
langfristiger geplant werden, oft sind die Treiber dann nicht die
Bibliotheken, sondern die Hochschulen selbst. Zudem wollen Hochschulen
den konkreten Nutzen von Nachhaltigkeitsprojekten für Lehre und
Forschung sehen, bevor sie darauf einsteigen.

Wer ein wenig über den Bibliothekstellerrand hinaus schaut, sieht zudem,
wie andere Kulturinstitutionen mit dem Wunsch nach mehr Nachhaltigkeit
umgehen: Museen, Schulen, kleine und mittlere Unternehmen (KMU).
Vielerorts haben Workshops stattgefunden, Mitarbeitende wurden
einbezogen, Ideen gesammelt, Best Practice eruiert -- Bibliotheken sind
in dieser Beziehung ja kein Sonderfall. Die Entwicklung zu einer
grüneren Organisation ist ein Prozess, der sich aus vielen kleinen
Schritten zusammensetzt. Zurzeit entstehen auch in der Schweiz erste
Leitfäden für Organisationen, wie sie ihre nachhaltige Entwicklung
gestalten können.

\hypertarget{v}{%
\section{V}\label{v}}

Biblio2030 ist eine Kampagne für die Mitglieder von Bibliosuisse und
alle Bibliotheken im ganzen Land. Die Kommission Biblio2030 unterstützt
die Mitglieder inhaltlich und mit konkreten Angeboten, Empfehlungen und
Materialien dabei, ihr Engagement für die nachhaltige Entwicklung
umzusetzen. Wo steht die Kommissionsarbeit nach knapp drei Jahren mit
dem Thema? Was ist gelungen, was ist schwierig und warum?

Gespräche mit Kolleg*innen und Studierenden der Informationswissenschaft
bestätigen uns darin, dass Biblio2030 gut unterwegs ist. Die Grundlagen
für eine Kampagne sind gelegt: Das Thema ist in vielen Bibliotheken
lanciert, nachhaltiges Leben ist vielen Verantwortlichen und
Mitarbeitenden in Bibliotheken ein echtes Anliegen. Das eigene
Verhalten, das Wirken der Organisation wird zunehmend auch nach grünen
Gesichtspunkten und Kriterien der sozialen Gerechtigkeit untersucht und
neue Wege werden beschritten. Zudem ist Nachhaltigkeit als Gegenstand
von studentischen Arbeiten im Fach Informationswissenschaft keine
Seltenheit mehr. Das alles ist ein schöner Erfolg. Aber, ganz ehrlich,
ist der Funke wirklich übergesprungen? Verstehen sich die Bibliotheken
in der Schweiz tatsächlich bereits als Akteurinnen und Plattformen für
die Debatte über Nachhaltigkeit? Und finden sich in den Öffentlichen und
Wissenschaftlichen Bibliotheken in der Schweiz auch Angebote, die über
den Medienverleih zum Thema, Informationsangebote und Veranstaltungen
hinausgehen? Da sind wir realistisch: Einzelne Beispiele sind vorhanden,
und es gibt Bibliotheken, die ein ganzheitliches Verständnis von
Nachhaltigkeit entwickelt haben und mutig Neues ausprobieren. Aber
flächendeckend ist das gewiss noch nicht der Fall. Die SDG-Welle ist
noch nicht über die Bibliotheken geschwappt, weiterhin bleibt viel zu
tun.

Das hängt nicht zuletzt auch damit zusammen, dass nachhaltige
Entwicklung als Prozess zu verstehen ist und nicht wie ein Befehl von
oben erteilt werden kann. Nur, wer für sich selbst nachvollzogen hat,
warum Nachhaltigkeit DAS Thema des Augenblicks ist, wird andere --
Kolleg*innen, Vorgesetzte, Benutzer*innen -- dafür begeistern können.
Das braucht Zeit, Gespräche, Austausch, Selbstkritik, Reflexion,
Lektüre, Inspiration, et cetera.

Bibliosuisse und insbesondere die Kommission Biblio2030 verstehen sich
als Gesprächspartner für alle, die sich im Kontext Nachhaltigkeit und
Bibliotheken engagieren wollen, die nach Lösungen für ein Problem
suchen, das nur gemeinsam angegangen werden kann und die bereit sind,
ihre eigenen Handlungsspielräume mutig zu eruieren und beherzt
auszuschöpfen.

%autor
\begin{center}\rule{0.5\linewidth}{0.5pt}\end{center}

\textbf{Heike Ehrlicher und Franziska Baetcke} bilden gemeinsam mit
Amélie Vallotton Preisig das Präsidium der Kommission Biblio2030 des
Schweizer Bibliotheksverbands Bibliosuisse.

\textbf{Heike Ehrlicher} ist stellvertretende Geschäftsführerin von
Bibliosuisse (\url{http://www.bibliosuisse.ch/})

\textbf{Franziska Baetcke} ist Direktorin der Stiftung Bibliomedia
Schweiz (\url{http://www.bibliomedia.ch/de}) und eines von 16
Vorstandsmitgliedern von Bibliosuisse.

\end{document}
