\documentclass[a4paper,
fontsize=11pt,
%headings=small,
oneside,
numbers=noperiodatend,
parskip=half-,
bibliography=totoc,
final
]{scrartcl}

\usepackage[babel]{csquotes}
\usepackage{synttree}
\usepackage{graphicx}
\setkeys{Gin}{width=.4\textwidth} %default pics size

\graphicspath{{./plots/}}
\usepackage[ngerman]{babel}
\usepackage[T1]{fontenc}
%\usepackage{amsmath}
\usepackage[utf8x]{inputenc}
\usepackage [hyphens]{url}
\usepackage{booktabs} 
\usepackage[left=2.4cm,right=2.4cm,top=2.3cm,bottom=2cm,includeheadfoot]{geometry}
\usepackage{eurosym}
\usepackage{multirow}
\usepackage[ngerman]{varioref}
\setcapindent{1em}
\renewcommand{\labelitemi}{--}
\usepackage{paralist}
\usepackage{pdfpages}
\usepackage{lscape}
\usepackage{float}
\usepackage{acronym}
\usepackage{eurosym}
\usepackage{longtable,lscape}
\usepackage{mathpazo}
\usepackage[normalem]{ulem} %emphasize weiterhin kursiv
\usepackage[flushmargin,ragged]{footmisc} % left align footnote
\usepackage{ccicons} 
\setcapindent{0pt} % no indentation in captions

%%%% fancy LIBREAS URL color 
\usepackage{xcolor}
\definecolor{libreas}{RGB}{112,0,0}

\usepackage{listings}

\urlstyle{same}  % don't use monospace font for urls

\usepackage[fleqn]{amsmath}

%adjust fontsize for part

\usepackage{sectsty}
\partfont{\large}

%Das BibTeX-Zeichen mit \BibTeX setzen:
\def\symbol#1{\char #1\relax}
\def\bsl{{\tt\symbol{'134}}}
\def\BibTeX{{\rm B\kern-.05em{\sc i\kern-.025em b}\kern-.08em
    T\kern-.1667em\lower.7ex\hbox{E}\kern-.125emX}}

\usepackage{fancyhdr}
\fancyhf{}
\pagestyle{fancyplain}
\fancyhead[R]{\thepage}

% make sure bookmarks are created eventough sections are not numbered!
% uncommend if sections are numbered (bookmarks created by default)
\makeatletter
\renewcommand\@seccntformat[1]{}
\makeatother

% typo setup
\clubpenalty = 10000
\widowpenalty = 10000
\displaywidowpenalty = 10000

\usepackage{hyperxmp}
\usepackage[colorlinks, linkcolor=black,citecolor=black, urlcolor=libreas,
breaklinks= true,bookmarks=true,bookmarksopen=true]{hyperref}
\usepackage{breakurl}

%meta
%meta

\fancyhead[L]{E. Hermann\\ %author
LIBREAS. Library Ideas, 38 (2020). % journal, issue, volume.
\href{http://nbn-resolving.de/}
{}} % urn 
% recommended use
%\href{http://nbn-resolving.de/}{\color{black}{urn:nbn:de...}}
\fancyhead[R]{\thepage} %page number
\fancyfoot[L] {\ccLogo \ccAttribution\ \href{https://creativecommons.org/licenses/by/4.0/}{\color{black}Creative Commons BY 4.0}}  %licence
\fancyfoot[R] {ISSN: 1860-7950}

\title{\LARGE{Naturschutzgebiet Bibliothek}}% title
\author{Elisa Herrmann} % author

\setcounter{page}{1}

\hypersetup{%
      pdftitle={Naturschutzgebiet Bibliothek},
      pdfauthor={Elisa Herrmann},
      pdfcopyright={CC BY 4.0 International},
      pdfsubject={LIBREAS. Library Ideas, 38 (2020).},
      pdfkeywords={Bibliothek, Biodiversität, Biodiversity Heritage Library, Schädlinge, Insekten, Schimmelpilze, ausgestorbene Tierarten},
      pdflicenseurl={https://creativecommons.org/licenses/by/4.0/},
      pdfcontacturl={http://libreas.eu},
      baseurl={http://libreas.eu},
      pdflang={de},
      pdfmetalang={de}
     }



\date{}
\begin{document}

\maketitle
\thispagestyle{fancyplain} 

%abstracts
\begin{abstract}
\noindent
\textbf{Kurzfassung}: Die Biodiversität in Bibliotheken ist oft erst auf
den zweiten Blick erkennbar. In, zwischen und auf Büchern haben ganz
besondere Arten ihre biologische Nische eingerichtet -- teils vom
Menschen gewollt, teils ungewollt. Der Beitrag gibt einen Einblick in
die Vielfalt an Organismen in Bibliotheken und die Rolle von
Informationsinfrastrukturen zum Schutz von Flora und Fauna der Welt.

\begin{center}\rule{0.5\linewidth}{0.5pt}\end{center}

\textbf{Abstract}: Biodiversity in libraries is often only visible at
second glance. Very special species have established their biological
niche in, on and between books - partly wanted by humans, partly
unwanted. This article provides an insight into the diversity of
organisms in libraries and the role of information infrastructures in
the protection of the earth's flora and fauna.
\end{abstract}

%body
\enquote{Als Wüste werden die vegetationslosen oder vegetationsarmen
Gebiete der Erde bezeichnet.}\footnote{Wikipedia: Wüste.
  \url{https://de.wikipedia.org/wiki/W\%C3\%BCste} (zuletzt aufgerufen
  am 20.08.2020)} Ich zähle die Pflanzen im Hauptraum der Bibliothek:
Ein alter Hibiskus, eine Grünlilie und ein Kaktus auf der Fensterbank.
Trocken ist es hier auch. Und ein bisschen staubig. Und doch: So wie die
Wüste lebt, so lebt auch die Bibliothek. Oftmals im Verborgenen bietet
sie den Lebensraum für Organismen, die auf, zwischen und teilweise nur
in den Büchern leben.

Hinter den schweren Türen der Panzerschränke dokumentieren reich
illustrierte und aufwändig kolorierte Rara-Bände die Tier- und
Pflanzenwelt der Erde -- vermutlich sind hier mehr Arten zu sehen als im
artenreichsten Zoo Europas. Auf jeden Fall gibt es in Bibliotheken Tiere
zu sehen, die in keinem Tierpark zu bestaunen sind: Völlig
selbstverständlich stehen Einhörner neben Nashörnern im
Gessner\footnote{Gessner, Conrad: Historiae animalvm. Digital verfügbar
  in der Biodiversity Heritage Library.
  \url{https://doi.org/10.5962/bhl.title.125499} (zuletzt aufgerufen am
  20.08.2020)}, fabelhafte Tierwesen verzaubern in Romanen und die 1976
\enquote{entdeckte} Steinlaus (\emph{Petrophaga lorioti)} lebt im
Pschyrembel\footnote{Pschyrembel Online: Steinlaus.
  \url{https://www.pschyrembel.de/Steinlaus/K0LHT} (zuletzt aufgerufen
  am 20.08.2020)} fort.

Wen wundert es da, dass Bibliotheken Tiere als Schutzsymbol oder
Maskottchen ausgewählt haben? Neben dem \enquote{Bücherwurm} sind es vor
allem Eulen, die als Symbol der griechischen Göttin der Weisheit,
Athene, zahlreiche Bibliotheken wie die Library of Congress zieren. In
den vergangenen Jahren haben eher Vierbeiner zwischen den Regalreihen
ihr Zuhause gefunden. Neben Therapiehunden vor allem über 300 offizielle
Bibliothekskatzen.\footnote{Open Education Database: A Quick Guide to
  Library Cats.
  \url{https://oedb.org/ilibrarian/quick-guide-library-cats/} (zuletzt
  aufgerufen am 20.08.2020)} Etwas exotischer fiel die Wahl der
Haustiere in der Ernst Mayr Library des Museum of Comparative Zoology
der Harvard Universität aus. \emph{Gromphadorhina portentosa,} eher
bekannt als Madagaskar-Fauchschabe, kann von Studierenden an den
Leseplatz ausgeliehen werden (im Terrarium), sollte Therapiehund Cooper
in der benachbarten Countway Library mal ausgebucht sein. \footnote{Ernst
  Mayr Library (2012): Library Pets.
  \url{https://library.mcz.harvard.edu/blog/library-pets} (zuletzt
  abgerufen am 20.08.2020)}

So sehr wir bestimmte Arten verehren, um das Wissen in
Bibliothekssammlungen (symbolisch) zu schützen, so sehr bemühen wir uns,
andere Organismen aus den Räumen fernzuhalten. Schädlinge wie Nagetiere
oder Wanzen ernähren sich gerne von pflanzlichen Materialien (Papier)
und stellen eine Bedrohung für die Bestandserhaltung dar. Bislang ist
noch keine Art bekannt, die ausschließlich in Bibliotheken lebt (im
Gegensatz zu einer Spinne, deren einziger natürlicher Lebensraum das
Finnische Naturkundemuseum ist\footnote{Nicholls, Henry (2016): The
  museum filled with venomous spiders that just won't die.
  \url{https://www.bbc.com/future/article/20160413-the-museum-filled-with-poisonous-spiders-that-just-wont-die}
  (zuletzt aufgerufen am 20.08.2020)}), und doch gibt es Arten, die in
Archiven und Bibliotheken einen paradiesischen Lebensraum gefunden
haben, zum Beispiel der Papiersilberfisch (\emph{Ctenolepisma
longicaudata}). Dieses synantrophe Tier ernährt sich fast ausschließlich
von Papier und Karton indem es Zellulosen mit körpereigener Cellulase in
Zucker zerlegt, der dann verdaut werden kann. Papier-Silberfische sind
in ihrer Umgebung nicht sehr anspruchsvoll (Temperaturen zwischen 20 und
24 Grad Celsius, trockene Umgebung, meidet Licht) und können bis zu 300
Tage ohne Nahrung überleben -- ein Albtraum für das \emph{Integrated
Pest Management}. Und wenn die Umgebung von trocken zu feucht wechselt
und damit für den Papiersilberfisch unwirtlich wird, erobern
(Schimmel-)Pilze das Biotop \enquote{Buch}.

Weder \emph{Ctenolepisma longicaudata} noch Schimmelpilze werden von der
International Union for Conservation of Nature (IUCN) auf der Roten
Liste der gefährdeten Arten\footnote{IUCN, International Union for
  Conservation of Nature: The IUCN Red List Of Threatened Species.
  \url{https://www.iucnredlist.org/} (zuletzt abgerufen 20.08.2020)}
geführt. Einige der Bücher, die sie bedrohen, enthalten jedoch Hinweise
auf ausgestorbene, bedrohte oder vom Aussterben bedrohte Tiere, Pflanzen
und andere Organismen. Als der berühmte Illustrator John Gould 1838
seine Expedition nach Australien unternahm, dokumentierte er die
eindrucksvolle Tierwelt des roten Kontinents. Auf über 180 Tafeln
stellten sie Säugetiere, die ihnen auf der Expedition begegneten, in
möglichst natürlicher Umgebung dar und publizierten diese mit
Begleittexten im Werk \enquote{The Mammals of Australia}.\footnote{Gould,
  John: The Mammals of Australia. Digital verfügbar in der Biodiversity
  Heritage Library: \url{https://doi.org/10.5962/bhl.title.112962}
  (zuletzt aufgerufen am 20.08.2020)} Aus heutiger Sicht dokumentiert es
die schwindende Artenvielfalt Australiens. Die letzten Aufzeichnungen
über das Wallaby wurden 1930 in Westaustralien verifiziert und 1956
wurde das Mondnagelkänguruh für ausgestorben erklärt. Die einzige
verbreitete Darstellung dieser Art ist die Goulds Abbildung in
\enquote{The Mammals of Australia}.\footnote{Gould, John: Onychogalea
  Lunata. In: The Mammals of Australia. Bd.2, Tafel LV. Digital
  verfügbar in der Biodiversity Heritage Library.
  \url{https://www.biodiversitylibrary.org/page/49740861} (zuletzt
  abgerufen am 20.08.2020)}

Letztlich kann man von Glück reden, dass die Art zumindest dokumentiert
wurde. ExpertInnen gehen heute davon aus, dass lediglich 14\% der Arten
auf dem Land und nur 9\% der Arten im Wasser bekannt sind.\footnote{Mora
  et al.~(2011): How Many Species Are There on Earth and in the Ocean.
  \url{https://doi.org/10.1371/journal.pbio.1001127} (zuletzt aufgerufen
  am 20.08.2020)} Die Biodiversitätsforschung befindet sich also in
einem Wettlauf gegen die Zeit und braucht starke Partner:
Informationseinrichtungen. Bibliotheken und Archive bewahren das Wissen
um die bisher bekannte Flora und Fauna der Erde und liefern so wichtige
Grundlagen für Artbeschreibungen, die Dokumentation des Klimawandels und
Forschungsmethoden allgemein.

Insbesondere in naturwissenschaftlichen Bibliotheken und Archiven
eröffnet sich den Lesenden eine teils vergessene Tier- und Pflanzenwelt.
Um den Zugang zu diesen Quellen für die Biodiversitätsforschung zu
erleichtern, gründeten 2006 zehn Bibliotheken, Archive und
Informationseinrichtungen aus Naturkundemuseen, Botanischen Gärten und
ähnlichen Einrichtungen das Konsortium der Biodiversity Heritage Library
(BHL).\footnote{Biodiversity Heritage Library: History of BHL.
  \url{https://about.biodiversitylibrary.org/about/history-of-bhl/}
  (zuletzt aufgerufen am 20.08.2020)} Seither wurden über die BHL circa
58 Mio. Seiten, aus 160.000 Titeln beziehungsweise 261.000 Bänden
digitalisiert und open access zur Verfügung gestellt. Die mittlerweile
20 Konsortiumsmitglieder\footnote{Biodiversity Heritage Library: BHL
  Consortium.
  \url{https://about.biodiversitylibrary.org/about/bhl-consortium/}
  (zuletzt aufgerufen am 20.08.2020)}, darunter das Museum für
Naturkunde Berlin als einzige Institution aus dem deutschsprachigen
Raum, engagieren sich zudem in Verlagsverhandlungen, um urheberrechtlich
geschützte Literatur bereits vor dem Ablauf der Schutzfrist open access
anzubieten.

Neben der reinen Literaturbereitstellung werden die darin enthaltenen
Informationen aufbereitet und mit Daten aus anderen Datenportalen,
insbesondere aus dem Bereich Biodiversität, verlinkt. So sind die
OCR-Volltexte in der BHL alle auf taxonomische Namen automatisch
analysiert und mit den Einträgen in der Encyclopedia of Life verknüpft.
Dies ermöglicht es zum einen die Artbeschreibung in ihrem historischen
Kontext darstellen zu können und zum anderen bietet es Forschenden einen
weiteren Zugang zu weiterführenden Informationen zu einer Art.

Wie wichtig die freie Zugänglichkeit zu Biodiversitätsliteratur ist, hat
die SARS-CoV-2-Pandemie in jüngster Zeit gezeigt. Wie in vielen anderen
Bereichen unseres Lebens waren die gewohnten Wege nicht mehr gangbar --
Archive und Bibliotheken eine Weile geschlossen. Die Versorgung von
Forschenden, Studierenden, PädagogInnen und SchülerInnen mit
wissenschaftlicher Literatur war nur über elektronische Ressourcen
möglich und Open-Access-Bibliotheken wie die BHL haben schwere
Einschnitte in der Literaturversorgung abgemildert. Zusätzlich zu ihren
verschiedenen Tools und Services für die wissenschaftliche Arbeit,
stellte das BHL-Konsortium Material zusammen, um den Fernunterricht
durch kuratierte Sammlungen, Bildressourcen, aber auch Malbücher zu
unterstützen. Die MitarbeiterInnen der BHL haben zudem stark an der
Verbesserung des Zugangs zu den Sammlungen gearbeitet, indem sie die
Metadaten für digitalisierte Bände und die darin enthaltenen
Bildnachweise verbessern.

Neben der Bewahrung des Wissens über die biologische Vielfalt engagiert
sich die BHL auch aktiv für die Sensibilisierung der Öffentlichkeit in
Naturschutzfragen. Gegenwärtig untersucht die \emph{Earth Optimism
Campaign}\footnote{Biodiversity Heritage Library Blog: Earth Optimism
  2020.
  \url{https://blog.biodiversitylibrary.org/category/campaigns/earth-optimism-2020}
  (zuletzt aufgerufen am 20.08.2020)} das Leben und die Arbeit von
Personen, die entscheidend dazu beigetragen haben, den Naturschutz in
Wissenschaft und Gesellschaft zu etablieren. Ursprünglich für April 2020
geplant und wegen der SARS-CoV-2-Pandemie verschoben, wurde die Kampagne
im Juli aufgenommen und wird bis Ende dieses Jahres 2020 wegweisende
Literatur zum Naturschutz beleuchten.

Wir haben gesehen, dass Bibliotheken für einige Tiere, Pflanzen und
Organismen ein realer Lebensraum sind. Ausgestorbene Arten leben in den
Büchern weiter und manche leben ausschließlich auf dem Papier. Sie
engagieren sich (nicht nur) im Rahmen der Sustainable Development Goals
(SDGs) der Vereinten Nationen\footnote{United Nations: The 17 Goals.
  \url{https://sdgs.un.org/goals} (zuletzt aufgerufen am 20.08.2020)}
für den Erhalt von Biodiversität und tragen so zu einem ökologischerem
Bewusstsein bei. Bibliotheken sind daher gar nichts so weit weg von den
Merkmalen von Naturschutzgebieten, die \enquote{\ldots{} aus
wissenschaftlichen, naturwissenschaftlichen {[}\ldots{]} oder wegen
ihrer Seltenheit, besonderen Eigenart oder hervorragenden Schönheit.
\ldots{}}\footnote{Bundesnaturschutzgesetz (BNatSchG): §23
  Naturschutzgebiete.
  \url{http://www.gesetze-im-internet.de/bnatschg_2009/__23.html}
  (zuletzt aufgerufen am 20.08.2020)} schützenswert sind.

Dieser Artikel entstand mit großer Unterstützung von Constance Rinaldo,
Librarian of the Ernst Mayr Library and Museum of Comparative Zoology
Archives und Tomoko Y. Steen, Ph.D., Department of Microbiology and
Immunology, Georgetown University.

%autor
\begin{center}\rule{0.5\linewidth}{0.5pt}\end{center}

\textbf{Elisa Herrmann}, M.A.~LIS, ist seit 2019
Wissenschaftliche Leiterin der Bibliothekssammlung und
Informationsversorgung am Museum für Naturkunde -- Leibniz-Institut für
Evolutions- und Biodiversitätsforschung in Berlin. Der Schwerpunkt ihrer
Arbeit liegt in der Entwicklung der Bibliothek als moderne
Informationsinfrastruktur für die Forschung am Museum. Sie betreut im
Rahmen der Sammlungserschließung des Museums die Retrodigitalisierung
der Bibliotheksbestände und ist in diesem Zusammenhang auch an der
strategischen Weiterentwicklung der Biodiversity Heritage Library
beteiligt.

\end{document}
