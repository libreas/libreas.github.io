\documentclass[a4paper,
fontsize=11pt,
%headings=small,
oneside,
numbers=noperiodatend,
parskip=half-,
bibliography=totoc,
final
]{scrartcl}

\usepackage[babel]{csquotes}
\usepackage{synttree}
\usepackage{graphicx}
\setkeys{Gin}{width=.4\textwidth} %default pics size

\graphicspath{{./plots/}}
\usepackage[ngerman]{babel}
\usepackage[T1]{fontenc}
%\usepackage{amsmath}
\usepackage[utf8x]{inputenc}
\usepackage [hyphens]{url}
\usepackage{booktabs} 
\usepackage[left=2.4cm,right=2.4cm,top=2.3cm,bottom=2cm,includeheadfoot]{geometry}
\usepackage{eurosym}
\usepackage{multirow}
\usepackage[ngerman]{varioref}
\setcapindent{1em}
\renewcommand{\labelitemi}{--}
\usepackage{paralist}
\usepackage{pdfpages}
\usepackage{lscape}
\usepackage{float}
\usepackage{acronym}
\usepackage{eurosym}
\usepackage{longtable,lscape}
\usepackage{mathpazo}
\usepackage[normalem]{ulem} %emphasize weiterhin kursiv
\usepackage[flushmargin,ragged]{footmisc} % left align footnote
\usepackage{ccicons} 
\setcapindent{0pt} % no indentation in captions

%%%% fancy LIBREAS URL color 
\usepackage{xcolor}
\definecolor{libreas}{RGB}{112,0,0}

\usepackage{listings}

\urlstyle{same}  % don't use monospace font for urls

\usepackage[fleqn]{amsmath}

%adjust fontsize for part

\usepackage{sectsty}
\partfont{\large}

%Das BibTeX-Zeichen mit \BibTeX setzen:
\def\symbol#1{\char #1\relax}
\def\bsl{{\tt\symbol{'134}}}
\def\BibTeX{{\rm B\kern-.05em{\sc i\kern-.025em b}\kern-.08em
    T\kern-.1667em\lower.7ex\hbox{E}\kern-.125emX}}

\usepackage{fancyhdr}
\fancyhf{}
\pagestyle{fancyplain}
\fancyhead[R]{\thepage}

% make sure bookmarks are created eventough sections are not numbered!
% uncommend if sections are numbered (bookmarks created by default)
\makeatletter
\renewcommand\@seccntformat[1]{}
\makeatother

% typo setup
\clubpenalty = 10000
\widowpenalty = 10000
\displaywidowpenalty = 10000

\usepackage{hyperxmp}
\usepackage[colorlinks, linkcolor=black,citecolor=black, urlcolor=libreas,
breaklinks= true,bookmarks=true,bookmarksopen=true]{hyperref}
\usepackage{breakurl}

%meta
%meta

\fancyhead[L]{W. Kaiser\\ %author
LIBREAS. Library Ideas, 38 (2020). % journal, issue, volume.
\href{https://doi.org/10.18452/23477}{\color{black}https://doi.org/10.18452/23477}
{}} % doi 
\fancyhead[R]{\thepage} %page number
\fancyfoot[L] {\ccLogo \ccAttribution\ \href{https://creativecommons.org/licenses/by/4.0/}{\color{black}Creative Commons BY 4.0}}  %licence
\fancyfoot[R] {ISSN: 1860-7950}

\title{\LARGE{Soziale Arbeit mit Wohnungslosen in Bibliotheken: Befürwortung von mehr Kooperationen mit Sozialarbeiter\*innen}}% title
\author{Wolfgang Kaiser} % author

\setcounter{page}{1}

\hypersetup{%
      pdftitle={Soziale Arbeit mit Wohnungslosen in Bibliotheken: Befürwortung von mehr Kooperationen mit Sozialarbeiter\*innen},
      pdfauthor={Wolfgang Kaiser},
      pdfcopyright={CC BY 4.0 International},
      pdfsubject={LIBREAS. Library Ideas, 38 (2020).},
      pdfkeywords={Bibliothek, Obdachlosigkeit, Armut, Soziale Arbeit},
      pdflicenseurl={https://creativecommons.org/licenses/by/4.0/},
      pdfcontacturl={http://libreas.eu},
      baseurl={http://libreas.eu},
      pdflang={en},
      pdfmetalang={en}
     }



\date{}
\begin{document}
\sloppy

\maketitle
\thispagestyle{fancyplain} 

%abstracts
\begin{abstract}
\noindent
Der Beitrag thematisiert die mangelnden Kooperationen von
deutschsprachigen Bibliotheken mit Sozialarbeiter*innen beim Umgang mit
dem Klientel der Wohnungslosen. Es werden Anregungen formuliert, in
welcher Form dies für (öffentliche) Bibliotheken besser gelingen kann
ist. Zudem wird anhand der Sozialen Arbeit in den Bibliotheken in den
USA deutlich gemacht, dass durch die gezielte Nutzbarmachung der
Fähigkeiten und Kenntnisse von Sozialarbeiter*innen Bibliothekar*innen
entlastet werden können beziehungsweise durch Studiengänge wie etwa
Library Social Work eine Symbiose entstehen kann.

\begin{center}\rule{0.5\linewidth}{0.5pt}\end{center}
\noindent
This article refers to the lack of cooperation in germanspeaking
libraries with social workers, in particular regarding homeless people.
Suggestions, how and to what extent it may be useful for (public)
libraries to work together, are formulated. By explaining how social
workers collaborate in libraries in the USA, it discloses that the
skills and the knowledge of social workers can contribute to the relief
of librarians. Respectively study courses like Library Social Work can
create a symbiosis of both professions.
\end{abstract}

%body
\hypertarget{einfuxfchrung-in-die-thematik}{%
\section{Einführung in die
Thematik}\label{einfuxfchrung-in-die-thematik}}

Im vergangenen Jahr entschied sich der wohnungslose Giovanni Maramotti,
der mit den Sicherheitskräften einer bekannten öffentlichen Bibliothek
in Berlin mehrfach aneinandergeriet,\footnote{\url{https://taz.de/Obdachlos-ohne-Krankenversicherung/!5648706/}}
diese Bibliothek nie wieder aufzusuchen. Eigentlich könnte das für
manche Bibliotheksmitarbeiter*in eine gute Nachricht sein, doch diese
Geschichte landete in der Presse und brachte der Einrichtung negative PR
ein.

Es wurde deutlich, dass sich Sicherheitsmitarbeiter*innen um den Mann
\enquote{kümmerten} und nicht etwa Streetworker*innen. An einem
exemplarischen Kommentar des pensionierten Bibliothekars Walter
Scheithauer aus Wien, der auf einen Blogbeitrag von bibliothekarisch.de
aus dem Jahre 2011\footnote{\url{http://blog.bibliothekarisch.de/blog/2011/01/30/die-waermste-bibliothek-aller-zeiten-oder-wie-die-stadtbibliothek-in-hangzhou-auf-twitter-eine-hohe-rensonanz-erfuhr/}}
reagierte, wird klar wie überfordert das Bibliothekspersonal in
Großstadtbibliotheken ist, um mit unterschiedlichen Menschen
menschlicher beziehungsweise professioneller umzugehen.

Scheithauer forderte damals neben Security-Mitarbeiter*innen auch schon
Sozialarbeiter*innen als Unterstützung für das überforderte
Bibliothekspersonal. Denn die Konflikte sind mit Sicherheit keine
Einzelfälle. In Vorbereitung dieses Textes wurden verschiedene
Bibliotheken angeschrieben beziehungsweise über die Mailingliste Forum
ÖB Anfragen gerichtet. Die Offenheit war in den allermeisten Fällen
nicht gegeben, weder für Antworten dem Autor gegenüber, noch für die
Offenheit und Transparenz mit der Thematik umzugehen. Telefonate mit
Bibliothekarinnen erschienen mir äußerst floskelhaft und enthielten
zahlreiche Allgemeinplätze, wie sie in vielen Leitbildern enthalten
sind. Die Stadtbibliothek Stuttgart ging dagegen mit der Thematik
transparent und schonungslos um, so dass es deren Umgang mit der
Problematik sogar als Positivbeispiel in die Stuttgarter Zeitung
schaffte.\footnote{\url{https://www.stuttgarter-zeitung.de/inhalt.streetwork-projekt-europaviertel-in-stuttgart-bibliothek-hilft-jugendlichen-weiter.2bebc08d-625d-4007-9c1d-3c015f47d4b3.html}}

\hypertarget{inwiefern-der-kundenbegriff-die-wahrnehmung-des-wohnungslosen-als-problemkunden-verstuxe4rkt}{%
\section{Inwiefern der Kundenbegriff die Wahrnehmung des
Wohnungslosen als Problemkunden
verstärkt}\label{inwiefern-der-kundenbegriff-die-wahrnehmung-des-wohnungslosen-als-problemkunden-verstuxe4rkt}}

Im Gegensatz zum deutschsprachigen Bibliothekswesen hat sich in der
Sozialen Arbeit der Kundenbegriff nicht durchgesetzt. Christian Stark
kennzeichnete 2006 den Begriff der Kundenorientierung als
\enquote{Mittel zum Zweck} Profite zu erreichen.\footnote{Stark, 2006,
  S. 3} Bezogen auf die Bibliotheksarbeit, heißt das im Umkehrschluss,
dass Kund*innen in der Bibliothek als Konsument*innen betrachtet werden?
In der Sozialen Arbeit dagegen sollte der Mensch in seiner
Ganzheitlichkeit gesehen werden, die Prinzipien lauten Empathie,
Akzeptanz und Authentizität. Bezogen auf Mitarbeiter*innen in
Bibliotheken stellte Carolin Schneider 2006 in ihrer Diplomarbeit
zutreffend fest, dass Angehörige des Berufsstandes der
Bibliothekar*innen meist der Mittelschicht entstammen und seltener
wirklich in der Lage sind, Empathie gegenüber Obdachlosen zeigen zu
können.\footnote{Schneider, 2006, S. 52} Eine wirkliche
Auseinandersetzung fand bis dato im bibliothekarischen Kontext nicht
statt.\footnote{Zschau/Jobmann, 2013, S. 6} Gerhard Zschau und Peter
Jobmann wiesen in ihrer Masterarbeit darauf hin, dass der Begriff Kunde
das Ziel verfolgt, Leistungen zu messen und zu steigern.\footnote{Ebda.
  S. 8} Damit ist klar, dass die Gruppe der Wohnungslosen sicherlich
nicht zu den Wunschkund*innen einer Öffentlichen Bibliothek zählt.
Schneider ging sogar noch einen Schritt weiter, indem sie konstatierte,
dass, egal welches Verhalten Wohnungslose an den Tag legen, diese immer
als \enquote{Problemnutzer*innen} wahrgenommen werden. Sie stieß bei
ihren Recherchen allerdings nicht nur auf Ablehnung, sondern fand auch
Angehörige der Profession, die Motivation für Engagement
zeigen.\footnote{Schneider, 2006, S. 53} Klar ist: Mit dem Kundenbegriff
verändert sich das Rollenverständnis, was Besucher*innen in Bibliotheken
für eine Bedeutung zukommt.\footnote{Zschau/Jobmann, 2013, S. 9 f.}

\hypertarget{soziale-arbeit-in-uxf6ffentlichen-bibliotheken}{%
\section{Soziale Arbeit in öffentlichen
Bibliotheken}\label{soziale-arbeit-in-uxf6ffentlichen-bibliotheken}}

Um mehr über Einstellungen in Bibliotheken zum Thema zu erfahren
verschickte der Autor zweimal, zunächst im April und dann im September
2019, eine Anfrage an Bibliotheken gerichtet, deren Betreff
\enquote{Anfrage: Soziale Arbeit \& obdachlose Menschen in öffentlichen
Bibliotheken?}. Die erste Anfrage wurde in die bibliothekarische
Mailingliste Forum-ÖB gesetzt, welche die größte Liste für Öffentlichen
Bibliotheken im deutschsprachigen Raum darstellt. Die zweite Anfrage
wurde gezielt an Emailadressen von Großstadtbibliotheken der Hauptstädte
aller Bundesländer gerichtet und zusätzlich an die Stadtbibliotheken
Köln und Frankfurt, die aufgrund ihrer Bedeutung und Größe noch
hinzugenommen wurden. Auf die erste Anfrage in der Mailingliste Forum ÖB
meldeten sich insgesamt nur drei Bibliothekarinnen, mit der Antwort,
dass es in deren Städten keine Erfahrungen mit dieser Gruppe gibt.
Generell ist eine äußerst geringe Antwortbereitschaft -- gerade von
Großstadtbibliotheken zu beobachten gewesen. Zudem verwundert dies
gerade in Bibliotheken in großen Städten, da doch eine Kooperation mit
Einrichtungen der Wohnungslosenhilfe Sinn machen würde.

Soziale Arbeit findet in den meisten öffentlichen Bibliotheken
Deutschlands nicht gezielt, gewollt oder geplant statt. Die
Stadtbibliothek München kooperiert seit kurzem, ähnlich wie die
Büchereien Wien, wie im folgenden Kapitel dargestellt wird, mit
Streetworker*innen der Teestube \enquote{Komm} des Evangelischen
Hilfswerkes.\footnote{Telefonat vom 14.05.2020 mit Frau Waltraud
  Leitmeier, Mitarbeiterin der Stadtbibliothek München}

\hypertarget{sozialarbeiterinnen-als-netzwerkpartner-am-beispiel-der-buxfcchereien-wien}{%
\subsection{Sozialarbeiter*innen als Netzwerkpartner am Beispiel der
Büchereien
Wien}\label{sozialarbeiterinnen-als-netzwerkpartner-am-beispiel-der-buxfcchereien-wien}}

Über den Artikel im Wiener Kurier \enquote{Obdachlose in Büchereien:
Mitarbeiter überfordert}\footnote{Rieger, 2016} stieß der Autor erstmals
im Jahr 2016 auf diese Thematik und erfuhr ein Jahr später vom Leiter
der Büchereien Wien, dass an diesen eine Fortbildung durch Streetworker
von \enquote{wieder wohnen}, Betreute Unterkünfte wohnungslose
Menschengemeinnützige GmbH und FONDS SOZIALES WIEN durchgeführt
wurde.\footnote{E-mail von Christian Jahl vom 12.01.2017, dem Leiter der
  Büchereien Wien} Mitarbeiter*innen von SAM (Mobilen Sozialen Arbeit im
öffentlichen Raum) besuchen die Hauptbücherei in regelmäßigen Abständen.
Dabei findet ein Austausch mit Mitarbeiter*innen, mit dem
Sicherheitsdienst und (potenziellen) Klienten statt. Zudem gibt es die
Möglichkeit, in dringenden Notfällen Streetworker*innen telefonisch
anzufordern.\footnote{E-mail von Christian Jahl vom 12.05.2019, dem
  Leiter der Büchereien Wien} Es handelte sich um drei zweistündige
Workshops zum Thema \enquote{Umgang mit Obdachlosen und anderen
auffälligen Personen in Theorie und Praxis}, die im Jahr 2017 von zwei
Sozialarbeiter*innen von \enquote{wieder wohnen} umgesetzt wurden. Es
wurden gängige Stereotypen und Vorurteile aufgebrochen. Darüber hinaus
wurde auf Einrichtungen verwiesen, die Nutzer*innen der Bibliotheken
empfohlen werden können, wenn diese ihren Aufenthalt in der Bibliothek
zum Beispiel aufgrund von Störungen beenden müssen. In einem weiteren
Schritt fanden praktische Übungen und Rollenspiele statt. Insgesamt
nahmen an diesen Workshops 30 Bibliothekar*innen teil und die Resonanz
war durchweg positiv.\footnote{Email von Karin Claudi vom 25.02.2019,
  Qualitätsmanagement, Aus- und Fortbildung\\
  Büchereien Wien} Habakzeh Hassan, der Teamleiter der
Straßensozialarbeit von Obdach Wien gemeinnützige GmbH (das umbenannte
\enquote{wieder wohnen}), betonte aber auch, dass die Kooperation nur
auf Anfrage zwischen den Büchereien Wiens und den Streetworker*innen
Bestand hat. Ziel dieser Workshops war es auch bei den
Bibliotheksmitarbeiter*innen eine Multiplikatorenfunktion zu
erreichen.\footnote{E-mail von Habakzeh Hassan vom 02.05.2019,
  Streetworker von SAM}

\hypertarget{sozialarbeiterinnen-als-netzwerkpartnerinnen-am-beispiel-der-stadt-zuxfcrich}{%
\subsection{Sozialarbeiter*innen als Netzwerkpartner*innen am Beispiel
der Stadt
Zürich}\label{sozialarbeiterinnen-als-netzwerkpartnerinnen-am-beispiel-der-stadt-zuxfcrich}}

Eine weitere Einrichtung, die mit sozialen Einrichtungen der
Wohnungslosenhilfe kooperiert, ist die Pestalozzi-Bibliothek in Zürich.
Gaby Mattmann erläuterte in ihrem Artikel in der schweizerischen
Bibliothekszeitschrift Arbido aus dem Jahr 2017 \enquote{«Sie, der da
stinkt!» -- Vom Umgang mit Kunden am Rande der Gesellschaft}, warum eine
Kooperation dringend geboten war.\footnote{Mattmann, 2017, Online
  verfügbar unter:
  \url{https://arbido.ch/fr/edition-article/2017/le-potentiel-de-la-diversit\%C3\%A9/sie-der-da-stinkt-vom-umgang-mit-kunden-am-rande-der-gesellschaft}}
Diese Bibliothek arbeitet mit dem Café Yucca der Zürcher Stadtmission
und mit dem city-Treffunkt des Stadtzürcher Sozialdepartements zusammen.
Beide Einrichtungen bieten einen niederschwelligen Zugang für
wohnungslose Menschen. Dort erhalten diese eine preisgünstige Mahlzeit,
die Möglichkeit Kleidung waschen zu lassen oder auch Kleiderspenden zu
erhalten. Darüber hinaus erhält der Einzelne eine Beratung.

In einer Emailantwort durch Felix Hüppi, dem damaligen Chefbibliothekar
der Pestalozzi-Bibliothek vom 23.04.2019, wurde klar, dass es sich
lediglich um einen jährlichen Austausch handelt.
Bibliotheksbesucher*innen, von denen angenommen wird, dass diese
gegebenenfalls wohnungslos sein könnten, werden an städtische
Einrichtungen verwiesen und erhalten Flyer zu Angeboten der
Wohnungslosenhilfe.

Hüppi erwähnte ebenso die Einrichtung SIP (Sicherheit, Intervention und
Prävention)\footnote{\url{https://www.stadt-zuerich.ch/sd/de/index/stadtleben/sip.html}}
der Stadt Zürich ebenso in seiner Email und bezeichnete diese als eine
halb soziale und halb polizeiliche Stelle. Die Mitarbeiter*innen trugen
dazu bei, dass sich die Situation in der Bibliothek entschärfte und
deeskalierte\textbf{.}\footnote{\url{https://arbido.ch/fr/edition-article/2017/le-potentiel-de-la-diversit\%C3\%A9/sie-der-da-stinkt-vom-umgang-mit-kunden-am-rande-der-gesellschaft}}
Zudem schrieb Hüppi in seiner Antwort, dass es eine \enquote{wirkliche
Kooperation} mit der Schuldenberatung gibt, die in den Räumlichkeiten
der Bibliothek Beratungen zu Finanzen und Schulden anbietet. Die
Interessent*innen können diese unangemeldet nutzen.\footnote{Email von
  Felix Hüppi vom 23.04.2019, ehemaliger Chefbibliothekar, PBZ
  Pestalozzi-Bibliothek Zürich} Dabei wäre es natürlich denkbar, dass
auch Wohnungslose diese Beratung in Anspruch nehmen, da diese
niedrigschwelliger ist als andere.

\hypertarget{sozialarbeiterinnen-als-teil-der-belegschaft-einer-bibliothek-in-den-usa}{%
\subsection{Sozialarbeiter*innen als Teil der Belegschaft einer
Bibliothek in den
USA}\label{sozialarbeiterinnen-als-teil-der-belegschaft-einer-bibliothek-in-den-usa}}

Nachdem es unter anderem in der San Francisco Public Library zu
Drogenmissbrauch, zu Gewalt und Sex in Waschräumen gekommen war, stellte
diese Einrichtung 2009 als erste öffentlichen Bibliothek in den USA
überhaupt eine*n Sozialarbeiter*in ein.\footnote{\url{https://rabble.ca/news/2019/08/how-canadas-libraries-are-bridging-social-service-gaps}}
Die entstehenden Kosten werden teilweise vom Gesundheitsreferat der
Stadt übernommen. Ein sogenannter \enquote{psychiatric social
worker}\footnote{\url{https://www.salon.com/2013/03/07/public_libraries_the_new_homeless_shelters_partner}}
leitet ein Team von sogenannten \enquote{Health and Safety Associates}.
Einige der Beschäftigten war selbst einmal obdachlos. Die Aufgabe des
Teams ist, nach den Klienten zu sehen, die den Anschein erwecken, dass
sie Hilfe benötigen. Weiterhin achtet es darauf, dass insbesondere die
wohnungslosen Besucher*innen in der Bibliothek die Verhaltensregeln
einhalten. In den ersten drei Jahren dieser personellen Erweiterung
wurde 1.200 wohnungslosen Menschen geholfen.\footnote{\url{https://scholarworks.sjsu.edu/cgi/viewcontent.cgi?article=1184\&context=ischoolsrj}}
Die Schwerpunkte der Hilfe liegen auf der Versorgung mit Essen, der
Vermittlung von Wohnraum und Hygienemöglichkeiten, der Leistung
medizinischer Versorgung und dem Angebot von Dienstleistungen für
psychisch kranke Menschen. Laut Schätzungen der Sozialarbeiterin Leah
Esguerra half die Bibliothek mehr als 60 Menschen dabei, wieder einen
festen Wohnsitz, konkret in Form von einer Wohnung, zu bekommen.
Inzwischen stellten mehr und mehr Bibliotheken in den USA neben
Sozialarbeiter*innen, auch Krankenpfleger*innen und andere
\enquote{Outreach}-Mitarbeiter*innen ein, um wohnungslosen Menschen zu
helfen. Das Sozialarbeitsprogramm der San Francisco Public Library
entwickelte sich zu einem Good-Practice-Beispiel mit Modellcharakter. Es
inspirierte seitdem zahlreiche andere städtische Bibliothekssysteme in
den USA ähnliche Konzepte umzusetzen.\footnote{\url{https://www.salon.com/2013/03/07/public_libraries_the_new_homeless_shelters_partner/}}

\hypertarget{aufgaben-fuxfcr-sozialarbeiterinnen-im-umgang-mit-wohnungslosen-in-bibliotheken}{%
\section{Aufgaben für Sozialarbeiter*innen im Umgang mit
Wohnungslosen in
Bibliotheken}\label{aufgaben-fuxfcr-sozialarbeiterinnen-im-umgang-mit-wohnungslosen-in-bibliotheken}}

Die meisten nordamerikanischen Bibliotheken beschäftigen jedoch keine
Sozialarbeiter*innen direkt. Vielmehr wird nach dem sogenannten
\enquote{\emph{Referral-based model"} gearbeitet. Das heißt, die in
Bibliotheken angebotenen sozialen Dienstleistungen werden nicht direkt
von der Bibliothek angeboten, sondern per Kontakt zu kommunalen
Einrichtungen vermittelt, die wiederum den Klienten in den Bibliotheken
weiterhelfen. Die wichtigsten Aufgaben sind das Networking und der
Aufbau von Beziehungen innerhalb der Kommune. Dies geschieht, indem
Netzwerkpartner in die Bibliothek eingeladen werden. Justine Janis, eine
Sozialarbeiterin der Chicago Public Library, drückte es so aus, dass die
Bibliothek für viele Wohnungslose ohnehin schon ein vertrauter Ort ist
und man den Klienten da}abholen" kann, wo er sich gerade
befindet.\footnote{\url{https://www.washingtonpost.com/posteverything/wp/2016/01/27/what-happens-when-libraries-are-asked-to-help-the-homeless-find-shelter/}}
Leah Esguerra, die Sozialarbeiterin der San Francisco Library, baute
beispielsweise Kontakte mit dem städtischen Gesundheitsamt und dem
Wohnungsamt auf.\footnote{\url{https://socialwork.du.edu/news/library-social-work}}
Die Sozialarbeiterin der Denver Public Library, Elissa Hardy, gab in
einem Interview der Universität Denver Auskunft über die typischen
Aufgaben sozialer Arbeit in Bibliotheken. Was sie und ihre Kolleg*innen
tun, ist die Schulung von Mitarbeiter*innen. Dazu zählt die Durchführung
von Fortbildungen zum Umgang mit gesundheitlichen Notfällen und darüber,
wie Mitarbeiter*innen einer Bibliothek einen Ansatz erlernen, der ihnen
dabei hilft, mit traumaerfahrenen Menschen umzugehen. Zudem klärt sie
über psychische Erkrankungen, Drogenmissbrauch und über die
Traumathematik auf. Diese dadurch erworbenen Kompetenzen ermöglichen den
Mitarbeiter*innen ein besseres Verständnis und mehr Mitgefühl zu
entwickeln. Das trägt auch dazu bei, dass nicht mehr sofort die Polizei
verständigt wird beziehungsweise der Klient nicht sofort der Bibliothek
verwiesen wird. Die Stigmatisierungen, denen Wohnungslose häufig
ausgesetzt sind, konnten dadurch reduziert werden. Des Weiteren gibt es
sogenannte \enquote{peer navigators}, welche über Erfahrungen als
Wohnungslose verfügen. Sie befinden sich bereits in der Erholungsphase
ihrer psychischen Erkrankungen und/oder ihrer Sucht und/oder ihrer
Wohnungslosigkeit. Diese gehen durch die Bibliothek und verteilen Snacks
oder Kleidung an die Wohnungslosen. Außerdem wissen die Sozialarbeiterin
Elissa Hardy und ihr Team, wie sie mit unter dem Einfluss von
Betäubungsmitteln stehenden Menschen umgehen und schnellstmöglich
reagieren, wenn Menschen in der Bibliothek eine Überdosis Opioide zu
sich nehmen. Sie hat je einen Kollegen und eine Kollegin. Letztere ist
auf das Thema Jugend und Familien spezialisiert und ihr Kollege auf die
Themen Migration und Flüchtlingsarbeit. Die Arbeit verteilt sich auf 26
Zweigstellen in Denver. Im letzten Jahr wurden insgesamt 3.500 Kontakte
mit Klienten dokumentiert. Außerdem arbeitet Hardy an der Universität
von Denver als außerordentliche Professorin, da dort der Studiengang
\enquote{Library Social Work} angeboten wird. Sie setzt sich dafür ein,
dass diese spezielle Ausrichtung der sozialen Arbeit mehr Anerkennung
und eine höhere Wertschätzung innerhalb der Berufsverbände erfährt.

\hypertarget{fazit}{%
\section{Fazit}\label{fazit}}

Das deutschsprachige Bibliothekswesen scheint die Thematik der Sozialen
Arbeit mit Wohnungslosen in (öffentlichen) Bibliotheken nach wie vor
weitgehend zu vernachlässigen. Weder in den Berufsverbänden oder der
Ausbildung beziehungsweise dem Studium scheint ein Bewusstsein oder eine
Einsicht in die Notwendigkeit vorhanden zu sein. Gerade in Großstädten
sind die Herausforderungen akut, weshalb mehr als sinnvoll ist, sich
differenzierter mit dieser Gruppe von Menschen und ihren Besonderheiten
auseinanderzusetzen.

In der Praxis gibt es Fortbildungen für Mitarbeiter*innen, in denen
diese lernen, mit schwierigen Nutzer*innen der Bibliothek umzugehen. Zu
dieser Klientel werden pauschal wohnungslose Menschen gezählt. Sowohl
die Äußerungen des Streetworkers, der mit den Büchereien Wiens
kooperiert, als auch die des ehemaligen Chefbibliothekars der
Pestalozzi-Bibliothek in Zürich machten deutlich, dass es sich hierbei
nur um punktuelle unregelmäßige Kontakte handelt, die nach Bedarf wieder
reaktiviert werden können. Das im Abschnitt vorgestellte referral-based
model könnte für Großstadtbibliotheken hierzulande ein Ansatz sein, der
pragmatisch und unkompliziert umgesetzt werden könnte. Trotz der
Tatsache, dass Deutschland im Vergleich zu den angelsächsischen Ländern
über ein vergleichsweise gutes Sozialsystem verfügt, könnten
Streetworker*innen, Bibliotheken als öffentliche Einrichtungen im Sinne
einer aufsuchenden Sozialarbeit regelmäßig und routinemäßig besuchen, in
denen tatsächlich regelmäßig Wohnungslose anzutreffen sind. Des Weiteren
wären nicht nur Fortbildungen zur Thematik um Wohnungslosigkeit und
Obdachlosigkeit in den Einrichtungen selbst, sondern auch in
Ausbildungseinrichtungen künftiger Bibliothekar*innen dringend
empfehlenswert.

\hypertarget{literaturverzeichnis}{%
\section{Literaturverzeichnis}\label{literaturverzeichnis}}

Barrows, Paul Kaidy (2014). Serving the needs of homeless library
patrons: Legal issues, ethical concerns, and practical
approaches.SLISStudent Research Journal, 4(2). Online verfügbar unter:
\url{http://scholarworks.sjsu.edu/slissrj/vol4/iss2/3}

Czimmer-Gauss, Barbara (2017): Streetwork-Projekt Europaviertel in
Stuttgart: Bibliothek hilft Jugendlichen weiter. Online verfügbar unter:
\url{https://www.stuttgarter-zeitung.de/inhalt.streetwork-projekt-europaviertel-in-stuttgart-bibliothek-hilft-jugendlichen-weiter.2bebc08d-625d-4007-9c1d-3c015f47d4b3.html}

Kaiser, Wolfgang: Die «wärmste Bibliothek aller Zeiten» oder wie die
Stadtbibliothek Hangzhou auf Twitter eine hohe Rensonanz erfuhr. In:
Bibliothekarisch.de vom 30.01.2011. Online verfügbar unter:
\url{http://blog.bibliothekarisch.de/blog/2011/01/30/die-waermste-bibliothek-aller-zeiten-oder-wie-die-stadtbibliothek-in-hangzhou-auf-twitter-eine-hohe-rensonanz-erfuhr/}

Mattmann, Gaby (2017): «Sie, der da stinkt!» -- Vom Umgang mit Kunden am
Rande der Gesellschaft. In: Arbido (1) 2017. Online verfügbar unter:
\url{https://arbido.ch/fr/edition-article/2017/le-potentiel-de-la-diversit\%C3\%A9/sie-der-da-stinkt-vom-umgang-mit-kunden-am-rande-der-gesellschaft}

Nieves, Evelyn (2013): Public libraries: The new homeless shelters. In:
Salon.com vom 07.03.2013. Online verfügbar unter:
\url{https://www.salon.com/2013/03/07/public_libraries_the_new_homeless_shelters_partner/}

Rieger, Lisa (2016): Obdachlose in Büchereien: Mitarbeiter überfordert.
In: Kurier vom 21.12.2016. Online verfügbar unter:
\url{https://kurier.at/chronik/obdachlose-in-buechereien-mitarbeiter-ueberfordert/236.841.158}

Robinson, Olivia (2019): How Canada's libraries are bridging
social-service gaps. In: Rabble.ca vom 20.08.2019. Online verfügbar
unter:
\url{https://rabble.ca/news/2019/08/how-canadas-libraries-are-bridging-social-service-gaps}

Schneider, Carolin (2006). Bibliothekarische Angebote für Obdachlose in
England: Mit einem Vergleich zur bibliothekarischen Praxis in
Deutschland (Arbeiten zur Bibliotheks- und Dokumentationspraxis, N.F.,
Bd. 1). Zugl.: Köln, Fachhochschule, Diplomarbeit, 2004. Hannover:
Koechert.

Schramm, Christian (2019): Obdachlos ohne Krankenversicherung:Die Leiden
des Giovanni. In: taz vom 28.12.2019. Online verfügbar unter:
\url{https://taz.de/Obdachlos-ohne-Krankenversicherung/!5648706/}

Stadt Zürich -- Sozialdepartement (2020): Sicherheit Intervention
Prävention sip züri. Online verfügbar unter:
\url{https://www.stadt-zuerich.ch/sd/de/index/stadtleben/sip.html}

Stark, Christian (2006): Klient oder Kunde? Kritische Überlegungen zum
Kundenbegriff in der Sozialen Arbeit. Online verfügbar unter:
\url{http://www.sozialearbeit.at}

University of Denver (2018): Library Social Work. GSSW Adjunct Prof.
Elissa Hardy is helping to define a new social work specialty.
26.06.2018. Online verfügbar unter:
\url{https://socialwork.du.edu/news/library-social-work}

Vartabedian, Marc (2016): What happens when libraries are asked to help
the homeless find shelter. In: Washington Post vom 26.01.2016. Online
verfügbar unter:
\url{https://www.washingtonpost.com/posteverything/wp/2016/01/27/what-happens-when-libraries-are-asked-to-help-the-homeless-find-shelter/}

Zschau, Gerhard ; Jobmann, Peter: Auf dem Weg zur demokratischen
Bibliothek. -- Berlin, Freie Univ., Masterarb., 2013. Online verfügbar
unter: \url{http://demokratische-bibliothek.de/}

\emph{Onlinequelle wurden am 10.10.2020 abgerufen.}

%autor
\begin{center}\rule{0.5\linewidth}{0.5pt}\end{center}

\textbf{Wolfgang Kaiser}, *1981, Diplom-Bibliothekar, zuletzt seit 2018
im Sozialen Beratungsdienst in einem städtischen Notunterkunftsheim für
Wohnungslose in München tätig, aktuell Master Soziale Arbeit (cand.) KSH
München. Kontakt: wolfgang\_kaiser@ymail.com

\end{document}
