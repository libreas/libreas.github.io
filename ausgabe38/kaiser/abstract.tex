Der Beitrag thematisiert die mangelnden Kooperationen von
deutschsprachigen Bibliotheken mit Sozialarbeiter*innen beim Umgang mit
dem Klientel der Wohnungslosen. Es werden Anregungen formuliert, in
welcher Form dies für (öffentliche) Bibliotheken besser gelingen kann
ist. Zudem wird anhand der Sozialen Arbeit in den Bibliotheken in den
USA deutlich gemacht, dass durch die gezielte Nutzbarmachung der
Fähigkeiten und Kenntnisse von Sozialarbeiter*innen Bibliothekar*innen
entlastet werden können beziehungsweise durch Studiengänge wie etwa
Library Social Work eine Symbiose entstehen kann.

\begin{center}\rule{0.5\linewidth}{0.5pt}\end{center}

\noindent This article refers to the lack of cooperation in germanspeaking
libraries with social workers, in particular regarding homeless people.
Suggestions, how and to what extent it may be useful for (public)
libraries to work together, are formulated. By explaining how social
workers collaborate in libraries in the USA, it discloses that the
skills and the knowledge of social workers can contribute to the relief
of librarians. Respectively study courses like Library Social Work can
create a symbiosis of both professions.
