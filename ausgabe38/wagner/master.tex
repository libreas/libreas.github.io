\documentclass[a4paper,
fontsize=11pt,
%headings=small,
oneside,
numbers=noperiodatend,
parskip=half-,
bibliography=totoc,
final
]{scrartcl}

\usepackage[babel]{csquotes}
\usepackage{synttree}
\usepackage{graphicx}
\setkeys{Gin}{width=.4\textwidth} %default pics size

\graphicspath{{./plots/}}
\usepackage[ngerman]{babel}
\usepackage[T1]{fontenc}
%\usepackage{amsmath}
\usepackage[utf8x]{inputenc}
\usepackage [hyphens]{url}
\usepackage{booktabs} 
\usepackage[left=2.4cm,right=2.4cm,top=2.3cm,bottom=2cm,includeheadfoot]{geometry}
\usepackage{eurosym}
\usepackage{multirow}
\usepackage[ngerman]{varioref}
\setcapindent{1em}
\renewcommand{\labelitemi}{--}
\usepackage{paralist}
\usepackage{pdfpages}
\usepackage{lscape}
\usepackage{float}
\usepackage{acronym}
\usepackage{eurosym}
\usepackage{longtable,lscape}
\usepackage{mathpazo}
\usepackage[normalem]{ulem} %emphasize weiterhin kursiv
\usepackage[flushmargin,ragged]{footmisc} % left align footnote
\usepackage{ccicons} 
\setcapindent{0pt} % no indentation in captions

%%%% fancy LIBREAS URL color 
\usepackage{xcolor}
\definecolor{libreas}{RGB}{112,0,0}

\usepackage{listings}

\urlstyle{same}  % don't use monospace font for urls

\usepackage[fleqn]{amsmath}

%adjust fontsize for part

\usepackage{sectsty}
\partfont{\large}

%Das BibTeX-Zeichen mit \BibTeX setzen:
\def\symbol#1{\char #1\relax}
\def\bsl{{\tt\symbol{'134}}}
\def\BibTeX{{\rm B\kern-.05em{\sc i\kern-.025em b}\kern-.08em
    T\kern-.1667em\lower.7ex\hbox{E}\kern-.125emX}}

\usepackage{fancyhdr}
\fancyhf{}
\pagestyle{fancyplain}
\fancyhead[R]{\thepage}

% make sure bookmarks are created eventough sections are not numbered!
% uncommend if sections are numbered (bookmarks created by default)
\makeatletter
\renewcommand\@seccntformat[1]{}
\makeatother

% typo setup
\clubpenalty = 10000
\widowpenalty = 10000
\displaywidowpenalty = 10000

\usepackage{hyperxmp}
\usepackage[colorlinks, linkcolor=black,citecolor=black, urlcolor=libreas,
breaklinks= true,bookmarks=true,bookmarksopen=true]{hyperref}
\usepackage{breakurl}

%meta
%meta

\fancyhead[L]{Wagner, Schumann, Riesenweber\\ %author
LIBREAS. Library Ideas, 38 (2020). % journal, issue, volume.
\href{https://doi.org/10.18452/23473}{\color{black}https://doi.org/10.18452/23473}
{}} % doi 
\fancyhead[R]{\thepage} %page number
\fancyfoot[L] {\ccLogo \ccAttribution\ \href{https://creativecommons.org/licenses/by/4.0/}{\color{black}Creative Commons BY 4.0}}  %licence
\fancyfoot[R] {ISSN: 1860-7950}

\title{\LARGE{Libraries4Future -- über die Initiative und Best Practices}}% title
\author{Janet Wagner, Tim Schumann, Christina Riesenweber} % author

\setcounter{page}{1}

\hypersetup{%
      pdftitle={Libraries4Future -- über die Initiative und Best Practices},
      pdfauthor={Janet Wagner, Tim Schumann, Christina Riesenweber},
      pdfcopyright={CC BY 4.0 International},
      pdfsubject={LIBREAS. Library Ideas, 38 (2020).},
      pdfkeywords={Bibliothek, Grüne Bewegung, professionelles Engagement, Klimaschutzbewegung},
      pdflicenseurl={https://creativecommons.org/licenses/by/4.0/},
      pdfcontacturl={http://libreas.eu},
      pdflang={de},
      pdfmetalang={de}
     }



\date{}
\begin{document}

\maketitle
\thispagestyle{fancyplain} 

%abstracts

%body
\hypertarget{ruxfcckblick}{%
\section{Rückblick}\label{ruxfcckblick}}

Die Initiative Libraries4Future (L4F) wurde im Sommer 2019 gemeinsam von
Vertreter*innen des Netzwerks Grüne Bibliothek und des LIBREAS-Vereins
gegründet. Mit Gründung der Initiative wurden Grundsätze formuliert, die
es für Einzelpersonen und/oder Bildungseinrichtungen ermöglicht, sich in
punkto Umwelt- und Klimaschutz zu positionieren. Die Grundsätze wurden
auf der gleichzeitig gelaunchten Webseite
(\url{http://www.libraries4future.org}) veröffentlicht, kurz zuvor wurde
bereits der Twitter-Account aktiviert (@Libraries4F). Die Liste der
Unterzeichner*innen wuchs rasch
(\url{https://libraries4future.org/liste-der-unterzeichner/}).

Von Anfang an bemühten wir uns um den Aufbau als internationale
Initiative. Die Grundsätze wurden in viele verschiedene Sprachen
übersetzt und auf der Webseite veröffentlicht. Inzwischen gibt es sie in
18 Sprachversionen. Ähnlich wie bei den FridaysForFuture-Bewegungen
gelang es uns als \enquote{Ortsgruppe Berlin} gemeinsame Treffpunkte vor
Berliner Bibliotheken zu benennen und gemeinsam zu den
Klimastreikdemonstrationen zu gehen. Auf der Webseite wurde zudem ein
Blog eingerichtet, der über weltweite Demonstrationen von
Bibliotheksmitarbeiter*innen und Menschen im Umfeld von Bibliotheken und
Libraries4Future berichtet.

\hypertarget{best-practice-das-museum-und-die-bibliothek-des-naturkundemuseums-berlin}{%
\section{Best Practice -- Das Museum und die Bibliothek des
Naturkundemuseums
Berlin}\label{best-practice-das-museum-und-die-bibliothek-des-naturkundemuseums-berlin}}

Das Museum für Naturkunde in Berlin sowie die Bibliothek liegen nah am
Invalidenpark, dem festen Treffpunkt und Protestort für die
FridaysForFuture-Demonstrationen in Berlin. Museums- und
Bibliotheksmitarbeitende haben aktiv Kontakt aufgenommen zu den
FFF-Gruppen, um einen engeren Austausch mit Forschenden und
Aktivist*innen zu den gemeinsamen Themen Biodiversität, Artensterben und
Klimawandel zu ermöglichen. Treffpunkt ist das sogenannte
Experimentierfeld.

\begin{quote}
\enquote{Jeden Freitag ab 14 Uhr und im Anschluss an die
FridaysForFuture-Demonstrationen lädt das Museum für Naturkunde zum
Austausch mit WissenschaftlerInnen des Museums sowie anderer
wissenschaftlicher Institute ein. In den ersten Monaten fanden diese
Dialoge jeweils an Thementischen statt, an denen unterschiedliche
Probleme und Lösungen im Kontext des Klima- und Umweltschutzes
adressiert und gemeinsam mit den SchülerInnen und Studierenden
weitergedacht wurden.} (Museum für Naturkunde Berlin, S. 14)
\end{quote}

Schüler*innen Berlins erhielten zudem freitags am Nachmittag freien
Eintritt in das Museum. In den Räumlichkeiten des Experimentierfelds
finden Workshops und weitere Austauschformate in engem Kontakt zu
Wissenschaftler*innen statt.

Dieses Beispiel zeigt anschaulich, wie Bildungseinrichtungen
Austauschforum und Begegnungsort sein können. So geht gesellschaftlicher
Diskurs!

\hypertarget{libraries4future-in-zeiten-von-corona}{%
\section{Libraries4Future in Zeiten von
Corona}\label{libraries4future-in-zeiten-von-corona}}

Dann kam das Jahr 2020 und die Corona-Pandemie bringt seitdem das Leben
vielerorts zum regelrechten Stillstand. Initiativen benötigen
Zeitressourcen, Austausch, Ideen und Engagement, um Lebendigkeit und
Tatkraft nicht zu verlieren. Dies war und ist aktuell in Zeiten von
Corona um vieles schwerer. Im Mai 2020 versuchten wir virtuell weiterhin
durch Partizipation und Vernetzung L4F sichtbarer zu machen. In
Zusammenarbeit mit dem \enquote{Team von morgen} bot sich im Sommer 2020
eine wunderbare Möglichkeit, virtuell aktiv zu werden.

\begin{quote}
\enquote{Die Karte von morgen ist eine interaktive Onlineplattform für
Initiativen des Wandels und für nachhaltige Unternehmen.} (von morgen
2020) Die Aktion \enquote{Karte von Morgen} rief auf zu einem
Mapathon-Wochenende. Die Mission dahinter: \enquote{Durch das Kartieren,
Verbinden und Verbreiten zukunftsweisender Entwicklungen, geben wir
Orientierung für gemeinwohlorientiertes Leben und Wirtschaften.} (von
morgen 2020)
\end{quote}

Die virtuellen Begegnungen der unterschiedlichen ForFuture-Bewegungen
waren spannend. Zugleich ist es ein wichtiges, verbindendes Gefühl zu
wissen, wie viele Menschen sich trotz der Pandemie-Situation
zusammenfinden und die universelle Dringlichkeit von Umwelt- und
Klimaschutz nicht aus den Augen verlieren. An der ein oder anderen
Reaktion war jedoch auch abzulesen, dass es eher verwunderte:
Bibliotheken positionieren sich im Klima- und Ressourcenschutz?
Bibliotheken kennen die Nachhaltigkeitsziele und die Agenda 2030 und
richten ihre Arbeit und ihr Handeln danach aus?

So wurde bei der Kartierungsaktion hier wichtige Aufklärungsarbeit
geleistet und alle Bibliotheken auf der Unterzeichnerliste wurden in die
virtuelle Karte eingetragen. L4F fordert in seinen Grundsätzen unter
anderem den gesellschaftlichen Diskurs und Austausch. In der
\enquote{Karte von morgen} finden sich unzählige potenzielle Partner für
Bibliotheken, um diesen Austausch und auch zukünftige Partnerschaften
und Vernetzungen zu realisieren. Die Visualisierung macht eine schnelle
regionale Suche sehr einfach: Gibt es für meine Bibliothek eine NGO,
eine Initiative, ein Verein mit dem ich vielleicht die nächste
Veranstaltung zum Thema Nachhaltigkeit, Regionalität, Gleichstellung,
Mobilität in meiner Stadt/Gemeinde, biologische Vielfalt et cetera
anfragen kann? Die Zielsetzungen sind nahezu übereinstimmend: Eine
lebenswerte Zukunft für alle Menschen weltweit unter Beachtung von
intra- und intergenerationeller Gerechtigkeit. Umweltbildung,
Bereitstellung von Fakten und Bildungsangebote, um die globalen
Zusammenhänge unseres Welt(öko)systems zu verstehen, ist und soll eine
wichtige Aufgabe von Bibliotheken sein.

Diese Aufgabe richtet sich nicht nur an die Nutzer*innen von
Bibliotheken. Das eigene Team in der Bibliothek sollte offen über die
Mehrdimensionalität von Nachhaltigkeit nachdenken, darüber diskutieren
und gleichzeitig anhand der Grundsätze von L4F Leitbilder, Ziele und
Taten folgen lassen. Neben der bestehenden Komplexität zum Thema
Nachhaltigkeit sind die fünf Grundsätze von L4F eine erste Orientierung
für Menschen in Bibliotheken. Die Überlegungen als Akteur*innen sind
vielschichtig: Wie realisieren wir ressourcenschonendes Arbeiten,
klimafreundliche Mobilität zur und von der Arbeit, Kooperation mit
regionalen Partner*innen und die Umsetzung von Bildung für nachhaltige
Entwicklung für die Zielgruppen in der jeweiligen Einrichtung? Dies sind
erste Ansätze, die sich aus den Grundsätzen heraus ergeben und die es zu
beantworten gilt. Das direkte Handeln gelingt, wenn systemische
Voraussetzungen geschaffen und wenn gleiche Ziele und Werte von Allen
getragen werden. Die eigene Überzeugung, die intrinsische Motivation
einzelner Personen spielen eine bedeutende Rolle, um im Team einer
Bibliothek Bewusstsein zu schaffen für eine gemeinsame Rolle als
Akteur*in im Umwelt- und Klimaschutz. Gemeinsame Ziele und Werte
schaffen dafür eine gute Voraussetzung:

\begin{quote}
\enquote{Werte sind die Kerne intrinsischer Motivation, also der
Persönlichkeit. Für die nachhaltige Entwicklung besonders bedeutsam, da
sie generationsübergreifend sind, bilden sie auch in neuen,
unvorhersehbaren Situationen eine Richtschnur für das Handeln.} (Ibisch
et al.~2018, S. 105)
\end{quote}

\hypertarget{best-practice-bibliothekssystem-der-freien-universituxe4t-berlin}{%
\section{Best Practice: Bibliothekssystem der Freien Universität
Berlin}\label{best-practice-bibliothekssystem-der-freien-universituxe4t-berlin}}

An der Freien Universität Berlin entwickelten sich gemeinsame Werte im
Zusammenhang mit einer strategischen Neuentwicklung für das gesamte
Bibliothekssystem. Bei diesem partizipativen Prozess kam es neben der
Entwicklung von Vision, Mission und Leitbild zu einem Werteverständnis,
bei dem Nachhaltigkeit eine wichtige Rolle einnimmt.

Über den Winter 2020 wurde an der Freien Universität Berlin erstmalig
eine Strategie für das gesamte Bibliothekssystem, bestehend aus der
Universitätsbibliothek, den Fachbereichsbibliotheken, dem Center für
Digitale Systeme und dem Universitätsarchiv entwickelt.\footnote{Mehr
  Informationen zum Change-Projekt finden Sie auf den Webseiten der
  Universitätsbibliothek der Freien Universität Berlin:
  \url{https://www.fu-berlin.de/sites/ub/ueber-uns/wandel/index.html}.}
In einem umfassenden Mitgestaltungsprozess wurden erste Entwürfe für die
Strategie mit vielen Kolleg*innen besprochen und in einem Workshop Ende
Januar mit mehr als 70 Personen überarbeitet. Hier wurde deutlich, dass
sich die Mitarbeitenden eine explizite Benennung des
Nachhaltigkeitsthemas in der strategischen Ausrichtung unserer Arbeit
wünschen. Als einer von sieben Werten, die unsere Zusammenarbeit leiten
sollen, wurde deswegen \enquote{Nachhaltigkeit und Verantwortung}
festgeschrieben: \enquote{Wir nehmen unsere Verpflichtung gegenüber den
zukünftigen Generationen wahr. Deswegen prüfen wir unsere Entscheidungen
stets auf ihre Nachhaltigkeit im ökologischen, ökonomischen und sozialen
Sinn und richten unser Handeln entsprechend verantwortungsbewusst aus.}

Trotz der erschwerten Bedingungen durch die Pandemie haben wir seitdem
einiges in Bewegung gebracht: Eine Arbeitsgruppe hat in Kooperation mit
der Stabsstelle \enquote{Nachhaltigkeit und Energie} der Freien
Universität ein Konzept entwickelt, um das Thema Nachhaltigkeit in allen
Teilen des Bibliothekssystems zu verankern. Den Auftakt dazu bildete ein
virtueller Workshop im August 2020, an dem mehr als 50 Kolleg*innen
teilgenommen haben. Hier haben wir grundlegende Aspekte der
Nachhaltigkeitsarbeit vorgestellt, die Interessen der Teilnehmenden
dokumentiert und werden nun Vertiefungsthemen wie \enquote{Nachhaltige
Mobilität} oder \enquote{Nachhaltigkeit in der Büroausstattung} in
konkrete Handlungsansätze ausarbeiten.

Dieser Prozess wird sich bis ins Jahr 2021 ziehen, \enquote{gelebt}
werden die Werte bereits. Bewusstseinsbildung und Transparenz über
ressourcensparendes Arbeiten und die Verantwortlichkeit im eigenen
Handeln gilt es zu stärken.

Die fünf Grundsätze von L4F sind hierbei nicht nur Orientierung, sondern
machen sehr deutlich, wie Haltung und Handlung ineinander gehen können
und müssen.

\hypertarget{libraries4future-wie-geht-es-weiter}{%
\section{Libraries4Future -- wie geht es
weiter?}\label{libraries4future-wie-geht-es-weiter}}

Die Libraries4Future-Initiative wird bisher nur von sehr wenigen
Menschen aktiv getragen, was die Arbeit sehr stark beeinträchtigt. Das
ist problematisch, da sie in Zukunft stärker und sichtbarer werden muss,
um besser als Ansprechpartner*in und als Akteur*in zur Verfügung zu
stehen und lokale Initiativen unterstützen zu können.

Um stärker und sichtbarer zu werden, ist vor allem der Aufbau von
Kooperationen mit anderen Initiativen innerhalb und außerhalb von
Bibliotheken geplant. Etwas erschwert wird dieses Ziel durch das immer
noch vorherrschende Image von Bibliotheken, dass sich dort alles ‚nur um
das Buch' dreht.

Ganz im Sinne der \enquote{4Future-Bewegung} soll die
Libraries4Future-Initiative von Regionalgruppen getragen werden, die
sich noch organisieren müssen. Damit soll vor allem dem Aspekt Rechnung
getragen werden, dass die Voraussetzungen sowie die Ziele lokal sehr
unterschiedlich sein können und nur lokal angegangen werden können.
Startpunkte für regionale Gruppen können zum Beispiel lokale Barcamps
sein, die von Libraries4Future organisiert werden. Oftmals ist erst die
direkte Begegnung der Zündfunke Kooperationen und Initiativen zu
starten!

Gleichzeitig soll es der Libraries4Future-Initiative aber auch
ermöglichen, durch verschiedene regionale Gruppen, globaler zu agieren
und mehr Kraft zu entfalten.

Damit wir gemeinsam diese Ziele erreichen können, brauchen wir
Unterstützung! Helft uns, unseren Blog mit Leben zu füllen und unser
Mailpostfach zu betreuen! Helft uns, Netzwerke mit anderen Gruppen
aufzubauen! Helft uns, Ortsgruppen zu gründen! Libraries4Future steht
erst am Anfang seiner Entwicklung. Wir brauchen jede helfende Hand, um
weiter vorwärts zu kommen!

Deswegen:

\begin{itemize}
\tightlist
\item
  unterstützt die Libraries4Future-Initiative,
\item
  helft uns, gemeinsam stärker zu werden,
\item
  helft uns, die Grundsätze von Libraries4Future auch in der eigenen
  Bibliothek umzusetzen,
\item
  helft uns, Bibliotheken zu Akteur*innen für Klimagerechtigkeit werden
  zu lassen!
\end{itemize}

Bei Interesse bitte einfach eine Mail an:

info@libraries4future.org

\hypertarget{literaturverzeichnis}{%
\section{Literaturverzeichnis}\label{literaturverzeichnis}}

Ibisch, Pierre L.; Molitor, Heike; Conrad, Alexander; Walk, Heike;
Mihotovic, Vanja; Geyer, Juliane (Hg.) (2018): Der Mensch im globalen
Ökosystem. Eine Einführung in die nachhaltige Entwicklung. Unter
Mitarbeit von Michael Succow und Marlehn Thieme. Gesellschaft für
Ökologische Kommunikation mbH. München: oekom.

Museum für Naturkunde Berlin: Dialog im Forschungsmuseum 2019. DOI:
\url{https://doi.org/10.7479/rpkp-ht60}.

von morgen (2020): von morgen - Alles Gute auf einer Karte - von morgen.
Online verfügbar unter \url{http://blog.vonmorgen.org/}, zuletzt
aktualisiert am 20.09.2020, zuletzt geprüft am 20.09.2020.

%autor
\begin{center}\rule{0.5\linewidth}{0.5pt}\end{center}

\textbf{Janet Wagner} ist Mit-Initiatorin von Libraries4Future, aktiv im
Netzwerk Grüne Bibliothek und Bibliothekarin an der Philologischen
Bibliothek der Freien Universität Berlin.

\textbf{Tim Schumann} ist Mit-Initiator von Libraries4Future, aktiv im
Netzwerk Grüne Bibliothek und leitet die Heinrich-Böll-Bibliothek in
Berlin.

\textbf{Christina Riesenweber} ist Projektmanagerin beim Projekt
Organisationsentwicklung in der Universitätsbibliothek der Freien
Universität Berlin
(\url{https://www.fu-berlin.de/sites/ub/ueber-uns/team/riesenweber/index.html})

\end{document}
