\documentclass[a4paper,
fontsize=11pt,
%headings=small,
oneside,
numbers=noperiodatend,
parskip=half-,
bibliography=totoc,
final
]{scrartcl}

\usepackage[babel]{csquotes}
\usepackage{synttree}
\usepackage{graphicx}
\setkeys{Gin}{width=.4\textwidth} %default pics size

\graphicspath{{./plots/}}
\usepackage[ngerman]{babel}
\usepackage[T1]{fontenc}
%\usepackage{amsmath}
\usepackage[utf8x]{inputenc}
\usepackage [hyphens]{url}
\usepackage{booktabs} 
\usepackage[left=2.4cm,right=2.4cm,top=2.3cm,bottom=2cm,includeheadfoot]{geometry}
\usepackage{eurosym}
\usepackage{multirow}
\usepackage[ngerman]{varioref}
\setcapindent{1em}
\renewcommand{\labelitemi}{--}
\usepackage{paralist}
\usepackage{pdfpages}
\usepackage{lscape}
\usepackage{float}
\usepackage{acronym}
\usepackage{eurosym}
\usepackage{longtable,lscape}
\usepackage{mathpazo}
\usepackage[normalem]{ulem} %emphasize weiterhin kursiv
\usepackage[flushmargin,ragged]{footmisc} % left align footnote
\usepackage{ccicons} 
\setcapindent{0pt} % no indentation in captions

%%%% fancy LIBREAS URL color 
\usepackage{xcolor}
\definecolor{libreas}{RGB}{112,0,0}

\usepackage{listings}

\urlstyle{same}  % don't use monospace font for urls

\usepackage[fleqn]{amsmath}

%adjust fontsize for part

\usepackage{sectsty}
\partfont{\large}

%Das BibTeX-Zeichen mit \BibTeX setzen:
\def\symbol#1{\char #1\relax}
\def\bsl{{\tt\symbol{'134}}}
\def\BibTeX{{\rm B\kern-.05em{\sc i\kern-.025em b}\kern-.08em
    T\kern-.1667em\lower.7ex\hbox{E}\kern-.125emX}}

\usepackage{fancyhdr}
\fancyhf{}
\pagestyle{fancyplain}
\fancyhead[R]{\thepage}

% make sure bookmarks are created eventough sections are not numbered!
% uncommend if sections are numbered (bookmarks created by default)
\makeatletter
\renewcommand\@seccntformat[1]{}
\makeatother

% typo setup
\clubpenalty = 10000
\widowpenalty = 10000
\displaywidowpenalty = 10000

\usepackage{hyperxmp}
\usepackage[colorlinks, linkcolor=black,citecolor=black, urlcolor=libreas,
breaklinks= true,bookmarks=true,bookmarksopen=true]{hyperref}
\usepackage{breakurl}

%meta
%meta

\fancyhead[L]{Y. Fadeeva\\ %author
LIBREAS. Library Ideas, 38 (2020). % journal, issue, volume.
\href{https://doi.org/10.18452/23479}{\color{black}https://doi.org/10.18452/23479}
{}} % doi 
\fancyhead[R]{\thepage} %page number
\fancyfoot[L] {\ccLogo \ccAttribution\ \href{https://creativecommons.org/licenses/by/4.0/}{\color{black}Creative Commons BY 4.0}}  %licence
\fancyfoot[R] {ISSN: 1860-7950}

\title{\LARGE{Wie steht es mit Open Access in den \enquote{Buchfächern}? Erfahrungen aus dem Projekt OGeSoMo}}% title
\author{Yuliya Fadeeva} % author

\setcounter{page}{1}

\hypersetup{%
      pdftitle={Wie steht es mit Open Access in den "Buchfächern"? Erfahrungen aus dem Projekt OGeSoMo},
      pdfauthor={Yuliya Fadeeva},
      pdfcopyright={CC BY 4.0 International},
      pdfsubject={LIBREAS. Library Ideas, 38 (2020).},
      pdfkeywords={Bibliothek, Open Access, OGeSoMo, Open-Access-Bücher, Geisteswissenschaften, Sozialwissenschaften, Projektbericht},
      pdflicenseurl={https://creativecommons.org/licenses/by/4.0/},
      pdfcontacturl={http://libreas.eu},
      baseurl={http://libreas.eu},
      pdflang={de},
      pdfmetalang={de}
     }



\date{}
\begin{document}

\maketitle
\thispagestyle{fancyplain} 

%abstracts

%body
Das BMBF-geförderte Projekt Open Access in den
Geistes- und Sozialwissenschaften mit dem Schwerpunkt
Monografien (OGeSoMo),\footnote{OGeSoMo hatte eine Laufzeit von
  26 Monaten (1.3.2018--30.4.2020) und beinhaltete eine 100 \% Stelle
  einer wissenschaftlichen Mitarbeiterin an der UB sowie eine 50 \%
  Stelle einer wissenschaftlichen Mitarbeiterin am Institut für
  Germanistik. Die Leitung des Projekts lag an der UB der UDE bei
  Dorothee Graf.} angesiedelt an der Universitätsbibliothek (UB) und dem
germanistischen Institut der Universität Duisburg-Essen (UDE),
unterstützte 2018--2020 die Entwicklung von Open Access (OA) in den
sogenannten Buchdisziplinen an der Universitätsallianz Ruhr
(UAR).\footnote{Die Anbindung an die Universitätsallianz Ruhr (UAR)
  erfolgte durch die OA-Beauftragten der anderen Universitäten der
  Allianz (neben der UDE die Ruhr-Universität Bochum und TU Dortmund)
  sowie durch die Förderung von Autor*innen aus der UAR.} Dank einer
Anschubfinanzierung von 75.000 € und beinahe ebenso hoher
Eigenbeteiligung der UDE wurde eine große Anzahl von Monografien und
Sammelbänden in Open Access überführt beziehungsweise als OA
erstpubliziert. Im Folgenden werden einige Eckpunkte und Ergebnisse des
Projekts vorgestellt.

\hypertarget{einleitendes-zu-open-access-in-den-geistes--und-sozialwissenschaften}{%
\section{Einleitendes zu Open Access in den Geistes- und
Sozialwissenschaften}\label{einleitendes-zu-open-access-in-den-geistes--und-sozialwissenschaften}}

Das Thema Open Access als Zukunftsformat der Wissenschaftskommunikation
rückt immer weiter ins Blickfeld der bibliothekarischen,
publizistischen, wissenschaftlichen und wissenschaftspolitischen
Gegenwart. Üblicherweise wird bei Open Access auf Meilensteine wie die
Budapest Open Access Initiative (2002)\footnote{\url{https://budapestopenaccessinitiative.org}.}
und die Berliner Erklärung (2003)\footnote{\url{https://openaccess.mpg.de/Berliner-Erklaerung}.} verwiesen, wo der offene -- kosten- und
schrankenlose -- Zugang zu wissenschaftlichen Arbeiten beziehungsweise
Ergebnissen wissenschaftlicher Forschung im Internet genannt wird. Neben
dem reinen Zugang geht es bei Open Access auch um die freie, wenngleich
durch Lizenzen geregelte, Nutzung\footnote{Die Berliner Erklärung
  fordert für alle Nutzer*innen ``a free, irrevocable, worldwide, right
  of access to, and a license to copy, use, distribute, transmit and
  display the work publicly and to make and distribute derivative works,
  in any digital medium for any responsible purpose, subject to proper
  attribution of authorship (community standards, will continue to
  provide the mechanism for enforcement of proper attribution and
  responsible use of the published work, as they do now), as well as the
  right to make small numbers of printed copies for their personal
  use.''} der Ergebnisse. Ein darüber hinausgehendes inhaltliches
Verständnis von Open Access existiert bislang noch nicht.\footnote{Diese
  Offenheit erlaubt die Unabhängigkeit von tradierten Publikationsformen
  nach selbständigen und unselbständigen Arbeiten und bietet die
  Möglichkeit, das Feld der grauen Literatur neu zu gestalten; mit
  Implikationen für den Begriff einer wissenschaftlichen Arbeit. Hier
  sind eine ganze Reihe von Unterscheidungen möglich, zum Beispiel
  hinsichtlich der Kategorien (Digital-Analog-)Form, Kosten und Zugang
  (vergleiche Burovikhina 2020, S. 111f.) beziehungsweise der Lesarten
  des Qualitätsbegriffs sowie auch möglicher Differenzierungen nach Ort,
  Zeitpunkt, Person/Rolle, Textform und Rechtslage, vergleiche Graf und
  Fadeeva (2020), Abschnitte 4.1.1 und 4.1.2).} Im Projektkontext wurde
unter Open Access ein verlagsproduziertes, kostenloses E-Book mit einer
Creative-Commons-Lizenz (CC-Lizenz)\footnote{\url{https://creativecommons.org/licenses/}.}
verstanden, das entweder zeitgleich mit der Druckfassung erschien -- als
Gold-OA -- oder nach einer Embargofrist durch den Verlag kostenlos wurde
und eine CC-Lizenz erhielt -- als Grün-OA.

In STM-Disziplinen (englisch für sciences, technologies und medicine),
die hauptsächlich im Zeitschriftenformat publizieren und direkt von
Zeitschriften-Rankings (und den wenigen Großverlagen, die diese
Zeitschriften besitzen) abhängig sind, ist OA eine etablierte Praxis.
Dort dreht sich die Diskussion um hohe Kosten (für
Forschungsinstitutionen und deren Bibliotheken), Marktbeherrschung und
Gewinnmargen (Großverlage). Die Lösung wird in der Transformation von
einer Subskriptions- zu einer Publikationsfinanzierung (Projekt DEAL,
APCs) und alternativen Publikationsmodellen wie Scholar-led-Initiativen
gesehen. Ganz andere Fragen, Praktiken und Sorgen zeigen sich in den
deutschsprachigen Geistes- und Gesellschaftswissenschaften. Hier ist
Open Access weder etabliert noch umfassend bekannt und wird häufig mit
Vorbehalten verbunden, die Schreckensszenarien näherstehen als den
tatsächlichen Problemen. Open Access in diesen sogenannten
Buchdisziplinen ist ein multifaktorielles Phänomen, das es in seiner
Komplexität erst einzubetten gilt, bevor Bewertungen oder Empfehlungen
sinnvoll getroffen werden können. Die Wissenschaftskommunikation findet
hier zu einem viel größeren Anteil in Form von Monografien und
Sammelbänden statt. Das karrierebestimmende Element hängt für die
Autor*innen und Herausgeber*innen am Renommee der einzelnen Verlage, die
im DACH-Raum ein zu den englischsprachigen Monokulturen im STM-Bereich
geradezu konträres Bild ergeben, nämlich das eines diversen, sehr
spezialisierten Biotops zumeist kleiner und mittelständischer
Verlage.\footnote{Ferwerda, Pinter und Stern (2017, S. 59). An dieser
  Stelle sind zwei Anmerkungen angebracht. Gerade weil die
  Verlagslandschaft so kleinteilig ist, sind auch die hier im
  Allaussagemodus getroffenen Behauptungen mit entsprechender Vorsicht
  zu verstehen. Keine Aussage trifft voll auf die Gesamtheit der
  DACH-Verlage zu, sondern soll Tendenzen und häufige Punkte erfassen.
  Für quantitativ präzisere Aussagen vergleiche zum Beispiel Kaier, und
  Lackner (2019). Außerdem sind Praxis und Renommee internationaler
  Verlage zu beachten, in denen Angehörige deutscher Universitäten
  selbstverständlich ebenfalls publizieren.} Das begründet oft eine
enge, dauerhafte Beziehung zwischen Autor*innen und Herausgeber*innen
und ``ihren'' Verlagen. Diese Verlagslandschaft ist heterogen in ihrer
Einstellung gegenüber Open Access, sehr häufig jedoch zögerlich, was
unter anderem an fehlenden Kapazitäten finanzieller, personeller oder
technischer Art liegt.\footnote{Auch wenn es Verlage gibt, die sich sehr
  im OA engagieren.} Dazu kommen wirtschaftliche Existenzsorgen und
daraus resultierende Risikoscheu. Die Verlage bieten also nicht oft von
allein ein tragfähiges Konzept für eine OA-Publikation an oder haben gar
eine explizite Policy zu diesem Thema.\footnote{\url{https://www.uni-due.de/ogesomo/zwischenergebnisse}.}
Ähnlich schlecht steht es mit der Initiative der Autor*innen. Selbst
namhafte und rege publizierende Geistes- und Sozialwissenschaftler*innen
wissen mit dem Ausdruck ``Open Access'' beziehungsweise ``OA'' häufig
wenig anzufangen oder haben sehr vage und teils falsche Vorstellungen.
So findet sich beispielsweise oft die Sorge über den Verlust der
Kontrolle über ihre Arbeiten oder den Untergang der
Wissenschaftsfreiheit. Oder sie befürchten unseriöse
Publikationspraktiken (`predatory publishing') und vor allem einen
vermeintlich unvermeidbaren Qualitätsverlust wissenschaftlicher Arbeiten
durch OA.\footnote{Kleineberg und Kaden (2017).} Damit einhergehend
sehen sie eine Gefahr, ihren wissenschaftlichen Namen zu ``verbrennen''
-- eine in der Zeit des akademischen Publikationsdrucks (`publish or
perish') und sehr prekärer Beschäftigungsverhältnisse, insbesondere in
den Geisteswissenschaften,\footnote{\url{https://www.sueddeutsche.de/karriere/wissenschaft-karriere-befristet-1.4484574-0},
  \href{https://www.zeit.de/2020/37/geisteswissenschaften-gesellschaft-coronavirus-forschung-bibliotheken-systemrelevant}{https://www.zeit.de/2020/37/geisteswissenschaften-gesellschaft-coronavirus-forschung-bibliotheken-systemrelevant},
  \url{https://www.br.de/fernsehen/ard-alpha/sendungen/campus/akademiker-sicherheit-uni-jobs-befristet-100.html}.}
durchaus nachvollziehbare Haltung. Gleichzeitig schätzen die meisten
Wissenschaftler*innen die digitale Verfügbarkeit benötigter Literatur in
der eigenen Forschung und Lehre und erwarten deren schnelle und
unkomplizierte Bereitstellung von ihren Bibliotheken. Letztere erfüllen
zunehmend auch forschungsnahe Aufgaben, darunter auch die Beratung und
Publikationsunterstützung von Wissenschaftler*innen ihrer Einrichtungen
und in vielen Fällen auch die direkte Förderung durch einen eigenen
Publikationsfonds.

In diesem Zusammenhang muss spätestens jetzt das oftmals entscheidende
Thema genannt werden, nämlich die zum Teil immensen Kosten einer
OA-Publikation in einem Verlag. Diese Kosten kommen zu den
branchenüblichen Publikationskosten (früher auch
``Druckkostenzuschuss'') hinzu, so dass eine private Deckung der eigenen
Monographie oder eines Sammelbandes kaum realistisch ist. Entsprechend
hängt die Veröffentlichung im Open Access (noch mehr als im Closed
Access) zumeist von der Möglichkeit einer Förderung ab, zum Beispiel
durch Dritt- oder Haushaltsmittel, die diese Kosten übernehmen kann.
Hier kommen diverse Optionen in Frage, darunter Projekte, in deren
Rahmen die Arbeit am/zum jeweiligen Buch durchgeführt wurden, externe
Förderungen speziell für die Veröffentlichung im OA, aber auch
Publikationsfonds der eigenen Einrichtungen.

In der Charakterisierung der komplexen Situation kommen zwei Adjektive
besonders häufig vor, nämlich ``neu'' und ``unübersichtlich'': neue
Akteur*innen (Intermediäre wie zum Beispiel Knowledge Unlatched), neue
Vorgaben und Möglichkeiten aus der Wissenschaftspolitik und
Förderlandschaft, die Suche nach neuen Geschäftsmodellen und Formen der
Zusammenarbeit (publizieren mit Verlagen, auf Repositorien oder
\emph{scholar-led,} OA-Stellung bereits publizierter Titel oder auch die
Überführung von Zeitschriften in OA, sogenanntes journal flipping).
Bibliotheken stehen vor neuen Aufgaben wie Förderung,
Publikationsunterstützung, Repositoriumsbetrieb, neue Programme,
Workflows. Last but not least, das große Thema Metadaten. Beherrscht
wird das Bild aber von einer diffusen Unübersichtlichkeit, in der zwar
manche Akteur*innen bereits aktiv auf OA setzen und sich profilieren.
Für die Meisten handelt es sich um eine verunsicherte, atomisierte und
konfliktgeladene Mischung aus ``böhmischen Dörfern'' und einer Arena.
Wissenschaftler*innen stehen als Forschende und Lehrende vor einer Fülle
neuer, nicht-traditioneller Quellen, deren Verwendung und
Kategorisierung noch völlig unklar ist.\footnote{Vergleiche Fußnote 4.}
Als Autor*innen reagieren sie mit zurückhaltendem Abwarten. Verlage
haben (durchaus begründete) Existenzängste, scheuen riskante
Investitionen und sehen in Bibliotheken und verstärkt aufkommenden
Universitätsverlagen öffentlich geförderte Konkurrenz.

\hypertarget{die-entstehung-von-ogesomo-an-der-bibliothek-der-universituxe4t-duisburg-essen}{%
\section{Die Entstehung von OGeSoMo an der Bibliothek der
Universität
Duisburg-Essen}\label{die-entstehung-von-ogesomo-an-der-bibliothek-der-universituxe4t-duisburg-essen}}

Die Idee, die sich im Projekt OGeSoMo manifestierte, entwickelte sich an
der UB der UDE aus dem steigenden Bedarf, die genannten Schwierigkeiten,
Umstellungen und Fragen zu adressieren und problemorientiert zu einer
Verbesserung der Situation beizutragen. Konkrete Anhaltspunkte
kristallisierten sich an der UB zum einen in vermehrten
OA-Beratungsfragen aus den Geistes- und Sozialwissenschaften sowie mit
einer stetig wachsenden Anzahl und Art der forschungsnahen und
publikationsunterstützenden Aufgaben. Zum anderen zeigte sich der
Transformationsprozess in der akademischen Publikationslandschaft immer
deutlicher -- mit zahlreichen Veränderungen, für die gewohnte Abläufe
und Rollenverteilungen nicht mehr passen. Neu entstandene Anforderungen
-- digitaler, technisch-handwerklicher, juristischer, kommunikativer,
infrastruktureller, finanzieller und betriebswirtschaftlicher Art --
betreffen (mit unterschiedlichen Schwerpunkten) \emph{alle
Akteur*innen}, nicht nur die Bibliotheken.

\hypertarget{diversituxe4t-als-programm-ziele-perspektiven-aufgaben}{%
\section{Diversität als Programm -- Ziele, Perspektiven,
Aufgaben}\label{diversituxe4t-als-programm-ziele-perspektiven-aufgaben}}

Das Projekt wurde in Zusammenarbeit mit einem Lehrstuhl der Germanistik
an der UDE sowie drei Partnerverlagen in insgesamt fünf Arbeitspaketen
durchgeführt. OGeSoMo sollte ähnlich multiperspektivisch und
breitgefächert angelegt werden, wie sich die Problemlage selbst
präsentiert. Im Bereich der Geistes- und Sozialwissenschaften ist die
Förderung von OA vor allem die Förderung der \emph{Bekanntheit von OA}
und einer kommunikativen Vermittlung zwischen den unterschiedlichen
Beteiligten mit ihren komplexen, teils gegensätzlichen
Erwartungshaltungen. Hier spielte die Interaktion mit Autor*innen,
Lehrenden, Studierenden, Verlagen und Intermediären eine Schlüsselrolle,
um eine möglichst interdisziplinäre, inklusive und kooperative
Perspektive zu erlangen. Daher beschäftigten sich drei der sechs
Projektziele von OGeSoMo mit dieser Thematik: (1) Stärkung von
Bewusstsein, (2) Verbreitung von Wissen, (3) Informationsangebote. Über
die gesamte Projektdauer verteilte, ineinandergreifende
Awareness-Maßnahmen (Arbeitspaket 5) hatten zum Ziel, die heterogenen
Beteiligtengruppen zu erreichen, mit unterschiedlichsten
Aufklärungsangeboten für OA zu sensibilisieren und das Bewusstsein für
diese Publikationsform zu steigern. Die Awareness-Strategie\footnote{Vergleiche
  dazu die ausführliche Darstellung in Fadeeva, Falkenstein-Feldhoff und
  Graf (2020).} war theoretisch und praktisch angelegt, indem sie eine
OA-spezifische, zielgerichtete Problemanalyse zur Grundlage der
Entwicklung konkreter Handlungs- und Werbemaßnahmen machte. Zu diesen
praktischen Maßnahmen gehörten zum Beispiel Schulungen und Vorträge,
durchgängig Beratungen an allen beteiligten Universitäten,
Präsentationen des Projekts auf wissenschaftlichen und
bibliothekarischen Tagungen und im Verlagswesen, einem
interdisziplinären Workshop sowie mehreren Publikationen.\footnote{Graf,
  Burovikhina und Leinweber (2019); Graf, Fadeeva und
  Falkenstein-Feldhoff (2020).} Zudem verfügt(e) OGeSoMo über eine
digitale Präsenz in Form von Blog- und Newsletterbeiträgen, einer
Homepage samt einer dauerhaften Archivierungsseite auf dem
Publikationsserver DuEPublico2.\footnote{\url{https://www.uni-due.de/ogesomo/}
  und \url{https://duepublico2.uni-due.de/go/OGeSoMo}.} Eine
umfangreiche Sammlung nachnutzbarer Materialien mit Handreichungen zu
Brennpunktfragen wie Lizenzen und Urheberrecht, dem OA-Sammelband zum
Projekt\footnote{Graf, Fadeeva und Falkenstein-Feldhoff (2020).} mit
weiterführenden Analysen und Diskussionen und zahlreiche Poster,
Präsentationen, Literaturliste und Informationen zum Workflow stehen
Interessierten zur Verfügung.

Die anderen drei Projektziele adressierten kooperative, finanzielle und
praktisch-technische Desiderata sowie wissenschaftlich-didaktische
Anwendungsgebiete: (4) OA-Förderung, (5) Geschäftsmodelle, (6) OA in der
Lehre. Mit Hilfe der Anschubfinanzierung (75.000 € des BMBF sowie Mittel
des Publikationsfonds der UDE) wurden mehr als vierzig Monografien und
Sammelbände von Wissenschaftler*innen der drei Universitäten der UAR in
den Open Access überführt und stehen nun dauerhaft unter CC-Lizenz über
die Verlagsseiten sowie die jeweiligen universitären Repositorien der
UAR-Bibliotheken zur weltweit kostenlosen Nutzung zur Verfügung.
Außerdem erfolgten anhand der Projekttitel, der
Universitätsbibliographie der UDE sowie einer Stichprobe
geisteswissenschaftlicher Dissertationen der UDE mehrere empirische
Untersuchungen. Dazu zählen

\begin{itemize}

\item
  eine begleitende Erhebung über die Online-Nutzungen der geförderten
  Titel (zusammen mit dem Intermediär Knowledge Unlatched) sowie den
  Verkauf der Printausgaben;\footnote{Vergleiche Falkenstein-Feldhoff
    und Graf (2020).}

  \begin{itemize}
  
  \item
    Ergebnis: Alle Titel erschienen sowohl gedruckt als auch in der
    jeweiligen OA-Variante (Gold oder Grün). Unter den geförderten
    Titeln waren 24 Sammelwerke, neun Monografien und sieben
    Dissertationen.
  \end{itemize}
\item
  eine vergleichende Verkaufsanalyse der kostenpflichtigen E-Books mit
  den OA-Ausgaben und, zusätzlich, der Auswirkungen für den Verkauf der
  Printversionen;\footnote{Vergleiche ebenda.}

  \begin{itemize}
  
  \item
    Ergebnis: Die Befürchtung von Verkaufsausfällen wurde teilweise
    bestätigt, ist innerhalb der Paketpreise allerdings nicht in einer
    eindeutigen Zuordnung einzelner Titel angebbar.
  \end{itemize}
\item
  eine Erhebung der meistgewählten Verlage für geisteswissenschaftliche
  Publikationen an der UDE im Zeitraum 2007--2018, ermittelt aus 3.083
  Publikationen in der Universitätsbibliografie;

  \begin{itemize}
  
  \item
    Ergebnis: Insgesamt gab es Publikationen in über 500 Verlagen. Die
    Liste\footnote{Innerhalb der Geisteswissenschaften waren die Verlage
      De Gruyter, LIT, Peter Lang, Springer, Shaker und transcript UVRR
      besonders präsent, in den Sozialwissenschaften Barbara Budrich,
      Campus, Nomos, LIT, Routledge, Springer VS, transcript, siehe
      \url{https://doi.org/10.17185/duepublico/71013}.} der 17
    beliebtesten Verlage führten De Gruyter und Peter Lang mit jeweils
    157 und 129 Publikationen an. Diese Liste diente dann als Grundlage
    für die Erhebung der OA-Policies der jeweiligen Verlage.\footnote{Es
      gab sechs Rückmeldungen auf 24 Anfragen nach OA-Policies. Auf 25
      Anfragen nach Standardverlagen erfolgten eine positive und zwei
      negativen Antworten.}
  \end{itemize}
\item
  eine Erhebung über das Publikationsverhalten (Verlag versus
  Repositorium) der Nachwuchswissenschaftler*innen der UDE am Beispiel
  der veröffentlichten Dissertationen in der Soziologie und Germanistik
  im Zeitraum 2007--2017;\footnote{Vergleiche Graf und Fadeeva (2020).}

  \begin{itemize}
  
  \item
    Ergebnis: Während innerhalb der Germanistik über 80 \% der Arbeiten
    im klassischen Verlagsmodell erschienen sind, setzten Soziolog*innen
    mit 51 \% zu einem deutlicheren Teil auf die
    Repositoriumspublikation.\footnote{Vergleiche
      \url{https://doi.org/10.17185/duepublico/71224} sowie
      \url{https://doi.org/10.17185/duepublico/71107}.}
  \end{itemize}
\item
  eine Erhebung über die Nutzung von Online-Publikationen als Quellen in
  germanistischen Dissertationen.\footnote{Vergleiche Burovikhina
    (2020).}

  \begin{itemize}
  
  \item
    Ergebnis: Unter den 16.534 Quellenangaben sind nur 6 \%
    Online-Quellen. Zehn der 55 Dissertationen verwenden ausschließlich
    gedruckte Quellen. Eine Differenzierung hinsichtlich des OA-Status
    einer Publikation ist innerhalb der gängigen Fachkonventionen nicht
    gegeben.
  \end{itemize}
\end{itemize}

Im Rahmen der Open-Access-Stellung von über 40 Monografien und
Sammelbänden standen praktische, technische und betriebswirtschaftliche
Details sowohl allgemein verlegerischer Workflows als auch
OA-spezifischer Aspekte im Mittelpunkt des Interesses (Ziel 5 --
Geschäftsmodelle). Die Auswahl der zu fördernden Titel erfolgte nach
folgenden Kriterien:\footnote{Unter
  \url{https://doi.org/10.17185/duepublico/71014} findet sich eine
  entsprechende Handreichung.} Zugehörigkeit der Autor*innen zu einer
der UAR-Universitäten, die erfolgte oder geplante Publikation in einem
der drei Partnerverlage Barbara Budrich, Peter Lang und transcript
(woraus die inhaltliche Zuordnung zu den Geistes- und
Sozialwissenschaften bereits erfolgte), positive Begutachtung der
Monografie beziehungsweise des Sammelbandes durch ein \emph{editorial
board} oder \emph{peer review}, Zustimmung zur Teilnahme an der
Datenerhebung. Die Herangehensweise war durch Kooperation bestimmt. Die
Kontaktaufnahme mit den Autor*innen erfolgte durch Verlage und
Projektbeteiligte. Die kooperierenden Verlage verwendeten ein
Finanzierungsmodell aus Print- und Zusatzkosten der
OA-Ausgabe,\footnote{Hier gibt es verschiedene Bezeichnungen: OA-Gebühr
  bzw. OA \emph{fee}, BPC für \emph{book processing charge} als
  Entsprechung zu APC, \emph{article processing charge,} für
  Zeitschriftenartikel.} die zwischen 350 und 5.000 € lag, je nach
Verlag und OA-Variante (Grün oder Gold). Insgesamt wurden 19 Gold- und
der Rest als Grün-Titel gefördert.

Der germanistische Projektteil befasste sich mit dem Potential von OA
als nachhaltige Ressource im E-Learning und adressiert damit Ziel 6 (OA
in der Lehre).\footnote{Vergleiche Beißwenger (2020), siehe auch
  \url{https://doi.org/10.17185/duepublico/71099}.} Noch bevor die
Folgen der Corona-Pandemie zu vielen Anfragen zu weiteren
Anwendungsoptionen von "\textsc{textlabor"} führten, wurde die als
Prototyp an der RWTH entwickelte, mediendidaktische Moodle-Anwen\-dung zur
kooperativen Erarbeitung von Texten im Blended Learning eingesetzt.
Michael Beißwenger und Veronika Burovikhina untersuchten im
Inverted-Classroom-Modell\footnote{Siehe dazu zum Beispiel Beisswenger
  und Burovikhina (2019).} drei Semester lang in insgesamt elf
Seminaren/ fünfzehn Seminareinheiten die studienzentrierte, digital
gestützte Textannotation als Verfahren. Die Evaluationsergebnisse durch
die Teilnehmer*innen waren sehr positiv und eröffneten wichtige
Einsichten in die Lehrpraxis. Die Anwendung \textsc{Textlabor} kann mit
weitaus größerem Nutzen eingesetzt werden, wenn die Texte in OA
vorliegen, da Nachnutzung und Vernetzung über Einzelseminare hinaus
möglich wird. Vor allem könnten dann ganze Texte bearbeitet werden und
nicht nur die urheberrechtliche erlaubten 15 \% des Umfangs -- eine im
geisteswissenschaftlichen Studium durchaus problematische Einschränkung.

\hypertarget{probleme-von-heute-agenda-fuxfcr-morgen}{%
\section{Probleme von heute -- Agenda für
morgen}\label{probleme-von-heute-agenda-fuxfcr-morgen}}

Wie die Verstetigung von Anfragen auf Förderung durch den
Publikationsfonds der UDE zeigt, ist OA für viele Geistes- und
Sozialwissenschaftler*innen dieser Universität inzwischen ein
anzustrebendes Publikationsmodell, wenngleich bislang überwiegend noch
im tradierten Verlagsmodell. Das Anliegen, OA in den Geistes- und
Sozialwissenschaften systematisch in den Fokus der Aufmerksamkeit und
mit all seinen Implikationen in den Wissenschaftsdiskurs zu rücken, ist
nur zum Teil gelungen. OGeSoMo hat jedoch unterschiedlichste Probleme
ans Licht gebracht, die theoretische und praktische Aspekte von OA
betreffen und neue Fragen aufwerfen. Diese heterogenen Punkte sind teils
allgemeiner Natur, teils sehr konkret, technisch-digital und behandeln
sehr detaillierte Spezialgebiete. Sie sind nicht nur negativer Art,
sondern können fruchtbarer als Fragen, Einsichten und Perspektiven für
weitere Arbeit gefasst werden. Viele von ihnen können bereits heute
thematisiert werden, andere Antworten werden sich erst in den kommenden
Jahren abschließend formen.

\textbf{Fragen}

Besonders offen traten Unklarheiten in der Verwendung zentraler Begriffe
zutage: Was ist eine OA-Publikation? Wie kann die Qualität einer
wissenschaftlichen Publikation abseits der Verlagsreputation gefasst
werden? Wie offen sollte der Review(-prozess) sein? Daneben zeigten sich
zahlreiche praktische Fragen und Desiderata: Welche Formen des
Co-Publishing können etabliert werden? Welche finanziellen Modelle
(Crowd-Funding, konsortiale Modelle, BPC) bieten eine echte Möglichkeit,
OA nach Ablauf konkreter Förderungen nachhaltig zu implementieren?
Welche Workflows funktionieren und wo besteht Standardisierungsbedarf?
Welche neuen Einnahmequellen ermöglichen OA-Publikationen für Verlage
und Intermediäre, zum Beispiel durch kostenpflichtige Bereitstellung von
Nutzungsdaten? Wie kann OA auf struktureller Ebene etabliert werden, um
Open Science als großes Konzept der Zukunft umzusetzen? Welche Rolle
spielt Scholar-led Publishing in Deutschland? Wie können die anfallenden
digital-infrastrukturellen Herausforderungen\footnote{Als Stichworte
  seien hier beispielhaft genannt: Statistische Erhebung von
  Nutzungsdaten, Datenaustausch und Mapping zwischen Systemen des
  Buchhandels und der Bibliothekswelt, der Umstieg auf einen
  XML-basierten und damit medienneutralen Workflow.} effizient und fair
getragen werden (und von wem)?

\textbf{Einsichten und Perspektiven}

Rollen und Zuständigkeiten im Publikationsgeschehen müssen neu verteilt
werden. Das Festhalten an tradierten Abläufen (ausschließliche
Rechteübertragung an den Verlag, Verlagsrenommee als Druckmittel für die
Karriere) ist nicht mehr zielführend. Nötig ist zum Beispiel die
Dezentralisierung verlegerischer Aufgaben, die bisher pauschal einer
Institution zugewiesen wurden. Sie sollten analysiert, transparent
gemacht und neu verteilt werden: Bibliotheken können ihre vermittelnde
Position zwischen Wissenschaftler*innen, Verlagen und Intermediären
sowie ihre Mittel zur ideellen, technischen und finanziellen Förderung
und Publikationsunterstützung nutzen. Sie können Wissenschaftler*innen
beraten, Publikationen auf Repositorien dauerhaft frei verfügbar machen,
die aktive Nutzung des Zweitveröffentlichungsrechts anregen und
unterstützen, Metadaten produzieren und ihren Austausch technisch
voranbringen, Nutzungsdaten erheben und kommunizieren, und vieles mehr.

Ein anderes Beispiel zeigt die ENABLE-Community.\footnote{\url{https://enable-oa.org}.}
In diesem Zusammenschluss verschiedener Akteur*innen zur Förderung von
OA in den Geistes- und Sozialwissenschaften werden Co-Publishing-Modelle
gemeinsam erarbeitet. Dabei werden bekannte Vorstellungen darüber, wofür
Verlage, Autor*innen, Bibliotheken oder Buchhandel verantwortlich sind
und welche Gewinne und Investitionen zu erwarten sind (oder wegfallen),
durch andere Fragen abgelöst. Diese lauten zum Beispiel

\begin{itemize}

\item
  Was braucht es, um eine gute OA-Publikation zu erstellen?
\item
  Welche Kosten entstehen an welcher Stelle?
\item
  Welche Schritte können von wem geleistet werden?
\item
  Welche Finanzierungsmodelle sind nachhaltig?
\end{itemize}

Abschließend bleibt festzuhalten, dass eine einheitliche Lösung
technischer, finanzieller, rechtlicher et cetera Fragen zurzeit nicht in
Sicht ist. Das wird von zahlreichen Akteur*innen auch nicht gewünscht.
Verschiedene Ansätze (Verlags-, Co- und Self-Publishing, digital und
print, Closed und Open Access) existieren nebeneinander und werden der
Komplexität der Transformationssituation am ehesten gerecht; sie werden
vermutlich mittelfristig die Diversität der Publikationslandschaft
erhalten. Entscheidend sind jedoch eine aufgeschlossene Haltung und
regelmäßige Kommunikation zwischen allen Beteiligten, so wie es mit
OGeSoMo erprobt und gewinnbringend praktiziert wurde. Das
Kommunikationsgebot gilt für neue Formen der Zusammenarbeit im
Co-Publishing (zum Beispiel zwischen Verlagen und Bibliotheken), zeigt
aber auch gute Wirkung in offenen Austauschformaten zwischen Autor*innen
und Verlagen. Genauso ist die (fach-)wissenschaftliche Diskussion
kritischer Fragen unumgänglich, zum Beispiel nach Qualitätsmerkmalen und
der Rolle der Verlagsreputation. Hier eröffnet Open Access neue Wege in
der Möglichkeit, das klassische Modell der Verlagspublikation insgesamt
zu erweitern. Das gilt nicht nur für die Aufbereitung und Modifizierung
der digitalen Nutzung wissenschaftlicher Arbeiten (Formate: html, xml,
epub, PDF; Funktionen: Durchsuchbarkeit, Verschlagwortung, Verlinkung \&
Aktualisierbarkeit, multimediale Darstellung, offene Begutachtung). Auch
die Verteilung der verschiedenen Aufgaben und damit einhergehender
Rechte und Pflichten innerhalb der Publikation stehen zur Disposition
(Herstellung und Satz, Verarbeitung der Metadaten, Hosten, Vertrieb,
Druck, Marketing et cetera) Mit OGeSoMo ist ein explorativer Schritt für
Open Access in den Buchfächern getan. Jetzt gilt es, die vielfältigen
Anknüpfungsmöglichkeiten und Diskussionsofferten zu nutzen.

\pagebreak

\hypertarget{literaturverzeichnis}{%
\section{Literaturverzeichnis}\label{literaturverzeichnis}}

Beißwenger, Michael: \enquote{Innovative Möglichkeiten der Arbeit mit digitalen
(Open-Access-)Publi\-kationen in der Lehre: Kooperative Texterschließung
mit dem \textsc{textlabor}}, in: Graf, Dorothee, Yuliya Fadeeva und
Katrin Falkenstein-Feldhoff (Hrsg.): \emph{Bücher im Open Access. Ein
Zukunftsmodell für die Geistes- und Sozialwissenschaften?} Opladen,
Berlin, Toronto: Barbara Budrich 2020, S.~140--151.
\url{https://doi.org/10.17185/duepublico/72237}.

Beisswenger, Michael und Veronika Burovikhina: \enquote{Von der Black Box in
den Inverted Classroom: Texterschließung kooperativ gestalten mit
digitalen Lese- und Annotationswerkzeugen}, in: Führer,
Felician-Michael und Carolin Führer (Hrsg.): \emph{Dissonanzen in der
Deutschlehrerbildung. Theoretische, empirische und hochschuldidaktische
Rekonstruktionen und Perspektiven}, Bad Heilbrunn: Klinkhardt 2019,
S.~193--222.

Burovikhina, Veronika: \enquote{Verarbeitung und Nutzung digitaler
Publikationen in Forschung und Lehre am Institut für Germanistik an der
Universität Duisburg-Essen}, in: Graf, Dorothee, Yuliya Fadeeva und
Katrin Falkenstein-Feldhoff (Hrsg.): \emph{Bücher im Open Access. Ein
Zukunftsmodell für die Geistes- und Sozialwissenschaften?} Opladen,
Berlin, Toronto: Barbara Budrich 2020, S.~111--122.
\url{https://doi.org/10.17185/duepublico/72237}.

Fadeeva, Yuliya, Katrin Falkenstein-Feldhoff und Dorothee Graf:
\enquote{Awareness-Konzept: theoretisch und praktisch}, in: Graf, Dorothee,
Yuliya Fadeeva und Katrin Falkenstein-Feldhoff (Hrsg.): \emph{Bücher im
Open Access. Ein Zukunftsmodell für die Geistes- und
Sozialwissenschaften?} Opladen, Berlin, Toronto: Barbara Budrich 2020,
S.~123--140. \url{https://doi.org/10.17185/duepublico/72237}.

Falkenstein-Feldhoff, Katrin und Dorothee Graf: \enquote{Explorative Studie der
Verkaufs- und Nutzungszahlen}, in: Graf, Dorothee, Yuliya Fadeeva und
Katrin Falkenstein-Feldhoff (Hrsg.): \emph{Bücher im Open Access. Ein
Zukunftsmodell für die Geistes- und Sozialwissenschaften?} Opladen,
Berlin, Toronto: Barbara Budrich 2020, S.~89--110.
\url{https://doi.org/10.17185/duepublico/72237}.

Ferwerda, Eelco, Frances Pinter und Niels Stern: A Landscape Study On
Open Access And Monographs: Policies, Funding And Publishing In Eight
European Countries, Zenodo 2017.
\url{http://doi.org/10.5281/zenodo.815932}.

Graf, Dorothee, Veronika Burovikhina und Natalie Leinweber:
\enquote{Zukunftsmodell Monografien im Open Access}, \emph{O-bib. Das offene
Bibliotheksjournal} 6/4 (2019), S.~164--177.
\url{https://doi.org/10.5282/o-bib/2019H4S164-177}.

Graf, Dorothee und Yuliya Fadeeva: \enquote{Einleitung und abschließende
Evaluation des Projekts: Was bleibt nach OGeSoMo (zu tun)?}, in: Graf,
Dorothee, Yuliya Fadeeva und Katrin Falkenstein-Feldhoff (Hrsg.):
\emph{Bücher im Open Access. Ein Zukunftsmodell für die Geistes- und
Sozialwissenschaften?} Opladen, Berlin, Toronto: Barbara Budrich 2020,
S.~14--42. \href{https://doi.org/10.17185/duepublico/72237}{https://doi.org/10.17185/due\-publico/72237}.

Graf, Dorothee, Yuliya Fadeeva und Katrin Falkenstein-Feldhoff (Hrsg.):
\emph{Bücher im Open Access. Ein Zukunftsmodell für die Geistes- und
Sozialwissenschaften?} Opladen, Berlin, Toronto: Barbara Budrich 2020.
\url{https://doi.org/10.17185/duepublico/72237}.

Kaier, Christian und Karin Lackner: \enquote{Open Access aus der Sicht von
Verlagen}, \emph{Bibliothek Forschung und Praxis} 43/1 (2019),
S.~194--205. \url{https://doi.org/10.1515/bfp-2019-2008}.

Kleineberg, Michael und Ben Kaden: \enquote{Open Humanities?
ExpertInnenmeinungen über Open Access in den Geisteswissenschaften},
\emph{LIBREAS} 32 (2017), S.~1--31.
\url{https://libreas.eu/ausgabe32/kleineberg/}.

%autor
\begin{center}\rule{0.5\linewidth}{0.5pt}\end{center}

\textbf{Yuliya Fadeeva}, Dr.~phil., arbeitet im Open-Access-Bereich der
Universitätsbibliothek Duisburg-Essen. Sie war am Abschluss des
OGeSoMo-Projektes beteiligt, insbesondere an der Erstellung des
Sammelbandes.

\end{document}
