Das BMBF-geförderte Projekt Open Access in den Geistes- und
Sozialwissenschaften mit dem Schwerpunkt Monografien (OGeSoMo),
angesiedelt an der Universitätsbibliothek (UB) und dem germanistischen
Institut der Universität Duisburg-Essen (UDE), unterstützte 2018--2020
die Entwicklung von Open Access (OA) in den sogenannten Buchdisziplinen
an der Universitätsallianz Ruhr (UAR). Dank einer Anschubfinanzierung
von 75.000 € und beinahe ebenso hoher Eigenbeteiligung der UDE wurde
eine große Anzahl von Monografien und Sammelbänden in Open Access
überführt bzw. als OA erstpubliziert. Im Folgenden werden einige
Eckpunkte und Ergebnisse des Projekts vorgestellt.
