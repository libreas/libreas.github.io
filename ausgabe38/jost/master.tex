\documentclass[a4paper,
fontsize=11pt,
%headings=small,
oneside,
numbers=noperiodatend,
parskip=half-,
bibliography=totoc,
final
]{scrartcl}

\usepackage[babel]{csquotes}
\usepackage{synttree}
\usepackage{graphicx}
\setkeys{Gin}{width=.4\textwidth} %default pics size

\graphicspath{{./plots/}}
\usepackage[ngerman]{babel}
\usepackage[T1]{fontenc}
%\usepackage{amsmath}
\usepackage[utf8x]{inputenc}
\usepackage [hyphens]{url}
\usepackage{booktabs} 
\usepackage[left=2.4cm,right=2.4cm,top=2.3cm,bottom=2cm,includeheadfoot]{geometry}
\usepackage{eurosym}
\usepackage{multirow}
\usepackage[ngerman]{varioref}
\setcapindent{1em}
\renewcommand{\labelitemi}{--}
\usepackage{paralist}
\usepackage{pdfpages}
\usepackage{lscape}
\usepackage{float}
\usepackage{acronym}
\usepackage{eurosym}
\usepackage{longtable,lscape}
\usepackage{mathpazo}
\usepackage[normalem]{ulem} %emphasize weiterhin kursiv
\usepackage[flushmargin,ragged]{footmisc} % left align footnote
\usepackage{ccicons} 
\setcapindent{0pt} % no indentation in captions

%%%% fancy LIBREAS URL color 
\usepackage{xcolor}
\definecolor{libreas}{RGB}{112,0,0}

\usepackage{listings}

\urlstyle{same}  % don't use monospace font for urls

\usepackage[fleqn]{amsmath}

%adjust fontsize for part

\usepackage{sectsty}
\partfont{\large}

%Das BibTeX-Zeichen mit \BibTeX setzen:
\def\symbol#1{\char #1\relax}
\def\bsl{{\tt\symbol{'134}}}
\def\BibTeX{{\rm B\kern-.05em{\sc i\kern-.025em b}\kern-.08em
    T\kern-.1667em\lower.7ex\hbox{E}\kern-.125emX}}

\usepackage{fancyhdr}
\fancyhf{}
\pagestyle{fancyplain}
\fancyhead[R]{\thepage}

% make sure bookmarks are created eventough sections are not numbered!
% uncommend if sections are numbered (bookmarks created by default)
\makeatletter
\renewcommand\@seccntformat[1]{}
\makeatother

% typo setup
\clubpenalty = 10000
\widowpenalty = 10000
\displaywidowpenalty = 10000

\usepackage{hyperxmp}
\usepackage[colorlinks, linkcolor=black,citecolor=black, urlcolor=libreas,
breaklinks= true,bookmarks=true,bookmarksopen=true]{hyperref}
\usepackage{breakurl}

%meta
%meta

\fancyhead[L]{R. Jost, Red. LIBREAS\\ %author
LIBREAS. Library Ideas, 38 (2020). % journal, issue, volume.
\href{https://doi.org/10.18452/23470}{\color{black}https://doi.org/10.18452/23470}
{}} % doi 
\fancyhead[R]{\thepage} %page number
\fancyfoot[L] {\ccLogo \ccAttribution\ \href{https://creativecommons.org/licenses/by/4.0/}{\color{black}Creative Commons BY 4.0}}  %licence
\fancyfoot[R] {ISSN: 1860-7950}

\title{\LARGE{Interview mit Rina Jost}}% title
\author{Rina Jost, Redaktion LIBREAS} % author

\setcounter{page}{1}

\hypersetup{%
      pdftitle={Interview mit Rina Jost},
      pdfauthor={Rina Jost, Redaktion LIBREAS},
      pdfcopyright={CC BY 4.0 International},
      pdfsubject={LIBREAS. Library Ideas, 38 (2020).},
      pdfkeywords={Interview, Bibliothek, Kunst, Kommunikation, Illustration},
      pdflicenseurl={https://creativecommons.org/licenses/by/4.0/},
      pdfcontacturl={http://libreas.eu},
      baseurl={http://libreas.eu},
      pdflang={de},
      pdfmetalang={de}
     }



\date{}
\begin{document}

\maketitle
\thispagestyle{fancyplain} 

%abstracts

%body
\emph{LIBREAS: Vielen Dank, dass Sie sich bereit erklärt haben, unsere
Fragen zu beantworten. Wir haben gemerkt, dass eine ganze Anzahl von
Designer*innen und Künstler*innen, wenn sie für Bibliotheken arbeiten,
gerne als ein Symbol auf Tiere oder Pflanzen zurückgreifen. Uns
interessiert, warum das so ist. Gleichzeitig wollen wir die Möglichkeit
auch nutzen, um von Ihrer Seite zu hören, wie eine solche Zusammenarbeit
mit Bibliotheken abläuft. (Wir denken sonst, ehrlich gesagt, eher von
Seiten der Bibliotheken her.)}

\emph{Wann kamen Sie das erste Mal in Ihrer Arbeit mit Bibliotheken in
Kontakt? Hatte das bei Ihnen \enquote{Vorläufer}, beispielsweise eigene
Erfahrungen aus der Kindheit und Jugend?}

Rina Jost (RJ): Als Kind habe ich sehr gerne Zeit in Bibliotheken
verbracht. Die Bibliotheksstunde war eine meiner Lieblingslektionen in
der Schule und auch privat haben meine Eltern mich und meine Schwester
oft in die Bibliothek mitgenommen, um über das Wochenende (oder für die
Ferien) Comics, Bücher und Filme auszuleihen. Insofern haben mich
Bibliotheken früh geprägt -- auch heute gehe ich manchmal zum Stöbern
und für Inspiration in die Bibliothek. Im Auftragsverhältnis bin ich mit
dem Bücherfest Frauenfeld und der Briefmarke für die Nationalbibliothek
mit Bibliotheken in Kontakt gekommen. Eine unverbindliche Anfrage vor
vier Jahren für die Bibliothek Rapperswil-Jona ist versandet, aber ich
habe gesehen, dass sie mittlerweile mit einer Berufskollegin arbeiten.

\emph{LIBREAS: Unterscheiden sich für Sie als Designerin Bibliotheken
als Kund*innen von anderen Einrichtungen?}

RJ: Nein. Ich arbeite mit Kund*innen aus diversen Branchen zusammen,
darunter Privatpersonen, Einrichtungen aus dem Kultursektor und
kommerziell ausgerichtete Firmen.

\emph{LIBREAS: Wir haben, wenn wir ehrlich sind, wenig Ahnung von den
kreativen Prozessen, die bei einem Design-Prozess stattfinden. Sie haben
auf Ihrer Homepage einige Skizzen aus dem Schaffensprozess
veröffentlicht
(\url{https://rinajost.ch/Briefmarke-125-Jahre-Nationalbibliothek}). Da
ist sichtbar, dass Sie mit der Idee einer Recherche in der Bibliothek
begonnen haben -- aber am Ende unter Wasser gelandet sind und dann dort
ein Fisch auftaucht.}

RJ: Für einen umfassenderen Einblick in den kreativen Prozess empfehle
ich das Making Of für den OITGG Festivalauftritt:
\url{https://rinajost.ch/making-of-oitgg}

\emph{LIBREAS: Wie gehen Sie allgemein vor, um zu Motiven für solche
Arbeiten zu gelangen? Und wie sind Sie bei dieser konkreten Arbeit zu
diesen Motiven gelangt? Wann ist der Fisch hinzugekommen?}

RJ: Zu Beginn einer Auftragsarbeit steht das Briefing und oft eine
Recherche. Danach skizziere ich relativ frei und ohne Schere im Kopf.
Diese ersten Ideen sind oft sehr naheliegend und noch nicht besonders
interessant. Während des Prozesses erweitere ich dann meinen
Gedankenspielraum sowohl zeichnerisch als auch konzeptuell. Ein
Leitmotiv war die Forscherin, die in der Bibliothek auf
Entdeckungsreise/Expedition geht. Ich habe mir auch überlegt, was man
auf keinen Fall in der Bibliothek haben möchte (Feuer, Wasser,
Lebensmittel, \ldots) um der Bildwelt einen interessanten Aspekt
hinzufügen zu können. So ist die etwas absurde Idee der
Unterwasserbibliothek entstanden, in die meine Protagonistin nun
eintaucht. Die Tiere (Krabbe, Fisch, Tintenfisch) sind erst in der
Animation für die Augmented Reality dazugekommen, um die Unterwasserwelt
zu bevölkern. Als Tiercharaktere lenken sie nicht von der Protagonistin
ab, sondern unterstützen das Narrativ. Sie sind natürlich auch
Sympathieträger.

\emph{LIBREAS: Stach das Projekt für Sie heraus aus Ihrem sonstigen
Schaffen oder würden Sie die eher als typisch für Ihre Projekte ansehen?
Für andere Projekte haben Sie auch auf Augmented Reality
zurückgegriffen, aber nicht für alle. Warum in diesem Fall doch? War es
vielleicht eine Vorgabe der Post oder der Nationalbibliothek?}

RJ: Es war keine Vorgabe. Die Idee entstand während des kreativen
Prozesses und wurde von mir eingebracht. Es war der Nationalbibliothek
wichtig, nicht als 'verstaubte Bibliothek' dargestellt zu werden,
sondern innovativ aufzutreten und das Spektrum der Dienstleistungen
abzubilden. Die Nationalbibliothek sammelt in diversen Medien und unter
anderem auch an der Schnittstelle von analog und digital. Da hat sich
Augmented Reality als erweitertes Konzept angeboten, weil es perfekt zum
Briefing passt und den innovativen Aspekt hervorhebt. Ausserdem wird die
Betrachter*in auf einer Meta-Ebene so selbst zur Forscher*in. Es ist
übrigens die erste AR-Briefmarke in der Schweiz.

Das Medium der Augmented Reality machte bei dieser Auftragsarbeit
konzeptuell Sinn. Der Effekt soll nicht im Vordergrund stehen, sondern
die inhaltliche Idee unterstützen. Bei meinen privaten Arbeiten nutze
ich AR eher als Spielerei und um meine handwerklichen und technischen
Fähigkeiten zu erweitern.

\emph{LIBREAS: Sie haben eine spezifisch anthropomorphe Form gewählt --
der Krebs hat etwas sehr menschliches. Das machen auch andere
Designer*innen und Künstler*innen. Denken Sie, nur so sind Tiere und
Bibliotheken miteinander zu verbinden? Passen Tiere als solche sonst
nicht zu Bibliotheken?}

RJ: Dies ist eine gestalterisch-kommunikative Entscheidung, die nicht
viel mit Bibliotheken zu tun hat. Mit einer anthropomorphen Figur kann
sich der/die Betrachter*in besser identifizieren und das Spektrum an
Ausdrücken und Emotionen, die transportiert werden können, ist viel
grösser als bei einer naturalistischen/wissenschaftlichen Darstellung.

Gerade wenn sich die Kommunikation an Kinder richtet, wird dieser Effekt
häufig genutzt. Ein grosser Teil der \emph{illustrierten} visuellen
Kommunikation in Bibliotheken richtet sich vermutlich an ein junges
Publikum, weshalb Bibliothekar*innen möglicherweise überproportional oft
mit anthropomorphen Formen konfrontiert sind. Die Wahl einer
anthropomorphen Figur ist natürlich auch eine Möglichkeit, Personen mit
verschiedensten Hintergründen anzusprechen. Wenn die Sympathiefigur
hingegen ein Mensch ist, stellt sich die Frage, wie man Diversität
abbildet. Ist die Figur männlich, weiblich oder nonbinär? Welche
Hautfarbe hat sie? Welche Kleidung trägt sie? Die anthropomorphe Form
kann ein Weg sein, solche gestalterischen Fragestellungen zu umschiffen.

In Lehrmitteln, Wissensbüchern oder Museen stehen dann wieder andere
Aspekte im Vordergrund und es wird öfter auf die wissenschaftliche
Illustration / naturalistische Abbildung zurückgegriffen.

Sie sehen, es hat vermutlich mehr mit der Art und Absicht der visuellen
Kommunikation zu tun als mit der Branche selbst. Im Fall der Briefmarke
stehen nun mal die Sichtbarkeit und auch der Verkauf der Briefmarke im
Vordergrund, weshalb die Figuren möglichst sympathisch wirken sollen.

\emph{LIBREAS: Wie kam der Kontakt für dieses Projekt zustande? Sind Sie
mit Ideen auf die Nationalbibliothek oder die Post zugegangen oder kamen
die Kolleg*innen zu ihnen? War das der übliche Weg, wie solche Projekte
zustande kommen oder eher ungewöhnlich?}

RJ: In einem ersten Schritt ist meines Wissens die Nationalbibliothek
auf die Post zugegangen. Ich wurde dann direkt von der Post angefragt
und mit zwei anderen Gestalter*innen zu einem (vergüteten)
Gestaltungswettbewerb für die Sondermarke eingeladen. Es gab ein
Briefing mit einem Vertreter der Nationalbibliothek und einer
Vertreterin der Post. Eine Jury der Post entschied über die
eingereichten Entwürfe. Die Post war auch meine direkte Kundin. Dieser
Prozess scheint gängig zu sein im Hinblick auf Sondermarken.

Für andere Illustrationsaufträge werde ich in der Regel direkt von den
Endkund*innen angefragt, manchmal laufen Anfrage und Kommunikation auch
über eine Werbeagentur oder über ein Organisationskomitee.

\emph{LIBREAS: Welche Ziele hatte denn, Ihrer Meinung nach, die
Bibliothek konkret mit den Projekt? Gab es besondere Herausforderungen,
zum Beispiel seitens der Bibliotheksleitung?}

RJ: Die Nationalbibliothek wollte auf ihr Jubiläumsjahr aufmerksam
machen. In dessen Rahmen hat sie auch viele Veranstaltungen geplant.
Leider haben die Pandemie und der Lockdown einen Teil dieser Aktivitäten
auf Eis gelegt und sämtliche Kommunikation wurde gestoppt. Ich denke,
für die Nationalbibliothek war die Briefmarke ein Schauplatz von vielen.
Es gab keine grosse Enthüllung, sondern war wohl eher dazu gedacht, ein
breites Publikum auf das Angebot und das Jubiläumsjahr der
Nationalbibliothek aufmerksam zu machen.

\emph{LIBREAS: Welche Erfahrungen nehmen Sie aus den Projekten für die
Zusammenarbeit mit Bibliotheken mit? Was wünschen Sie sich für Ihre
zukünftige Arbeit bei solchen Projekten von Bibliotheken? Würden Sie
anderen Designer*innen empfehlen, auch solche Projekte durchzuführen
oder eher nicht?}

RJ: Da meine Kundin für die Briefmarke die Post war, kann ich nicht viel
zur Zusammenarbeit mit der Nationalbibliothek sagen, abgesehen vom gut
kommunizierten Briefing.

Ich habe bereits mit Mitarbeiter*innen der Kantonsbibliothek Thurgau
gearbeitet. Eine geplante illustrierte Lesung im Rahmen des Bücherfests
2020 musste leider verschoben/abgesagt werden. Die Kommunikation, das
Engagement und die Motivation der Bibliotheksmitarbeiter*innen ist
grossartig und sehr ansteckend.

Die visuelle Kommunikation und insbesondere natürlich die Illustration
ist ein gutes Tool für Bibliotheken um ein breites Publikum anzusprechen
und ihre Kommunikation attraktiv zu gestalten. Je nach Anwendungsbereich
kann Illustration auch unterstützend Wissen vermitteln oder auch
überraschen und zum Nachdenken anregen. Die Bildsprache gehört zu
unserem Repertoire wie das gesprochene und geschriebene Wort und ist in
Bibliotheken bereits sehr präsent (Kinderbücher, Comics, Filme,
Titelbilder, Signaletik, Gebrauchsanweisungen, Karten, ...). Ich denke,
es kann sehr wertvoll sein, dieses Potential bewusster zu nutzen.

%autor
\begin{center}\rule{0.5\linewidth}{0.5pt}\end{center}

\textbf{Rina Jost} (*1987) lebt und arbeitet als selbstständige
Illustratorin in Frauenfeld. Sie hat an der HSLU Design \& Kunst
studiert und mit dem BA in Visueller Kommunikation, mit Vertiefung in
Illustration abgeschlossen. Ihre Arbeit wurde u.a. von der Society of
Illustrators (NY), den World Illustration Awards (UK, Shortlist) und den
Hiii Illustration Awards (China) anerkannt. 2017 wurde sie mit dem
ersten Förderpreis der Stadt Frauenfeld ausgezeichnet.

\url{https://www.rinajost.ch} \textbar{}
\url{https://instagram.com/rinajost} (@rinajost)

\end{document}
