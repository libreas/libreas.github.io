Mit den DEAL-Verträgen steht das Versprechen im Raum, administrative und
(autor*innenseitig) finanzielle Hürden für Open Access zu reduzieren.
Für Bibliotheken bedeuten die DEAL-Verträge neben der Zentralisierung
von Finanzströmen häufig auch mehr Verantwortung in anderen Bereichen:
Information über die Verträge allgemein und Rechte von Autor*innen,
Verifikation berechtigter Artikel, direkte Beratung bei Fragen rund um
Workflows und Creative-Commons-Lizenzen und vieles mehr. Der Beitrag ist
eine Ausarbeitung des Vortrags beim 1. DEAL Praxis-Webinar (Voigt 2020)
und gibt einen Überblick, in welchen Handlungsfeldern die
Universitätsbibliothek der Technischen Universität (TU) Berlin aktiv
wird. Ein besonderer Fokus liegt dabei auf der Kommunikation mit den
Wissenschaftler*innen, um den größtmöglichen Anteil an Open Access und
Offenheit aus den DEAL-Verträgen herauszuholen.
