\documentclass[a4paper,
fontsize=11pt,
%headings=small,
oneside,
numbers=noperiodatend,
parskip=half-,
bibliography=totoc,
final
]{scrartcl}

\usepackage[babel]{csquotes}
\usepackage{synttree}
\usepackage{graphicx}
\setkeys{Gin}{width=.4\textwidth} %default pics size

\graphicspath{{./plots/}}
\usepackage[ngerman]{babel}
\usepackage[T1]{fontenc}
%\usepackage{amsmath}
\usepackage[utf8x]{inputenc}
\usepackage [hyphens]{url}
\usepackage{booktabs} 
\usepackage[left=2.4cm,right=2.4cm,top=2.3cm,bottom=2cm,includeheadfoot]{geometry}
\usepackage{eurosym}
\usepackage{multirow}
\usepackage[ngerman]{varioref}
\setcapindent{1em}
\renewcommand{\labelitemi}{--}
\usepackage{paralist}
\usepackage{pdfpages}
\usepackage{lscape}
\usepackage{float}
\usepackage{acronym}
\usepackage{eurosym}
\usepackage{longtable,lscape}
\usepackage{mathpazo}
\usepackage[normalem]{ulem} %emphasize weiterhin kursiv
\usepackage[flushmargin,ragged]{footmisc} % left align footnote
\usepackage{ccicons} 
\setcapindent{0pt} % no indentation in captions

%%%% fancy LIBREAS URL color 
\usepackage{xcolor}
\definecolor{libreas}{RGB}{112,0,0}

\usepackage{listings}

\urlstyle{same}  % don't use monospace font for urls

\usepackage[fleqn]{amsmath}

%adjust fontsize for part

\usepackage{sectsty}
\partfont{\large}

%Das BibTeX-Zeichen mit \BibTeX setzen:
\def\symbol#1{\char #1\relax}
\def\bsl{{\tt\symbol{'134}}}
\def\BibTeX{{\rm B\kern-.05em{\sc i\kern-.025em b}\kern-.08em
    T\kern-.1667em\lower.7ex\hbox{E}\kern-.125emX}}

\usepackage{fancyhdr}
\fancyhf{}
\pagestyle{fancyplain}
\fancyhead[R]{\thepage}

% make sure bookmarks are created eventough sections are not numbered!
% uncommend if sections are numbered (bookmarks created by default)
\makeatletter
\renewcommand\@seccntformat[1]{}
\makeatother

% typo setup
\clubpenalty = 10000
\widowpenalty = 10000
\displaywidowpenalty = 10000

\usepackage{hyperxmp}
\usepackage[colorlinks, linkcolor=black,citecolor=black, urlcolor=libreas,
breaklinks= true,bookmarks=true,bookmarksopen=true]{hyperref}
\usepackage{breakurl}

%meta
\expandafter\def\expandafter\UrlBreaks\expandafter{\UrlBreaks%  save the current one
  \do\a\do\b\do\c\do\d\do\e\do\f\do\g\do\h\do\i\do\j%
  \do\k\do\l\do\m\do\n\do\o\do\p\do\q\do\r\do\s\do\t%
  \do\u\do\v\do\w\do\x\do\y\do\z\do\A\do\B\do\C\do\D%
  \do\E\do\F\do\G\do\H\do\I\do\J\do\K\do\L\do\M\do\N%
  \do\O\do\P\do\Q\do\R\do\S\do\T\do\U\do\V\do\W\do\X%
  \do\Y\do\Z}
%meta

\fancyhead[L]{Redaktion LIBREAS\\ %author
LIBREAS. Library Ideas, 38 (2020). % journal, issue, volume.
\href{https://doi.org/10.18452/23482}{\color{black}https://doi.org/10.18452/23482}
{}} % doi 
\fancyhead[R]{\thepage} %page number
\fancyfoot[L] {\ccLogo \ccAttribution\ \href{https://creativecommons.org/licenses/by/4.0/}{\color{black}Creative Commons BY 4.0}}  %licence
\fancyfoot[R] {ISSN: 1860-7950}

\title{\LARGE{Das liest die LIBREAS, Nummer \#7 (Herbst / Winter 2020)}}% title
\author{Redaktion LIBREAS} % author

\setcounter{page}{1}

\hypersetup{%
      pdftitle={Das liest die LIBREAS, Nummer \#7 (Herbst / Winter 2020)},
      pdfauthor={Redaktion LIBREAS},
      pdfcopyright={CC BY 4.0 International},
      pdfsubject={LIBREAS. Library Ideas, 38 (2020).},
      pdfkeywords={Literaturübersicht, Bibliothekswissenschaft, Informationswissenschaft, Bibliothekswesen, Rezension,literature overview, library science, information science, library sector, review},
      pdflicenseurl={https://creativecommons.org/licenses/by/4.0/},
      pdfcontacturl={http://libreas.eu},
      baseurl={},
      pdflang={de},
      pdfmetalang={de}
     }



\date{}
\begin{document}

\maketitle
\thispagestyle{fancyplain} 

%abstracts

%body
Beiträge von Ben Kaden (bk), Karsten Schuldt (ks), Michaela Voigt (mv),
Eva Bunge (eb)

\hypertarget{zur-kolumne}{%
\section{1. Zur Kolumne}\label{zur-kolumne}}

Ziel dieser Kolumne ist es, eine Übersicht über die in der letzten Zeit
erschienene bibliothekarische, informations- und
bibliothekswissenschaftliche sowie für diesen Bereich interessante
Literatur zu geben. Enthalten sind Beiträge, die der LIBREAS-Redaktion
oder anderen Beitragenden als relevant erschienen.

Eine Themenvielfalt sowie ein Nebeneinander von wissenschaftlichen und
nicht-wissenschaft\-lichen Ansätzen werden angestrebt. Traditionelle
Publikationen sollen ebenso erwähnt werden wie Blogbeiträge oder Videos
beziehungsweise TV-Beiträge.

Hinweise auf auf erschienene Literatur oder Beiträge in anderen Formaten
sind gern gesehen und können der Redaktion mitgeteilt werden (Siehe
\href{http://libreas.eu/about/}{Impressum}, Mailkontakt für diese
Kolumne ist
\href{mailto:zeitschriftenschau@libreas.eu}{\nolinkurl{zeitschriftenschau@libreas.eu}}.)
Die Koordination der Kolumne liegt bei Karsten Schuldt. Verantwortlich
für die Inhalte sind die jeweiligen Beitragenden. Die Kolumne
unterstützt den Vereinszweck des LIBREAS-Vereins zur Förderung der
bibliotheks- und informationswissenschaftlichen Kommunikation.

LIBREAS liest gern und viel Open-Access-Veröffentlichungen. Wenn sich
Beiträge dennoch hinter eine Bezahlschranke verbergen, werden diese
durch \enquote{{[}Paywall{]}} gekennzeichnet. Zwar macht das Plugin
\href{http://unpaywall.org/}{Unpaywall} das Finden von legalen
Open-Access-Versionen sehr viel einfacher. Als Service an der
Leserschaft verlinken wir OA-Versionen, die wir vorab finden konnten,
jedoch auch direkt. Für alle Beiträge, die dann immer noch nicht frei
zugänglich sind, empfiehlt die Redaktion Werkzeuge wie den
\href{https://openaccessbutton.org/}{Open Access Button} oder
\href{https://core.ac.uk/services/discovery/}{CORE} zu nutzen oder auf
Twitter mit
\href{https://twitter.com/hashtag/icanhazpdf?src=hash}{\#icanhazpdf} um
Hilfe bei der legalen Dokumentenbeschaffung zu bitten.

\hypertarget{artikel-und-zeitschriftenausgaben}{%
\section{2. Artikel und
Zeitschriftenausgaben}\label{artikel-und-zeitschriftenausgaben}}

Rose-Wiles, Lisa M. ; Shea, Gerard ; Kehnemuyi, Kaitlin (2020).
\emph{Read in or check out: A four-year analysis of circulation and
in-house use of print books}. In: The Journal of Academic Librarianship,
46(4), 102157. \url{https://doi.org/10.1016/j.acalib.2020.102157}
{[}Paywall{]}

Für die Bibliothek einer kleineren, privaten Universität in New Jersey
untersuchten die Autorinnen, wie sich die Nutzung von Medien in der
Bibliothek im Vergleich zur Ausleihe selber entwickelte. Dabei sammelten
sie über vier Jahre Daten (2015--2018) zu den Medien, die sich in
Buchwägen zum Zurückstellen ansammelten. Es ist klar, dass so nicht die
gesamte In-House Nutzung erfasst wurde, sondern Studierende bekanntlich
auch selber Medien zurückstellen oder absichtlich verstellen können.
Dennoch ergibt dies zumindest Vergleichswerte.

Alle Nutzungsformen von Medien gehen dabei zurück, teilweise massiv: die
Ausleihe von physischen Beständen, Nutzung In-House und auch die Nutzung
elektronischer Medien. Am geringsten ist der Rückgang bei der Fernleihe.
Einige der Tendenzen, die der Artikel beschreibt, sind wohl auf den
lokalen und nationalen Kontext zurückzuführen.

Hervorzuheben ist aber ein Ergebnis: Nachdem die Nutzung elektronischer
Medien über lange Zeit stieg und es möglich war zu vermuten, dass die
Nutzung von physischen Medien durch elektronische ersetzt würde (wogegen
allerdings auch schon viele Studien zur konkreten Mediennutzung
sprachen), ging in diesem Beispiel auch die Nutzung elektronischer
Medien zurück. Es gibt also einen allgemeinen Rückgang der
Mediennutzung, nicht einen für spezifische Medienformen. (In den Daten
der Bibliotheksstatistiken im DACH-Raum zeigt sich dieser Trend noch
nicht so eindeutig. Es wird aber interessant sein, ihn im Blick zu
behalten.) (ks)

\begin{center}\rule{0.5\linewidth}{0.5pt}\end{center}

Lund, Brady D. ; Waltson, Matthew (2020). \emph{Anxiety-uncertainty
management theory as a prelude to Mellon's Library Anxiety}. In: The
Journal of Academic Librarianship 46 (2020) 4, 102160,
\url{https://doi.org/10.1016/j.acalib.2020.102160} {[}Paywall{]}

Dieser Artikel scheint eine Nebenfrage zu behandeln, aber es geht am
Ende tatsächlich um einen wichtigen Punkt. Das Thema
\enquote{Bibliotheksangst} wird immer wieder einmal -- im DACH-Raum vor
allem in Abschlussarbeiten -- behandelt. Es zählt nicht zu den
meist-besprochenen Themen, aber es gibt ein ständiges Interesse an ihm.
Anderswo werden immer wieder einmal Studien in Bibliotheken unternommen
oder Projekte angegangen, Bibliotheksangst zu reduzieren.

Lund \& Waltson diskutieren nun, welches Konzept von Angst hinter diesen
Arbeiten stehen. Sie postulieren, dass vor allem psychologische Konzepte
genutzt werden, welche diese Angst aus Prädispositionen der einzelnen
Menschen erklären. Dies würde aber den Effekt haben, dass diese Angst
als im Kern nicht veränderbar angesehen würde -- weder von den Personen
noch von den Bibliotheken. Zudem würde der Blick weniger auf einzelne
Personen und eher auf Gruppen gelegt (zum Beispiel im Sinne von
Bibliotheksangst von Studierenden im ersten Semester). Stattdessen
schlagen sie vor, \enquote{anxiety-uncertainty management theory} zu
nutzen, mit der Informationsverhalten vor dem Hintergrund von
Erwartungen und Erfahrungen einzelner Individuen erklärt wird. Auf
Erwartungen und Erfahrungen sei viel besser einzugehen als auf mehr oder
minder feste Charaktereigenschaften. Insoweit würde mit einem solchen
theoretischen Wechsel auch ein Raum geschaffen für Bibliotheken, aktiv
auf Bibliotheksangst zu reagieren. (ks)

\begin{center}\rule{0.5\linewidth}{0.5pt}\end{center}

Mazarakis, Athanasios; Bräuer, Paula (2020). \emph{Gamification of an
open access quiz with badges and progress bars: An experimental study
with scientists.} In: GamiFIN Conference 2020, Levi, Finland, April
1--3, 2020. \url{http://ceur-ws.org/Vol-2637/paper7.pdf}

Eine Studie mit 28 Wissenschaftler*innen in Deutschland ergab, dass der
Einsatz von Bausteinen aus dem Game Design (Progress Bar, Badges)
messbar zur Motivation zur Auseinandersetzung mit einem Thema, in diesem
Fall Open Access, beiträgt. (bk)

\begin{center}\rule{0.5\linewidth}{0.5pt}\end{center}

Keller, Alice: „\emph{Lust ja, aber keine Zeit!": Publikationsverhalten
von Bibliothekaren und Informationswissenschaftlern.} In: Bibliothek -
Forschung und Praxis, 44(2), 231--245.
\url{https://doi.org/10.1515/bfp-2020-0019}

In ihrer Studie untersuchte Alice Keller die Gründe für das Phänomen,
dass es zumindest im deutschsprachigen Raum eine Herausforderung
darstellt, \enquote{Artikel für subskriptionsbasierte Zeitschriften der
Bibliotheks- und Informationswissenschaften zu akquirieren}. Dabei
pointiert der Titel des Aufsatzes bereits einen Teil des Ergebnisses.
Ein anderer, von ihr ebenfalls benannter Grund, ist die hohe Zahl von
deutschsprachigen Fachzeitschriften, die in Konkurrenz um einen
vergleichsweise überschaubaren Pool an potentiellen Autor*innen stehen.
Insgesamt flossen 468 Rückmeldungen auf die Umfrage in die
SurveyMonkey-basierte Auswertung ein. Ein interessantes Ergebnis ist,
dass das Publizieren von Artikeln für die eigene Entwicklung
nachgeordnet erscheint, also im Umkehrschluss in diesem Berufsfeld nicht
zwingend relevant für die Karriere ist. Wo geschrieben wird, geschieht
dies oft aus einer inneren Motivation und mit dem Ziel der
\enquote{allgemeinen Berufsentwicklung}. Festzuhalten sind weiterhin
eine hohe Affinität zu Open Access, sowie der Trend, dass die
Publikationsfreude mit zunehmendem Alter wächst, was die Autorin auf
ungleich verteilte zeitliche Freiräume zurückführt. Dabei mag auch eine
Rolle spielen, dass das Schreiben von Artikeln nur für einen Teil der im
Berufsfeld Aktiven überhaupt zum Arbeitsprofil zählt. So lässt sich als
alternatives Fazit zu \enquote{Lust ja, aber keine Zeit!} auch
festhalten: Wer im deutschen Bibliothekswesen schreiben will, muss es
also wirklich wollen und wer nicht will, kann dennoch Karriere machen.
Ob dies auch für den Kernbereich der Bibliotheks- und
Informationswissenschaft im wirklich wissenschaftlichen Sinn gilt,
bleibt unklar, da die Auswahl des Samples die Zugehörigkeit zu einer
wissenschaftlicher Community und der bibliothekarischen Fachcommunity
nicht gesondert differenziert. (bk)

\begin{center}\rule{0.5\linewidth}{0.5pt}\end{center}

Andrae, M., Blumesberger, S., Edler, S., Ernst, J., Fiedler, S.,
Haslinger, D., Neustätter, G. und Trieb, D. (2020)
„\emph{Barrierefreiheit für Repositorien. Ein Überblick über technische
und rechtliche Voraussetzungen}", Mitteilungen der Vereinigung
Österreichischer Bibliothekarinnen und Bibliothekare, 73(2). S.
259--277. \url{https://doi.org/10.31263/voebm.v73i2.3640}

Barrierefreiheit geht alle an, auch Repositorienbetreiber*innen.
Mitunter mag dieser Teilaspekt im Alltag hinten anstehen, wenn dieser
vor allem aus inhaltlicher Betreuung (in Hinblick auf Contentakquise im
Sinne von Steigerung des Open-Access-Outputs wie auch von (formaler und
technischer) Qualitätssicherung), sachgemäßer Erschließung und
technischer Weiterentwicklung des Repositoriums besteht. Andrae et al.
arbeiten deutlich heraus, warum der barrierefreien Gestaltung von
Repositorien (mehr) Aufmerksamkeit geschenkt werden sollte: Rechtliche
Vorgaben sind das Eine, der eigene Anspruch beziehungsweise Wahrnehmung
des eigentlichen Auftrags (Inhalte frei zugänglich machen, in der Regel
nach den Prinzipien von Open Access) das Andere. Die Autor*innengruppe
fokussiert, verständlicherweise, auf die Österreichische Rechtslage.
Alle anderen Absätze dürften jedoch uneingeschränkt auch für
Repositorienbetreiber*innen in anderen Ländern Relevanz entfalten und
einen nützlichen Überblick über die zahlreichen Bereiche (Design,
Guidelines und Hilfetexte, Sprache, Dateiformate und vieles mehr)
darstellen, die bei der Überarbeitung in den Blick genommen werden
sollten. (mv)

\begin{center}\rule{0.5\linewidth}{0.5pt}\end{center}

Garner, Jane (2020). \emph{Experiencing time in prison: the influence of
books, libraries and reading}. In: Journal of Documentation 76 (2020) 5,
1033--1050, \url{https://doi.org/10.1108/JD-07-2019-0128} {[}Paywall{]}

Gefangene in Gefängnissen nutzen die Bibliothek und das Lesen vor allem,
um Zeit zu überbrücken. Die Studie befragte eine Anzahl von Gefangenen
in sieben Einrichtungen in verschiedenen australischen Bundesstaaten
(und soll Teil einer grösseren Studie zur Bibliotheksnutzung in
Gefängnissen in diesem Land sein). Das Ergebnis war eigentlich
durchgängig gleich. Gefangene haben viel \enquote{unstrukturierte} Zeit,
die sie aber nur eingeschränkt nutzen können, da ihnen nur wenige
Freizeitmöglichkeiten zur Verfügung stehen. Garner betont mehrfach, dass
dieses Ergebnis konträr zu anderen Studien wäre, welche die Nutzung von
Öffentlichen Bibliotheken untersuchen würden. In solchen würde
\enquote{Zeit rumbringen} als Grund für die Nutzung einer Bibliothek
nicht vorkommen. (Anmerkung: In Studien zur Nutzung von Bibliotheken
durch Menschen ohne festen Wohnsitz kommt dies aber schon manchmal vor.
Diese werden von Garner nicht referiert.)

In einem langen Teil des Textes schildert Garner die Bibliotheken in den
Gefängnissen, die sie besucht hat. Es liest sich wie ein Versuch, die
(bibliothekarische) Öffentlichkeit auf deren Zustand aufmerksam zu
machen. Für ein so reiches Land wie Australien ist er nämlich
beschämend. Nur eine der sieben Bibliotheken hat eine ausgebildete
Bibliothekarin und einen festen Etat. Die anderen Bibliotheken werden
von Gefangenen geleitet und haben fast nie einen Erwerbungsetat. Die
Bestände bestehen vor allem aus Geschenken von Gefangenen selber. Der
Zugang zu ihnen ist fast immer beschränkt. Während das Ergebnis von
Garner ernstzunehmen ist, wären die von ihr beschriebenen Bibliotheken
auch gar nicht in der Lage, andere Aufgaben (beispielsweise Bildung) zu
unterstützen. (ks)

\begin{center}\rule{0.5\linewidth}{0.5pt}\end{center}

Rekdal, Ole Bjørn (2014): \emph{Academic Urban Legends}. In: Social
Studies of Science 44 (2014) 4, 638--654,
\url{https://doi.org/10.1177\%2F0306312714535679}

Die Qualität wissenschaftlicher Forschung war zuletzt immer wieder
Thema, sowohl in den Medien als auch im Fachdiskurs. Die
Geschäftspraktiken der Predatory Publishers haben diesem Problem eine
größere mediale Aufmerksamkeit beschert, aber die systematischen Effekte
der \enquote{Publish-or-Perish}-Mentalität sind in Bibliotheken schon
lange bekannt und haben nicht weniger einen Einfluss auf die Qualität
(und Quantität) wissenschaftlicher Publikationen.

Ole Bjørn Rekdal untersucht in seinem Aufsatz, welche Folgen ungenaue,
fehlerhafte oder übereilte Zitate in wissenschaftlichen
Veröffentlichungen auf den Forschungsprozess haben können. Er macht dies
exemplarisch an der urbanen Legende fest, dass Spinat viel Eisen
enthalten soll. Dabei stößt er auf eine Erklärung für die Entstehung
dieser Legende, die sich aber wiederum selbst als (sich inzwischen
selbst perpetuierende) Legende herausstellt.

Rekdals Aufsatz ist sehr lesenswert geschrieben und auch sechs Jahre
nach seiner Veröffentlichung noch äußerst relevant. (eb)

\hypertarget{covid-19-und-die-bibliotheken-erste-welle}{%
\subsection{2.1 COVID-19 und die Bibliotheken, Erste
Welle}\label{covid-19-und-die-bibliotheken-erste-welle}}

Wang, Ting ; Lund, Brady (2020). \emph{Announcement Information Provided
by United States' Public Libraries during the 2020 COVID-19 Pandemic}.
In: Public Library Quarterly, 39(4), 283--294.
\url{https://doi.org/10.1080/01616846.2020.1764325}

In einem der ersten wissenschaftlichen Texte, die sich auf die
Aktivitäten (Öffentlicher) Bibliotheken während der COVID-19 Pandemie
beziehen, zeigten Wang \& Lund welche Informationen von Public Libraries
in den USA verbreitet wurden. Sie sammelten aus einer nach Grösse der
Gemeinden gesteuerten Zufallsauswahl von Bibliotheken die jeweiligen
Hinweise von deren Homepage zwischen dem 14.03. und 12.04.2020. Die
Auswertung erfolgt auf der Basis früherer Studien zu übernommenen Rollen
von Bibliotheken während realer Krisen. In diesen wurden schon
verschiedene mögliche Rollen benannt. Die Frage war nun, welche in der
Pandemie 2020 übernommen wurden.

In der Auswertung zeigt sich, dass die meisten Bibliotheken vor allem
darüber informierten, dass sie geschlossen hatten, dass Veranstaltungen
ausfielen und welche Angebote trotzdem angeboten wurden. Rund die Hälfte
der Bibliotheken vermittelte Hinweise zu Vorsichtsmassnahmen
(insbesondere Hygiene- und Abstandsgebote). 69\,\% vermittelten Links zu
vertrauenswürdigen Quellen für Gesundheitsinformationen und 53\,\%
Hinweise zum Finden weiterer vertrauenswürdiger Quellen. Wang \& Lund
postulieren, dass eine Information, welche durch die jeweilige
Bibliotheksleitung gegengezeichnet wurde, mehr Gewicht hätte als
Informationen ohne eine solche direkte Zuordnung. Bei 26\,\% der
Bibliotheken fanden sie eine solche Unterschrift. Im Untersuchungsmonat
veränderten einige Bibliotheken die Informationen, welche sie zur
Verfügung stellten und zwar dahingehend, dass sie mehr Wert auf
Informationen über sich selber (Schliessungen und Angebote) legten.
Grundsätzlich lieferten Bibliotheken aus urbanen Gegenden mehr
Informationen als solche aus suburbanen oder ländlichen Gemeinden. (ks)

\begin{center}\rule{0.5\linewidth}{0.5pt}\end{center}

Tammaro, Anna Maria (2020). \emph{COVID 19 and Libraries in Italy}. In:
International Information \& Library Review, 52(3), 216--220.
\url{https://doi.org/10.1080/10572317.2020.1785172}

Die kurze Kolumne von Tammaro zur Arbeit von italienischen Bibliotheken
während der COVID-19-Krise basiert auf einer zeitnahen Umfrage, bei
denen 70 Einrichtungen -- eine eher kleine Zahl -- antworteten.
Grundsätzlich ist das Bild aber, dass (auch) in Italien Bibliotheken
nach der Schliessung der physischen Räume begannen, digitale Angebote zu
machen. Vorhandene Angebote wurden beworben, Veranstaltungen im
digitalen Raum organisiert. Es wurde versucht, auch gedruckte Medien
zugänglich zu halten. Insbesondere wurde die Fernleihe aufrechterhalten
und massiv genutzt. Tammaro berichtet, dass zumindest in den ersten
Monaten die Nutzungszahlen der digitalen Angebote explodiert seien.

Die Krise hätte die strukturellen Probleme der italienischen
Bibliotheken aber auch sichtbarer gemacht: Eine ganze Reihe von
Kolleg*innen, die prekär beschäftigt waren, verloren ihren Arbeitsplatz.
Fehlende strategische Vorstellungen für die Weiterentwicklung
bedeuteten, dass viele Bibliotheken jetzt in der Pandemie schnell
Entscheidungen treffen mussten. Die massiven Unterschiede in Etat und
Infrastruktur zwischen Norden und Süden Italiens seien ebenso spürbar
geworden. (ks)

\begin{center}\rule{0.5\linewidth}{0.5pt}\end{center}

Walsh, Benjamin ; Rana, Harjinder (2020). \emph{Continuity of Academic
Library Services during the Pandemic: The University of Toronto
Libraries' Response}. In: Journal of Scholarly Publishing, e51404,
\url{https://doi.org/10.3138/jsp.51.4.04} {[}Paywall{]}

Während alle Bibliotheken in der COVID-19-Pandemie mit den gleichen
Herausforderungen umzugehen hatten, haben bislang nur relativ wenige
ihre Lösungen zusammenfassend dargestellt. Dieser Texte tut dies für das
recht grosse Bibliothekssystem der University of Toronto. Die Lösungen
sind wohl die, welche auch in anderen Bibliotheken zu finden waren:
Umstellung auf Online-Lösungen, Anstieg des Beratungsbedarfs von
Studierenden und Forschenden, mit denen umzugehen war, Ausweitung des
Angebots von elektronischen Medien. Besonders ist vielleicht, dass die
Bibliotheken der University -- die aufgrund der Grösse ihres Bestandes
auch an gedruckten Medien für andere Bibliotheken ansonsten eine Basis
für die Fernleihe darstellen -- auch begannen, E-Books in die Fernleihe
zu geben. Die Fragen am Ende des Artikels, wie sich die Nutzung der
elektronischen Medien und des Raumes der Bibliothek nach der Pandemie
entwickeln werden, werden sich in allen anderen Bibliotheken stellen.
(ks)

\begin{center}\rule{0.5\linewidth}{0.5pt}\end{center}

Craft, Anna R. (2020). \emph{Remote Work in Library Technical Services:
Connecting Historical Perspectives to Realities of the Developing
COVID-19 Pandemic}. In: Serials Review, 46(3), 227--231.
\url{https://doi.org/10.1080/00987913.2020.1806658}

Die Autorin dieser Kolumne, welche explizit im Frühjahr 2020 geschrieben
wurde, nimmt die rasante Veränderung der Arbeit in vielen Bibliotheken
hin zu remote work zu Beginn der Pandemie zum Anlass, einen Überblick
zur schon vorhandenen Literatur zu solcher Arbeitsweise in Bibliotheken
zu geben. Sie zeigt, dass Projekte und Überlegungen dazu schon in den
frühen 1990er Jahren starteten und schon einiges an Wissen vorhanden
ist. Gerade Katalogarbeit wurde schon länger von ausserhalb der
Bibliothek erledigt. Bibliotheken hätten während der Pandemie auf diesem
Wissen aufbauen können.

Aus der Literatur erwähnt sie, dass remote work verändert, wie Aufgaben
erteilt und erledigt werden, wie der notwendige Zeitraum für diese
Aufgaben bestimmt wird und wie sie evaluiert wird. Zudem erwähnt sie,
dass remote work Probleme wie das Gefühl von Isolation, die
Notwendigkeit, remote in Teams zu kommunizieren sowie das Finden der
richtigen Work-Life-Balance mit sich bringen. Für die Zeit nach der
Pandemie ruft sie dazu auf, die Erfahrungen aus Bibliotheken mit remote
work zu sammeln und zu diskutieren. (ks)

\begin{center}\rule{0.5\linewidth}{0.5pt}\end{center}

Yuvaraj, Mayank (2020). \emph{Global responses of health science
librarians to the COVID-19 (Corona virus) pandemic: a desktop analysis}
(Regular Feature: International Perspectives And Initiatives). In:
Health Information and Libraries Journal, e12312,
\url{https://doi.org/10.1111/hir.12321}

In dieser Kolumne gibt die Autorin eine ganz kurze Übersicht von
Initiativen von Bibliotheken im Bezug auf COVID-19, die auf Homepages
von englisch-sprachigen Bibliotheksverbänden abgebildet wurden. Das ist
wenig instruktiv -- es wurden Informationen und Flyer zu Massnahmen
verbreitet, einige Bibliothekar*innen arbeitet weiter vor Ort, weil sie
ihre Trägereinrichtungen unterstützen mussten -- und zeigt vor allem,
dass während der Krisenzeit ein geschwinder Aktivismus herrschte. (ks)

\begin{center}\rule{0.5\linewidth}{0.5pt}\end{center}

Grassel, Allison (2020). \emph{Programming in Time of Pandemic: The Year
Libraries Went Touchless}. In: Children and Libraries, 18 (2020) 3,
\url{https://doi.org/10.5860/cal.18.3.3}

Eine ganz kurze Übersicht darüber, wie einige Öffentliche Bibliotheken
in den USA in den ersten Monaten der Pandemie reagierten, liefert diese
Sammlung. Mit dabei: Storytime Online oder per Telefon, Herzen im
Fenster, virtuelle Veranstaltungen, Einladungen zu Spaziergängen,
Lieferung von Büchern. Viel Sympathisches, aber wenig über die konkrete
Situation hinaus weisendes. (ks)

\begin{center}\rule{0.5\linewidth}{0.5pt}\end{center}

Garner, Jessica C. ; Logue, Natalie K. (2020). \emph{Navigating the
COVID-10 slipstream: A case study on living and managing access services
during a global pandemic}. In: Journal of Access Services {[}Latest
Articles{]}, \url{https://doi.org/10.1080/15367967.2020.1818570}

Dieser Text gibt einen umfassenden Überblick zur Arbeit der Bibliothek,
genauer der \enquote{Access Services}, an der Georgia Southern
University. Grundsätzlich hat die Bibliothek wohl das erlebt, was andere
Bibliotheken auch erlebt haben: Schnelle Umstellung zur Lieferung von
Medien per Post, hier ergänzt um Laptops, die auch von der Bibliothek
verborgt werden. Umstellung auf Arbeit von Daheim. Probleme mit der
Aufrechterhaltung der Fernleihe. Hervorzuheben ist, dass -- wenn auch
weiter hinten -- auch die Belastungen für das Personal beschrieben
werden.

Die Situation ist am Ende des Text selbstverständlich noch nicht
geklärt. Vorerst hat die Bibliothek wieder eingeschränkt geöffnet, aber
weder die finanzielle Zukunft noch die Frage, ob, wann und wie sie
wieder vollständig öffnet, ist geklärt. Dies müsste im Idealfall ein
weiterer Text, nach der Pandemie, zeigen. (ks)

\begin{center}\rule{0.5\linewidth}{0.5pt}\end{center}

Kachan, Åsa (2020). \emph{Vulnerabilities exposed and the opportunity to
respond: Reflections on public libraries in the time of COVID-19}.
(Information Management Public Lectures) Halifax: School of Information
Management, Dalhousie University, 06.10.2020,
\url{https://www.youtube.com/watch?v=mNQAwgZrM8c}

Kein Text, sondern ein öffentlicher Vortrag der CEO der Halifax Public
Libraries darüber, was diese Bibliotheken während der Pandemie getan und
entschieden haben. Die Übersicht ist umfassend, aber auch gefährlich:
Kachan zählt alle möglichen Aufgaben auf, welche die Bibliotheken hätten
und landet sehr schnell bei der Hilfe für Menschen ohne festen Wohnsitz
oder mit wenig sozialem Kapital. Sie überhäuft die Bibliothek mit
Aufgaben, die alle wichtig sind und doch erschlagen und die tatsächliche
bibliothekarische Praxis zum Dienst an den Menschen überhöhen. Genau für
solche Beschreibungen wurde das Konzept des Vocational Awe
{[}\enquote{Berufliche Ehrfurcht}, vergleiche Ettarh, Fobazi (2018).
Vocational Awe and Librarianship: the lies we tell ourselves. In: In The
Library With The Lead Pipe, 10.01.2018,
\url{http://www.inthelibrarywiththeleadpipe.org/2018/vocational-awe/}{]}
geprägt. Dieses an Selbstüberhöhung grenzende Selbstbild der Bibliothek
als inhärent gut, essentiell und unantastbar sogar zugleich zu
übersteigerten Erwartungen auch an sich selbst, was sich in
idealistischen Abstrichen an den Arbeitsbedingungen und teils auch in
Überbelastungen bis zum Burn-Out niederschlägt.

In diesem Zusammenhang gleichzeitig interessant und gefährlich ist das
Konzept der Bibliotheken als \enquote{second responders}, die nicht --
wie die \enquote{first responders} wie Feuerwehr oder medizinischer
Notdienst -- sofort bei einer Katastrophe helfen, aber unmittelbar
danach. (ks)

\begin{center}\rule{0.5\linewidth}{0.5pt}\end{center}

Roesner, Elke (2020).\emph{Viel Lärm um alles} -- \emph{ZB MED und die
COVID-19-Pandemie: ZB MED unterstützt die Forschung rund um das
Corona-Virus SARS-CoV-2}. In: Information. Wissenschaft \& Praxis 71
(2020) 4: 195--198, \url{https://doi.org/10.1515/iwp-2020-2089}
{[}Paywall{]}

In ihrem Bericht von der Arbeit der ZB MED während der Pandemie schlägt
Elke Roesner einen recht selbst-lobenden Ton an. Die ZB MED sei
\enquote{als nationaler Information Hub für die überregionale
Informationsversorgung zuständig} (Roesner 2020: 195) und hätte deshalb
als \enquote{Zentrale Fachbibliothek so umfangreich wie nur irgend
möglich am Laufen gehalten werden müssen} (Roesner 2020: 195). Sie
beschreibt dann zuerst, wie an der ZB erst in einem Blog valide Quellen
zur Pandemie gesammelt wurden, dann mit dem ZB MED COVID-19 Hub ein
eigenes Angebot, inklusive Repository, dafür geschaffen wurde. Erst
später beschreibt sie, wie kurzfristig für das Personal das Arbeiten im
Homeoffice ermöglicht wurde und postuliert -- ehrlich gesagt, ohne es
argumentativ hergeleitet zu haben --, dass die Krise gezeigt hätte, dass
\enquote{Open Science {[}...{]} der einzig gangbare Weg in der
Wissenschaft {[}ist{]}} (Roesner 2020: 197).

Der Artikel ist vor allem für eine Haltung zu erwähnen, die vielleicht
nach der Krise noch einmal breiter diskutiert werden muss: Die ZB MED --
und viele andere Bibliotheken -- erklärten sich selber zu einer in der
Krise wichtigen Institution, deren Arbeit unbedingt aufrecht erhalten
werden musste. Ob das so war, ob die Nutzer*innen oder das Personal das
überhaupt so sah, ist nicht klar. Darstellungen wie die von Roesner
hinterlassen den Beigeschmack, als wären andere mögliche Positionen
einfach unter der Behauptung begraben worden, als Einrichtung wichtiger
zu sein als viele andere. (ks)

\begin{center}\rule{0.5\linewidth}{0.5pt}\end{center}

Barbian, Jan-Pieter (2020). \emph{Eine existenzielle Krise: Die Folgen
der Corona-Pandemie für die Öffentlichen Bibliotheken am Beispiel der
Stadtbibliothek Duisburg.} In: BuB 72 (2020) 7: 417--421 {[}gedruckt{]}

Barbian schildert die Situation an der Stadtbibliothek Duisburg, deren
Leiter er ist: Wie am Beginn der Krise Veranstaltungen abgesagt werden
mussten, wie die Bibliothek geschlossen, dafür der Zugang zu digitalen
Angeboten ausgeweitet und die digitale Anmeldung in der Bibliothek
ermöglicht wurde. Auch, wie später der Bibliotheksdienst wieder langsam
aufgenommen wurde. Im Gegensatz zu den vielen hier diskutierten Texten
erwähnt Barbian auch die Sorgen um die Sicherheit der Mitarbeitenden,
die sich in der Bibliothek gemacht wurden. (Das ist vor allem im
Vergleich erstaunlich und sagt vielleicht auch etwas über den
Stellenwert aus, das dem Wohlbefinden des Personals in unterschiedlichen
Bibliotheken eingeräumt wird.) Weiterhin postuliert er, nachdem er die
virtuellen Konferenzen des Jahres diskutiert hat, dass \enquote{nach wie
vor der Grundsatz {[}gelte{]}, dessen Bestätigung wir in den vergangenen
Monaten schmerzhaft erleben mussten: Wir alle sind und bleiben soziale
Wesen.} (Barbian 2020: 421) Nachvollziehbar ist das allerdings aus
seinen vorhergehenden, eher praxisorientierten Ausführungen nicht.

Im einführenden Absatz des Textes stellt Barbian eine Reihe von Fragen:
\enquote{Wie sind die Öffentlichen Bibliotheken mit dieser völlig
neuartigen Herausforderung im Hinblick auf ihre Kunden umgegangen? Wie
haben sich Mitarbeiter auf die grundlegende veränderten
Rahmenbedingungen {[}...{]} eingestellt? Welche Konsequenzen hat
{[}...{]} die Verlagerung der Angebote auf digitale Medien für den
zukünftigen Bestandsaufbau? Was wird aus der in den vergangenen zehn
Jahren vorangetriebenen Konzeption der Bibliothek als \enquote{Dritter
Ort} {[}...{]}? Wie verläuft in Zukunft der Erfahrungs- und
Meinungsaustausch im Bibliothekswesen {[}...{]} : alles nur noch
online?} (Barbian 2020: 417) Diese Fragen beantwortet der Text nicht,
aber es sind Fragen, die es lohnen würde, sie weiter zu diskutieren.
(ks)

\begin{center}\rule{0.5\linewidth}{0.5pt}\end{center}

Ceynowa, Klaus (2020). \emph{Lessons from Lockdown: Was
wissenschaftliche Bibliotheken aus der Corona-Krise lernen könn(t)en}.
In: ZfBB 67 (2020) 3-4: 150--154.
\url{https://doi.org/10.3196/1864295020673428} {[}gedruckt, Paywall{]}

Beobachtungen in der Bayerischen Staatsbibliothek während der
COVID-19-Pandemie verarbeitete deren Direktor zu einer Polemik über die
Zukunftsvorstellungen Wissenschaftlicher Bibliotheken. Er konstatiert,
dass die Vorstellungen einiger Bibliotheken -- konkret greift er die
Sächsische Staats- und Landesbibliothek Dresden auf -- darauf hin zielen
würden, den Bestand in den Hintergrund zu rücken und sich eher als
\enquote{Dienstleister {[}...{]} rund um den Forschungskreislauf und des
vielfältigen Vermittlers auf dem Feld der Wissenschaftskommunikation}
verorten zu wollen. Demgegenüber hätte die Krise gezeigt, dass die
Nutzenden vor allem Interesse an Beständen und an Lesesälen hätten.
Gleich mit Beginn der Krise sei die Nachfrage nach den Beständen --
sowohl gedruckten als auch online angebotenen -- gestiegen. So schnell
als möglich wurde nach Wegen gesucht, Lesesäle wieder zu öffnen (mit
Hygienekonzepten). Nach anderen Angeboten, also dem, was sonst als
\enquote{zukunftsträchtig} gelten würde, sei praktisch nicht gefragt
worden. \enquote{Es zeigt sich:} -- so das Fazit -- \enquote{Bestände
und Sammlungen, seien sie analog oder digital, sind nach wie vor das
Alleinstellungsmerkmal der wissenschaftlichen Bibliothek.}(Ceynowa
2020:153). (ks)

\hypertarget{neue-angebote-in-uxf6ffentliche-bibliotheken}{%
\subsection{2.2 Neue Angebote in Öffentliche
Bibliotheken}\label{neue-angebote-in-uxf6ffentliche-bibliotheken}}

Nguyen, Linh Cuong (2020). \emph{The Impact of Humanoid Robots on
Australian Public Libraries}. In: Journal of the Australian Library and
Information Association, 69(2), 130--148.
\url{https://doi.org/10.1080/24750158.2020.1729515} {[}Paywall{]}

Eine ganze Reihe von Öffentlichen Bibliotheken schaffte in den letzten
Jahren humanoide Roboter an oder plante dies zumindest. Texte und
Informationen, die dazu im DACH-Raum publiziert wurden, waren eher etwas
zurückhaltend mit konkreten Erfahrungen. Die Studie von Nguyen ist da
konkreter: Es wurden Bibliothekar*innen in australischen Public
Libraries interviewt, welche solche Roboter einsetzen, und
Veranstaltungen mit diesen Robotern beobachtet und diese Daten dann
zusammengefasst. Stellenweise liest sich der Text etwas sehr einseitig
positiv. Auf mögliche Kritik wird zum Beispiel kaum eingegangen, Grenzen
der Arbeit mit diesen Robotern tauchen eher indirekt auf. Er stellt eher
die Meinungen und Verhaltensweisen derer da, welche von diesen Robotern
überzeugt sind.

Das einschränkend vermerkt benennt Nguyen vier Themen, für die die
Roboter eingesetzt werden: (1) \enquote{Community builder}, also das
Zusammenbringen von Menschen, (2) \enquote{Teacher}, das Anstossen von
Interesse zum Lernen, (3) \enquote{Aide}, also zur Unterstützung bei
einfachen Aufgaben, (4) \enquote{Challenger}, als Objekt, das
Herausforderungen stellt, beispielsweise motiviert, neues Wissen (über
Roboter und Programmieren) zu erwerben.

Grundsätzlich werden die Roboter als Unterstützung anderer Aufgaben der
Bibliotheken eingesetzt und motivieren vor allem, indem sie Unterhaltung
bieten. Zudem unterstreicht auch Nguyen, dass der Einsatz dieser Roboter
einen relevanten Arbeitsaufwand bedeutet. (ks)

\begin{center}\rule{0.5\linewidth}{0.5pt}\end{center}

Copeland, Andrea ; Yoon, Ayoung ; Zhang, Sheng (2020). \emph{Data Reuse
Practices and Expectations for Data Resources and Services among Public
Library Users}. In: Public Library Quarterly 2020 {[}Latest Articles{]}
\url{https://doi.org/10.1080/01616846.2020.1773749} {[}Paywall{]}

Dieser Artikel basiert auf einer Umfrage, die unter Nutzer*innen
Öffentlicher Bibliotheken in Indiana durchgeführt wurde. Eher weniger
Personen wurden befragt, der Kontext von Umfrage und Forschungsfrage ist
eindeutig die USA. Die Ergebnisse lassen sich nicht übertragen. Aber die
Annahmen, die hinter der Studie stehen, können Anregung für eine Debatte
sein: Sollen Öffentliche Bibliotheken aktiv Daten anbieten? Sollten sie
dabei beraten, diese zu finden, zu analysieren und aufzubereiten?
Sollten sie diese Aufgabe übernehmen und die notwendigen Skills dazu zu
vermitteln?

Die Ergebnisse in dieser Studie deuten darauf hin, dass zumindest in den
USA Daten -- nicht reine Texte, sondern zum Beispiel demographische
Daten, empirische Daten über Schulen oder sozio-ökonomische Daten --
weit häufiger von Nutzer*innen gesucht und verarbeitet werden, als
erwartet wurde. Dabei schätzten die Befragten, dass sie selber leicht
bis sehr überdurchschnittliche Kenntnisse darüber haben, wie Daten
analysiert und aufbereitet werden. Gewünscht wurde vor allem eine
Unterstützung bei der Suche von Daten. Gleichwohl schliessen die
Autor*innen, dass Öffentliche Bibliotheken sich stärker dabei engagieren
sollten, Nutzer*innen bei der Nutzung von Daten zu unterstützen und dies
auch Teil ihrer Weiterbildungs- und Veranstaltungsangebote werden zu
lassen.

Wie gesagt lassen sich die Ergebnisse nicht einfach in den DACH-Raum
übersetzen. Relevant sind die Fragen trotzdem, insbesondere auch die
Hinweise, dass viele Nutzer*innen sich selber schon als Personen
ansehen, die ausreichende Data Literacy haben, um andere Personen darin
zu unterrichten (und deshalb nicht von der Bibliothek erwarten, dass
diese dies auch anbietet) und dass Öffentliche Bibliotheken, wenn sie
diese Aufgaben annehmen wollen, dies pro-aktiv angehen müssen, da
hierfür tatsächlich weitergehende Kompetenzen notwendig sind, die nicht
bei allen Bibliothekar*innen vorausgesetzt werden können. (ks)

\begin{center}\rule{0.5\linewidth}{0.5pt}\end{center}

Lenstra, Noah ; Campana, Kathleen (2020). \emph{Spending Time in Nature:
How Do Public Libraries Increase Access?}. In: Public Library Quarterly
{[}Latest Articles{]}
\url{https://doi.org/10.1080/01616846.2020.1805996}

Öffentliche Bibliotheken im DACH-Raum organisieren seit einigen Jahren
kontinuierlich neue Angebote und Veranstaltungsreihen, welche die
Aufgaben von Bibliotheken erweitern sollen. Stichworte wir Dritter Ort,
Makerspace oder Bibliothekscafé sind in diesem Zusammenhang verbreitet.
Dies scheint für viele Bibliotheken schon eine Veränderung zu sein.

Mit Blick auf die Literatur gerade aus dem US-amerikanischen, aber auch
australischen oder kanadischen Bibliothekswesen scheinen diese
Entwicklungen aber langsam und wenig einfallsreich. Kolleg*innen in
diesen Ländern besetzen immer wieder Themen (und zeigen auch oft, dass
die Themen in anderen Bibliotheken schon Verbreitung gefunden haben,
also keine reinen Ideen oder Hypes sind), die immer weiter ausgreifen.

Diese Text hier argumentiert zum Beispiel, dass Bibliotheken einen
Zugang zur Natur bieten sollten. Damit sind von der Bibliothek
organisierte Spaziergänge, gemeinsames Gärtnern oder auch Seed Libraries
gemeint. Im Text selber wird auch über eine Umfrage berichtet, die
zeigt, dass solche Angebote in US-amerikanischen Öffentlichen
Bibliotheken tatsächlich breit existieren. (Wobei \enquote{StoryWalks},
ein organisierter Spaziergang, bei dem durch eine*n Bibliothekar*in
gleichzeitig auf die Landschaft und ein Buch eingegangen wird, die
verbreitetste Form darstellen.) Erstaunlich ist dabei nicht das Thema
selber, sondern dass sich solche Themensetzungen immer wieder neu
finden. Sie vermitteln zumindest den Eindruck einer, im Vergleich zum
DACH-Raum, viel grösseren Dynamik. (ks)

\begin{center}\rule{0.5\linewidth}{0.5pt}\end{center}

Kelly, Laura ; Bolanos, Cinthya (2020). \emph{From Outreach to
Translanguaging: Developing a Bilingual Storytime}. In: Children and
Libraries, 18 (2020) 3, \url{https://doi.org/10.5860/cal.18.3.28}

Die beiden Autorinnen berichten darüber, wie sie eine monatliche
Spanisch-englische \enquote{Storytime} (eine Veranstaltung für Kinder,
in denen gemeinsam gelesen, gesungen und gesprochen wird) in der Public
Library in Memphis einrichteten und -- auf der Basis von systematisch
gesammelten Rückmeldungen -- veränderten. Die Ergebnisse lassen sich
vielleicht nicht vollständig in andere Sprachen und Kulturen übertragen,
bieten aber eine Anregung für einen Prozess der ständigen Verbesserung.

Kelly und Bolanos halten als Lessons Learned fest: (1) Eher langsamer
vorgehen und weniger in einer Veranstaltung machen, dafür aber mehr Zeit
haben, dass sich alle mit dem, was gelesen, gesungen und so weiter wird,
beschäftigen, (2) regelmässig zwischen Englisch und Spanisch wechseln,
auch manchmal Dinge zweimal sagen, lesen und so weiter, in beiden
Sprachen, weil Bilingualismus in ihrem Fall heisst, beide Sprachen auf
einmal zu benutzen, nicht voneinander getrennt, (3) Outreach
beziehungsweise Werbung da machen, wo die Leute auch sind, in diesem
Fall auf Facebook und über die Schulen, nicht -- wie am Anfang
angenommen -- in Spanisch sprachigen Läden oder Restaurants. (ks)

\begin{center}\rule{0.5\linewidth}{0.5pt}\end{center}

\emph{Bibliotheque(s): Revue de l'Association des Bibliothécaires de
France}. (2020) 100/101 (April 2020) {[}gedruckt{]}

Die Zeitschrift des Verbandes der französischen Bibliothekar*innen
publizierte Anfang diesen Jahres ihre 100ste (und 101ste) Ausgabe. Die
runde Nummer wurde genutzt, um \enquote{mehr als 100} (genauer 178)
Vorschläge dazu zu versammeln, was insbesondere Öffentliche Bibliotheken
ändern können. Die Sammlung ist eklektisch, die Vorschläge werden je in
Bild und sehr kurzem Text vorgestellt. Es geht darum,
Bibliotheksangebote zu verändern, mehr aus dem Raum der Bibliothek
herauszugehen und Angebote an anderen Orten (Institutionen, auf der
Strasse, mobil) anzubieten, aber auch die Bibliothek ökologisch
umzubauen und von Angebot und Personal her diverser zu machen (mit neuem
Personal, der positiven Sichtbarmachung von Unterschieden). Im Gegensatz
vielleicht zum Selbstbild vieler Bibliotheken zeigt sich hier, dass
Bibliotheken (in Frankreich) ständig auch darüber nachdenken sich zu
verändern. Selbstverständlich ist das Heft eine Möglichkeit für andere
Bibliotheken, sich (wieder einmal) \enquote{Anregungen zu holen}.
Interessant ist aber eher, wie sehr diese Sammlung eine eigentlich sehr
aktive Bibliothekswelt zeigt. (ks)

\hypertarget{uxf6ffentliche-bibliotheken-andere-themen}{%
\subsection{2.3 Öffentliche Bibliotheken, andere
Themen}\label{uxf6ffentliche-bibliotheken-andere-themen}}

Spencer-Bennett, Kate (2020). \emph{Libraries in women's lives: everyday
rhythms and public time}. In: Educational Review {[}Latest Articles{]},
\url{https://doi.org/10.1080/00131911.2020.1803796}

Wissen Bibliotheken, warum Menschen sie nutzen? Sicherlich entwickeln
sie aktuell viele Vorstellungen davon, warum Menschen sie in Zukunft,
wenn neue Angebote eingeführt sind, nutzen werden. Aber eigentlich
wissen sie wenig darüber, wie sie im Alltag von Menschen verankert sind.
Spencer-Bennett postuliert in dieser Studie, in welcher sie Frauen in
Birmingham, UK, befragte, dass für diese die Bibliothek und der
Bibliotheksbesuch Teil der alltäglichen \enquote{Rhythmen} sind. Sie
greift dabei auf Henri Lefebvre zurück, der die Nutzung von Städten mit
dem Konzept der Analyse von Rhythmen untersuchte und dabei die Bedeutung
der alltäglichen Handlungen und Strukturen hervorhob. Das funktioniert
sehr gut: Spencer-Bennett kann zeigen, dass die Bedeutung der Bibliothek
zumindest für die Frauen, die sie befragte -- und die zum Teil gerade
nicht aus der Mehrheitsgesellschaft oder dem Mittelstand stammen --
darin liegt, dass diese Teil ihres Alltags ist. Es geht ihnen vor allem
darum, sie regelmässig aufzusuchen, entweder als Teil der sinngebenden
Struktur oder als Zeit und Ort, der sie von anderen Aufgaben für eine
Zeit befreit. Die konkreten Angebote treten dabei zurück, der Besuch
selber ist wichtig. Das Ergebnis hat praktische Relevanz, weil es einen
Hinweis gibt, worauf zu achten wäre, wenn man Bibliotheken verändern
will: Weniger auf die vorzeigbaren neuen Angebote und mehr auf die
Frage, wie die Bibliothek sich in den Alltag der Menschen einfügt
beziehungsweise einfügen könnte. (ks)

\begin{center}\rule{0.5\linewidth}{0.5pt}\end{center}

Kimmel, Sue ; Lancaster, Krystal (2020). \emph{Where are the Books about
Trains? A Case Study Exploring. Reorganization of the Children's Section
in a Small Public Library}. In: New Review of Children's Literature and
Librarianship, 25(1--2), 20--32.
\url{https://doi.org/10.1080/13614541.2020.1774267} {[}Paywall{]}

Dieser Text ist zu empfehlen als Beispiel dafür, wie unterschiedlich
Bibliothekssysteme und deren Entwicklung sein können und wie wenig das
in den einzelnen Ländern bedacht wird. Inhaltlich geht es darum, wie in
einer Kinderbibliothek in den USA der Bestand nach Interessenskreisen
aufgestellt wurde (wie man im DACH-Raum sagen würde). Im Artikel wird
beschrieben, wie dabei vorgegangen wurde und mit dem Begriff des
\enquote{Boundary Object} sogar versucht, dies theoretisch zu
reflektieren.

Interessant ist dabei aber vor allem, dass das als neu besprochen wird.
Obgleich der Text auch andere Beispiele aus den USA anführt, in denen
die verbreitete Dewey Decimal Classification für andere
Aufstellungssystematiken und -systeme verlassen wurde, scheint es für
diese Bibliothek eine neue Erfahrung gewesen zu sein festzustellen, dass
Nutzer*innen -- hier vor allem Kinder und Jugendliche -- sich mit dieser
Systematik nicht immer zurecht finden und dass es andere Möglichkeiten
der Aufstellung gäbe. Dies ist für Öffentliche Bibliotheken im DACH-Raum
keine irgendwie neue Feststellung, vielmehr ist die Aufstellung des
Bestandes nach Interessenskreisen so normal, dass sie kaum noch
thematisiert wird. (ks)

\begin{center}\rule{0.5\linewidth}{0.5pt}\end{center}

Scott, Dani ; Saunders, Laura (2020). \emph{Neutrality in public
libraries: How are we defining one of our core values?}. In: Journal of
Library and Information Science (Online First),
\url{https://doi.org/10.1177/0961000620935501} {[}Paywall{]}

In diesem Text wird über eine Umfrage dazu berichtet, was
Bibliothekar*innen in US-amerika\-nischen Bibliotheken unter
\enquote{Neutralität} verstehen. Konkrete Fragestellung und Ergebnisse
sind sehr US-amerikanisch, beispielsweise scheint schon bei den Fragen
die Vorstellung durch, dass es zu Themen praktisch immer eine
\enquote{liberale} und eine \enquote{konservative} Meinung gäbe, was
sich ja auch im Parteiensystem der USA, aber nicht unbedingt dem in
anderen Staaten, widerspiegelt. Zudem ist die Argumentation des Textes
an Dokumenten der ALA orientiert, die nur für die USA Geltung haben.

Interessant ist aber, dass die Studie davon ausgeht -- und auch zeigt,
dass das stimmt --, dass \enquote{Neutralität} als Wert für Bibliotheken
zwar stark diskutiert wird, aber das darunter auch in der
bibliothekarischen Diskussion unterschiedliche Sachen verstanden werden.
Die Ergebnisse der Umfrage, gerade die offenen Antworten, zeigen, dass
unter Neutralität zumeist verstanden wird, \enquote{objektiv dabei zu
sein, Informationen zu verbreiteten}, dass aber andere Verständnisse
ebenso existieren und wohl die bibliothekarische Arbeit prägen.
Gleichzeitig zeigen sie, dass von einer ganzen Anzahl von Antwortenden
Neutralität als Mythos angesehen wird, der vor allem den Effekt hat, den
Status Quo zu erhalten und der denen nützt, die gesellschaftlich eh
schon ihre Sicht durchsetzen können.

Die Autorinnen schliessen, dass eine Klärung, was überhaupt unter
Neutralität verstanden wird, notwendig wäre, um die Diskussionen im
Bibliothekswesen weiterzuführen. (ks)

\begin{center}\rule{0.5\linewidth}{0.5pt}\end{center}

Pontis, Devendra Dilip ; Winberry, Joseph ; Finn, Bonnie ; Hunt,
Courtney (2020). \emph{What is innovative to public libraries in the
United States? A perspective of library administrators for classifying
innovations.} In: Journal of Librarianship and Information Science,
52(3), 792--805. \url{https://doi.org/10.1177/0961000619871991}
{[}Paywall{]}, \url{http://hdl.handle.net/1811/91766} {[}OA-Version{]}

Ist es sinnvoll und notwendig, dass, was in Öffentlichen Bibliotheken
\enquote{Innovation} genannt wird, zu klassifizieren? Die Autor*innen
dieser Studie tun dies auf der Basis der vom Urban Libraries Council's
vergebenen Preise für innovative Projekte US-amerikanischer
Bibliotheken. Die Ergebnisse -- eine Kategorisierung in Program,
Process, Partnership und Technology -- sind vielleicht nicht so spannend
wie die Grundfrage: Was meinen Bibliotheken überhaupt, wenn sie
\enquote{Innovation} sagen?

Der Artikel diskutiert, dass Bibliotheken an sich immer wieder die
Bedeutung von Innovation diskutieren, dass aber wenig darüber
nachgedacht wird, was genau das heisst. Bislang hätten Überlegungen
dieser Art vor allem darin bestanden, Kategorien von Innovation aus
anderen Bereichen für das Bibliothekswesen zu adaptieren, was allerdings
den Eindruck hinterliess, dass dies wenig Erkenntnisfortschritt gebracht
hat. Die Studie soll auf der Basis tatsächlicher Projekte aus dem
Bibliothekswesen eine neue, passendere Kategorisierung liefern. Dazu
wurden die genannten Daten -- also ausgezeichnete Projekte -- analysiert
und codiert. (Dies könnte man im DACH-Raum auch mit den bekannten
Preisen für innovative Projekte im Bibliotheksbereich reproduzieren.)

Aber dann, in der Diskussion dieser Ergebnisse, scheint nicht mehr klar
zu sein, was diese neue Kategorisierung Bibliotheken genau bringen
könnte: Weiss man so mehr, was Innovationen sind, wie sie durchgeführt
werden, welche Sinn und Effekt die haben? Es wird behauptet, dass
Bibliotheken den Artikel nutzen können, indem sie in diesem zitierte
Beispiele anschauen -- aber das könnten sie auch so mit den Preisträgern
machen, ohne Kategorisierung. Sicherlich: Eine Studie dieser Art
erarbeitet neues Wissen und muss nicht sofort klären, wozu dieses Wissen
später genutzt wird. Aber dennoch hinterlässt es einen unfertigen
Eindruck, so, als wäre das Thema \enquote{Innovation in Bibliotheken}
gar nicht zu fassen. (ks)

\begin{center}\rule{0.5\linewidth}{0.5pt}\end{center}

Lawrence, E.E. (2020). \emph{On the problem of oppressive tastes in the
public library.} In. Journal of Documentation, 76 (2020) 5: 1091--1107,
\url{https://doi.org/10.1108/JD-01-2020-0002} {[}Paywall{]}

Soll bei der Beratung der Nutzer*innen immer deren Leseinteresse im
Vordergrund stehen oder haben Bibliothekar*innen weitere Prinzipien zu
beachten? Dieses Essay von E.E. Lawrence geht direkt die mögliche
Konsequenz einer bibliothekarischen Praxis an, die vielleicht im
DACH-Raum nicht so intensiv diskutiert und reflektiert wird wie in den
USA, aber doch praktiziert wird: Die Empfehlung von Literatur durch die
Bibliotheken.

Lawrence referiert sehr richtig, dass Bibliotheken einst den Anspruch
hatten, die Nutzer*innen dabei zu unterstützen, einen literarischen
Geschmack zu entwickeln und sich \enquote{hinauf zu lesen}, aber dass
dieser Anspruch aufgegeben wurde. Die Interessen der Nutzenden stehen
seit Jahrzehnten im Zentrum der Beratungsaktivitäten, das Ziel ist es
heute, das Lesen an sich zu fördern. Dies steht aber unter Umständen,
wie Lawrence auch richtig bemerkt, im Widerspruch zu anderen Zielen, die
Bibliotheken befördern wollen, insbesondere Demokratie und soziale
Gerechtigkeit. Der Autor geht dies mit der Diskussion einer
hypothetischen Nutzerin an, die gerne Romance Novels liesst, in denen
die Handelnden Personen alle Weiss sind. Soll die Bibliothek jetzt den
Anspruch aufgeben, Diversität zu befördern und dieser Nutzerin nur
solche Bücher empfehlen?

Lawrence löst dieses Problem, indem er -- angelehnt an feministische
Kritik -- klar trennt zwischen den Interessen dieser Nutzerin.
Einerseits wäre es vollkommen richtig, ihr Interesse an Romance Novels
zu beachten, aber es wäre falsch \enquote{oppressive tastes}, hier
Rassismus, zu unterstützen. Die Bibliothek soll nicht zum Lesen
\enquote{guter Literatur} erziehen, aber sie kann den eigenen Anspruch,
ein Ort der Demokratie und für alle zu sein, nicht einlösen, indem sie
Vorlieben unterstützt, die gleichzeitig Unterdrückungsverhältnisse
reproduzieren. Der Essay sagt noch nicht, was genau das praktisch
heissen würde. Aber die Diskussion sollte auch im DACH-Raum aufgenommen
werden. Es ist klar, dass nicht beides geht: Den Geschmack der
Nutzer*innen einfach als gegeben annehmen und gleichzeitig eine
offenere, diversere Gesellschaft repräsentieren zu wollen. (ks)

\begin{center}\rule{0.5\linewidth}{0.5pt}\end{center}

Cole, Natalie ; Stenström, Cheryl (2020). \emph{The Value of
California's Public Libraries}. In: Public Library Quarterly {[}Latest
Articles{]}, \url{https://doi.org/10.1080/01616846.2020.1816054}

Der Grund, diesen Artikel zu besprechen, sind nicht die konkreten
Ergebnisse, sondern die Fragestellung. Cole und Stenström beschreiben in
ihm einen Ansatz, um den \enquote{Wert} von Öffentlichen Bibliotheken zu
messen beziehungsweise nachzuweisen. Sie tun das, indem sie (a) zuerst
ökonomische und soziale Werte trennen, dann (b) auf der Basis von
Literatur, Daten und Interviews einen längeren Abschnitt dazu
formulieren, wo die Bibliotheken in Kalifornien Einfluss hätten und dann
(c) jede dieser Aussagen noch einmal argumentativ ausbreiten.

Der Eindruck, den dieser Text hinterlässt, ist, dass man das alles schon
mehrfach gelesen hat. Mit leicht anderen Foki und Daten, aber
grundsätzlich ähnlich. So oft schon wurde versucht, den Wert von
Bibliotheken nachzuweisen. Noch öfter hört man von Projekten, in denen
das angestrebt wurde, die dann aber einschlafen, abgebrochen werden,
keine Erwähnung mehr finden. Dieses Projekt sticht hervor, weil es
beendet wurde. Aber die Frage, die es aufwirft, ist: Warum? Warum wird
immer wieder nach dem \enquote{Wert der Bibliotheken} gefragt? Reichen
die bisherigen Nachweise nicht? Wem reichen sie nicht? Sind sie nicht
überzeugend? Oder wird so versucht, ein (angenommenes) Problem zu lösen,
das vielleicht gar nicht existiert? Oder eines, das nicht so gelöst
werden kann?

Der Text vermittelt den Eindruck, dass es notwendig wäre, endlich einmal
weiter zu gehen: Bibliotheken haben unterschiedliche positive Einflüsse.
Das ist jetzt oft genug gezeigt und argumentiert worden. Was jetzt? (ks)

\hypertarget{monographien-und-buchkapitel}{%
\section{3. Monographien und
Buchkapitel}\label{monographien-und-buchkapitel}}

Killick, Selena ; Wilson, Frankie (2019). \emph{Putting Library
Assessment Data to Work}. London: facet publishing, 2019

Der Titel des Buches und auch das Vorwort verspricht zu diskutieren, wie
Bibliotheken all die Daten, die sie über die Nutzung und die Interessen
von Nutzenden schon sammeln, einsetzen können, um sich zu entwickeln.
Das stimmt aber nicht. Vielmehr liefert das Buch, jeweils mit kurzen
Einleitungen und dann \enquote{case studies} genannten Berichten aus
einzelnen Bibliotheken, mit welche Methoden (nationale oder lokale
Umfragen, Interviews und ähnliches) vor allem Bibliotheken aus
Grossbritannien versuchen zu verstehen, welche Interessen ihre Nutzenden
haben. Das eigentliche Thema, nämlich wie die Ergebnisse dieser Umfragen
et cetera in die Arbeit der Bibliotheken umgesetzt werden, bleibt sehr
skizzenhaft. Der Fokus liegt eher darauf, wie die Instrumente (also
Fragebögen und so weiter) konstruiert und wie die Daten nachher
dargestellt werden.

Bestimmt ist das für Bibliotheken, die darüber nachdenken, auch ein
regelmässiges Assessment einzuführen, eine hilfreiche Sammlung. Aber am
Ende hinterlässt das Buch einen schlechten Eindruck: Nicht nur wird vor
allem die Sammlung von Daten diskutiert, nicht ihre Nutzung. Es gibt
auch keine Diskussion der Grenzen einen solchen Vorgehens. Weder werden
die Grenzen der einzelnen Methoden bedacht noch diskutiert, ob die
Vorstellung, dass, wenn man nur richtig fragen würde, die Nutzenden alle
sagen würden, wie die Bibliothek \enquote{perfekt} werden kann,
überhaupt stimmen kann. Es ist eine ganz eingeschränkte Sicht darauf,
wie Bibliotheken funktionieren und wie Bibliotheksnutzung funktioniert
-- eine, die von anderen Autor*innen wohl als neoliberal beschrieben
würde.

Interessant ist aus dem Vorwort zu lernen, dass es zwei Konferenzserien
zum \enquote{Library Assessment} gibt: LibPMC -- International
Conference on Performance Measurement in Libraries (seit 1995, eher in
Grossbritannien, \url{https://libraryperformance.org/}) und Library
Assessment Conference (seit 2006, in den USA,
\url{https://www.libraryassessment.org/}). (ks)

\begin{center}\rule{0.5\linewidth}{0.5pt}\end{center}

Henry, Jo ; Eshleman, Joe ; Moniz, Richard (2018). \emph{The
Dysfunctional Library. Challenges and Solutions to Workplace
Relationships.} Chicago: ALA Editions, 2018

Henry, Jo ; Eshleman, Joe ; Moniz, Richard (2020). \emph{Cultivating
Civility: Practical Ways to Improve a Dysfunctional Library.} Chicago:
ALA Editions, 2020

Die beiden Bücher von Henry, Eshleman und Moniz sind aufeinander
bezogen. Das erste Buch schildert vor allem Probleme und ist dafür
hervorzuheben, dass es zeigt, wie schlecht die Arbeitsbedingungen in
einigen Bibliotheken (in der USA, aber wohl auch darüber hinaus) sind.
\enquote{Dysfunctional} wird hier eine Bibliothek genannt, wenn die
Arbeit das Personal krank macht, die persönlichen Beziehungen im
Personal und zwischen Personal und Management unerträglich schlecht sind
und die normale Arbeit dem Personal Angst macht. Es ist wichtig, dass
dies angesprochen wird, auch um verständlich zu machen, dass dies kein
Problem in einer Bibliothek ist, sondern in viel zu vielen Bibliotheken
vorkommt. Es eröffnet die Möglichkeit, darüber nachdenken, was zu ändern
wäre, damit die Bibliothek nicht dysfunktional ist.

Das zweite Buch möchte Lösungen für diesen Zustand anbieten.
Grundsätzlich wäre das zu begrüssen, aber am Ende verbleiben die
Lösungen dabei, alle Beteiligten zu ermahnen, sich klar zu werden, wer
sie sind und sein wollen, wer die anderen sind und was sie wollen und
sich ansonsten zu bemühen, zivil miteinander umzugehen. Dazu werden
viele, viele einzelne \enquote{Techniken} benannt -- also kleine
Übungen, Strategien und so weiter. Richtig ist wohl, dass diese
unterteilt werden in Techniken für Individuen, Teams, Führungskräfte und
Organisation. Das alles hinterlässt aber einen unzureichenden Eindruck:
So, als wäre alles nur eine Frage des Wohlverhaltens aller, nicht ein
systematischen Problem. Irritierend ist, dass die Autor*innen sehr wohl
mit aktueller, kritischer Literatur zum Bibliothekswesen und der Arbeit
in ihm vertraut sind: \enquote{Vocational awe} wird genauso eingeführt
wie Literatur zur Geschichte der Desegregation von Bibliotheken {[}siehe
weiter unten in dieser Kolumne{]} und andere Themen. Das ganze Buch
hinterlässt den Eindruck, bei allen guten Vorsätzen, sich nicht an
Lösungen zu trauen, die über individuelle Verhaltensänderungen
hinausgehen -- ein wenig, wie man sich den US-amerikanischen
Liberalismus vorstellt: Nicht zum Beispiel die gewerkschaftliche oder
gesellschaftliche Organisation soll eine Veränderung herbeiführen,
sondern immer nur die einzelnen Individuen.

Zu erwähnen ist aber auch, dass beide Bücher beim Verlag der ALA
erschienen sind und dass damit der US-amerikanische Verband auch eine
Position bezüglich der Arbeitskultur in Bibliotheken bezieht: Sie soll
das Personal nicht kaputt machen und muss dafür auch erst einmal
thematisiert werden. (ks)

\begin{center}\rule{0.5\linewidth}{0.5pt}\end{center}

Miles, Malcom (2015). \emph{Limits to Culture: Urban Regeneration vs.
Dissident Art}. London: PlutoPress, 2015

Mörsch, Carmen (2019). \emph{Die Bildung der A\_n\_d\_e\_r\_e\_n durch
Kunst: Eine postkoloniale und feministische historische Kartierung der
Kunstvermittlung} (Studien zur Kunstvermittlung; 2). Wien: Zaglossus,
2019

Diese beiden Bücher handeln gar nicht direkt von Bibliotheken (auch wenn
der britische Public Libraries Act von 1850 jeweils erwähnt wird), aber
sind doch relevant, weil sie den Einsatz von Kunst zur
\enquote{Regenerierung} von Städten (Miles 2015) thematisieren
beziehungsweise den Einsatz von Kunst zur \enquote{Hebung der unteren
Schichten} seit dem 19. Jahrhundert (Mörsch 2019), beide mit einem Fokus
auf Grossbritannien. Beide Themen sind miteinander verwoben -- und es
sind solche, auf die sich Öffentliche Bibliotheken immer wieder
beziehen, wenn sie sich zum Beispiel heute als Orte beschreiben, wo sich
Communities bilden oder solche, die mit verschiedenen Angeboten Menschen
erreichen wollen, \enquote{die sonst nicht in die Bibliothek kommen}.
Beide Bücher zielen nicht auf Antworten, sondern problematisieren eher.
(Miles betont sogar, dass er zu dem Teil der Universität gehört, die
neue Fragen aufwerfen, nicht so dem, der gleich Lösungen bietet.)

Mörsch zeigt, wie die Entwicklung von Kunstmuseen in Grossbritannien und
die dazugehörige Pädagogik letztlich immer wieder der Idee folgte, den
jeweiligen \enquote{Anderen} der britischen Gesellschaft zu erziehen,
\enquote{emporzuheben} zu den Ideen des Mittelstandes. Auch, wie dies
von ernsthaften Bemühungen gekennzeichnet war, nicht von Zynismus, und
wie viele Mittel dafür eingesetzt wurden. Gleichzeitig zeigt sie aber
auch, dass die Erfolge -- gemessen an der Idee, so eine ausgeglichene
Gesellschaft zu schaffen, in der Widersprüche aufgehoben würden, ohne
sie eigentlich anzugehen -- mindestens zweifelhaft sind, aber gleichwohl
im Denken von Kunstmuseen immer wieder neue Gruppen zu \enquote{Anderen}
erklärt wurden, an denen zu arbeiten sei. Der Misserfolg zeitigte also
eher eine Weiterentwicklung, nicht ein Überdenken.

Miles thematisiert Strategien, die eingesetzt werden -- oder wurden, er
äussert immer auch, dass die Zeit langsam vorbei sei --, um Städte und
Teile von Städten mittels Kunst und vor allem Kunstmuseen, die
gleichzeitig partizipativ agieren und Communities bilden sollen, zu
beleben. Einerseits zeigt er, dass diese Strategien eher Fragen als
Lösungen aufwerfen. Insbesondere, dass viele dieser Museen und Projekte
scheitern, ohne dass das Einfluss auf folgende Projekte zu haben scheint
(so wird immer wieder das Guggenheim in Bilbao als Beispiel für
gelungene Wiederbelebung gewählt, während die Stadt weiterhin eine
sinkende Zahl an Einwohner*innen hat). Auch, dass immer wieder
angeführte Beispiele für die Entstehung neuen Communities bei näherer
Betrachtung gerade das nicht getan haben, sondern eher soziale
Unterschiede verstärkt haben (das gilt für Agora genauso wie für Gärten
in viktorianischen Städten). Warum wird das dennoch immer wieder als
Begründung für die Neugründung von Museen oder partizipative Programme
dieser Museen genutzt? Auch hier wäre es leicht, von Zynismus
auszugehen, aber Miles zeigt in einem Kapitel, in welchem er Besuche zu
solchen Museen schildert auch, dass manchmal ihre Architektur -- deren
Grundsätzen moderne Bibliotheksbauten auch folgen -- absurd erscheinen,
aber die Museen selber sich ernsthaft bemühen, die Kunst in ihnen
tatsächlich oft gut ist und das Personal engagiert ist. Das ist nicht
das Problem.

Beide Bücher -- Miles noch mehr als Mörsch, die noch eher historisch
argumentiert -- hinterlassen vor allem Fragen. Fragen zu Diskursen und
Annahmen, die für Öffentliche Bibliotheken praktisch gleich gelten. Vor
allem lernt man, wie wenig eigentlich gut begründet ist an solchen
Diskursen, die Museums- und Bibliotheksarbeit heute prägen, wie viel
davon aber auch Traditionen aufgreift, die unreflektiert reproduziert zu
werden scheinen. Das ist bedenkenswert. (ks)

\begin{center}\rule{0.5\linewidth}{0.5pt}\end{center}

Wulf, Andrea (2012). \emph{Die Jagd auf die Venus: und die Vermessung
des Sonnensystems}. München: Bertelsmann, 2012.

Beim sogenannten Venustransit handelt es sich um eines der seltensten
vorhersagbaren astronomischen Ereignisse. Dabei zieht der Planet Venus
vor der Sonnenscheibe vorbei. Der letzte Venustransit war 2012, der
nächste wird erst im Jahr 2117 erwartet. Für die Astronomen des 18.
Jahrhunderts, die den Transit sowohl 1761 als auch 1769 beobachten
konnten, war dies ein Ereignis von ungemeiner Bedeutung, denn es
erlaubte es erstmals, die Entfernung der Erde von der Sonne zu
bestimmen. Dafür war es jedoch nötig, viele Beobachtungen durchzuführen,
die möglichst weit voneinander entfernt sein mussten. Also brachen aus
ganz Europa Forschungsreisende in die entlegendsten Gebiete der Erde
auf. James Cooks berühmte Reise, auf der er in Tahiti Halt machte und
die Küsten von Neuseeland und Australien erforschte, war zur Beobachtung
des Venustransits 1769 organisiert worden.

Andrea Wulf erzählt in ihrem Buch die Geschichte dieser Forschungsreisen
und die Anstrengungen zur Berechnung der Entfernung zur Sonne. Dabei
erfährt man gleichzeitig sehr viel über eines der ersten europaweiten
Forschungsprojekte, das seine Forschungsdaten über alle Ländergrenzen
hinweg teilte und so zu einem Ergebnis kam. Forscher aus Großbritannien,
Frankreich, Russland, Deutschland, Österreich, Skandinavien und anderen
Ländern -- Katholiken, Jesuiten, Anglikaner und Protestanten --
koordinierten ihre Aktivitäten und sammelten ihre Beobachtungen, um sie
in der Summe auswerten zu können. Und dies, obwohl während des ersten
Transits noch der Siebenjährige Krieg tobte und die Forscher auch nicht
zuletzt durch gegenseitige Rivalität (sowohl persönlicher als auch
nationaler Art) angetrieben wurden. (eb)

\hypertarget{uxfcber-bibliotheken-und-gesellschaft-in-frankreich}{%
\subsection{3.1 Über Bibliotheken (und Gesellschaft) in
Frankreich}\label{uxfcber-bibliotheken-und-gesellschaft-in-frankreich}}

Soulé, Véronique (2019). \emph{De squat en squat, une bibliothèque de
rue}. Montreuil: Éditions Quart Monde, 2019

Bibliothèques de rue -- Bibliotheken der Strasse -- sind in Frankreich
nicht selten. Sie werden vor allem von Freiwilligenorganisationen
angeboten, die Menschen in Armut, Sans Papiers oder ähnlich Entrechtete
unterstützen. Dieses kleine Heft stellt eine dieser Bibliotheken aus
Marseille vor, welche von der katholischen Organisation ATD Quart Monde,
die sich gegen extreme Armut engagiert, angeboten wird. Es ist in deren
Publikationsreihe erschienen, insoweit auch Darstellung der eigenen
Arbeit nach aussen. Verfasst von einer Journalistin liesst es sich als
Reportage.

Grundsätzlich funktionieren bibliothèques de rue so, dass Freiwillige
regelmässig dort, wo Menschen in Armut leben, vorbeikommen und gemeinsam
mit den dortigen Kindern lesen. Dies passiert oft im Freien, auf Decken
(daher der Name), offenbar auch verbunden mit Ritualen wie Anfangs- und
Endliedern. Die vorgestellte Bibliothek ist basiert im lokalen Haus von
ATD Quart Monde und folgt Familien von Roma, die keinen
Aufenthaltsstatus in Frankreich erhalten und deshalb immer wieder -- mit
Unterstützung durch die Zivilgesellschaft -- Häuser besetzen, um eine
Wohnung zu haben. Diese müssen sie ebenso regelmässig verlassen, die
Bibliothek folgt ihnen zu den neuen Wohnungen. (Daher der Titel des
Buches, \enquote{squat} ist ein besetztes Haus.) Die Publikation setzt
das als bekannt voraus und schildert in drei Teilen stattdessen, was bei
den wöchentlichen Veranstaltungen geschieht, stellt eine Zahl von
Freiwilligen vor und präsentiert die Sicht der Eltern, deren Kinder von
dieser Einrichtung profitieren. Alles sehr positiv, was aber bei dieser
Art von Publikation zu erwarten ist.

Grundsätzlich interessant für das Publikum im DACH-Raum dürfte sein,
dass diese Form von Unterstützung seit mindestens den 1980er Jahren in
Frankreich etabliert ist -- auch in unterschiedlichen Formen,
Zusammenhängen und Organisationsformen der Zivilgesellschaft. Diese Buch
stellt nur eine dieser Bibliotheken vor. (ks)

\begin{center}\rule{0.5\linewidth}{0.5pt}\end{center}

Rose, José (2020). \emph{Des bibliothèques pour Marseille: en finir avec
l'indolence}. Marseille: Éditions Gaussen, 2020

Das Erstaunlichste an diesem Buch ist wohl, dass es überhaupt existiert.
Der Autor ist weder Bibliothekar noch Journalist, sondern emeritierter
Soziologie-Professor der Universität in Marseille und Mitglied der
Vereinigung der Nutzer*innen der Öffentlichen Bibliotheken. Letztere
versteht sich nicht unbedingt als Freundeskreis, sondern offenbar eher
als Pressure-Group, welche die Mitbestimmung der Nutzer*innen an der
Bibliotheksentwicklung sicherstellen will. Eine solche Gruppe gibt es im
DACH-Raum wohl nicht. (Wohl deshalb ist es auch in einem Verlag, der vor
allem lokale Literatur aus Marseille und Umgebung verlegt, publiziert.)

Die Situation der Bibliotheken in Marseille wird hier als krisenhaft
bezeichnet. Der Autor formuliert zuerst, warum Bibliotheken wichtig
seien sowie dass und wie sie sich permanent verändern würden.
Anschliessend stellt er die Situation in Marseille dar und im letzten
Kapitel nennt er Entwicklungen, welche seiner Meinung nach notwendig
wären. Dabei klingt er in grossen Teilen wie ein Propagandist der
Bibliotheksverbände: Alle Themen, alle Stichworte, die sich über die
Bedeutung und Entwicklung von Bibliotheken in den Papieren der Verbände
finden, finden sich auch bei ihm wieder (das sind ähnliche wie im
DACH-Raum, auch wenn die Bedeutung des Lesens und der Leseförderung
sowie die Bibliothek als Ort der gelebten Demokratie in Frankreich noch
eher hervorgehoben wird). Die Probleme in Marseille zeigt er anhand
sinkender und, im Vergleich zu anderen Bibliothekssystemen in
Frankreich, niedriger Nutzungs- und Etatzahlen. Zudem erwähnt er -- und
vielleicht ist das Buch deshalb von ihm geschrieben worden, weil er in
der Position ist, dies offener zu äussern -- interne Probleme: Die
Leitung der Bibliotheken würde immer wieder ausgetauscht, die interne
Kommunikation sei schlecht, das Personal würde immer wieder demotiviert.

Dabei erfährt man aber auch, dass \enquote{Krise} immer relativ ist. In
Marseille -- mit rund 860.000 Einwohner*innen -- gibt es neun
Öffentliche Bibliotheken (im Buch noch acht, die neunte wurde im Oktober
2020 eröffnet). Allein so ein Netz kann kaum eine Stadt im DACH-Raum
vorweisen. Daneben gibt es aber weiterhin die Stiftung Office central
des bibliothèques, welche 33 Quartierbibliotheken, 20
Spitalbibliotheken, eine Kinderbibliothek und eine Ludothek (in einem
Spital) betreibt sowie eine Association culturelle d'espaces lectures et
d'écriture en Méditerranée, welche acht Lesesäle unterhält. Letztere und
die Öffentlichen Bibliotheken betreiben auch viele Angebote ausserhalb
der eigenen Räume, vor allem in Zusammenarbeit mit anderen
Einrichtungen. (Die bibliothèque de rue aus Marseille, welche im weiter
oben besprochen Buch vorgestellt wird, ist hier noch nicht mal erwähnt.)
An Bücher oder Raum zum Lesen zu gelangen, scheint in Marseille nicht
schwierig zu sein, auch wenn die Verteilung der Räume über die Stadt
ungleichmässig und das Stadtzentrum bevorzugt ist.

Der Autor des Buches verlangt trotzdem eine Entwicklung der
Bibliotheken. Für ihn ist es vor allem eine Frage von Entscheidungen auf
politischer Ebene, wie sich die Bibliotheken entwickeln können. Er
fordert von der Politik einen klaren Plan zur Entwicklung der
Bibliotheken, eine bessere Zusammenarbeit aller Akteur*innen sowie eine
Beachtung von Entwicklungen im Bibliothekswesen anderer Länder. Bei
letzterem ist er wenig innovativ: Geschaut werden soll nach Skandinavien
und in die USA. Alle Konzepte und Begriffe, die im DACH-Raum aus diesen
Ländern übernommen werden, werden auch bei Ihm aufgegriffen. Das
erstaunliche an dieser Aufzählung ist, dass sie keine Lösung für die
Probleme, die zuvor im Buch aufgezählt werden, darstellen. Der Dritte
Ort, der zum Beispiel auch erwähnt wird, wird die internen Probleme der
Bibliotheken, die der Autor als relevant erachtet, zum Beispiel nicht
tangieren. Das Buch ist engagiert und man ist verwundert, dass sich
jemand von ausserhalb des Bibliothekswesens so sehr für deren
Entwicklung interessiert. Aber herausragend neue Ideen werden in ihm
nicht entwickelt. (ks)

\begin{center}\rule{0.5\linewidth}{0.5pt}\end{center}

Bobis, Laurence (Hrsg.) ; Gouttebaron, Sylvie (Hrsg.) ; Alphant,
Marianne ; Bergounioux, Pierre ; Durif, Eugène ; Farge, Arlette (2018).
\emph{Des écrivains à la bibliothèque de la Sorbonne}. Paris: Éditions
de la Sorbonne, 2018

Bobis, Laurence (pres.) ; Gouttebaron, Sylvie (pres.) ; Cosnay, Marie ;
Galea, Claudine ; Rebotier, Jacques ; Taillandier, Fanny (2019).
\emph{Des écrivains à la bibliothèque de la Sorbonne - 2}. Paris:
Éditions de la Sorbonne, 2019

Die Bibliotheken der Sorbonne haben sich mit dem Literaturhaus in Paris
(La maison des écrivains et de la littérature) zusammengetan und bislang
zwei kleine Bücher herausgegeben, in welchem je vier
Schriftsteller*innen einen Text über ein Buch aus diesen Bibliotheken
publizieren sollen. {[}Anmerkung: Das dritte Buch dieser Reihe ist
unterdessen offenbar auch erschienen.{]} Diese Texte entstehen in einem
längeren Arbeitsprozess, bei dem die Schreibenden und die beiden
Institutionen zusammenarbeiten und bei denen manchmal auch vom
eigentlichen Thema abgeglitten wird. Das klingt sympathisch.
Offensichtlich ist die Reihe darauf ausgelegt, jetzt jährlich zu
erscheinen.

Grundsätzlich lesen sich die Bücher schnell und vergnüglich je an einem
Nachmittag weg. Einige Texte sind sehr spezifisch, da sie sich
tatsächlich auf genau ein Buch beziehen (beispielsweise Arlette Farge
auf eine autobiographische Beschreibung der Erziehung bei den Jesuiten
im 17. Jahrhundert), andere beziehen sich auf das Lesen an sich
(beispielsweise der Text von Marie Cosnay), wieder andere auf die
Sorbonne selber (Pierre Bergounioux) oder die Unmöglichkeit, sich in
diesen Bibliotheken nur ein Buch auszusuchen (Jacques Rebotier).
Teilweise hat dies auch Längen, wenn das Thema nicht interessant
scheint. Gleichzeitig bringen die Schreibenden ihre Kompetenzen ein:
teilweise mit poetischer Sprache, teilweise mit doppeldeutiger oder sehr
überlegter.

Ein wenig hinterlassen die Bücher aber dann schon den Eindruck von
Mitbringseln aus dem Souvenirshop für Literaturinteressierte, die beim
Paris-Besuch nicht dabei waren -- schon solche, die der Rezensent gerne
und dankend entgegennehmen (oder selber mitbringen) würde, aber auch
solche, die wenig längerfristigen Eindruck hinterlassen. (ks)

\begin{center}\rule{0.5\linewidth}{0.5pt}\end{center}

Agié-Carré, Sophie (Hrsg.) (2018). \emph{Seniors en bibliothèque}
(Collection Médiathèmes, 21). Paris: Association des Bibliothécaires de
France, 2018

Weniger beachtet, als es das Thema verdient hätte, hat sich in den
Öffentlichen Bibliotheken in den letzten Jahren die Arbeit für
Senior*innen einigermassen etabliert. Es kommt in der Realität wohl mehr
vor als in der bibliothekarischen Literatur. Insoweit ist dieser
Sammelband aus französischen Bibliotheken, der diese Arbeit auch einmal
darstellt, notwendig. Er zeigt, dass es viele Bibliotheken gibt, die
immer wieder ähnliche -- aber dadurch ja nicht weniger sinnvolle --
Angebote für diese Altersgruppe machen. Im Band werden viele
Einzelbeispiele dargestellt (wobei offenbar darauf geachtet wurde,
Bibliotheken und Bibliothekssysteme mit verschiedenen Voraussetzungen,
und gerade nicht nur aus Grossstädten, einzubeziehen). Umrahmt wird es
von zwei allgemeineren Kapiteln, die in das Thema einführen und die
Erfahrungen zusammenfassen. Kritisch ist zu sehen, dass viele Angebote
defizitorientiert sind: Es wird davon ausgegangen, dass Senior*innen
etwas fehlt (beispielsweise digitale Kompetenzen) und dass die
Bibliotheken dieses fördern müssten. Die Interessen und das vorhandene
Wissen von Senior*innen selber stehen weniger im Fokus. Im Grossen sind
das Buch und die vorgestellten Angebote aus Bibliotheken relativ
konventionell, was aber vielleicht auch zeigt, dass diese nicht
unbedingt neu sein müssen, sondern vor allem gut für die angesprochenen
Senior*innen. (ks)

\hypertarget{bibliotheksgeschichte}{%
\subsection{3.2 Bibliotheksgeschichte}\label{bibliotheksgeschichte}}

Müller, Hans-Peter (2020). \emph{Umerziehung durch rote Bibliotheken:
SED-Bibliothekspolitik 1945/46 bis zum Ende des 1960er Jahre}. Berlin:
Simon Verlag für Bibliothekswissen, 2020

Erklärtermassen soll sich dieses Buch mit der Bibliothekspolitik SBZ und
DDR während eines Zeitraumes von 15 Jahren beschäftigen. Wäre das so,
würde das Buch wohl einen Beitrag zur Bibliotheksgeschichte leisten, da
diese Zeit und dieses Land bislang wenig -- aber auch nicht nie --
beachtet wurden. Aber das tut es nicht. Vielmehr scheint der Autor
gedanklich den Kalten Krieg immer noch nicht verlassen zu haben: Es
scheint ihm eher darum zu gehen, die DDR und deren Bibliothekspolitik zu
diskreditieren als eine kontextualisierte Geschichte zu präsentieren.
Das führt am Ende trotzdem zu einigen Erkenntnissen und liefert auch
eine gewisse Übersicht über einige bibliothekspolitische Prozesse. Aber
letztlich ist es nicht hilfreich, um zu verstehen, was in diesen Jahren
in diesem Land in Bibliotheken oder dessen Bibliothekspolitik passierte.

Schon der Aufbau des Buches zeigt das Ziel: Das erste Drittel wird
genutzt, um den Richtungsstreit im deutschen Volksbibliothekswesen in
den 1910ern bis 1930ern sowie die Bibliothekspolitik im
Nationalsozialismus zu schildern. Sicherlich muss bei
Geschichtsschreibung ein Kontext geliefert werden, aber dieser hier ist
gesucht: Es geht dem Autor darum, eine Kontinuität von der Position
Walter Hoffmanns im Richtungsstreit über den Nationalsozialismus zur DDR
zu zeigen und nicht zum Beispiel darum, die Entwicklungen von 1945 bis
1960 so darzustellen, dass Entwicklungsrichtungen gezeigt würden, die
auch genommen hätten werden können.

Die Hauptthese des Autors ist, dass Hoffmann und die \enquote{neue
Richtung} im Richtungsstreit eine extrem totalitäre Form der
Bibliotheksarbeit etabliert hätten, an die dann im Nationalsozialismus
angeschlossen und die dann auch in der DDR weitergeführt worden wäre.
Eine Sache, auf die er dabei fokussiert, ist die \enquote{empfehlende
Bibliographie} (Auswahllisten, die den Leser*innen vorgelegt wurden, um
ihre Lektüre auf ein Ziel hin zu orientieren), die bei Hoffmann
eingeführt worden sei, um die Leser*innen zu lenken und die dann im
Nationalsozialismus und auch in der DDR weiter benutzt worden wären.
Damit hätten die Bibliotheken nicht einen liberalen Zugang zu Medien
gewährt, sondern Lektüre explizit gesteuert. Aber damit diese These
irgendeinen Wert erhält, hätte sie getestet werden müssen: Ist die
\enquote{empfehlende Bibliographie} tatsächlich etwas, was das
Bibliothekswesen in Diktaturen auszeichnet? (Dieses Test wäre negativ
ausgefallen. Sie wurden beispielsweise in der Arbeit katholischer
Bibliotheken in Deutschland ebenso eingesetzt wie in Bibliotheken in
anderen politischen Systemen.) Auch hätte der Kontext breiter gefasst
werden müssen, insbesondere hätte die Bibliothekspolitik in der BRD
geschildert werden müssen und die anderen Bibliotheksformen, die Anfang
des 20. Jahrhunderts existierten, nicht nur die Volksbibliotheken. Nur
dann wäre sichtbar, ob die Entwicklung in der DDR besonders (und
diktatorisch, wie der Autor unterstellt) war oder nicht. (Dabei würde
die These aber scheitern, weil die Kontinuität, die der Autor für das
Bibliothekswesen in der DDR behauptet, zum Beispiel auch in der BRD
vorhanden war.)

Über weite Strecken nennt der Autor dann bibliothekspolitische
Entscheidungen und referiert bibliothekarische Konferenzen, die in der
SBZ und der DDR stattfanden und die, wie der Autor richtig bemerkt,
selbstverständlich nicht dem Wissensaustausch dienten, sondern Versuche
waren, Anweisungen über die Arbeit in Bibliotheken weiterzugeben. Er
beschreibt sie als Zwang und Versuch, das Bibliothekswesen in der DDR
dem in der Sowjetunion anzupassen. Abschliessend stellt er aber fest,
dass die Entwicklung in der DDR sich viel mehr auf Traditionen des
deutschen Bibliothekswesens bewegte, als das sie von aussen (hier von
der Sowjetunion) beeinflusst worden wären. Das ist dann tatsächlich die
relevante Erkenntnis, die zum Beispiel zu Fragen führen könnte, ob eine
solche Pfadabhängigkeit auch in den Bibliothekswesen anderen Länder zu
finden ist.

Aber durch seinen Fokus, die DDR und deren Bibliothekssystem von
vorneherein als diktatorisch zu beschreiben und in der Kontinuität mit
dem Nationalsozialismus stellen zu wollen (was sich zum Teil bis in die
Terminologie zeigt), schafft es der Autor kaum, tatsächliche
Besonderheiten oder Entwicklungen, die spezifisch für der DDR wären,
aufzuzeigen. Der Fokus auf \enquote{Bibliothekspolitik} führt dazu, dass
vor allem der Ablauf von Regelungen, Gesetzen und Anweisungen
geschildert wird.

Ganz am Anfang erwähnt der Autor, dass 1964 schon ein Buch von Martin
Thilo zum Bibliothekswesen in der SBZ/DDR erschien {[}Martin Thilo
(1964). Das Bibliothekswesen in der sowjetischen Besatzungszone (Bonner
Berichte aus Mittel- und Ostdeutschland). Bonn ; Berlin: Deutscher
Bundes-Verlag, 1964{]}, welches schon einen guten -- wenn auch selber
nicht von Polemik freien -- Einblick in dessen Entwicklung gibt. Das ist
auch so. Letztlich hat der Autor diesem älteren Werk wenig hinzugefügt,
das Buch von Thilo ist weiterhin eine bessere Quelle. (ks)

\begin{center}\rule{0.5\linewidth}{0.5pt}\end{center}

Wiegand, Wayne A. ; Wiegand, Shirley A. (2018). \emph{The Desegregation
of Public Libraries in the Jim Crow South: Civil Rights and Local
Activism}. Baton Rouge: Louisiana State University Press, 2018

Dieses Buch zeigt zum ersten Mal in einem breiten Überblick für alle
Südstaaten in den USA, wie die Öffentlichen Bibliotheken im Zuge der
Bürgerrechtsbewegung Mitte der 1960er Jahre \enquote{integriert} -- also
die Rassentrennung beim Zugang zu bibliothekarischen Diensten aufgehoben
-- wurden. Sichtbar wird dabei, dass Bibliotheken explizit Teil dieser
Geschichte sind, auch wenn zum Beispiel die \enquote{Integration} von
Schulen mehr im Fokus der Geschichtsschreibung steht. Die
Bürgerrechtsbewegung der 1960er hatte die Bibliotheken immer mit im
Blick.

Gleichzeitig zeigt die Übersicht auch, wie unterschiedlich die
Integration lokal gehandhabt wurde, was ebenso nicht nur für
Bibliotheken gilt: Teilweise waren Proteste und Sit-Ins notwendig, um
die Integration zu erzwingen, die dann teilweise gewalttätig beantwortet
wurden. Teilweise fand die Integration \enquote{still} statt, also ohne
Proteste. Es hing immer von den lokalen Strukturen ab, insbesondere
davon, wie stark der Rassismus in Politik, Verwaltung und
Zivilgesellschaft verankert war.

Nach einem einführenden Kapitel zur Bürgerrechtsbewegung und deren
Aktionsformen im Allgemeinen geht das Buch jeden einzelnen südlichen
Bundesstaat durch und präsentiert einzelne Fälle. Auch dabei zeigt sich,
dass die Spannweite innerhalb eines Bundesstaats grösser war als
zwischen ihnen. Oft werden Sit-Ins geschildert, die den Ausschlag zur
Integration gaben: Fast immer Jugendliche, auch schon 13-Jährige,
betraten die \enquote{weisse} Bibliothek und verlangten, dort Mitglied
zu werden, was ihnen verweigert wurde. Dann setzten sie sich an Tische
und begannen zu lesen, bis sie verhaftet wurden. Diese Proteste wurden
dokumentiert und skandalisiert. Über kurz oder lang führten solche
Proteste dazu, dass alle Bibliothekssysteme integriert wurden --
teilweise über absurde Zwischenstufen, beispielsweise wurden oft für
eine \enquote{Probezeit} Tische und Stühle entfernt, um zu schauen, ob
eine Integration möglich wäre. In seiner Ansammlung liest sich das
mitunter ermüdend, weil sich die Argumente, Strategien und Protestformen
wiederholen. Aber selbstverständlich war jeder Protest notwendig, jede
Teilnahme an einem solche Sit-In mutig und ist es somit auch richtig,
diese sichtbar zu machen. Gleichzeitig versucht das Buch, die Geschichte
der Desegregation der Bibliotheken mit der US-amerikanischen Geschichte
zu verbinden: Wann immer jemand, die oder der später wichtige Posten in
Parlament, Bundesgerichtshof oder anderen relevanten Institutionen
eingenommen hat oder aber in der Bürgerrechtsbewegung aktiv war und
gleichzeitig mit diesen Bibliotheken in Zusammenhang gebracht werden
kann -- weil er oder sie an den Protesten beteiligt war oder durch diese
Proteste Zugang zu Bibliotheken erhielt --, wird dies erwähnt. Dies
hinterlässt teilweise den Eindruck, dass nur diese Personen, die später
andere wichtige Positionen einnahmen, relevant gewesen wären. Zum Teil
überschreibt damit der pädagogische Impetus leider die Geschichte der
Personen, die lokal aktiv waren, aber nach dem Erreichen der Integration
Karrieren machten, die nicht so beeindruckend klingen.

Relevant für die heutige Zeit ist das letzte Kapitel, welches sich mit
der Haltung der American Library Association beschäftigt: Es zeigt, dass
die ALA sich praktisch nicht engagierte, zumindest nicht offiziell.
Während sie sich in anderen Bereichen gegen Zensur einsetzte und in
anderen Ländern mittels Bibliotheksprojekten Demokratie förderte,
schwieg sie praktisch zur Segregation in den Südstaaten und nahm
beispielsweise sowohl Verbände auf, die segregiert waren also auch
solche, die integriert waren. Es waren bibliothekarische Publikationen
ausserhalb des Verbandes, die das erst als Skandal thematisierten. Zudem
gab es Bibliothekar*innen, die unter der Hand Hilfe anboten. Aber als
Verband war die ALA tunlichst darauf bedacht, sich nicht zu engagieren.
(ks)

\begin{center}\rule{0.5\linewidth}{0.5pt}\end{center}

Pettegree, Andrew ; der Weduwen, Arthur (2019). \emph{The Bookshop of
the World. Making and Trading Books in the Dutch Golden Age}. New Haven;
London: Yale University Press, 2019

Diese umfassende Geschichte von Druckgewerbe und Buchhandel der
Niederlande im 17. Jahrhundert baut auf zwei Argumenten auf: (1) Der
rasante Aufstieg zur grössten Buchhandelsnation in diesem Jahrhundert
basierte darauf, dass in den Niederlanden der Zwischenhandel -- also das
Importieren und Exportieren von Druckwerken -- etabliert und damit eine
internationale Rolle eingenommen wurde. Dies basierte auf einem grossen
und schnell wachsenden heimischen Markt für Printprodukte, auf dem
schnell das notwendige Kapital für die Etablierung dieses
Zwischenhandels erwirtschaftet werden konnte. (2) Ein Grossteil dieses
heimischen Marktes ist heute nur noch indirekt nachweisbar. Die heute
noch vorhandenen Druckwerke aus den Niederlanden des 17. Jahrhunderts
liefern ein falsches Bild davon, wie dieser Markt tatsächlich aussah.

Das Buch beschreibt umfassend, immer eingebunden in die niederländische
Geschichte selber, wie dieser Markt -- mit Druckereigewerbe, Buchhandel
und angrenzenden Berufen -- wuchs und auch über Krisen hinweg lange eine
gute Position behielt. Es beginnt mit der erfolgreichen Revolte gegen
die spanische Herrschaft (1581) und endet mit dem 17. Jahrhundert
selber. In dieser Zeit hätte sich die Niederlande zu einem Land mit weit
verbreiteter Alphabetisierung entwickelt (aus religiösen Gründen, da der
Pietismus darauf basierte, dass alle Menschen die Bibel auch tatsächlich
lesen sollten; wegen der direkten staatlichen Förderung des auf Handel
und moderne Wirtschaft setzenden Staates; aber auch wegen der rasant
zunehmenden Verschriftlichung von Verwaltung und Herrschaft in den
Gemeinden selber). Zudem hätte der wirtschaftliche Aufschwung zu mehr
frei verfügbaren Kapital geführt, auch bei der normalen Bevölkerung, so
dass überhaupt ein ausreichend grosser Markt entstehen konnte.

Dies hätte dazu geführt, dass sich ein Markt für zahllose Druckwerke
etablieren konnte, die zu einem grossen Teil ephemeren Charakter gehabt
hätten: Zahllose Verlautbarungen des Staates, der Länder, der Gemeinden
und so weiter, erste Zeitungen, politisch-religiöse Broschüren, die
gerade in Krisenzeiten in grosser Zahl geschrieben und verbreitet
wurden, Bücher für die wachsende Zahl an Schulen, religiöse
Kleinliteratur (Liederbücher, Katechismen und so weiter),
Selbsthilfebücher (Einführungen in die Mathematik, das Rechnungswesen,
medizinische Ratgeber und so weiter) sowie billig produzierte
literarische Werke, die sich die allgemeine Bevölkerung leisten konnte.
Zudem etablierte sich schnell ein Second-Hand-Markt durch Auktionen.
Diese ganzen Drucke lieferten, so die Argumentation, die finanzielle und
infrastrukturelle Basis für grössere Projekte -- sowohl Druckprojekte
als auch den erwähnten Zwischenhandel auf europäischer Ebene.
Insbesondere die Druckereien plus diese zulieferenden Gewerbe (zum
Beispiel für Stiche), die für diese ganzen \enquote{kleinen} Aufträge
aufgebaut wurden, konnten dann erst die Meisterwerke der Druckkunst
hervorbringen, welche uns heute noch in zahlreichen Ausgaben vorliegen.

Was beide Autoren immer und immer wieder betonen, ist, dass die Werke
der damaligen Zeit, die uns heute in Bibliotheken vorliegen, nicht die
Werke sind, welche das hauptsächliche Publikationsaufkommen und die
Nutzung von Druckwerken im 17. Jahrhundert in den Niederlanden
ausmachten. Was auf uns überkommen ist, ist, was als sammelwürdig
angesehen wurde. Vieles andere, was viel weiterer verbreitet war, ist
fast vollständig verschwunden. Die Autoren zeigen dies immer wieder
anhand von Werken, von denen uns aus späteren Auflagen noch ein Exemplar
bekannt ist, von vorhergehenden Auflagen aber keines. Sie können auch
immer wieder zum Beispiel aus Anzeigen oder Katalogen Hinweise auf Werke
liefern, die heute gar nicht mehr vorliegen. An sich ist das nicht
überraschend: Die Vorstellung, dass es Einrichtungen bräuchte, welche
die gesamten Publikationen eines Landes sammeln müssten, etablierte sich
bekanntlich erst Ende des 19. Jahrhunderts (und dann auch nie
vollständig). Pettegree und der Weduwen zeigen noch einmal, dass deshalb
eine umfassende Geschichte von Mediennutzung einer Epoche nur auf der
Basis der Werke, die uns aus der Zeit zuvor überkommen sind, nicht
möglich ist. (ks)

\begin{center}\rule{0.5\linewidth}{0.5pt}\end{center}

Kuttner, Sven ; Kempf, Klaus (Hrgs.) (2018). \emph{Buch und Bibliothek
im Wirtschaftswunder: Entwicklungslinien, Kontinuitäten und Brüche in
Deutschland und Italien während der Nachkriegszeit (1949-1965).}
(Beiträge zum Buch- und Bibliothekswesen, 63) Wiesbaden: Harrassowitz
Verlag, 2018

Dieser Band versammelt Beiträge, die auf einer Tagung des Wolfenbütteler
Arbeitskreises für Bibliotheks-, Buch- und Mediengeschichte von
Autor*innen aus Deutschland und Italien gehalten wurden. Es ist eine
sehr bunte Sammlung und zumindest in diesem Band -- vielleicht nicht auf
der Tagung selber -- wurde leider versäumt das Potential, die
Entwicklung in zwei Ländern zu vergleichen, zu nutzen. Die Beiträge
stehen lose nebeneinander, ohne Bezug aufeinander zu nehmen oder auch
nur inhaltlich das Gleiche zu behandeln, so dass auch ein indirekter
Vergleich schwierig ist. Das Themenfeld ist sehr breit, was ebenso zum
Eindruck einer gewissen Beliebigkeit beiträgt. Der Band beginnt mit
einem Artikel (Christof Dipper), welcher diskutiert, ob der Begriff
\enquote{Wirtschaftswunder} überhaupt für beide Staaten anzuwenden ist.
Als Kontext ist das begrüssenswert, aber leider geht der Artikel selber
nicht auf Buch- und Bibliothekwesen ein, gleichzeitig ist es der
einzige, der wirklich einen Vergleich angeht.

Ansonsten sind die Beiträge auf Einzelfragen fokussiert. Für sich selber
sind viele -- aber nicht alle -- interessant, einige sehr tiefgehend
(beispielsweise Annemarie Kaindl und Maximilian Schreiber zum Plan, in
München Staats- und Universitätsbibliothek in einem Gebäude
zusammenzuführen), andere reissen eher oberflächlich längerfristige
Entwicklungen an (beispielsweise Birgit Dankert zum Öffentlichen
Bibliothekswesen in Deutschland, beginnende Ende der 1940er, aber ended
bei Debatten, die auf die 1970er und 1980er vorausdeuten). Zudem
konzentrieren sich, wie der Titel andeutet, die Autor*innen entweder auf
das Bibliothekswesen oder das Buchwesen, beziehungsweise jeweils auf
Teile davon. Der Rezensent fühlt sich inhaltlich nicht in der Lage, die
Beiträge zum Buchwesen zu bewerten. Gleichwohl ist der interessanteste
des Bandes, von Christine Haug zu Leihbuchromanen und
Leihbuchroman-Verlagen, gerade aus diesem Bereich. Haug zeigt, dass auch
der Bereich von \enquote{kommerziellen Leihbüchereien} und dem dort
vertriebenen Lesestoffen einer historisch-kritischen Analyse bedarf und
nicht einfach aus der Geschichte von Buch- und Bibliothekswesen
ausgeschlossen oder gar, wie in den 1950ern noch üblich, als
\enquote{Massenware} abgewertet werden darf.

Für den Bibliotheksbereich hervorzuheben sind die Beiträge von Klaus
Kempf, Jürgen Badendreier und Sven Kuttner, die alle drei jeweils die
Debatten innerhalb der Wissenschaftlichen Bibliotheken in Deutschland in
den späten 1940ern und 1950ern darstellen. Diese seien, so Kempf, kein
Neuanfang gewesen, sondern dem Impetus des \enquote{Wiederaufbaus}
verpflichtet gewesen: Es wurde an die konservative Kulturkritik der
1920er und 1930er Jahre angeschlossen, die einen Unterschied zwischen
\enquote{Geistigem} und \enquote{Materialismus} postulierte. Der
Nationalsozialismus wurde als Ergebnis des Materialismus, der
Massenkultur und der zu wenigen Bildung bezeichnet. Mit naheliegenden
Fragen (beispielsweise der Verstrickung des Bibliothekswesens mit dem
nationalsozialistischen Regime oder der Frage, ob nicht gerade diese
Kulturkritik mit den Nationalsozialismus überhaupt ermöglicht hätte)
wurde sich nicht beschäftigt. Auswirkungen hatte das dahingehend -- so
alle drei Autoren --, dass die Universitätsbibliotheken vor allem Wert
auf die Vollständigkeit des Bestandes legten, Forschung als langsame
Tätigkeit imaginierten und so auch den Zugang zum Bestand gestalteten.
Die Interessen von Studierenden und Forschenden wurden so nicht bedient.
Deshalb -- so wieder alle drei -- wuchs die Bedeutung von Zweig-,
Fakultäts-, Instituts- und andere Spezialbibliotheken, die von den
Lehrstühlen selber getragen wurden und unabhängig von den
Universitätsbibliotheken agierten. Erst die Universitätsneugründungen
sowie der Generationenwechsel im Bibliothekswesen in den 1960ern und
Akteure von ausserhalb des Bibliothekswesens erzwangen in den 1960ern
und 1970ern eine Änderung. Die Beiträge verweisen implizit sehr richtig
darauf, dass zum Beispiel Vorstellungen davon, wie Wissenschaft, Bildung
und Lesen funktioniert, nicht einfach innerbibliothekarische Debatten
sind, sondern konkrete Auswirkung auf die Form und Funktion von
Bibliotheken und deren Nutzung haben.

Der Band enthält zahlreiche Beiträge zum italienischen Bibliotheks- und
Buchwesen. Die Beiträge sind im Italienischen belassen worden und
jeweils durch einen englischen Abstract ergänzt. Letztere sind leider
manchmal schwer zu lesen und hinterlassen den Eindruck, in Grammatik und
Wortschatz sehr vom Italienischen geprägt zu sein. Inhaltlich stellen
diese Beiträge vor allem Übersichten dar, beispielsweise zur Entwicklung
des italienischen Öffentlichen Bibliothekswesen oder des
Bibliotheksverbandes. Dabei halten sich die Autor*innen oft eng an
offizielle Quellen und liefern wenig Quellen- oder andere Kritik. Sie
zeichnen vor allem das Bild einer stetigen, wenn auch steinigen,
Entwicklung hin zum heutigen Zustand. (ks)

\begin{center}\rule{0.5\linewidth}{0.5pt}\end{center}

Wiederkehr, Ruth (2018). \emph{Lesen, schreiben, beten, heilen. Die
Bibliothek des mittelalterlichen Klosters Hermetschwil} (Murensia; 6).
Zürich: Chronos Verlag, 2018

Eine ganze Anzahl Klöster oder Stiftungen, welche die Geschichte
aufgehobener Klöster lebendig erhalten, geben Publikationsreihen zu
dieser Geschichte heraus. Wieder in einer ganzen Anzahl dieser Reihen
werden dann auch Hefte zur jeweiligen Bibliothek oder Büchern aus dieser
veröffentlicht, was folgerichtig ist, schliesslich waren (und sind)
diese Bibliotheken ein wichtiger Ort des Klosterlebens und
repräsentierten Teile der eigenen Identität. {[}In der letzten Nummer
dieser Kolumne wurde schon eine andere dieser Publikationen
besprochen.{]}

Das Kloster Hermetschwil (Kanton Aargau), dessen mittelalterliche und
frühneuzeitliche Bibliothek Thema der Publikation von Ruth Wiederkehr
ist, besteht weiterhin. Die Sammlung wurde nicht, wie in vielen anderen
Fällen, mit der Reformation oder einer staatlich dekredierten Aufhebung
des Klosters beendet. Zudem wurde nicht, wie in anderen Fällen, nur die
Auswahl einer Bibliothek überliefert, aus der ephemere Werke entfernt
wurden. Sie enthält deshalb relativ viele Bücher für den alltäglichen
Gebrauch -- private Predigtensammlungen, Heilbücher und so weiter --,
die in anderen Fällen wohl nach ausgiebiger Benutzung vernichtet wurden.
Alle die Bücher dieser Bibliothek aus dem 12. bis 16. Jahrhundert sind
digitalisiert auch bei e-codicies einzusehen (vergleiche Suche nach
\enquote{Hermetschwil, Benediktinerinnenkloster} in
\enquote{Library/Collection} unter
\url{https://www.e-codices.unifr.ch}).

Wiederkehr geht deswegen in ihrer Publikation auch weniger auf die
Materialität oder Ausgestaltung der Bücher ein, wenngleich das Heft
reich bebildert ist. Vielmehr konzentriert sie sich darauf, sehr
lebendig die Benutzung von Büchern im Kloster darzustellen. Es geht mehr
um ihren Inhalt, vor allem der \enquote{Kleinschriften}, und das
eigentliche Klosterleben, wie es sich aus diesen rekonstruieren lässt.
Das alles basiert auf reichhaltiger Forschungsliteratur, ist aber
durchaus für die breite Öffentlichkeit dargestellt. (ks)

\hypertarget{social-media}{%
\section{4. Social Media}\label{social-media}}

Twitter: \#ShowYourBibOutift,
\url{https://twitter.com/hashtag/ShowYourBibOutfit?f=live}

Ein Trend, bei dem wir uns ehrlich gesagt fragen, ob wir einfach zu alt
dafür geworden sind oder ob wir es einfach \enquote{don't get it}, der
aber offenbar bei einer Anzahl Kolleg*innen beliebt ist, ist, unter dem
Hashtag \#ShowYourBibOutfit Fotos von sich selber zu posten und dabei zu
zeigen, in welcher Kleidung man in der Bibliothek arbeitet. Hier und da
gibt es kulturelle Anspielungen, selbstverständlich viele T-Shirts mit
Verweisen auf das Lesen. Grundsätzlich scheint viel Legeres getragen zu
werden. (ks)

\begin{center}\rule{0.5\linewidth}{0.5pt}\end{center}

Gutknecht, Christian: \emph{Der Schweizer 57 Mio EUR Elsevier Deal},
13.08.2020,
\url{https://wisspub.net/2020/08/13/der-schweizer-57-mio-eur-elsevier-deal/}.

Herb, Ulrich: \emph{Open Access Transformation in Switzerland \&
Germany}, 19.08.2020,
\url{https://scidecode.com/2020/08/19/open-access-transformation-in-switzerland-germany/}.

57 Million Euro, so viel zahlen die Schweizer Hochschulen für den
zunächst auf vier Jahre ausgelegten Vertrag mit Elsevier, der das Lesen
sowie das Open-Access-Publizieren in Elsevier-Zeitschriften regelt.
Gemeinhin werden solchen Konstrukte als
Open-Access-Transformations\-ver\-träge bezeichnet, geht es doch -- so
zumindest das erklärte Ziel der Forschungseinrichtungen samt
Bibliotheken -- um die sukzessive Überführung von Zeitschriften in ein
Open-Access-Geschäftsmo\-dell. Dass der Schweizer Elsevier-Vertrag ein
Open-Access-Transformationsvertrag ist, daran lässt das erste Resümee
von Christian Gutknecht Zweifel aufkommen. Gutknecht legt die wunden
Punkte (niedrige Anzahl de facto Open Access publizierter Artikel und
für diese eine niedrige Quote an \enquote{echter}
Open-Access-kompatibler Lizenzierung mit CC BY, Fortführung von double
dipping, mangelnde Transparenz über Kosten für einzelne Schweizer
Einrichtungen) und schließt nahezu resignativ mit: \enquote{Man hat sich
nun mit dem Geld 30\,\% OA bei Elsevier gekauft. Für die 100\,\% die
swissuniversities bis 2024 erreichen will, muss aber noch sehr viel
passieren.}

Wenige Tage später nahm Ulrich Herb einen Vergleich des Schweizer
Elsevier-Deals mit den deutschen DEAL-Verträgen für Wiley und
SpringerNature vor: Er stellt die Eckdaten dieser drei Verträge vor
sowie Details wie Standardlizenz, berechtigte Artikel und Kosten
gegenüber und arbeitet abschließend Ähnlichkeiten und Unterschiede
heraus. (mv)

\begin{center}\rule{0.5\linewidth}{0.5pt}\end{center}

TU Berlin -- Architekturmuseum: Videoblog \enquote{\emph{Sehstücke}},
\url{https://architekturmuseum.ub.tu-berlin.de/index.php?p=659}

Wie können Museen (oder auch andere auf Ausstellungsformate
ausgerichtete Kultureinrichtungen) in der Pandemie-bedingten Zeit lang
anhaltender Schließung beziehungsweise Nutzungseinschränkung ihre
Bestände zugänglich machen? Digitalisierung, klar, das ist naheliegend,
und vielerorts bereits erfolgt. Doch geht da noch mehr?
\enquote{Sehstücke} heißt die Reihe, in der das Architekturmuseum der TU
Berlin seit Mai 2020 im zweiwöchentlichen Rhythmus besondere Werke aus
dem eigenen Bestand präsentiert: In Videos von circa fünf Minuten werden
Detailaufnahmen der Objekte gezeigt und Details beziehungsweise Kontext
zum Werk geliefert. (mv)

\begin{center}\rule{0.5\linewidth}{0.5pt}\end{center}

Springville Public Library: \emph{LOL * Social Distancing and Cleaning
at Springville Public Library}, YouTube 02.05.2020,
\url{https://youtu.be/HY2PkpIcbQo}

Warum haben Bibliotheken eigentlich wegen Covid-19 geschlossen? Die
Bibliothekar*innen der öffentlichen Bibliothek in Springville haben die
hygienischen Bedenken szenisch umgesetzt. (mv)

\hypertarget{konferenzen-konferenzberichte}{%
\section{5. Konferenzen,
Konferenzberichte}\label{konferenzen-konferenzberichte}}

\emph{2020 Virtual Library Publishing Forum} (Library Publishing
Coalition), 4.--8. Mai 2020,
\url{https://librarypublishing.org/library-publishing-forum/}

Die Library Publishing Coalition ist eine vor allem, aber nicht nur, in
den USA und Kanada basierte Vereinigung von Bibliotheken, welche
gleichzeitig als Verlage agieren. Die Zahl derer, die im Verzeichnis der
Coalition aufgeführt werden
(\url{https://librarypublishing.org/directory-year/directory-2020/} --
ohne vollständig zu sein) ist, angesichts dessen, dass die damit
zusammenhängenden Aufgaben in der bibliothekarischen Literatur kaum
besprochen werden, erstaunlich hoch: Offenbar betätigen sich
Bibliotheken in diesem Bereich.

Die im Mai 2020 aus bekannten Gründen digital durchgeführte Konferenz
ist anhand zahlreicher Videos und Folien nachzuvollziehen, leider nicht
die Diskussionen zu den Vorträgen. Auffällig ist, dass davon wieder
viele erste Ergebnisse oder Überlegungen präsentieren und dass
gleichzeitig weiterreichende Themen aus dem Bereich Open Access (zum
Beispiel APCs) besprochen wurden. Trotzdem ist sie ein Hinweis darauf,
dass auch dieses Thema gemeinsam mehr besprochen werden könnte, als dies
bisher der Fall ist.

Hervorzuheben für die LIBREAS -- weil es uns als Redaktion irgendwie
auch betrifft -- sind das Projekt von Sarah Severson und Jessica Lange
\enquote{Documenting labour in Canadian, independent scholarly journal
publishing} (bei dem jetzt schon klar ist, dass die meisten unabhängigen
wissenschaftlichen Zeitschriften, zumindest in Kanada, vom Engagement
kleiner Redaktionsteams leben) (\url{https://youtu.be/2qtvV8YJEuA}) und
die Diskussion von Robert Browder, Aaron Mccollough, Andrew Lockett und
Lara Speicher \enquote{Editorial Control in Library Publishing: Who Does
What and Why?} (\url{https://youtu.be/HNMh3pJ5s7M}). (ks)

\begin{center}\rule{0.5\linewidth}{0.5pt}\end{center}

Peters, Timothy ; Dickinson, Thad E. (2020). \emph{A History of the
Distance Library Services Conference}. In: Journal of Library \&
Information Services in Distance Learning, 14(2), 96--109.
\url{https://doi.org/10.1080/1533290X.2020.1809600}

Dieser Text ist ein Abschlussbericht für die im Titel genannte
Konferenz, welche fast 40 Jahre lang alle zwei Jahre durchgeführt wurde.
Eingestellt wurde sie aus zwei Gründen: Zum ersten finanziellen, da der
Etat der ausrichtenden Bibliothek (Central Michigan University
Libraries) immer enger geworden ist. Zum anderen stellen die Autoren
fest, dass die Gruppe der \enquote{Distance Learners} (also
Studierender, die nicht direkt auf dem Campus lernen, sondern über
externe Programme) und die Gruppe \enquote{normaler} Studierender immer
weniger zu unterscheiden seien, da diese auch verstärkt Online-Kurse
belegen würden. Nicht zuletzt wären die technischen Lösungen, um
Distance Learners zu unterstützen, heute in jeder (US-amerikanischen)
Bibliothek vorhanden. Diese vorzustellen und ihren Einsatz zu
diskutieren, sei nicht mehr nötig. {[}Gleichwohl wollten zwei andere
Bibliotheken die Konferenz fortführen, mussten sie aber 2020 wegen der
COVID-19-Pandemie absagen.{]}

Was den Text interessant macht, ist, dass er weniger auf den Inhalt der
Konferenzen der letzten Jahre eingeht und viel mehr in den Vordergrund
rückt, wie die Konferenz organisiert wurde, also wie die Konferenzorte
ausgewählt, die Beiträge bewertet und das Programm zusammengestellt
wurden. Einen solchen konzisen Einblick erhält man selten. (ks)

\hypertarget{populuxe4re-medien-zeitungen-radio-tv-etc.}{%
\section{6. Populäre Medien (Zeitungen, Radio, TV
etc.)}\label{populuxe4re-medien-zeitungen-radio-tv-etc.}}

Claus-Jürgen Göpfert: Bibliotheks-Chef Frank Scholze: \emph{„Diese
Pandemie ist ein Digital-Beschleuniger"}. In: Frankfurter Rundschau /
fr.de, 29.04.2020,
\url{https://www.fr.de/kultur/literatur/bibliotheks-chef-frank-scholze-diese-pandemie-digital-beschleuniger-13744460.html}

Im Interview mit der Frankfurter Rundschau sieht der neue
Generaldirektor der Deutschen Nationalbibliothek, Frank Scholze, in der
Coronakrise einen \enquote{Digital-Beschleuniger} auch für sein Haus, da
ein Großteil der Mitarbeiter*innen im Home-Office arbeiten muss und
zugleich digitale Prozesse in vielen Bereichen -- er nennt exemplarisch
die Rolle Corona-relevanter Preprints -- einen neuen Stellenwert
erhalten. Zugleich muss die Deutsche Nationalbibliothek aufgrund ihres
Profils als Präsenz- und Archivbibliothek selbstverständlich dauerhaft
auf Lesesäle setzen. Zugleich zeigt sich auch bei ihr die Anforderung
der Bibliothek als Ort der Begegnung und Raum für direkte Kommunikation.
Entsprechend sieht Frank Scholze auch perspektivisch den Ort der
Bibliothek als deren Wesenskern. (bk)

\begin{center}\rule{0.5\linewidth}{0.5pt}\end{center}

Alison Flood: \emph{German library pays £2.5m for 'friendship book', 400
years after it first tried to buy it.} In: The Guardian / guardian.com,
27.08.2020
\url{https://www.theguardian.com/books/2020/aug/27/german-friendship-book-sold-library-das-grosses-stammbuch-germany-herzog-august-bibliothek}

Die Herzog August Bibliothek Wolfenbüttel erwarb über Vermittlung des
Auktionshauses Sotheby's für € 2,8 Millionen das „Freundschaftsalbum"
(Album Amicorum) von Philipp Hainhofer, das Herzog August der Jüngere
bereits im Jahr 1648 vergeblich für die Bibliothek zu beschaffen
versucht hatte. (bk)

\begin{center}\rule{0.5\linewidth}{0.5pt}\end{center}

Natasha Pulley: \emph{The Midnight Library by Matt Haig review -- a
celebration of life's possibilities.} In: The Guardian / guardian.com,
27.08.2020
\url{https://www.theguardian.com/books/2020/aug/27/the-midnight-library-by-matt-haig-review-a-celebration-of-lifes-possibilities}

Das Motiv der Bibliothek prägt die Fantasy-Novel \enquote{The Midnight
Library} von Matt Haig (Edinburgh: Canongate, 2020) und zwar bis in den
Titel hinein. Das Motto ist \enquote{One Library, infinite Lives} und
auf Seite 2 liest man: \enquote{The library was a little shelter of
civilization.} Die Grundidee umreißt die Rezensentin Natasha Pulley in
ihrer Kurzbesprechung im Guardian: Die Protagonistin Nora ist an einem
Tiefpunkt in ihrem Leben und versucht, sich das Leben zu nehmen.
\enquote{But instead of death, what Nora finds is a library in which
each volume represents a version of her life where she made different
choices.} (bk)

\begin{center}\rule{0.5\linewidth}{0.5pt}\end{center}

o.A.: \emph{\enquote{Design Thinking} zur neuen Bücherei}. In:
Stuttgarter Zeitung, Esslingen. 09.09.2020, S. 20

Für den 23. September 2020 lud die Stadtverwaltung von Esslingen zu
einem Vortrag zum Thema "Design Thinking in Bibliotheken", gehalten von
Julia Bergemann. Hintergrund ist das Ergebnis eines Bürgerentscheids aus
dem Februar 2019, bei dem die Menschen in Esslingen für die Renovierung
der dortigen Bibliothek im Bebenhäuser Pfleghof gestimmt hatten. Der
Prozess wird vom \enquote{Bücherei-Berater Andreas Mittrowann}
begleitet. (bk)

\begin{center}\rule{0.5\linewidth}{0.5pt}\end{center}

Kathrin Götze: \emph{Einbruch in Stadtbibliothek: Diebe stehlen 80
Rollen Toilettenpapier}, In: Hannoversche Allgemeine / haz.de,
30.03.2020,
\url{https://www.haz.de/Umland/Neustadt/Neustadt-Einbruch-in-Stadtbibliothek-Diebe-stehlen-80-Rollen-Klopapier}

Diese Überschrift wird zukünftigen Generationen wie ein falsch datierter
Aprilscherz vorkommen, zumindest wenn sie keine genaue Kenntnis des
Warenmangels in der frühen ersten Corona-Lockdownphase haben. Neben
Toilettenpapier (und in Folge verwandte Produkte wie Küchenrolle und
Taschentücher) betraf dies vor allem lagerungsfähige Grundnahrungsmittel
(Nudeln, Reis, Mehl) und Backzutaten (ein Kiosk in Berlin-Kreuzberg warb
gar \enquote{Masken und Hefe hier erhältlich} mit einem Aufsteller auf
der Straße). Gegen den Diebstahl in der Stadtbibliothek in Hannover
wurde Anzeige erstattet; nicht zu ermitteln ist, ob der/die Dieb*in
gefasst werden konnte. (mv)

\hypertarget{abschlussarbeiten}{%
\section{7. Abschlussarbeiten}\label{abschlussarbeiten}}

Grest, Anett (2020). \emph{Bibliotherapie in Bibliotheken}. (Berliner
Handreichungen zur Bibliotheks- und Informationswissenschaft, 456)
Berlin: Institut für Bibliotheks- und Informationswissenschaft der
Humboldt-Universität zu Berlin, 2020 {[}Zugleich Masterarbeit, 2016{]}.
\url{https://doi.org/10.18452/21618}.

In der Übersicht zum Forschungsstand dieser Arbeit vermerkt die Autorin,
dass die Bibliotherapie -- also der Einsatz von Literatur, Lesen,
Schreiben und dem Sprechen darüber zu therapeutischen Zwecken -- in der
bibliothekarischen Literatur zwar schon behandelt wurde, dies aber eher
selten, und dass gleichzeitig in anderen Bereichen, vor allem der
Literatur zu Kreativtherapien, die Bibliotherapie weit öfter bedacht
wird. Ihre Masterarbeit wollte dies ändern.

Sie beschreibt dazu verschiedene Formen des Einsatz von Bibliotheraphie
und führt sechs Interviews mit Praktiker*innen. Diese agieren vor allem
im therapeutischen Kontext. Am Ende will sie so den Status Quo zu dieser
Therapieform in Deutschland aufzeigen.

Während die Arbeit interessante Aspekte anspricht und einen Überblick zu
verschiedenen Settings liefert, ist sie wenig aussagekräftig zum Thema,
das im Titel angesprochen wird: Bibliotheken kommen nur am Rande vor --
bei der Frage, ob diese ein gutes Setting bieten würden, was zum Teil
bejaht, zum Teil verneint wird, und bei der Frage, wie die in der
Therapie benutzte Literatur ausgewählt wird. Es geht eher um die
eigentliche Therapie, für die die Autorin zeigt, dass sie sinnvoll sein
kann. Worum es in der Arbeit auch nicht geht, ist die eigentlich für
Bibliotheken näherliegende Form von Bibliotherapie, nämlich die durch
die Lesenden selbst-gesteuerte, nicht die durch Therapeut*innen
angeleitete. (ks)

\hypertarget{weitere-medien}{%
\section{8. Weitere Medien}\label{weitere-medien}}

Sarah Turnnidge: \emph{No, A School Librarian Didn't Arrange Those Books
To Troll Boris Johnson}. In: HuffPost UK. 27.08.2020,
\url{https://www.huffingtonpost.co.uk/entry/castle-rock-school-librarian-didnt-arrange-book-boris-johnson_uk_5f4767b7c5b64f17e138dbdd}

Im August 2020 machte besonders auf Twitter ein Aufnahme des englischen
Premierministers Boris Johnson die Runde, die ihn vor einem
Bibliotheksregal der Schulbibliothek der Castle Rock School in
Leicestershire zeigte, in dem offenbar mittels gewitzter Anordnung von
Büchern eine Art visueller Subtweet kommuniziert wurde
(\url{https://twitter.com/NicholasPegg/status/1298647435829563393}).
In der Tat sollte die Anordnung von Philip Pullmans \enquote{The Subtle
Knife} über Ray Bradburys \enquote{Fahrenheit 451} bis hin zu Roald
Dahls \enquote{The Twits} eine versteckte Botschaft senden. Diese galt
aber nicht Boris Johnson, wie die HuffPost (UK) mitteilte. Die
Schulbibliothekarin, die nach eigener Aussage aufgrund mangelnder
Unterstützung den Posten im Februar 2020 verließ, adressierte mit der
sprechenden Aufstellung eigentlich die Schulleitung. Dass die Anordnung
auch sechs Monate nach ihrem Rückzug unverändert war und so Boris
Johnsons Ansprache an die Schulkinder einen unerwartet pointierte
Hintergrundgestaltung bot, dürfte den Grund für ihre Resignation
unterstreichen. Insofern: \enquote{The reality is actually a bit of a
sadder story. If someone had done it for Boris Johnson that would have
been amazing but it's just a weird twist of fate.} (bk)

\begin{center}\rule{0.5\linewidth}{0.5pt}\end{center}

Anita Brookner: \emph{Look At Me}. London: PENGUIN BOOKS, 2016
{[}1983{]}, S. 10

Eine Beobachtung von Francis Hinton, Protagonistin und Bibliothekarin in
Anita Brookners Roman \enquote{Look At Me} zu den Nutzer*innen ihrer
Bibliotheken: \enquote{You get a lot of borderline cases in libraries.}
(bk)

%autor

\end{document}
