Der Text zeigt anhand einer essayistisch selektiven Rückschau in die
Zeit vor den Digital Humanities bibliotheks- und
informationswissenschaftliche Ansätze zur Entwicklung hypertextueller
Werkzeuge für Bibliographie-Verwaltung und Strukturierung des
wissenschaftlichen Diskurses - eine zukunftsweisende Idee für eine
digitale Geisteswissenschaft zur Unterstützung geisteswissenschaftlicher
Denkarbeit jenseits von reinem `distant thinking'.
