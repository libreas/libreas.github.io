\documentclass[a4paper,
fontsize=11pt,
%headings=small,
oneside,
numbers=noperiodatend,
parskip=half-,
bibliography=totoc,
final
]{scrartcl}

\usepackage{synttree}
\usepackage{graphicx}
\setkeys{Gin}{width=.4\textwidth} %default pics size

\graphicspath{{./plots/}}
\usepackage[ngerman]{babel}
\usepackage[T1]{fontenc}
%\usepackage{amsmath}
\usepackage[utf8x]{inputenc}
\usepackage [hyphens]{url}
\usepackage{booktabs} 
\usepackage[left=2.4cm,right=2.4cm,top=2.3cm,bottom=2cm,includeheadfoot]{geometry}
\usepackage{eurosym}
\usepackage{multirow}
\usepackage[ngerman]{varioref}
\setcapindent{1em}
\renewcommand{\labelitemi}{--}
\usepackage{paralist}
\usepackage{pdfpages}
\usepackage{lscape}
\usepackage{float}
\usepackage{acronym}
\usepackage{eurosym}
\usepackage[babel]{csquotes}
\usepackage{longtable,lscape}
\usepackage{mathpazo}
\usepackage[normalem]{ulem} %emphasize weiterhin kursiv
\usepackage[flushmargin,ragged]{footmisc} % left align footnote

\usepackage{listings}

\urlstyle{same}  % don't use monospace font for urls

\usepackage[fleqn]{amsmath}

%adjust fontsize for part

\usepackage{sectsty}
\partfont{\large}

%Das BibTeX-Zeichen mit \BibTeX setzen:
\def\symbol#1{\char #1\relax}
\def\bsl{{\tt\symbol{'134}}}
\def\BibTeX{{\rm B\kern-.05em{\sc i\kern-.025em b}\kern-.08em
    T\kern-.1667em\lower.7ex\hbox{E}\kern-.125emX}}

\usepackage{fancyhdr}
\fancyhf{}
\pagestyle{fancyplain}
\fancyhead[R]{\thepage}

%meta
%meta

\fancyhead[L]{J. Waack \\ %author
LIBREAS. Library Ideas, 30 (2016). % journal, issue, volume.
\href{http://nbn-resolving.de/
}{}} % urn
\fancyhead[R]{\thepage} %page number
\fancyfoot[L] {\textit{Creative Commons BY 3.0}} %licence
\fancyfoot[R] {\textit{ISSN: 1860-7950}}

\title{\LARGE{Die Bibliografie von New Orleans. Zu einem Gedicht}} %title %title
\author{Juliane Waack} %author

\setcounter{page}{1}

\usepackage[colorlinks, linkcolor=black,citecolor=black, urlcolor=blue,
breaklinks= true]{hyperref}

\date{}
\begin{document}

\maketitle
\thispagestyle{fancyplain} 

%abstracts

%body
Die Bibliografie in der Lyrik ist schwer zu finden, ähnlich wie
artverwandte Themen der Ordnung, auch wenn das Gedicht an und für sich
ordnet, Klänge sortiert, Bedeutungen aneinanderreiht und in Beziehung
setzt.

Doch was die Lyrik kann, das thematisiert sie selten, sie ist
Nutznießerin der sprachlichen Möglichkeiten, der Grammatik,
Rechtschreibung, Phonetik und Semantik. Wenn sie über die Grundlagen
spricht, die ihr diese Sinnes- und Klangweite ermöglichen, dann setzt
sie diese in einen Kontext, der aus der Theorie in eine fließende
Lebenswelt schlüpft und dort ein neues Gewand erhält. Auch das ist ein
Merkmal der Lyrik, sich dem Fremden annähern, mit dem Bekannten
fremdeln.

So handelt \enquote{New Orleans Bibliography}\footnote{Foster, Tonya M.
  (2002) \enquote{New Orleans Bibliography} aus \enquote{Callaloo},
  Volume 25, Number 1, Winter 2002.
  \url{https://muse.jhu.edu/article/6713}} der amerikanischen Autorin
und Professorin der Literatur Tonya M. Foster nicht von
Artikelsammlungen und Verzeichnissen, sondern lässt ein Leben der
Eindrücke und Sprache alphabetisch vor unseren Augen die Stationen
ablaufen. Von Flora und Fauna, Kultur und Mythos, Sagen und
Sprichwörtern zur Geographie und Geschichte hin zu den Dingen, die
Kindheit, Erwachsensein, die Jugend und das Alter, Frau sein, Mann sein
und in New Orleans sein ausmachen.

\begin{quote}
\emph{dark-skinned, daughters, dead-end, Desire Projects, desire unmet
is desire multiplied, dirty rice, do, Dorothy, due}

\emph{Elysian Fields, Erato, etouffe, Euterpe, Ezekiel}

\emph{Father John's cough syrup, filé, first born, first born done died,
fleur de lys, flood, front porch, \enquote{fur true?}}
\end{quote}

Die Bibliografie dokumentiert dabei, ohne Dinge zu referenzieren, die
man in Bibliotheken findet. Kein Stück Papier kann \enquote{I like
coffee; I like tea; I like a colored boy and he likes me} so ins Leben
rufen, wie es die erröteten Wangen auf dem Schulhof können, kein
\enquote{light-skinned, lighter than a paper bag} kann schwarz auf weiß
die Unterschiede zwischen Schwarz und Weiß wiedergeben. \enquote{New
Orleans Bibliography} ist ein Verzeichnis der Erinnerung und es erzählt
eine Geschichte, die nur teilweise zugänglich für denjenigen wird, der
sie liest.

Im Gegensatz zur wissenschaftlichen Bibliografie händigt sie uns nicht
jede Referenz klar dokumentiert mit Jahreszahl, Verlag und Format aus,
sondern versteckt sich immer wieder auch hinter der Sprache und dem
Erlebten einer Stimme, die uns fremd ist. \enquote{Katie, kickback, kick
your ass} wird niemand in einem Lexikon nachschlagen können. Das Bild
hinter der Alliteration bleibt verschwommen, ungenau. Ebenso wie
\enquote{Miss Myrtle, Miss Tit}, das Persönliche neben dem Profanen. Wie
sehr sie miteinander vertraut waren, wie sehr sie einander berührt
haben, ob sie ein und dieselbe Person waren, nur aus anderen Augen
gesehen, aus anderen Mündern gerufen im Leben der Erzählerin, das bleibt
uns verborgen. Diese Bibliografie ist die Assoziationskette einer
eigenen Geschichte, einer einzigartigen Geschichte. Sie ist so nicht nur
eine Aneinanderreihung von Wörtern, die dem Leser oft in ihrer wahren
Bedeutung verloren bleiben, sondern auch eine Referenz auf unsere eigene
Bibliografie, unser \enquote{morning, mosquitos, mourning}, unsere
Klänge, Namen, Bilder und Wörter, die weitergereicht werden, um ein
Leben zu ordnen, von A bis Z.

\begin{quote}
\emph{\enquote{a girl who looks like her father is born for luck}}
\end{quote}

%autor

\end{document}