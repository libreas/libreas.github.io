\documentclass[a4paper,
fontsize=11pt,
%headings=small,
oneside,
numbers=noperiodatend,
parskip=half-,
bibliography=totoc,
final
]{scrartcl}

\usepackage{synttree}
\usepackage{graphicx}
\setkeys{Gin}{width=.4\textwidth} %default pics size

\graphicspath{{./plots/}}
\usepackage[ngerman]{babel}
\usepackage[T1]{fontenc}
%\usepackage{amsmath}
\usepackage[utf8x]{inputenc}
\usepackage [hyphens]{url}
\usepackage{booktabs} 
\usepackage[left=2.4cm,right=2.4cm,top=2.3cm,bottom=2cm,includeheadfoot]{geometry}
\usepackage{eurosym}
\usepackage{multirow}
\usepackage[ngerman]{varioref}
\setcapindent{1em}
\renewcommand{\labelitemi}{--}
\usepackage{paralist}
\usepackage{pdfpages}
\usepackage{lscape}
\usepackage{float}
\usepackage{acronym}
\usepackage{eurosym}
\usepackage[babel]{csquotes}
\usepackage{longtable,lscape}
\usepackage{mathpazo}
\usepackage[normalem]{ulem} %emphasize weiterhin kursiv
\usepackage[flushmargin,ragged]{footmisc} % left align footnote

\usepackage{listings}

\urlstyle{same}  % don't use monospace font for urls

\usepackage[fleqn]{amsmath}

%adjust fontsize for part

\usepackage{sectsty}
\partfont{\large}

%Das BibTeX-Zeichen mit \BibTeX setzen:
\def\symbol#1{\char #1\relax}
\def\bsl{{\tt\symbol{'134}}}
\def\BibTeX{{\rm B\kern-.05em{\sc i\kern-.025em b}\kern-.08em
    T\kern-.1667em\lower.7ex\hbox{E}\kern-.125emX}}

\usepackage{fancyhdr}
\fancyhf{}
\pagestyle{fancyplain}
\fancyhead[R]{\thepage}

%meta
%meta

\fancyhead[L]{E. W. Steinhauer \\ %author
LIBREAS. Library Ideas, 30 (2016). % journal, issue, volume.
\href{http://nbn-resolving.de/
}{}} % urn
\fancyhead[R]{\thepage} %page number
\fancyfoot[L] {\textit{Creative Commons BY 3.0}} %licence
\fancyfoot[R] {\textit{ISSN: 1860-7950}}

\title{\LARGE{Die Nutzung einer \enquote{Schattenbibliothek} im Licht des Urheberrechts}} %title %title
\author{Eric W. Steinhauer} %author

\setcounter{page}{1}

\usepackage[colorlinks, linkcolor=black,citecolor=black, urlcolor=blue,
breaklinks= true]{hyperref}

\date{}
\begin{document}

\maketitle
\thispagestyle{fancyplain} 

%abstracts
\begin{abstract}
Nachveröffentlichung von
\url{https://ub-deposit.fernuni-hagen.de/receive/mir_mods_00000825} /
CC-BY
\end{abstract}

%body
\section{Hinführung\protect\footnote{Dieser
  Beitrag stellt lediglich eine persönliche Meinungsäußerung des
  Verfassers dar. Das Manuskript wurde am 14. Dezember 2016
  abgeschlossen.}}

Ohne einen breiten und zuverlässigen Zugang zu Fachpublikationen ist
wissenschaftliches Arbeiten nicht möglich. Als Folge von
Marktkonzentrationen liegt das wissenschaftliche Publikationswesen vor
allem im Bereich der Zeitschriften derzeit in den Händen einiger weniger
internationaler Konzernverlage. Diese vertreiben ihre Inhalte für sie
selbst hoch profitabel an wissenschaftliche Einrichtungen, vor allem an
Bibliotheken. Da wissenschaftliche Veröffentlichungen im
Forschungsprozess nicht substituiert werden können, gelingt es einigen,
besonders wichtigen Verlagen, sehr ambitionierte und zudem stetig
steigende Preise für ihre Produkte zu verlangen.

Die Erwerbungsetats von Bibliotheken und Forschungseinrichtungen können
mit diesen Preissteigerungen schon seit vielen Jahren nicht mehr
mithalten. Die Folge dieser Entwicklung ist auf der einen Seite, dass
sich viele Forschungseinrichtungen notwendige Fachliteratur nicht mehr
in der erforderlichen Breite leisten können oder dass der Erwerb der
Produkte der Konzernverlage zu Lasten von Büchern und anderen
Publikationen kleiner und mittlerer Wissenschaftsverlage geht. Man
spricht in diesem Zusammenhang auch von der so genannten
Zeitschriftenkrise.\footnote{Keller, Alice: Elektronische Zeitschriften
  im Wandel, Wiesbaden 2001, S. 78 ff.}

Dem bezahlten Zugang bei den Konzernverlagen stellt die
Open-Access-Bewegung als Reaktion auf die Zeitschriftenkrise die Idee
frei zugänglicher wissenschaftlicher Inhalte gegenüber. Sie hat in den
letzten Jahren beachtliche Erfolge erzielt und ist dabei, insbesondere
in den Natur- und Technikwissenschaften, zu einem führenden
Publikationsparadigma zu werden. Dessen ungeachtet spielen
kostenpflichtige Publikationen in Konzernverlagen immer noch eine große
Rolle, nicht zuletzt deshalb, weil sie im Wissenschaftsmanagement für
die Messung wissenschaftlicher Reputation von hoher Bedeutung sind.

Die Zeitschriftenkrise erschwert zwar den Zugang zu wissenschaftlichen
Publikationen, macht ihn aber nicht unmöglich. So können Bibliotheken
vor Ort nicht verfügbare Inhalte durch ihre Lieferdienste wie Fernleihe
oder Dokumentdirektlieferung relativ schnell besorgen. Über persönliche
Kontakte unter Wissenschaftlerinnen und Wissenschaftlern haben sich
zudem gut funktionierende Netzwerke etabliert, über die nahezu jede
Veröffentlichung erreichbar ist. Beide Wege sind legal, denn § 53a UrhG
gestattet jedenfalls in Papierform die Dokumentlieferung durch
Bibliotheken, und die zwischen einzelnen Personen übermittelte Kopie ist
eine nach § 53 Abs. 2 S. 1 Nr. 1 UrhG zulässige Vervielfältigung zum
eigenen wissenschaftlichen Gebrauch, die auch ein anderer als der
Begünstigte herstellen darf.

Bibliotheken und persönliche Kontakte als Alternativen zu einem direkt
beim Verlag lizenzierten Zugang haben freilich den Nachteil der
Umständlichkeit und der zeitlichen Verzögerung. Auf dieses Problem
reagieren so genannte \enquote{Schattenbibliotheken}, die über das
Internet eine große Zahl von Veröffentlichungen an einer Stelle für den
sofortigen Zugriff verfügbar machen. Die derzeit bekannteste und wohl
auch größte \enquote{Bibliothek} dieser Art ist Sci-Hub aus Kasachstan
mit rund 58 Millionen Wissenschaftspublikationen.\footnote{Zu nennen
  wäre hier auch Library Genesis oder LibGen mit 52 Millionen Artikeln.}

Die Nutzung von Sci-Hub ist denkbar einfach und funktioniert nach dem
von Google bekannten \enquote{Ein-Schlitz-Prinzip}:

\begin{figure}
\centering
\includegraphics{img/abbildung.png}
\caption{}
\end{figure}

Man braucht nur die DOI einer gewünschten Publikation einzugeben und hat
nach zwei Klicks sofort den Volltext auf dem Bildschirm. Gerade diese
Einfachheit hat dazu geführt, dass Sci-Hub auch von Wissenschaftlerinnen
und Wissenschaftlern genutzt wird, die einen legalen Zugang über
Verlagsangebote an ihren Institutionen haben.\footnote{Eine Studie an
  der Universität Utrecht kommt zu dem Ergebnis, dass 75\% der dort über
  Sci-Hub bezogenen Artikel auch legal (Lizenzen/Open Access) zugänglich
  waren, vgl. \emph{Kramer}, Sci-Hub: access or convenience? A Utrecht
  case study.
  \url{https://im2punt0.wordpress.com/2016/06/20/sci-hub-utrecht-case-study-part-1/}
  und
  \url{https://im2punt0.wordpress.com/2016/06/20/sci-hub-access-or-convenience-a-utrecht-case-study-part-2/}.
  Siehe zur Frage der Nutzergruppen von Sci-Hub auch Bohannon, John:
  Who`s downloading pirated papers? Everyone. in: Science 352 (2016), S.
  508-512.}

Es steht außer Frage, dass das Angebot von Sci-Hub eine massive
Urheberrechtsverletzung darstellt und rechtswidrig ist. Doch gilt dies
auch für die Nutzung dieser Schattenbibliothek?

\section*{Vervielfältigung zum eigenen wissenschaftlichen
Gebrauch?}\label{vervielfuxe4ltigung-zum-eigenen-wissenschaftlichen-gebrauch}

Die übliche Art, Publikationen über Sci-Hub zu beziehen, besteht darin,
einen Aufsatz mit Hilfe seiner DOI, die meist leicht auf
Verlagswebseiten oder in Literaturdatenbanken zu finden ist, aufzurufen
und den so erhaltenen Text entweder abzuspeichern oder für die intensive
Lektüre auszudrucken. Die in beiden Fällen erzeugte Vervielfältigung
greift in das Recht des Urhebers aus § 16 UrhG bzw. in dem Verlag meist
umfassend und ausschließlich eingeräumte Nutzungsrechte ein. Sie ist
mangels einer vertraglich mit dem Rechteinhaber vereinbarten Nutzung nur
dann zulässig, wenn eine Schrankenbestimmung eingreift. In Betracht
kommt hier insbesondere die Vervielfältigung zum eigenen
wissenschaftlichen Gebrauch nach § 53 Abs. 2 S. 1 Nr. 1 UrhG.

Auf den ersten Blick scheinen die Voraussetzungen unproblematisch
gegeben zu sein. Die offensichtliche Rechtswidrigkeit der Vorlage
schließt nach § 53 Abs. 1 S. 1 UrhG lediglich eine Privatkopie aus. Da
dieses Erfordernis gerade nicht für die anderen Anwendungsfälle von § 53
UrhG, also auch nicht bei Vervielfältigungen für den eigenen
wissenschaftlichen Gebrauch, gilt,\footnote{So ausdrücklich
  \emph{Loewenheim}, in: Schricker/Loewenheim, Urheberrecht, 4. Aufl.
  2010, § 53, Rn. 21.} könnte man im Gegenschluss folgern, dass es auf
die Legalität der Quelle in diesem Fall eben nicht ankommt und daher
Vervielfältigungen, die über Sci-Hub ermöglicht werden, rechtlich nicht
zu bean­standen sind.

Doch so einfach ist die Sache nicht. Als im Zuge des so genannten
\enquote{Zweiten Korbes} § 53 Abs. 1 UrhG im Jahr 2007 um das
Erfordernis einer nicht offensichtlich rechtswidrigen Quelle ergänzt
worden ist, wollte der Gesetzgeber dies nicht als Einschränkung einer
vorher nahezu schrankenlosen Privatkopierfreiheit, sondern lediglich als
Klarstellung eines allgemeinen Rechtsgrundsatzes verstanden
wissen.\footnote{Vgl. BT-Drs. 15/1066, S. 2: \enquote{Durch Ergänzung
  des § 53 Abs. 1 Satz 1 -- neu -- UrhG wird klargestellt, dass die nach
  § 53 Abs. 1 -- neu -- UrhG privilegierten Privatkopien nur zulässig
  sind, wenn sie aus legalem Ausgangsmaterial gewonnen werden};
  \emph{Busch}, Zur urheberrechtlichen Einordnung der Nutzung von
  Streamingangeboten, in: GRUR 2011, 502.}

Es liegt daher nahe, auch bei Vervielfältigungen für den eigenen
wissenschaftlichen Gebrauch die Verwendung einer rechtmäßig zugänglich
gemachten Vorlage zu fordern. Als Anknüpfungspunkt im Wortlaut der
Schrankenbestimmung könnte dabei das Merkmal der \enquote{Gebotenheit}
dienen. Dieser unbestimmte Rechtsbegriff unterliegt der Auslegung. Ein
wichtiger Auslegungsmaßstab ist dabei die Richtlinie 2001/29/EG des
Europäischen Parlaments und des Rates vom 22. Mai 2001 zur
Harmonisierung bestimmter Aspekte des Urheberrechts und der verwandten
Schutzrechte in der Informationsgesellschaft
(InfoSoc.-Richtlinie).\footnote{Durch Auslegung des Merkmals der
  Gebotenheit hat der BGH bereits bei § 52a UrhG das mit Blick auf § 53a
  UrhG naheliegende Gegenschlussargument bei der Frage nach dem Vorrang
  angemessener Verlagsangebote neutralisiert, vgl. BGH ZUM 2014, S. 530
  (Meilensteine der Psychologie).} In seinem Urteil über die
Verleihbarkeit von E-Books durch Bibliotheken hat der EuGH jüngst den
allgemeinen Grundsatz angeführt, dass eine Vervielfältigung oder
sonstige auf einer urheberrechtlichen Ausnahmebestimmung gestützte
Nutzung auf Grundlage einer unrechtmäßigen Quelle dem Ziel des
Urheberschutzes, wie er durch die europäischen Richtlinien gewährt wird,
zuwiderlaufe und daher nicht erlaubt werden könne.\footnote{Vgl. EuGH
  C-174/15, Rn. S. 66-71.}

Von daher wird man auch eine Vervielfältigung für den eigenen
wissenschaftlichen Gebrauch, die über die Nutzung von Sci-Hub ermöglicht
wird, als nicht geboten im Sinne von § 53 Abs. 2 S. 1 Nr. 1 UrhG werten
müssen. Sie ist damit rechtswidrig.

\section*{Rechtmäßige Nutzung?}\label{rechtmuxe4uxdfige-nutzung}

Diese Rechtswidrigkeit bezieht sich auf eigene Ausdrucke oder dauerhafte
Abspeicherungen. In beiden Fällen sind jedoch neben dem bloßen Aufruf
des über Sci-Hub angebotenen Werkes noch weitere Handlungen
erforderlich. Wie aber ist die Situation zu bewerten, wenn ein Werk bei
Sci-Hub lediglich aufgerufen und am Bild­schirm gelesen wird, ohne dass
darüber hinaus dauerhafte Vervielfältigungen erstellt werden? Anders
gefragt: Ist auch das bloße Lesen eines Werkes bei Sci-Hub ebenfalls
unzulässig?

Einschlägig zur Beantwortung dieser Frage ist nicht § 53 UrhG, sondern §
44a Nr. 2 UrhG. Danach sind vorübergehende, rein technisch bedingte
Vervielfältigungshandlungen erlaubt, um eine rechtmäßige Nutzung eines
Werkes zu ermöglichen. Diese Vervielfältigungen dürfen zudem keinen
eigenen wirtschaftlichen Wert haben.

Die zunächst entscheidende Frage ist, ob das bloße Lesen am Bildschirm
eines über Sci-Hub bezogenen Werkes eine \enquote{rechtmäßige Nutzung}
im Sinne des Gesetzes ist. Spontan würde man sagen, dass die Nutzung
einer illegalen Schattenbibliothek wohl kaum rechtmäßig genannt werden
könne.\footnote{So im Ergebnis Busch, Thomas: Zur urheberrechtlichen
  Einordnung der Nutzung von Streamingangeboten. in: GRUR 2011, S.
  501-503.} Doch darauf kommt es in diesem Fall nicht an. Entscheidend
bei § 44a UrhG ist nämlich allein die Perspektive des Nutzenden und das,
was er oder sie in einer konkreten Nutzungssituation tut.

Bewertet werden muss daher allein der Vorgang des Lesens am Bildschirm.
Nach ganz herrschender Meinung in der Urheberrechtswissenschaft ist die
bloße Wahrnehmung eines urheberrechtlich geschützten Werkes, der rein
rezeptive Werkgenuss also, von den urheberrechtlichen
Ausschließlichkeitsrechten gar nicht erfasst und damit eine per se
rechtmäßige Nutzung.\footnote{Vgl. auch Erwägungsgrund 33 der
  InfoSoc.-Richtlinie: \enquote{Eine Nutzung sollte als rechtmäßig
  gelten, soweit sie vom Rechtsinhaber zugelassen bzw. nicht durch
  Gesetze beschränkt ist.} Nicht durch Gesetze beschränkt wäre eben der
  rezeptive Werkgenuss.}

In der Literatur wird diese Ansicht, die vor allem im Zusammenhang mit
dem \emph{Streaming} von illegal ins Netz gestellten Kinofilmen oder
Pornographie diskutiert worden ist, von gewichtigen Stimmen
geteilt.\footnote{Zum Diskussionsstand vgl. \emph{Dreier}, in:
  Dreier/Schulze, UrhG, 5. Aufl. 2015, § 44a, Rn. 8 sowie \emph{Wiebe},
  in: Spindler/Schuster, Recht der elektronischen Medien, 3. Aufl. 2015,
  § 44a UrhG, Rn. 10 f.} Selbst das BMJV ist der Ansicht, dass der bloße
Werkgenuss im Internet urheberrechtlich unbedenklich ist.\footnote{Vgl.
  BT-Drs. 18/246, S. 3} Da zudem die Vervielfältigung des über Sci-Hub
zugänglichen Werkes auf dem Bildschirm spätestens mit dem Schließen des
Browsers bzw. des Herunterfahrens des Rechners gelöscht wird, ist sie
auch hinreichend flüchtig.\footnote{Vgl\emph{. von Welser}, in:
  Wandtke/Bullinger, UrhG, 4. Aufl., 2014, § 44a, Rn. 3.} Schließlich
kann die Vervielfältigung als solche wirtschaftlich nicht verwertet
werden, da sie nicht weitergegeben werden kann und überdies nicht
vollständig ist, weil ja immer nur ein Ausschnitt des Werkes am
Bildschirm zu sehen ist. Damit wäre das bloße Lesen von Aufsätzen aus
einer Schattenbibliothek offenbar legal!\footnote{So im Ergebnis auch
  \emph{Dreier}, in: Dreier/Schulze, UrhG, 5. Aufl., 2015, § 53, Rn.
  12b, wonach in den Fällen, in denen eine Kopie nach § 53 UrhG zulässig
  ist, eine Nutzung nach § 44a UrhG gleichwohl legal sein kann. Siehe
  auch Starcke, Andreas: Kinoticket oder Online-Stream? In: JURA 2016,
  S. 1290-1292.}

An dieser Stelle freilich könnte man das oben genannte Argument des EuGH
anführen, wonach es nicht sein könne, unrechtmäßige Quellen zum
Ausgangspunkt für weitere, dann aber legale Nutzungen zu machen. Dieses
Argument ist jedoch eher schwach.

Zum einen geht es beim bloßen Werkgenuss gerade nicht um Nutzungen im
Sinne des Urheberrechts. So wie niemand auf die Idee kommt, das Lesen
von Raubdrucken oder das Anhören illegal gebrannter CDs zu verbieten, so
wird man auch das Lesen von Raubkopien am Bildschirm nicht anders zu
beurteilen haben.\footnote{So argumentieren Fangerow, Kathleen; Schulz,
  Daniela: Die Nutzung von Angeboten auf www.kino.to - Eine
  urheberrechtliche Analyse des Film-Streamings im Internet. In: GRUR
  2010, S. 681. \emph{Ensthaler}, Streaming und Urheberrechtsverletzung,
  in: NJW 2014, S. 1554 freilich wirft die Frage auf, \enquote{ob dieser
  freie Werkgenuss in der digitalen Welt erhalten bleiben soll.}}

Zum anderen gibt es auch einen entscheidenden Unterschied zu einer
dauerhaft angefertigten Kopie. Wer sich mit Hilfe einer
Schattenbibliothek selbst gleichsam eine eigene Schattenbibliothek
aufbaut, ist in der Lage, die abgespeicherten Inhalte anderen zur
Verfügung zu stellen und damit selbst Teil eines illegalen Netzwerkes zu
werden. Diese Möglichkeit besteht beim bloßen Lesen nicht. Um einen
Aufsatz erneut zu konsultieren, muss er jedes Mal neu in der
Schattenbibliothek aufgerufen werden. Wird die Schattenbibliothek
abgeschaltet, ist auch die Lesemöglichkeit nicht mehr gegeben. Hier wird
deutlich, dass die Anfertigung eigener Kopien gegenüber dem bloßen Lesen
eine ganz andere Qualität hat.\footnote{So differenziert im Ergebnis
  auch \emph{Wiebe}, in: Spindler/Schuster, Recht der elektronischen
  Medien, 3. Aufl. 2015, § 44a UrhG, Rn. 11, anders dagegen Marly,
  Jochen: Bildschirmkopien, Cache-Kopien und Streaming als
  urheberrechtliche Herausforderungen. In: EuZW 2014, S. 619.}
Rechteinhaber können den Schaden, den ihnen eine Schattenbibliothek
zufügt, mit Blick auf die \enquote{Raubleser} sofort unterbinden, wenn
es ihnen gelingt, die Schattenbibliothek abzuschalten. Kopien jedoch,
die Nutzer der Schattenbibliothek angefertigt haben, können die
Rechteinhaber praktisch nicht mehr kontrollieren. Wenn ein hohes
Schutzniveau im Bereich des Urheberrechts das Ziel ist, muss die
Rechtsordnung diesen Kontrollverlust, der beim bloßen Lesen nicht
eintritt, verhindern.

\section*{Weiterführende
Überlegungen}\label{weiterfuxfchrende-uxfcberlegungen}

Die offenbar nicht eindeutig rechtswidrige Nutzung einer eindeutig
rechtswidrigen Schattenbibliothek wirft einige Fragen auf. Zunächst wäre
zu überlegen, ob eine Auslegung von § 44a UrhG, wie hier vorgeschlagen,
in einem möglichen Rechtsstreit vor Gericht Bestand hätte. Prognosen
sind hier natürlich schwierig, einige Grundlinien möglicher
Argumentationen lassen sich gleichwohl aufzeigen. Denkbar ist zunächst,
beim Merkmal der \enquote{rechtmäßigen Nutzung} anzusetzen und die
Nutzung einer Schattenbibliothek auch im bloßen Lesezugriff als
rechtswidrig zu bewerten.\footnote{Vgl. \emph{Busch}, Zur
  urheberrechtlichen Einordnung der Nutzung von Streamingangeboten, in:
  GRUR 2011, S. 501.} Allerdings müsste dann die sinnvolle Doktrin vom
urheberrechtsfreien Werkgenuss relativiert werden.\footnote{Nach
  \emph{Rehbinder/Peukert}, Urheberrecht, 17. Aufl., 2015, Rn. 608 soll
  sie bei der digitalen Werk­nutzung nicht mehr gelten, so auch Marly,
  Jochen: Bildschirmkopien, Cache-Kopien und Streaming als
  urheberrechtliche Herausforderungen. In: EuZW 2014, S. 618:
  \enquote{reine Werknutzung zugleich zustimmungspflichtige
  Vervielfältigung.}}

Interessanter ist es daher, das Merkmal der \enquote{eigenen
wirtschaftlichen Bedeutung} näher zu betrachten.\footnote{Vgl. Marly,
  Jochen: Bildschirmkopien, Cache-Kopien und Streaming als
  urheberrechtliche Heraus­forderungen. In: EuZW 2014, S. 619.} Bislang
wird hier meist auf die Vervielfältigungshandlung bzw. die dabei
erzeugten Kopien abgestellt und nach deren unmittelbarer
wirtschaftlicher Bedeutung gefragt. Eine solche Bedeutung setzt aber in
der Regel die Möglichkeit voraus, dass die erzeugten Kopien irgendwie
weitergegeben werden können.

Hier freilich stellt sich das Problem, dass eine Weitergabe von
flüchtigen oder begleitenden Kopien schon vom Begriff schwer möglich
ist. Soll also das Merkmal der \enquote{eigenen wirtschaftlichen
Bedeutung} einen substanziellen Anwendungsbereich haben, so könnte man
überlegen, die wirtschaftliche Bedeutung der fraglichen
Vervielfältigungen umfassender zu bestimmen. Hierfür spricht, dass § 44a
UrhG ja nicht vom \enquote{Wert} der Vervielfältigung spricht, was den
Blick auf die Kopie selbst verengt, sondern von der \enquote{Bedeutung},
womit auch ökonomische Umstände und Konsequenzen der erzeugten Kopie
gemeint sein können. Eine \enquote{wirtschaftliche Bedeutung} dieser Art
kommt einer flüchtigen Lesekopie in mehrfacher Hinsicht zu.

Zunächst kann man darauf abstellen, dass der Betrachter eines illegal
zugänglich gemachten Aufsatzes sich den eigenen Erwerb erspart. Dieses
Argument ist bei wissenschaftlichen Fachaufsätzen wenig sinnvoll. Das
liegt daran, dass diese Aufsätze in aller Regel nicht von den Leserinnen
und Lesern erworben,\footnote{Dass einzelne Beiträge für 30 \euro{} und
  mehr einzeln bezogen werden können, spielt in der Praxis keine Rolle.
  Solche Angebote zur Phantasiepreisen werden von Leserinnen und Lesern
  an Hochschulen und anderen Forschungseinrichtungen nicht genutzt.}
sondern über einen durch Bibliotheken und Forschungseinrichtungen
bereitgestellten und für die Nutzerinnen und Nutzer selbst kostenfreien
Zugang genutzt werden. Steht ein solcher Zugang nicht zur Verfügung,
kann der gewünschte Beitrag gegen eine sehr geringe Gebühr über die
Fernleihe bezogen werden. Die Nutzung einer Schattenbibliothek erspart
hier offenbar kein Geld. Oder doch?

Tatsächlich ist auch ein bloß lesender Zugriff Gegenstand eigener, auf
individuelle Endnutzerinnen und Endnutzer zugeschnittener kommerzieller
Angebote. Ein Beispiel hierfür ist der Dienst \enquote{DeepDyve}
(\url{https://www.deepdyve.com}). Dort kann gegen Zahlung von 40 \$ im
Monat auf über 10.000 wissenschaftliche Zeitschriften lesend zugegriffen
werden. Genau diese Kosten ersparen sich Nutzerinnen und Nutzer von
Sci-Hub, wenn sie Inhalte, die sie über Angebote wie DeepDyve zu
angemessenen Bedingungen legal nutzen können, nicht dort, sondern über
eine Schattenbibliothek beziehen. Insoweit kann man mit guten Gründen
argumentieren, dass auch flüchtige Kopien für das bloße Lesen am
Bildschirm eine wirtschaftliche Bedeutung haben, jedenfalls dann, wenn
eben dieses Lesen gegen ein Entgelt zu angemessene Bedingungen angeboten
wird, wenn es also ein entsprechendes Geschäftsmodell gibt, das durch
die Nutzung illegaler Quellen unmittelbar beeinträchtigt wird.\footnote{Vgl.
  LG München I, in: MMR 2007, S. 329 sowie im Ergebnis Ensthaler,
  Jürgen: Streaming und Urheberrechtsverletzung. In: NJW 2014, S. 1555,
  der § 44a UrhG immer dann verneint, wenn die in Frage stehende Nutzung
  einer wirtschaftlich bedeutsamen Verwertungsmöglichkeit entspricht.}

Schließlich kann noch überlegt werden, welche Auswirkung die Nutzung von
Schattenbibliotheken wie Sci-Hub auf die weitere Verwertung von
wissenschaftlichen Publikationen hat. Dabei geht es weniger um den
unmittelbaren Vertrieb von Inhalten, sondern mehr darum, dass im Sinne
von Big Data die Nutzung digital verfügbarer Publikationen eine immer
wichtiger werdende Quelle für wissenschaftsunterstützende
Dienstleistungen wie die Messung von Reputation und dergleichen wird.
Nutzungsdaten sind aber nur dann aussagekräftig, wenn sie die
tatsächliche Nutzung auch weitgehend vollständig erfassen. Vor diesem
Hintergrund stellen Schattenbibliotheken ein ernstes Problem für alle
Art bibliometrischer Erfassung von Nutzungsvorgängen und darauf
aufbauender Verwertung in anderen Informationsprodukten dar.\footnote{Darauf
  weist etwa McNutt, Marcia: My love-hate of Sci-Hub. In: Science 352
  (2016), S. 497 hin.}

Auch wenn diese Auswirkungen wohl zu abstrakt sind, um sie im Merkmal
der \enquote{eigenen wirtschaftlichen Bedeutung} im Rahmen von § 44a
UrhG zu berücksichtigen, so sind sie doch für die informationsfachliche
Einordnung der Nutzung von Schattenbibliotheken nicht unwichtig. Die
klammheimliche Freude, dass durch das Angebot von Schattenbibliotheken
wie Sci-Hub nun den wenig geliebten Konzernverlagen eins ausgewischt
wird,\footnote{Vgl. Russel, Carrie; Sanchez, Ed: Sci-Hub unmasked --
  Piracy, information policy, and your library. In: College and Research
  Libraries News 77 (2016), S. 122.} könnte so auch für Freunde des
offenen Zugangs getrübt werden, soweit sie sich für bessere
Reputationsmessungen als den hierfür im Grunde vollkommen ungeeigneten
Impact-Faktor interessieren, denn Schattenbibliotheken werden ihre
Nutzerdaten sicher nicht dauerhaft und zuverlässig zur Verfügung
stellen.\footnote{Allerdings gibt es bei Sci-Hub eine bemerkenswerte
  Offenheit, vgl. Elbakyan, Alexandra; Bohannon, John: Data from: Who`s
  downloading pirated papers? Everyone. Dryad Digital Repository. 2006.
  Online unter: \url{http://dx.doi.org/10.5061/dryad.q447c}.}

Da hier zudem ein mögliches neues Geschäftsfeld gerade für große
Konzernverlage liegt, werden auch diese die Herausforderungen, die eine
Schattenbibliothek wie Sci-Hub nicht nur beim Zugang, sondern vor allem
beim Nutzungskomfort bietet, sehr ernst nehmen müssen. Diese Verlage
laufen sonst Gefahr, durch überzogene Renditeerwartungen bei dem von
ihnen vertriebenen Content, die zu Zugangsengpässen bei Bibliotheken und
damit zu einem Ausweichen auf Schattenbibliotheken führen, die für sie
wirtschaftlich hochinteressante Entwicklung neuer Analyseinstrumente für
die wissenschaftliche Reputationsmessung selbst zu hintertreiben.

Als Ergebnis kann an dieser Stelle festgehalten werden, dass mit Blick
auf die Existenz legaler kommerzieller Angebote für die bloße Lektüre
wissenschaftlicher Beiträge im Internet auch das Lesen von Inhalten über
Sci-Hub von wirtschaftlicher Bedeutung ist, so dass eine Berufung auf §
44a Nr. 2 UrhG für diese Inhalte ausscheidet, wenn man das Merkmal der
\enquote{eigenen wirtschaftlichen Bedeutung} auch auf ersparte eigene
Aufwendungen bzw. die Existenz eigenständiger Verwertungswege für die
bei Sci-Hub gewählte Nutzungsform erstrecken möchte.

\section*{Fazit}\label{fazit}

Im Licht des Urheberrechts ist und bleibt Sci-Hub für den Nutzer eine
Schattenbibliothek. Ausdrucke und Abspeicherungen von über Sci-Hub
zugänglich gemachten Werken sind nicht zulässig. Selbst das bloße Lesen
findet in einer rechtlichen Grauzone statt. Auch wenn die Nutzung von
Sci-Hub daher nicht unbedenklich ist, wird in der Praxis vielleicht
weniger das nicht restlos klare Recht, sondern am Ende das eigene
Gewissen darüber entscheiden, ob man den Verlockungen von 58 Millionen
frei zugänglichen Wissenschaftspublikationen nachgeben will oder nicht.
Rechtliche Gründe sprechen wohl dafür, das besser nicht zu tun.

\section*{Informationsethischer
Nachtrag}\label{informationsethischer-nachtrag}

Die informationsethische Frage, die Schattenbibliotheken mit
wissenschaftlichen Inhalten aufwerfen, ist nur vordergründig einfach zu
lösen. Einerseits gilt hier die Besonderheit, dass wissenschaftliche
Autorinnen und Autoren in aller Regel nicht vergütet werden und ihnen
daher, im Gegensatz etwa zu Künstlerinnen und Künstlern bei der Musik-
und Filmpiraterie, kein persönlicher Schaden zugefügt wird.

Andererseits bedingt ein professionelles wissenschaftliches
Publikationswesen hohe Kosten, so dass die Verlage als wesentliche
Akteure in diesem System um dessen Funktionsfähigkeit willen auf
Einnahmen angewiesen sind.

Letztlich ist alles eine Frage der Balance. Schattenbibliotheken
reagieren hier auf offenbar unausgewogene Entwicklungen der letzten
Jahre mit dem Risiko freilich, neben den schwarzen Schafen unter den
Verlagen auch dem für die Wissenschaft unverzichtbaren Publikations- und
Reputationssystem als solchem einen irreparablen Schaden zuzufügen.
Daher muss die Bereitstellung einfacher und bezahlbarer legaler
Zugriffsmöglichkeiten das Ziel sein. Dass populäre Schattenbibliotheken
aber erst den dafür notwendigen Handlungsdruck erzeugen müssen,
erschwert mit Blick auf das anzustrebende Ziel eine eindeutige
informationsethische Einordnung ihrer Nutzung. \footnote{Gegen die
  Nutzung von Sci-Hub aus ethischer Sicht sprechen sich Saleem, Faheed;
  Hasaali, Mohamed Azmi; ul Haq, Noman: Sci-Hub \& ethical issues. In:
  Research in Social and Administrative Pharmacy 13 (2017), S. 253 aus.
  Kritisch aus Bibliothekssicht sind Russel, Carrie; Sanchez, Ed;
  Sci-Hub unmasked -- Piracy, information policy, and your library. In:
  College and Research Libraries News 77 (2016), S. 125, die Open Access
  und bessere Dokumentlieferung als Lösung anbieten. Eine realistische
  Sicht auf die seit Jahren übliche Praxis des digitalen Austausches von
  Fachpublikationen bietet Heathers, James: Why Sci-Hub Will Win.
  02.05.2016, online unter:
  \url{https://medium.com/@jamesheathers/why-sci-hub-will-win-595b53aae9fa\#.tscdkaz0v}.
  Siehe auch Steinhauer, Eric W: Urheberrechtsnovelle - Das Urheberrecht
  in der Wissenschaft, oder \enquote{The Dirty Way Of Information}. In:
  H-Soz-Kult, 27.09.2007
  \url{http://www.hsozkult.de/debate/id/diskussionen-938}.}

%autor
\begin{center}\rule{0.5\linewidth}{\linethickness}\end{center}

\textbf{Prof.~Dr.~Eric W. Steinhauer}, Verwaltungsdirektor an der
Fernuniversität in Hagen, Honorarprofessor an der Humboldt-Universität
zu Berlin.

Prof.~Dr.~Eric W. Steinhauer hat Rechtswissenschaft, katholische
Theologie, Philosophie, Politik- und Erziehungswissenschaft in Münster
und Hagen studiert. Er promovierte zum Dr.~jur. in Münster. Nach dem
Bibliotheksreferedariat in Freiburg/Brsg. und München war er
wissenschaftlicher Bibliothekar in Ilmenau, Magdeburg und arbeitet in
dieser Position derzeit in Hagen. Seit 2014 hat er eine Honorarprofessor
am Institut für Bibliotheks- und Informationswissenschaft an der
Humboldt-Universität zu Berlin. Seine Arbeitsschwerpunkte sind
Bibliotheks- und Urheberrecht, digitales kulturelles Gedächtnis,
Kulturwissenschaft der Bibliothek.

\end{document}
