\documentclass[a4paper,
fontsize=11pt,
%headings=small,
oneside,
numbers=noperiodatend,
parskip=half-,
bibliography=totoc,
final
]{scrartcl}

\usepackage[babel]{csquotes}
\usepackage{synttree}
\usepackage{graphicx}
\setkeys{Gin}{width=.4\textwidth} %default pics size

\graphicspath{{./plots/}}
\usepackage[ngerman]{babel}
\usepackage[T1]{fontenc}
%\usepackage{amsmath}
\usepackage[utf8x]{inputenc}
\usepackage [hyphens]{url}
\usepackage{booktabs} 
\usepackage[left=2.4cm,right=2.4cm,top=2.3cm,bottom=2cm,includeheadfoot]{geometry}
\usepackage[labelformat=empty]{caption} % option 'labelformat=empty]' to surpress adding "Abbildung 1:" or "Figure 1" before each caption / use parameter '\captionsetup{labelformat=empty}' instead to change this for just one caption
\usepackage{eurosym}
\usepackage{multirow}
\usepackage[ngerman]{varioref}
\setcapindent{1em}
\renewcommand{\labelitemi}{--}
\usepackage{paralist}
\usepackage{pdfpages}
\usepackage{lscape}
\usepackage{float}
\usepackage{acronym}
\usepackage{eurosym}
\usepackage{longtable,lscape}
\usepackage{mathpazo}
\usepackage[normalem]{ulem} %emphasize weiterhin kursiv
\usepackage[flushmargin,ragged]{footmisc} % left align footnote
\usepackage{ccicons} 
\setcapindent{0pt} % no indentation in captions
\usepackage{xurl} % Breaks URLs

%%%% fancy LIBREAS URL color 
\usepackage{xcolor}
\definecolor{libreas}{RGB}{112,0,0}

\usepackage{listings}

\urlstyle{same}  % don't use monospace font for urls

\usepackage[fleqn]{amsmath}

%adjust fontsize for part

\usepackage{sectsty}
\partfont{\large}

%Das BibTeX-Zeichen mit \BibTeX setzen:
\def\symbol#1{\char #1\relax}
\def\bsl{{\tt\symbol{'134}}}
\def\BibTeX{{\rm B\kern-.05em{\sc i\kern-.025em b}\kern-.08em
    T\kern-.1667em\lower.7ex\hbox{E}\kern-.125emX}}

\usepackage{fancyhdr}
\fancyhf{}
\pagestyle{fancyplain}
\fancyhead[R]{\thepage}

% make sure bookmarks are created eventough sections are not numbered!
% uncommend if sections are numbered (bookmarks created by default)
\makeatletter
\renewcommand\@seccntformat[1]{}
\makeatother

% typo setup
\clubpenalty = 10000
\widowpenalty = 10000
\displaywidowpenalty = 10000

\usepackage{hyperxmp}
\usepackage[colorlinks, linkcolor=black,citecolor=black, urlcolor=libreas,
breaklinks= true,bookmarks=true,bookmarksopen=true]{hyperref}
\usepackage{breakurl}

%meta
%meta

\fancyhead[L]{U. Preuß\\ %author
LIBREAS. Library Ideas, 46 (2024). % journal, issue, volume.
\href{https://doi.org/10.18452/31224}{\color{black}https://doi.org/10.18452/31224}
{}} % doi
\fancyhead[R]{\thepage} %page number
\fancyfoot[L] {\ccLogo \ccAttribution\ \href{https://creativecommons.org/licenses/by/4.0/}{\color{black}Creative Commons BY 4.0}}  %licence
\fancyfoot[R] {ISSN: 1860-7950}

\title{\LARGE{Digitaler Wandel in der Kultur – Bibliotheken und Förderung im Land Brandenburg}}% title
\author{Ulf Preuß} % author

\setcounter{page}{1}

\hypersetup{%
      pdftitle={Digitaler Wandel in der Kultur – Bibliotheken und Förderung im Land Brandenburg},
      pdfauthor={Ulf Preuß},
      pdfcopyright={CC BY 4.0 International},
      pdfsubject={LIBREAS. Library Ideas, 46 (2024).},
      pdfkeywords={Digitalisierung, Brandenburg, Fördermittel, Bibliotheken, Kulturerbe},
      pdflicenseurl={https://creativecommons.org/licenses/by/4.0/},
      pdfurl={https://doi.org/10.18452/31224},
      pdfdoi={10.18452/31224},
      pdflang={de},
      pdfmetalang={de}
     }



\date{}
\begin{document}

\maketitle
\thispagestyle{fancyplain} 

%abstracts
\begin{abstract}
\noindent
\textbf{Zusammenfassung}: Die sich in immer kürzeren Zyklen verändernden
digitalen Rahmenbedingungen bieten Kultureinrichtungen neue Chancen und
stellen gleichzeitig große Herausforderungen für die damit verbundenen
Anpassungsprozesse dar. Jede Kultureinrichtung hat dabei einen
individuellen, konkreten Handlungsrahmen, in welchem die kulturelle
Aufgabe realisiert wird, so auch die Bibliotheken. Im Land Brandenburg
wurden kulturpolitische Rahmenbedingungen auf strategischer und
konzeptioneller Ebene neu ausgerichtet. Mit einer neuen Förderlinie
steht ein umfassendes Förderwerkzeug zur Verfügung. Für die Förderung
von Digitalisierungsmaßnahmen spielen dabei institutionelle
Digitalstrategien eine Schlüsselrolle. Der Beitrag ordnet die Richtlinie
in diesem Kontext kurz ein.

\begin{center}\rule{0.5\linewidth}{0.5pt}\end{center}
\end{abstract}

%body
\section{Digitaler Wandel}\label{digitaler-wandel}

Aufgrund der vielen technologischen und damit einhergehenden
gesellschaftlichen Veränderungen der letzten Jahre, wie
massenmarkttaugliche Miniaturisierung und Mobilisierung von
Digitaltechnik (zum Beispiel Smartphone Apple iPhone 1 von 2007), Social
Media (beispielsweise facebook Start 2004) oder praktisch nutzbare
KI-Anwendungen (unter anderem Veröffentlichung von ChatGPT in 2022),
stehen die Kultureinrichtungen -- wie viele Lebens- und Arbeitsbereiche
-- vor der permanenten Herausforderung der Anpassung an ein rasant
verändertes Umfeld. Im Normalfall stehen den Kultureinrichtungen nur
geringe Investitionsmittel zur Verfügung, was zu einer weiteren Öffnung
der Schere zwischen Anpassungsdruck und Anpassungsfähigkeit führt. Im
Zusammenhang mit der Corona-Pandemie und Fördermöglichkeiten zur
Bewältigung der pandemiebedingten Auswirkungen setzte in vielen
Bereichen ein kurzfristiger Digitalisierungsschub ein. Plötzlich
verfügbare, teils umfangreiche Fördermittel stellten viele Einrichtungen
vor neue Herausforderungen. Insbesondere die klare Formulierung von
Anforderungen und das Projektmanagement waren -- und sind auch sonst oft
-- unterschätzte Projektbestandteile, welche sich negativ auf die
Realisierbarkeit auswirken können.

\section{Kulturpolitik und Förderung im Land
Brandenburg}\label{kulturpolitik-und-fuxf6rderung-im-land-brandenburg}

Bereits Ende 2018 veröffentlichte die Landesregierung Brandenburg die
\enquote{Zukunftsstrategie Digitales Brandenburg}.\footnote{(2018)
  Landesregierung Brandenburg/Staatskanzlei: Zukunftsstrategie Digitales
  Brandenburg. URL:
  \url{https://www.demografie-portal.de/DE/Publikationen/2018/zukunftsstrategie-digitales-brandenburg.pdf?__blob=publicationFile&v=2}
  (Letzter Aufruf: 1.12.2024)} In dieser ressortübergreifenden Strategie
wurden insgesamt 202 Maßnahmen beschrieben, davon für den Kulturbereich:

\begin{itemize}
\item
  Nr. 126 \enquote{Sicherung des filmkulturellen Erbes}
\item
  Nr. 127 \enquote{Kulturelle Bildung}
\item
  Nr. 128 \enquote{Sicherung und Präsentation des kulturellen Erbes und
  des Kulturgutes}
\item
  Nr. 129 \enquote{Entwicklung von Kultureinrichtungen zu modernen
  Kulturbetrieben}
\item
  Nr. 130 \enquote{Digitale Vermittlung kultureller Inhalte und
  kultureller Angebote}
\end{itemize}

Basierend auf den Maßnahmen 127--130\footnote{Ebenda S. 29f.} wurde die
Förderlinie \enquote{Förderung und Begleitung des digitalen Wandels im
Kulturbereich im Land Brandenburg} (DiWa) durch das Ministerium für
Wissenschaft, Forschung und Kultur entwickelt. Dieses Programm steht
Kulturinstitutionen aller Kulturbereiche (Kulturschaffende,
kulturbewahrende und kulturvermittelnde Einrichtungen sowie
sozio-kulturellen Institutionen) mit Sitz in Brandenburg zur Verfügung.
Die Förderlinie ist zudem in die Digitale Agenda\footnote{(2021) MWFK:
  Digital Agenda. URL:
  \url{https://mwfk.brandenburg.de/sixcms/media.php/9/MWFK_digitaleAgenda.pdf}
  (Letzter Aufruf: 1.12.2024)} und die Kulturpolitsche Strategie
2024\footnote{(2024) MWFK: Kulturpolitische Strategie 2024. URL:
  \url{https://mwfk.brandenburg.de/sixcms/media.php/9/Kulturstrategiebf.pdf}
  (Letzter Aufruf: 1.12.2024)} des Ministeriums für Wissenschaft,
Forschung und Kultur (MWFK) eingebettet. Erstmalig konnten Anträge 2021
für eine Förderung im Jahr 2022 gestellt werden.

Die DiWa-Förderlinie umfasste zunächst die Felder:

\begin{itemize}
\item
  (1) Entwicklung einer Digitalstrategie und Qualifikation,
\item
  (2) IT-Ausstattung und
\item
  (3) prototypische Entwicklung neuer künstlerischer Inhalte und
  Vermittlungsformen.
\end{itemize}

Bereits seit 2012 werden durch das Land Brandenburg zudem Fördermittel
für den Bereich der retrospektiven Digitalisierung von Kulturgut zur
Verfügung gestellt. Diese Förderung geht wiederum auf den Beschluss der
Kultusministerkonferenz (KMK) von 2009 zur Entwicklung der Deutschen
Digitalen Bibliothek zurück, welche zudem eine Unterstützung der
Einrichtungen durch die jeweiligen Bundesländer vorsah. Im
Förderzeitraum 2013--2023 konnten insgesamt 109 Projekte, mit zusammen
über 200 Projektbeteiligungen, in Verbund- und Einzelprojekten
realisiert werden.\footnote{FH Potsdam: Digitalisierung in der Kultur
  mit Landesförderung. URL:
  \url{https://www.fh-potsdam.de/hochschule-karriere/organisation/assoziierte-einrichtungen/koordinierungsstelle-brandenburg-digital/digitalisierungsprojekte}
  (Letzter Aufruf: 1.12.2024)} Ab dem Förderzeitraum 2024 wurde diese
Förderlinie in die umfassendere DiWa-Förderlinie\footnote{(2024) MWFK:
  Fördergrundsätze des Ministeriums für Wissenschaft, Forschung und
  Kultur zur \enquote{Förderung und Begleitung des digitalen Wandels im
  Kulturbereich im Land Brandenburg 2025}. URL:
  \url{https://mwfk.brandenburg.de/sixcms/media.php/9/FG\%20DIWA.pdf}
  (Letzter Aufruf: 1.12.2024)} integriert. Dieses Förderprogramm bietet
aktuell somit folgende Maßnahmenbereiche:

\begin{itemize}
\item
  Modul A) Strategie und Qualifikation,
\item
  Modul B) Digitale Infrastruktur,
\item
  Modul C) Retrospektive Digitalisierung und
\item
  Modul D) Kunst und Vermittlung.
\end{itemize}

Die zentrale Verwaltung der Förderlinie obliegt dem MWFK. Für die
Beratung im Vorfeld der Antragstellung steht die Koordinierungsstelle
Brandenburg-digital\footnote{FH Potsdam: Koordinierungsstelle
  Brandenburg-digital. URL:
  \url{https://www.fh-potsdam.de/hochschule-karriere/organisation/assoziierte-einrichtungen/koordinierungsstelle-brandenburg-digital}
  (Letzter Aufruf: 1.12.2024)} an der Fachhochschule Potsdam zur
Verfügung.

\section{Digitalstrategie}\label{digitalstrategie}

Kernelement der Förderung ist die Digitalstrategie der jeweiligen
Einrichtung. Falls diese noch nicht vorhanden ist, können Mittel für
Beratungsleistungen und Dienstleistungen Dritter oder projektbezogene
zusätzliche Personalausgaben zur Erarbeitung einer eigenen
Digitalstrategie beantragt werden. Erst mit dem Vorliegen der Strategie
können Maßnahmen aus den anderen Modulen beantragt werden, da eine
möglichst zielgerichtete und nachhaltige Förderung angestrebt wird.

Aufgrund der sehr unterschiedlichen Kulturbereiche (von Theater,
Orchester, Sozio-Kultur, Musik und Kunstschulen bis hin zu Museen,
Archiven und Bibliotheken), der zumeist kommunalen oder freien
Trägerschaft (zum Beispiel gemeinnützige Stiftungen und Vereine) und oft
sehr knappen personal- und finanziellen Ressourcen, sind Art und Umfang
der jeweiligen Strategie sehr unterschiedlich. Ausgangspunkt zur
Erstellung einer Digitalstrategie ist die jeweilige Aufgabe der
Einrichtung. Daraus ergibt sich das Handeln der Einrichtung inklusive
der Auswahl und Nutzung von Systemen.

Im Wesentlichen sollte eine Digitalstrategie auf die ganzheitliche
Betrachtung der Kultureinrichtung und ihrer digitalen
Entwicklungspotentiale abzielen. Ein zentraler Bereich ist die eigene
Verwaltung und Organisation der Einrichtung an sich, unter anderem
Gebäude-, Personal- und Finanzverwaltung. Ein weiterer Bereich ist der
jeweilige inhaltliche Gegenstand der Arbeit der Kultureinrichtung. Dies
kann im Kontext einer Bibliothek beispielsweise der eigene und/oder
lizenzierte Bestand oder der zur Verfügung stehende öffentliche Raum
(Dritter Ort) sein. Zudem haben alle Kultureinrichtungen den Bereich der
zielgruppenspezifischen Kulturangebote, welche eine nutzungszentrierte
Sicht erfordern. Für Bibliotheken mit historischen Beständen könnte dies
unter anderem das Angebot einer digitalen Zugänglichkeit durch ein
digitales Sammlungssystem mit webbasiertem Zugriff sein. Durch eine
vergleichende Betrachtung der aktuellen Situation mit den digitalen
Erfordernissen, lassen sich Entwicklungsbereiche transparent
dokumentieren und Prioritäten für die Umsetzung ableiten.

Eine Digitalstrategie beinhaltet, unter Verweis auf gegebenenfalls
bestehende Dokumente, im Wesentlichen folgendes:

\begin{itemize}
\item
  Kurzfassung der Aufgabe der jeweiligen Einrichtung (zum Beispiel der
  Bibliothek)
\item
  Kurzfassung der aktuell und mittelfristig verfügbaren Ressourcen
  (unter anderem Personal und Budget)
\item
  Kurzfassung dessen, was im Bereich der Digitalität bereits erfolgte
  (Ist-Stand)
\item
  Betrachtung der Digitalanforderungen

  \begin{itemize}
  \item
    Bereich Verwaltung und Organisation (zum Beispiel Finanzverwaltung)
  \item
    Bereich des jeweiligen Gegenstandes der Einrichtung (was muss im
    Vorfeld der Kulturangebote vorhanden sein)
  \item
    Bereich Kulturangebote (nutzungszentrierte Sicht)
  \end{itemize}
\item
  Soll/Ist-Abgleich
\item
  Übersicht der künftigen Handlungsfelder, inklusive einer Priorisierung
\end{itemize}

\section{Digitalisierung der
Bibliotheken}\label{digitalisierung-der-bibliotheken}

Bibliotheken können im Zusammenhang mit dem Thema Digitalisierung auf
eine sehr lange aktive Entwicklung zurückgreifen. Die digitale
Zusammenarbeit im Rahmen der Katalogisierung und übergreifenden
Zugänglichkeit ist für die meisten Bibliotheken alltägliches Geschäft.
Die Erschließung erfolgt in überregionalen Verbünden, wie dem
Gemeinsamen Bibliotheksverbund (GBV), welcher durch die
Universitätsbibliothek Potsdam\footnote{Universität Potsdam: Über die
  Universitätsbibliothek. URL:
  \url{https://www.ub.uni-potsdam.de/de/ueber-uns/ueber-die-universitaetsbibliothek}
  (Letzter Aufruf: 1.12.2024)} genutzt wird. Zusätzliche
Rechercheoptionen und weitergehende Funktionalitäten bieten regionale
Verbünde, wie der Kooperativen Bibliotheksverbund Berlin-Brandenburg
(KOBV).\footnote{Zuse-Institut Berlin: Kooperativer Bibliotheksverbund
  Berlin-Brandenburg. URL: \url{https://www.kobv.de/} Aufruf: 1.12.2024)}
Dieser ist ein Zusammenschluss von Öffentlichen, Wissenschaftlichen und
Spezialbibliotheken und bietet Dienstleistungen in den Bereichen
Recherche, Archivierung, Katalogisierung und Hosting. Neben den lokalen
Beständen bieten Bibliotheken auch Zugriff auf externe Ressourcen.
Hierfür entwickelten sich verschiedene Lizenzmodelle, wie zum Beispiel
den Nationallizenzen der Deutschen Forschungsgemeinschaft
(DFG),\footnote{DFG: Nationallizenzen. URL:
  \url{https://www.nationallizenzen.de/} (Letzter Aufruf: 1.12.2024)}
dem Angebot onleihe\footnote{Verbund der Öffentlichen Bibliotheken
  Berlin: onleihe. URL: \url{https://voebb.onleihe.de/} Aufruf:
  1.12.2024)} für E-Books, Zeitungen und Zeitschriften oder dem Zugriff
auf das Portal filmfriends\footnote{filmwerte GmbH: filmfriends. URL:
  \url{https://www.filmfriend.de/} (Letzter Aufruf: 1.12.2024)} für
Öffentliche Bibliotheken. Darüber hinaus entwickelten sich im Rahmen der
erweiterten Aufgabeninterpretation insbesondere der Öffentlichen
Bibliotheken weitere Bereiche mit digitalen Dimensionen. Hierzu gehört
das Konzept \enquote{Dritter Ort},das die Bibliotheksnutzenden stärker
in den Fokus rückt. Neben dem sogenannten ersten Ort (dem Zuhause) und
dem zweiten Ort (dem Arbeitsplatz) wird als ein solcher \emph{dritter}
Ort ein Sozialraum verstanden. Dieser wird mit bestimmten Eigenschaften,
wie Neutralität, Inklusion, Erreichbarkeit und Zugänglichkeit, offene
Atmosphäre und kontinuierliche Weiterentwicklung, in Verbindung gesetzt.
Die Bedeutung der Arbeit mit Bibliotheksbeständen rückt dabei etwas
zugunsten anderer Wünsche und Anforderungen an den realen, öffentlich
zugänglichen und nutzbaren Raum, den die Bibliotheken bieten, in den
Hintergrund. In diesem Zusammenhang wurden zusätzliche Angebote wie
Makerspaces, Gaming-Areas, Medienwerkstätten, Repair-Cafés,
Bibliotheksgärten und -cafés oder digitale, analoge und interaktive
Veranstaltungsprogramme entwickelt.\footnote{(2020) Bibliotheksportal:
  Der Dritte Ort. Ein vielbeachtetes Konzept im Bibliothekswesen. URL:
  \url{https://bibliotheksportal.de/informationen/die-bibliothek-als-dritter-ort/dritter-ort/}
  (Letzter Aufruf: 1.12.2024)} Die Bibliotheken waren und sind daher oft
Vorreiter im Kontext Digitalisierung.

Die genannten Ansätze im Bibliothekswesen sind wegweisend, gehen aber
auch mit größeren Veränderungsprozessen hinsichtlich der hierfür
erforderlichen räumlichen, personellen, IT-infrastrukturellen und
letztlich finanziellen Ressourcen einher. Die Formulierung einer
ganzheitlichen Digitalstrategie hilft den Einrichtungen, ihre
Realsituation für die Mitarbeitenden, den Träger und potenzielle
Förderer transparent dar- und Entwicklungsleitlinien und konkrete
Handlungsfelder herauszustellen. Hierbei können sich Bibliotheken zudem
auf übergreifende Bibliotheksentwicklungskonzepte\footnote{dbv:
  Bibliotheksentwicklungspläne. URL:
  \url{https://www.bibliotheksverband.de/entwicklungsplaene} (Letzter
  Aufruf: 1.12.2024)} beziehen.

Da die Öffentlichen Bibliotheken in kommunaler Trägerschaft sind,
bedeutet dies für die Digitalstrategie, dass eine Einbindung in die
strategische (mittel- und langfristige) Entwicklung des Trägers erfolgen
muss. Liegt eine entsprechende Strategie vor, so können sich
Bibliotheken auch am vorgenannten DiWa-Förderprogramm im Land
Brandenburg beteiligen. Förderfähig wären dabei zum Beispiel
IT-Anschaffungen zur Umsetzung von Maßnahmen im Kontext des Konzeptes
\enquote{Dritter Ort}, die retrospektive Aufbereitung regional
bedeutender historischer Bestände inklusive anteiliger, projektbezogener
Elemente der Bestandserhaltung, die funktionale Erweiterung der
webbasierten Vermittlungs- und Nutzungsangebote und
Qualifikationsmaßnahmen zum Aufbau notwendiger eigener Kompetenzen.

Neben dem vorgestellten Förderprogramm stehen den Bibliotheken
vielfältige regionale und überregionale Förderangebote zur Verfügung.
Auch mit Blick auf diese Fördermöglichkeiten stellt die Digitalstrategie
eine sinnvolle Kommunikationsgrundlage zur Verfügung.
Bibliotheksspezifische Informationen zu Förderoptionen bieten zum
Beispiel die Landesfachstelle für Archive und Öffentliche Bibliotheken
Brandenburg\footnote{FH Potsdam: Bibliotheksberatung. URL:
  \url{https://www.fh-potsdam.de/hochschule-karriere/organisation/assoziierte-einrichtungen/landesfachstelle-archive-und-oeffentliche-bibliotheken-brandenburg/bibliotheksberatung}
  (Letzter Aufruf: 1.12.2024)} und der dbv\footnote{dbv:
  Fördernewsletter. URL:
  \url{https://www.bibliotheksverband.de/foerdernewsletter} (Letzter
  Aufruf: 1.12.2024)}.

%autor
\begin{center}\rule{0.5\linewidth}{0.5pt}\end{center}

\textbf{Ulf Preuß} (\url{https://orcid.org/0000-0003-0626-1968}) studierte, nach einer
14-jährigen Tätigkeit in den Bereichen Personal- und Rechnungswesen
als Soldat der Bundeswehr, Bibliotheksmanagement BA und
Informationswissenschaften MA an der FH Potsdam. Seit Ende 2012 leitet
er die Koordinierungsstelle Brandenburg-digital. Damit verbunden ist die
Funktion einer Geschäftsstelle für den informellen,
kulturspartenübergreifenden Arbeitskreis Brandenburg.digital. Seit 2013
ist er darüber hinaus in Nebentätigkeit engagiert in verschiedenen
Studien- und Weiterbildungsprogrammen der FU Berlin, der HU Berlin, der
HTW Berlin, der Donau-Universität Krems (Österreich) und der FH Potsdam,
in den Themenfeldern bestandsschonende retrospektive Digitalisierung,
digitale Präsentation und digitale Archivierung.

\end{document}