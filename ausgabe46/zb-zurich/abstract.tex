\textbf{Zusammenfassung:} Für die Zentralbibliothek Zürich (ZB) hat die
Erhaltung des Bestandes eine hohe Bedeutung. Eine eigene Abteilung ist
für die Konservierung und Restaurierung der wertvollen analogen Bestände
verantwortlich. Neben Arbeiten am konkreten Objekt befasst sie sich auch
mit allgemeinen bestandserhaltenden Maßnahmen wie zum Beispiel dem
gebäudeübergreifenden Klimamonitoring. Im Rahmen der
Schutzdigitalisierung bereitet sie die Objekte auf verschiedene Arten
für das Scannen vor, so dass diese im Digitalisierungsprozess keine
Schäden nehmen. Dies gilt auch für Digitalisierungsprojekte mit anderer
Zielstellung. Ein zunehmend größeres Arbeitsfeld ist die Erhaltung der
verschiedenen digitalen ZB-Bestände, wozu neben Retrodigitalisaten auch
«born-digital»-Objekte gehören. Hierfür hat die ZB neue Infrastrukturen
aufgebaut und betreibt seit 2023 ein digitales Langzeitarchiv nach
OAIS-Standard.\\

\noindent
\textbf{Abstract:} For the Zurich Central Library (ZB), the preservation
of its collection is of great importance. A dedicated department is
responsible for the conservation and restoration of valuable analogue
materials. In addition to working on specific objects, it also focuses
on general preservation measures, such as cross-building climate
monitoring. As part of digitization projects, this department prepares
objects in various ways for scanning, ensuring that they do not incur
damage during the digitization process. An increasingly significant area
of work is the preservation of the various digital ZB collections, which
include both retro-digitized materials and \enquote{born-digital} objects. To
support this, the ZB has established new infrastructures and has been
operating a digital long-term archive based on the OAIS standard since
2023.
