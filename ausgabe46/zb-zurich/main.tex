\documentclass[a4paper,
fontsize=11pt,
%headings=small,
oneside,
numbers=noperiodatend,
parskip=half-,
bibliography=totoc,
final
]{scrartcl}

\usepackage[babel]{csquotes}
\usepackage{synttree}
\usepackage{graphicx}
\setkeys{Gin}{width=.4\textwidth} %default pics size

\graphicspath{{./plots/}}
\usepackage[ngerman]{babel}
\usepackage[T1]{fontenc}
%\usepackage{amsmath}
\usepackage[utf8x]{inputenc}
\usepackage [hyphens]{url}
\usepackage{booktabs} 
\usepackage[left=2.4cm,right=2.4cm,top=2.3cm,bottom=2cm,includeheadfoot]{geometry}
\usepackage[labelformat=empty]{caption} % option 'labelformat=empty]' to surpress adding "Abbildung 1:" or "Figure 1" before each caption / use parameter '\captionsetup{labelformat=empty}' instead to change this for just one caption
\usepackage{eurosym}
\usepackage{multirow}
\usepackage[ngerman]{varioref}
\setcapindent{1em}
\renewcommand{\labelitemi}{--}
\usepackage{paralist}
\usepackage{pdfpages}
\usepackage{lscape}
\usepackage{float}
\usepackage{acronym}
\usepackage{eurosym}
\usepackage{longtable,lscape}
\usepackage{mathpazo}
\usepackage[normalem]{ulem} %emphasize weiterhin kursiv
\usepackage[flushmargin,ragged]{footmisc} % left align footnote
\usepackage{ccicons} 
\setcapindent{0pt} % no indentation in captions
\usepackage{xurl} % Breaks URLs

%%%% fancy LIBREAS URL color 
\usepackage{xcolor}
\definecolor{libreas}{RGB}{112,0,0}

\usepackage{listings}

\urlstyle{same}  % don't use monospace font for urls

\usepackage[fleqn]{amsmath}

%adjust fontsize for part

\usepackage{sectsty}
\partfont{\large}

%Das BibTeX-Zeichen mit \BibTeX setzen:
\def\symbol#1{\char #1\relax}
\def\bsl{{\tt\symbol{'134}}}
\def\BibTeX{{\rm B\kern-.05em{\sc i\kern-.025em b}\kern-.08em
    T\kern-.1667em\lower.7ex\hbox{E}\kern-.125emX}}

\usepackage{fancyhdr}
\fancyhf{}
\pagestyle{fancyplain}
\fancyhead[R]{\thepage}

% make sure bookmarks are created eventough sections are not numbered!
% uncommend if sections are numbered (bookmarks created by default)
\makeatletter
\renewcommand\@seccntformat[1]{}
\makeatother

% typo setup
\clubpenalty = 10000
\widowpenalty = 10000
\displaywidowpenalty = 10000

\usepackage{hyperxmp}
\usepackage[colorlinks, linkcolor=black,citecolor=black, urlcolor=libreas,
breaklinks= true,bookmarks=true,bookmarksopen=true]{hyperref}
\usepackage{breakurl}

%meta
%meta

\fancyhead[L]{N. Sarad, L. Fuhrer und J. Reiling\\ %author
LIBREAS. Library Ideas, 46 (2024). % journal, issue, volume.
\href{https://doi.org/10.18452/}{\color{black}https://doi.org/10.18452/}
{}} % doi 
\fancyhead[R]{\thepage} %page number
\fancyfoot[L] {\ccLogo \ccAttribution\ \href{https://creativecommons.org/licenses/by/4.0/}{\color{black}Creative Commons BY 4.0}}  %licence
\fancyfoot[R] {ISSN: 1860-7950}

\title{\LARGE{Bestandserhaltung analog und digital. Ein Praxisbericht aus der Zentralbibliothek Zürich}}% title
\author{Nadine Sarad, Lea Fuhrer, Jesko Reiling} % author

\setcounter{page}{1}

\hypersetup{%
      pdftitle={Bestandserhaltung analog und digital. Ein Praxisbericht aus der Zentralbibliothek Zürich},
      pdfauthor={Nadine Sarad, Lea Fuhrer, Jesko Reiling},
      pdfcopyright={CC BY 4.0 International},
      pdfsubject={LIBREAS. Library Ideas, 46 (2024).},
      pdfkeywords={Bestandserhaltung, Digitalisierung, Langzeitarchivierung, preservation, digitization, long-term archiving},
      pdflicenseurl={https://creativecommons.org/licenses/by/4.0/},
      pdfurl={https://doi.org/10.18452/},
      pdfdoi={10.18452/},
      pdflang={de},
      pdfmetalang={de}
     }



\date{}
\begin{document}

\maketitle
\thispagestyle{fancyplain} 

%abstracts
\begin{abstract}
\noindent
\textbf{Zusammenfassung:} Für die Zentralbibliothek Zürich (ZB) hat die
Erhaltung des Bestandes eine hohe Bedeutung. Eine eigene Abteilung ist
für die Konservierung und Restaurierung der wertvollen analogen Bestände
verantwortlich. Neben Arbeiten am konkreten Objekt befasst sie sich auch
mit allgemeinen bestandserhaltenden Maßnahmen wie zum Beispiel dem
gebäudeübergreifenden Klimamonitoring. Im Rahmen der
Schutzdigitalisierung bereitet sie die Objekte auf verschiedene Arten
für das Scannen vor, so dass diese im Digitalisierungsprozess keine
Schäden nehmen. Dies gilt auch für Digitalisierungsprojekte mit anderer
Zielstellung. Ein zunehmend größeres Arbeitsfeld ist die Erhaltung der
verschiedenen digitalen ZB-Bestände, wozu neben Retrodigitalisaten auch
«born-digital»-Objekte gehören. Hierfür hat die ZB neue Infrastrukturen
aufgebaut und betreibt seit 2023 ein digitales Langzeitarchiv nach
OAIS-Standard. \\

\noindent
\textbf{Abstract:} For the Zurich Central Library (ZB), the preservation
of its collection is of great importance. A dedicated department is
responsible for the conservation and restoration of valuable analogue
materials. In addition to working on specific objects, it also focuses
on general preservation measures, such as cross-building climate
monitoring. As part of digitization projects, this department prepares
objects in various ways for scanning, ensuring that they do not incur
damage during the digitization process. An increasingly significant area
of work is the preservation of the various digital ZB collections, which
include both retro-digitized materials and \enquote{born-digital} objects. To
support this, the ZB has established new infrastructures and has been
operating a digital long-term archive based on the OAIS standard since
2023.
\end{abstract}

%body
\section{Allgemeine präventive Aufgaben der Abteilung
Bestandserhaltung}\label{allgemeine-pruxe4ventive-aufgaben-der-abteilung-bestandserhaltung}

Die ZB ist eine der wenigen Institutionen in der Schweiz, die über eine
eigene Abteilung für Bestandserhaltung verfügt. Diese ist für den Erhalt
der umfassenden und diversen Bestände zuständig, die in den hausinternen
mehrstöckigen Magazinen und in externen Depots aufbewahrt werden. Die
Abteilung Bestandserhaltung beschäftigt elf festangestellte
Mitarbeitende und setzt sich zusammen aus der Ausrüsterei, der
Buchbinderei sowie der Konservierung und Restaurierung von Bibliotheks-
und Archivgut. Während sich die Ausrüsterei um den Buchdurchlauf der
Bibliothek kümmert und unter anderem Neuzugänge mit Signaturen und einem
elektronischen \enquote*{Tag} versieht, beschäftigt sich die
Buchbinderei des Hauses mit kleineren Reparaturen, der Vorbereitung für
Neubindungen durch externe Dienstleister:innen und unterstützt die
Restauratorinnen zum Beispiel auch bei internen Ausstellungen.

Zu den Hauptaufgaben der sieben Buch- und Papierrestauratorinnen gehören
neben konservatorischen Maßnahmen, wie zum Beispiel der Trockenreinigung
von Objekten, auch praktische Restaurierungsarbeiten wie Einband- und
Papierrestaurierungen an den Beständen der Spezialsammlungen.

In Kooperation mit dem Gebäudemanagement etablierte die
Bestandserhaltung ein integriertes Schädlingsbekämpfungsmanagement
(Integrated Pest Management, IPM) und ein nachhaltiges und
gebäudeübergreifendes Klimamonitoring. Zusammen mit einer externen Firma
entwickelte sie einen Notfallplan zur Rettung der Dokumente im
Havariefall. Diese konservatorischen, präventiven Aspekte der
Bestandserhaltung haben in den letzten zwanzig Jahren deutlich an
Relevanz gewonnen und sich in der Zentralbibliothek etabliert. Innerhalb
eines Jahrzehnts wuchs die ehemalige kleine Buchbinderei auf fast das
Dreifache an und der Schwerpunkt verschob sich in Richtung Erhalt der
Bestände.

\section{Neueres Aufgabenfeld: Schutzdigitalisierung und
Vorbereitungsarbeiten für
Digitalisierungsprojekte}\label{neueres-aufgabenfeld-schutzdigitalisierung-und-vorbereitungsarbeiten-fuxfcr-digitalisierungsprojekte}

Die ZB digitalisiert seit zwei Jahrzehnten kontinuierlich ihre Bestände.
Die damit verbundenen Ziele sind die Erleichterung des Zugangs und der
Schutz der Originale. Für die Abteilung Bestandserhaltung ist in diesem
Zusammenhang über die Jahre ein neues Aufgabenfeld entstanden. Sie sorgt
dafür, die Dokumente für die Digitalisierung vorzubereiten und
sicherzustellen, dass diese durch die Digitalisierung keinen Schaden
erleiden. Es geht zudem darum, die Scanner und die Mitarbeitenden vor
Verunreinigungen durch Staub oder Schmutz zu schützen, weshalb das
Bibliotheksgut auch trockengereinigt wird. Im aktuellen
Google-Digitalisierungsprojekt, das die ZB seit Mitte 2023 zusammen mit
der Universitätsbibliothek Basel, der Universitätsbibliothek Bern und
der Zentral- und Hochschulbibliothek Luzern durchführt, zählt es etwa zu
den Aufgaben der Bestandserhaltung, die von der Bestandslogistik
bereitgestellten Bücher vor der Digitalisierung zu begutachten. Die
beiden verantwortlichen Restauratorinnen prüfen den allgemeinen Zustand
eines jeden Buches und ob das Buch dem umfangreichen Katalog von
Anforderungen, die Google an die Scanbarkeit von Werken anlegt, genügt.
Monatlich werden auf diese Weise 5.000 Bücher ‚verarbeitet'. In anderen
Digitalisierungsprojekten legen die Restauratorinnen fest, bis zu
welchem Öffnungswinkel Bücher geöffnet und auf den Scanner gelegt werden
können. Zudem führen sie, falls notwendig, konservatorische
Sicherungsarbeiten an den ausgewählten Büchern durch, die leichte
Schäden aufweisen, um möglichst viele Bücher digitalisieren zu können.

\pagebreak
\section{Objektschonende Digitalisierung ermöglicht Zugang zu
fragilen
Dokumenten}\label{objektschonende-digitalisierung-ermuxf6glicht-zugang-zu-fragilen-dokumenten}

Die Abteilung ist organisatorisch dem Bereich \enquote{Spezialsammlungen
/ Digitalisierung} angegliedert, in welchem auch die bestandsbesitzenden
Abteilungen Handschriften, Alte Drucke und Rara, Graphische Sammlung und
Fotoarchiv, Karten und Panoramen, Musik sowie Turicensia (das sind
Zürcher Drucke und Medien nach 1800) versammelt sind. Zum Bereich gehört
im Weiteren auch die Abteilung \enquote{Digitale Produktion und
Plattformen} und darin das Digitalisierungszentrum (Digiz) der ZB, mit
dem die Bestandserhaltung seit gut einem Jahrzehnt zunehmend stärker
zusammenarbeitet.

Der Bestandserhaltung und der Abteilung Digitale Produktion und
Plattformen ist gemein, dass sie prozessbasiert arbeiten und
abteilungsübergreifend kooperieren. Besonders der Prozess der
Eigendigitalisierung der ZB kann nur durch eine engmaschige
Projektplanung beider Abteilungen erfolgreich erfolgen. Die Prüfung
eines Objekts durch die Bestandserhaltung ist fester Bestandteil des
Digitalisierungsworkflows und erfolgt nach der Erfassung des
Digitalisierungsauftrags immer als erster Schritt. Zu den Aufgaben der
Bestandserhaltung zählen, wie oben bereits erwähnt, die Festlegung der
Öffnungswinkel der zu digitalisierenden Bücher sowie Teil- oder auch
Vollrestaurierungen. Die vorausgehende Prüfung und Behandlung durch die
Bestandserhaltung gewährleisten eine objektschonende Digitalisierung.

Die vom Digiz erstellten Digitalisate, die nach gängigen Standards der
Kulturgüterdigitalisierung produziert werden, werden auf den Plattformen
\href{http://www.e-rara.ch/}{e-rara.ch},
\href{http://e-manuscripta.ch}{e-manuscripta.ch} und Zurich Open
Plattform (ZOP, erreichbar unter
\href{http://zop.zb.uzh.ch}{zop.zb.uzh.ch}) publiziert und sind so für
Benutzer:innen auf der ganzen Welt digital abruf- und nutzbar. Neben der
besseren und breiteren Nutzbarkeit der Objekte durch die Digitalisierung
werden auf diesem Weg auch die Originale geschont, da sie seltener
konsultiert werden müssen. Die Digitalisierung erlaubt es darüber
hinaus, dass fragile Werke, die für die Benutzung aus konservatorischen
Gründen eigentlich gesperrt sind, gleichwohl zugänglich gemacht werden
können. Bücher, die nur mit einem sehr kleinen Winkel geöffnet werden
dürfen, können mit Spezialscannern wie etwa dem Wolfenbütteler
Buchspiegel digitalisiert werden und werden so in digitaler Form
einsehbar gemacht.

\section{Erhaltung der digitalen
Bestände}\label{erhaltung-der-digitalen-bestuxe4nde}

Durch die qualitativ hochwertige Digitalisierung erhält die ZB eine
digitale Sicherheitskopie der analogen Originale. Mit der
fortschreitenden Digitalisierung stellt sich auch die Frage, wie die
Erhaltung der stetig wachsenden digitalen Bestände sichergestellt werden
kann. Seit 2023 betreibt die ZB deshalb ein eigenes digitales
Langzeitarchiv nach dem Standard des Open Archival Information System
(OAIS) in Zusammenarbeit mit der Firma docuteam AG (Softwarelösung
\enquote{docuteam cosmos}). Ziel ist es, alle digitalen Dokumente, die
von den Abteilungen der Spezialsammlungen als archivwürdig bewertet
werden, in das digitale Langzeitarchiv zu überführen und so langfristig
zu erhalten. Explizit von der Archivierung ausgeschlossen sind zum
aktuellen Zeitpunkt kommerzielle oder frei zugängliche digitale Inhalte
(insbesondere Streamingangebote), für welche die ZB lediglich
Zugriffsrechte besitzt, sowie Inhalte wie E-Books und E-Journals, bei
denen die Zentralbibliothek ihre Archivierungsrechte im Rahmen der
Beteiligung an internationalen Kooperationen wahrnimmt (LOCKSS,
Portico).

Seit Oktober 2023 werden alle neu veröffentlichten Digitalisate auf den
Plattformen e-rara.ch und e-manuscripta.ch direkt im neuen digitalen
Langzeitarchiv gesichert. Aktuell läuft zudem eine Einspeisung der
Altdaten dieser zwei Plattformen, welche seit der Aufschaltung von
e-rara im Jahr 2010 und von e-manuscripta 2013 entstanden sind. Für das
nächste Jahr ist die Implementierung des Workflows für die
Zeitungsdigitalisate der ZB, die auf der Plattform
\href{http://e-newspaperarchives.ch}{e-newspaperarchives.ch} publiziert
werden, sowie die Anbindung von ZOP an das digitale Langzeitarchiv
geplant.

Neben dieser großen und stetig wachsenden Menge an Retrodigitalisaten
bewahrt die ZB in ihren Sammlungen auch sogenannte
\enquote{born-digital} Bestände auf. Diese Bestände, die bereits
ursprünglich in digitaler Form erzeugt worden sind, entsprechen entweder
dem Turicensia-Sammelauftrag, zum Beispiel neue Zürcher Publikationen,
die als PDF akquiriert werden, oder gelangen durch Archivübernahmen in
die Spezialsammlungen der Zentralbibliothek. Insgesamt verwaltet die ZB
aktuell circa 250 TB an potenziell archivwürdigen Daten.

\section{Zukünftige Herausforderungen bei der Bewahrung des
digitalen
Bestandes}\label{zukuxfcnftige-herausforderungen-bei-der-bewahrung-des-digitalen-bestandes}

Eine Herausforderung für die digitale Langzeitarchivierung stellen
insbesondere die \enquote{born-digital} Bestände in den Vor- und
Nachlässen der Spezialsammlungen dar. Obwohl es sich im Vergleich zu den
Retrodigitalisaten um eine überschaubare Datenmenge handelt, gibt es
hier eine Vielzahl von Dateiformaten. Häufig ist auch der Status der
Dateien nicht eindeutig geklärt, weshalb für diese Bestände eine
aufwändige Bewertung und Erschließung auf Einzeldokumentstufe nötig sein
wird. Bei diesen Datenbeständen liegt der Fokus daher aktuell auf
Abklärungs- und Vorbereitungsarbeiten. Unter anderem wurde ein
Zwischenarchiv für die digitale Langzeitarchivierung eingerichtet,
welches aus einer gesicherten Serverinfrastruktur mit einem regelmäßigen
Back-up besteht. Die Daten, die vorgängig an den diversen Speicherorten
und auf den verschiedensten Speichermedien gesichert waren, werden
zentral, nach Abteilungen geordnet, im Zwischenarchiv gesichert. Das
Zwischenarchiv bildet die Voraussetzung, um auch für all diese
Datenbestände eine möglichst effiziente Lösung für die digitale
Langzeitarchivierung zu erarbeiten und verhindert einen Datenverlust
infolge von veralteten oder beschädigten Datenträgern.

Eine weitere Herausforderung für die Bestandserhaltung sind die in der
ZB aufbewahrten audiovisuellen Medien. Video- und Audiobestände können
sowohl konservatorisch bedroht als auch von technologischer Obsoleszenz
betroffen sein. In solchen Fällen stellt die Digitalisierung häufig die
beste Möglichkeit dar, dieses Kulturgut für zukünftige Generationen zu
sichern. Aus diesem Grund startet die ZB ab 2025 ein mehrjähriges
Digitalisierungsprojekt, welches die digitale Sicherung der unikalen
audiovisuellen Medien in den Beständen der Spezialsammlungen zum Ziel
hat.

Die Digitalisierung nach hohen Qualitätsstandards ist hier Bestandteil
einer umfassenden Bestandserhaltungsstrategie der ZB. Da die Abteilung
Bestandserhaltung mit diesen Medientypen aktuell noch keine Expertise
besitzt, wird die ZB bei diesen Medien mit externen Dienstleister:innen
zusammenarbeiten.

%autor
\begin{center}\rule{0.5\linewidth}{0.5pt}\end{center}

\textbf{Nadine Sarad} (M.A.) ist Restauratorin und seit Anfang des
Jahres Abteilungsleitung der Bestandserhaltung in der Zentralbibliothek
Zürich. Sie studierte Restaurierung und Buchgeschichte in London
(Großbritannien) und erwarb ihre praktischen Erfahrungen hauptsächlich
als selbständige Restauratorin in Deutschland. Weiterhin arbeitete sie
viele Jahre als Projektrestauratorin, Werkstattkoordinatorin und zuletzt
als Teamleitung im Sachgebiet Bestandserhaltung im Historischen Archiv
Köln.

\textbf{Lea Fuhrer} (M.A.) ist wissenschaftliche Mitarbeiterin in der
Abteilung Digitale Produktion und Plattformen und für die digitale
Langzeitarchivierung bei der Zentralbibliothek Zürich verantwortlich.
Sie hat Deutsche Sprach- und Literaturwissenschaft und Kulturanalyse an
der Universität Zürich studiert. Von 2017 bis 2023 war sie bei der
Fotostiftung Schweiz als wissenschaftliche Mitarbeiterin tätig. Daneben
hat sie berufsbegleitend den MAS ALIS (Master of Advanced Studies in
Archival, Library and Information Science) an den Universitäten Bern und
Lausanne absolviert.

\textbf{Jesko Reiling} (PD Dr.) leitet seit mehreren Jahren die
Abteilung Digitale Produktion und Plattformen. Nach dem
Germanistik-Studium an der Universität Zürich arbeitete er viele Jahre
als Dozent an der Universität Bern und habilitierte sich 2018 an der
Universität Fribourg. Danach war er in verschiedenen Forschungs- und
Editionsprojekten tätig.

\end{document}