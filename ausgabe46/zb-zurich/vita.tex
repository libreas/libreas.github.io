\begin{center}\rule{0.5\linewidth}{0.5pt}\end{center}

\textbf{Nadine Sarad} (M.A.) ist Restauratorin und seit Anfang des
Jahres Abteilungsleitung der Bestandserhaltung in der Zentralbibliothek
Zürich. Sie studierte Restaurierung und Buchgeschichte in London
(Großbritannien) und erwarb ihre praktischen Erfahrungen hauptsächlich
als selbständige Restauratorin in Deutschland. Weiterhin arbeitete sie
viele Jahre als Projektrestauratorin, Werkstattkoordinatorin und zuletzt
als Teamleitung im Sachgebiet Bestandserhaltung im Historischen Archiv
Köln.

\textbf{Lea Fuhrer} (M.A.) ist wissenschaftliche Mitarbeiterin in der
Abteilung Digitale Produktion und Plattformen und für die digitale
Langzeitarchivierung bei der Zentralbibliothek Zürich verantwortlich.
Sie hat Deutsche Sprach- und Literaturwissenschaft und Kulturanalyse an
der Universität Zürich studiert. Von 2017 bis 2023 war sie bei der
Fotostiftung Schweiz als wissenschaftliche Mitarbeiterin tätig. Daneben
hat sie berufsbegleitend den MAS ALIS (Master of Advanced Studies in
Archival, Library and Information Science) an den Universitäten Bern und
Lausanne absolviert.

\textbf{Jesko Reiling} (PD Dr.) leitet seit mehreren Jahren die
Abteilung Digitale Produktion und Plattformen. Nach dem
Germanistik-Studium an der Universität Zürich arbeitete er viele Jahre
als Dozent an der Universität Bern und habilitierte sich 2018 an der
Universität Fribourg. Danach war er in verschiedenen Forschungs- und
Editionsprojekten tätig.
