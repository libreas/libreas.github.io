\documentclass[a4paper,
fontsize=11pt,
%headings=small,
oneside,
numbers=noperiodatend,
parskip=half-,
bibliography=totoc,
final
]{scrartcl}

\usepackage[babel]{csquotes}
\usepackage{synttree}
\usepackage{graphicx}
\setkeys{Gin}{width=.4\textwidth} %default pics size

\graphicspath{{./plots/}}
\usepackage[ngerman]{babel}
\usepackage[T1]{fontenc}
%\usepackage{amsmath}
\usepackage[utf8x]{inputenc}
\usepackage [hyphens]{url}
\usepackage{booktabs} 
\usepackage[left=2.4cm,right=2.4cm,top=2.3cm,bottom=2cm,includeheadfoot]{geometry}
\usepackage[labelformat=empty]{caption} % option 'labelformat=empty]' to surpress adding "Abbildung 1:" or "Figure 1" before each caption / use parameter '\captionsetup{labelformat=empty}' instead to change this for just one caption
\usepackage{eurosym}
\usepackage{multirow}
\usepackage[ngerman]{varioref}
\setcapindent{1em}
\renewcommand{\labelitemi}{--}
\usepackage{paralist}
\usepackage{pdfpages}
\usepackage{lscape}
\usepackage{float}
\usepackage{acronym}
\usepackage{eurosym}
\usepackage{longtable,lscape}
\usepackage{mathpazo}
\usepackage[normalem]{ulem} %emphasize weiterhin kursiv
\usepackage[flushmargin,ragged]{footmisc} % left align footnote
\usepackage{ccicons} 
\setcapindent{0pt} % no indentation in captions
\usepackage{xurl} % Breaks URLs

%%%% fancy LIBREAS URL color 
\usepackage{xcolor}
\definecolor{libreas}{RGB}{112,0,0}

\usepackage{listings}

\urlstyle{same}  % don't use monospace font for urls

\usepackage[fleqn]{amsmath}

%adjust fontsize for part

\usepackage{sectsty}
\partfont{\large}

%Das BibTeX-Zeichen mit \BibTeX setzen:
\def\symbol#1{\char #1\relax}
\def\bsl{{\tt\symbol{'134}}}
\def\BibTeX{{\rm B\kern-.05em{\sc i\kern-.025em b}\kern-.08em
    T\kern-.1667em\lower.7ex\hbox{E}\kern-.125emX}}

\usepackage{fancyhdr}
\fancyhf{}
\pagestyle{fancyplain}
\fancyhead[R]{\thepage}

% make sure bookmarks are created eventough sections are not numbered!
% uncommend if sections are numbered (bookmarks created by default)
\makeatletter
\renewcommand\@seccntformat[1]{}
\makeatother

% typo setup
\clubpenalty = 10000
\widowpenalty = 10000
\displaywidowpenalty = 10000

\usepackage{hyperxmp}
\usepackage[colorlinks, linkcolor=black,citecolor=black, urlcolor=libreas,
breaklinks= true,bookmarks=true,bookmarksopen=true]{hyperref}
\usepackage{breakurl}

%meta
%meta

\fancyhead[L]{E. Euler\\ %author
LIBREAS. Library Ideas, 46 (2024). % journal, issue, volume.
\href{https://doi.org/10.18452/31226}{\color{black}https://doi.org/10.18452/31226}
{}} % doi 
\fancyhead[R]{\thepage} %page number
\fancyfoot[L] {\ccLogo \ccAttribution\ \href{https://creativecommons.org/licenses/by/4.0/}{\color{black}Creative Commons BY 4.0}}  %licence
\fancyfoot[R] {ISSN: 1860-7950}

\title{\LARGE{VG WORT und Open Access: Ein Balanceakt für Wissenschaftler*innen}}% title
\author{Ellen Euler} % author

\setcounter{page}{1}

\hypersetup{%
      pdftitle={VG WORT und Open Access: Ein Balanceakt für Wissenschaftler*innen},
      pdfauthor={Ellen Euler},
      pdfcopyright={CC BY 4.0 International},
      pdfsubject={LIBREAS. Library Ideas, 46 (2024).},
      pdfkeywords={VG Wort, Open Access, Wissenschaft, Verwertungsgesellschaft, Vergütung, Urheberrecht, scholarly communication, copyright, royalties},
      pdflicenseurl={https://creativecommons.org/licenses/by/4.0/},
      pdfurl={https://doi.org/10.18452/31226},
      pdfdoi={10.18452/31226},
      pdflang={de},
      pdfmetalang={de}
     }



\date{}
\begin{document}

\maketitle
\thispagestyle{fancyplain} 

%abstracts
\begin{abstract}
\noindent
\textbf{Zusammenfassung:} Der folgende Beitrag geht der Frage nach, wie
sich die Teilnahme an Ausschüttungen durch die VG WORT mit den
Prinzipien von Open Access und der Wissenschaftsethik in Einklang
bringen lässt und beleuchtet das Spannungsverhältnis zwischen der
Förderung des Gemeinwohls durch freien Zugang zu Wissen und der
Wahrnehmung von gesetzlichen Vergütungsansprüchen, insbesondere im
Kontext öffentlich finanzierter Wissenschaft.
\end{abstract}

%body
\section{Autorschaft ist nicht gleich
Autorschaft}\label{autorschaft-ist-nicht-gleich-autorschaft}

Autorschaft ist vielfältig und wird durch unterschiedliche Motivationen
geprägt, die stark von der Art der Texte und dem beruflichen Kontext der
Autorinnen abhängen. Autorinnen können literarische, journalistische,
dramatische oder wissenschaftliche Texte verfassen, die in
unterschiedlichen Kontexten -- analog oder digital -- verbreitet werden.
Die Motivation hinter der Veröffentlichung solcher Werke kann stark
variieren: Für manche steht der Wunsch im Vordergrund, Wissen zu teilen
und eine breitere Öffentlichkeit oder Fachcommunity zu erreichen. Für
andere sind finanzielle Vergütungen zentral, um den Lebensunterhalt zu
sichern.

Ökonomisch betrachtet sind Autorinnen als Urheberinnen der Texte an den
Verwertungen derselben angemessen zu beteiligen (§\,11 UrhG). Während
diese Beteiligung und Vergütung für freie Schriftstellerinnen und
Journalistinnen ohne feste Anstellung und festes Gehalt, sowie für
Autorinnen von Romanen, Sachbüchern und anderen nichtwissenschaftlichen
Texten, die von den Erlösen aus Buchverkäufen und deren
(Zweit-)Verwertung leben, eine hervorgehobene Bedeutung hat, gilt das
nicht für Autorinnen, die fest in Redaktionen oder Verlagen angestellt
sind und wissenschaftliche Autorinnen, die im Rahmen einer (Fest-)
Anstellung forschen und publizieren.

Auf die Beteiligung an Einnahmen aus vergütungspflichtigen gesetzlich
erlaubten Nutzungen ihrer Texte, zum Beispiel Vergütung für die
Anfertigung einer Kopie mit einem Vervielfältigungsgerät, die nur über
eine Verwertungsgesellschaft (hier die VG WORT e.\,V., ohne Datum)
geltend gemacht werden können, sind sie erst recht nicht angewiesen. Ist
es also gerechtfertigt, dass sie an der jährlichen Ausschüttung der
Einnahmen der VG Wort teilnehmen können? Oder sollten
Wissenschaftlerinnen der VG WORT gar nicht erst als Mitglied beitreten?
Diese Frage stellt sich umso mehr, wenn Wissenschaftlerinnen den im
Rahmen der öffentlich finanzierten Tätigkeit entstandenen Text der
Allgemeinheit kostenfrei zugänglich gemacht haben und mittels einer
freien Lizenz umfassende einfache Nutzungsrechte eingeräumt, also im
Open Access publiziert haben und ist regelmäßig Auslöser hitziger
Debatten, so beispielsweise einer über die Mailingliste IPOA-Forum von
September 2024 (IPOA-Forum, 2024) geführten Diskussion aus Anlass eines
Beitrages bei irights.info (Wiese \& Lange, 2024), der die Vereinbarkeit
der Mitgliedschaft von wissenschaftlichen Autor*innen in der VG Wort mit
der freien Lizenzierbarkeit, zum Beispiel mit Creative-Commons-Lizenzen
herausgestellt hat.

\section{VG Wort und ihre Rolle in der
Wissenschaft}\label{vg-wort-und-ihre-rolle-in-der-wissenschaft}

Zunächst kurz zur Rolle der VG Wort in der Wissenschaft: Die VG Wort ist
ein rechtsfähiger Verein kraft staatlicher Verleihung, in dem sich
Wortautorinnen und deren Verlegerinnen zur gemeinsamen Verwertung von
Urheberrechten zusammengeschlossen haben. Sie nimmt als einzige
Verwertungsgesellschaft in Deutschland die ihr vertraglich anvertrauten
urheberrechtlichen Befugnisse von Wortautorinnen und deren Verlegerinnen
wahr (BGH, 2024). Im Wissenschaftsbereich kommt ihr vor allem die
Aufgabe zu, Vergütungen für die gesetzlich erlaubten Nutzungen für
Unterricht, Wissenschaft und Institutionen geltend zu machen (§\,60h Abs.\,4 UrhG). Die Vergütungen werden nach einem definierten Schlüssel auf die
wahrnehmungsberechtigten Autorinnen ausgeschüttet, soweit diese die VG
WORT zur Wahrnehmung der gesetzlichen Vergütungsansprüche und der
Geräte- und Speichermedienvergütung mittels eines Wahrnehmungsvertrages
(WV VG WORT, 2024) beauftragt haben.

Verwertungsgesellschaften haben aber nicht nur eine ökonomische, sondern
auch kulturfördernde und soziale Funktion. Sie vertreten zwar die
Interessen ihrer Mitglieder, sollen jedoch nicht einseitig wie
Berufsverbände agieren. Stattdessen sind sie verpflichtet, die Rechte
ausgleichend wahrzunehmen und ein Gleichgewicht zwischen den
Bedürfnissen der Urheberinnen und der Gesellschaft herzustellen. Zur
Förderung der Wissenschaft beteiligt die VG Wort Herausgeberinnen und
den Förderungsfonds Wissenschaft, dessen einzige Gesellschafterin sie
ist, entsprechend den Bestimmungen ihres Verteilungsplans und ihrer
Satzung an den Einnahmen aus den gesetzlichen Vergütungsansprüchen der
Urheberinnen. Der Förderungsfonds Wissenschaft vergibt
Druckkostenzuschüsse für das Erscheinen wissenschaftlicher Werke und
Fachwerke, Zuschüsse für Forschungen, aus denen wissenschaftliche Werke
oder Fachwerke hervorgehen sollen, sowie Zuschüsse für sonstige
Maßnahmen zur Förderung des wissenschaftlichen Schrifttums und des
Fachschrifttums (Satzung Förderungsfonds Wissenschaft, 2024).

Die Stärkung der Rechteinhaberschaft wissenschaftlicher Autorinnen kann
in einem Widerspruch zu den erforderlichen Nutzungsfreiheiten in der
Wissenschaft stehen. Das zeigte sich insbesondere im Ringen um die
Entfristung von §\,52a UrhG, nach dem urheberrechtlich geschützte Inhalte
unter bestimmten Voraussetzungen für Unterrichts- und Forschungszwecke
einem bestimmt abgegrenzten Personenkreis öffentlich zugänglich gemacht
werden durften, etwa indem sie in schulische oder universitäre Intranets
eingestellt werden. Somit ist fraglich ist, ob die Ziele von Open Access
dadurch konterkariert werden, dass Verwertungsgesellschaften an
wissenschaftliche Autorinnen Einnahmen für gesetzlich erlaubte Nutzungen
auch dann ausschütten, wenn es sich um eine Open-Access-Veröffentlichung
handelt.

\section{Zwischen Gewinn- und
Gemeinwohlinteressen}\label{zwischen-gewinn--und-gemeinwohlinteressen}

Die Open-Access-Bewegung verfolgt das Ziel, öffentlich finanzierte
Forschung für alle frei zugänglich zu machen. Dieses Gemeinwohlinteresse
steht jedoch in einem Spannungsfeld mit den ökonomischen Interessen von
Verlagen, Autorinnen und den gesetzlichen Vergütungsregelungen.

Nach der Berliner Erklärung über den offenen Zugang zu
wissenschaftlichem Wissen aus 2003 soll öffentlich finanziertes Wissen
der Öffentlichkeit umfassend (kosten- und barrierefrei) zur Verfügung
stehen. Der Begriff umfasst hier die Ermöglichung weitergehender
Nutzungsfreiheiten, die über die gesetzlich erlaubten Nutzungen aus
Abschnitt 6 des Urheberrechtsgesetzes hinausgehen.

Das rechtliche Instrument für die weitergehende Einräumung von
Nutzungsfreiheiten ist die freie Lizenzierung (unentgeltliche Einräumung
einfacher Nutzungsrechte für die allgemeine Öffentlichkeit gem. §\,32
Abs.\,3 S.\,3 UrhG). Wissenschaftliche Autorinnen machen hiervon im
Rahmen der ihnen grundrechtlich eingeräumten Freiheiten, darüber zu
entscheiden, ob, wie und wo sie ihre Ergebnisse öffentlich zugänglich
machen (Art.\,5 Abs.\,3 GG), eingeschränkt Gebrauch: in manchen
Wissenschaftsdisziplinen fast umfassend, in anderen, insbesondere der
Rechtswissenschaft, äußerst zurückhaltend. Ob sich hieran etwas ändert,
seit der Kodex zur guten wissenschaftlichen Praxis der Deutschen
Forschungsgemeinschaft (DFG) von 2019 bis Ende Juli 2023
rechtsverbindlich an den Hochschulen in Deutschland, die weiterhin
Drittmittel von der DFG erhalten können wollen, umzusetzen war, bleibt
abzuwarten. Leitlinie 13 des Kodex betont zwar die Publikationsfreiheit,
es wird in der Erläuterung jedoch der Erwartung Ausdruck verliehen, dass
die Publikation \enquote{wann immer möglich}, den
\textbf{FAIR}-Prinzipien (\enquote{\textbf{F}indable,
\textbf{A}ccessible, \textbf{I}nteroperable, \textbf{R}eusable})
entsprechend in anerkannten Archiven und Repositorien zugänglich gemacht
wird. Durch die weiche Formulierung \enquote{\emph{unter
Berücksichtigung der Gepflogenheiten des betroffenen Fachgebiets}}\,''
werden sich zurückhaltende Disziplinen voraussichtlich auch weiter
zurückhaltend verhalten können.

Die Zurückhaltung in der Rechtswissenschaft ist nicht zuletzt darauf
zurückzuführen, dass sie zu den Disziplinen zählt, in denen sich mit
Publikationen (selbstständigen Monografien ebenso wie unselbstständigen
Periodika) Geld verdienen lässt. Möglich ist das, weil und solange die
Publikation nicht nur kostendeckend ist, sondern Gewinne produziert, an
denen die Autorinnen beteiligt werden können. In der Regel geschieht
dies durch Verkauf von Zugängen oder gedruckten Exemplaren. Zugang hat
dann nur, wer sich diesen leisten kann, oder Zugang zu einer Bibliothek
hat, die diesen finanziert.

Die Open-Access-Bewegung will unter anderem diese strukturellen
Barrieren überwinden und zielt auf einen kostenfreien Zugang für die
breite Öffentlichkeit als Minimalkonsens ab. Wenn jedoch keine Zugänge
mehr verkauft werden können, dann müssen andere Modelle zur Finanzierung
der Publikationskosten gefunden werden. Eine Vielzahl von
Transformationsansätzen für die Finanzierung der Publikationskosten hat
sich mittlerweile etabliert. Den Transformationsmodellen unter
Einbeziehung kommerzieller Verlage ist gemein, dass diese immer auch die
Gewinne der Verlagsseite mitfinanzieren. Je mehr Gewinne jedoch für die
Open-Access-Transformation mitfinanziert werden müssen, umso schwieriger
gelingt diese. Individuelle Gewinn- und Gemeinwohlinteressen stehen hier
in einem Spannungsverhältnis.

Es lässt sich hier argumentieren, dass zwar nichts dagegen spricht, wenn
Verlage oder/und wissenschaftliche Autorinnen Einnahmen aus
Publikationen erzielen, das vorrangige Ziel der Wissenschaft -- die
freie Verbreitung und Anknüpfbarkeit von Wissen -- muss jedoch im
Vordergrund stehen und sichergestellt werden und die wissenschaftlichen
sowie ethischen Standards dürfen nicht untergraben werden.

Erlöse erzielen Wissenschaftlerinnen für ihre Publikationen auch aus der
Teilnahme an Ausschüttungen für gesetzlich erlaubte Nutzungen -- selbst
dann, wenn sie diese frei lizenziert haben. Fraglich ist, wie das
rechtlich und wissenschaftsethisch zu bewerten ist.

\section{Vereinbarkeit freie Lizenzierung und VG WORT
Mitgliedschaft}\label{vereinbarkeit-freie-lizenzierung-und-vg-wort-mitgliedschaft}

Die Voraussetzungen für die rechtliche Vereinbarkeit der freien
Lizenzierung mit dem Abschluss des Wahrnehmungsvertrag der VG WORT (WV
VG WORT, 2024) wurden bereits in einem Praxisbericht (Reda, 2024) und
einem Leitfaden (Wiese \& Lange, 2024) thematisiert und im Ergebnis
bejaht. Für Autorinnen mit Wahrnehmungsvertrag mit der VG WORT bleibt
festzuhalten, dass Werke unter freier Lizenzierung in allen Varianten
(mit und ohne Vorbehalt gegen kommerzielle Nutzungen) bereitgestellt
werden können, ohne dass der Anspruch auf eine Vergütung im Bereich der
gesetzlichen Vergütungsansprüche verloren geht. Der Wahrnehmungsvertrag
mit der VG WORT steht hierzu praktisch nicht in Konflikt. Die VG WORT
ist sich der im Wissenschaftsbereich praktizierten Lizenzierungen ohne
Vorbehalt gegen kommerzielle Nutzungen bewusst und akzeptiert diese ohne
Einschränkungen für durch sie wahrgenommene \emph{gesetzliche
Vergütungsansprüche}, auf die sich freie Lizenzen gerade nicht beziehen.
Rechtstheoretische Konkordanz auch im Hinblick auf \emph{vertragliche
Vergütungsansprüche} und laut Wahrnehmungsvertrag ausschließlich
eingeräumte Rechte lässt sich durch Ausnehmung dieser Rechte von der
Wahrnehmung (gem. §\,13 WV VG WORT, 2024) herstellen. Selbst wenn die
Wahrnehmung vertraglicher Nutzungen ausgenommen wird, besteht nach dem
WV VG WORT weiterhin ein Anspruch auf Teilnahme an den Ausschüttungen
für gesetzlich erlaubte Nutzungen (siehe §\,4 S.\,3 WV VG WORT, 2024).

In der Teilnahme an den jährlichen Ausschüttungen durch
wissenschaftliche Autorinnen, deren Publikationen im Rahmen einer
öffentlich finanzierten Tätigkeit entstehen, kann losgelöst von der
rechtlichen Zulässigkeit ein Widerspruch gesehen werden. Begründen lässt
sich diese Ansicht damit, dass hierbei eine Mehrfachvergütung geschehe.
So seien die Beiträge schon über die Tätigkeit öffentlich finanziert und
wenn gegebenenfalls Open-Access-Publikationsgebühren öffentlich
finanziert würden, sowie zusätzlich Einnahmen für gesetzliche Nutzungen
aus öffentlichen Haushalten über die VG WORT ausgeschüttet würden, dann
sei das wissenschaftsethisch bedenklich. Sieht man sich die einzelnen
Kosten an, dann lässt sich nachvollziehbar vertreten, dass mit dem
Gehalt für wissenschaftliche Autorinnen neben der Forschung auch die
Publikation derselben abgegolten ist. Die Infrastruktur (zum Beispiel
IT, Mitarbeitende, Labor etc.) dafür wird ebenso durch den Arbeitgeber
bereitgestellt, wie die Arbeitszeit vergütet wird.

Diese rein gemeinwohlorientierte Betrachtung ist nachvollziehbar,
versäumt es jedoch, die individuelle Perspektive angemessen zu
berücksichtigen, und stellt diese somit in ein unauflösbares
Spannungsverhältnis, das es aber gerade aufzulösen gilt, wenn
wissenschaftliche Autorinnen die Open-Access-Transformation unterstützen
sollen.

Da Wissenschaftlerinnen selten als Autorinnen einer bestimmten Anzahl an
Publikationen oder überhaupt für die Publikation, sondern abstrakt für
die Tätigkeit in Wissenschaft, Forschung und Lehre eingestellt werden,
schulden sie keine konkrete Publikation und ihre grundrechtlich
gewährleistete Freiheit beinhaltet nicht nur die freie Entscheidung
\enquote{ob}, sondern auch \enquote{wie} publiziert wird. Ob
Wissenschaftlerinnen durch Satzung der Anstellungskörperschaft
verpflichtet werden können, zusätzlich zur freien Entscheidung über
einen Verlag zu veröffentlichen, dieselbe Publikation auch im
Repositorium der Einrichtung nach einem Embargo von 12\,Monaten
öffentlich zugänglich zu machen, ist (für Deutschland) durch das
Bundesverfassungsgericht noch zu entscheiden und gegenwärtig noch offen.
(Fischer, 2023)

Nehmen wir einmal vom Gedanken Abstand, dass Wissenschaftlerinnen qua
Satzung oder anderer rechtlicher Vorgaben verpflichtet werden müssen,
ihre Ergebnisse Open Access zu veröffentlichen und fragen uns
stattdessen, was sie in Anerkennung ihrer grundrechtlich gewährleisten
Freiheiten in Wissenschaft, Forschung und Lehre sowie Eigentumsgarantie
dazu motivieren könnte, dann müssen wir die individuelle Perspektive und
Motivationslage verstehen, die für oder gegen die kosten- und
barrierefreie Open-Access-Publikation spricht. Hier müssen wir
feststellen, dass diese in den unterschiedlichen Wissenschaftsbereichen
und angesichts der unterschiedlichen Gattungen von Publikationsformaten
sehr verschieden ist.

Reputationslogiken, (nicht vorhandene offene) Infrastrukturangebote und
Publikationsservices stehen gleichbedeutend neben der individuellen
Perspektive, mit verschiedenen Publikationsformaten Gewinne erzielen zu
können. So kann zum Beispiel mit qualitativ hochwertigen, innovativen
Lehr- und Lernmaterialien fächerübergreifend Gewinn erwirtschaftet
werden und manche Disziplinen, die sich spezifisch auch an die Praxis
wenden, können mit Fach- und Handbüchern, oder um nochmal auf die
Rechtswissenschaften zu sprechen kommen, Rechtskommentaren, Einnahmen
generieren. Wissenschaftlerinnen werden nicht konkret und ausschließlich
dafür bezahlt, diese Formate zu erarbeiten, und in der Wissenschaft gibt
es häufig auch nicht die passenden Infrastrukturen und
Publikationsservices, um sie hierbei zu unterstützen. Entstehen diese
Formate ohne explizite Einräumung zeitlicher Kapazitäten innerhalb der
Arbeitszeit trotzdem, dann nicht selten durch Selbstausbeutung und in
Zusammenarbeit mit spezialisierten Verlagen, die wie oben beschrieben,
die Wissenschaftlerinnen an ihren durch Verkaufserlöse erzielten
Einnahmen beteiligen.

\section{Auflösung}\label{aufluxf6sung}

Wie lässt sich nun in einem kapitalistischen Akkumulationssystem das
Spannungsverhältnis von individuellen Gewinn- und Gemeinwohlinteressen
auflösen?

Bezogen auf die Ausgangsfrage, ob wissenschaftliche Autorinnen, die auf
eigene Einnahmen durch Verkauf im Fall der OA lizenzierten kostenfrei
bereitgestellten Publikation verzichten, zusätzlich auch auf die
Teilnahme an den Ausschüttungen der Einnahmen für gesetzlich erlaubte
Nutzungen durch die VG WORT verzichten sollten, ist wichtig zu
verstehen, dass die Kosten für Vergütungsansprüche aus
vergütungspflichtigen gesetzlichen Nutzungen losgelöst davon entstehen,
ob die wissenschaftliche Texte Open Access verfügbar sind oder nicht.
Zudem hat der Gesetzgeber eine Entscheidung über die
Vergütungspflichtigkeit bestimmter gesetzlicher Erlaubnisse unabhängig
davon getroffen, ob die sie geltend machenden Werkurheberinnen die
genutzten Werke im Rahmen einer öffentlich finanzierten Tätigkeit
geschaffen haben.

Wenn wissenschaftliche Autorinnen, die nicht Open Access
veröffentlichen, zusätzlich an der Ausschüttung der VG WORT teilnehmen,
sollten hiervon erst recht solche Gebrauch machen können, die Open
Access veröffentlichen. Ansonsten würde der Anteil der Ausschüttung, der
auf sie entfiele, zusätzlich auf die Ersteren aufgeteilt und diese
zusätzlich belohnt.

Erschwerend kommt hinzu, dass auch der Verlag, der gegebenenfalls schon
durch eine OA-Publikationsgebühr für entgangene Gewinne entschädigt
wurde, als wahrnehmungsberechtigte Gruppe bei der VG WORT an den
Ausschüttungen für gesetzlich erlaubte Nutzungen beteiligt wird.

Wenn wissenschaftlichen Autorinnen, die ihre Beiträge Open Access
veröffentlichen, nicht nur auf Verkaufserlöse, sondern zusätzlich auch
auf die ihnen zustehende Beteiligung an den Einnahmen für gesetzlich
erlaubte Nutzungen verzichten, würden davon in dem geltenden System die
kommerziellen Verlage und nicht transformationswilligen
Wissenschaftlerinnen profitieren. Dies würde transformationswillige
Wissenschaftlerinnen doppelt benachteiligen.

Die Entscheidung darüber, ob ein Wahrnehmungsvertrag mit der VG WORT
geschlossen wird, ist die individuelle Entscheidung einer jeden
publizierenden Wissenschaftlerin. Diese Entscheidung gerade da in Frage
zu stellen, wo diese sich für eine Open-Access-Lizenzierung entschieden
hat und damit regelmäßig dagegen optiert hat, eigene Erlöse aus der
Publikation abseits der Beteiligung an der Ausschüttung für gesetzliche
Nutzungen als Ausgleich für entgangene Einnahmemöglichkeiten im Hinblick
auf diese Nutzungen zu erzielen, lädt wissenschaftliche Autorinnen nicht
dazu ein, an der Open-Access-Transformation teilzunehmen.

Dabei ist es gerade für den Bereich Lehre und Bildung wichtig, dass
wissenschaftliche Autorinnen frei lizenzierte und damit veränderbare,
aktuelle Lehr- und Lernmaterialien bereitstellen. Wenn aber nicht nur
auf die Möglichkeit der Einnahmen für den Verkauf, sondern darüber
hinaus auch auf die Vergütungen für die Meldung bei der VG WORT
verzichtet werden soll, sinkt die Motivation weiter, diese als
Open-Access-Publikation zu veröffentlichen. Viele Autorinnen legen
nämlich großen Wert darauf, dass die freie Lizenzierung die Beteiligung
an den Ausschüttungen durch die VG WORT gerade nicht ausschließt.

Wenn auch der Ansatz der VG Wort Rechtepositionen von
(wissenschaftlichen) Urheberrinnen zu stärken und gesetzliche
Nutzungserlaubnisse mithin eng zu fassen und an eine Vergütung zu
koppeln, mit dem Ansatz möglichst umfassender Nutzungsmöglichkeiten in
der Wissenschaft in einem Spannungsverhältnis steht, so sollte dennoch
davon abgesehen werden orthodoxe Korrektheit auf Seiten der
Wissenschaftlerinnen zu erwarten, zumal diese in einem
Motivationskonflikt stecken.

Eine mögliche Auflösung dieses Motivationskonfliktes könnte darin
liegen, Anreizmodelle und Vergütungsstrukturen für Open-Access-Werke aus
der Wissenschaft zu entwickeln, die sowohl die Interessen der Autorinnen
als auch das Gemeinwohl berücksichtigen -- etwa durch eine Reform der VG
WORT, die speziell auf die Besonderheiten der Transformation des
Wissenschaftsbereichs eingeht und an transformationswillige Autorinnen
eventuell besondere Ausschüttungen vorsieht.

Auch Initiativen, die nicht mehr nur das Endprodukt, die OA-Publikation,
in den Blick nehmen, sondern vielmehr den Weg dahin innovieren und
kollaborative, egalitäre, diverse und gemeinschaftliche Autorinnenschaft
in den Vordergrund stellen (siehe hierzu zum Beispiel
OpenRewi e.\,V., \url{https://openrewi.org/}), die durch innovative
Publikationsservices unterstützt werden, können innerhalb des
existierenden Systems diziplinbezogen zu einer Auflösung des
Spannungsverhältnisses individueller Reputations- und Gewinn-, sowie
gemeinwohlorientierter Publikationsinteressen beitragen. Der vorliegende
Beitrag versteht sich insoweit als ein weiterer Denkanstoß in diese
Richtung.

\section{Fazit}\label{fazit}

Die Frage, ob Wissenschaftler*innen an den Ausschüttungen gesetzlicher
Vergütungsansprüche der VG Wort teilnehmen sollten, wenn und obwohl sie
ihre Werke unter freien Lizenzen veröffentlichen, bleibt ein
umstrittenes Thema im Spannungsfeld zwischen Gemeinwohlinteressen und
individuellen Vergütungsansprüchen. Einerseits sollen öffentlich
finanzierte Publikationen der Allgemeinheit nicht nur kostenfrei
zugänglich, sondern auch kostenfrei nutzbar zur Verfügung stehen,
zumindest in dem gesetzlich erlaubten Umfang -- ein Grundgedanke, der im
Open-Access-Prinzip und der Berliner Erklärung von 2003 verankert ist.
Andererseits steht Wissenschaftlerinnen eine finanzielle Entschädigung
für die Nutzung ihrer Werke zu, insbesondere wenn diese unter
Schrankenregelungen erfolgen, die durch die Creative-Commons-Lizenzen
nicht erfasst werden.

Solange es keine klaren, auf die besonderen Bedürfnisse der
verschiedenen Wissenschaftsdisziplinen und Werkgattungen zugeschnittenen
Anreizmechanismen gibt, welche die Open-Access-Transformationskosten in
den Blick nehmen, wird das Spannungsverhältnis zwischen individuellen
Gewinn- und Gemeinwohlinteressen bestehen bleiben. Die Teilnahme an den
Ausschüttungen der VG WORT schafft zwischenzeitlich im geltenden System
eine gewisse Balance der Anreize und gleicht den Motivationskonflikt
transformationswilliger wissenschaftlicher Autorinnen zumindest etwas
aus.

\section{Literatur}\label{literatur}

\subsection{Beiträge}\label{beitruxe4ge}

\emph{Berliner Erklärung über den offenen Zugang zu wissenschaftlichem
Wissen} (2003), \url{https://openaccess.mpg.de/Berliner-Erklaerung/}

BGH - I ZR 135/23 - Beschluss des I. Zivilsenats vom 21.11.2024

DFG (2019). \emph{Leitlinien zur Sicherung guter wissenschaftlicher
Praxis}, \url{https://perma.cc/V765-8LQT}

Fischer, Georg (2024). \emph{Zweitveröffentlichungsrecht und Causa
Konstanz: Bundesverfassungsgericht vor Entscheidung}. In: iRights info,
17.04.2023, \url{https://perma.cc/3UJV-NX8H}

IPOA-Forum (2024). \emph{ipoa-forum Archiv},
\url{https://lists.fu-berlin.de/pipermail/ipoa-forum/}

Reda, Felix (2024). \emph{Kein Widerspruch: Open Access und Vergütung
durch die VG Wort}. In. iRights info, 26.07.2024,
\url{https://perma.cc/V975-P7H9}

Satzung Förderungsfonds Wissenschaft vom 25. Januar 2024,
\url{https://perma.cc/Y6UZ-UJPP}

Wiese, Robert ; Lange, Marc (2024). \emph{Open Access und die VG Wort:
Was es bei wissenschaftlichen Texten zu beachten gilt}. In: iRights
info, 11.09.2024, \url{https://perma.cc/32MP-2BCR}

VG Wort e.\,V. (ohne Datum). \emph{Allgemeines},
\url{https://perma.cc/X789-V6JZ}

WV VG Wort (2024). \emph{Wahrnehmungsvertrag}, \url{https://perma.cc/2REX-2L3P}

\subsection{Gesetze}\label{gesetze}

GG - Grundgesetz für die Bundesrepublik Deutschland,
\url{https://www.gesetze-im-internet.de/gg/}

UrhG - Gesetz über Urheberrecht und verwandte Schutzrechte
(Urheberrechtsgesetz), \url{https://www.gesetze-im-internet.de/urhg/}

%autor
\begin{center}\rule{0.5\linewidth}{0.5pt}\end{center}

\textbf{Ellen Euler} ist Professorin für Bibliothekswissenschaft -- Open
Access/Open Data an der FH Potsdam,
\url{https://www.fh-potsdam.de/hochschule-netzwerk/personen/ellen-euler}.

\end{document}