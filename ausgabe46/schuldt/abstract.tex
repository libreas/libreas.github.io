\textbf{Zusammenfassung}: Die schweizerische Bibliotheksstatistik umfasst
seit 2021 die Daten aller öffentlich zugänglichen Bibliotheken. Diese
Daten werden allerdings kaum für Forschungen über Bibliotheken genutzt.
Im Artikel wird die Statistik und deren Datenqualität vorgestellt,
anschliessend das Dashboard \enquote{BiblioCheck}, welches eine Darstellung der
Daten für die allgemein öffentlichen Bibliotheken ermöglicht.
Hauptsächlich argumentiert der Text aber dafür, die Daten für
weitergehende Fragen zu nutzen. Anhand von Verteilungen und Vergleichen
zwischen Gruppen sowie Korrelationen führt er beispielhaft vor, was für
Aussagemöglichkeiten in den Daten noch vorhanden sind.

\begin{center}\rule{0.5\linewidth}{0.5pt}\end{center}

\textbf{Abstract}: Since 2021, the Swiss library statistic encompasses
the data of all publicly accessible libraries. However, this data is
rarely used for research on libraries. The article presents these
statistics and their data quality, followed by a discussion of the
\enquote{BiblioCheck} dashboard, which makes it possible to display the data
for public libraries. However, the text mainly argues in favor of using
the data for further questions. Using distributions and comparisons
between groups as well as correlations as examples, it demonstrates the
potential information that is still to be obtained from the data.
