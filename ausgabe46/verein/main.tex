\documentclass[a4paper,
fontsize=11pt,
%headings=small,
oneside,
numbers=noperiodatend,
parskip=half-,
bibliography=totoc,
final
]{scrartcl}

\usepackage[babel]{csquotes}
\usepackage{synttree}
\usepackage{graphicx}
\setkeys{Gin}{width=.4\textwidth} %default pics size

\graphicspath{{./plots/}}
\usepackage[ngerman]{babel}
\usepackage[T1]{fontenc}
%\usepackage{amsmath}
\usepackage[utf8x]{inputenc}
\usepackage [hyphens]{url}
\usepackage{booktabs} 
\usepackage[left=2.4cm,right=2.4cm,top=2.3cm,bottom=2cm,includeheadfoot]{geometry}
\usepackage[labelformat=empty]{caption} % option 'labelformat=empty]' to surpress adding "Abbildung 1:" or "Figure 1" before each caption / use parameter '\captionsetup{labelformat=empty}' instead to change this for just one caption
\usepackage{eurosym}
\usepackage{multirow}
\usepackage[ngerman]{varioref}
\setcapindent{1em}
\renewcommand{\labelitemi}{--}
\usepackage{paralist}
\usepackage{pdfpages}
\usepackage{lscape}
\usepackage{float}
\usepackage{acronym}
\usepackage{eurosym}
\usepackage{longtable,lscape}
\usepackage{mathpazo}
\usepackage[normalem]{ulem} %emphasize weiterhin kursiv
\usepackage[flushmargin,ragged]{footmisc} % left align footnote
\usepackage{ccicons} 
\setcapindent{0pt} % no indentation in captions
\usepackage{xurl} % Breaks URLs

%%%% fancy LIBREAS URL color 
\usepackage{xcolor}
\definecolor{libreas}{RGB}{112,0,0}

\usepackage{listings}

\urlstyle{same}  % don't use monospace font for urls

\usepackage[fleqn]{amsmath}

%adjust fontsize for part

\usepackage{sectsty}
\partfont{\large}

%Das BibTeX-Zeichen mit \BibTeX setzen:
\def\symbol#1{\char #1\relax}
\def\bsl{{\tt\symbol{'134}}}
\def\BibTeX{{\rm B\kern-.05em{\sc i\kern-.025em b}\kern-.08em
    T\kern-.1667em\lower.7ex\hbox{E}\kern-.125emX}}

\usepackage{fancyhdr}
\fancyhf{}
\pagestyle{fancyplain}
\fancyhead[R]{\thepage}

% make sure bookmarks are created eventough sections are not numbered!
% uncommend if sections are numbered (bookmarks created by default)
\makeatletter
\renewcommand\@seccntformat[1]{}
\makeatother

% typo setup
\clubpenalty = 10000
\widowpenalty = 10000
\displaywidowpenalty = 10000

\usepackage{hyperxmp}
\usepackage[colorlinks, linkcolor=black,citecolor=black, urlcolor=libreas,
breaklinks= true,bookmarks=true,bookmarksopen=true]{hyperref}
\usepackage{breakurl}

%meta

%meta

\fancyhead[L]{Vorstand LIBREAS-Verein\\ %author
LIBREAS. Library Ideas, 46 (2024). % journal, issue, volume.
\href{https://doi.org/10.18452/31228}{\color{black}https://doi.org/10.18452/31228}
{}} % doi 
\fancyhead[R]{\thepage} %page number
\fancyfoot[L] {\ccLogo \ccAttribution\ \href{https://creativecommons.org/licenses/by/4.0/}{\color{black}Creative Commons BY 4.0}}  %licence
\fancyfoot[R] {ISSN: 1860-7950}

\title{\LARGE{In eigener Sache: Bericht über die Aktivitäten des LIBREAS-Vereins 2023/2024}}% title
\author{Vorstand LIBREAS-Verein} % author

\setcounter{page}{1}

\hypersetup{%
      pdftitle={In eigener Sache: Bericht über die Aktivitäten des LIBREAS-Vereins 2023/2024},
      pdfauthor={Vorstand LIBREAS-Verein},
      pdfcopyright={CC BY 4.0 International},
      pdfsubject={LIBREAS. Library Ideas, 46 (2024)},
      pdfkeywords={Vereinsbericht, LIBREAS},
      pdflicenseurl={https://creativecommons.org/licenses/by/4.0/},
      pdfcontacturl={http://libreas.eu},
      baseurl={},
      pdflang={de},
      pdfmetalang={de},
      pdfdoi={10.18452/31228},
      pdfurl={https://doi.org/10.18452/31228tttt}
     }



\date{}
\begin{document}

\maketitle
\thispagestyle{fancyplain} 

%abstracts

%body
\emph{Vorbemerkung: Der Vorstand des LIBREAS-Vereins veröffentlicht den
Tätigkeitsbericht im Sinne der Transparenz jeweils in der auf die
Mitgliederversammlung folgenden Ausgabe der LIBREAS.Library Ideas in
gekürzter Form. Personenbeziehbare Daten werden dabei ausgelassen,
sofern nicht die ausdrückliche Zustimmung der betreffenden Person(en)
vorliegt. Ebenso werden Details ausgelassen, die das Vereinsvermögen
betreffen. Sie können durch Mitglieder des Vereins beim Vereinsvorstand
jederzeit erfragt werden beziehungsweise werden in den Protokollen der
Versammlungen aufgeschlüsselt und mit den Mitgliedern geteilt.}

\section{Berichtszeitraum}\label{berichtszeitraum}

Der Bericht bezieht sich auf den Zeitraum zwischen der
Mitgliederversammlung 2023 (15.11.2023) bis zur Mitgliederversammlung
2024 (13.11.2024).

\section{Vorstand}\label{vorstand}

Dem Vereinsvorstand gehörten im Berichtszeitraum Matti Stöhr
(Vorsitzender), Dr.\,Karsten Schuldt (stellvertretender Vorsitzender),
Jana Rumler (Schriftleiterin), Dr.\,Maxi Kindling (Finanzerin) und Ben
Kaden (Ressort LIBREAS.Library Ideas) an. Der Vorstand hat sich
regelmäßig getroffen und bei Bedarf virtuell ausgetauscht.

\section{Mitglieder}\label{mitglieder}

Der LIBREAS-Verein hatte mit Stand 28.10.2024 50\,Mitglieder. Davon
waren 47 persönliche Mitglieder sowie drei Fördermitglieder.

\section{Vereinsfinanzen}\label{vereinsfinanzen}

Die Einnahmen des LIBREAS-Vereins setzten sich im Haushaltsjahr
2023/2024 aus den Mitgliedsbeiträgen und Spenden zusammen. Ausgaben
wurden getätigt für das Hosting der Webauftritte, Kontoführungsgebühren
und die Servicepauschale für die \emph{Libraries4Future}-Website. Die
Kasse wird jährlich geprüft und das Ergebnis im Rahmen der
Mitgliederversammlung berichtet. Es gab keine Beanstandungen.

Im September 2024 hat das Finanzamt die Gemeinnützigkeit des Vereins und
damit die Freistellung von der Körperschaftssteuer auf Grundlage der
eingereichten Unterlagen für die Jahre 2021--2023 bestätigt. Die nächste
Prüfung erfolgt im Jahr 2027.

\section{LIBREAS-Ausgaben/Redaktion}\label{libreas-ausgabenredaktion}

Der Schwerpunkt der Vorstandstätigkeit, die sehr eng mit der
Redaktionstätigkeit verbunden ist, lag im Berichtszeitraum auf der
Erstellung der LIBREAS-Ausgaben Ausgaben \#44 Grassroots Open Access
(\url{https://libreas.eu/ausgabe44/}) und \#45 The Sound of Libraries
(\url{https://libreas.eu/ausgabe45/}) und der Vorbereitung der Ausgabe
\#46 \enquote{Bestandserhaltung}. Zudem wurde der Call für die Ausgabe
\#47 \enquote{Lug und Trug (im Wissenschaftssystem)} vorbereitet.\\

\section{Kommunikation}\label{kommunikation}

Der Social-Media-Kanal Mastodon
(\url{https://openbiblio.social/@libreas}) wurde bespielt. Zusätzlich
wurde in geringem Maß der LIBREAS.Instagram-Kanal
(\url{https://www.instagram.com/libreas.libraryideas/}) gepflegt.

\section{Libraries4Future}\label{libraries4future}

Der LIBREAS-Vorstand hat sich im vergangenen Jahr weiterhin mit dem
\emph{Netzwerk Grüne Bibliothek} ausgetauscht und in geringem Maß den
Mastodon-Kanal der Initiative gepflegt. Der Verein hat weiterhin die
Betreuung der Website (\url{https://libraries4future.org/}) finanziell
unterstützt. Die Initiative sucht nach wie vor dringend nach Leuten, die
sich aktiv einbringen können, um unter anderem den Mastodon-Kanal zu
betreuen.

%autor

\end{document}