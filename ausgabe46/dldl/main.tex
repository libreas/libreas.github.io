\documentclass[a4paper,
fontsize=11pt,
%headings=small,
oneside,
numbers=noperiodatend,
parskip=half-,
bibliography=totoc,
final
]{scrartcl}

\usepackage[babel]{csquotes}
\usepackage{synttree}
\usepackage{graphicx}
\setkeys{Gin}{width=.4\textwidth} %default pics size

\graphicspath{{./plots/}}
\usepackage[ngerman]{babel}
\usepackage[T1]{fontenc}
%\usepackage{amsmath}
\usepackage[utf8x]{inputenc}
\usepackage [hyphens]{url}
\usepackage{booktabs} 
\usepackage[left=2.4cm,right=2.4cm,top=2.3cm,bottom=2cm,includeheadfoot]{geometry}
\usepackage[labelformat=empty]{caption} % option 'labelformat=empty]' to surpress adding "Abbildung 1:" or "Figure 1" before each caption / use parameter '\captionsetup{labelformat=empty}' instead to change this for just one caption
\usepackage{eurosym}
\usepackage{multirow}
\usepackage[ngerman]{varioref}
\setcapindent{1em}
\renewcommand{\labelitemi}{--}
\usepackage{paralist}
\usepackage{pdfpages}
\usepackage{lscape}
\usepackage{float}
\usepackage{acronym}
\usepackage{eurosym}
\usepackage{longtable,lscape}
\usepackage{mathpazo}
\usepackage[normalem]{ulem} %emphasize weiterhin kursiv
\usepackage[flushmargin,ragged]{footmisc} % left align footnote
\usepackage{ccicons} 
\setcapindent{0pt} % no indentation in captions
\usepackage{xurl} % Breaks URLs

%%%% fancy LIBREAS URL color 
\usepackage{xcolor}
\definecolor{libreas}{RGB}{112,0,0}

\usepackage{listings}

\urlstyle{same}  % don't use monospace font for urls

\usepackage[fleqn]{amsmath}

%adjust fontsize for part

\usepackage{sectsty}
\partfont{\large}

%Das BibTeX-Zeichen mit \BibTeX setzen:
\def\symbol#1{\char #1\relax}
\def\bsl{{\tt\symbol{'134}}}
\def\BibTeX{{\rm B\kern-.05em{\sc i\kern-.025em b}\kern-.08em
    T\kern-.1667em\lower.7ex\hbox{E}\kern-.125emX}}

\usepackage{fancyhdr}
\fancyhf{}
\pagestyle{fancyplain}
\fancyhead[R]{\thepage}

% make sure bookmarks are created eventough sections are not numbered!
% uncommend if sections are numbered (bookmarks created by default)
\makeatletter
\renewcommand\@seccntformat[1]{}
\makeatother

% typo setup
\clubpenalty = 10000
\widowpenalty = 10000
\displaywidowpenalty = 10000

\usepackage{hyperxmp}
\usepackage[colorlinks, linkcolor=black,citecolor=black, urlcolor=libreas,
breaklinks= true,bookmarks=true,bookmarksopen=true]{hyperref}
\usepackage{breakurl}

%meta

%meta

\fancyhead[L]{Redaktion LIBREAS\\ %author
LIBREAS. Library Ideas, 46 (2024). % journal, issue, volume.
\href{https://doi.org/10.18452/31229}{\color{black}https://doi.org/10.18452/31229}
{}} % doi 
\fancyhead[R]{\thepage} %page number
\fancyfoot[L] {\ccLogo \ccAttribution\ \href{https://creativecommons.org/licenses/by/4.0/}{\color{black}Creative Commons BY 4.0}}  %licence
\fancyfoot[R] {ISSN: 1860-7950}

\title{\LARGE{Das liest die LIBREAS, Nummer \#15 (Herbst-Winter 2024)}}% title
\author{Redaktion LIBREAS} % author

\setcounter{page}{1}

\hypersetup{%
      pdftitle={Das liest die LIBREAS, Nummer \#15 (Herbst-Winter 2024)},
      pdfauthor={Redaktion LIBREAS},
      pdfcopyright={CC BY 4.0 International},
      pdfsubject={LIBREAS. Library Ideas, 46 (2024)},
      pdfkeywords={Literaturübersicht, Bibliothekswissenschaft, Informationswissenschaft, Bibliothekswesen, Rezension, literature overview, library science, information science, library sector, review},
      pdflicenseurl={https://creativecommons.org/licenses/by/4.0/},
      pdfcontacturl={http://libreas.eu},
      baseurl={},
      pdflang={de},
      pdfmetalang={de},
      pdfdoi={10.18452/31229},
      pdfurl={https://doi.org/10.18452/31229}
     }



\date{}
\begin{document}

\maketitle
\thispagestyle{fancyplain} 

%abstracts

%body
Beiträge von Karsten Schuldt (ks), Eva Bunge (eb), Ben Kaden (bk), Maxi
Kindling (mk), Viola Voß (vv)

\hypertarget{zur-kolumne}{%
\section{1. Zur Kolumne}\label{zur-kolumne}}

Ziel dieser Kolumne ist es, eine Übersicht über die in der letzten Zeit
erschienene bibliothekarische, informations- und
bibliothekswissenschaftliche sowie für diesen Bereich interessante
Literatur zu geben. Enthalten sind Beiträge, die der LIBREAS-Redaktion
oder anderen Beitragenden als relevant erschienen.

Themenvielfalt sowie ein Nebeneinander von wissenschaftlichen und
nicht-wissenschaftlichen Ansätzen wird angestrebt und auch in der Form
sollen traditionelle Publikationen ebenso erwähnt werden wie
Blogbeiträge oder Videos beziehungsweise TV-Beiträge.

Gerne gesehen sind Hinweise auf erschienene Literatur oder Beiträge in
anderen Formaten. Diese bitte an die Redaktion richten. (Siehe
\href{http://libreas.eu/about/}{Impressum}, Mailkontakt für diese
Kolumne ist
\href{mailto:zeitschriftenschau@libreas.eu}{\nolinkurl{zeitschriftenschau@libreas.eu}}.)
Die Koordination der Kolumne liegt bei Karsten Schuldt, verantwortlich
für die Inhalte sind die jeweiligen Beitragenden. Die Kolumne
unterstützt den Vereinszweck des LIBREAS-Vereins zur Förderung der
bibliotheks- und informationswissenschaftlichen Kommunikation.

LIBREAS liest gern und viel Open-Access-Veröffentlichungen. Wenn sich
Beiträge dennoch hinter einer Bezahlschranke verbergen, werden diese
durch \enquote{{[}Paywall{]}} gekennzeichnet. Zwar macht das Plugin
\href{http://unpaywall.org/}{Unpaywall} das Finden von legalen
Open-Access-Versionen sehr viel einfacher. Als Service an der
Leserschaft verlinken wir jedoch auch direkt OA-Versionen, die wir vorab
finden konnten. Für alle Beiträge, die dann immer noch nicht frei
zugänglich sind, empfiehlt die Redaktion (neben
\href{http://unpaywall.org/}{Unpaywall}) die Browser-Plugins
\href{https://openaccessbutton.org/}{Open Access Button} oder
\href{https://core.ac.uk/services/discovery/}{CORE} zu nutzen sowie auf
dem favorisierten Social-Media-Kanal mit
\href{https://mastodon.social/tags/icanhazpdf}{\#icanhazpdf}, um Hilfe
bei der legalen Dokumentenbeschaffung zu bitten.

Die bibliographischen Daten der besprochenen Beiträge aller Ausgaben
dieser Kolumne finden sich in der öffentlich zugänglichen Zotero-Gruppe:
\url{https://www.zotero.org/groups/4620604/libreas_dldl/library}.

\hypertarget{artikel-und-zeitschriftenausgaben}{%
\section{2. Artikel und
Zeitschriftenausgaben}\label{artikel-und-zeitschriftenausgaben}}

\hypertarget{vermischte-themen}{%
\subsection{2.1 Vermischte Themen}\label{vermischte-themen}}

Hobart, Elizabeth (2024). \emph{Describing Games for Special Collections
Libraries}. In: RBM. A Journal of Rare Books, Manuscripts, and Cultural
Heritage 25 (2024) 1, 50--64, \url{https://doi.org/10.5860/rbm.25.1.50}

Es wird immer seltener, dass Bibliothekar*innen während ihrer Ausbildung
einen Katalogisierungskurs besuchen. Deshalb ist dieser Einblick in die
konkrete Katalogisierungspraxis interessant: Anhand der Aufnahme von
zwei Spielen aus der Mitte des 19. Jahrhunderts in den
Bibliothekskatalog der Penn State University (Pennsylvania) wird
gezeigt, welche Schwierigkeiten sich weiterhin stellen, wenn
Nonbook-Materialien katalogisiert werden müssen, insbesondere wenn sie
dann noch in Fremdsprachen (Französisch und Deutsch) vorliegen.

Es ist ein wenig so, wie es früher in der Ausbildung gewesen sein muss,
wenn schon die Grundlagen der Katalogisierung \enquote{durchgenommen}
wurden und jetzt zu den viel spannenderen Sonderfällen übergegangen
wurde: Ein Beispiel dafür, wann die normalen Regelwerke nicht
ausreichend sind, sondern lauter Sonderregelungen gefunden werden
müssen. (ks)

\begin{center}\rule{0.5\linewidth}{0.5pt}\end{center}

Carroll, Mary ; Garrison, Kasey ; Oddone, Kay ; Wakeling, Simon (2024).
\emph{School libraries in Australia: A preliminary analysis of the
Knowledge Bank of Australian and New Zealand School Libraries}. In: IFLA
Journal (Online First), \url{https://doi.org/10.1177/03400352241246442}

Die Autor*innen haben eine Datenbank mit alle greifbaren Texten über
Schulbibliotheken in Australien und Neuseeland aufgebaut, in diesem Text
werten sie die aus Australien aus. Dabei präsentieren sie einige
beschreibenden Statistiken (beispielsweise die Zahl der Texte pro Jahr),
kontextualisieren die beschriebenen Ergebnisse aber auch nochmal.

Im Ganzen liest sich der Text aber vor allem als die Beschreibung einer
krisenhaften Situation. Alle Autor*innen sind an der School of
Information and Communication Studies der Charles Sturt University
(Wagga Wagga) angestellt -- und damit an der letzten Einrichtung, welche
eine Ausbildung für Schulbibliothekar*innen in Australien und Neuseeland
anbietet. Dies war einmal anders: Während der 1960er und 1970er Jahre
interessierte sich die australische Politik mehr für den nachhaltigen
Aufbau von Schulbibliotheken sowie einer Infrastruktur, welche diese
unterstützt (Ausbildung, Beratungsstellen, Forschung). In den letzten
Jahren wurde dies zurückgefahren: Die Infrastrukturen wurden
zurückgebaut oder ganz geschlossen, die Politik führt lieber
kurzfristige, dafür grosse Projekte durch, in denen neue
Schulbibliotheken aufgebaut werden – ohne für ihre nachhaltige
Finanzierung oder Ausstattung mit Personal zu sorgen. Eine übergreifende
Forschung existiert auch nicht mehr.

Die Autor*innen haben die genannte Datenbank erst aufgebaut, weil sie
diesen Wandel als Gefahr sahen und zumindest eine Wissensquelle für die
weitere Arbeit zur Verfügung stellen wollten. Getrieben von dieser
Krisenstimmung scheinen sie dann auch die Datenbank auszuwerten. Sie
zeigen, dass sich die Geschichte der Schulbibliotheken in Australien bis
zum Anfang des 20. Jahrhunderts anhand von Literatur nachweisen lässt
(und nicht erst seit den 1960er Jahren). Sie zeigen auch, wie sich die
wandelnde Politik direkt auf die Schulbibliotheken und die Literatur
über diese auswirken -- mehr Infrastruktur und nachhaltige Finanzierung
führten zu mehr Literatur, die Umstellung auf kurzfristige Projekte
führte auch zu weniger (und anderer) Literatur. Zudem stellen sie fest
(das ist vielleicht einer der wenigen Lichtblicke), dass die
Bibliotheksverbände nach dem \enquote{Rückzug} der Politik mehr
Verantwortung dafür übernommen hätten, für Schulbibliotheken Marketing
und Lobbyarbeit zu betreiben. (ks)

\begin{center}\rule{0.5\linewidth}{0.5pt}\end{center}

Hider, Philip ; Wakeling, Simon ; Marshall, Amber ; Garner, Jane (2024).
\emph{Public Library Services in Rural Australia: Challenges and
Prospects}. In: Journal of the Australian Library and Information
Association, 73 (2024) 2: 122--147,
\url{https://doi.org/10.1080/24750158.2024.2315338}

Mittels einer Umfrage wurde die Situation der Öffentlichen Bibliotheken
in Australien ausserhalb der Städte und grösseren Siedlungen erhoben
(hier \enquote{outer regional}, \enquote{remote} und \enquote{very
remote}). Die Fragen bezogen sich auch darauf, welche Auswirkungen die
COVID-19Pandemie hatte und wie die Zukunft dieser Bibliothekssysteme
gesehen wird. Begründet wird diese Umfrage unter anderem damit, dass in
der Literatur, aber auch zum Beispiel in der Arbeit des australischen
Bibliotheksverbandes, die Bibliotheken in den urbanen Zentren im Fokus
stehen, während über die in \enquote{ländlichen} Gegenden kaum etwas
bekannt sei.

Die Ergebnisse zeigen eine erstaunliche Konstanz: Die Bibliothekssysteme
scheinen recht gut aufgestellt zu sein und sich langsam zu entwickeln,
allerdings immer auf Basis geringer Ressourcen. Es werden viele
Fahrbibliotheken betrieben, deren Zahl sich in den letzten Jahren auch
kaum verändert hat. Tendenziell sind die Bibliotheksgebäude auch in den
\enquote{remote regions} in den jeweils grössten Siedlungen
untergebracht. Interessant sind die Antworten auf die Fragen nach der
COVID-19-Pandemie und der generellen Zukunftsperspektive. Die
Bibliotheken wären durch die Pandemie mehr zu Orten geworden, an denen
die Bevölkerung Hilfe bei IT-Problemen sucht und auch Zugang zu
digitalen Medien nutzt. Aber ansonsten hat nur eine geringe Zahl von
Bibliothekar*innen den Eindruck, als hätte sich viel verändert. Vielmehr
sind die meisten der Meinung, dass die Situation Ende 2023 (als die
Umfrage durchgeführt wurde) nicht gross anders wäre als 2019. Auch für
die Zukunft sehen die meisten keine grosse Veränderung. Zwar gäbe es
immer Probleme mit der Finanzierung und beispielsweise dem Verschleiss
bei den Fahrzeugen der Fahrbibliotheken, aber nur eine kleine Zahl von
Bibliothekar*innen sieht grosse Veränderungen auf die Bibliotheken
zukommen. (ks)

\begin{center}\rule{0.5\linewidth}{0.5pt}\end{center}

Kann-Rasmussen, Nanna (2023). \emph{When librarians speak up:
justifications for and legitimacy implications of librarians' engagement
in social movements.} In: Journal of Documentation 79 (2023) 1: 36--51,
\url{https://doi.org/10.1108/JD-02-2022-0042} {[}Paywall{]}

Die Autorin versucht, eine soziologische Erklärung dafür zu finden, wie
Bibliothekar*innen und Öffentliche Bibliotheken anhand von Beispielen
aus Dänemark und Schweden ihre Beteiligung an gesellschaftspolitischen
Themen – im Fall dieses Artikels im Bereich LGBTQ+-Rechte, ökologische
Bewegung und Antirassismus – begründen und wie diese Begründung
zusammenhängt mit der grundsätzlichen Absicherung der Bibliotheken
gegenüber ihren Trägern. Sie basiert diese Überlegungen nicht nur auf
Interviews mit aktiven Bibliothekar*innen, sondern auch auf Studien, die
sich darüber Gedanken machen, warum sich Bibliotheken überhaupt
entwickeln. Letztlich plädiert sie dafür, aus der französischen
Soziologie das Framework \enquote{orders of worth} (von Luc Boltanski
und Laurent Thévenot) zu übernehmen. Dieses beschreibt, wie
Organisationen in modernen Gesellschaften sich und ihre Arbeit
absichern, indem sie jeweils verschiedene Werte vertreten und
verschiedene Argumente mobilisieren, die sich jeweils an
unterschiedliche Träger richten und dabei auch disparate Ziele anstreben
können. Öffentliche Bibliotheken würden \enquote{funktionieren}, weil
sie neben den gesellschaftspolitischen Zielen – die auch nicht immer
eindeutig, sondern oft für Interpretationen offen sind – auch andere
Aufgaben betonen, beispielsweise die Literaturversorgung der
Bevölkerung.

Die Argumentation der Autorin wird nicht alle Leser*innen überzeugen.
Aber es ist eine gute Anregung dazu, über das Bestehen und die
Weiterentwicklung von Bibliotheken anders nachzudenken, als nur jeweils
ein Ziel oder eine mögliche Entwicklung zu betonen. (ks)

\begin{center}\rule{0.5\linewidth}{0.5pt}\end{center}

Lea, Mary Margaret ; Emmelhainz, Celia (2024). \emph{Organizers of
Museum History: Honoring the Labor of Librarians and Archivists in the
Bureau of American Ethnology}. In: International Journal of
Librarianship 9 (2024) 2: 87--102,
\url{https://doi.org/10.23974/ijol.2024.vol9.2.365}

Basierend auf einem historischen Projekt, welches die Arbeit von Frauen
im Smithsonian Institut erforscht, präsentieren die Autor*innen hier die
Biographien von acht Bibliothekarinnen und Archivarinnen, die im 19. und
20. Jahrhundert tätig waren. Zudem diskutieren sie, dass sich deren
Namen eher in den \enquote{Acknowledgments} von wissenschaftlichen
Publikationen finden, als an anderer Stelle. Es sei schwer, genau zu
klären, was ihre jeweilige Tätigkeit in den jeweiligen
Forschungsprojekten war, aber es sei sichtbar, dass sie immer dazu
beitrugen, dass am Smithsonian überhaupt geforscht werden konnte. Bis in
die zweite Hälfte des 20. Jahrhunderts sei diese Arbeit von Frauen aber
regelmässig abgewertet worden. (ks)

\begin{center}\rule{0.5\linewidth}{0.5pt}\end{center}

Themenschwerpunkt \enquote{Koloniale Kontexte in Bibliotheken} (2024).
In: o-bib 11 (2024) 3, \url{https://www.o-bib.de/bib/issue/view/335}

In diesem Schwerpunkt sind Beiträge versammelt, die aus einem Workshop,
welcher Ende 2023 an der Staatsbibliothek zu Berlin Preußischer
Kulturbesitz stattfand. Dabei handelt es sich nicht um die damals
gehaltenen Vorträge, sondern jeweils um Artikel, die in gewisser Weise
von Autor*innen-Kollektiven auf Basis des Workshops geschrieben wurden.
Der Anspruch ist umfangreich: Es soll geklärt werden, was
\enquote{kolonialer Bibliotheksbestand} konkret heisst, gleichzeitig
soll die Arbeit an der Dekolonisierung dieser Bestände beschrieben und
zudem in den Kontext bibliothekarischer Arbeiten integriert werden.

Viele der Beiträge verweisen dann zuerst auf breitere Diskussionen, die
Museen, Archive, andere Kultureinrichtungen und die politische Ebene
einbeziehen. Anschliessend stellen sie oft die Schwierigkeiten mit dem
Begriff \enquote{Kolonalismus} und seinem Verständnis dar: Was als
kolonial gilt, wie es zum Beispiel bei den Erwerbungsgeschichten von
Beständen zu beachten ist oder wie die einzelnen Materialien
interpretiert werden, ist nicht mit einer einfachen Definition zu
fassen. Es bleibt immer die Notwendigkeit, sich im Einzelfall und über
mehrere Anspruchsgruppen hinweg zu verständigen. Anschliessend gehen die
meisten Beiträge auf konkrete Beispiele von Beständen, Katalogisierungs-
oder Digitalisierungsprojekten ein. Dabei zeigt sich immer wieder, dass
mit jedem Bestand neue Fragen auftauchen. In diesem Rahmen vorgestellt
werden aber unter anderem Arbeitsgruppen und Netzwerke, die im
Bibliotheksbereich in den letzten Jahren entstanden sind sowie die
Arbeit an einem neuen Metadatenfeld im MARC21-Standard (361) für die
Beschreibung der Provenienz von Materialien.

Der letzte Beitrag, eine Diskussion von Forschenden aus Deutschland und
Afrika über Bibliotheksbestände, führt dann noch einmal vor Augen, dass
viele dieser Projekte und Diskussionen sehr europäisch sind. Was heute
in Deutschland als \enquote{koloniale Literatur} gilt, kann in anderen
Zusammenhängen als deutsch-afrikanische Literatur mit einem eigenen Wert
gelten.

Der Schwerpunkt schafft es, dem Anspruch, einen Überblick zu liefern,
gerecht zu werden. Er ist ein sinnvoller Einstieg in das Thema. Etwas
irritierend ist allerdings, dass er selber in zwei Punkten, ohne das zu
reflektieren, eine Art von übergreifender Inanspruchnahme vornimmt.
Einerseits wird in den Beiträge oft vom \enquote{deutschsprachigen Raum}
geschrieben, aber ausser in einem Beitrag, in welchem auch die
Universitätsbibliothek Wien vorkommt, wird nur über deutsche
Bibliotheken geschrieben. Andererseits wird ständig so geschrieben, als
seien \enquote{koloniale Sammlungen} Teil aller Bibliotheken, dabei ist
dieses spezifische Thema – im Gegensatz zu Wirkungen kolonialer
Wissensproduktion – nur für eine Anzahl von Bibliotheken relevant. (ks)

\begin{center}\rule{0.5\linewidth}{0.5pt}\end{center}

Brundy, Curtis ; Thornton, Joel B. (2024). \emph{The paper mill crisis
is a five-alarm fire for science: what can librarians do about it?} In:
Insights 37 (2024) 11: 1--7, \url{https://doi.org/10.1629/uksg.659}.

Brundy und Thornton geben einen kurzen Überblick über aktuelle
Entwicklungen im Bereich der Paper Mills und unethischer
Publikationspraktiken. Dabei nehmen sie auch die möglichen strukturellen
Ursachen im Wissenschaftssystem sowie Reaktionen der Verlage in den
Blick. Schließlich nehmen sie auch Bibliotheken und Bibliothekar*innen
-- die durch die bibliotheksseitige Finanzierung von
Publikationsgebühren nun enger mit dem wissenschaftlichen
Publikationsprozess verknüpft seien als je zuvor -- in die Pflicht,
einen Teil zum Kampf gegen wissenschaftsschädigende Geschäftspraktiken
beizutragen. Dazu machen sie konkrete Vorschläge, wie dies in der Praxis
umgesetzt werden kann. (eb)

\begin{center}\rule{0.5\linewidth}{0.5pt}\end{center}

Feinmann, Jane (2024). '\emph{Substandard and unworthy': why it's time
to banish bad-mannered reviews}. In: Nature / nature.com. 23.09.2024.
\url{https://doi.org/10.1038/d41586-024-02943-z}

Der Artikel problematisiert die Tonalität von Peer-Review-Gutachten, die
in gar nicht so wenigen Fällen unprofessionell bis beleidigend zu sein
scheint. Damit weist er auf einen weiteren Nachteil des nach wie vor als
Standard geltenden Ansatzes zur Beurteilung der Qualität
wissenschaftlicher Manuskripte hin. Laut einer Studie aus dem Jahr 2019
haben um die 60\,\% der Publizierenden aus dem MINT-Bereich
entsprechende Erfahrungen machen müssen. (vergleiche Nyssa J. Silbiger,
Amber D. Stubler (2019). \emph{Unprofessional peer reviews
disproportionately harm underrepresented groups}. In: STEM. PeerJ
7:e8247, \url{https://doi.org/10.7717/peerj.8247}) Erwartungsgemäß hat
dies negative Auswirkungen gerade bei Junior-Forschenden, da solche
Reviews als demotivierend und damit auch als Angriff auf das
Selbstvertrauen wirken. Dies wirkt sich zugleich schädlich auf die
Forschungsproduktivität und auf die Forschungskultur als Ganzes aus.
Besonders betroffen scheinen auch hier Frauen und Minderheiten zu sein.

Entsprechend relevant wäre es, das Aufkommen solcher Kommentare
einzuhegen. Eine Rolle fällt dabei den Redakteur*innen der Journals zu,
die entsprechende Formulierungen erkennen und entfernen können. Zudem
können die Begutachtenden entsprechend sensibilisiert und geschult
werden. Weiterhin wird der Ansatz eines \enquote{transparent peer
review} vorgeschlagen. Bei diesem werden neben der eigentlichen
Publikation auch die Gutachten veröffentlicht. Die Erfahrungen damit
sind indes noch überschaubar: Bei einem entsprechenden Angebot von IOP
Publishing (IOPP) nutzen bisher nur etwa die Hälfte der Autor*innen die
Möglichkeit zur ergänzenden Publikation der Reviews. Auch die doppelte
Anonymisierung, bei der sowohl Reviewer*innen als auch Autor*innen
anonymisiert werden, reduziert offenbar die Zahl übergriffiger und
abwertender Formulierungen. (bk)

\begin{center}\rule{0.5\linewidth}{0.5pt}\end{center}

Caffrey, Carolyn ; Perry, Katie ; Withorn, Tessa ; Lee, Hannah ; Philo,
Thomas ; Clarke, Maggie ; Eslami, Jillian ; Galoozis, Elizabeth ; Kohn,
Katie Paris ; Ospina, Dana ; Chesebro, Kimberly ; Clawson, Hallie ;
Dowell, Laura (2024). \emph{Library instruction and information literacy
2023.} In: Reference Services Review 52 (2024) 3: 298--384,
\url{https://doi.org/10.1108/RSR-07-2024-0036} {[}Paywall{]}

Dieser Beitrag stellt eine annotierte Liste von englischsprachigen
Beiträgen zum Thema \enquote{library instruction} – also Einführungen in
die Bibliotheksnutzung und Nutzer*innenschulungen von Bibliotheken –
dar, die im Jahre 2023 erschienen sind. Insgesamt sind dies 374
Beiträge, welche für diesen Beitrag gelesen, geordnet, zusammengefasst
und bewertet wurden. Eine solche Übersicht erscheint in dieser
Zeitschrift jährlich seit 1973. Dabei wird jeweils Vollständigkeit
angestrebt, obgleich mit der sprachlichen Eingrenzung selbstverständlich
jeweils nur ein Teilbereich der bibliothekarischen Literatur sichtbar
wird. Wer sich mit diesem Themenfeld beschäftigt, wird diese Liste
kennen.

In dieser Ausgabe hat die Gruppe, welche die Liste erstellt, aber eine
Anzahl von vorhergehenden Überlegungen an den Anfang gestellt, die aus
einer Art Selbstreflektion entstanden sind. Sie schildern, welche
Grenzen ihre Annotationen haben – beispielsweise, dass sie immer als
Bibliothekar*innen und Bibliothekswissenschaftler*innen beschreiben,
aber nicht aus einer pädagogischen Perspektive oder aber, dass sie zwar
anstreben, die Liste nicht als Wertung gelten zu lassen, aber es schwer
ist, diesen Eindruck zu vermeiden. Zudem schildern sie ihr Vorgehen beim
Erstellen der Liste: (1) Welche Datenbankrecherchen sie durchführen, (2)
wie sie die Ergebnisse jeweils in eine gemeinsam zu bearbeitende Liste
transformieren, (3) wie sie die Annotationen verfassen und (4) wie sie
diese jeweils gemeinsam in der Gruppe besprechen, bevor sie
veröffentlicht werden. Zudem beschreiben sie die grundsätzliche
Kategorisierung, die sie für die Liste vorgenommen haben. Interessant
ist dies, weil solche fachlichen, annotierten Liste zur Literatur über
bibliothekarische Themen heute kaum noch erstellt werden. (Wenn man
nicht diese Kolumne hier als so eine Liste werten wollen würde.) Die
Darstellung am Anfang dieses Beitrags gibt eine Übersicht dazu, welche
Arbeit nötig wäre, um weitere solcher Listen – für andere Themen oder
für andere Sprachen – zu erstellen. Etwas, was grundsätzlich zu
begrüssen wäre. (ks)

\hypertarget{inklusion-und-safe-spaces}{%
\subsection{2.2 Inklusion und Safe
Spaces}\label{inklusion-und-safe-spaces}}

Rondinelli, Morgan (2024). \emph{What's Missing in Conversations about
Libraries and Mental Illness}. In: In the Library with the Lead Pipe,
19. Juni 2024,
\url{https://www.inthelibrarywiththeleadpipe.org/2024/conversations-about-libraries/}

In diesem persönlichen Essay fasst die Autorin, selber Bibliothekarin,
die mit Zwangsstörung lebt, zusammen, was in den letzten Jahren an
englisch-sprachiger Literatur zum Thema Bibliotheken und psychische
Störungen publiziert wurde, um es mit ihren eigenen Erfahrung
abzugleichen. Dabei kommt sie zu grundsätzlich positiven Einschätzungen:
Zumindest in ihrem beruflichen Umfeld sei sie und ihr Umgang mit ihrer
Zwangsstörung akzeptiert. In der Literatur vermisst sie aber eine
tiefere Auseinandersetzung mit der Situation von Bibliothekar*innen, die
mit solchen Störungen leben. Zumeist würde sich in der Forschung auf
Nutzer*innen konzentriert. (ks)

\begin{center}\rule{0.5\linewidth}{0.5pt}\end{center}

Berget, Gerd (2024). \emph{What is the role of public libraries and
books in the everyday lives of adults with intellectual disability?}.
In: Journal of Librarianship and Information Science {[}Online First{]},
\url{https://doi.org/10.1177/09610006241257278}

Mithilfe von Interviews von 25 Personen, die als \enquote{care giver}
oder Lehrpersonen direkten Kontakt mit Personen haben, die mit
intellektuellen Beeinträchtigungen leben, sollte in dieser Studie
erhoben werden, wie letztere Öffentliche Bibliotheken und Bücher nutzen.
Dabei ging es immer um die Situation in Norwegen. Relevant ist dies
auch, weil es in diesem Land eine Stiftung gibt, welche Bücher
publiziert, die extra für Menschen mit intellektuellen
Beeinträchtigungen adaptiert werden. (Im Laufe des Artikels wird auch
vermittelt, was die Herausforderungen bei solchen Adaptierungen sind.)

Im Ergebnis zeigt sich, dass die Bibliotheken, Bücher im Allgemeinen und
auch die adaptierten Bücher im Alltag kaum genutzt werden. Bibliotheken
werden beispielsweise kaum besucht. Aber selbst dann, wenn dies
passiert, wird kaum auf bibliothekarische Angebote zurückgegriffen. Auch
Bestände mit den genannten adaptierten Büchern werden kaum genutzt. Die
Bibliothek wird vor allem als Ort genutzt, der an sich besucht werden
kann. Berget führt dies darauf zurück, dass die \enquote{care giver} und
Lehrpersonen zu wenig über die vorhandenen Angebote informiert sind.
(ks)

\begin{center}\rule{0.5\linewidth}{0.5pt}\end{center}

Zeuner, Philipp ; Buchert, Caleb ; Fischer, Yvonne ; Baumann, Nik ;
Frick, Claudia ; Ramünke, Sabrina (2024). \emph{Bibliotheken als Safe(r)
Spaces für die LGBTQIA+ Community?: Hands-on Lab auf der BiblioCon
2024}. In: API Magazin 5 (2024) 2,
\url{https://doi.org/10.15460/apimagazin.2024.5.2.209}

Während in der letzten Ausgabe der LIBREAS. Library Ideas ein Artikel
der Queerbrarians zur grundsätzlichen Frage, was queer sein in der
Bibliothekspraxis heisst, veröffentlicht wurde, organisierte die gleiche
Gruppe auf der BiblioCon im Juni 2024 ein HandsOn-Lab, das an der Frage
arbeiten sollte, ob und wie Bibliotheken safe spaces für Menschen aus
der LGBTQAI+ Community sein beziehungsweise werden könnten. (Da das Lab
auf der BiblioCon stattfand, die vor allem von Bibliothekar*innen aus
dem DACH-Raum besucht wird, implizit also auch Bibliotheken aus diesen
Ländern.) Der Text ist ein Bericht über das Lab und dessen Ergebnisse.
Da die Veranstaltung überfüllt war, kann man von einem hohen Interesse
ausgehen. (Obgleich, wie so oft bei diesen Konferenzen, auch auf der
BiblioCon 2024 ständig Klagen über zu kleine Räume erhoben wurden, nicht
nur bei diesem Thema.)

Insgesamt schliessen die Autor*innen, dass (a) das Interesse am Thema
gross ist, (b) bei vielen, aber nicht allen, Bibliothekar*innen ein
Willen vorhanden ist, die Bibliothek möglichst offen und \enquote{safe}
zu gestalten, dass allerdings (c) kein Raum wirklich sicher sein kann,
sondern nur sicherer als andere Räume. Als Hauptherausforderung wird (d)
postuliert, dass es zu wenige Hilfestellungen, beispielsweise Leitfäden,
für die konkrete Praxis geben würde. (ks)

\begin{center}\rule{0.5\linewidth}{0.5pt}\end{center}

Albro, Maggie ; Stark, Rachelr Keiko ; Kauffroath, Kelli (2024).
\emph{Checking Out Our Workspaces: An Analysis of Negative Work
Environment and Burnout Utilizing the Negative Acts Questionnaire and
the Copenhagen Burnout Inventory for Academic Librarians.} In: Evidence
Based Library and Information Practice, 19 (2024) 3: 2--22,
\url{https://doi.org/10.18438/eblip30472}

In dieser Studie wurde vermutet, dass Wissenschaftliche
Bibliothekar*innen in den USA, für die dies die zweite oder dritte
Karriere darstellt (also die nicht direkt in den Beruf eingestiegen
sind), verstärkt unter Burnout leiden. Die Gründe dafür scheinen sehr
US-spezifisch zu sein und es zeigt sich dann in der Auswertung auch,
dass diese Vermutung nicht stimmt. Vielmehr gibt es Verweise darauf,
dass es an sich ein Level an Burnout unter den Bibliothekar*innen gibt,
der aber nicht dadurch bestimmt wird, ob jemand direkt in den Beruf
eingestiegen ist oder über andere Wege. Zudem zeigt sich, dass andere
Personen und Institutionen auf dem Campus sich eher weigern, mit der
Bibliothek zusammenzuarbeiten, wenn diese als \enquote{schlechter
Arbeitsplatz}, also als Institution mit einer schlechten Arbeitskultur
für das Personal, bekannt ist.

Was an dieser Studie heraussticht, ist, dass sie dafür nicht neue
Umfrageinstrumente erstellt, sondern explizit auf etablierte Umfragen
aus der Psychologie zurückgreift, die sowohl theoretisch abgesichert
sind als auch standardisiert und getestet wurden. Hier wurden sie für
Wissenschaftliche Bibliothekar*innen genutzt (indem nur diese in einer
Umfrage inkludiert wurden). Dadurch lassen sich aber -- wie es die
Autor*innen tun -- die Ergebnisse besser in den \enquote{normalen
Levels} von Burnout in der US-amerikanischen Bevölkerung verorten. Und
sie liessen sich in Zukunft auch besser nachnutzen. In diesem Sinne ist
die Studie vorbildhaft. (ks)

\hypertarget{forschungsdaten}{%
\subsection{2.3 Forschungsdaten}\label{forschungsdaten}}

Irene V. Pasquetto, Zoë Cullen, Andrea Thomer, Morgan Wofford, (2024):
\emph{What is research data \enquote{misuse}? And how can it be
prevented or mitigated?} In: Journal of the Association for Information
Science and Technology, 1--17, \url{https://doi.org/10.1002/asi.24944}

Die Autor*innen untersuchen Varianten des Missbrauchs von
Forschungsdaten insbesondere in offener beziehungsweise publizierter
Form. Eine Schlussfolgerung lautet, dass eine missbräuchliche Nutzung
nicht gänzlich ausgeschlossen werden kann. Umso mehr sollten sich
professionelle Datenvermittler*innen (\enquote{Data intermediaries})
proaktiv mit dem Thema und dem Einhegen beziehungsweise Management von
Missbrauchsszenarien befassen. Eine Lösung sehen die Autor*innen in
einem expliziten Sichtbarmachen und Kommunizieren des Problems.

Weiterhin identifizieren die Autor*nnen eine zweite auffällige
Diskrepanz: Über Open Science Settings werden zunehmend Forschungsdaten
als Open Research Data der Allgemeinheit zugänglich. Zugleich sind die
meisten Guidelines auf Fach- und Expertencommunities zugeschnitten, was
dazu führt, dass man externen Nutzenden wenig Handhabe an die Hand gibt.
Hier werden Data-Literacy- und Citizen-Science-Programme für bestimmte
öffentliche Zielgruppen erwähnt, wobei jedoch für die generell
extrawissenschaftliche Bereitstellung und Kontextualisierung
ausreichende Lösungen offenbar ein Desiderat bleiben.

Ein weiterer Mehrwert über diese vertiefende Problematisierung hinaus
liegt in einer Art Typologie des Forschungsdatenmissbrauchs, der sich
laut der Autor*innen in mindestens sieben Formen aufschlüsseln lässt:

\begin{enumerate}
\def\labelenumi{\arabic{enumi}.}
\item
  Analytische Fehler: die Verwendung falscher oder fehlerhafter Methoden
  während der Datenanalyse, die zu ungenauen oder unzuverlässigen
  Ergebnissen führen;
\item
  Fehlinterpretationen: das Missverstehen der Bedeutung oder
  Implikationen von Forschungsdaten, was möglicherweise zu falschen
  Schlussfolgerungen führt;
\item
  Falschdarstellungen: die absichtliche oder unbeabsichtigte Verzerrung,
  Veränderung oder Auslassung von Daten, die einen falschen oder
  unvollständigen Eindruck vermitteln oder eine bestimmte Agenda
  unterstützen können;
\item
  Rufschädigung: die Schädigung der Reputation der ursprünglichen
  Datenerheber*innen, -analyst*innen oder -kurator*innen durch
  Handlungen wie unterlassene Datenzitation, verweigerte oder
  ungerechtfertigte Autor*innenschaft oder Erzeugung öffentlicher
  Beschämung für wahrgenommene Mängel in den Daten;
\item
  Verletzung der Privatsphäre und Geoprivatsphäre: die Verletzung der
  Vertraulichkeitsvereinbarungen mit Personen oder die Offenlegung
  sensibler standortbezogener Informationen, die möglicherweise Schäden
  oder Unbehagen verursachen
\item
  Ausnutzung (\enquote{exploitation}): Unethische oder schädliche
  Nutzung von Forschungsdaten zum persönlichen Vorteil oder Gewinn, oft
  auf Kosten der Personen, auf die sich die Daten beziehen;
\item
  Unkritische Verwendung voreingenommener und anstößiger Daten: das
  Versäumnis, Datenquellen, die Vorurteile, anstößige Inhalte oder
  diskriminierende Elemente enthalten, kritisch zu evaluieren und zu
  hinterfragen, wodurch schädliche Stereotypen oder Narrative
  fortgeführt werden. (bk)
\end{enumerate}

\begin{center}\rule{0.5\linewidth}{0.5pt}\end{center}

Pinto, Fabiana (2023): \emph{Trilha histórica sobre prática da publição
de dados de pesquisa.} In: Perspectivas em Ciência da Informação. inf.
28 (2023), \url{https://doi.org/10.1590/1981-5344/45978}

In diesem Aufsatz wird die Geschichte des Publizierens von Daten
beziehungsweise Forschungsdaten von der Vorgeschichte (beginnend mit dem
Ishango-Knochen) bis in die Gegenwart anhand einiger Beispiele
nachgezeichnet. Der Datenbegriff wird in diesem Zusammenhang sehr
inklusiv ausgelegt. So sind Höhlenmalereien dahingehend als
Datenpublikation interpretierbar, als das in ihnen Erkenntnisse über die
Welt in einer grafischen Form dokumentiert wurde. Die Intentionalität
der diese Quellen anfertigenden Personen bleibt naturgemäß unklar,
könnte aber generell bei der definitorischen Frage, wann ein Inhalt auch
eine Forschungsdatenpublikation ist, durchaus relevant werden. Der
Aufsatz betont allerdings stärker, dass sich solche Artefakte aus
heutiger Sicht als Forschungsdaten lesen lassen. Aufschlussreich ist das
zitierte Beispiel einer heilpflanzenkundlichen Publikation mit dem Titel
\enquote{De Materia Medica} von Pedanius Dioscorides aus dem 6.
Jahrhundert. Hierbei handelt es sich um eine Systematisierung und
Dokumentation von Erfahrungswissen, die durchaus mit heutigen
Vorstellungen einer Fach- oder wissenschaftlichen Publikation
korrespondiert. Auch Datenaufzeichnungen von Galileo Galilei werden als
Beispiele einer Art Forschungsdatenpublikation erwähnt. Komplett
einsichtig wird die historische Kontinuität schließlich am Beispiel von
Marie Curies Laborbuch. Tatsächlich verdeutlicht der Aufsatz, dass jede
Land- oder Seekarte und jede statistische Aufzeichnung als Vorläufer
dessen zu verstehen ist, was wir heute als Forschungsdatenpublikation
ansehen. Der Aufsatz gibt damit einen sehr schönen kursorischen
Überblick über die mediologische und methodische Entwicklung von
Datenpublikationen. Die gegenwärtige digitale Praxis hat dabei, wie auch
Pinto betont, eigene Logiken, Regeln und Standards und ermöglicht
darüber hinaus die Abbildung und Publikation ungleich größerer
Datenmengen. Zum Ausblick unterstreicht Pinto noch einmal die Rolle von
Open Science für die wissenschaftliche Kommunikation, in diesem Fall
also im Sinne der transparenten Sichtbarmachung der eigenen Forschung
für die Peers. In Brasilien ist dies offenbar weitgehend durch ein
Angebot entsprechender Datenrepositorien abgesichert. (bk)

\hypertarget{open-access}{%
\subsection{2.4 Open Access}\label{open-access}}

Alfred Früh ; Rika Koch (2024): \emph{Ein neuer Blick auf Open Access:
Wissenschaftliches Publizieren aus Sicht. des öffentlichen
Beschaffungsrechts}. In: sui generis \#unbequem 2024, S. 65--75.
\url{https://doi.org/10.21257/sg.253}

In dem Artikel setzen sich die beiden Inhaber*innen
rechtswissenschaftlicher Professuren der Universität Basel und der
Fachhochschule Bern mit der Frage auseinander, inwieweit
rechtswissenschaftliche Publikationen nach den Prinzipien von Open
Access verfügbar gemacht werden sollten. Sie beziehen dazu eindeutig
Position: \enquote{Es ist {[}...{]} nicht einzusehen, weshalb die
Rechtswissenschaft eine geringere Verpflichtung zur Offenheit treffen
sollte als andere Disziplinen. Wenn schon, gilt das Gegenteil.
Schliesslich schützen Rechtsnormen individuelle und gesellschaftliche
Erwartungshaltungen; entsprechend sollten Informationen über deren
Gehalt und Auslegung erst für alle frei zugänglich sein.} Im Artikel
betrachten sie das 2021 revidierte Beschaffungsrecht von Schweizer Bund
und Kantonen und wenden es auf das wissenschaftliche Publizieren im
traditionellen \enquote{Reader/Library-Pay-Modell} einerseits sowie das
\enquote{Author-Pay-Modell} andererseits an. Nach einer Einführung in
die aus Sicht der Autor*innen derzeit maßgeblichen Wege
wissenschaftlicher Veröffentlichungen nach diesen beiden Ansätzen
(Bibliotheken zahlen für den Zugang oder Autor*innen für das
Publizieren) prüfen sie, inwieweit das geltende Beschaffungsrecht für
beide Ansätze geltend gemacht werden kann. Zum
\enquote{Reader/Library-Pay-Modell} halten sie zunächst fest, dass es
durch diesen Ansatz \enquote{auf vergleichsweise begrenzten Märkten
leicht {[}ermöglicht wird{]}, eine erhebliche Marktmacht zu erlangen.}
Mögliche Abhilfe böte hier theoretisch das Kartellrecht, aber den
\enquote{Einsatzmöglichkeiten kartellrechtlicher Instrumente {[}sind{]}
aufgrund zeitlicher und finanzieller Restriktionen enge Grenzen
gesetzt.} Für das öffentliche Beschaffungsrecht stellen sie fest, dass
beide Modelle beziehungsweise Hochschulen und deren Angehörige
grundsätzlich dem öffentlichen Beschaffungsrecht unterliegen,
\enquote{wenn sie bei gewerblich tätigen Dritten – den Verlagen – Leistungen erwerben.} Somit unterlägen sie auch den
beschaffungsrechtlichen Prinzipien. Im Beitrag legen die Autor*innen
einen Fokus auf das für die Beschaffung im öffentlichen Bereich seit
2021 geltende Prinzip der Nachhaltigkeit. Sie führen dazu die
Begrifflichkeiten der wirtschaftlichen, sozialen und digitalen
Nachhaltigkeit aus. Maßgabe ist es, dass öffentliche Mittel
\enquote{wirtschaftlich nachhaltig ausgegeben werden und sich nachhaltig
auf die Gesellschaft auswirken bzw. einen möglichst grossen
gesellschaftlichen Mehrwert generieren.} An diese Verantwortung seien
insbesondere Institutionen gebunden, denen eine Bildungsfunktion
zukommt. Sie kommen entsprechend zu dem Schluss, dass für die Prüfung
der \emph{wirtschaftlichen Nachhaltigkeit} entscheidend ist, ob zwischen
der \enquote{Leistung der Verlage und dem von ihnen eingeforderten
Entgelt ein Missverhältnis vorliegt. Ist dies der Fall, könnte aus dem
Beschaffungsrecht eine Verpflichtung zum Publizieren in Open-Access-Form
abgeleitet werden. Noch deutlicher für die Prüfung einer
Open-Access-Pflicht beim Abschluss mit Verlagshäusern sprechen die
Gesichtspunkte der \emph{sozialen Nachhaltigkeit} und der
\emph{digitalen Nachhaltigkeit}.}

Der sui generis Textgattung geschuldet, führt an der einen und anderen
Stelle die verkürzte Darstellung zu missverständlichen Formulierungen;
auch kommt die Darstellung der komplexen Finanzierungsmodelle für Open
Access vor dem Hintergrund der aktuellen Open-Access-Transformation zu
kurz, so dass beispielsweise Transformationsverträge beziehungsweise
PAR-Modelle und Diamond-Ansätze der konsortialen und damit nicht
Autor*innen-basierten Finanzierung unberücksichtigt bleiben. Die
Autor*innen sind sich aber wohl bewusst, dass dieser Beitrag der Rubrik
\enquote{unbequem} nur ein erster Schritt sein kann. So ziehen sie
selbst das entsprechende Fazit: \enquote{Die beschaffungsrechtliche
Befassung mit dem wissenschaftlichen Publizieren steht noch ganz am
Anfang. So fehlt es beispielsweise noch an empirischen und rechtlichen
Untersuchungen zum Missverhältnis zwischen Leistung und Gegenleistung
der Verlage im traditionellen Modell, an einer Ausarbeitung der sozialen
und digitalen Nachhaltigkeitsdimension sowie an einem vergleichenden
Blick ins Ausland.} Eine Fortsetzung der Forschung ist mehr als
wünschenswert und zeigt einmal mehr die Bedeutung der Kostentransparenz
im wissenschaftlichen Publikationswesen auf. (mk)

\hypertarget{bestandserhaltung}{%
\subsection{2.5 Bestandserhaltung}\label{bestandserhaltung}}

Pasqui, Valdo (2024): \emph{Digital Curation and Long-Term Digital
Preservation in Libraries}. In: JLIS.It 15 (1):109--125,
\url{https://doi.org/10.36253/jlis.it-567}.

Dieser Artikel aus Italien reflektiert über die Herausforderungen für
die Bestandserhaltung digitaler Inhalte. Diese wurden bekanntlich in den
vergangenen zwei Jahrzehnten zu einem zentralen Angebot in Bibliotheken.
Sie umfassen neben Sammlungen digitalisierter Bestände auch genuin
digitale Materialien. Die Langzeitarchivierung bleibt eine
Herausforderung, weil sie die Verfügbarhaltung einschließt. Sie sollte
als \enquote{Digital preservation by design} von Beginn an mitgedacht
werden. Zudem sind digitale Bestandserhaltung (Digital Preservation) und
digitaler Sammlungskuration (Digital Curation) im Sinne einer
fortlaufenden aktiven Pflege der digitalen Inhalte und ihrer Vermittlung
eng miteinander verwoben. Digital Curation schließt neben dem Erhalt der
Inhalte selbst auch die Metadatenpflege und das
Zugänglichkeitsmanagement, also Auffindbarkeit und Verfügbarkeit, ein.
Ein besonderer, im Vergleich zum Management statischer Medien, Anspruch
ergibt sich durch die Entwicklungen im Bereich Open Science und
beispielsweise der FAIR-Prinzipien, die zu einer dynamischen Nutzung von
Inhalten führen. Dafür braucht es passende Strategien, die überwiegend
noch nicht vorliegen. Der Ansatz der Digital Curation gewinnt dabei
weiter an Bedeutung. Zudem verlangt Open Science, dass der offene Zugang
und die Nachnutzbarkeit parallel zu den Materialien an sich langfristig
gesichert sind. Weiterhin behandelt der Aufsatz Fragen der Infrastruktur
und der Infrastrukturökonomie für die digitale Langzeitarchivierung.
(bk)

\hypertarget{monographien-und-buchkapitel}{%
\section{3. Monographien und
Buchkapitel}\label{monographien-und-buchkapitel}}

\hypertarget{vermischte-themen-1}{%
\subsection{3.1 Vermischte Themen}\label{vermischte-themen-1}}

Adolpho, Kalani Keahi ; Krueger, Stephen G. ; McCracken, Krista (edit.)
(2023). \emph{Trans and Gender Diverse Voices in Libraries}. (Series of
Gender and Sexuality in Information Science, 13) Sacramento, CA: Library
Juice Press, 2023 {[}gedruckt{]}

In diesem Buch sind Beiträge von Bibliothekar*innen aus den USA und
Kanada versammelt, die sich allesamt als Trans oder Gender Diverse
beschreiben. Inhaltlich und stilistisch konnten die Autor*innen selber
bestimmen, worüber sie schreiben wollten, aber die meisten wählten die
Form von Essays und Berichten, um in ihnen über ihre eigenen Erfahrungen
in der bibliothekarischen Ausbildung und im Bibliotheksalltag
nachzudenken. Die Texte richten sich in den meisten Fällen an andere
(potentielle) Bibliothekar*innen, die sich selber in diesem Spektrum
verorten. Einige Texte sprechen auch direkt Bibliotheken, das
Bibliothekswesen als Ganzes oder Ausbildungseinrichtungen an, vor allem
um Forderungen zu stellen. Aber hauptsächlich vermittelt das Buch – explizit gewollt – den Eindruck von und für Kolleg*innen mit den
gleichen Erfahrungen geschrieben zu sein, schon um ihnen zu zeigen, dass
sie mit ihren Erfahrungen nicht alleine sind.

Man könnte vermuten, dass der Rezensent (männlich, cis, heterosexuell)
nicht die richtige Person wäre, um dieses Buch hier anzuzeigen. Aber die
Herausgeber*innen betonen in der Einleitung extra, dass es auch für das
ganze Bibliothekswesen gedacht sei – es sollte genutzt werden, um
zuzuhören; um zu verstehen, wie die Erfahrungen von Kolleg*innen, die
Trans oder Gender Diverse sind, tatsächlich sind und auch, um das
gesamte Bibliothekssystem zu ändern. Hauptsächlich sollte das
Bibliothekswesen – so wieder die Herausgeber*innen – lernen, dass die
Betroffenen wütend sind, weil es zwar grosse Versprechungen dahingehend
gibt, dass Bibliotheken safe spaces seien und dass das Bibliothekswesen
als Ganzes Diversität fördern würde, aber das diese Versprechen oft
nicht eingehalten werden. Vielmehr seien heteronormative Strukturen und
Verhaltensweisen sowie transphobe Haltungen weiterhin vorhanden und
viele offizielle Positionen von Bibliotheksverbänden oder Bibliotheken
würden wenig mehr sein als Lippenbekenntnisse.

Der Rezensent kann selbstverständlich nur aus seiner Position werten, ob
das Buch dieses Ziel erreicht. (Ob das Ziel, dass sich Trans und Gender
Diverse Personen im Bibliothekswesen angesprochen fühlen und ermutigt,
ihre eigenen Erfahrungen zu reflektieren, erreicht wird, müssen diese
beantworten.) Hierzu ist die Wertung aber zwiespältig: Ein grosser Teil
der Beiträge vermittelt diese Wut nämlich nicht. Insbesondere am Beginn
des Buches schreiben Autor*innen immer wieder, dass sie im Alltag von
anderen Kolleg*innen Unterstützung erhalten würden und die Erfahrungen
im Bibliothekswesen grösstenteils positiv wären. Das gilt nicht für alle
Texte. Einige sind sehr explizit darin, vor allem transphobes Verhalten
von Lehrpersonen und anderen Bibliothekar*innen zu benennen sowie darauf
hinzuweisen, wenn die angeblichen Ziele von Bibliotheken und
Ausbildungseinrichtungen, Diversität zu fördern, mit den realen
Strukturen zusammenstossen, die zum Beispiel immer wieder in technischen
Systemen eine klare geschlechtliche Markierung von Personen erzwingen.
Aber eingebettet in die vielen positiv gestimmten Texte vermittelt sich
eher das Bild, dass die Bibliotheken zumindest in den USA und Kanada
grosse Schritte hin zum \enquote{safe space librarianship} gemacht
haben, wenn auch noch eine Anzahl von Einzelpersonen dem entgegensteht.
(Allerdings erwähnen die Herausgeber*innen auch, dass eine ganze Reihe
von angedachten Texten nicht geschrieben wurden, weil sich
Bibliothekar*innen nicht sicher fühlten, diese zu publizieren. Auch in
den Texten wird mehrfach vermittelt, dass sie extra anonymisiert wurden,
um nicht auf die jeweiligen Autor*innen oder die im jeweiligen Text
besprochenen Einrichtungen zurückgeführt werden zu können. Das
vermittelt am Ende aber einen vielleicht positiveren Eindruck, da
\enquote{positive} Texte so viel eher veröffentlicht wurden.) Im Ganzen
vermittelt das Buch den Eindruck, dass es wichtig ist, diese Schritte
proaktiv weiterzugehen – und beispielsweise nicht darauf zu warten, dass
Transpersonen im Bibliothekswesen sich erst beschweren, bevor sich etwas
ändert –, aber auch, dass schon ein grosser Teil des Weges zurückgelegt
wurde. (ks)

\begin{center}\rule{0.5\linewidth}{0.5pt}\end{center}

Priestner, Andy ; Martin, Marisa (edit.) (2024). \emph{User Experience
in Libraries: Yearbook 2024}. Lincolnshire: UX in Libraries, 2024
{[}gedruckt{]}

\enquote{UX in Libraries} ist ein Sammelbegriff für das, was im
Deutschen wohl eher \enquote{Nutzer*innen\-for\-schung} genannt wird: Der
Einsatz von mehr oder minder klar definierten Methoden in Bibliotheken,
um zu erfahren, wie Nutzer*innen bestimmte Angebote (Services,
Schulungen, den Raum Bibliothek und so weiter) wahrnehmen und wie diese
Angebote verändert werden sollten. Die Methoden stammen teilweise aus
der Forschung (beispielsweise Interviews, Mapping, Beobachtungen),
teilweise aus der Marktforschung (unter anderem Rapid Prototyping oder
Sketching Exercises). Es geht dabei immer darum, Daten zu erheben, mit
denen Entscheidungen in der Bibliothek getroffen werden können und nicht
um wissenschaftliche Fragestellungen. Insoweit geht es nicht darum, ob
die Methoden valide oder die Ergebnisse reproduzierbar sind, sondern
nur, ob sie nutzbare Ergebnisse hervorbringen. Ausserdem geht es
praktisch immer darum, Nutzer*innen einer Bibliothek zu befragen. Die
Nutzung von Ergebnissen aus anderen Bibliotheken, der Vergleich zwischen
Bibliotheken oder aber die Einbettung von Studien und Ergebnissen in die
Fachliteratur ist praktisch nicht Teil von \enquote{UX in
Libraries}-Projekten.

Nichtsdestotrotz gibt es um das Schlagwort \enquote{UX in Libraries}
eine grösstenteils englisch-sprachige Community (die aber auch
Kolleg*innen aus skandinavischen Ländern, den Niederlanden oder
Deutschland umfasst). Ein wenig getrieben wird sie offenbar von Andy
Priestner, der dieses Buch mit herausgegeben hat, zudem 2021 schon ein
Handbook publizierte, in welchem verschiedene Methoden vorgestellt
werden, und ausserdem als Berater für Bibliotheken tätig ist, welche
diese Methoden einsetzen. Aber sie umfasst fraglos weitere Kolleg*innen
in verschiedensten Bibliotheken.

Seit 2016 – mit Unterbrechung von 2020–2021 – findet jährlich eine
Tagung dieser Community statt (UXLibs), welche jeweils in einem
Tagungsband dokumentiert wird. Das vorliegende Buch ist der Band für das
Treffen 2023 in Brighton, UK. Hauptsächlich sind hier die gehaltenen
Referate sowie Berichte über Workshops versammelt. Diese geben einen
Einblick darin, was für Projekte in Bibliotheken durchgeführt werden.
Sinnvoll ist dies für weitere Bibliotheken, die nach möglichen Methoden
für vergleichbare Projekte suchen. Sichtbar ist, dass es eine lebendige
Community ist und dass Bibliotheken von den Ergebnissen der Projekte
profitieren. Aber gleichzeitig werden auch die Grenzen dieses Ansatzes
klar: (1) In jeder Bibliothek scheinen die Projekte jeweils neu
aufgesetzt und nur für die jeweils lokale Bibliothek ausgewertet zu
werden. Ein gemeinsamer Aufbau von Wissen scheint nicht stattzufinden.
(2) Die Projekte tendieren alle dazu, Nutzer*innen direkt einzubinden.
Das gilt in der Community als grundsätzlich positiv, lässt für
Aussenstehende aber schnell den Eindruck aufkommen, dass hier Mitarbeit
mehr oder minder erzwungen wird. Nutzende sollen ständig irgendwas
zeichnen, ausprobieren oder mit Legosteinen basteln. (3) Eine wirkliche
Methodenreflektion oder Theoriebildung findet nicht einmal ansatzweise
statt. Bei den Methoden scheint es teilweise wichtiger zu sein, noch
eine neue zu erfinden, als darüber nachzudenken, was die Möglichkeiten
und Grenzen der schon benutzten Methoden sind. Die Ergebnisse der
einzelnen Bibliotheken werden, wie gesagt, auch nie zusammengeführt, um
zum Beispiel allgemeine Aussagen über Nutzer*innen in Bibliotheken zu
generieren (und anschliessend zu prüfen).

In einer ganzen Anzahl der Beiträge im Band wird explizit auf Fragen der
(mangelnden) Diversität der befragten Nutzer*innen eingegangen. Die
Kolleg*innen in den Bibliotheken machen sich Gedanken dazu, wie sie eine
grössere Vielfalt von Personen erreichen können. (Allerdings, wie
gesagt, ohne über die verwendeten Methoden selber nachzudenken, obgleich
auch die einen Einfluss darauf haben könnten, wer teilnimmt und wer
nicht.)

Sicherlich lohnt es sich, von Zeit zu Zeit wahrzunehmen, was in dieser
Community unternommen wird. Es ist nicht die einzige Community von
Bibliothekar*innen, die sich mit Fragen der Nutzer*innenforschung
beschäftigt – zu nennen wäre auch die um die Zeitschrift \emph{Evidence
Based Library and Information Practice} als mehr wissenschaftlich
orientierte –, aber eine der lebendigsten. Das wird mit diesen
Tagungsbänden ermöglicht. Das Buch selber vermittelt mit seinem Layout
und Inhalt – insbesondere seitenlangen Bilderstrecken von social events
der Konferenz – aber auch schnell den Eindruck, dass die Community vor
allem für Kolleg*innen offen ist, die solche sozialen Interaktionen und
den Fokus auf praktische Fragen mögen sowie sich nicht scheuen, in
Workshops mit Lego zu arbeiten. Also: Es ist nicht für alle. (Der Rezent
selber ist davon zum Beispiel vollkommen abgeschreckt und wird gewiss
nicht zu der im Januar 2025 stattfindenden UXLibs9 nach Liverpool
fahren. Aber andere Personen werden sich durch das Yearbook vielleicht
genau dazu angespornt fühlen.) (ks)

\begin{center}\rule{0.5\linewidth}{0.5pt}\end{center}

Barbakoff, Audrey ; Lenstra, Noah (2024). \emph{The 12 Steps to a
Community-Led Library}. Chicago: ALA Editions, 2024 {[}gedruckt{]}

In den englischsprachigen Bibliothekswesen gibt es, mehr noch als im
DACH-Raum, eine Reihe von Berater*innen, die oft mit dem jeweils
gleichen Programm oder Angebot Bibliotheken bei deren Entwicklung
unterstützen. Wenn sie erfolgreich sind, können sie dies teilweise
hauptberuflich betreiben. In der vorstehenden Besprechung ist der
genannte Andy Priestner ein solcher Berater. Die beiden Autor*innen
dieses Buches hier sind für ihr Programm und für die USA ebenfalls als
Berater*innen tätig, wobei Noah Lenstra zudem an der University of North
Carolina arbeitet. Ihr Angebot besteht darin, Bibliotheken dabei zu
unterstützen, partizipativer zu werden und dabei explizit
Personengruppen einzubeziehen, die von Bibliotheken viel weniger
erreicht werden.

Das vorliegende Buch ist praktisch dieses Beratungsangebot in
schriftlicher Form. Grundsätzlich geht es darum, Bibliotheken dahin zu
bringen, Partizipation als die Abgabe von Macht zu verstehen:
Nutzer*innen sollen die Entwicklung von Bibliotheken und deren Angeboten
direkt mitbestimmen können und nicht nur – wie dies laut den Autor*innen
heute der Normalfall wäre – mit ihren Meinungen einbezogen werden, aber
ohne die Macht, Entscheidungen zu treffen. Der Grossteil des Buches geht
darauf ein, wie ein solches Ziel konkret in Bibliotheken umgesetzt
werden kann. Es ist sehr praxisorientiert und thematisiert
beispielsweise, was sich für das Personal ändert, welche Formen von
Workshops, Partizipationsmöglichkeiten, Evaluationen des Erfolgs dieser
Anstrengungen und so weiter existieren. Der inhaltliche Teil – in
welchem neben dem Prinzip Partizipation auch die Grundidee von
Diversität erläutert wird – ist dagegen recht kurz. Ganz offensichtlich
ist das Buch aus der Beratungsarbeit der beiden Autor*innen entstanden
und enthält auch viele konkrete Beispiele.

Allerdings: Die Sprache ist gewöhnungsbedürftig. Stellenweise liest sich
das Buch so, als würde zu Kindern gesprochen und nicht zu ausgebildeten
Bibliothekar*innen. Zudem wird, warum auch immer, von der ersten
Autor*in durchgehend als abwesende Person gesprochen (im Sinne von:
\enquote{Dr.~Barbakoffs Forschung zeigte dies und das.}). Zudem ist
gegen das Ziel, Partizipation und Diversität im Bibliothekswesen zu
fördern, nichts einzuwenden. Allerdings hat man beim Lesen schnell den
Eindruck, dass alles schon mehrfach und auch reflektierter gelesen zu
haben, wenn auch teilweise unter anderen Schlagwörtern. Es ist eine
Handreichung, die vor allem Bibliotheksleitungen bei der Umsetzung
dieser Ziele helfen kann, aber inhaltlich kein originärer Beitrag. (ks)

\begin{center}\rule{0.5\linewidth}{0.5pt}\end{center}

Weber, Jürgen (2024). \emph{Sammeln nach 1998: Wie Provenienzforschung
die Bibliotheken verändert}. (Phänomenologie der Bibliothek:
Redescriptions, 1). Bielefeld: transcript Verlag, 2024,
\url{https://doi.org/10.14361/9783839472248}

Mit diesem Band wurde im transcript Verlag die neue Reihe
\enquote{Phänomenologie der Bibliothek: Redescriptions} eröffnet. Es ist
noch nicht klar, was diese Reihe inhaltlich anstrebt. Aber sie soll
offenbar spezifisch bibliothekswissenschaftlich sein, was zu begrüssen
ist, schon weil damit der transcript-Verlag zu den sehr wenigen Verlagen
hinzutritt, in denen im DACH-Raum über bibliothekswissenschaftliche und
bibliothekarische Themen publiziert wird.

Was das Buch darstellt, ist aber ein kein eigenständiges Werk, sondern
eine Zusammenstellung von Artikeln und Buchkapiteln, welche Jürgen Weber
in den letzten rund 15 Jahren anderswo (recht oft in der
\emph{Zeitschrift für Bibliothekswesen und Bibliographie})
veröffentlichte. Sie sind teilweise aktualisiert, aber für dieses Buch
selber wurde nur die Einleitung neu geschrieben. Zudem beziehen sich die
meisten Texte konkret auf Bibliotheken aus Weimar, vor allem auf die
Herzogin Anna Amalia Bibliothek, an welcher Weber auch tätig ist.
Zumeist beschäftigen sich diese mit Fragen der Provenienzforschung,
inklusive der Beschreibung einzelner Restitutionsfälle sowie
Überlegungen dazu, ob Provienznachweise zum Normalfall im
bibliothekarischen Bestandsmanagement werden sollten. Nur zwei Artikel,
die versuchen, Sammlungen und Sammlungsstruktur von Bibliotheken
theoretisch zu fassen, fallen aus diesem Rahmen. Allerdings: Diese
beiden Texte arbeiten mit jeweils unterschiedlichen theoretischen
Ansätzen, die sich nicht ergänzen. In anderen Beiträgen gibt es weitere
Verweise zu Theorien des Sammelns (vor allem aus der Forschung um Museen
und Archive). Das öffnet immer wieder andere Blickwinkel auf
Bibliotheksbestände, aber gleichzeitig fehlt dadurch ein umfassendes
Konzept, mit dem eine konkrete Frage – zu erwarten wäre die im Titel des
Buches genannte \enquote{Wie Provenienzforschung die Bibliotheken
verändert} – bearbeitet würde. (ks)

\hypertarget{bibliotheks--und-buchgeschichte}{%
\subsection{3.2 Bibliotheks- und
Buchgeschichte}\label{bibliotheks--und-buchgeschichte}}

Röhner, Barbara (2016). \emph{Von Reproduktionsausstellungen zum
Bildverleih: Ideen- und Entwicklungsgeschicht von Artotheken in der
DDR.} (Hallesche Beiträge zur Kunstgeschichte, 12) Halle an der Saale:
Universitätsverlag Halle-Wittenberg, 2016 {[}gedruckt{]}

Artotheken sind Einrichtungen, die Kunst verleihen. Im DACH-Raum sind
heute einige von ihnen in Öffentlichen Bibliotheken angesiedelt, andere
sind alleinstehende Institutionen. Es gab aber eine eigene Geschichte
dieser Einrichtungen in der DDR, wo sie fast alle Abteilungen von
Bibliotheken waren. Ab den 1970er Jahren waren diese recht gut
etabliert, inklusive Artotheken in Gewerkschaftsbibliotheken von
Grossbetrieben und Bibliotheken der Nationalen Volksarmee, aber auch
durch die Betreuung von Artotheken als eigener Arbeitsbereich durch das
Zentralinstitut für Bibliothekswesen. Fast alle diese Einrichtungen
wurden nach 1989 geschlossen. Das Buch arbeitet diese Geschichte auf. Es
stammt aus der Kunstgeschichte, was die gesamte Darstellung massiv
beeinflusst.

Eine Besonderheit der Artotheken in der DDR war, dass sie – bis auf
Ausnahmen – keine Originalkunstwerke erwarben und verliehen, sondern
Reproduktionen. (Röhner widerspricht dabei explizit Konrad Umlauf, der
sie deshalb gerade nicht als Artotheken gelten lässt.) Zuerst arbeitet
Röhner deshalb die Idee von Reproduktionen als Form der Kunstvermittlung
auf, wie sie in der DDR diskutiert wurde, mit einer Übersicht der
Diskussion in den Jahrzehnten zuvor. Dabei inkludiert werden andere
Debatten und Entwicklungen in der Kulturlandschaft der DDR.
Anschliessend wird auf die Geschichte der Kunstreproduktionen und der
(wenigen) Verlage eingegangen, die diese in der DDR produzierten. Und
zwar so ausführlich, dass erst auf Seite 216 (von 414 Textseiten vor den
Verzeichnissen und Anhängen) überhaupt die erste Artothek in der
Berliner Stadtbibliothek angesprochen wird.

Erst dann geht das Buch die Geschichte von verschiedenen Artotheken
durch, gestützt auf die zeitgenössische Literatur, zahlreiche
Archivmaterialien und Interviews. Selten in der Bibliotheksgeschichte
wird so tiefgehend und intensiv auf die Geschichte einzelner
bibliothekarischer Einrichtungen eingegangen, zumal noch mit einem
kritischen Blick. Das ist positiv hervorzuheben. Dadurch erhält man
Einblick in Überlegungen zu Artotheken und in deren konkrete Praxis.
Diese waren immer davon geprägt, lokal unterschiedliche Lösungen zu
finden, sowie sich gleichzeitig in den eigenen Veröffentlichungen
positiver darzustellen, als sie in der Realität wohl waren. Insbesondere
vergleicht Röhner Behauptungen in der Literatur mit Photographien aus
den tatsächlichen Artotheken und kann Differenzen zeigen. Gleichzeitig
wird sichtbar, dass die meisten grösseren Artotheken vom Engagement
einzelner Bibliothekar*innen abhingen – was aber auch dazu führte, dass
sie teilweise in Kleinstädten erstaunliche Grössen erreichen konnten.
Wieder mit Ausnahmen stand in ihnen der erzieherische beziehungsweise
pädagogische Aspekt im Mittelpunkt.

Wie gesagt, ist dies eine kunsthistorische Arbeit. Deshalb ist
verständlich, dass Röhner aber auch Fehler macht, wenn es um die
bibliothekarische Seite geht. Teilweise werden Personen falsch
zugeordnet (Hugo Heimann wird zum Beispiel Vertreter des
Volksbüchereiwesens genannt, obwohl die von ihm gegründete Bibliothek
explizit keine Volksbücherei sein sollte, sondern eine
Arbeiterbibliothek) und teilweise werden Arbeitsmittel falsch verstanden
(insbesondere interpretiert sie Auswahlverzeichnisse, die von
Bibliotheken als Erschliessungsmittel für ihre Nutzer*innen
herausgegeben wurden, als Hilfsmittel für den Bestandsaufbau anderer
Bibliotheken). Es ist im Einzelfall zu diskutieren, ob dies im Detail zu
falschen Aussagen führt. Im Ganzen aber ist dieses Buch ab Seite 216
eine übersichtliche, teilweise durch zu viele geschilderte Details aber
auch repetitive Geschichte.

Eine Frage, die allerdings nicht beantwortet wird, ist die, was
eigentlich das Problem mit Kunstreproduktionen – im Gegensatz zu
Originalen – sein soll, insbesondere wenn es um kunstpädagogische Ziele
geht. Der Fokus auf Kunstreproduktionen wird die ganze Zeit als
Besonderheit der Artotheken in der DDR angeführt und hat dann auch dazu
geführt, dass sie nach 1989 grösstenteils geschlossen wurden. Vielleicht
ist es für Röhner, die in der Kunstgeschichte arbeitet, keine
diskussionswürdige Frage, sondern Wissen, das als bekannt voraussetzt
werden kann. Aber gerade, weil nicht nur im Text selber, sondern auch
noch im letzten Absatz ihres Buches, auf Walter Benjamins \enquote{Das
Kunstwerk im Zeitalter seiner technischen Reproduzierbarkeit} verwiesen
wird, wäre eine Klarstellung zu wünschen gewesen. (ks)

\begin{center}\rule{0.5\linewidth}{0.5pt}\end{center}

Hinterthür, Bettina (2006). \emph{Noten nach Plan: die Musikverlage in
der SBZ/DDR - Zensursystem, zentrale Planwirtschaft und deutsch-deutsche
Beziehungen bis Anfang der 1960er Jahre}. (Beiträge zur
Unternehmensgeschichte, 23) Stuttgart: Franz Steiner Verlag, 2006
{[}gedruckt{]}

Im weiter oben besprochenen Buch von Barbara Röhner wird die Geschichte
der Verlage besprochen, welche in der DDR Kunstreproduktionen
veröffentlichten. In gewisser Weise als Kompagnon kann dieses Buch von
Bettina Hinterthür gelesen werden, welches ebenfalls ausgesprochen
umfangreich die Geschichte spezifischer Verlage in der DDR darstellt,
nämlich derjenigen, welche Notendrucke und musikwissenschaftliche
Literatur publizierten. Hinterthür beschränkt sich dabei auf die Jahre
bis Anfang der 1960er, aber in den Ereignissen und Ergebnissen gleicht
die Darstellung der von Röhner. (Gleichwohl ist das Buch von Hinterthür
über 500 Seiten stark. Für die gesamte Geschichte bis 1989/1990 hätte es
wohl eines tausendseitigen Werkes bedurft.) Es gab in der Sowjetischen
Besatzungszone nach Ende des Zweiten Weltkrieges eine Anzahl von
betreffenden Verlagen – teilweise mit zerstörten Produktionsanlagen,
aber teilweise auch mit gut erhaltenen Druckereien und Lagerbeständen – und gleichzeitig Versuche, erst der KPD, dann der SED, diese in die
Kulturpolitik einzubinden. Dies funktionierte nicht sofort, sondern erst
in einem langen Prozess, was teilweise damit zu tun hatte, dass die
Partei – im Gegensatz zur Literatur – für die spezifischen Bereiche
Musik und Kunst vor 1945 keine eigenständigen Konzeptionen entwickelte
hatte, sondern diese erst im Laufe der Zeit erarbeitete. Teilweise
widersetzten sich aber auch die Verlagsinhabenden oder -mitarbeitenden
selber der Vereinnahmungsversuche. Eine Besonderheit der Musikverlage
war zudem, dass nicht nur Parallelverlage in Westdeutschland gegründet
wurden, die den Anspruch erhoben, die eigentlichen Rechtsnachfolger der
Verlage zu sein, sondern dass ein Grossteil der Einnahmen der Verlage
aus den Aufführungsrechten und nicht dem Verkauf der Notendrucke
stammte.

Hinterthür schildert auf der Basis umfangreicher Archivrecherchen die
Entwicklungen der Verlage, inklusive der Auseinandersetzungen um
Druckmöglichkeiten, Papierzuteilungen, kulturelle Kampagnen der SED und
Ansprüchen der Parallelverlage. Dies ist einigermassen deckungsgleich zu
der Geschichte, die Röhner darstellte. Was bei Hinterthür heraussticht,
ist, dass auf der einen Seite wohl alle vorliegenden Archivalien
ausgewertet und kleinteilig dargestellt wurden (nicht selten werden
einige Abschnitte lang die Interaktionen einzelner Personen
dargestellt), was das Lesen des Buches teilweise schwer macht, aber sich
Hinterthür auf der anderen Seite mit einer eigenständigen Bewertung fast
vollständig zurückhält. Zum Teil wird direkt aus den Archivalien
zitiert, als ob diesen einfach vertraut werden könnte, obgleich es sich
fast immer um politisch aufgeladene Dokumente handelt. (ks)

\begin{center}\rule{0.5\linewidth}{0.5pt}\end{center}

Schneider, Johannes Ulrich (2024). \emph{Andrew Carnegies Bibliotheken:
Über Moderne und Öffentlichkeit}. (Themen, 110) München: Carl Friedrich
von Siemens Stiftung, 2024 {[}gedruckt{]}

In der Geschichte der anglo-amerikanischen Public Libraries (in den USA,
Kanada, Grossbritannien, Irland und einigen anderen ehemaligen
britischen Kolonien) ist Andrew Carnegie (1835--1919) eine der
bestimmenden Personen in der ersten Hälfte des 20. Jahrhunderts. Der
Stahlmagnat war und ist für seine Philantrophie bekannt, wobei er einen
Schwerpunkt auf die Öffentlichen Bibliotheken legte und den Bau von rund
2.500 von ihnen finanzierte. Relevant war dabei nicht nur, dass er viel
Geld gab, sondern dass diese Förderung nach einiger Zeit zu einem
eigenen, effizient organisierten Unternehmen wurde, unter anderem mit
eigenen Richtlinien für die Förderung. Die bekannteste dieser Regeln
ist, dass Carnegie voraussetzte, dass die jeweilige Gemeinde die
Finanzierung der Unterhaltskosten einer Bibliothek garantieren musste,
welche jährlich mindestens 10\,\% der Fördersumme umfassen musste, die
von Carnegie für die Errichtung und Einrichtung der jeweiligen
Bibliothek zur Verfügung gestellt wurde. Mit diesen Regelungen gilt
Carnegie auch als Person, die die Philantrophie in das
Industriezeitalter überführte.

Carnegie, \enquote{seine} Bibliotheken, aber auch andere Aspekte seiner
Philantrophie (inklusive des Widerstands gegen diese, unter anderem von
der organisierten Arbeiter*innenschaft, die ihn als
gewerkschaftsfeindlichen Industriellen ansah), sind schon oft Thema von
Beiträgen und Monographien gewesen. Zahlreiche Aspekte seines Lebens und
Wirkens sind beschrieben, ausgeleuchtet und auch kritisiert worden. Das
vorliegende Buch stellt jetzt einen ausgearbeiteten Vortrag dar, welchen
Johannes Ulrich Schneider -- bis zu seiner Pensionierung 2022 unter
anderen Direktor der Universitätsbibliothek Leipzig, dort aber jetzt
auch weiter Professor für Philosophie -- im Rahmen eines Förderprogramms
der Carl Friedrich von Siemens Stiftung gehalten hat. Im Rahmen dieses
Programms hatte er zu Carnegie geforscht. Das Buch ist deshalb aber kein
Forschungsbericht, sondern spricht eher in lockerer Folge
unterschiedliche Aspekte Carnegies und der Carnegie Libraries an. Ein
wenig -- obgleich es eine wissenschaftliche Publikation ist, in der
beispielsweisen Aussagen auch immer nachgewiesen sind und Argumente
gebildet werden -- liesst es sich wie ein lockeres Gespräch über das
Thema.

Die Frage ist nun, was dieses Buch zu den schon vorhandenen Beiträgen
hinzufügt. Heraus ragt, dass es eine explizit deutschsprachige
Publikation ist, während die meisten anderen aus dem
anglo-amerikanischen Raum stammen und in Englisch vorliegen. Schneider
schaut ein wenig von aussen auf Carnegie, nicht aus Gesellschaften, die
direkt von dessen Tätigkeiten geprägt wurden. Ansonsten hat das Buch
aber keine explizite Forschungsfrage. Es ist keine reine Biographie,
keine explizite Bibliotheksgeschichte, sondern ein wenig die Versammlung
verschiedener Punkte. So insistiert Schneider darauf, Carnegies
Philantrophie breiter zu schildern, als nur auf die Bibliotheken
bezogen. Unter anderem war Carnegie stark in der Friedensbewegung
engagiert, aber mit einer spezifischen, die liberal-kapitalistische
Gesellschaftsordnung in den Mittelpunkt stellenden Position.
Gleichzeitig betont Schneider, dass man Carnegie in seiner Zeit verorten
müsse, weder als unhinterfragbaren Held noch als nur kritisch zu sehende
Persönlichkeit.

Dadurch, dass das Buch viele Themen anspricht, können wohl alle
Leser*innen die Punkte finden, die sie am meisten interessieren. Für den
Rezensenten hier waren dies die Anmerkungen zu den Grenzen der
Bibliotheksgeschichte. Schneider betont, dass gerade in der
englischsprachigen Bibliotheksgeschichte die Geschichte der Public
Libraries oft mit der Förderung Carnegies beginnt, obgleich es zuvor und
zeitgleich auch andere Formen von öffentlich zugänglichen Bibliotheken
gab, die zur Geschichte der Public Libraries gehören. Gleichzeitig
argumentiert Schneider mehrmals, dass es in der Bibliotheksgeschichte
eigentlich nicht möglich wäre, zu untersuchen, wie Nutzer*innen die
Bibliotheken tatsächlich wahrnahmen, was sie wie gelesen haben und so
weiter. Alles, was zu schildern wäre, seien die Entwicklungen von
Gebäuden, Beständen und Diskursen. (Ein Thema, das er auch für die
zeitgenössischen Bibliotheken schon besprochen hat, vergleiche
Schneider, Johannes Ulrich (2018). \emph{Lesen als Arbeiten in der
Bibliothek}. In: Bonte, Achim ; Rehnolt, Julian (Hrsg.) (2018).
\emph{Kooperative Informationsinfrastrukturen als Chance und
Herausforderung}. Berlin: De Gruyter Saur, 2018: 277-288.) (ks)

\begin{center}\rule{0.5\linewidth}{0.5pt}\end{center}

Soilihi Mzé, Hassan (2023). \emph{Geöffnet - gelenkt - umgebaut:
Universitätsbibliothek Leipzig, Deutsche Bücherei und Leipziger
Stadtbibliothek zwischen institutioneller Reorganisation und politischer
Instrumentalisierung (1945-1968/69)}. Leipzig: Leipziger
Universitätsverlag, 2023 {[}gedruckt{]}

Diese Arbeit, gleichzeitig eine geschichtswissenschaftliche
Dissertation, schildert für die drei im Titel genannten Bibliotheken
deren Entwicklung in der Sowjetischen Besatzungszone und der frühen DDR.
In ihr wird postuliert, dass man diese drei als Beispiele für
Entwicklungen Wissenschaftlicher Bibliotheken in der DDR (zumindest bis
1968/69) ansehen kann. Dieses Postulat ist schwer zu belegen, da es
bislang keine etablierte Übersicht zur Bibliotheksgeschichte der DDR
gibt, man also schwerlich sagen kann, ob die Entwicklungen in Leipzig
aus denen im restlichen Land herausstachen oder nicht. Sicherlich, die
Deutsche Bücherei -- heute die Deutsche Nationalbibliothek, Standort
Leipzig -- war eine sehr besondere Bibliothek, die eigentlich nur mit
anderen Nationalbibliotheken verglichen werden kann. Aber die
Universitätsbibliothek Leipzig und die Stadtbibliothek (die heute als
wissenschaftliche Bibliothek nicht mehr existiert, was ein Thema des
Buches ist), hatten ihre Parallelorganisationen in anderen Städten.

Ein Thema, welches Soilihi Mzé umtreibt, ist der Umbau der genannten
Bibliotheken hin zu \enquote{sozialistischen Bibliotheken}. Dieses
Postulat, dass die Bibliotheken in der DDR zu \enquote{sozialistischen
Bibliotheken} werden müssten, wurde spätestens mit der Gründung der DDR
explizit vertreten. Gleichwohl war es -- wie unter anderem in diesem
Buch gezeigt wird -- nie ganz einfach zu bestimmen, was darunter zu
verstehen sei. Die Bibliotheken sollten auf die Aufgaben ausgerichtet
werden, eine sozialistisch Gesellschaft aufzubauen und dann in dieser
Gesellschaft zu funktionieren. Aber hiess dies zum Beispiel, dass
Bibliotheken möglichst effektiv, in zentralisierten Netzwerken und unter
einheitlicher Kontrolle arbeiten sollten? Oder hiess es, dass die
Bibliothekar*innen möglichst ideologisch geschult sein müssten? Hiess
es, dass alle wichtigen Funktionen mit Personen besetzt sein sollten,
die der KPD nach 1946 der SED, angehörten? Im Buch wird beschrieben,
dass dies jeweils neu verstanden wurde und deshalb auch immer wieder mit
unterschiedlichen Ansätzen umzusetzen versucht wurde.

Grundsätzlich geht Soilihi Mzé, basierend auf noch vorhandenen Akten und
Unterlagen der drei Bibliotheken, die strukturellen Veränderungen durch,
die in den untersuchten Jahren stattfanden. Dem Titel getreu folgt die
Arbeit einem Aufbau von drei Perioden: Erst die Säuberung von Beständen
und Personal nach dem Nationalsozialismus, mit anschliessender
Wiedereröffnung der Bibliotheken (\enquote{geöffnet}), dann die
Übernahme der Leitungsfunktionen durch SED-Mitglieder bis Ende der
1950er Jahre (\enquote{gelenkt}) und schliesslich der Umbau der
Bibliotheken bis zum Erlass der \enquote{Bibliotheksverordnung der
Deutschen Demokratischen Republik} 1968 (\enquote{umgebaut}), wobei der
Erlass von Soilihi Mzé als Anfang einer neuen Periode gewertet wird,
deren Geschichte noch zu schreiben sei. Dabei geht das Buch jeweils
nacheinander die Entwicklungen in den drei Bibliotheken durch. Soilihi
Mzé interessiert, wie die Deutsche Bücherei als Bibliothek mit
gesamtdeutschen Anspruch agierte, aber gleichzeitig in die Politik der
SED (welche in diesen Jahrzehnten wechselte von der Betonung einer
Zusammengehörigkeit von BRD und DDR hin zur Betonung der
Eigenständigkeit der DDR) eingebunden wurde. Bei der
Universitätsbibliothek geht es sehr um den Umbau der Zweigliedrigkeit
(also dem autonomen Bestehen von Instituts- und Fachbibliotheken neben
der Universitätsbibliothek) zum eingliedrigen System (bei dem die
Institutsbibliotheken, inklusive der Hoheit über Bestand und Etat, der
Universitätsbibliothek und dessen Leiters unterstehen sollten); ein
Umbau, der allerdings bis 1968 nicht vollständig gelang. Bei der
Leipziger Stadtbibliothek, die sich als Wissenschaftliche Bibliothek der
Bürgerschaft -- betrieben neben der Volksbücherei -- verstand und
beispielsweise Bestände von Inkunabeln betreute, steht vor allem die
Auflösung derselben im MIttelpunkt der Darstellung sowie ihre
Zusammenführung mit der Volksbücherei.

Soilihi Mzé interpretiert all dies immer wieder auch als ideologisch
motivierte Aktivitäten. Beispielsweise sei es bei der Auflösung der
Stadtbibliothek darum gegangen, dem \enquote{Bürgertum} der Stadt einen
identitätsbildenden Kristallisationspunkt zu nehmen. Es ist eine
Schwierigkeit dieser Arbeit, dass Soilihi Mzé dafür zwar immer Argumente
bringen kann, also oft explizit zeigt, dass es tatsächlich ein erklärtes
Ziel der SED war, \enquote{bürgerliches Denken} auszuschalten. Aber
gleichzeitig drängen sich immer wieder Zweifel auf: Zweigliedrige
Universitätsbibliothekssysteme zu eingliedrigen umzubauen oder aber
historisch angelegte Stadtbibliotheken mit Volksbüchereien zu vereinen
oder aber sie ganz aufzulösen, war keine Eigenheit der DDR. Das
passierte in der BRD, in Österreich und der Schweiz auch -- ebenso, wie
es da immer wieder neu scheiterte und bis heute immer wieder neu
angegangen wurde. Grundsätzlich wird in der Arbeit zudem durchgehend
behauptet, dass ein Grund für die ganzen Aktivitäten die Orientierung an
der Sowjetunion gewesen wäre. Gerade dies wird aber nicht explizit
gezeigt. Ständig drängt sich die Frage auf, ob die DDR mit den Umbauten
nicht auch Trends folgte, die sich aus gesellschaftlichen Veränderungen
ergaben -- Stichwort: Ausweitung von Bildungsmöglichkeiten -- und die
mit ideologischen Zielen in Übereinstimmung gebracht wurden. Zwar wird
für die Universitätsbibliothek erwähnt, dass es ähnliche Bestrebungen
schon vor 1933 und nach 1945 auch anderswo gab, aber so interpretiert,
als hätte die SED diese Bestrebungen nur okkupiert, um eigene Ziele
durchzusetzen.

Was in dem Buch zu lernen ist, sind die grundsätzlichen Veränderungen in
den drei genannten Bibliotheken: Wann wurden Entscheidungen getroffen?
Welche Entscheidungen wurden getroffen? Welche Personen wurden wann
entlassen, welche befördert? Das ist alles, teilweise sehr kleinteilig,
dargestellt. Am meisten wird -- wohl weil darüber mehr Akten vorlagen --
über die Zeit der Säuberung der Bestände nach dem Nationalsozialismus
und der Wiedereröffnung der Bibliotheken berichtet. Gleichzeitig
beschreiben grosse Teile des Buches auch vor allem
Personalentwicklungen: Wer wurde abgesetzt? Wann etablierte die SED
eigene Kader in den Bibliotheken und Bibliotheksleitungen? All das nimmt
erstaunlich viel Platz im Text ein. Entwicklungen des Bestandes oder der
konkreten Nutzung geraten dabei immer wieder in den Hintergrund.
Teilweise vermittelt das Buch den Eindruck, es sei es einfach nur darum
gegangen, die Entscheidungsfunktionen in den Bibliotheken zu
zentralisieren und dann die wichtigen Positionen mit Mitgliedern der SED
zu besetzen. Es bleibt am Ende eine Unsicherheit zurück, ob dies schon
\enquote{sozialistische Bibliotheken} ausgemacht hat oder ob nicht auch
mehr auf andere Aspekte hätte eingegangen werden müssen, um den realen
Unterschied zwischen diesen Bibliotheken (und den Entwicklungen in
ihnen) und denen in anderen Ländern darzustellen. (ks)

\begin{center}\rule{0.5\linewidth}{0.5pt}\end{center}

Knowlton, Steven A. ; Pozzi, Ellen M. ; Sly, Jordan S. ; Spunaugle,
Emily D. (edit.) (2024). \emph{Libraries without borders: New directions
in library history}. Chicago: ALA editions, 2024 {[}gedruckt{]}

Der \emph{Library History Round Table} ist eine Arbeitsgruppe in der ALA
– vergleichbar mit einer Sektion im \emph{Bibliosuisse} oder einer
Kommission im \emph{dbv} –, welche die Beschäftigung mit
Bibliotheksgeschichte vorantreibt. Dazu organisiert sie unter anderem
jährlich Seminare – praktisch kleine Konferenzen – und gibt die
Zeitschrift \emph{Libraries: Culture, History, and Society} heraus. Die
Vorträge auf den Seminaren und die Beiträge in der Zeitschrift
beschäftigen sich verständlicherweise zumeist mit der US-amerikanischen
Bibliotheksgeschichte. Aber der Round Table ist auch bemüht, diesen
Fokus zu erweitern. Seit einigen Jahren gibt es explizite Bemühungen,
die Geschichte marginalisierter Gruppen – also zum Beispiel
afroamerikanischer Bibliothekar*innen – sichtbar zu machen.

Das vorliegende Buch ist nun praktisch die Publikation der meisten
Vorträge, die auf dem Anfang 2020 abgehaltenen Seminar gehalten wurden.
Zwar wird in der Einführung betont, dass es sich nicht um die
Konferenzveröffentlichung handeln würde, aber es ist nicht klar, was
genau das heissen soll. Eventuell ist damit einfach gemeint, dass die
Beiträge noch überarbeitet wurden.

Ansonsten liest sich das Buch wie eine weitere Veröffentlichung der
Zeitschrift, auch wenn es eine eigenständige Monographie darstellt. Das
vorgebliche Thema, \enquote{libraries without borders}, wird nur in
einigen Beiträgen am Rande aufgegriffen, ansonsten aber ignoriert. Die
Beiträge beschäftigen sich mit sehr verschiedenen Themen und sind auch
sehr unterschiedlich. Als inhaltlich interessant hervorzuheben ist ein
Beitrag zur Geschichte von Bibliotheken der mariologischen Bewegung, die
von Klerus und Laienschaft getragen im frühen 20. Jahrhundert Maria in
den Mittelpunkt der katholischen Spiritualität stellen wollte und dazu
unter anderem zahllose Publikationen erstellte, die dann wiederum,
teilweise systematisch, oft aber auch unsystematisch, in Bibliotheken
gesammelt und erschlossen wurden. (Henry Handley: \enquote{\emph{Thank
you, father, for your grand cooperation'': Outreach and the Founding of
the Marian Library}, 27--54) Zudem zu erwähnen ist ein Beitrag, der sich
mit dem Phänomen von}Jahre zu spät zurückgegebenen Bibliotheksbüchern''
beschäftigt, die regelmässig in der Presse und bibliothekarischen
Publikationen erwähnt werden. In diesem wird danach gefragt, welchen
Diskursfiguren die Darstellung dieser Fälle folgt. (John DeLooper:
\emph{Better late than never: Stories of Long-Overdue Books}, 79--103).
(ks)

\begin{center}\rule{0.5\linewidth}{0.5pt}\end{center}

Lenhard, Philipp (2024). \emph{Café Marx: Das Institut für
Sozialforschung von den Anfängen bis zu Frankfurter Schule}. München:
C.H. Beck, 2024 {[}gedruckt{]}

Wie im Titel sichtbar, stellt dieses Buch die Gesamtgeschichte des
\emph{Instituts für Sozialforschung} dar, also der Einrichtung, die in
der Weimarer Republik gegründet wurde, um eine Aktualisierung
marxistischer Theoriebildung nach der 1918/1919 gescheiterten Revolution
ausserhalb der verschiedenen linken Parteien zu ermöglichen, die dann
nach 1933 mit mehreren Zweigstellen (Genf, Paris, London, New York,
Kalifornien) im Exil existierte und nach 1949 nach Frankfurt am Main
zurückkehrte, um (wieder) Institut der dortigen Universität zu werden.
Die Arbeit geht auf die Strukturen des Instituts, auf die prägenden
Persönlichkeiten und die wichtigsten Publikationen aus dem Institut
(also unter anderem die \emph{Dialektik der Aufklärung} und die
\emph{Negative Dialektik}) ein.

Relevant im Zusammenhang hier ist, dass in einem Kapitel (\emph{In der
Bibliothek: Geschlechterverhältnisse und soziale Hierarchien}, 102--116)
auch die Bibliothek des Instituts thematisiert wird, zumindest die,
welche bis 1933 im ersten Institutsgebäude existierte. Dabei wird der
wichtige Punkt gemacht, dass über die \enquote{wichtigen
Persönlichkeiten} des Instituts (Theodor W. Adorno, Max Horkheimer,
Friedrich Pollock, Felix Weil und andere – mit der Ausnahme von Hannah
Arendt, die mit dem Institut in Verbindung stand, aber sich auch bald
abwandte, praktisch alles Männer) schon sehr viel geforscht wurde, aber
über das Personal, welches das Institut mit ihrer alltäglichen Arbeit
ermöglichten, praktisch gar nicht. Lenhard führt dies vor allem anhand
der Bibliothek vor (im Laufe des Buches aber auch noch an anderen
Personen, beispielsweise dem Hauswart des Gebäudes). Dabei dargestellt
werden die Biographien dieser Mitarbeiter*innen, fast alles Frauen,
erwähnt wird aber auch, dass diese immer grosse Lücken enthalten. Man
weiss nicht viel über sie und auch nicht über ihre Arbeit. Über die
Bibliothek erfährt man in der Darstellung, dass sie gleichzeitig als
Archiv arbeitete. (Ein Ziel des Instituts war es damals, möglichst alle
Dokumente zur Geschichte der Arbeiterbewegung zu sammeln und zu
veröffentlichen, damit sie der Forschung zur Verfügung stünden.) Über
die restliche Bestandsarbeit ist wenig zu erfahren. Die Bibliotheken in
den \enquote{Zweigstellen} oder nach der Rückkehr des Instituts nach
Frankfurt werden nicht thematisiert. (ks)

\begin{center}\rule{0.5\linewidth}{0.5pt}\end{center}

Scarpatetti, Beat von (2022). \emph{Bücherliebe und Weltverachtung: Die
Bibliothek des Volkspredigters Heynlin von Stein und ihr Geheimnis}.
Basel: Schwabe Verlag, 2022 {[}gedruckt, OA-Version:
\url{https://doi.org/10.24894/978-3-7965-4473-6}{]}

Der Autor beziehungsweise Ersteller dieses Buches war seit den späten
1960er Jahren als Mediävist und Katalogisierer mittelalterlicher
Schriften in Basel, Paris und St.~Gallen aktiv. Zudem publizierte er,
aus einer christlich-theologischen Sicht, zum Thema Ökologie. Die
vorliegende Arbeit ist in gewisser Weise ein Spätwerk beziehungsweise
der Abschluss einer über 20-jährigen Obsession. (In der Festschrift zu
seinem 75. Geburtstag (Egli, Daniel (Hrgs.) ; Meyer, Kurt (Redak.) ;
Welti, Manfred (Redak.) (2016). \emph{Kultur und Ökologie: Festschrift
zum 75. Geburtstag Beat von Scarpatetti}. Binningen: Verein Ökogemeinde
Binningen, 2026) wird dargestellt, dass dies nur eine seiner Obsessionen
neben der Ökologie, Vertretung von \enquote{Autofreien}, Musik und
Handschriftenkatalogisierung war.)

Es geht ihm um die Bibliothek eines spätmittelalterlichen Theologen,
Heynlin von Stein, die in grossen Teilen weiterhin in der
Universitätsbibliothek Basel (in ihrer Funktion als Kantonsbibliothek)
erhalten ist. Heynlin war einerseits für seine Zeit, die zweite Hälfte
des 16. Jahrhunderts, einflussreich, weit gereist und erfolgreich. Er
war an verschiedenen Universitäten aktiv, in Paris an der Sorbonne sogar
Rektor. Später war er in Basel und im oberrheinischen Raum als Prediger
aktiv, wo er hauptsächlich zum Verzicht auf weltliche Aktivitäten und
Freuden sowie zur Hinwendung zu Glauben und Gebet aufrief. Obgleich auch
damit erfolgreich, zog er sich in seinen letzten Jahren als
Ordensmitglied in die \enquote{Kleine Klause} in Basel zurück. Nachdem
diese später aufgelöst wurde, ging deren Bibliothek an die
Kantonsbibliothek über, was deren heutigen Standort erklärt.

Der Autor dieses Buches hat sich schon länger mit Heynlin befasst. Unter
anderem erstellte er einen Lexikoneintrag über ihn. Hier hat er jetzt
einen vollständigen Katalog der Bibliothek erarbeitet -- einschliesslich
einiger Bücher, die nicht in Basel stehen --, inklusive einer
eingehenden Beschreibung der Bücher, teilweise Biographien ihrer Autoren
sowie eingehender Beschreibungen von Ausstattung und Annotationen. Das
ist nicht unberechtigt: Die Bibliothek war für ihre Zeit gross. 287
Signaturen werden hier nachgewiesen, die teilweise mehrere
zusammengebundene Werke umfassen. Die Bibliothek wurde aufgebaut, als
der Medienwandel hin zum Druck gerade stattfand -- sie enthält Drucke,
Handschriften und Werke des Autors selbst. Zudem wurden alle Bücher
sorgsam ausgestattet -- also nicht einfach möglichst billig gebunden --
und vom Autor aktiv genutzt, der in den meisten Annotationen
hinterlassen hat. (Beziehungsweise, wie Scarpatetti vermutet, von
anderen Personen im Auftrag Heynlins hinterlassen wurden.) Die
Bibliothek ist also eine historische Quelle, die hier der Forschung für
verschiedene Fragestellungen eröffnet wird.

Für die Druckgeschichte relevant ist -- was allerdings schon bekannt war
--, dass Heynlin als Rektor der Sorbonne die Erstdrucker Frankreichs
(Ulrich Gering, Martin Crantz und Michael Friburger) nach Paris holte,
um ihre Druckerei 1470 an der Universität selber einzurichten.

Scarpatetti stellt diesem Katalog eine rund 100-seitige Einführung
voraus. In dieser schildert er nicht nur die Bibliothek selber und
argumentiert dafür, sie weiter zu erforschen. Vielmehr nutzt er sie, um
in gewisser Weise alle Überlegungen und Forschungen zu Heynlin, die er
im Laufe der Jahre angesammelt hat, zu präsentieren. So diskutiert er
über lange Seiten, ob Heynlin eventuell der unehelicher Spross einer
Adelsfamilie war. Zudem geht er nicht nur auf die Inhalte der Bücher in
Heynlins Bibliothek ein, sondern verbindet sie mit einer Einführung in
das Thema \enquote{Weltabgewandheit} in der christlichen Theologie seit
der Spätantike. Am Ende verfolgt er dieses Thema weiter bis in die
Jetztzeit, um schliesslich wieder bei seinem anderen Thema, Ökologie und
Theologie, zu landen -- ohne auch nur noch eine Verbindung zu Heynlin
herzustellen. Zudem ist das ganze in einer sehr spezifischen Sprache
geschrieben, die in gewisser Weise dem Stereotyp eines weltabgewandten,
sich Forschung und Katalogisierung widmenden Mediävisten aus der Schweiz
entspricht: Es wird zum Beispiel vorausgesetzt, dass die Leser*innen
Deutsch, Französisch und Latein sprechen (nur an wenigen Stellen werden
die lateinischen Zitate übersetzt, die französischen gar nicht),
Dokumente werden \enquote{aufs Netz gestellt}, ständig wird von
\enquote{wir} gesprochen, wenn der Autor selber gemeint ist. Das hat
alles seinen eigenen Charme, aber in gewisser Weise vermittelt es den
Eindruck, der Autor selber könnte mit dem Titel seiner Arbeit,
\enquote{Bücherliebe und Weltverachtung}, beschrieben werden -- selbst,
wenn er eigentlich gegen diese theologische Tradition und aus
ökologischen Gründen für ein aktives Handeln in der Welt argumentiert.
(ks)

\hypertarget{weitere-wissenschaftliche-medien-konferenzberichte-abschlussarbeiten}{%
\section{4. Weitere wissenschaftliche Medien (Konferenzberichte,
Abschlussarbeiten)}\label{weitere-wissenschaftliche-medien-konferenzberichte-abschlussarbeiten}}

{[}Diesmal keine Beiträge{]}

\hypertarget{populuxe4re-medien-social-media-zeitungen-radio-tv}{%
\section{5. Populäre Medien (Social Media, Zeitungen, Radio,
TV)}\label{populuxe4re-medien-social-media-zeitungen-radio-tv}}

Kehlmann, Daniel (2024). \emph{\enquote{Wir fühlen nicht, was wir doch
wissen} -- Die Politik muss die Schöpfungskraft der Kunst, sie muss die
demokratische Gesellschaft vor der heranstürmenden Macht der künstlichen
Intelligenz (KI) schützen. Eine Rede.} In: Süddeutsche Zeitung, Nr. 154,
6./7. Juli 2024, S. 17,
\url{https://www.sueddeutsche.de/kultur/daniel-kehlmann-25-jahre-bundeskulturpolitik-kuenstliche-intelligenz-lux.a4cfns2UkoVT6NKGgFRtc}
{[}Paywall{]}

Abgedruckt ist hier eine Rede, die Daniel Kehlmann beim Festakt
\enquote{25+1 Jahre Bundeskulturpolitik} am 05.07.2024 im
Bundeskanzleramt hielt. Kehlmann kontextualisiert die rasante
Weiterentwicklung der Large Language Models -- das eigene, vier Jahre
alte Buch zum Thema wird bezeichnet als \enquote{ganz und gar und
profund veraltet \ldots{} es liest sich jetzt wie ein Text über die
ersten Eisenbahnen} -- und zeigt anhand klarer und erschreckend
realistischer Beispiele auf, welche gravierenden Auswirkungen der
unregulierte, alltägliche Einsatz von KI potentiell für unsere
Gesellschaft haben kann: \enquote{..., dass wir Desinformation in einem
Ausmaß erleben werden, gegen das alles Bisherige wie eine freundliche
Diskussion unter Gleichgesinnten aussieht.} Die Rede endet mit der
klaren Aufforderung an die Politik, zeitnah aktiv zu werden. (eb)

\begin{center}\rule{0.5\linewidth}{0.5pt}\end{center}

Christiane Peitz, Matthias Horx: \emph{Hat die Zukunft eine Zukunft,
Herr Horx?.} In: Tagesspiegel, 29.12.1999 SEITE 032 / Kultur

In einem Interview zur Jahreswende 1999/2000 betont Zukunftsforscher
Matthias Horx, dass es einen \enquote{digital backlash} geben und der
\enquote{Hype der totalen Computerisierung {[}...{]} zu Ende gehen}
wird. (bk)

\begin{center}\rule{0.5\linewidth}{0.5pt}\end{center}

{[}Schwerpunkt Bestandserhalt{]} Ouellette, Jennifer (2024). \emph{That
book is poison: Even more Victorian covers found to contain toxic dyes}.
In: arstechnica, 19.08.2024,
\url{https://arstechnica.com/science/2024/08/that-book-is-poison-even-more-victorian-covers-found-to-contain-toxic-dyes/}

Während des \enquote{viktorianischen Zeitalters} -- also der britischen
Kultur und Gesellschaft der Regierungszeit Queen Victorias (1837--1901)
-- wurde viel Arsen verwendet, beispielsweise als Bestandteil von Tinte
oder in der Lebensmittelindustrie. Dies war Teil des industriellen
Fortschritts, der unter anderem immer neue Farben für Künstler*innen und
die Druckindustrie hervorbrachte. Aber gleichzeitig führte dieses Arsen
zu Vergiftungen, teilweise über eine lange Zeit. Zahlreiche Arbeiten
haben sich mit diesem Phänomen beschäftigt. (zum Beispiel Hawksley,
Lucinda (2016). \emph{Bitten by witch fever : Wallpaper \& arsenic in
the Victorian home}. London : Thames \& Hudson, 2016, das sich unter
anderem mit dem Problem beschäftigt, welches der Erhalt von Tapeten, die
mit solcher Arsen-durchsetzten Farbe gedruckt wurden, im National
Archive in London heute macht.)

Der Artikel hier geht auf das Problem von Bibliotheken ein, Bücher zu
erhalten, die mit solchen Farben gedruckt wurden. Diese Bücher können
nicht einfach zur Nutzung freigegeben werden. Es werden kurz einige
Bibliotheken angesprochen, die aktuell dieses Problem angehen, zudem
wird das \enquote{Poison Book Project} der University of Delaware
(\url{https://sites.udel.edu/poisonbookproject/arsenic-bookbindings/})
vorgestellt. Ein*e Forscher*in des Projektes wird zitiert und auf eine
betreffende Pressemitteilung verwiesen. (Diese Mitteilung scheint der
\enquote{Auslöser} des Artikels gewesen zu sein.) Letztlich liefert der
Text wieder einmal einen Überblick zu diesem Thema und zeigt, dass sich
Bibliotheken auch mit solchen explizit vergifteten Medien (hier nicht
als Metapher für den Inhalt verstanden) befassen müssen. (ks)

\begin{center}\rule{0.5\linewidth}{0.5pt}\end{center}

Packham, Allfie (2024). \emph{\enquote*{A shell of the place it used to
be}: readers on the importance of libraries - and their fragile future.}
In: The Guardian, 06. September 2024,
\url{https://www.theguardian.com/books/article/2024/sep/06/essential-for-me-readers-on-the-importance-of-libraries}

Public Libraries sind in Grossbritannien seit Jahrzehnten Gegenstand von
Sparrunden auf verschiedenen politischen Ebenen. Gleichzeitig haben sich
Kampagnenformen gegen solche Sparvorhaben etabliert. Dazu zählen
Beiträge in verschiedenen Medien, die sich gegen solche Massnahmen
richten -- beispielsweise in einem Editorial, das kurz nach dem hier
besprochenen Beitrag in der gleichen Tageszeitung erschien
(\url{https://www.theguardian.com/commentisfree/article/2024/sep/08/the-guardian-view-on-public-libraries-these-vital-spaces-provide-much-more-than-books})
-- oder aber Beiträge, in denen Leser*innen berichten, warum sie ihre
jeweilige Bibliothek wichtig finden und wie sie diese nutzen. Der
Artikel hier ist ein Beispiel für diese: Nachdem der Guardian Daten über
Public Libraries auswertete und zeigte, dass im gesamten Königreich in
den letzten Jahren rund 160 von ihnen geschlossen wurden und viele
weitere ihre Angebote einschränken mussten
(\url{https://www.theguardian.com/books/article/2024/sep/03/more-than-180-uk-public-libraries-closed-or-handed-to-volunteers-since-2016}),
berichten in diesem Beitrag nun eine Anzahl von Personen, warum sie
Bibliotheken besuchen.

Das liest sich für Bibliotheken sehr positiv: Sie sind Orte des Lesens;
Orte, in die man sich zurückziehen kann; Orte, die sicher sind. Es ist
motivierend, dass eine breite Öffentlichkeit Bibliotheken als wichtig
erachtet und dies auch laut sagt. Aber gleichzeitig ist auffällig, was
sie nicht erwähnen: Auch in Grossbritannien werden im Öffentlichen
Bibliothekswesen Debatten darum geführt, wie man Public Libraries
modernisieren sollte: Makerspaces, Bildungsangebote, Third Place sind
alles keine Fremdworte, sondern sie werden, wenn möglich, umgesetzt.
Aber wenn Leser*innen -- also die Leute, die Bibliotheken nutzen und
offenbar auch wichtig finden -- daran denken, was sie wichtig finden,
dann scheinen sie vor allem an das zu denken, was Öffentliche
Bibliotheken gerne als \enquote{traditionelle Angebote} beschreiben.
(ks)

\begin{center}\rule{0.5\linewidth}{0.5pt}\end{center}

Walther, Christian (2024). \emph{Fernsehdoku zu DDR und Mauerfall}. In:
taz, 16.10.2024,
\url{https://taz.de/Fernsehdoku-zu-DDR-und-Mauerfall/!6040124/}

Der Autor hat im NDR-Archiv Aufnahmen einer Diskussionsveranstaltung im
Französischen Dom (Berlin), auf der am 09. November 1989 über die
Entwicklung in der DDR gesprochen wurde, genau an dem Tag, an dem die
\enquote{Berliner Mauer fiel} -- also die Grenzübergänge geöffnet wurden
-- aufgespürt. Es diskutierten hier Personen, welche später in der
Wendezeit eine wichtige Rolle spielten, aber ohne wahrzunehmen, was
gerade ausserhalb der Kirche passierte. Es ist fraglos ein Zeitdokument
und der Autor wollte es veröffentlicht wissen. Allerdings war dies nicht
einfach. Er schildert die Wege durch die Archive, die er nehmen musste,
bevor er sein Ziel erreichte. Sicherlich: Das ist in Bibliotheken etwas
anders, aber es ist ein gutes Beispiel dafür, wie Nutzer*innen bestimmte
Regelungen und Entscheidungen von Archiven wahrnehmen -- nämlich nicht
unbedingt positiv, selbst wenn der Autor am Ende sein Ziel erreichte.
(ks)

\hypertarget{weitere-medien}{%
\section{6. Weitere Medien}\label{weitere-medien}}

Bemme, Jens (2024). \emph{Zehn einfache Regeln für grafische
Zusammenfassungen}. In: Blog SLUB Open Science Labs, 28.10.2024,
\url{https://osl.hypotheses.org/13736}

Im Blog des Open Science Labs der SLUB Dresden erschien Ende Oktober
2024 ein Hinweis auf einen Artikel, der zehn kompakte Ratschläge und
zahlreiche Tool-Tipps für die Erstellung grafischer Abstracts gibt.
(Jambor, Helena Klara / Bornhäuser, Martin (2024): \emph{Ten simple
rules for designing graphical abstracts}. In: PLoS Computational Biology
20.2:e1011789. \url{https://doi.org/10.1371/journal.pcbi.1011789})

In den Zeitschriften, mit denen ich üblicherweise zu tun habe, sind
solche Abstracts noch nicht verbreitet. Das einzige Exemplar, welches
ich bislang bewusst wahrgenommen habe, hat sich dafür durch ein
Eichhörnchen qualifiziert. (Urban, Christian et al.~(2024).
\emph{Ancient Mycobacterium leprae genome reveals medieval English red
squirrels as animal leprosy host}. In: Current Biology 34 (2024) 10:
2221 - 2230.e8, \url{https://doi.org/10.1016/j.cub.2024.04.006})

Als spezielle Form der Wissenschaftskommunikation an der Schnittstelle
von wissenschaftlichen Texten und Informationsgrafik finde ich solche
\enquote{bildlichen Zusammenfassungen} aber durchaus interessant.
Vielleicht kann der Artikel Anregungen liefern, mal über einen
grafischen Abstract für die nächste eigene Publikation nachzudenken,
auch wenn er nicht gefordert ist? Durch das Konzentrieren auf die
Kernaussagen des Textes und ihre Verständlichkeit und das
\enquote{Ausprobieren} an interessierten Testleser*innen -- Rule 10:
Before, during, after: Feedback! -- kann ja auch der Text selbst noch
Verbesserungen erfahren. (vv)

%autor

\end{document}