\begin{center}\rule{0.5\linewidth}{0.5pt}\end{center}

\textbf{Michel Roth} (Orcid: 0000-0002-0300-9110), geboren in Altdorf
(Uri), ist Komponist und Professor für Komposition, Musiktheorie und
Artistic Research an der Hochschule für Musik der Musik-Akademie Basel
(FHNW). Er forscht und publiziert über musikalische Anwendungen von
Spieltheorie und Kybernetik (Promotion an der Universität Basel),
Organologie der zeitgenössischen Musik und alpine Klangsoziologie
(Singende Seile'').

\url{https://www.fhnw.ch/de/personen/michel-roth}

\textbf{Felicitas Erb} ist klassische Sängerin und Musikvermittlerin,
spezialisiert auf Lied- und Konzertgesang. Für ihre CD-Einspielungen
erhielt sie zahlreiche Auszeichnungen, darunter eine Nominierung für den
Opus Klassik als Sängerin des Jahres 2023. Als Musikvermittlerin war sie
am Opernhaus Zürich tätig und arbeitet nun an ihrem Forschungsprojekt
Musikvermittlung als kreativ-kritische Praxis'' an der Hochschule für
Musik Basel.

\textbf{Anna Alexay} studiert Schulmusik II C mit Hauptfach
Musikwissenschaft in Basel. Im August 2023 hat sie Michel Roth bei
seinem Installationsprojekt Seilsender am Alpentöne-Festival in Altdorf
assistiert.

\textbf{Oleksandra Katsalap} ist eine ukrainische Musikerin. Sie
studiert Komposition und Klavier an Hochschule für Musik Basel bei
Michel Roth, Caspar Johannes Walter und Tobias Schabenberger. Ihr
Hauptinteresse liegt an die Grenze zwischen Musik und Performance.

\textbf{Oliver Rutz} studierte Komposition und Musiktheorie an der
Hochschule für Musik in Basel. Wenn er nicht in einer Musikbibliothek
sitzt, interessiert er sich für gesellschaftliche Themen im urbanen und
alpinen Raum. Oliver Rutz bildet sich derzeit weiter im journalistischen
und publizistischen Bereich.
