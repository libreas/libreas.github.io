\textbf{Zusammenfassung}: Die von Hans Scharoun entworfene
Staatsbibliothek am Kulturforum ist ein zentraler Ort wissenschaftlicher
wie literarischer Textproduktion in Berlin. Neben der hohen Raumqualität
dieser Ikone moderner Bibliotheksarchitektur befördert auch ihre
charakteristische, von Wim Wenders sogar in seinem Film Der Himmel über
Berlin (1987) ästhetisierte Geräuschkulisse, ganz wesentlich die
kreative Schreibarbeit im Lesesaal. In diesem Beitrag geht es um ein in
Kooperation mit dem Hörverlag speak low sowie der
Medienwissenschaftlerin Hannah Wiemer realisiertes Projekt der
Staatsbibliothek zur akustischen Vermessung ihres Scharoun-Gebäudes am
Kulturforum. Am Vorabend von dessen Generalinstandsetzung soll damit
sowohl dem Inspirationspotential der Sounds of Stabi nachgegangen als
auch und vor allem die Möglichkeit geschaffen werden, sich die
Staatsbibliothek nach Hause zu holen -- gerade während ihrer
baubedingten Schließzeit.

\begin{center}\rule{0.5\linewidth}{0.5pt}\end{center}

\textbf{Summary}: The Staatsbibliothek at the Kulturforum, designed by
Hans Scharoun, is a central space for scientific and literary text
production in Berlin. In addition to the high spatial quality of this
icon of modern library architecture, its characteristic soundscape,
which was even aestheticized by Wim Wenders in his film Wings of Desire
(1987), also significantly promotes creative writing in the reading
room. This article discusses a project realized by the Staatsbibliothek
in collaboration with the publishing house speak low and media scientist
Hannah Wiemer to conduct an acoustic survey of the Scharoun building at
the Kulturforum. On the eve of its upcoming renovation, the aim is both
to explore the inspirational potential of the Sounds of Stabi and, above
all, to create the opportunity to take the Staatsbibliothek home --
especially during its construction-related closure.
