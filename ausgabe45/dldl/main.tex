\documentclass[a4paper,
fontsize=11pt,
%headings=small,
oneside,
numbers=noperiodatend,
parskip=half-,
bibliography=totoc,
final
]{scrartcl}

\usepackage[babel]{csquotes}
\usepackage{synttree}
\usepackage{graphicx}
\setkeys{Gin}{width=.4\textwidth} %default pics size

\graphicspath{{./plots/}}
\usepackage[ngerman]{babel}
\usepackage[T1]{fontenc}
%\usepackage{amsmath}
\usepackage[utf8x]{inputenc}
\usepackage [hyphens]{url}
\usepackage{booktabs} 
\usepackage[left=2.4cm,right=2.4cm,top=2.3cm,bottom=2cm,includeheadfoot]{geometry}
\usepackage[labelformat=empty]{caption} % option 'labelformat=empty]' to surpress adding "Abbildung 1:" or "Figure 1" before each caption / use parameter '\captionsetup{labelformat=empty}' instead to change this for just one caption
\usepackage{eurosym}
\usepackage{multirow}
\usepackage[ngerman]{varioref}
\setcapindent{1em}
\renewcommand{\labelitemi}{--}
\usepackage{paralist}
\usepackage{pdfpages}
\usepackage{lscape}
\usepackage{float}
\usepackage{acronym}
\usepackage{eurosym}
\usepackage{longtable,lscape}
\usepackage{mathpazo}
\usepackage[normalem]{ulem} %emphasize weiterhin kursiv
\usepackage[flushmargin,ragged]{footmisc} % left align footnote
\usepackage{ccicons} 
\setcapindent{0pt} % no indentation in captions
\usepackage{xurl} % Breaks URLs

%%%% fancy LIBREAS URL color 
\usepackage{xcolor}
\definecolor{libreas}{RGB}{112,0,0}

\usepackage{listings}

\urlstyle{same}  % don't use monospace font for urls

\usepackage[fleqn]{amsmath}

%adjust fontsize for part

\usepackage{sectsty}
\partfont{\large}

%Das BibTeX-Zeichen mit \BibTeX setzen:
\def\symbol#1{\char #1\relax}
\def\bsl{{\tt\symbol{'134}}}
\def\BibTeX{{\rm B\kern-.05em{\sc i\kern-.025em b}\kern-.08em
    T\kern-.1667em\lower.7ex\hbox{E}\kern-.125emX}}

\usepackage{fancyhdr}
\fancyhf{}
\pagestyle{fancyplain}
\fancyhead[R]{\thepage}

% make sure bookmarks are created eventough sections are not numbered!
% uncommend if sections are numbered (bookmarks created by default)
\makeatletter
\renewcommand\@seccntformat[1]{}
\makeatother

% typo setup
\clubpenalty = 10000
\widowpenalty = 10000
\displaywidowpenalty = 10000

\usepackage{hyperxmp}
\usepackage[colorlinks, linkcolor=black,citecolor=black, urlcolor=libreas,
breaklinks= true,bookmarks=true,bookmarksopen=true]{hyperref}
\usepackage{breakurl}

%meta

%meta

\fancyhead[L]{Redaktion LIBREAS\\ %author
LIBREAS. Library Ideas, 45 (2024). % journal, issue, volume.
\href{https://doi.org/10.18452/...}{\color{black}https://doi.org/10.18452/...}
{}} % doi 
\fancyhead[R]{\thepage} %page number
\fancyfoot[L] {\ccLogo \ccAttribution\ \href{https://creativecommons.org/licenses/by/4.0/}{\color{black}Creative Commons BY 4.0}}  %licence
\fancyfoot[R] {ISSN: 1860-7950}

\title{\LARGE{Das liest die LIBREAS, Nummer \#14 (Frühling–Sommer 2024)}}% title
\author{Redaktion LIBREAS} % author

\setcounter{page}{1}

\hypersetup{%
      pdftitle={Das liest die LIBREAS, Nummer \#14 (Frühling–Sommer 2024)},
      pdfauthor={Redaktion LIBREAS},
      pdfcopyright={CC BY 4.0 International},
      pdfsubject={LIBREAS. Library Ideas, 45 (2024)},
      pdfkeywords={Literaturübersicht, Bibliothekswissenschaft, Informationswissenschaft, Bibliothekswesen, Rezension, literature overview, library science, information science, library sector, review},
      pdflicenseurl={https://creativecommons.org/licenses/by/4.0/},
      pdfcontacturl={http://libreas.eu},
      baseurl={},
      pdflang={de},
      pdfmetalang={de},
      pdfdoi={10.18452/...},
      pdfurl={https://doi.org/10.18452/...}
     }



\date{}
\begin{document}

\maketitle
\thispagestyle{fancyplain} 

%abstracts

%body
Beiträge von Eva Bunge (eb), Ben Kaden (bk), Karsten Schuldt (ks),
Michaela Voigt (mv), Najko Jahn (nj), Viola Voß (vv)

\hypertarget{zur-kolumne}{%
\section{1. Zur Kolumne}\label{zur-kolumne}}

Ziel dieser Kolumne ist es, eine Übersicht über die in der letzten Zeit
erschienene bibliothekarische, informations- und
bibliothekswissenschaftliche sowie für diesen Bereich interessante
Literatur zu geben. Enthalten sind Beiträge, die der LIBREAS-Redaktion
oder anderen Beitragenden als relevant erschienen.

Themenvielfalt sowie ein Nebeneinander von wissenschaftlichen und
nicht-wissenschaftlichen Ansätzen wird angestrebt und auch in der Form
sollen traditionelle Publikationen ebenso erwähnt werden wie
Blogbeiträge oder Videos beziehungsweise TV-Beiträge.

Gerne gesehen sind Hinweise auf erschienene Literatur oder Beiträge in
anderen Formaten. Diese bitte an die Redaktion richten. (Siehe
\href{http://libreas.eu/about/}{Impressum}, Mailkontakt für diese
Kolumne ist
\href{mailto:zeitschriftenschau@libreas.eu}{\nolinkurl{zeitschriftenschau@libreas.eu}}.)
Die Koordination der Kolumne liegt bei Karsten Schuldt, verantwortlich
für die Inhalte sind die jeweiligen Beitragenden. Die Kolumne
unterstützt den Vereinszweck des LIBREAS-Vereins zur Förderung der
bibliotheks- und informationswissenschaftlichen Kommunikation.

LIBREAS liest gern und viel Open-Access-Veröffentlichungen. Wenn sich
Beiträge dennoch hinter einer Bezahlschranke verbergen, werden diese
durch \enquote{{[}Paywall{]}} gekennzeichnet. Zwar \linebreak macht das Plugin
\href{http://unpaywall.org/}{Unpaywall} das Finden von legalen
Open-Access-Versionen sehr viel einfacher. Als Service an der
Leserschaft verlinken wir OA-Versionen, die wir vorab finden konnten,
jedoch auch direkt. Für alle Beiträge, die dann immer noch nicht frei
zugänglich sind, empfiehlt die Redaktion Werkzeuge wie den
\href{https://openaccessbutton.org/}{Open Access Button} oder
\href{https://core.ac.uk/services/discovery/}{CORE} zu nutzen sowie auf
dem favorisierten Social-Media-Kanal mit
\href{https://mastodon.social/tags/icanhazpdf}{\#icanhazpdf} um Hilfe
bei der legalen Dokumentenbeschaffung zu bitten.

Die bibliographischen Daten der besprochenen Beiträge aller Ausgaben
dieser Kolumne finden sich in der öffentlich zugänglichen Zotero-Gruppe:
\url{https://www.zotero.org/groups/4620604/libreas_dldl/library}.

\hypertarget{artikel-und-zeitschriftenausgaben}{%
\section{2. Artikel und
Zeitschriftenausgaben}\label{artikel-und-zeitschriftenausgaben}}

\hypertarget{vermischte-themen}{%
\subsection{2.1 Vermischte Themen}\label{vermischte-themen}}

Csiszar, Alex (2023). Provincializing Impact: From Imperial Anxiety to
Algorithmic Universalism. In: James Evans und Adrian Johns (Hrsg.).
\emph{Beyond Craft and Code: Human and Algorithmic Cultures, Past and
Present.} Osiris, Vol. 38. University of Chicago Press.
\url{https://doi.org/10.1086/725131}

Der Wissenschaftshistoriker Alex Csiszar (Harvard University) beschreibt
vom International Catalogue of Scientific Literature (ICSL) der Royal
Society of London im späten 19. Jahrhundert ausgehend die historischen
Bedingungen, unter denen die Bibliometrie in der Zeit des Kalten Krieges
eine besondere Rolle im Diskurs über die Entwicklung der Wissenschaft
spielen konnte. Er argumentiert, dass selektive Zitationsindizes wie der
Science Citation Index (SCI) nie ein neutrales Suchinstrument waren.
Vielmehr seien Vorstellungen von Produktivität, Qualität und Wirkung in
die Datenbank eingeflossen. Der Aufsatz zeichnet nach, wie es dazu kam,
dass nordamerikanische, westeuropäische und englischsprachige Quellen in
den Indizes auch heute noch bevorzugt werden. (nj)

\begin{center}\rule{0.5\linewidth}{0.5pt}\end{center}

Modero, Gina C. (2023). \emph{The Special Collections Reading Room: A
Study of Culture and Its Impact on the Researcher Experience}. In: RBM:
A Journal of Rare Books, Manuscripts, and Cultural Heritage 24 (2023) 2,
\url{https://doi.org/10.5860/rbm.24.2.129}

In dieser Studie wendet die Autorin anthropologische Fragestellungen und
Methoden auf Sonderlesesäle in Bibliotheken in den USA an. Sie fragt,
wie diese Lesesäle von einer jeweils eigenen Kultur geprägt sind und wie
sich diese Kultur ausdrückt, beispielsweise im Aufbau der Säle, in der
Infrastruktur und den Hinweisschildern, in offiziellen Regeln und im
Verhalten von Bibliothekar*innen und Leser*innen. Allerdings scheint ihr
das Ergebnis, nämlich dass alle diese Lesesäle eine jeweils eigene
Kultur haben, schon relativ früh festgestanden zu haben. Es wird schnell
zu Beginn des Artikels genannt und dann weiter an Beispielen
exemplifiziert, aber nicht hergeleitet. Grundsätzlich sind im Text die
Darstellung der Theorie und Methode ständig mit den eigentlichen
Ergebnissen vermischt. Es fehlt ein roter Faden. Insoweit ist es vor
allem ein Nachweis, dass man mit anthropologischen Methoden auch
Bibliotheken untersuchen kann. Aber die Argumente, die für das Ergebnis
angeführt werden, überzeugen durch ihre sporadische Darstellung im Text
nicht wirklich. Sie scheinen hier vor allem subjektive Interpretationen
der Autorin zu sein. (ks)

\begin{center}\rule{0.5\linewidth}{0.5pt}\end{center}

Larsen, Håkon (2024). \emph{Managing Norwegian public libraries as civil
public spheres: recent controversies}. In: Journal of Documentation 80
(2024) 1, 116-130, \url{https://doi.org/10.1108/JD-02-2023-0036}
{[}Paywall{]}

Mit der Revision des norwegischen Bibliotheksgesetzes wurde 2014
explizit festgeschrieben, dass Öffentliche Bibliotheken einen Ort für
gesellschaftliche Debatten bieten müssen. Dabei wird sich -- auch in
diesem Text hier -- kontinuierlich auf Jürgen Habermas berufen. Die
Vorstellung allerdings, welche im Gesetz festgeschrieben wurde, scheint
einer relativ naiven Auffassung davon zu folgen, was
\enquote{gesellschaftliche Debatte} heisst und wie diese stattfindet --
nämlich praktisch als einfaches Gespräch. Dabei ist sie
selbstverständlich immer konfliktbehaftet, was man bei Habermas lernen
könnte, weil es ihm in seinen Arbeiten ja darum geht, zu verstehen, wie
sich die Gesellschaft durch Kommunikation entwickelt.

Die Umsetzung dieser Vorschrift obliegt den Bibliotheken selbst. Larsen
versammelt in seinem Text nun Vorfälle der letzten Jahre, in welchen die
Umsetzung zu Auseinandersetzungen führte. Oft sind es explizite
rassistische oder antimuslimische Gruppen, die Bibliotheken für
Veranstaltungen nutzen wollten, was zu der Frage führte, ob Bibliotheken
dies im Sinne einer Debatte zulassen sollen oder nicht -- und wenn
nicht, warum nicht. Es gab aber auch Auseinandersetzungen darum, ob
weiter Harry Potter-bezogene Veranstaltungen angeboten werden sollten,
nachdem ihre Autorin zu einer der prominentesten
\enquote{Anti-Trans-Aktivist*innen} wurde. Der Text beschreibt diese
Auseinandersetzungen und ihre jeweiligen \enquote{Lösungen}
(beispielsweise, dass Veranstaltungen von rassistischen Gruppen
stattfinden konnten, wenn sie \enquote{neutral} moderiert wurden, allen
Personen offen standen und explizit auch anderen Positionen Raum gegeben
wurde). Er beschreibt dies als Lerneffekte, sowohl für Bibliotheken als
auch für die Politik und die Öffentlichkeit. So ist heute etabliert,
dass Veranstaltungen zwar untersagt werden können, wenn sie eine Gefahr
für Personal und Nutzer*innen darstellen, aber nicht aus rein
politischen Gründen. Damit wird auch gezeigt, dass die Idee,
Bibliotheken müssten Ort von Debatten und Öffentlichkeit sein,
notwendigerweise zu Konflikten führen wird.

Erstaunlich an dem Text oder der Situation in Norwegen im Allgemeinen
ist, dass das nicht vorhergesehen wurde. Wie gesagt: Das solche
Öffentlichkeit zu Konflikten führt, ist eine Grunderkenntnis bei
Habermas, auf den sich ständig berufen wird. Es wäre aber auch zu lernen
gewesen, wenn man woher in anderen Ländern, in denen Bibliotheken
ähnlichen Prämissen folgen, geschaut hätte (beispielsweise Kanada). Es
scheint, als hätte das norwegische Bibliothekswesen an dieser Stelle
einerseits zu wenig weit über sich selber hinausgeschaut (eine
Einschätzung, die vielleicht vom Eindruck des Rezensenten bei
persönlichen Kontakten mit norwegischen Kolleg*innen geprägt ist, bei
denen die Kolleg*innen eigentlich immer nur die Situation in den
skandinavischen Ländern vor Augen hatten) und andererseits zu sehr einen
gesellschaftlichen Konsenswillen angenommen, der in der Realität nicht
gegeben ist. Für Bibliotheken im DACH-Raum ist der Text eine
Lernmöglichkeit, da sich ähnliche Fragen auch ohne gesetzliche
Bestimmung im Bibliotheksalltag stellen. (ks)

\begin{center}\rule{0.5\linewidth}{0.5pt}\end{center}

Roy, Mantra ; Chatterjee, Sutapa (2024). \emph{Barriers in LIS
Scholarship in India: Some Observations}. In: International Journal of
Librarianship 8 (2024) 4: 114-127,
\url{https://doi.org/10.23974/ijol.2024.vol8.4.330}

In Indien -- wie auch in einigen anderen Staaten -- ist es notwendig,
dass Bibliothekar*innen, um auf bestimmten Stellen in Wissenschaftlichen
Bibliotheken befördert zu werden, forschen und ihre Forschung
publizieren. Zudem gibt es mehrere Stellen, an denen aktiv
Bibliothekswissenschaft betrieben wird. Die Situation ist also noch
anders als im DACH-Raum, wo bekanntlich eine Trennung von Wissenschaft
und Karrieren in Wissenschaftlichen Bibliotheken existiert. Und dennoch
ist die Sichtbarkeit dieser indischen Forschung in der internationalen
-- was heisst, vor allem der englischsprachigen --
Bibliothekswissenschaft gering.

Die Studie versucht zu klären, warum dies so ist. Dazu wurden Literatur
gesichtet sowie zwei Umfragen und Interviews durchgeführt. Das führt ein
wenig zum Eindruck eines Stückwerks, in dem vieles irgendwie integriert
wurde. Dennoch sind die Ergebnisse interessant. So werden zuerst externe
Faktoren genannt: Das Wissenschaftssystem ist auf das Englische
orientiert, was eine Hürde für viele Kolleg*innen (nicht nur) in Indien
darstellt. Darüber hinaus ist es mehr und mehr auf finanzstarke
Institutionen konzentriert, die zum Beispiel Article Processing Charges
(APC) zahlen können oder institutionelle Repositorien betreiben. Weitere
externe Faktoren finden sich auf der Ebene von Institutionen: In den
meisten Hochschulen in Indien werden die Bibliothekar*innen nicht dabei
unterstützt, zu forschen. Sie erhalten keine oder zu wenig Arbeitszeit
dafür, keine Beratung oder andere Unterstützung. Es gibt aber auch
Faktoren, die direkt bei den Bibliothekar*innen selber angesiedelt sind:
Sie haben wenig Erfahrung darin, zu publizieren. Das müssten sie üben.
Zudem sind sie wenig über die Möglichkeiten, beispielsweise des
Publizierens in Open Access ohne Kosten, informiert. Alles in allem
scheint die Situation also von verschiedenen Hürden geprägt, die auf
unterschiedliche Weise angegangen werden können. (ks)

\begin{center}\rule{0.5\linewidth}{0.5pt}\end{center}

\emph{Case Studies in Library Publishing} 1 (2023) 1,
\url{https://cslp.pubpub.org/issue-1}

Die erste Ausgabe dieser neuen Zeitschrift erschien Ende 2023. Sie
versteht sich als Publikationsort für das \enquote{Library Publishing}.
Damit wird im englischsprachigen Bibliothekswesen die Aufgabe von
Bibliotheken verstanden, Publikationen von Forschenden oder Studierenden
zu unterstützen. Das kann auf die technische Infrastruktur beschränken,
aber auch ausgeweitet werden bis hin zum Betrieb eines eigenen Verlags,
der zumeist als reiner Open Access Verlag konzipiert wird. Dazwischen
finden sich weitere Services, wie Beratung, die Übernahme des
\enquote{Copy Editing} oder das Klären von Lizenzen. Selbstverständlich
gibt es einige dieser Angebote auch in einigen deutschsprachigen
Bibliotheken. Aber in den englischsprachigen scheinen sie sich in den
letzten Jahren mehr zu etablieren und gleichzeitig scheinen die damit
beschäftigten Kolleg*innen auch mehr und mehr zu kooperieren.

Die erste Ausgabe der Zeitschrift stellt nun in sieben Beiträgen (sechs
aus dem englischsprachigen Raum, einer aus der Ukraine) konkrete
Publikationsprojekte vor. Allen Artikeln sind Lessons Learned
vorangestellt. Zudem sind sie fast alle sehr kleinteilig und konkret.
Das wird dann von Interesse sein, wenn eine Bibliothek selber
vergleichbare Services aufbauen will. Interessanter ist aber, dass sich
das \enquote{Feld} des Library Publishings mit dieser Zeitschrift weiter
zu festigen scheint. (ks)

\begin{center}\rule{0.5\linewidth}{0.5pt}\end{center}

Albergaria, Matheus (2024). \emph{The economics of libraries}. In:
Journal of Information Science {[}Online first{]},
\url{https://doi.org/10.1177/01655515241233741} {[}Paywall{]}

Der Artikel fasst in Auswahl zusammen, was in den letzten Jahren im Feld
der \enquote{ökonomischen Forschung} -- ergo der Betriebswirtschaft --
über Bibliotheken geforscht wurde. Laut dem Autor steigt die Zahl der
Studien in diesem Bereich an. Die Studien beschäftigen sich auf der
einen Seite mit Frage der Effekte der gemeinsamen Nutzung eines
geteilten Gutes -- also der Medien in Bibliotheken, die in endlicher
Zahl vorhanden sind, durch Nutzer*innen, die jeweils nach eigenen
Interessen handeln würden -- und auf der anderen Seite mit dem Versuch,
die Nutzung von Bibliotheken mit anderen, ökonomisch relevanten
Kennzahlen -- beispielsweise der Zahl der Patente, die in verschiedenen
Städten angemeldet werden -- zu kombinieren. Der Autor betont auch, dass
es eine Tradition der Kritik solcher Studien und vor allem des
\enquote{Weltbildes} der Ökonomie gibt, schliesst aber, dass es ein
wachsendes und deshalb für die Bibliotheks- und Informationswissenschaft
relevantes Feld darstellt. (ks)

\begin{center}\rule{0.5\linewidth}{0.5pt}\end{center}

Centerwall, Ulrika (2024). \emph{In plain sight: School librarian
practices within infrastructures for learning}. In: Journal of
Librarianship and Information Science 56 (2024) 1: 211-222,
\url{https://doi.org/10.1177/09610006221140881} {[}Paywall{]}

Für die hier vorgestellten Studie wurden zwölf Schulbibliothekar*innen
aus Schweden über ihre Arbeit interviewt. Die jeweiligen Bibliotheken
gelten als \enquote{Best Practice}, dass heisst sie wurden in einem
jährlich von der betreffenden Gewerkschaft durchgeführten Wettbewerb als
solche ausgezeichnet. Die Autor*in betont, dass sie deshalb herausragend
sind. In Schweden gibt es zwar die gesetzliche Bestimmung, dass alle
Schüler*innen Zugang zu einer solchen Bibliothek haben sollen. Dies
würde jedoch nicht überall durchgesetzt. Gut ausgestattet, inklusive
ausgebildeter Bibliothekar*innen, seien nur einige Bibliotheken.

Für die befragten Bibliothekar*innen stellt die Studie nun heraus, dass
die Bibliotheken Teil der schulischen Infrastruktur sein können, es dazu
aber aktiver Arbeit der Bibliothekar*innen selbst bedarf. Sie müssen die
Zusammenarbeit mit Lehrpersonen initiieren und aufrechterhalten und
Lehrpersonen kontinuierlich darüber informieren, was ihre Aufgaben und
Möglichkeiten sind. Sie müssen auch dafür sorgen, dass sie als
professionelles Personal wahrgenommen mit eigener Ausbildung und
Kompetenz werden, nicht als \enquote{Hilfspersonal}. Dabei betont die
Autorin, dass sich Lehrpersonen als Profession in der Schule besser
positionieren können, da sie zum Beispiel auf zahlreiche etablierte
Standards und Richtlinien zurückgreifen können -- implizit deutet sie
damit an, dass solche gemeinsam in der schulbibliothekarischen
Profession erarbeitete Standards hilfreich wären. (ks)

\begin{center}\rule{0.5\linewidth}{0.5pt}\end{center}

Guss, Samantha ; Cunningham, Sojourna ; Stout, Jennifer (2024).
\emph{Not all staying is the same: Unpacking retention and turnover in
academic libraries}. In: In The Library With The Lead Pipe, 10.04.2024,
\url{https://www.inthelibrarywiththeleadpipe.org/2024/not-all-staying/}

Warum bleiben Bibliothekar*innen auf ihren konkreten Arbeitsstellen,
obwohl sie eigentlich wechseln wollen, also in anderen Bibliotheken
arbeiten möchten? Die Frage ist, so die Autor*innen dieser Studie,
relevant, da es in Bibliotheken grundsätzlich ein Problem gäbe, Personal
\enquote{zu halten}, aber gleichzeitig oft übersehen wird, dass nicht
alle Bibliothekar*innen gleich die Stellen wechseln würden, wenn sie
sich nicht (mehr) mit ihrer aktuellen Arbeitsstelle identifizieren.
Stattdessen würden sie oft verbleiben, was schlecht für sie selbst --
ihr Wohlbefinden, teilweise ihre Gesundheit und ihre professionelle
Entwicklung -- und für die jeweilige Bibliothek -- mit Auswirkungen auf
die Arbeitskultur, die Leistungsfähigkeit des Personals und auch die
Weiterentwicklung der Bibliothek -- wäre. Nicht nur könnte sich das
Personal besser entwickeln, wenn es offener mit seinem Missbehagen
umgehen würde. Auch Bibliotheken als Arbeitsplatz würden sich besser
entwickeln können, wenn sie nicht nur danach fragten, warum jemand
\enquote{gegangen} sei, sondern auch das Unbehagen des Personals, dass
sich nicht zu diesem Schritt entschloss, auswerten.

Die Studie geht erst theoretisch und dann mittels zehn Interviews vor.
Es zeigt sich, dass es offenbar eine ganze Reihe von Bibliothekar*innen
gibt, die lange mit ihrem Arbeitsplatz und ihrer Situation unzufrieden
sind, aber teilweise Jahre brauchen, ihre Positionen zu verlassen. Ein
Grund dafür sei, dass es zwar oft besondere Situationen, Ereignisse oder
Umstrukturierungen sind, die sich dann letztlich dazu bringen, doch die
Stelle zu wechseln (im Text werden sie als Trigger bezeichnet), aber das
Unwohlsein sich über einen längeren Zeitraum aufbaut. Oft arrangieren
sich Bibliothekar*innen mit der Situation, bis sie irgendwann
\enquote{nicht mehr können}. Zwei Gründe, warum sie länger bleiben, sind
die Arbeitsmarktsituation -- ergo die Schwierigkeit, eine andere Stellen
zu finden -- und die grundsätzlich Zufriedenheit mit der Arbeit selber,
also zum Beispiel der Umgang mit Nutzer*innen. Während die Ergebnisse
dieser Studie spezifisch US-amerikanisch sind, da sie dort durchgeführt
wurde, lässt sich vermuten, dass es auch im DACH-Raum eine Reihe
Bibliothekar*innen in unbefriedigenden Situationen gibt, die -- wenn man
die Parallele zieht -- vor allem ihre Arbeit machen, aber ansonsten
weder sich noch die Einrichtung weiterentwickeln. (ks)

\begin{center}\rule{0.5\linewidth}{0.5pt}\end{center}

Engström, Lisa ; Eckerdal, Johanna Rivano (2024). \emph{Bringing on the
Social: Infrastructuring Libraries Through Zine-making Workshops}. In:
The Journal of Creative Library Practice,
\url{https://creativelibrarypractice.org/2024/03/14/bringing-on-the-social-infrastructuring-libraries-through-zine-making-workshops/}

Interessant an dieser Studie ist die Methodik. Herausgefunden werden
sollte, wie (schwedische) Bibliothekar*innen eine Öffentliche Bibliothek
wahrnehmen, also konkret, wie sie in der Praxis den Anspruch, dass diese
\enquote{ein Ort} beziehungsweise \enquote{ein Treffpunkt} sein soll,
verstehen. Dazu wurde von den Autor*innen ein \enquote{Fanzine Workshop}
mit acht Bibliothekar*innen veranstaltet. Die Teilnehmenden versammelten
sich in einer konkreten Bibliothek und erhielten dann Aufgaben, die in
Fanzines -- also selbst produzierten, kleinen Heften -- über die
Bibliothek endeten. Es ging darum, durch den Raum zu gehen, ihn
wahrzunehmen, Notizen, Photos oder Skizzen zu machen und diese
Materialien in Zweierteams zu ordnen, auszuwählen sowie schliesslich zu
den genannten Fanzines zusammenzukleben. Die Autor*innen nutzten die
Fanzines anschliessend als Datenmaterial. Sie werteten sie daraufhin
aus, wie die Bibliothekar*innen den Raum und seine Nutzung sahen, was
sie als wichtig und was als weniger wichtig ansehen. Die Ergebnisse
waren wenig überraschend, sondern spiegelten den Diskurs über die
Bibliothek als \enquote{offenen, demokratischen Ort}, der sich auch in
der Literatur zu Öffentlichen Bibliotheken im DACH-Raum findet. (ks)

\begin{center}\rule{0.5\linewidth}{0.5pt}\end{center}

Adolpho, Keahi ; Krueger, Stephen G. (2024). \emph{Decistifying trans
and gender diverse inclusion in library work: A literature review.} In:
In The Library With The Lead Pipe, 24.04.2024,
\url{https://www.inthelibrarywiththeleadpipe.org/2024/decistifying/}

Ausgangspunkt für diese Literaturübersicht ist das Unbehagen der
Autor*innen über den \linebreak \enquote{Stand} der Literatur zu Trans- und
Genderthemen in der (englischsprachigen) bibliothekarischen Literatur.
Es gibt zu diesem Thema zwar eine wachsende Anzahl von Texten, aber
diese -- so die Kritik -- würde praktisch immer wieder nur die
Grundlagen erklären. (Sie nennen dies das \enquote{Trans 101}, also
praktisch den Einführungskurs.) Hingegen fehlten konkrete Arbeiten dazu,
wie Bibliotheken (ganz umgreifend gemeint -- als Sammlungen, als
Kataloge, als Orte, als Arbeitsplatz) zu Orten werden könnten, in denen
Trans- und Genderthemen normaler Alltag und damit \enquote{normalisiert}
sind. Stattdessen würde immer wieder neu erklärt, welche
unterschiedlichen sexuellen Identitäten es gäbe, dass es einen
Unterschied zwischen Identität und sexuellen Präferenzen gäbe oder wie
Sprache und Geschlecht \enquote{funktionieren}. Es scheint den
Autor*innen, als bliebe man bisher bei den ersten Schritten stehen.

Um diese Aussage zu untermauern, führen sie in diesem Artikel eine
weitgehend systematische Literatursichtung durch. Sie versammeln
Literatur der letzten Jahre und besprechen sie einzeln. Auch wenn das
nicht ihr Ziel war, liefern sie damit eine Handreichung für
Bibliothekar*innen, die sich für das Thema interessieren. Ungewollt ist
dies also ein weiteres \enquote{Trans 101}, aber vielleicht eines, dass
ein \enquote{Weitergehen} motiviert, da so sichtbar wird, dass die
Grundlagen eigentlich alle schon mehrfach erklärt wurden. (ks)

\hypertarget{forschungsdatenmanagement-und-forschungsfuxf6rderung}{%
\subsection{2.2 Forschungsdatenmanagement und
Forschungsförderung}\label{forschungsdatenmanagement-und-forschungsfuxf6rderung}}

Schmidt, Birgit ; Chiarelli, Andrea ; Loffreda, Lucia ; Sondervan,
Jeroen (2023). \emph{Emerging Roles and Responsibilities of Libraries in
Support of Reproducible Research}. In: LiberQuarter 33 (2023),
\url{https://doi.org/10.53377/lq.14947}

Der Titel des Textes impliziert eine Studie, aber er ist eigentlich ein
Policy-Dokument, das die Arbeit einer Arbeitsgruppe von
Bibliothekar*innen und anderen Personen aus dem Bereich
Forschungsinfrastruktur in Europa zusammenfasst. Es scheint sich auf
einer breiten Literaturbasis abzustützen, die immer wieder argumentativ
herangezogen und auch im Literaturverzeichnis dargestellt wird. Die
Darstellung selbst ist jedoch erstaunlich wenig mit dieser Literatur
verbunden. Eher werden hier grundsätzliche Ergebnisse der Arbeitsgruppe
dargestellt, der darin aber vertraut werden muss. Zumindest die
dargestellten Ergebnisse -- dass es verschiedene Entwicklungen im
Bereich Forschungsdatenmanagement gibt und dass sich diese in den
verschiedenen europäischen Ländern und Institutionen unterscheiden --
scheinen recht gering für den Aufwand zu sein, eine länderübergreifende
Arbeitsgruppe einzurichten. (Zu vermuten ist, dass die Zusammenarbeit
einen Beitrag zur Vernetzung geleistet hat, die im Text nicht sichtbar
wird.) Der Text wird als Beispiel dieser Art von Literatur zu gelten,
die in den letzten Jahren immer häufiger im Bibliotheksbereich zu
erscheinen scheint -- Dokumente, die eigentlich eine Positionsbestimmung
darstellen (was seine Berechtigung hat), aber im Format einer
wissenschaftlichen Arbeit daherkommen (was sie allerdings nicht sind).
(ks)

\begin{center}\rule{0.5\linewidth}{0.5pt}\end{center}

Hackett, Cody ; Kim, Jeonghyun (2024). \emph{Planning, implementing and
evaluating research data services in academic libraries: a model
approach}. In: Journal of Documentation 80 (2024) 1, 27-38,
\url{https://doi.org/10.1108/JD-01-2023-0007} {[}Paywall{]}

Der Titel des Textes ist irreführend. Es geht nicht um ein weiteres
Modell für Services im Bereich Forschungsunterstützung an
Wissenschaftlichen Bibliotheken, vielmehr ist es eine Übersicht zu schon
vorhandenen Modellen dieser Art. Gezeigt wird, (a) dass es schon eine
Reihe von Modellen gibt, die einzuordnen versuchen, was Bibliotheken in
diesem Bereich anbieten oder anbieten sollten, (b) dass sich diese
Modelle teilweise sehr voneinander unterscheiden und (c) dass sich der
Bereich in Bewegung befindet. (ks)

\hypertarget{wissenschaftskommunikation-und-wissenschaftliche-zeitschriften}{%
\subsection{2.3 Wissenschaftskommunikation und wissenschaftliche
Zeitschriften}\label{wissenschaftskommunikation-und-wissenschaftliche-zeitschriften}}

Koçak, Zafer (2023). \emph{Misleading Metrics: Predatory Trade Expands}.
In: Trakya University Journal of Natural Sciences 24 (2023) 2, 1-3,
\url{https://doi.org/10.23902/trkjnat.1368563}

In diesem kurzen Editorial gibt der Autor einen Überblick über Praktiken
irreführender bibliometrischer Online-Angebote im Kontext von Predatory
Journals. Neben einer inhaltlichen Einführung werden einige der
häufigsten \enquote{misleading metrics} aufgelistet, ihre
Charakteristiken aufgelistet und wichtige Publikationen zum Thema
erwähnt. Der Text eignet sich gut, um einen schnellen Überblick über das
Thema zu erhalten. (eb)

\begin{center}\rule{0.5\linewidth}{0.5pt}\end{center}

Dejan Pajić, Aleksandra Babić, Tanja Jevremov (2023): \emph{Open access
practice in personality research: a bibliometric perspective.}
Primenjena Psihologija, 16(4).
\url{https://doi.org/10.19090/pp.v16i4.2511}

Die Autor*innen legen eine bibliometrisch grundierte fachspezifische
Analyse des Open Access-Verhaltens der Community der
Persönlichkeitsforschung in der Psychologie vor. Sie ist Teil einer
Sonderausgabe zum Thema \enquote{Promoting Open Science Principles}.
Über Scopus werteten die Autor*innen über 57.000 Publikationen aus dem
Feld hinsichtlich des Publikationsstatus aus und konnten eine generelle
Open-Access-Quote von 31\% ermitteln. Während im Gesamtbestand Green-OA
dominiert, lassen sich im Zeitverlauf die größten Zuwächse im Bereich
Gold-OA ermitteln. Der grüne Weg verliert sogar an Relevanz. Allerdings
stellt die Studie fest, dass sich Gold-OA für die Domäne der
Persönlichkeitsforschung hinsichtlich des wissenschaftlichen und
gesellschaftlichen Impacts beziehungsweise Reichweite als gegenüber den
Erwartungen deutlich weniger wirksam als bei Green-OA erweist. Dies ist
insofern problematisch, weil es die Bewegung von Green-OA zu Gold-OA
gibt. Die Autor*innen vermuten, dass die steigende Akzeptanz von Gold-OA
zu einer Abnahme der Motivation für Green-OA, also die
Zweitveröffentlichung über Repositorien führt. (bk)

\begin{center}\rule{0.5\linewidth}{0.5pt}\end{center}

Eve, Martin Paul (2024). \emph{Digital Scholarly Journals Are Poorly
Preserved: A Study of 7 Million Articles}. In: Journal of Librarianship
and Scholarly Communication 12 (2024) 1,
\url{https://doi.org/10.31274/jlsc.16288}

Diese Studie stammt direkt aus der Forschung von crossref, also der
zentralen Registrierungsagentur für DOIs. Insoweit sind die
Fragestellung und auch die Handlungsempfehlungen, die am Ende gegeben
werden, aus den Interessen der Organisation abgeleitet. Dennoch sind die
Ergebnisse der Studie für Bibliotheken relevant. Sie untersucht mithilfe
einer Datenanalyse, den Archivierungsstatus wissenschaftlicher Artikel,
selbstverständlich immer behaftet mit dem Problem jeder Datenanalyse,
dass sie von der Qualität der vorhandenen Daten abhängig ist.

Aber: Es gibt grundsätzlich die Erwartung, dass Wissenschaftsverlage
sich um die Archivierung der von ihnen veröffentlichten Artikel kümmern.
Sie sollen sicherstellen, dass einmal vergebene DOIs auch immer auf eine
Version eines Artikels verweisen können. Dazu wird auch erwartet, dass
er an verschiedenen Stellen beziehungsweise in verschiedenen Systemen
archiviert ist.

Die Analyse zeigt nun -- wie schon der Titel des Artikels andeutet --,
dass dies selten der Fall ist. Einige wenige Verlage erfüllen diese
Erwartung, eine etwas grössere Zahl erfüllt sie einigermassen -- legt
also Kopien nur in einem System ab -- und der grösste Teil erfüllt sich
nicht. Es gibt dabei einige Tendenzen: Grössere Verlage schneiden
grundsätzlich besser ab, aber auch nicht alle. Nur Elsevier erfüllt alle
Ansprüche. Je kleiner die Verlage sind, umso eher ist der
Archivierungsstatus \enquote{ihrer} Artikel prekär. Dies stellt, wie der
Autor richtig bemerkt, ein Problem für die langfristige
Nachvollziehbarkeit von Wissenschaft dar. In seiner Auswertung geht er
hauptsächlich auf mögliche Konsequenzen für crossref ein, vor allem
strengere Durchsetzung von Anforderungen für die Vergabe von DOIs und
bessere Beratung ihrer Mitglieder. Bibliotheken müssen ihre eigenen
Schlüsse ziehen. (ks)

\begin{center}\rule{0.5\linewidth}{0.5pt}\end{center}

Haruto Hiraba, Yoshimasa Takeuchi, Kensuke Nishio, Hiroyasu Koizumi,
Takayuki Yoneyama, Hideo Matsumura: \emph{Current status of dental
journals published by Japanese organization}. In: Japanese Dental
Science Review. Vol. 60, 2024, S. 40-43,
\url{https://doi.org/10.1016/j.jdsr.2023.12.001}

Die Autor*innen analysieren die internationale Sichtbarkeit von
japanischen Fachzeitschriften aus dem Bereich der Zahnmedizin anhand der
Kriterien Journal Impact Factor (JIF), Eigenfactor (EF), Article
Influence Score (AIS) und Open-Access-Anteil. Im Journal Citation
Reports (JCR) sind 18 Publikationen nachgewiesen, 16 weitere Titel
werden nicht im JCR indexiert. Die Zeitschrift mit dem höchsten JIF
(6,6) ist The Japanese Dental Science Review, die damit derzeit auf
Platz fünf unter den zahnmedizinischen Journals weltweit steht. Die
Autor*innen betonen die Relevanz für Forschende in diesen Toptiteln zu
publizieren. Sie unterstreichen zugleich die Bedeutung der Publikation
in Gold-Open-Access-Zeitschriften, die den Standards des DOAJ
entsprechen. Die Studie geht davon aus, dass über Open Access die
Sichtbarkeit der Aufsätze weiter zunimmt. Die Autor*innen sprechen sich
daher für eine stärkere Umstellung der Zeitschriften auf
Gold-Open-Access aus. Sie erwarten dadurch höhere JIF, EF und AIS-Werte
und einen stärkeren Wettbewerb der Titel untereinander. Als die beiden
zentralen Herausforderungen für Open Access sehen sie die Frage der
Finanzierung und die der Qualitätssicherung. (bk)

\begin{center}\rule{0.5\linewidth}{0.5pt}\end{center}

Dueholm Müller, Sune ; Sæbø, Johan Ivar (2024). \emph{The
\enquote*{hijacking} of the Scandinavian Journal of Information Systems:
Implications for the information systems community}. In: Information
Systems Journal 34 (2024) 2: 364-383,
\url{https://doi.org/10.1111/isj.12481}

Eine (relativ) neue Art von Betrug im Bereich wissenschaftliche
Zeitschriften wird in diesem Artikel anhand eines konkreten Beispiels,
dem im Titel genannten \enquote{Scandinavian Information Systems
Journal}, geschildert, nämlich das \enquote{hijacking} von
Zeitschriften. Dies betrifft offenbar eine wachsende Zahl von
Zeitschriften und hat auch schon dazu geführt, dass das Projekt
\enquote{Retraction Watch} eine eigene Liste solcher Zeitschriften führt
\url{https://retractionwatch.com/the-retraction-watch-hijacked-journal-checker/}.

Bei diesem Betrug wird eine Zeitschrift aufgesetzt, die eine regulär
existierende Zeitschrift kopiert, inklusive Titel, Aussehen,
Einreichungssystem, ISSN und anderen Merkmalen. Im beschriebenen Fall
ging dies soweit, dass in der Scopus-Datenbank sogar die URL zur
falschen Zeitschrift als die der richtigen hinterlegt wurde.
Anschliessend werden Artikel, die eingereicht werden, für die
Publikation angenommen, aber direkt eine Article Processing Charge
verlangt. In anderen Fällen wurden offenbar eingereichte Artikel unter
dem Namen anderer Autor*innen veröffentlicht. Der Artikel erläutert
diesen Betrug; beschreibt, wie Redaktion der \enquote{richtigen}
Zeitschrift darauf aufmerksam wurde und wie sie versuchte, den Schaden
zu begrenzen (was schwierig war, weil offenbar die grossen Anbieter von
Datendiensten kaum auf Meldungen dieser Art reagieren). Zudem wird
beschreiben, welche Auswirkungen der Betrug zum Beispiel auf Autor*innen
hat, die hoch bewertete Publikationen benötigen, um Karrieren starten
oder fortsetzen zu können, aber auch für die Reputation von
Zeitschriften. Die Autor*innen setzen sie auch an, darüber nachzudenken,
wie auf diesen Betrug reagiert werden kann, kommen dabei allerdings
nicht sehr weit. (ks)

\hypertarget{open-access}{%
\subsection{2.4 Open Access}\label{open-access}}

Chan, Jennifer ; Zhangm Erica ; Vrmeij, Hermine ; Riemer, John (2024).
\emph{Metadata Librarians for Open Access: A Path Towards Sustainable
Discovery and Impact for Open Access Resources.} In: International
Journal of Librarianship 8 (2024) 4: 30-41,
\url{https://doi.org/10.23974/ijol.2024.vol8.4.351}

Am Ende ist dieser Artikel ein Praxisbericht darüber, wie eine,
allerdings sehr grosse Universitätsbibliothek (University of California
Libraries) eine Position für eine*n Metadata Librarian im Open Access
Team eingerichtet hat. Diese Person ist für die Pflege der Metadaten
beziehungsweise Katalogisate für Open Access Publikationen im Katalog
der Bibliothek zuständig. Diese Schwerpunktsetzung hätte deren
Sichtbarkeit erhöht.

Interessant ist, dass die Autor*innen dies als ein Problem beschreiben:
Bibliotheken hätten in den letzten Jahren immer mehr Zeit und Ressourcen
im Bereich Open Access investiert, aber gerade nicht darin, deren
Metadaten zu pflegen. Dies führte dazu, dass andere Publikationen,
insbesondere solche, die physisch erscheinen, sichtbarer wären. Dabei
lägen die Kompetenzen von Bibliotheken gerade im Bereich Metadaten. Der
Artikel plädiert für die Aufnahme dieser Aufgabe in das Portfolio von
Open Access Offices. Ob es stimmt, dass dies eine neue Aufgabe wäre, wie
sie behaupten, untersuchen die Autor*innen nicht. Eventuell stimmt dies
nur für die USA oder die betreffende Bibliothek. (ks)

\hypertarget{monographien-und-buchkapitel}{%
\section{3. Monographien und
Buchkapitel}\label{monographien-und-buchkapitel}}

\hypertarget{vermischte-themen-1}{%
\subsection{3.1 Vermischte Themen}\label{vermischte-themen-1}}

Bonn, Maria ; Bolick, Josh ; Cross, Will (eds.) (2023). \emph{Scholarly
Communication Librarianship and Open Knowledge}. Chicago: Association of
College and Research Libraries, 2023. Open-Access-eBook- erreichbar über
\url{https://alastore.ala.org/content/scholarly-communication-librarianship-and-open-knowledge}
{[}OA-Version: \url{https://bit.ly/SCLAOK}{]} (Eine DOI würde dem Werk
gut tun.)

Ein neues Buzz-Word und neues Futter aus amerikanischer Produktion für
die Berufsbilddebatte im wissenschaftlichen Bibliothekswesen, also auch
für ``Fachreferent:innen und umzu", denkt man. Und rollt dann bei
einigen der (sehr kurzen) Beiträge angesichts der recht "heroischen"
Darstellungen ein wenig mit den Augen.

Die angesprochenen Themenbereiche und vor allem die zu jedem Beitrag
aufgeführten "Discussion questions" können aber für das Nachdenken über
die Lage an deutschen Bibliotheken durchaus "Knabberzeug" liefern. (vv)

\begin{center}\rule{0.5\linewidth}{0.5pt}\end{center}

Lo, Patrick ; Baker, David (2024). \emph{The Marketing of Academic,
National and Public Libraries Worldwide: Marketing, Branding, Community
Engagement}. Cambridge, Kidlington: Chandos Publishing, 2024,
\url{https://doi.org/10.1016/C2022-0-01950-4} {[}Paywall{]}

Von einem Buch mit dem Titel "The marketing of academic, national and
public libraries worldwide: marketing, branding, community engagement",
das 753 Seiten umfasst, könnte man eine Fülle von Fallbeispielen und
Good Practices aus dem Marketing für wissenschaftliche Bibliotheken
erwarten. In diesem Fall aber bekommt man eine Sammlung von Interviews
mit Kolleg:innen, die mehr oder weniger für das Marketing oder ähnliche
Aktivitäten in ihren Bibliotheken zuständig sind.

Die Antworten sind streckenweise durchaus interessant zu lesen. Aber
aufgrund einiger der (meines Erachtens viel zu groß formatierten)
Fragen, schwanken sie oft zwischen Marketing "für Bibliotheken" und "für
die jeweiligen Bibliothekar:innen" (bis hin zu "What would you like to
be remembered for when you retire?").

Die Auswahl der befragten Kolleg:innen hat einen USA-Schwerpunkt, aber
es gibt auch Beiträge aus anderen Ländern der Welt, darunter
Deutschland. Für die Reihenfolge im Band habe ich kein Muster erkennen
können. In einigen Fällen ist die Auswahl der Interviewten oder
zumindest die Aufnahme der Interviews in die Sammlung fragwürdig (zum
Beispiel "as I am not the marketing expert, I can\textquotesingle t
really answer the question", S. 243).

Hat man sich nach ein paar Seiten also von der Idee verabschiedet,
Inspirationen für Marketingmaßnahmen im eigenen Haus finden zu können,
kann man viel darüber erfahren, auf welchen Wegen die vorgestellten
Kolleg:innen in ihre Bibliothek und auf ihre Position geraten sind und
wie viele Bezeichnungen es für Aufgabenbereiche im weiten Umfeld von
Marketing und Community Engagement gibt. Man lernt viele Bibliotheken in
Kurzportraits kennen -- und kann nach der Lektüre dann vielleicht auch
darüber nachdenken, für was man denn selbst nach dem Abschied in den
Ruhestand (oder in Richtung einer anderen Aufgabe oder Bibliothek) in
Erinnerung bleiben möchte. (vv)

\begin{center}\rule{0.5\linewidth}{0.5pt}\end{center}

Jaminson, Andrea (2024). \emph{Decentering Whiteness in Libraries: A
Framework for Inclusive Collection Management Practices.} (Beta Phi Mu
Scholars Series) Lanham, Boulder, New York, London: Rowman \&
Littlefield, 2024 {[}gedruckt und als E-Book, Paywall{]}

Das kurze Buch bietet für US-amerikanische Bibliotheken eine Anleitung,
Medienbestände gezielt diverser zu gestalten. Das \enquote{decentering
whiteness} im Titel heisst hier vor allem, dass die Erfahrungen,
Perspektiven und Interessen anderer Personengruppen als der weissen
Majorität in den USA ebenso im Bestand vertreten sein sollten -- und
zwar als Normalfall, nicht zum Beispiel als gesonderte Abteilung. In den
letzten Jahren wurden eine ganze Anzahl solcher Arbeiten sowohl als
Monographien als auch in anderen Medienformaten veröffentlicht. Man kann
also von einer eindeutigen Bewegung hin zu diesem Thema und damit wohl
auch dem Ziel, in Bibliotheken solche diversen Bestände zu etablieren,
sprechen. Gleichzeitig -- ansonsten würden nicht immer neue Arbeiten
dazu publiziert werden -- ist dieses Ziel offensichtlich bislang nicht
erreicht worden.

Was dieses Buch auszeichnet, ist, dass es einerseits sehr kurz und
andererseits sehr praxisorientiert ist. Die Autorin argumentiert sowohl
mit ihrer eigenen Geschichte als afro-amerikanische Schüler*in, die ihre
eigene Erfahrung nicht in der Schulbibliothek ihrer sonst
afro-amerikanisch geprägten Schule wiederfand, als auch mit
grundsätzlichen Überlegungen dazu, dass eine diverse Gesellschaft auch
diverse Medienbestände benötigt, für die notwendige Arbeit im
Bestandsmanagement. Sie geht dann kurz weitere grundsätzliche Themen
durch, zum Beispiel die Geschichte der Positionen der ALA zum Thema.
Anschliessend steigt sie aber direkt in das Bestandsmanagement ein und
zeigt, wie das benannte Ziel über Bestandsmanagementplänen, deren
Umsetzung und Evaluation in der Praxis angegangen werden kann. Wert legt
sie dabei auf die von ihr selbst entwickelte \enquote{Measure of
Diversity}, einer Formel, mit der Bibliotheken den Stand der Diversität
ihrer Bestände \enquote{messen} können. All das ist für die direkte
Praxis in Bibliotheken geschrieben, inklusive Hinweisen auf
weiterführende Materialien und Beispiele.

Was für Leser*innen aus dem DACH-Raum mit diesem Werk offensichtlich
wird, ist, wie anders und strukturierter in den USA Bestandsmanagement
betrieben wird, insbesondere wie genau dieses in den
Bestandsmanagementplänen vorstrukturiert und durch diese gesteuert wird.
Gleichzeitig -- so zumindest der Eindruck -- scheinen Bibliotheken aber
auch viel \enquote{näher} am Inhalt der Medien zu sein, als dies im
DACH-Raum üblich ist, wo eine wachsende Anzahl von Medien durch Standing
Order oder vergleichbare Formen des \enquote{Massenkaufs} in die
Bibliotheken gelangen. Die Autorin verweist immer wieder darauf, dass
klar sein müsse, was in den Medien steht, dargestellt wird und so
weiter, und zum Beispiel auch auf Sammlungen von Rezensionen von
\enquote{diversen} Medientiteln, die nur Sinn ergeben, wenn Bibliotheken
tatsächlich einzelne Medien für ihren Bestand wählen und sie inhaltlich
bewerten. Das ist eine andere Kultur des Bestandsmanagement, was auch
heisst, dass Bibliotheken im DACH-Raum die Werkzeuge, Strukturen,
Formeln und so weiter nicht einfach aus den USA übernehmen können, wenn
sie selber das Ziel haben, ihre Bestände diverser zu machen. (ks)

\begin{center}\rule{0.5\linewidth}{0.5pt}\end{center}

Medaille, Ann (2024). \emph{The Librarian's Guide to Learning Theory:
Practical Applications in Library Settings}. Chicago: ALA editions, 2024
{[}gedruckt{]}

Diese kurze Publikation ist als Praxisbuch für Bibliothekar*innen
gedacht, die direkt -- also zum Beispiel in Kursen oder im Klassenraum
-- oder indirekt mit Bildung zu tun haben. Einleitend betont die
Autorin, dass Lerntheorien beschreiben, wie Lernen funktioniert und dass
sie deshalb grundlegend dafür sein sollten, wie Bildungsaktivitäten von
Lehrenden aufgebaut, geplant und durchgeführt werden. Sie bezieht dies
explizit auf Bibliothekar*innen. Dem ist grundsätzlich zuzustimmen, da
die explizite Beachtung theoretischer Vorannahmen tatsächlich zu einer
besseren Lehre führt -- weshalb diese Theorien einen hohen Stellenwert
in der Ausbildung von Lehrpersonen haben --, diese Theorien für
Bibliothekar*innen aber kaum aufbereitet sind.

Ein Problem, das sich aber auch in der Ausbildung von Lehrpersonen
stellt, ist, dass es eine ganze Reihe von Lerntheorien gibt, die immer
nur zum Teil empirisch bestätigt werden können. Insoweit gibt es
eigentlich keine \enquote{Meistertheorie}, sondern eine Vielzahl von
Ansätzen. Die Autorin betont am Anfang, dass sie verschiedene Theorien
vorstellen wird, aber am Ende ist es nur eine -- wenn auch die aktuell
die pädagogische Diskussion prägende -- Theorietradition, nämlich der
Konstruktivismus, den sie in verschiedenen Aspekten durchgeht. Anstatt
vieler Theorien sind die Kapitel eher Unterpunkten gewidmet,
beispielsweise der \enquote{self-regulation} von Lernenden oder der
Bedeutung von \enquote{dialogue} im Bildungsprozess. Dabei ist jedes
Kapitel gleich aufgebaut: Nach einer inhaltlichen Einführung, die
teilweise auch auf die historische Entwicklung des Themas eingeht, folgt
ein Abschnitt über die Bedeutung des Themas für Bibliotheken, dann einer
direkt für Lehrveranstaltungen von Bibliothekar*innen, um mit
Reflexionsfragen (sowie Fussnoten und Literatur) abzuschliessen. Das
Ganze ist also sehr praxisorientiert und sinnvoll für
Bibliothekar*innen, die sich damit beschäftigen wollen, wie sie ihre
Bildungsaktivitäten planen und durchführen wollen -- aber nur für eine
Lerntheorie und nicht, wie man durch den Titel vermuten könnte, für eine
Anzahl von ihnen. (ks)

\hypertarget{bibliotheks--und-buchgeschichte}{%
\subsection{3.2 Bibliotheks- und
Buchgeschichte}\label{bibliotheks--und-buchgeschichte}}

Widdersheim, Michael M. (2023). \emph{Circulation of Power: The
Development of Public Library Infrastructure in Greater Pittsburgh,
1924--2016}. (Current Topics in Library and Information Practice) Berlin
; Boston: Walter de Gruyter, 2023.
\url{https://doi.org/10.1515/9783111013404}. {[}gedruckt und als E-Book,
Paywall{]}

Auf der einen Seite ist dem Autor dieses Buches vieles zugute zu halten.
Er unternimmt den Versuch, eine Theorie der Bibliotheksentwicklung zu
erarbeiten und das auf der Basis eines konkreten, historischen
Beispiels. (Genauer: Er hat diese Theorie schon in einigen Artikeln
entwickelt, in diesem Buch exemplifiziert er sie noch einmal.) Mit der
Theorie will er beschreiben, warum sich Bibliothekssysteme entwickeln.
Gleichzeitig stellt er klar, dass er diese Theorie zur Diskussion
stellt. Er hätte sie an einem Beispiel entwickelt, sie müsse an weiteren
Beispielen geprüft werden. Und zuletzt stellt er die Geschichte des
Beispiels selbst, nämlich das heutige System Öffentlicher Bibliotheken
im Raum Pittsburgh und die die Bibliotheken unterstützende
Infrastruktur, in detaillierter Weise dar.

Allerdings: Die Theorie, die er entwickelt und explizit in einem eigenen
Kapitel darstellt, ist wenig überzeugend. Er postuliert, dass die
Entwicklung vor allem als eine Art Spiel von Kommunikation zwischen
verschiedenen Akteuren beschrieben werden könne. Die Akteure -- Vereine,
staatliche Akteure und Strukturen, Bibliotheken, Einzelpersonen,
gesellschaftliche Initiativen und andere -- würden wechseln. Aber was
sich wiederholte, wäre eine Thematisierung von Problemen durch einige
Akteure, auf die dann von anderen ab einer bestimmten Dringlichkeit
reagiert würde -- zustimmend, ablehnend oder konstruktiv. In einigen
Fällen würden die aufgeworfenen Probleme gelöst, beispielsweise durch
neue Infrastrukturen, in anderen würden sie ungelöst bleiben. Dies würde
sich zyklisch wiederholen. Der Autor weigert sich, andere mögliche
Gründe für diese Veränderungen zu prüfen. Solche -- zum Beispiel die
Bevölkerungsentwicklung, gesellschaftlicher und technologischer Wandel,
Veränderungen der Medienpraxis -- versteht er immer nur als Auslöser von
Kommunikation, nicht als Grund für Veränderungen in Bibliotheken selbst.
Zudem scheint seine Darstellung der entworfenen Theorie unnötig komplex
und gleichzeitig erstaunt, dass sie offenbar ohne Rückgriff auf andere
Theorien erstellt wurde, die sich mit Diskursen und Veränderungen
befassen (es sei nur an die Systemtheorie (Luhman) oder die Theorie des
kommunikativen Handelns (Habermas) erinnert).

Seine Darstellung der konkreten Geschichte der Bibliothekssysteme im
Grossraum Pittsburgh, die er im Buch ebenfalls liefert, fokussiert dann
auf kommunikative Akte. Im Ganzen ist das interessant, aber teilweise
ist es eine Darstellung davon, welcher Verein oder welche
Bürgermeisterin wann eine Brief geschrieben oder eine Rede gehalten hat.
An solchen Stellen ist den Darstellungen schwer zu folgen, weil sich die
Situationen über die Jahrzehnte sehr ähneln. (ks)

\begin{center}\rule{0.5\linewidth}{0.5pt}\end{center}

Senarclens, Vanessa de (Hrsg.) (2024). \emph{Bücher und ihre Wege:
Bibliomigration zwischen Deutschland und Polen seit 1939}. (Fokus: Neue
Studien zur Geschichte Polens und Osteuropas, 12). Paderborn: Brill
Schöningh, 2024. \url{https://doi.org/10.30965/9783657791750}.
{[}gedruckt und als E-Book, Paywall{]}

Auch wenn er in diesem Buch nicht benutzt wird, wäre wohl der Begriff
"Entanglement" von Bibliotheken und Geschichte geeignet, um den Inhalt
dieses Werkes zu beschreiben. Ausgehend von einer Konferenz versammelt
der Band Beiträge zu der Frage, wie Bibliotheken und Bestände im Laufe
der polnischen und deutschen Geschichte \enquote{wanderten}, aber auch
-- sehr oft und insbesondere während des Nationalsozialismus -- zerstört
wurden. Sichtbar wird dabei auch, dass es unterschiedliche Geschichten
gibt. Während in Polen die Erfahrung einer breit angelegten,
systematischen Zerstörung polnischer Bibliotheken -- und, wie in den
Beiträgen auch sichtbar wird, explizit jüdischer -- während des
Nationalsozialismus vorherrscht, wird auf deutscher Seite bislang vor
allem auf \enquote{Lücken} in Bibliotheksbeständen in Deutschland selbst
verwiesen, die durch den Verlauf des Krieges entstanden. Damit sind
unter anderem Bestände gemeint, die von deutschen Bibliotheken während
des Zweiten Weltkrieges ausgelagert wurden, sich nach 1945 auf
polnischem Boden befanden und dann zur Basis heutiger Bibliotheken
wurden. Aus polnischer Sicht galten sie als zumindest teilweiser Ersatz
für die vernichteten Bestände.

Die Beiträge liefern weder eine vollständige Geschichte noch geben sie
eine klare Aussage dazu, wie zum Beispiel mit diesen \enquote{alten
Beständen} heute umgegangen werden soll. In ihrer Grundhaltung plädieren
sie für eine Wahrnehmung gerade der polnischen Erfahrung durch die
deutsche Seite und heute für eine Zusammenarbeit der Bibliotheken beider
Länder. In vielen Beiträgen geht es darum, wie Bibliotheken zerstört
wurden oder welche Wege Bestände nahmen. Teilweise wird dies als
grossangelegte Geschichte erzählt, teilweise anhand von einzelnen
Büchern. (ks)

\begin{center}\rule{0.5\linewidth}{0.5pt}\end{center}

zur Lage, Julian (2022). \emph{Geschichtsschreibung aus der Bibliothek:
Sesshafte Gelehrte und globale Wissenszirkulation (ca. 1750--1815).}
(Wolfenbütteler Forschungen, 169). Wolfenbüttel: Herzog August
Bibliothek Wolfenbüttel, 2022 {[}gedruckt{]}

Die im Titel dieses Buches erwähnte \enquote{Bibliothek} wird erst sehr
spät zum Thema dieser kulturwissenschaftlichen Dissertation. Das
eigentliche Thema steht im Untertitel. Es geht um \enquote{sesshafte
Gelehrte} -- also solche, die vor allem von einem Ort aus arbeiteten und
keine grossen Reiseerfahrungen sammelten --, die Überblickswerke zu
globalen Themen vorlegten. Bei den vier hier tiefergehend besprochenen
Forschenden ging es immer darum, die Geschichte anderer Kontinente oder
gleich der ganzen Welt zu schreiben. Dabei stellte sich die Frage,
welche Quellen sie nutzen konnten, vor allem -- das eigentliche Thema --
ob sie Reisebeschreibungen nutzten und wie sie diese bewerteten. Ihre
Thesen über die Entwicklung der Welt mussten sie auf Quellen stützen,
die sie nicht direkt überprüfen konnten. Und gleichzeitig mussten sie
begründen, warum ihre Arbeitsweise \enquote{vom Schreibtisch aus} dafür
passend war.

Im Buch ist zu lernen, dass die Bewertung von Reiseberichten und der
Kritik dieser Berichte eine eigene Geschichte hat. Gleichzeitig erfährt
man, wie sich mit steigendem Anspruch an die Quellenkritik auch die
Verweisapparate -- Fuss- und Endnoten, Bibliographien, Anhänge --
entwickelten. Zudem gab es eine Geschichte der Begründung, warum eine
\enquote{sesshafte Forschung}, die vor allem Daten zusammentrug,
bewertete und daraus Thesen über die Entwicklung von Gesellschaften,
Menschen und der gesamten Welt ableitete, Vorteile gegenüber der
\enquote{eigenen Anschauung vor Ort} hätte. Im Buch wird die Arbeit vier
Forschender besprochen: Cornelius de Pauw, William Robertson, Johann
Gottfried Herder und Julius August Remer. Nur bei Remer geht der Autor
auf die konkrete Arbeitsweise des Forschers, inklusive dessen Nutzung
von Bibliotheken ein -- hier verstanden als Buchsammlungen, sowohl des
Forschenden selber, die von anderen Forschenden, Adligen und
Buchhändlern als auch von Universitätsbibliotheken selber. In diesem
Abschnitt finden sich auch konkrete Daten zur Nutzung von Büchern und
dem thematischen Aufbau von Bibliotheken. Die rund 250 Seiten der
Arbeit, die diesem Abschnitt vorangehen, lesen sich ein wenig wie ein
Prélude zu dieser intensiven Untersuchung.

Anzumerken ist zudem, dass der Autor im Einleitungsteil als theoretische
Basis Bruno Latours Analyse der Wissensproduktion aus \enquote{immutable
mobiles} -- Objekte, die bei Forschung im Feld erstellt werden, um
transportiert und dann in Laboren ausgewertet zu werden -- einführt. Das
scheint für Reiseberichte und deren Verwendung eigentlich ein sinnvolles
Modell zu sein. Jedoch kommt der Autor im Laufe der Untersuchung
erstaunlicherweise nicht mehr auf diese theoretische Basis zurück. (ks)

\begin{center}\rule{0.5\linewidth}{0.5pt}\end{center}

Smirnova, Victoria (2023). \emph{Medieval} Exempla \emph{in Transition:
Caesarius of Heisterbachs} Dialogus miraculourm \emph{and Its Readers.}
(Cistercian Studies Series ; 296) Collegeville, Minnesota: Liturgical
Press, 2023 {[}gedruckt{]}

Die Studie von Victoria Smirnova liest sich wie die Fingerübung einer
Mediävistin, die anhand eines spezifischen Buches und Autors zeigt, was
diese Wissenschaft an Wissen zu produzieren in der Lage ist. Fingerübung
deshalb, weil sie gar nicht einmal das gesamte Spektrum der
Forschungsmöglichkeiten nutzt -- beispielsweise werden Manuskripte nicht
materiell untersucht --, aber trotzdem eine erstaunliche, konsistente
Geschichte aufzeigen kann.

Bei dem konkreten Buch geht es um eine Sammlung von \enquote{Exempla},
Geschichten über moralisch richtiges Verhalten, über das Wirken Gottes
und des Teufels, die alle mit einer pädagogischen Absicht gesammelt
wurden. Sie sollten den Gläubigen, allen voran Mönchen und Nonnen des
Zististensier-Orden, ermöglichen, den moralisch richtigen Weg zu finden.
Genutzt wurden sie aber auch, wie Smirnova zeigt, als Hilfe für
Predigten. Die Sammlung, erstellt im frühen 13. Jahrhundert, fand erst
im Zististensier-Orden Verbreitung, dann auch bei einigen anderen Orden.
In der frühen Neuzeit wurde sie mehrfach gedruckt und machte damit den
Medienwandel des 16. Jahrhunderts mit. Gleichzeitig wurde sie zum Objekt
protestantischer Kritik, die sie als Beispiel für Aberglauben anführte,
und anschliessend auch der katholischen Seite, welche sie als vielleicht
naive, aber ehrliche Suche nach Gott interpretierte. Im Zeitalter der
Romantik wurde die Sammlung als vorgeblicher Ausdruck deutschen
mittelalterlichen Denkens \enquote{entdeckt} -- unter anderem von
Hermann Hesse. In den jüngerer Zeit wurde der Autor der Sammlung,
Caesarius von Heisterbach, zum Protagonisten verschiedener populärer
Romane, vor allem als Symbol für \enquote{mittelalterliche Weisheit} und
Mystik (bis hin zum Experten für Zeitreisen). Insoweit eignet sich das
Buch als Beispiel für eine Untersuchung.

Die Autorin geht nun das Buch -- beziehungsweise, vor dem Druck, die
\enquote{Textzeugen} des Buches --, die heute noch zu findenden
Annotationen und nachweisbaren Nutzungsweisen durch. Dies wird jeweils
anhand der verschiedenen in Bibliotheken vorhandenen Exemplare
durchgeführt sowie gerade für die Zeit des Mittelalters und der frühen
Neuzeit mit den jeweiligen Entwicklungen der verschiedenen
Ordensgemeinschaften kontextualisiert. Dabei beweist die Autorin, dass
sie den mediävistischen \enquote{Handwerkskoffer} gut beherrscht: Die
Geschichte mehrerer Jahrhunderte wird zusammengebracht, die Manuskripte
und ihre Provenienz werden untersucht, zwischen verschiedenen Sprachen
wird scheinbar aus dem Stegreif übersetzt (besonders beeindruckend bei
Texten, die ein spätmittelalterliches Latein und Deutsch miteinander
verbinden). Das alles macht, wie gesagt, den Eindruck eines
Fingerspiels, einer Vorstufe einer grossen, noch tiefergehenden Studie.
Was das Buch allerdings auch zeigt, ist, dass dieser Handwerkskoffer
immer weniger brauchbar wird, je weiter sich die Autorin vom Mittelalter
selber entfernt. Die Kontextualisierung der Frühdrucke ist noch
überzeugend, die Kontextualisierung in der Romantik dagegen scheint
schon nur noch in sehr groben Strichen vorgenommen worden zu sein. Die
Darstellung zeitgenössischer Romane des 20. und 21. Jahrhunderts -- also
in einer gänzlichen anderen Medienwelt zum Mittelalter -- beschränkt
sich auf eine Nacherzählung.

Gleichzeitig, dass sollte nicht überraschen, ist die Studie auch ein
Hinweis darauf, was die nachhaltige Arbeit von Bibliotheken zum Erhalt
von Manuskripten, Frühdrucken und älterer Literatur zu ermöglichen in
der Lage ist. Alle diese Dokumente fand die Autorin in Bibliotheken vor.
Gleichzeitig ist sie heute selber Wissenschaftliche Mitarbeiterin in
einer der grössten Bibliotheken mit einer solchen Sammlung, nämlich der
Bayerischen Staatsbibliothek in München. (ks)

\begin{center}\rule{0.5\linewidth}{0.5pt}\end{center}

Purdy, Jessica G. (2024). \emph{Reading Between the Lines: Parish
Libraries and their Readers in Early Modern England, 1558--1709.}
(Library of the Written Word, 120 ; The Handpress World, 98). Leiden,
Boston: Brill, 2024. \url{https://doi.org/10.1163/9789004363717}.
{[}gedruckt und als E-Book, Paywall{]}

Dieses Buch ist praktisch ein Reader zum im Titel genannten Thema
\enquote{Parish Libraries} in England in der Neuzeit, also praktisch der
Zeit der Reformation und Gegen-Reformation. Mit den Parish Libraries
sind Buchsammlungen gemeint, die in Kirchgemeinden in dieser Zeit
angelegt und für die Benutzung der Gemeindemitglieder freigegeben
wurden. Die aufeinander folgenden englischen König*innen legten Wert
darauf, ihre jeweilige religiöse Richtung in der Bevölkerung auch
mittels Büchern zu vermitteln. Sowohl die protestantischen als auch die
katholischen Regierungen machten den Kirchgemeinden Vorschriften dazu,
welche Bücher sie anzuschaffen hätten. Dies beförderte den Aufbau
solcher Bibliotheken.

Auch wenn die Autorin das Buch als \enquote{Studie} beschreibt, liest es
sich eher so, als hätte sie in den ersten zwei Teilen, in denen es erst
darum geht, wann und von wem die Bibliotheken gegründet wurden und dann,
wie genau sie ausgestattet oder wo sie in den Kirchen untergebracht
waren, die vorhandene Literatur zusammengefasst und teilweise neu
bewertet. Im dritten Teil dann widmet sie sich vier dieser Bibliotheken,
deren Bestände heute noch zu grossen Teilen erhalten sind, und wertet
sie bis ins kleinste Detail inhaltlich aus. Dabei kann sie zum Beispiel
zeigen, dass diese Bestände religiös recht offen waren, zumindest im
Rahmen der damaligen theologischen Auseinandersetzungen und dass sie
intensiv genutzt wurden, was heute noch an Lesespuren in den vorhandenen
Büchern nachvollzogen werden kann.

In den letzten Jahrzehnten hat es eine Anzahl von Teilstudien zu diesen
Bibliotheken gegeben, dieses Buch scheint einen gewissen Abschluss
darzustellen. Allerdings gelingt dies nicht immer überzeugend. An vielen
Stellen ist nicht klar, woher die Autorin ihr Wissen bezieht. Es scheint
ein wenig so, als würde sie bestimmte Aussagen für so bekannt halten,
dass sie nicht mehr nachgewiesen werden müssen. Irritierend ist das vor
allem im ersten Teil, wenn sie Schlüsse aus den Gründungsdaten der
Bibliotheken zieht -- beispielsweise dass sie erst in den Städten,
danach auf dem Land erfolgten, dann aber das ganze Land erfassten --,
die sie als letztgültigen Stand der Forschung darstellt. Dabei schreibt
sie selber, dass es fast nie genau bestimmt werden kann, durch wen und
wann eine dieser Bibliotheken tatsächlich gegründet wurde. Vieles dieser
Geschichte liegt im Dunkeln. Insoweit werden auch in Zukunft noch mehr
dieser Bibliotheken \enquote{auftauchen}. Die Autorin präsentiert ihre
Daten aber so, als wären sie abgeschlossen und weist sie auch nicht
einzeln nach. (ks)

\begin{center}\rule{0.5\linewidth}{0.5pt}\end{center}

Strickland, Forrest C. (2023). \emph{The Devotion of Collecting: Dutch
Ministers and the Culture of Print in the Seventeenth Century}. (Library
of the Written Word, 110 ; Handpress World, 89) Leiden, Boston: Brill,
2023. \url{https://doi.org/10.1163/9789004538191}. {[}gedruckt und als
E-Book, Paywall{]}

Die Niederlande waren im 17. Jahrhundert wohl das Land mit dem grössten
Buchmarkt. Gründe dafür waren die calvinistische Orientierung der damals
neuen Republik, die eine Bevölkerung hervorbrachte, welche hoch
alphabetisiert war, da sie die Bibel direkt lesen können sollte; eine
grosse Zahl von gebildeten Geistlichen, die aus den staatlicherseits
eingerichteten Universitäten stammten sowie das rasante
Wirtschaftswachstum, das die Niederlande zu einem Mittelpunkt eines
Netzes machten, an dem -- nicht nur im Buchmarkt -- viele
wirtschaftliche Stränge zusammenflossen. Nicht zuletzt regten zahlreiche
theologische Auseinandersetzungen, sowohl innerhalb des Calvinismus als
auch mit anderen Denominationen, die ständige Produktion und Konsumtion
von Literatur an.

In diesem Klima etablierte sich die Buchauktion als Teil des
Buchmarktes. Starben Personen, die eine einigermassen ansehnliche Anzahl
an Büchern besassen, wurden diese oft versteigert. Daneben wurden oft
auch Buchbestände von Verlagen versteigert, um schnell frisches Kapital
für andere Buchprojekte einzuwerben. Für viele dieser Auktionen wurden
Auktionskataloge gedruckt, die teilweise über das ganze Land und darüber
hinaus verbreitet wurden, damit auch Personen aus anderen Städten
mitbieten konnten. Für diese Kataloge entwickelten sich inoffizielle
Standards: Meist teilten sie die Bücher in Kategorien, erwähnten ihre
Sprache und so weiter. Und: Eine ganze Anzahl dieser Kataloge wurde in
Bibliotheken bis heute überliefert, obgleich sie eigentlich explizite
Verbrauchsliteratur darstellten.

Basis der vorliegenden Studie waren nun 2092 dieser Kataloge aus dem 17.
Jahrhundert, die heute noch vorhanden sind und die der Autor daraufhin
auswertete, welche Bücher in ihnen zur Versteigerung angezeigt wurden.
(Eventuell finden sich in Zukunft noch mehr. Es ist auch bekannt, dass
es noch mehr Auktionen gab, die zum Beispiel in Zeitungen der Zeit
erwähnt wurden.) Bei den Büchern handelte sich um solche, welche die
Eigentümer am Ende ihres Lebens besassen -- sicherlich nicht alle
Bücher, die sie gelesen hatten, da Bücher auch verliehen oder verschenkt
wurden oder weil Familien nicht alle Bücher versteigern wollten. Aber
sie bieten doch einen Einblick in die privaten Buchbestände der
damaligen Zeit. Der Autor hat die Einträge dieser Kataloge in einer
Datenbank versammelt und stellt die Auswertung dieser Daten dar. Sie
wird in den Kontext der damaligen Debatten und Kultur eingebunden. Es
ist ein narrativer Text, in den zahlreiche Tabellen integriert sind,
aber zum Beispiel auch auf Bilddokumente verwiesen wird. Das alles
geschieht sehr ausführlich. Immer beginnt der Autor ein Kapitel mit
einer Vignette, beispielsweise einer Auktion oder einem Theologen und
seinen Büchern, um dann anhand seiner Daten allgemeiner zu werden. Da
Priester und Theologen die meisten Bücher hatten, geht es dabei meist
auch um theologische Debatten.

Grundsätzlich entsteht das Bild einer alphabetisieren Gesellschaft, in
denen Fragen des Glaubens noch den Mittelpunkt des Denken darstellten,
aber auch zum Beispiel die Wirtschaft oder die Philosophie, die sich von
der Theologie zu lösen begann, eine immer grössere Rolle spielten.
Sichtbar wird auch, wie Latein als Sprache weiter vorherrschte, aber im
17. Jahrhundert dabei ist, langsam von Alltagssprachen abgelöst zu
werden. Und nicht zuletzt wird sichtbar, wie Bücher immer mehr zum
Allgemeingut wurden, also auch immer mehr \enquote{handliche} Ausgaben
erschienen. Das ist alles interessant, allerdings teilweise sehr
ausführlich und deshalb langsam zu lesen. (ks)

\begin{center}\rule{0.5\linewidth}{0.5pt}\end{center}

Bassermann-Jordan, Gabriele von ; Fromm, Waldemar ; Haug, Christine ;
Raabe, Christiane (Hrsg.) (2024). \emph{Jella Lepman: Journalistin,
Autorin, Gründerin der Internationalen Jugendbibliothek, Eine
Wiederentdeckung}. (Bavarian, Münchner Schriften zur Buch- und
Literaturgeschichte, Kleine Reihe ; 4). München: Allitera Verlag, 2024
{[}gedruckt{]}

Dieser Band versammelt die Vorträge einer Tagung, die 2020 stattfinden
sollte, aber aufgrund der Covid-19 Pandemie ausfiel. Die Tagung hätte
sich, wie im Titel angegeben, mit Jella Lepman beschäftigt, nicht nur
mit ihrer Rolle als Gründerin der 1949 in München eröffneten
Internationalen Jugendbibliothek, sondern auch mit anderen Aspekten
ihrer Arbeit. Stattdessen erfolgt dies nun über die Beiträge in diesem
Buch.

Lepman, in der Weimarer Republik Politikerin und Journalistin im
liberalen Milieu, floh -- verfolgt wegen dieser Tätigkeiten und ihrer
jüdischen Herkunft -- vor dem nationalsozialistischen Regime aus
Deutschland. Nach 1945 kehrte sie zurück, um die US-amerikanische
Besatzungsmacht bei deren Kulturarbeit zu unterstützen. Dabei hielt sie,
wie in mehreren Beiträgen betont, explizit Abstand zur deutschen
Bevölkerung und deutschen Behörden, beispielsweise indem sie generell
englisch sprach und eine US-amerikanische Uniform trug. Aber, sehr dem
Denken der Jahrhundertwende verhaftet, entwickelte sie die Hoffnung,
dass ein geistiger Neuaufbau Deutschlands möglich wäre, wenn man bei den
Kindern und Jugendlichen ansetzen und diese früh im Leben mit
demokratischen, weltoffenen Vorstellung in Kontakt bringen würde. Dies
ging sie unter anderem mit Ausstellungen von Kinderbüchern und
Kinderzeichnungen an, anschliessend mit der Gründung der genannten
Internationalen Jugendbibliothek, welche von Anfang an und bis heute
einen grossen Kinder- und Jugendbuchbestand mit einer reichhaltigen
Veranstaltungsarbeit verbindet.

Die Beiträge in diesem Band beleuchten die Person Lepman, ihre Arbeit
erst in der US-amerika\-nischen Verwaltung und später in der Münchner
Politik sowie auch ihre eigenen Kinderbücher und ihre Zusammenarbeit mit
Ernst Kästner. Stil und Fokus der Beiträge sind sehr unterschiedlich. Zu
bemerken ist aber auch, dass die Autor*innen sich nicht abgestimmt haben
oder redaktionell viel in ihre Beiträge eingegriffen wurde: Mehrere
Aussagen werden mehrfach gemacht, mehrere Vorgänge werden mehrfach
berichtet und dann unterschiedlich bewertet. Insbesondere gibt es
verschiedene Deutungen dazu, was über Lepmans Ideen zu sagen ist. Was
allerdings praktisch nicht zu finden ist, ist eine Geschichte der
Internationalen Jugendbibliothek selber: Zum Bestand, zur konkreten
Arbeit der Bibliothek oder deren Entwicklung, nachdem sie dann einmal
gegründet wurde, erfährt man praktisch nichts. (ks)

\hypertarget{weitere-wissenschaftliche-medien-konferenzberichte-abschlussarbeiten}{%
\section{4. Weitere wissenschaftliche Medien (Konferenzberichte,
Abschlussarbeiten)}\label{weitere-wissenschaftliche-medien-konferenzberichte-abschlussarbeiten}}

Zumstein, Philipp (2023). \emph{Der Weg ist nicht das Ziel: Über Ideale
und Irrwege bei der Open-Access-Transformation. Open-Access-Tage 2023
(OAT23), Berlin}. Folien zum Vortrag:
\url{https://doi.org/10.5281/zenodo.8388502}, Videoaufzeichnung:
\url{https://doi.org/10.5446/66708}.

Philipp Zumstein hält mit dem Vortrag bei den Open-Access-Tagen 2023,
was der streitbare Titel verspricht: Gegenübergestellt werden die
programmatischen Ziele (Forschende und/oder Entscheidungsebene
unterstützen, Verträge abschließen, OA-Anteil erhöhen, OA weltweit
ermöglichen, nachhaltige und faire Publikationslandschaft ermöglichen)
und anfallende Aufgaben im Bereich Publikationsservices
(Publikationsdatenmanagement, Publikationsservices, Abschluss und
Abwicklung von OA-Verträgen, Beratung von Wissenschaftler*innen,
Auswertung und Reporting).

Der Referent stellt die wichtige Frage, welchen Zielen bestimmte
Aufgaben dienen -- und hinterfragt, ob denn bestimmte Handlungs- oder
Aufgabenfelder (Bewirtschaftung Publikationsfonds, Einführung eines
Informationsbudgets) auch de facto der Umsetzung des übergeordneten
Ziels der Transformation dienen (können). Oder anders gesagt empfiehlt
er, vermeintlich offensichtliche Aufgabenstellungen kritisch zu
hinterfragen: Ob und welche Ziele mit bestimmten Aufgaben umgesetzt
werden \emph{können}. Und ob nicht manche Tätigkeiten bewusst
unterlassen oder niedrig priorisiert werden sollten. Zumstein
argumentiert dies am Beispiel der Informationsbudgets und einer
granularen Erfassung verschiedener Publikationskostenarten -- er
plädiert dafür, besser für Strukturen zu sorgen, damit "antiquierte"
Publikationsgebühren (etwa color oder page charges) nicht mehr gezahlt
werden -- anstatt immer besser darin zu werden, die verschiedenen
Kostenarten granular zu erfassen und umfangreich mit
Publikationsmetadaten zu verknüpfen.

Zumstein schließt den Vortrag mit Appellen und Empfehlungen, um eine
kritische Diskussion der Ziele der Open-Access-Transformation anzuregen.
Genau für eine solche Diskussion -- mit den Leitungen von Bibliotheken
und Hochschulen, wie auch innerhalb der OA-Community -- liefert er viele
Anregungen. (mv)

\begin{center}\rule{0.5\linewidth}{0.5pt}\end{center}

Schön, Margit ; Barbers, Irene; Mittermaier, Bernhard (2024).
\emph{Publikationskostenmonitoring: Aktueller Stand und
Herausforderungen von Publikationskosten an deutschen wissenschaftlichen
Einrichtungen}. Report. \url{https://doi.org/10.5281/zenodo.10810729}

Der Report genannte Bericht ist eine Darstellung und
Ergebniszusammenfassung eines Workshops mit über 200 Teilnehmenden sowie
einer Umfrage mit acht Teilnehmenden, die im Feld aktiv sind. Zudem
werden der Kontext geschildert und die Ergebnisse detailliert
dargestellt, was den Text recht umfangreich macht.

Grundsätzlich zeigt sich, dass das Thema \enquote{Publikationskosten}
von Open-Access-Veröffentli\-chungen in Bibliotheken und Hochschulen zwar
als Thema bekannt ist und zumindest bei den Personen, die am Workshop
teilnahmen, auch im Arbeitsalltag eine Bedeutung hat. Aber gleichzeitig
hat sich keine einheitliche Praxis etabliert. Vielmehr werden lokal
immer wieder andere Wege gegangen, um einen Überblick zu den anfallenden
Kosten zu erhalten.

Es gibt in der Soziologie und der Geschichtswissenschaft das Konzept der
\enquote{Pfadabhängigkeit}: Institutionen und Felder folgen tendenziell
einmal \enquote{eingeschlagenen} Pfaden. Wahrgenommene Probleme werden
eher mit ähnlichen Konzepten angegangen, die schon einmal ausprobiert
wurden; Lösungen werden eher in schon vorhandenen Strukturen und
Denkmustern integriert, als jeweils ganz neu entworfen zu werden. Dieser
Bericht hinterlässt stark den Eindruck einer solchen Pfadabhängigkeit im
deutschen Bibliothekswesen: In den letzten zwei Jahrzehnten wurden im
Bereich Open Access bestimmte Lösungen und Strukturen etabliert, zum
Beispiel institutionsübergreifende Mandate für Forschende,
Open-Access-Büros, die in Bibliotheken angesiedelt sind, inklusive der
Delegation der Verantwortung für Open-Access-Projekte an die
Hochschulbibliotheken. Beim Bericht fällt nun auf, dass die
Problembeschreibungen -- die Kosten fallen ohne Übersicht an
verschiedenen Stellen in Hochschulen an -- und Lösungsansätze -- die
Situation ist unübersichtlich und bedarf einer Vereinheitlichung, die
Verantwortung soll bei den Bibliotheken liegen -- sich dem recht
erfolgreichen Modell im Bereich Open Access ähneln. Andere Möglichkeiten
-- beispielsweise die Verlage zur Information zu verpflichten -- werden
gar nicht erst thematisiert. (ks)

\hypertarget{populuxe4re-medien-social-media-zeitungen-zeitschriften-radio-tv}{%
\section{5. Populäre Medien (Social Media, Zeitungen,
Zeitschriften, Radio,
TV)}\label{populuxe4re-medien-social-media-zeitungen-zeitschriften-radio-tv}}

Kristof, Nicholas (2023). \emph{We know the cure for loneliness. So why
do we suffer?} In: The New York Times -- International Edition.
September 8, 2023. S. 9,
\url{https://www.nytimes.com/2023/09/06/opinion/loneliness-epidemic-solutions.html?unlocked_article_code=JEOPFGKQBauhm0-RPny94Guinp1-eOMASsQy3iPuUjw-xSH_g7LVhlnzh4bRtJUSOaKtA1N0pX19hHM9_jN1HNQ18EEDeuCn21PbfKGDIpo8sgiwrZsjMqnVcJhUD0stLlbnLdUWTWN65q6XaXBVKtICx-7f1IsXMJ8QLdyaT2gHkEU8MX9XEFBoGnQbxykPTGARDA3Opi1GXOKH2rQ9gsAF7TOMIc8iZi2oyrQ1BOB6XfwSYquCqyUKipbI9Na3uzOBOnAmT_Pf5WmeEqkAdDWW3W3cAQO47OPjjB3YtPZDhuu4diDGG2_64FwvsFiXm0l0ZS1JD6OBzOdb4OhBuGry4Nar8GE\&smid=url-share}
{[}Paywall{]}

In einem ausführlichen Beitrag über Einsamkeit als soziale und
gesundheitliche Herausforderung -- Einsamkeit ist so
gesundheitsschädlich wie 15 Zigaretten am Tag, so eine Aussage --
verweist der Autor unter anderem auf Hilfs- und Public-Health-Programme.
Ein wichtiges Leitdokument für die USA ist der im Mai 2023 vorgelegte
Bericht \enquote{Our Epidemic of Loneliness and Isolation} (https://www.ncbi.nlm.nih.gov/books/NBK595227/, Office of
the Surgeon General, 2023), der an mehreren Stellen
Bibliotheken als Eckpfeiler einer der Einsamkeit vorbeugenden sozialen
Infrastruktur betont: \enquote{{[}The report{]} offers a strategy to
address loneliness that begins with building up the infrastructure that
enables social connection. That includes physical infrastructure, such
as parks and libraries, and also social infrastructure to weave together
volunteers or enthusiasts with similar interests.} Zudem erwähnt
Nicholas Kristof eine \enquote{Bibliothek der Dinge} in der britischen
Stadt Frome, die passenderweise SHARE FROME heißt, siehe
\href{https://sharefrome.org/}{\uline{https://sharefrome.org/}}. (bk)

\begin{center}\rule{0.5\linewidth}{0.5pt}\end{center}

Caspari, Heinrich: \emph{Lieferung an italienische staatliche
Bibliotheken betreffend}. In: Börsenblatt für den Deutschen Buchhandel.
Nr. 179, 03. August 1907, S. 10

In einem Brief an das Börsenblatt für die Rubrik \enquote{Sprechsaal}
teilt Heinrich Caspari von der Stuhrschen Buchhandlung in Berlin mit,
dass eine von seiner Buchhandlung versorgte staatliche Bibliothek in
Italien den Bezug nach offizieller Anordnung einstellen muss, da
entsprechende Abrechnungen nur noch innerhalb Italiens möglich sind. Der
Buchhändler fragt nach den Erfahrungen anderer Buchhandlungen. (bk)

\begin{center}\rule{0.5\linewidth}{0.5pt}\end{center}

o.A.: \emph{From Kenosha to the Children of Italy}. In: Wisconsin
Library Bulletin. January 1947, S. 26

Als Teil einer Woche des Buches wurde an der Boys and Girls Library in
Kenosha, Wisconsin, der ersten Kinder- und Jugendbibliothek der Stadt,
eine völkerverständigende beziehungsweise Kultur exportierende
Schatztruhe für Kinder in Italien zusammengestellt. Delourise I. Layman,
Bibliothekarin und damit mutmaßlich verantwortlich für die Aktion,
berichtet, dass Schüler*innen der Stadt eingeladen wurden, an der
Gestaltung der Kiste mitzuwirken. Es entstand unter anderem ein
\enquote{Scrapbook}, in dem die Kinder über ihr Leben in Kenosha
berichten. Für die Empfänger*innen in Italien wurden ein leeres Album
sowie Stifte, Farben und weitere Materialien beigelegt und zwar in der
Hoffnung, dass sie es ihren Altersgenoss*innen in Wisconsin gleichtun.
Außerdem fanden sich in der Schatztruhe dreißig Bücher zeittypischer
Kinderliteratur wie The Story of Little Black Sambo, Daniel Boone,
Millions of Cats und Make Way for Ducklings. (bk)

\begin{center}\rule{0.5\linewidth}{0.5pt}\end{center}

Schmitz, Jasmin: \emph{Beyond Predatory Publishing: Additional
Questionable Offers in Scholarly Publishing}. In: Scholarly
Communications in Transition. A Blog about Predatory and Other Phenomena
in Academia, 10.01.2024,
\url{https://in-transition.at/beyond-predatory-publishing-additional-questionable-offers-in-scholarly-publishing/}

Im Beitrag von Jasmin Schmitz werden Vorgehensweisen und Praktiken von
Paper Mills, die wissenschaftliche Artikel und deren Platzierung in
Zeitschriften verkaufen, beschrieben. Sie geht dabei näher auf deren
üblichen Geschäftspraktiken und Vorgehensweisen ein und spricht über die
Auswirkungen dieses wissenschaftlichen Fehlverhaltens auf die Forschung
insgesamt. Der Blogbeitrag bietet dabei einen übersichtlichen Einstieg
in aktuelle Entwicklungen und verleitet mit ausführlichen Referenzen zur
vertieften Recherche. (eb)

\begin{center}\rule{0.5\linewidth}{0.5pt}\end{center}

Oreskes, Naomi: \emph{Trouble in the Fast Lane. Scientific research
needs to slow down, not speed up}. In: Scientific American, Vol. 330,
No.~4 (April 2024), S. 69

In einer Art Op-Ed problematisiert die Autorin Probleme und Folgen
hochfrequenter Publikationskulturen in einer stark kompetitiv
ausgerichteten Wissenschaftswelt anhand der Zunahme wissenschaftlichen
Fehlverhaltens vor allem durch Nachlässigkeit bei der Erstellung von
wissenschaftlichen Veröffentlichungen. Eine Ursache ist nach der die
große Bedeutung der Publikationsquantität für die
Wissenschaftsevaluation, die zugleich mit einer generellen
Beschleunigungserzählung verknüpft wird, die auch von außerhalb der
Wissenschaft getrieben wird. Leitmantra: \enquote{Move fast and break
things}. Die Autorin erinnert daran, dass Wissenschaft sowohl in der
Erkenntnisfindung als auch in der Kommunikation Zeit braucht und
verweist auf absurd erscheinende Zahlen wie die in einer Studie
nachgewiesenen Zahl von 265 Autor*innen, die im Schnitt alle fünf Tage
ein Paper veröffentlichten. Der Artikel beschreibt also ein
wohlbekanntes Phänomen, das für die Bewertung der Forschung bislang
nicht gelöst ist. (bk)

\begin{center}\rule{0.5\linewidth}{0.5pt}\end{center}

McKie, Robin: '\emph{The situation has become appalling': fake
scientific papers push research credibility to crisis point}. In: The
Observer / guardian.com, 03.02.2024
\url{https://www.theguardian.com/science/2024/feb/03/the-situation-has-become-appalling-fake-scientific-papers-push-research-credibility-to-crisis-point}

Im \emph{Observer} wird vor dem Hintergrund stark steigender
Retraction-Zahlen für wissenschaftliche Aufsätze das Problem gefälschter
beziehungsweise mit unwissenschaftlichen Mitteln geschönten
Wissenschaftspublikationen aufgegriffen. Die Rekordzahl von 10.000
zurückgezogenen Artikeln im Jahr 2023 ist laut Expert*innen nur die
Spitze des Eisbergs. Für den in Kairo ansässigen Wissenschaftsgroßverlag
Hindawi, der besonders an- und auffällig für die Publikation solcher
Artikel scheint, zieht Wiley als Eigentümer jetzt die Reißleine, gibt
die Marke auf und ändert die Strukturen. Die Ursache für den rasanten
Zuwachs auch vorsätzlich gefälschter Forschung wird in einem vor allem
in China stark ausgeprägten Publikationsdruck für Nachwuchsforschende
gesehen, bei dem entsprechende Publikationslisten die Grundlage für ein
berufliches Fortkommen darstellen. Dies führte zu einer Art
Sekundärmarkt mit sogenannten \enquote{Paper mills}, die gezielt
gefälschte Studien produzieren, sowie Manipulationsstrukturen bis hin
zur Bestechung. Das Phänomen breitet sich zunehmend auf andere Länder
wie Indien, Russland oder Iran aus. Der hohe Anteil gefälschter
Forschung wirkt wiederum auf die wissenschaftliche Kommunikation und
Anschschlussforschungen sowie auch auf die Wahrnehmung von Forschung in
der Öffentlichkeit zurück, was sich während der Covid-Pandemie deutlich
zeigte. Die Langzeitfolgen könnten noch verheerender werden, wenn
Personen mithilfe gefälschter Forschung Karriere bis in
wissenschaftliche Steuerungspositionen machen. (bk)

\begin{center}\rule{0.5\linewidth}{0.5pt}\end{center}

Jochen Zenthöfer: \emph{Schwarzmarkt für den Zitatkauf}. In: Frankfurter
Allgemeine Zeitung, 21.02.2024, S. N4

Der Artikel berichtet über den Trend der Manipulation von Zitationen zur
Erhöhung von Zitationshäufigkeiten und damit dem quantitativen Standing
von Wissenschaftler*innen. Neben den traditionellen Zitierkartellen hat
sich eine Art Manipulationsmarkt entwickelt. Ein zitierter Experte
berichtet zudem sogar über parallele Publikationsstrukturen, bei denen
in Fake-Publikationen andere Fake-Publikationen zitiert werden, um
Impact-Factor-Angaben zu manipulieren. Er spricht auch von einer
\enquote{Verschmutzung des wissenschaftlichen Publizierens}. Anlass des
Artikels ist neben dem Aufzeigen des Phänomens auch ein Beklagen, dass
Google Scholar, Clarivate und andere Anbieter offenbar unzureichend oder
gar nicht auf diese Verzerrungseffekte reagieren, die angesichts der
Möglichkeiten von Künstlicher Intelligenz an Intensität und Umfang
zunehmen werden. Den \enquote{Silberstreif am Horizont} der
Wissenschaftsmanipulation sieht der Autor in einer Maßnahme der
chinesischen Regierung, die alle Hochschulen des Landes verpflichtete,
sämtliche Fälle zurückgezogener Publikationen zu melden und auf ein
wissenschaftliches Fehlverhalten zu prüfen. (bk)

\begin{center}\rule{0.5\linewidth}{0.5pt}\end{center}

Zoë Beery: \emph{Want a Synthesizer? Go Ahead, Take One}. In: New York
Times / nytimes.com 08.12.2023,
\url{https://www.nytimes.com/2023/12/08/nyregion/synth-library-brooklyn.html}
{[}Paywall{]} beziehungsweise
\url{https://web.archive.org/web/20240410171654/https://www.nytimes.com/2023/12/08/nyregion/synth-library-brooklyn.html}

In New York gibt es eine von Freiwilligen betriebene \enquote{Synth
Library NYC}, deren offensichtliches Anliegen der Verleih von
Synthesizern ist. Davon stehen 73 Geräte zur Verfügung, wie Zoë Beery
berichtet. Das damit verbundene zweite Anliegen ist nicht nur, Menschen
den Zugang zu den teuren und für viele unerschwinglichen Instrumenten zu
erleichtern, sondern auch der Aufbau einer Community um die
Synth-Klangkultur. Dafür gibt es bei Bedarf entsprechende Beratungen und
Einführungen durch Mitglieder dieser Community. Erwartungsgemäß gibt es
auch Workshops und öffentliche Veranstaltungen, die im Einklang mit dem
Wunsch der Betreibenden einen besonderen Zweck haben: \enquote{spread
the synth bug}. In den zweieinhalb Jahren der Existenz der
Synth-Bibliothek ging übrigens noch keines der Instrumente verloren.
(bk)

\begin{center}\rule{0.5\linewidth}{0.5pt}\end{center}

Rachel Felder: \emph{Colette's Sarah Andelman Is Back With Another
Idea}. In: New York Times / nytimes.com, 28.02.2024.
\url{https://www.nytimes.com/2024/02/28/style/sarah-andelman-bon-marche.html?unlocked_article_code=1.aE0.DDNA.O0AvBI2ZQZLS\&smid=url-share}
{[}Paywall{]}

In einem Beitrag über die Pariser Trendhändlerin Sarah Andelman
berichtet die Style-Abteilung der New York Times über deren aktuelles
Projekt einer auf Buch, Buchhandlungen und Bibliotheken bezogenen
Verkaufsausstellung im Kaufhaus Le Bon Marché Rive Gauche. Mit im
Angebot sind einige Produkte mit \enquote{Bibliotheksdüften}: Duftkerzen
(Byredo, Diptyque, Zigzag Island) sowie Keramikbleistifte, die neben
ihrer Schreibfähigkeit auch eine olfaktorische Spur zur Bibliothek von
Alexandria transportieren sollen (Officine Universelle Buly). Wer auf
der Webseite des Kaufhauses herumstöbert, entdeckt noch weitere
bibliotheksmodische Accessoires, beispielsweise \enquote{Library Card
Yellow}-Socken der amerikanischen Buchkulturvermarkter \enquote{Out of
Print}. Buch- und Bibliothekskultur sind offenbar nach wie vor lebendig,
zugleich aber außergewöhnlich genug, um als Distinktionsthema zu wirken.
(bk)

\begin{center}\rule{0.5\linewidth}{0.5pt}\end{center}

Sam Lubell: \emph{An Architect Builds Toward the Future on Mexico's
Border}. In: New York Times / nytimes.com. 01.03.2024.
\url{https://www.nytimes.com/2024/03/01/arts/design/mexico-border-architecture-canales.html?unlocked_article_code=1.aE0.Bi3H.o_5dfKmRwZF6\&smid=url-share}
{[}Paywall{]} beziehungsweise
\url{https://web.archive.org/web/20240326144835/https://www.nytimes.com/2024/03/01/arts/design/mexico-border-architecture-canales.html}

Die mexikanische Architektin Fernanda Canales entwarf für die
unmittelbar an der Grenze zwischen Mexiko und den USA gelegene und
entsprechend stark herausgeforderte Stadt Agua Prieta ein Kultur- und
Nachbarschaftszentrum inklusive Bibliothek. Der Artikel beschreibt kurz
architektonische Merkmale und die Einpassung in die örtliche Situation
(\enquote{a woven, bar-shaped building, its arched edges sitting
parallel to the striated, mural-saturated steel border wall about 10
feet away and just west of the town's international border crossing})
mit einer amphitheaterartigen Einlassung für Aufführungen und widmet
sich dann der symbolischen und sozialen Funktion. Die Platzierung der
Bibliothek wurde bewusst gewählt, um der einschüchternden und
buchstäblich separierenden Form einer stark befestigten Grenze einen
verbindenden Kontrapunkt entgegenzusetzen. Die Bibliothek ist nun
beispielsweise Ort eines lokalen Kulturfestivals und dient einem
Buchclub als Ort seiner samstäglichen Sitzungen. (bk)

\begin{center}\rule{0.5\linewidth}{0.5pt}\end{center}

k.A.: \emph{The Woman\textquotesingle s Library}. In: New York Times.
25.08.1860. S. 4. \url{https://nyti.ms/48BHaFH} {[}Paywall{]}

Die New York Times vermeldete am 25. August 1860 die anstehende
Eröffnung der ersten Bibliothek für Frauen in New York in einem
Universitätsgebäude am Washington Square. Damit sollte einem wachsenden
Informations- und Lesebedürfnis vor allem durch Immigration und
Industrialisierung zunehmenden Zahl von arbeitenden Frauen in der Stadt
Rechnung getragen werden. Die wenigen auch für Frauen zugänglichen
Bibliotheken der Stadt waren für die Zielgruppe, also die
Arbeiterinnenklasse, meist nicht nutzbar, da sie Nutzungsgebühren
erhoben. Diese lagen, so der Artikel, höher als die für Männer, obwohl
das Durchschnittseinkommen von Männern zu dieser Zeit dreimal höher war
als das von Frauen. Andere Bibliotheken schlossen Frauen von vornherein
aus, da in ihnen eine potentielle Ablenkung für die männlichen Nutzer
gesehen wurde. Einige hegten auch Sittlichkeitsbedenken. Der Eröffnung
der \enquote{Women's library} ging ein zweijähriger Prozess der
Abstimmung, des Aufbaus und vor allem auch der Durchsetzung voraus. Ein
Gegenargument lautete, dass die arbeitenden Frauen gar keine Zeit zur
Bibliotheksnutzung hätten und das Angebot daher eine Verschwendung wäre.
Mit der Einrichtung sollte der damals spürbaren Exklusion von Frauen aus
dem öffentlichen und kulturellen Leben entgegen getreten werden.
Durchgesetzt wurde die Initiative der ersten von \enquote{libraries for
their own} durch zwei Männer, Henry Ward Beecher und James T. Brady. Das
Anliegen war ausdrücklich auch, den Nutzerinnen einen Rückzugsort zu
geben. (bk)

\begin{center}\rule{0.5\linewidth}{0.5pt}\end{center}

Jane Margolies: \emph{Not Just for Scooby-Doo Anymore}. In: New York
Times, March 10, 2024, Section F, S. 9

In einem Bericht über die Popularität von versteckten Räumen in privaten
Neubauten in den USA stellt die Autorin einen Bezug zur
Bibliotheksarchitektur her. Konkret erwähnt sie die Bibliothek, die sich
der Unternehmer und Bankier J. Pierpont Morgan, heute Morgan Library \&
Museum, Anfang des 20. Jahrhunderts in New York vom Architekten Charles
McKim entwerfen ließ. In dieser wurden versteckte Türen in Regalen
untergebracht, hinter denen sich eine Art Betriebstreppenhaus für die
Bibliothekar*innen befanden. Weiterhin ließ sich Morgan in seinem
Arbeitszimmer ein verstecktes Regal in einem verschiebbaren Bücherregal
einbauen, in dem Titel untergebracht werden konnten, von denen
"vielleicht nicht jeder Gentleman möchte, dass es jeder sieht". Auf
einer Webseite der Bibliothek werden drei dieser Titel erwähnt: Fanny
Hill, La nouvelle Sapho und Le diable au corp. (bk)

\begin{center}\rule{0.5\linewidth}{0.5pt}\end{center}

Maya Pontone: Armed Groups in Haiti Ransack National Library. In:
Hyperallergic. April 4, 2024,
\url{https://hyperallergic.com/892251/armed-groups-in-haiti-ransack-national-library/}

Anfang April 2024 wurde die Nationalbibliothek Haitis in Port-au-Prince
von bewaffneten Gruppen geplündert. Inwieweit die Sammlungen selbst
betroffen waren, ist zum Zeitpunkt der Meldung nicht bekannt. Der
Generaldirektor der Bibliothek, Dangelo Neard, berichtet zunächst vom
Diebstahl von Möbeln sowie eines Generators, wies aber auch noch einmal
nachdrücklich auf die Bedrohung für die Rara, die Manuskriptsammlung
sowie das Zeitungsarchiv hin. (bk)

\begin{center}\rule{0.5\linewidth}{0.5pt}\end{center}

Zenthöfer, Jochen (2024). \emph{"Für die Nut­zung dau­er­haft gesperrt":
Umgang mit plagiierten Jura-Büchern}. In: Legal Tribune Online,
26.04.2024,
\url{https://www.lto.de/recht/hintergruende/h/plagiat-jura-dissertation-aberkennung-doktortitel-ausleihe-bibliothek-wissenschaft/}

Der Artikel stellt überblickshaft dar, wie Bibliotheken --
Wissenschaftliche und solche mit gesetzlichem Sammelauftrag -- mit
Büchern umgehen, in denen plagiiert wurde. Kurz: Wenn sie als
Pflichtexemplare in die Bibliotheken kamen, bleiben sie dort. Wenn
nicht, werden sie oft direkt ins Magazin gestellt oder ausgesondert. In
einer steigenden Anzahl von Fällen wird explizit im Katalog vermerkt,
dass es sich um Plagiatsfälle handelte. Ansonsten wird darauf bei der
Ausleihe hingewiesen. Der Text beschränkt sich wegen der Zielgruppe der
Publikation auf juristische Werke, aber selbstverständlich gilt die
Darstellung des Bestandsmanagements, die er liefert, auch für andere
Sachgruppen. (ks)

\begin{center}\rule{0.5\linewidth}{0.5pt}\end{center}

Tondo, Lorenzo: Plato's final hours recounted in scroll found in
Vesuvius ash. In: THE GUARDIAN / guardian.com, 29.04.2024
\url{https://www.theguardian.com/books/2024/apr/29/herculaneum-scroll-plato-final-hours-burial-site}

Eine neue Analyse einer im Jahr 1750 in Herculaneum entdeckten
Schriftrolle enthält einen Bericht über den letzten Abend Platons. Laut
Überlieferung verbrachte der sterbenskranke Philosoph seine letzten
Stunden mit Musik. Ein versklavtes thrakisches Mädchen, so der Bericht,
spielte für ihn auf der Flöte. Trotz seines Zustands ließ es sich Platon
anscheinend nicht nehmen, die Flötenspielerin ob ihres aus seiner Sicht
defizitären Rhythmusgefühls zu kritisieren. (bk)

\begin{center}\rule{0.5\linewidth}{0.5pt}\end{center}

Last Week Tonight With John Oliver: \enquote{Public Libraries}, (Season
11, Episode 10, Erstausstrahlung 5.5.2024), Streaming via HBO
\url{https://www.hbo.com/last-week-tonight-with-john-oliver/season-11/10-may-5-2024-public-libraries};
frei zugänglich via \url{https://www.youtube.com/watch?v=42xZB80sZaI}
(ca. 30 min)

In einer Folge vom Mai 2024 beschäftigt sich John Oliver in der
Late-Night-Talk- und News-Show \enquote{Last Week Tonight} mit
Öffentlichen Bibliotheken in den USA und den Kontroversen um
\enquote*{Book Bans}, die seit einigen Jahren die US-amerikanische
Öffentlichkeit beschäftigen und inzwischen auch im deutschen Diskurs
angekommen sind. Hintergrund ist, dass in den Vereinigten Staaten
zunehmend Bücher aufgrund ihres Inhalts verboten beziehungsweise deren
Zugang in (Öffentlichen oder Schul-) Bibliotheken eingeschränkt, oder
sie gar ganz aus dem Bestand entfernt werden sollen -- ginge es nach den
Vorstellungen einiger Aktivist*innen. Vielfach (nicht ausschließlich)
stehen dabei Medien mit LGBTQIA+ Themen im Mittelpunkt der Diskussion.
John Oliver erläutert in der Folge niedrigschwellig und für ein
Nicht-Fachpublikum Aspekte wie Bestandspflege (inklusive Gründe für
Aussonderungen), Auf- und Einteilung nach Zielgruppe (Kinder,
Jugendliche, Erwachsene) und den (eigentlich intuitiv verständlichen)
Einfluss von Budgetkürzungen auf die Bibliotheksarbeit (inklusive in der
Gemeinde ungewollte Bibliotheksschließungen). Sehempfehlung! Nicht
zuletzt ist es eine satirische Unterhaltungssendung mit Bildungs- und
Aufklärungsanpruch. (mv)

\begin{center}\rule{0.5\linewidth}{0.5pt}\end{center}

Muldowney, Decca (2023). \emph{Meet the Woman Training Parents How to
Get Books Banned}. In: The Daily Beast, 01.12.2023,
\url{https://www.thedailybeast.com/karen-england-is-teaching-parents-how-to-get-books-banned-starting-with-chino-valley}

Der Artikel stellt mit Karen England eine rechte, christliche Aktivistin
vor, welche die \enquote{Verbote} von Büchern in Öffentlichen und
Schulbibliotheken in den USA vorantreibt. Eingebettet ist dies in die
Geschichte einer kalifornischen Gemeinde, die tatsächlich ein Verbot von
\enquote{vulgären} Büchern in ihren Bibliotheken beschlossen hat, aber
auch der Gegenwehr gegen diesen Beschluss.

Es wird klar, dass es sich bei diesen Verboten um eine organisierte
politische Kampagne handelt, die sich gerade nicht durch einzelne
\enquote{empörte} Eltern erklärt, sondern einer mehr oder minder
straffen, seit Jahrzehnten aktiven Organisation. Es sind einige
Personen, welche diese Kampagne landesweit vorantreiben und eine kleine
Anzahl von Personen, welche diese jeweils lokal tragen. Der Eindruck
eines kulturellen Shifts, der von dem Grossteil der Bevölkerung getragen
wird, ist falsch. Eindrücklich beschreibt der Artikel auch, inklusive
einem Interview mit der im Mittelpunkt des Textes stehenden Karen
England, dass hinter dieser Kampagne vor allem US-amerikanisch,
fundamentalistisch-christliche Ziele stehen, auch wenn England mit der
Zeit gelernt hat -- und dies anderen beibringt -- eine Rhetorik zu
nutzen, die anderes verkündet.

Für den DACH-Raum relevant ist dieser Text, weil er zumindest die Frage
aufwirft, ob es solche Aktivist*innen wie England auch hier gibt, also
ob sich die -- so ja Wahrnehmung im Bibliothekswesen -- steigende Anzahl
von Versuchen, Bücher in Bibliotheksbeständen zu zensurieren, (auch)
durch so eine Kampagne erklären lässt. (ks)

\hypertarget{weitere-medien}{%
\section{6. Weitere Medien}\label{weitere-medien}}

Esoterica / Justin Sledge (2023). \emph{The Library of Alexandria - Myth
vs History}. 01.12.2023, (Video, 39:10 Minuten),
\url{https://youtu.be/iFsM56nN84o?feature=shared}

\emph{Esoterica} ist ein Youtube-Kanal, auf dem der
Religionswissenschaftler Justin Sledge
(\url{https://www.justinsledge.com}) Videos zu religiösen und
philosophischen Themen veröffentlicht. Sein Hauptfokus liegt dabei auf
dem europäischen Mittelalter, Alchemie und jüdischer Mystizistik. Die
Videos behandeln teilweise sehr spezielle Fragen, aber immer aus einer
wissenschaftlichen Perspektive -- wobei Slate erstaunlich oft seine
Abneigung gegen bestimmte, insbesondere französische, philosophische
Traditionen erwähnt und gleichzeitig betont, dass auch die
mittelalterliche Mystik und Alchemie eine hohe Rationalität aufwiesen.

Das hier erwähnte Video weicht, wie im Titel sichtbar, thematisch davon
ab. In ihm stellt Sledge vor, was über die (antike) Bibliothek von
Alexandria bekannt ist. In einer Sektion, in der er diskutiert, wie
viele Papyrusrollen überhaupt realistisch in der Bibliothek vorhanden
hätten sein können, thematisiert er auch diese Rollen und holt eine
solche aus dem Regal neben sich hervor. Diese hat er einst selber
beschrieben, praktisch als forschende Übung. Im Video führt er ihre
Materialität vor.

Zur Bibliothek selber fasst er grundsätzlich den aktuellen
Forschungsstand zusammen und kommentiert ihn aus seiner persönlichen
Perspektive. Man erfährt also eigentlich nichts Neues, wenn man sich mit
der Forschung einigermassen auskennt. Aber man erhält hier eine
sympathische, gleichzeitig recht nüchterne Darstellung. Es wird dabei
nicht nur thematisiert, dass die Bibliothek von vielen unrealistischen
Mythen umrankt ist, dass sie selbstverständlich nicht \enquote{von
Caesar} verbrannt wurde (und auch nicht von anderen Herrschenden),
sondern dass ihre Gründung, Finanzierung und ihr Niedergang abhängig war
von den unterschiedlichen politischen Interessen der wechselnden
Machthaber*innen in Ägypten. (ks)

\begin{center}\rule{0.5\linewidth}{0.5pt}\end{center}

Sophia Kimmig: \emph{Von Füchsen und Menschen}. München: Piper Verlag,
2022

In ihrem Buch über Füchse und die Stadtnatur erwähnt die Autorin den
Bibliotheksfuchs der Staatsbibliothek zu Berlin, Haus Unter den Linden,
der offenbar das einzigartige Klettertalent der Tiere dadurch zum
Ausdruck brachte, dass er hin und wieder auf Fensterbrettern auftauchte.
(vergleiche S. 29f.) (bk)

\begin{center}\rule{0.5\linewidth}{0.5pt}\end{center}

Robert Barry: \emph{compact disc}. New York: Bloomsbury Academic, 2020

In seiner Mediengeschichte der Compact Disc (CD) berichtet Robert Barry,
warum als Leitverpackung das Jewelcase gewählt wurde. Das Marketing von
Philips betonte, dass diese Lösung auf archivarische Ansprüche
zugeschnitten wurde, da die Angaben zum Inhalt der CD mit dem so
standardisierten und geschützten Labeling im Vergleich zu Schallplatten
in der Aufreihung im Regal, also auch bei großen Sammlungen, deutlicher
lesbar bleibt. (vergleiche S. 119) (bk)

\begin{center}\rule{0.5\linewidth}{0.5pt}\end{center}

Marc Masters: High Bias: \emph{The distorted history of the cassette
tape}. Chapel Hill: The University of North Carolina Press, 2023

In seiner Medien- und Kulturgeschichte der Kompaktkassette nicht zuletzt
als soziales Medium erwähnt Marc Masters auch einige Bezüge zur
Bibliothek. So schildert er als Inspiration des Multimedia-Künstlers
beziehungsweise \enquote{Cassette composers} Jason Zeh, das Medium ins
Zentrum seiner Arbeit zu rücken, dass dessen Vater sich Schallplatten
aus der Bibliothek auslieh, um sie zuhause auf Kassette zu kopieren und
zudem mit einem Fotokopiergerät aus den Schallplattencovern auch ein
entsprechendes Artwork für die Tape-Kopien anzufertigen. (vergleiche S.
78) Der Künstler Don Campau wiederum verteilte die Tapes seines Labels
Lonely Whistle im Rahmen seiner Guerilla-Dissemination nicht nur an
nichtsahnende Kunden an der Kasse seines Brotberufs sondern auch an
lokale Bibliotheken. (S. 65) Andere Künstler wie Aaron Dilloway nutzten
dagegen explizit für Bibliotheken hergestellte Abspieltechnologien,
nämlich den Telex C-1 Library of Congress Tape Player for the Blind, als
künstlerisches Werkzeug. (S. 74) Bibliotheksbestände spielen eine Rolle
für das Sichtbarmachen von besonderer Musik, wie das Beispiel von Mark
Gergis zeigt. Nachdem er sich für die Verbreitung syrischer Musik in der
westlichen Welt engagiert hatte, erstellte er eine Compilation namens
\emph{Cambodian Cassette Archives: Khmer and Pop Music Vol. 1} auf
Grundlage einer Sammlung seltener Aufnahmen der Asian Branch Library im
öffentlichen Bibliothekssystem von Oakland. Gergis berichtet, dass diese
nur auf Kassetten konservierte Musik im Ursprungsland längst
verschwunden war. (S. 122\,f.) Seine eigene über lange Jahre
zusammengetragene Sammlung syrischer Kassettenkultur digitalisiert und
präsentiert er über eine Seite namens Syrian Cassette Archives
(\url{https://syriancassettearchives.org/}). (bk)

\begin{center}\rule{0.5\linewidth}{0.5pt}\end{center}

British Library: \emph{Learning Lessons from the Cyber Attack. British
Library Cyber Incident Review}. 8 March 2024.
\url{https://www.bl.uk/home/british-library-cyber-incident-review-8-march-2024.pdf}.

In diesem Bericht über den Cyberangriff, der die britische
Nationalbibliothek im Oktober letzten Jahres fast völlig
handlungsunfähig machte, werden Hergang, Gründe und Auswirkungen des
Angriffs dargestellt und Pläne für den Wiederaufbau sowie Lessons
Learned aufgeführt. Beim Durchlesen fallen viele Punkte ins Auge, die so
nicht nur die British Library, sondern praktisch alle Bibliotheken
betreffen (viele veraltete Anwendungen und Systemstrukturen, wenig
Multi-Faktor-Authentifizierung et cetera). Er zeigt klar, wie
verletzlich nahezu alle unsere Systeme sind und ermutigt stark zur
Erstellung eines Notfallplans für Cyberattacken. (eb)

\begin{center}\rule{0.5\linewidth}{0.5pt}\end{center}

Byredo: \emph{Bibliothèque} (Duftkerze)

In der Produktbeschreibung stellt der schwedische Parfum- und
Lifestyle-Hersteller Byredo die These auf, dass Bibliotheken
\enquote{über Moden und Trends erhaben} sind und \enquote{uns in eine
Welt, in der die Zeit stillzustehen scheint} entführen. Dies versuchten
Firmengründer Ben Gorham und Parfümeur Jérôme Epinett 2017 in einem
mittlerweile schon fast Traditionsduft des Hauses in passende Duftnoten
zu übersetzen. Unlängst ergab sich die Gelegenheit, dies in einem
Redaktionsraum per Kerze auszuprobieren. Nach einem halben Arbeitstag
Beduftung lässt sich festhalten, dass es fantastisch wäre, wenn
Bibliotheken solch ein Aroma hätten, in der konkreten kerzenhaften
Umsetzung aber natürlich nichts im Duft an Bibliotheken erinnert. Dazu
ist er zu fruchtig. Das Veilchen in der Herznote sticht zumindest in der
Raumvariante jeden möglichen olfaktorischen Bezug zum (Einband)Leder.
Und auch das versprochene Birkenholz des Fonds drang kaum durch. Es ist
ein sehr schöner, durchaus prägnanter, fruchtig-komplexer Duft, der aber
die von der Bezeichnung intendierte Assoziation nicht findet. Mit
Bestnoten wurde das Thema verfehlt. (bk)

\begin{center}\rule{0.5\linewidth}{0.5pt}\end{center}

Publisher of Library Hi Tech News (2024). Retraction notice: Artificial
intelligence as enabler of future library services: how prepared are
librarians in African university libraries. In: \emph{Library Hi Tech
News}, 41 (2024) 3: 22. \url{https://doi.org/10.1108/LHTN-01-2024-0007}

Vielleicht ist das erste Fall dieser Art in der Bibliothekswissenschaft:
Ein Artikel über Artificial intelligence in Bibliotheken --
veröffentlicht im Oktober 2023 als ahead of print, also noch nicht ganz
\enquote{offiziell} -- wurde zurückgezogen, weil die Autor*innen
offenbar selber Artificial intelligence eingesetzt hatten, um zumindest
die Daten des Artikels zu generieren. (ks)

%autor

\end{document}