\documentclass[a4paper,
fontsize=11pt,
%headings=small,
oneside,
numbers=noperiodatend,
parskip=half-,
bibliography=totoc,
final
]{scrartcl}

\usepackage[babel]{csquotes}
\usepackage{synttree}
\usepackage{graphicx}
\setkeys{Gin}{width=.4\textwidth} %default pics size

\graphicspath{{./plots/}}
\usepackage[ngerman]{babel}
\usepackage[T1]{fontenc}
%\usepackage{amsmath}
\usepackage[utf8x]{inputenc}
\usepackage [hyphens]{url}
\usepackage{booktabs} 
\usepackage[left=2.4cm,right=2.4cm,top=2.3cm,bottom=2cm,includeheadfoot]{geometry}
\usepackage[labelformat=empty]{caption} % option 'labelformat=empty]' to surpress adding "Abbildung 1:" or "Figure 1" before each caption / use parameter '\captionsetup{labelformat=empty}' instead to change this for just one caption
\usepackage{eurosym}
\usepackage{multirow}
\usepackage[ngerman]{varioref}
\setcapindent{1em}
\renewcommand{\labelitemi}{--}
\usepackage{paralist}
\usepackage{pdfpages}
\usepackage{lscape}
\usepackage{float}
\usepackage{acronym}
\usepackage{eurosym}
\usepackage{longtable,lscape}
\usepackage{mathpazo}
\usepackage[normalem]{ulem} %emphasize weiterhin kursiv
\usepackage[flushmargin,ragged]{footmisc} % left align footnote
\usepackage{ccicons} 
\setcapindent{0pt} % no indentation in captions
\usepackage{xurl} % Breaks URLs

%%%% fancy LIBREAS URL color 
\usepackage{xcolor}
\definecolor{libreas}{RGB}{112,0,0}

\usepackage{listings}

\urlstyle{same}  % don't use monospace font for urls

\usepackage[fleqn]{amsmath}

%adjust fontsize for part

\usepackage{sectsty}
\partfont{\large}

%Das BibTeX-Zeichen mit \BibTeX setzen:
\def\symbol#1{\char #1\relax}
\def\bsl{{\tt\symbol{'134}}}
\def\BibTeX{{\rm B\kern-.05em{\sc i\kern-.025em b}\kern-.08em
    T\kern-.1667em\lower.7ex\hbox{E}\kern-.125emX}}

\usepackage{fancyhdr}
\fancyhf{}
\pagestyle{fancyplain}
\fancyhead[R]{\thepage}

% make sure bookmarks are created eventough sections are not numbered!
% uncommend if sections are numbered (bookmarks created by default)
\makeatletter
\renewcommand\@seccntformat[1]{}
\makeatother

% typo setup
\clubpenalty = 10000
\widowpenalty = 10000
\displaywidowpenalty = 10000

\usepackage{hyperxmp}
\usepackage[colorlinks, linkcolor=black,citecolor=black, urlcolor=libreas,
breaklinks= true,bookmarks=true,bookmarksopen=true]{hyperref}
\usepackage{breakurl}

%meta
%meta

\fancyhead[L]{D. Heß, S. Hein, K. Schuldt\\ %author
LIBREAS. Library Ideas, 45 (2024). % journal, issue, volume.
\href{https://doi.org/10.18452/...}{\color{black}https://doi.org/10.18452/...}
{}} % doi 
\fancyhead[R]{\thepage} %page number
\fancyfoot[L] {\ccLogo \ccAttribution\ \href{https://creativecommons.org/licenses/by/4.0/}{\color{black}Creative Commons BY 4.0}}  %licence
\fancyfoot[R] {ISSN: 1860-7950}

\title{\LARGE{Interview mit der Redaktion von \enquote{Forum Musikbibliothek}}}% title
\author{Dina Heß, Susanne Hein, Karsten Schuldt} % author

\setcounter{page}{1}

\hypersetup{%
      pdftitle={Interview mit der Redaktion von "Forum Musikbibliothek"},
      pdfauthor={Dina Heß, Susanne Hein, Karsten Schuldt},
      pdfcopyright={CC BY 4.0 International},
      pdfsubject={LIBREAS. Library Ideas, 45 (2024).},
      pdfkeywords={Interview, Musikbibliothek},
      pdflicenseurl={https://creativecommons.org/licenses/by/4.0/},
	  pdfurl={https://doi.org/10.18452/...},
	  pdfdoi={10.18452/...},
      pdflang={de},
      pdfmetalang={de}
     }



\date{}
\begin{document}

\maketitle
\thispagestyle{fancyplain} 

%abstracts

%body
Bibliotheken neigen bekanntlich dazu, sich in Arbeitsgruppen,
Kommissionen, Unterverbänden und so weiter zu organisieren. Sind diese
nur groß genug, dann -- so scheint es zumindest -- betreiben sie oft
auch eigene Publikationen. Manchmal eigene Blogs und Homepages, manchmal
Newsletter, aber manchmal auch eigene Zeitschriften. Musikbibliotheken
sind ein Beispiel dafür. Sie sind seit langem gut organisiert, inklusive
eines eigenen Weltverbandes (der International Association of Music
Libraries, Archives and Documentation Centres, IAML) mit eigenen
Ländergruppen, zum Beispiel in Deutschland, der Schweiz und Österreich,
die wiederum zum Teil eigene Arbeitsgemeinschaften haben. Für die
deutsche Bibliotheksstatistik stellen sie einen eigenen Bibliothekstyp
dar. Und mit \emph{Forum Musikbibliothek} existiert für sie seit 1980
eine kontinuierlich erscheinende Zeitschrift. Damit sind sie zwar nicht
der einzige Bibliothekstyp -- zum Beispiel haben die Museumsbibliotheken
mit den \emph{AKMB-news} auch eine eigene Zeitschrift -- aber doch einer
der wenigen.

Für den Schwerpunkt der \#45 der \emph{LIBREAS} erscheint es nur
passend, ein Interview quasi von Redaktion zu Redaktion über diese
Zeitschrift zu führen. Was macht sie aus -- auch im Vergleich zu breit
aufgestellten Zeitschriften wie der \emph{LIBREAS}? Gibt es spezifische
Herausforderungen bei der Redaktionsarbeit für Musikbibliotheken?
Welchen besonderen Blick bringen die Musikbibliotheken in das
Bibliothekswesen ein?

Im Folgenden also ein schriftlich geführtes Interview mit Dina Heß und
Susanne Hein aus der Schriftleitung von \emph{Forum Musikbibliothek}.
Die Fragen für die \emph{LIBREAS} stellte Karsten Schuldt.

\emph{Karsten Schuldt (KS): Werte Frau Hein, werte Frau Heß. Sie sind
beide, wenn ich richtig informiert bin, erst seit einigen Jahren in der
Schriftleitung von Forum Musikbibliothek tätig. Deshalb gleich die
Frage: Was hat Sie dazu getrieben, diese Verantwortung zu übernehmen?
Sie sind ja beide selber in Musikbibliotheken tätig -- ist das einfach
so, dass von den Kolleg*innen in den anderen Musikbibliotheken erwartet
wird, dass sie diese Aufgabe übernehmen, wenn sie \enquote{richtige
Musikbibliothekar*in} sein wollen, so als eine Form der rite de passage?
Oder welche Gründe hat es für Sie gegeben?}

Susanne Hein (SH): Zunächst eine kurze Information zur Organisation der
Zeitschrift: Wir beide gehören zur Schriftleitung, die die Artikel
einwirbt, inhaltlich verantwortet und auch grob die Linie der
Zeitschrift oder zumindest eines Heftes festlegt, teils auch in
Austausch mit dem Vorstand von IAML Deutschland. Auch über den Abdruck
unaufgefordert eingereichter Beiträge entscheiden wir. Unser Redakteur
dagegen korrigiert die Texte und später die Druckfahnen. Außerdem ist er
für den Rezensionsteil zuständig und übernimmt den Kontakt zur
Schriftsetzerin des Verlags.

Um die Frage nach der Motivation für dieses Amt zu beantworten:
Getrieben hat mich nichts und erwartet wurde es ebenfalls nicht. Ich
wurde überrascht, als die Präsidentin von IAML-Deutschland mich (und
einen mir bisher unbekannten Kollegen, Jonas Lamik) im Frühjahr 2020
fragte. Aber ich hatte Lust dazu, denn es macht Spaß, diese Hefte
gestalten zu können und die Ergebnisse zu sehen. Und ich bin sehr froh,
das Amt nicht alleine bedienen zu müssen, denn die meiste Arbeit fällt
in die Freizeit und es macht Freude, zu zweit in der Schriftleitung zu
sein. Zum Glück hilft uns außerdem ein Beirat mit Mitgliedern aus den
verschiedenen Musikbibliothekstypen sowie Vertretungen der
IAML-Ländergruppen Österreich und der Schweiz. Sie werben ebenfalls
Beiträge ein, vermitteln Rezensionen, übernehmen Interviews oder das
Schreiben des Editorials. Grundsätzlich sind unter den
Musikbibliothekar*innen sehr viele Kolleg*innen mit einer großen Portion
Idealismus. Die ist ebenfalls nötig für die vielen Ämter im Verband. Es
gibt nicht so viele Leute, die in Frage kommen -- mir hilft zum Beispiel
meine Berufserfahrung und die langjährige Verbandsarbeit, daher kenne
ich viele Menschen und bekomme oft mit, welche Projekte an welchen
Bibliotheken laufen.

Dina Heß (DH): Für mich ist das alles noch recht neu, weil ich erst seit
letztem Jahr die Schriftleitung mit Susanne zusammen mache. Vielleicht,
weil ich zwei Jahre in der Redaktion von \emph{o-bib. Das offene
Bibliotheksjournal} mitgearbeitet habe, haben mich Jonas Lamik und
Susanne Hein angesprochen, ob ich Interesse hätte, in der Schriftleitung
von \emph{Forum Musikbibliothek} mitzuarbeiten. Es hat mich natürlich
sehr gereizt, die ohnehin spannende Arbeit für ein Fachjournal nun in
meinem Fachgebiet nicht nur in unterstützender Rolle, sondern
mitgestaltend zu übernehmen.

Ich empfinde das nicht als notwendigen \emph{rite de passage}, aber
durch \emph{Forum Musikbibliothek} hatte ich schon nochmal einen
anderen, intensiveren Eintritt in die Fachcommunity. Ich bin sehr
glücklich darüber, mit Susanne eine so erfahrene Partnerin zu haben und
das Vorhaben \emph{Forum Musikbibliothek} auch gemeinsam mit unserem
Redakteur und dem Beirat zu \enquote{wuppen}.

\emph{KS: Zur Redaktionsarbeit: Musikbibliotheken liegen ja einigermaßen
\enquote{quer} zu den anderen Bibliothekstypen. Öffentliche
Bibliotheken, Wissenschaftliche Bibliotheken, Staats- und
Kantonsbibliotheken -- sie alle können Musikabteilungen haben, abgesehen
von ganz eigenständigen Musikbibliotheken. Stellt das ein Potential oder
eine Herausforderung für das Einwerben von Artikeln dar?}

SH: Natürlich ist das ein Potential -- einerseits gibt es Abwechslung
und Vielfalt, andererseits aber gibt es überall Gemeinsamkeiten quer
durch die Institutionen. Außer den von Ihnen genannten Typen sind in der
IAML Musikarchive und Musikdokumentationszentren vertreten.

DH: Als Musikhochschulbibliotheken sind wir eigentlich kein eigener
Bibliothekstyp, aber in der Praxis stellt unsere Arbeit sich oft als ein
sehr spannender Querschnitt aus Hochschulbibliothek, Öffentlicher
Musikbibliothek und Spezialbibliothek dar. Natürlich ist es immer
einfacher, in der eigenen Sparte Beiträge einzuwerben. Das ist auch ein
Grund, weshalb es sich für das Schriftleitungsduo bewährt hat, wenn es
mit Personen aus unterschiedlichen Richtungen innerhalb des
Musikbibliothekswesens besetzt ist.

\emph{KS: Grundsätzlich auch die Frage: Was unterscheidet
Musikbibliotheken eigentlich so sehr von anderen Bibliotheken, dass sie
ihre eigene Zeitschrift und ihren eigenen Verband betreiben? Wieder ist
das vielleicht mein spezifischer Blick, aber ich habe mich lange mit
Schulbibliotheken beschäftigt. Die hatten von 1975 bis 2000 auch ihre
eigene Zeitschrift, die damals wie Forum Musikbibliothek vom Deutschen
Bibliotheksinstitut herausgegeben wurde. Aber sie haben es nicht
geschafft, diese selbstständig weiterzuführen. Was macht dagegen
Musikbibliotheken so gut darin, solche Strukturen selber zu organisieren
und offenbar ja auch zu erhalten?}

SH: Wie gesagt, Musikbibliothekar*innen sind oft sehr motiviert. Unsere
Materie (vor allem in Bezug auf Noten und Tonträger) stellt auf sehr
vielen bibliothekarischen Gebieten andere beziehungsweise oft besondere
Anforderungen im Vergleich zu gängigen Medienarten an
Universalbibliotheken. Es geht um Katalogisierung und Recherche, um die
Marktsichtung und Erfahrung mit bestimmten Discoverysystemen,
Katalogisierungsprogrammen oder lizenzierbaren Streamingportalen bei
Noten und Musikaufnahmen. Ich vermute, dass es bei Schulbibliotheken
weniger Unterschiede zum allgemeinen Bibliothekswesen gibt oder auch,
dass die Unterschiede innerhalb der verschiedenen Schulbibliothekstypen
nicht so groß sind. Es tut gut und bringt uns fachlich weiter, sich dazu
auf Tagungen austauschen zu können. Auf jeden Fall sind wir dankbar,
dass die Träger der Musikbibliotheken diese Vernetzung unterstützen.
Vielleicht war das bei den Schulbibliotheken anders?

DH: In wissenschaftlichen und öffentlichen Musikbibliotheken haben wir
in der Regel alles, was andere Bibliotheken auch haben, aber eben
zusätzlich die Musik -- die sich in vielen Aspekten des
Bibliothekswesens etwas anders verhält. Das hat in großem Maße auch
damit zu tun, dass Noten anders genutzt werden. Wie ein literarisches
Werk kann auch ein musikalisches Werk natürlich Gegenstand von Forschung
sein. Aber im Fall von Noten wird eben auch daraus musiziert und das
wiederum in verschiedenen Kontexten: zum Beispiel privat als Hobby oder
professionell. Aus den verschiedenen Nutzungsarten ergeben sich
unterschiedliche Anforderungen an zum Beispiel Editionen oder
musikalische Ausgabeformen. Eine Sängerin, die eine Rolle aus Mozarts
Oper \enquote{Die Zauberflöte} oder Bernsteins \enquote{West Side Story}
einstudiert, braucht vielleicht den Klavierauszug, der Soloflötist eine
Flötenstimme, die Dirigentin die kritische Edition der Partitur und der
Regisseur vielleicht noch zusätzlich ein Libretto und wissenschaftliche
Literatur zum Werk. Das sind andere Anforderungen als die eines
Oberstufenschülers, der einen Aufsatz zu Mozarts Opern schreiben soll,
einer Forscherin, die sich mit Geschlechterrollen im Musiktheater
zwischen 1760 und 1960 beschäftigt oder einem Amateurmusiker, der auf
der Hochzeit seiner Tochter eine Arie aus der Zauberflöte oder der West
Side Story singen möchte.

\emph{KS: Für diese Ausgabe der LIBREAS habe ich mich mit der Geschichte
der öffentlichen Musikbibliotheken am Anfang des 20. Jahrhunderts
befasst. Damals erschienen Artikel zu musikbibliothekarischen Themen in
den breiter aufgestellten bibliothekarischen Publikationen,
beispielsweise in den Vorgängerinnen der heutigen BuB. Heute habe ich
den Eindruck, dass das fast nicht mehr der Fall ist. Würden Sie diese
Einschätzung teilen? Ist es halt nicht auch eine Gefahr, wenn so eine
Zeitschrift wie die Ihre besteht, dass dann die Musikbibliotheken im
gesamten Bibliothekswesen einigermassen \enquote{vergessen} gehen, weil
sie halt nicht mehr in der BuB oder so auftauchen?}

SH: Nein, das ergänzt sich alles ganz gut. In \emph{BuB} erscheinen
durchaus musikbibliothekarische Artikel, auch gelegentlich im
\emph{Bibliotheksdienst.} Außerdem erinnere ich mich an ein Sonderheft
Musikbibliotheken der \emph{ZfBB}. Teilweise stimmen wir das mit den
Autor*innen ab und haben sogar schon die Veröffentlichung einzelner
Beiträge in \emph{BuB} statt im \emph{Forum Musikbibliothek}
unterstützt, und zwar bei den Artikeln von Cortina Wuthe und Verena
Funtenberger (Heft 12/22), damit sie ein größeres Publikum bekamen.
Gerade ist wieder ein Sonderheft zum Thema Musikbibliotheken in
\emph{BuB} erschienen (Heft 2-3/2024, das letzte war 04/2016); darüber
wurden wir lange vorher aus der Musikbibliothekscommunity informiert.
Ein Problem ist vielleicht eher die musikbibliothekarische Präsenz auf
der und die Teilnahme an der \emph{BiblioCon}. Gelegentlich gibt es auch
dort musikbibliothekarische Inhalte und Musikbibliothekar*innen
profitieren bei der Tagungsteilnahme auf jeden Fall von
allgemein-bibliothekarischen Vorträgen, aber es ist für die einzelnen
Musikbibliotheks-Kolleg*innen schwierig, alle Termine zu berücksichtigen
beziehungsweise selbst mit Vorträgen präsent zu sein.

\emph{KS: Dazu eine Frage, die nichts direkt mit Ihrer Redaktionsarbeit
zu tun hat, aber doch bestimmt noch mehr Menschen als mich brennend
interessiert: Wer sind eigentlich die Musikbibliothekar*innen? Wie
verstehen die sich? Also: Würden die sich vor allem als Musiker*innen
beschreiben, die auch Bibliothekar*innen sind? Oder Bibliothekar*innen,
die an Musik interessiert sind? Ich frage das auch, weil mir scheint,
alle Musikbibliothekar*innen, die ich so treffe, spielen aktiv irgendein
Instrument. Das ist ja doch etwas anderes als in anderen Bibliotheken,
wo Kolleg*innen gerne einmal herausstellen, dass sie nicht vor allem
lesen, sondern aktiv Bestände managen. Da scheinen
Musikbibliothekar*innen viel näher an \enquote{ihrem Thema} dran zu
sein. Oder gibt es eine eigene Identität als Musikbibliothekar*in?}

SH: Es gibt sehr verschiedene Qualifikationen und Stellenprofile, was
unter anderem von den Musikbibliothekstypen und der Größe der jeweiligen
Bibliothek abhängt. In erster Linie kommen wir aus dem Bibliothekswesen
und der Musikwissenschaft. Lange Zeit gab es in Stuttgart an der
heutigen \emph{Hochschule der Medien} ein halbjähriges
musikbibliothekarisches Zusatzstudium im Anschluss an das
bibliothekarische Diplom-Examen und später einen Schwerpunkt
Musikinformationsmanagement im bibliothekarischen Masterstudiengang.
Aktuell bietet die \emph{Hochschule für Technik, Wirtschaft und Kultur
Leipzig} einen Masterstudiengang Bibliotheks- und
Informationswissenschaft mit der Profillinie Musikbibliotheken an. Viele
von uns haben mehrere Abschlüsse. Fast alle spielen ein Instrument oder
singen aktiv; ohne Notenkenntnisse und tieferes Verständnis für Musik
kann der Beruf nicht gut ausgeübt werden. Musiker*innen ohne
bibliothekarischen Abschluss sind mir auf bibliothekarischen Stellen
kaum bekannt. In den öffentlichen Musikbibliotheken werden dagegen
neuerdings Musikpädagog*innen eingestellt, dann aber mit dem Fokus auf
Programmarbeit, Workshops und Musikvermittlung. Die Bandbreite wird
sicher schon anhand unserer verschiedenen Antworten klar, da wir beide
aus unterschiedlichen Bibliothekstypen kommen. Eine eigene Identität ist
sicher übertrieben, aber die musikbibliothekarischen Gemeinsamkeiten
schweißen uns schon sehr zusammen.

DH: Unter den Orchesterbibliothekar*innen gibt es vereinzelt
Musiker*innen oder Musikwissenschaftler*innen ohne formale
bibliothekarische Ausbildung oder, in kleineren Häusern, auch andere
Quereinsteiger*innen. Häufig arbeitet man da aber auch auf eine ganz
andere Art mit den Musiker*innen des eigenen Orchesters und insbesondere
mit den Dirigent*innen zusammen und richtet beispielsweise die Stimmen
für eine Produktion ein. Das unterscheidet sich nochmal viel stärker von
der \enquote{klassischen} Bibliotheksarbeit, als vielleicht in unseren
beiden Sparten.\\
Natürlich arbeiten -- je nach Größe -- auch in öffentlichen und
wissenschaftlichen Musikbibliotheken nicht nur Menschen, die sowohl
musikalische als auch bibliothekarische Abschlüsse mitbringen. Ich halte
es für Musikbibliothekar*innen hier aber für sehr wichtig, sich nicht
nur in der Bibliothekswissenschaft auszukennen, sondern zusätzlich eben
auch, bis zu einem gewissen Grad Noten lesen zu können und das Vokabular
und die Konzepte der Musik und ihrer Notation zu kennen: Schlüssel,
Tonarten, Instrumente, Besetzungen, Epochen, Gattungen und Genres oder
die schon erwähnten musikalischen Ausgabeformen sind allein für die
Notensuche im Katalog fast unerlässlich; ebenso wichtig ist es, zu
wissen, dass eine Rockband anders musiziert als ein philharmonisches
Orchester, eine Opernsängerin anders als eine Popsängerin und ein
Konzertpianist anders als ein Jazztrompeter und dass sie infolgedessen
andere Bedarfe an eine Bibliothek haben und auch anders suchen.

\emph{KS: Etwas, was im Forum Musikbibliothek gleich auffällt, ist die
Rubrik \enquote{Personalia}, die ja immer wieder relativ viel Platz
einnimmt. Das vermittelt für mich als Außenstehenden schon den Eindruck
einer eng zusammenhängenden Community. Stimmt das? Wie wichtig ist denn
diese Rubrik? Gibt es in den Musikbibliotheken wirklich das Verlangen
danach, zu wissen, wer von welcher Musikbibliothek in eine andere
Musikbibliothek gewechselt ist oder wer auf welchen Posten aufgestiegen
ist? Ich frage auch, weil mir scheint, dass es solche Nachrichten früher
viel öfter in bibliothekarischen Zeitschriften gab, aber heute fast nur
noch in Ihrer Zeitschrift.}

SH: Diese Rubrik wurde erst 2012 eingeführt und nimmt meist nur wenig
Platz ein; im Heft 3/2023 fehlte sie sogar ganz. Bei einer Umfrage,
welche Rubriken unsere Leser*innen bevorzugen (Heft 3/2020) schnitt die
Rubrik Personalia zwar weniger gut ab, aber der Abstand zu den anderen
Rubriken war eher gering. Ja, wir sind eine übersichtliche und gut
vernetzte Community. Da interessiert es schon zu erfahren, welche Posten
mit welchen Personen besetzt werden oder es lohnt sich zu bilanzieren,
was wir einzelnen Kolleg*innen verdanken, die in den Ruhestand gehen.
Vermutlich würden sich auch in anderen Bibliothekstypen Kolleg*innen für
solche Artikel interessieren und es scheitert nur am Aufwand und der
Beitragsmenge?

DH: Wir arbeiten tatsächlich äußerst eng zusammen und tauschen uns
besonders innerhalb unser jeweiligen Sparten sehr viel aus -- zumindest
empfinde ich das so. Die große Bibliothekscommunity spielt dabei eine
viel kleinere Rolle. Das hängt mit den bereits angesprochenen
Spezialisierungen zusammen und auch mit den besonderen Bedingungen. In
Hochschulen ist das beispielsweise zusätzlich auch die Größe. Unsere
Musikhochschulbibliotheken haben grundsätzlich ja dieselben Aufgaben wie
eine Universitätsbibliothek, aber anstelle von 16.000 oder gar 30.000
Studierenden haben unsere Hochschulen häufig 400 bis maximal 1.700
Studierende. Gleichzeitig ist auch die Musikwelt -- also unsere
Nutzer*innen -- in einem viel höheren Maße vernetzt. In einzelnen
Disziplinen kennt jede*r buchstäblich jede*n.

\emph{KS: Was mir weiterhin in Ihrer Zeitschrift auffällt, sind die
recht vielen historischen Themen, also die Darstellungen zu
Musiker*innen und Komponist*innen, aber auch die vielen Projektberichte
zu Nachlässen. Haben Musikbibliotheken einen historischen Fokus? Gibt es
da nicht auch Widerspruch aus der Profession dagegen? Musikbibliotheken
wollen doch nicht nur historische Musik vermitteln.}

SH: Institutionelle IAML-Mitglieder sind überwiegend wissenschaftliche
Musikbibliotheken mit großen historischen Beständen und einem
entsprechenden Auftrag. Diese Bestände machen viel Arbeit in der
Erwerbung, Erschließung und Digitalisierung. Daher gibt es oft spezielle
Projekte und berichtenswerte Resultate. Es ist für alle interessant zu
erfahren, wo welche Bestände erworben und erschlossen wurden.
Gegenfrage: Was meinen Sie mit \enquote{historischer Musik}? Nahezu alle
Musikbibliotheken erwerben Partituren und Tonträger mit zeitgenössischer
klassischer Musik. Bei Jazz, Rock und Pop erscheinen im Verhältnis zu
Tonträgern nur wenig gedruckte Noten, diese gehören zum Profil vor allem
der Musikabteilungen an öffentlichen Bibliotheken und
Musikhochschulbibliotheken. Institutionen mit großen Tonträgersammlungen
schließen sich der IASA an, nicht der IAML, aber es gibt Schnittmengen.

DH: Unsere Bestände und auch die Vermittlung unserer Bestände sind so
vielfältig wie die Musik selbst und ein Musikstück ist -- egal wie alt
es ist -- immer auch in gewisser Weise \enquote{gegenwärtig}, wenn es
von jemandem musiziert wird. Manche Werke sind auch Jahrzehnte oder
Jahrhunderte nach ihrer Entstehung unverändert beliebt, manche werden
erst lange nach ihrem Entstehen durch die Musikwelt
\enquote{wiederentdeckt}. Das betrifft natürlich nicht nur die
klassische Musik, sondern auch Jazz, Musical oder die Popmusik des 20.
Jahrhunderts.\\
Die Musik selbst mag vor 400 oder 80 Jahren geschrieben worden sein,
wird aber heute noch -- und das fast täglich -- gespielt, erforscht,
gehört, bearbeitet und so weiter. Ich bezweifle, dass das bei
literarischen Werken in diesem Maße passiert -- am ehesten vielleicht
bei den Dramen von William Shakespeare oder den Romanen von Jane Austen.

Nehmen wir das Violinkonzert op. 35 von Pjotr Iljitsch Tschaikowski, das
1878 entstand. Ein Konzertmitschnitt vom September 2023, der auf
\emph{YouTube} verfügbar ist, hat jetzt, ein halbes Jahr später, über
eine Million Streams. Ich glaube nicht, dass so viele Menschen im
Zeitraum von einem halben Jahr 150 Jahre nach seiner Veröffentlichung
1878 den Roman \enquote{Vor dem Sturm} von Theodor Fontane rezipieren --
oder dass in Bibliotheken so häufig nach dem Roman gefragt wird, wie es
Bedarf nach dem Notenmaterial für das Violinkonzert gibt.

\emph{KS: Die Frage eben interessiert mich auch, weil mir ebenso
aufgefallen ist, dass doch recht viele, sagen wir einmal, digitale
Themen im Forum Musikbibliothek Platz finden. Einerseits viele Projekte
zur Digitalisierung von Nachlässen und Sammlungen. Andererseits aber
gerade auch aus Öffentlichen Bibliotheken Beiträge zu Makerspaces und
kuratierten Tracklists. Nehme ich das einfach falsch wahr? Ist das nicht
ein gewisser Widerspruch zu den vielen historischen Themen? Oder bildet
das einfach die Breite der Musikbibliotheken ab?}

SH: Warum sollte das ein Widerspruch sein? Auch die Digitalisierung ist
ein Thema, an der sich Musikbibliotheken (und wenige weitere
Spezialbibliotheken) mehr abarbeiten müssen als die allgemeinen
Bibliotheken, aber das ist doch eminent wichtig, wenn wir nicht als
Fossilien enden wollen?!

DH: Ich sehe da auch überhaupt keinen Widerspruch. Wir bewegen uns in
der Musik immer im \enquote{Zeitlosen} oder zeitübergreifend und die
Digitalisierung von Beständen hilft, die Brücke zu schlagen. Gute
Digitalisierung eröffnet zudem neue Arten, mit Musik zu arbeiten, aber
auch Musik zu erleben. Gleichzeitig ändert sich der praktische Umgang
mit Noten auch im Konzertbetrieb. Eine E-Note muss -- genauso wie ihr
gedrucktes Pendant -- viel mehr Anwendungskontexte erlauben als ein
E-Book. Gleichzeitig beschäftigt sich auch die Musikwissenschaft als
eine der Geisteswissenschaften zunehmend mit Themen der Digital
Humanities.\\
Der Musikmarkt hat sich ebenso grundlegend geändert wie der Filmmarkt,
vom physischen Medium hin zum Streaming. Das geht für Bibliotheken zum
Beispiel mit neuen Lizenzmöglichkeiten einher.

\emph{KS: Wenn wir schon bei den Themen im Forum Musikbibliothek sind:
Welche Themen würden Sie sich denn in Zukunft mehr wünschen? Sowohl als
Schriftleitung als auch als aktive Musikbibliothekar*innen?}

SH: Welche Themen wir uns beziehungsweise unsere Leser*innen sich mehr
wünschen, ist sonnenklar: Praxisberichte zu Managementfragen und
Erfahrungen mit verschiedenen Software-Anwendungen, Datenbanken oder
Streamingportalen. Außerdem Erfahrungen mit Erwerbungsmodellen, bei
denen bestimmte Tätigkeiten aus Lektorat, Katalogisierung und
ausleihfertiger Bearbeitung an Firmen aus dem Buch- und Musikalienhandel
vergeben werden. Es gibt aber nicht genug Kolleg*innen, die darüber
schreiben möchten, unter anderem weil es heikel ist beziehungsweise zu
viel Interna tangiert.

DH: Da kann ich nur zustimmen: Was vielen von uns am meisten hilft, sind
Erfahrungen aus der Praxis und Überlegungen, wie sich unser Angebot in
diesem Jahrzehnt und darüber hinaus entwickeln könnte. Mich persönlich
treibt aber aktuell auch vermehrt die Frage um, was die Bedarfe unserer
Nutzenden in ein paar Jahren sein werden -- wohin sich Musikpraxis und
\mbox{-forschung} entwickeln und welche Rolle Bibliotheken dabei spielen
werden.

\emph{KS: Okay. Eine Frage habe ich mir für das Ende aufgehoben, weil
sie vielleicht in ein Wespennest stößt. Deshalb lassen Sie mich kurz
vorher sagen, dass ich persönlich ein großer Freund von physischen
Medien bin -- ich lese, was ich kann, gedruckt; ich höre, was ich kann,
direkt von Vinyl. Aber auch, weil die LIBREAS so wie viele andere
bibliothekarische Zeitschriften einen anderen Weg gehen: Warum erscheint
Forum Musikbibliothek eigentlich weiterhin gedruckt? Ich weiss, nach
zwölf Monaten publizieren Sie alle Artikel als Open Access. Aber dennoch
ist der Hauptpublikationsweg ja weiterhin die gedruckte Ausgabe. Ist das
von den Musikbibliotheken so gewollt? Wird das auch in Zukunft so sein?}

SH \& DH: Momentan stellen wir uns diese Frage nicht -- eine Mehrheit
möchte noch das gedruckte Exemplar und mit der erwähnten OA-Strategie
sind wir zufrieden. Es wäre außerdem eher eine Entscheidung des
IAML-Vorstands beziehungsweise der Mitgliederversammlung, denn
\emph{Forum Musikbibliothek} ist (auch) eine Verbandszeitschrift.

\emph{KS: Frau Heß, Frau Hein, haben Sie vielen Dank für Ihre
Antworten.}

%autor
\begin{center}\rule{0.5\linewidth}{0.5pt}\end{center}

\textbf{Interviewpartnerinnen}

Dina Heß

Dina Heß leitet seit 2021 die Bibliothek der Folkwang Universität der Künste. 
Sie hat Mathematik mit Nebenfach Musikwissenschaft an der Universität Bonn und 
der University of Leeds studiert, außerdem Bibliothekswissenschaft an der TH Köln. 
Aktuell ergänzt sie ihr Studium an der TH Köln mit einem MALIS-Abschluss. Darüber 
hinaus ist sie Sprecherin der Arbeitsgemeinschaft der Kunst- und Musikhochschulbibliotheken 
in Nordrhein-Westfalen.

Susanne Hein

Susanne Hein, Musikwissenschaftlerin, M.A. und Bibliothekswissenschaftlerin, MALIS, 
leitet die Musikbibliothek der Zentral- und Landesbibliothek Berlin. Von 2002 bis 2013 
war sie daneben Lehrbeauftragte im MALIS-Fernstudiengang am Institut für Bibliotheks- und Informationswissenschaft
der HU Berlin, von 2003 bis 2009 Präsidentin von IAML-Deutschland. Sie ist Autorin des Tutorials 
\enquote{Musikrecherche} auf der Website des Deutschen Musikinformationszentrums Bonn (MIZ).

Susanne Hein und Dina Heß sind außerdem passionierte Chorsängerinnen. Die Schriftleitung von Forum 
Musikbibliothek bestellen sie seit 2020 bzw. 2023.

\textbf{Interviewer}

Karsten Schuldt
 
Dr. Karsten Schuldt, wissenschaftlicher Projektleiter am Schweizerischen Institut für Informationswissenschaft,
Fachhochschule Graubünden (Chur) und Redakteur der LIBREAS. Library Ideas.

\end{document}