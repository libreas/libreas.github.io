\textbf{Abstract:} The article introduces the concept of library music.
We document different terms and use cases as well as the diversity of
genres, pseudonyms and composers. Metadata is of special importance to
make this utilitarian music findable and increase its reuse. The section
on economics discusses marketing strategies, rights management and
experiments.

\begin{center}\rule{0.5\linewidth}{0.5pt}\end{center}

\textbf{Kurzfassung:} Der Artikel führt das Konzept Library Music ein.
Wir dokumentieren verschiedene Nutzungsbedingungen, Nutzungsszenarien
und die Verschiedenartigkeit der abgebildeten Genres, der Pseudonyme und
Diversität der Komponist*innen. Der Abschnitt zu Metadaten arbeitet
deren Relevanz für die Auffindbarkeit und Nachnutzung vorproduzierter
Musik heraus. Die wirtschaftlichen Aspekte werden anhand von
Marketingstrategien, Rechtemanagement und Experimenten besprochen.
