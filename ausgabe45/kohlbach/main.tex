\documentclass[a4paper,
fontsize=11pt,
%headings=small,
oneside,
numbers=noperiodatend,
parskip=half-,
bibliography=totoc,
final
]{scrartcl}

\usepackage[babel]{csquotes}
\usepackage{synttree}
\usepackage{graphicx}
\setkeys{Gin}{width=.4\textwidth} %default pics size

\graphicspath{{./plots/}}
\usepackage[ngerman]{babel}
\usepackage[T1]{fontenc}
%\usepackage{amsmath}
\usepackage[utf8x]{inputenc}
\usepackage [hyphens]{url}
\usepackage{booktabs} 
\usepackage[left=2.4cm,right=2.4cm,top=2.3cm,bottom=2cm,includeheadfoot]{geometry}
\usepackage[labelformat=empty]{caption} % option 'labelformat=empty]' to surpress adding "Abbildung 1:" or "Figure 1" before each caption / use parameter '\captionsetup{labelformat=empty}' instead to change this for just one caption
\usepackage{eurosym}
\usepackage{multirow}
\usepackage[ngerman]{varioref}
\setcapindent{1em}
\renewcommand{\labelitemi}{--}
\usepackage{paralist}
\usepackage{pdfpages}
\usepackage{lscape}
\usepackage{float}
\usepackage{acronym}
\usepackage{eurosym}
\usepackage{longtable,lscape}
\usepackage{mathpazo}
\usepackage[normalem]{ulem} %emphasize weiterhin kursiv
\usepackage[flushmargin,ragged]{footmisc} % left align footnote
\usepackage{ccicons} 
\setcapindent{0pt} % no indentation in captions
\usepackage{xurl} % Breaks URLs

%%%% fancy LIBREAS URL color 
\usepackage{xcolor}
\definecolor{libreas}{RGB}{112,0,0}

\usepackage{listings}

\urlstyle{same}  % don't use monospace font for urls

\usepackage[fleqn]{amsmath}

%adjust fontsize for part

\usepackage{sectsty}
\partfont{\large}

%Das BibTeX-Zeichen mit \BibTeX setzen:
\def\symbol#1{\char #1\relax}
\def\bsl{{\tt\symbol{'134}}}
\def\BibTeX{{\rm B\kern-.05em{\sc i\kern-.025em b}\kern-.08em
    T\kern-.1667em\lower.7ex\hbox{E}\kern-.125emX}}

\usepackage{fancyhdr}
\fancyhf{}
\pagestyle{fancyplain}
\fancyhead[R]{\thepage}

% make sure bookmarks are created eventough sections are not numbered!
% uncommend if sections are numbered (bookmarks created by default)
\makeatletter
\renewcommand\@seccntformat[1]{}
\makeatother

% typo setup
\clubpenalty = 10000
\widowpenalty = 10000
\displaywidowpenalty = 10000

\usepackage{hyperxmp}
\usepackage[colorlinks, linkcolor=black,citecolor=black, urlcolor=libreas,
breaklinks= true,bookmarks=true,bookmarksopen=true]{hyperref}
\usepackage{breakurl}

%meta
%meta

\fancyhead[L]{J. Kohlbach\\ %author
LIBREAS. Library Ideas, 45 (2024). % journal, issue, volume.
\href{https://doi.org/10.18452/29143}{\color{black}https://doi.org/10.18452/29143}
{}} % doi 
\fancyhead[R]{\thepage} %page number
\fancyfoot[L] {\ccLogo \ccAttribution\ \href{https://creativecommons.org/licenses/by/4.0/}{\color{black}Creative Commons BY 4.0}}  %licence
\fancyfoot[R] {ISSN: 1860-7950}

\title{\LARGE{Von wegen ein Stilles Örtchen}}% title
\author{Julia Kohlbach} % author

\setcounter{page}{1}

\hypersetup{%
      pdftitle={Von wegen ein Stilles Örtchen},
      pdfauthor={Julia Kohlbach},
      pdfcopyright={CC BY 4.0 International},
      pdfsubject={LIBREAS. Library Ideas, 45 (2024).},
      pdfkeywords={Essay, Wissenschaftliche Bibliothek, Alltag, Lautstärke},
      pdflicenseurl={https://creativecommons.org/licenses/by/4.0/},
	  pdfurl={https://doi.org/10.18452/29143},
	  pdfdoi={10.18452/29143},
      pdflang={de},
      pdfmetalang={de}
     }



\date{}
\begin{document}

\maketitle
\thispagestyle{fancyplain} 

%abstracts

%body
Es ist nur noch eine Woche bis zur Abgabe meiner Hausarbeit Zeit, da
muss ich heute endlich etwas zu Papier bringen. Ich setze mich an meinen
Schreibtisch und beginne die ersten Informationen in Sätze zu verpacken.
Doch meine WG-Mitbewohner sind mal wieder so laut, dass ich mich nicht
auf das Schreiben konzentrieren kann. Wütend packe ich die Bücher,
meinen Laptop und eine Wasserflasche zusammen und verlasse die Wohnung.

Schnellen Schrittes gehe ich zur Straßenbahnhaltestelle und fahre
Richtung Universitätsbibliothek -- da ist es wenigstens leise. Dort
angekommen schließe ich meinen Rucksack in einem Spind ein, nehme meine
Lernmaterialien und suche mir einen gemütlichen Arbeitsplatz.

Ich bleibe direkt im Erdgeschoss und richte mich an einem Tisch am
Fenster ein, mit Blick auf den Bibliotheksgarten, in welchem gerade die
Kirschbäume blühen. Ich lasse keine Zeit verstreichen -- öffne meinen
Laptop und schreibe an meiner Hausarbeit weiter. Doch lange kann ich
mich nicht konzentrieren -- ständig erfüllt ein lästiges Piepen den
ganzen Bibliotheksraum. Ich blicke mich um und sehe, dass an der
Ausleihe eine Schar von Nutzenden steht. Der Barcodescanner kommt kaum
zur Ruhe -- die Geräuschkulisse erinnert eher an einen Supermarkt;
keinesfalls an einen stillen Ort zum Lesen und Arbeiten. So kann ich auf
keinen Fall meine Hausarbeit schreiben. Also packe ich meine Sachen
wieder zusammen und gehe eine Etage weiter nach oben.

Dort suche ich mir wieder einen Platz am Fenster, aber diesmal ohne
Aussicht ins Grüne, sondern auf ein neu entstehendes Hochhaus. Und
wieder öffne ich meinen Laptop und arbeite weiter. Nach gut einer Stunde
trifft schallendes Geschirrgeklapper auf meine Ohren -- direkt unter mir
ist die Cafeteria, wo gerade der Mittagsbetrieb anläuft. Das darf doch
alles nicht wahr sein. Was ist aus dem stillen Ort Bibliothek nur
geworden? Um nicht gleich wieder das Weite zu suchen, beschließe ich
auch eine kurze Pause zu machen, ein Brötchen zu essen und frische Luft
zu schnappen. Nach einer Stunde wird es leiser und ich gehe an meinen
Arbeitsplatz zurück, wo ich mich ins Schreiben meines nächsten Kapitels
vertiefe. Doch nach kurzer Zeit ertönt erneut das Stapeln von Tassen und
Tellern. Jetzt reicht es mir -- ich packe meine Sachen zusammen und
beschließe in die zweite Etage zu gehen.

In der Abteilung der Sprachwissenschaften kann ich leider keinen
Fensterplatz finden und beziehe einen Sitzplatz direkt im
Aufgangsbereich. Doch irgendwie komme ich nicht mehr in meinen
Schreibfluss zurück, schiebe meinen Laptop beiseite und wälze zunächst
umherliegende Fachliteratur. Aber für das Lesen der schwierigen
Textpassagen fehlt mir sämtliche Konzentration. Ständig laufen
Studierende an mir vorbei. Die Toilettentür fällt ins Schloss und wird
direkt wieder geöffnet. Der Fahrstuhl fährt hoch und runter.
Konzentration ist da Fehlanzeige, meine Gedanken schweifen zu sämtlichen
Geräuschen ab. Und plötzlich wird mir bewusst, wie laut eine Bibliothek
doch ist. Von wegen es ist ein stiller und leiser Ort -- es müssen nur
die Ohren gespitzt werden und man merkt, wie viele Geräusche auch in
einem so ehrwürdigen Gebäude vorherrschen.

Noch ein letztes Mal wechsele ich den Platz -- packe meine Sachen also
wieder zusammen und stiefele in die oberste Etage, wo nur wenige
Arbeitsplätze zur Verfügung stehen. Aber ich kann noch ein lauschiges
Plätzchen ergattern und lasse mich nieder. Und endlich finde ich hier
auch die notwendige Ruhe -- keine piependen Scanner, kein klingelndes
Telefon, kein Geklapper und Gemurmel aus der Cafeteria und keine
stumpfen Schritte von Nutzenden. Meine Finger sprinten nur so über die
Tastatur und ich schreibe Seite für Seite. Als ich am vorletzten Absatz
angekommen bin, beschließe ich eine kurze Pause zu machen, lehne mich
zurück und atme mit einem tiefen Seufzer aus. Dieser erklingt im nun
doch stillen Raum so laut, dass ich mir erschrocken die Hand vor den
Mund halte. Der Student neben mir schaut auf und wir beginnen beide zu
lächeln.

Als plötzlich das Plätschern von Regentropfen auf dem Glasdach der
Bibliothek zu hören ist, können wir uns beide das Lachen über einen
erneuten Bibliothekssound nicht verkneifen. Schmunzelnd klappen wir die
Laptops zu, packen unsere Sachen zusammen und verlassen gemeinsam die
Universitätsbibliothek. Einen Ort, dem so viel Ruhe nachgetragen wird
und doch unzählige Geräusche im Inneren bereit hält.

%autor
\begin{center}\rule{0.5\linewidth}{0.5pt}\end{center}

\textbf{Julia Kohlbach} studierte Bibliotheks- und
Informationswissenschaft an der HTWK Leipzig. Seit 2018 arbeitet sie an
der ThULB Jena.

\end{document}