\documentclass[a4paper,
fontsize=11pt,
%headings=small,
oneside,
numbers=noperiodatend,
parskip=half-,
bibliography=totoc,
final
]{scrartcl}

\usepackage[babel]{csquotes}
\usepackage{synttree}
\usepackage{graphicx}
\setkeys{Gin}{width=.4\textwidth} %default pics size

\graphicspath{{./plots/}}
\usepackage[ngerman]{babel}
\usepackage[T1]{fontenc}
%\usepackage{amsmath}
\usepackage[utf8x]{inputenc}
\usepackage [hyphens]{url}
\usepackage{booktabs} 
\usepackage[left=2.4cm,right=2.4cm,top=2.3cm,bottom=2cm,includeheadfoot]{geometry}
\usepackage[labelformat=empty]{caption} % option 'labelformat=empty]' to surpress adding "Abbildung 1:" or "Figure 1" before each caption / use parameter '\captionsetup{labelformat=empty}' instead to change this for just one caption
\usepackage{eurosym}
\usepackage{multirow}
\usepackage[ngerman]{varioref}
\setcapindent{1em}
\renewcommand{\labelitemi}{--}
\usepackage{paralist}
\usepackage{pdfpages}
\usepackage{lscape}
\usepackage{float}
\usepackage{acronym}
\usepackage{eurosym}
\usepackage{longtable,lscape}
\usepackage{mathpazo}
\usepackage[normalem]{ulem} %emphasize weiterhin kursiv
\usepackage[flushmargin,ragged]{footmisc} % left align footnote
\usepackage{ccicons} 
\setcapindent{0pt} % no indentation in captions
\usepackage{xurl} % Breaks URLs
\usepackage{makecell}
\renewcommand{\arraystretch}{1.5}

%%%% fancy LIBREAS URL color 
\usepackage{xcolor}
\definecolor{libreas}{RGB}{112,0,0}

\usepackage{listings}

\urlstyle{same}  % don't use monospace font for urls

\usepackage[fleqn]{amsmath}

%adjust fontsize for part

\usepackage{sectsty}
\partfont{\large}

%Das BibTeX-Zeichen mit \BibTeX setzen:
\def\symbol#1{\char #1\relax}
\def\bsl{{\tt\symbol{'134}}}
\def\BibTeX{{\rm B\kern-.05em{\sc i\kern-.025em b}\kern-.08em
    T\kern-.1667em\lower.7ex\hbox{E}\kern-.125emX}}

\usepackage{fancyhdr}
\fancyhf{}
\pagestyle{fancyplain}
\fancyhead[R]{\thepage}

% make sure bookmarks are created eventough sections are not numbered!
% uncommend if sections are numbered (bookmarks created by default)
\makeatletter
\renewcommand\@seccntformat[1]{}
\makeatother

% typo setup
\clubpenalty = 10000
\widowpenalty = 10000
\displaywidowpenalty = 10000

\usepackage{hyperxmp}
\usepackage[colorlinks, linkcolor=black,citecolor=black, urlcolor=libreas,
breaklinks= true,bookmarks=true,bookmarksopen=true]{hyperref}
\usepackage{breakurl}

%meta
%meta

\fancyhead[L]{C. Frick et al.\\ %author
LIBREAS. Library Ideas, 45 (2024). % journal, issue, volume.
\href{https://doi.org/10.18452/...}{\color{black}https://doi.org/10.18452/...}
{}} % doi 
\fancyhead[R]{\thepage} %page number
\fancyfoot[L] {\ccLogo \ccAttribution\ \href{https://creativecommons.org/licenses/by/4.0/}{\color{black}Creative Commons BY 4.0}}  %licence
\fancyfoot[R] {ISSN: 1860-7950}

\title{\LARGE{Queer sein ist (bibliotheks)politisch}}% title
\author{Claudia Frick, Philipp Zeuner, Caleb Buchert,\\ Daniela Markus, Norma Fötsch, Yvonne Fischer,\\ Emma Wieseler, Sabrina Ramünke, Nik Baumann} % author

\setcounter{page}{1}

\hypersetup{%
      pdftitle={Queer sein ist (bibliotheks)politisch},
      pdfauthor={Claudia Frick, Philipp Zeuner, Caleb Buchert, Daniela Markus, Norma Fötsch, Yvonne Fischer, Emma Wieseler, Sabrina Ramünke, Nik Baumann},
      pdfcopyright={CC BY 4.0 International},
      pdfsubject={LIBREAS. Library Ideas, 45 (2024).},
      pdfkeywords={LGBTQIA+, queer, Safe Space, Diversität, Bestand,  Nutzende, Kollegium, Antidiskriminierung, Netzwerk},
      pdflicenseurl={https://creativecommons.org/licenses/by/4.0/},
      pdfurl={https://doi.org/10.18452/...},
      pdfdoi={10.18452/...},
      pdflang={de},
      pdfmetalang={de}
     }



\date{}
\begin{document}

\maketitle
\thispagestyle{fancyplain} 

%abstracts
\begin{abstract}
\noindent
\textbf{Kurzfassung}: Seit November 2023 gibt es die \textit{Queerbrarians}, ein 
deutschsprachiges Netzwerk queerer Bibliotheksmenschen. Anhand der beim 
ersten Treffen zusammengetragenen Themen legt dieser Beitrag neben den Beweggründen 
für die Entstehung des Netzwerks auch dessen Ideen und Vorstellungen einer 
queerfreundlicheren Bibliothekswelt dar. \textit{Queerness} ist in Bibliotheken im 
deutschsprachigen Raum ein noch immer unterrepräsentiertes Thema, egal ob vor 
oder hinter der Theke, im Bestand oder in den Katalogen. Das muss sich ändern. 
Die \textit{Queerbrarians} und dieser Beitrag möchten diese Veränderung anstoßen, erläutern, 
warum sie wichtig und wertvoll ist und für eine queere Perspektive sensibilisieren. 
Dass dabei (bibliotheks-) politische Überlegungen ins Spiel kommen, ist beinahe 
unvermeidlich, denn queer sein ist (bibliotheks)politisch.
\end{abstract}

%body
\section{Queerness und
Bibliotheken}\label{queerness-und-bibliotheken}

Bibliotheken sind von Menschen und für Menschen und bilden daher
zwangsläufig durch ihre Mitarbeitenden, Nutzenden, Angebote und Bestände
menschliche Lebensrealitäten sowie menschliche Bedürfnisse und
menschliches Wissen in allen Facetten ab. Das schließt auch die
Diversitätsdimensionen \emph{Identität} und \emph{Orientierung} ein
(Charta der Vielfalt o. J.; Elsheri et al.~2022, Supplement Figure 2;
Timmo D. 2022). Im Deutschen wird meist von Geschlechtsidentität
gesprochen, im Englischen treffender von
\href{https://lgbtqia.fandom.com/wiki/Gender_identity}{Gender
\emph{Identity}}. Unter Orientierung können nach dem
\href{https://lgbtqia.mywikis.wiki/wiki/Split_Attraction_Model}{Split-Attraction
Model} romantische und sexuelle Orientierung gefasst werden (Glass
2022). Alle diese Aspekte vereinen wir in diesem Beitrag unter den
Begriffen \emph{Queerness} und \emph{queer}. Gemeint sind damit
Menschen, die sich als Teil der
\href{https://lgbtqia.mywikis.wiki/wiki/LGBT}{LGBTQIA+} Community
identifizieren, sowie ihre Lebensrealitäten und ihre Bedürfnisse.

Wir wählen \emph{Queerness} und \emph{queer} mit Bedacht und Absicht,
nicht nur für diesen Beitrag, sondern auch für die \emph{Queerbrarians},
das Netzwerk queerer Bibliotheksmenschen, um das es in diesem Beitrag
geht. Der Begriff \emph{queer} ist nicht unkontrovers und wurde lange
als Beleidigung verwendet, bevor die Community ihn für sich positiv
besetzt hat (Diversity Arts Culture o. J.). Er wird heute teilweise
sogar eingesetzt, um engere Label zu vermeiden (Log 2022) oder als
Synonym für \emph{LGBTQIA+} (Wright 2024). Nicht alle Mitglieder der
Community verwenden ihn jedoch als Selbstbezeichnung oder nehmen ihn als
übergeordnetes Label an. Wieder andere nehmen sich ganz explizit aus der
Community heraus, distanzieren sich von ihr und dem Begriff
\emph{queer}. Die jüngsten und bekanntesten Beispiele dafür sind Jens
Spahn, ehemaliger Bundesminister für Gesundheit und Mitglied der CDU,
sowie Alice Weidel, Bundestagsabgeordnete und Mitglied der AfD
(Achterberg 2023). Jens Spahn verewigte sich mit den Worten \enquote{Ich
bin nicht queer, ich bin schwul.}, Alice Weidel mit \enquote{Ich bin
nicht queer, sondern ich bin mit einer Frau verheiratet, die ich seit
zwanzig Jahren kenne.} Dem möchten wir nicht widersprechen, denn
\emph{queer} ist ein politischer Begriff und beinhaltet den Einsatz
gegen Diskriminierung und für Gleichbehandlung aller Mitglieder der
LGBTQIA+ Community. Dass wir hier und im Netzwerk also \emph{queer}
verwenden, ist an sich schon (bibliotheks)politisch.

Einige Lesende wird es zudem irritieren, dass hier von \emph{wir}
geschrieben wird. Das ist nicht immer üblich in fachlichen Beiträgen.
Wir erkennen jedoch an, dass unsere Identifikation als Teil der LGBTQIA+
Community sowohl einen Einfluss auf die hier präsentierten Perspektiven
als auch auf die Sichtbarkeit queerer Stimmen in der Bibliothekswelt
hat. Im deutschsprachigen Raum ist \emph{Queerness} in Bibliotheken im
Vergleich zu den USA ein noch immer unterrepräsentiertes Thema (Gerlach
2023). Laut dem Loop-Modell besteht eine Wechselwirkung zwischen dem
Selbstverständnis von Bibliotheken, der bibliothekswissenschaftlichen
Forschung und Praxis und den persönlichen Positionierungen
bibliothekarischer Akteur*innen (Gerlach 2023). Die angesprochene
Unterrepräsentation ist daher auch auf die Zurückhaltung bei
persönlichen Positionierungen sowie bei Artikulationen der eigenen
Betroffenheit durch Agierende in Forschung und Praxis zurückzuführen.
Diese Wechselwirkung möchten wir, als Teil der \emph{Queerbrarians},
sowohl mit dem \emph{wir} in diesem Beitrag als auch die
\emph{Queerbrarians} mit ihrer Gründung positiv verstärken.\footnote{Wir
  schreiben hier zudem als Teil der \emph{Queerbrarians} und nicht für
  unsere Einrichtungen und Bibliotheken.} Wir orientieren uns dabei an
jenen, die vor uns kamen, uns den Weg geebnet haben und deren Offenheit
dazu führte, dass andere sich weniger alleine fühlten und mit Fragen zu
Namensänderungen, Ratschlägen zum Coming-Out im Kollegium und vielem
mehr auf sie zukommen konnten (Walters 2023).

\section{Queerbrarians}\label{queerbrarians}

Entstanden ist die Idee zu den \emph{Queerbrarians} in kleiner Runde bei
der Planung und Einreichung eines Hands-On-Labs für die BiblioCon 2024
(Zeuner et al.~2024). Die Idee von \enquote{Mehr Glitzer? How To
LGBTQIA+ Safe Space für Bibliotheken} war eine offene Diskussion und die
Beantwortung der Fragen, wie man selbst eine Safe Person für die
LGBTQIA+ Community sein und die eigene Bibliothek zu einem Safe(r) Space
machen kann (The Roestone Collective 2014; Minkov 2021). Hierbei wurde
ein wenig frustriert festgestellt, dass der Anstoß zur Diskussion und
die Gestaltung des Hands-On-Labs mal wieder in den Händen der Community
selbst liegen.

Wenn die LGBTQIA+ Community Veränderung will, muss sie diese selber
schaffen. Ein Dilemma, denn Overburdening, also die Überlastung von
Mitgliedern der Community durch zusätzliche emotionale und tatsächliche
Arbeit rund um das Thema \emph{Queerness} (Heinze 2021), ist keine zu
unterschätzende Zusatzbelastung zur ohnehin vorhandenen beruflichen
Tätigkeit. Auch beim ersten Treffen der \emph{Queerbrarians} wurde das
thematisiert: Die Zusatzbelastung durch die unbeabsichtigt eingenommene
oder zugeschriebene Rolle als zentrale Ansprechperson für das Thema,
wenn man sich mit \emph{Queerness} auseinandersetzt, sich für
Sichtbarkeit und gegen Diskriminierung einsetzt oder sich outet, mischt
sich mit dem Bedürfnis, endlich Veränderung zu sehen.

Roberto schrieb 2011 zu seinen Erfahrungen als
\href{https://lgbtqia.fandom.com/wiki/Transgender}{trans} Mann in seiner
Bibliothek:

\begin{quote}
\enquote{Accept that you will be That Transgender Library Staffer for a
while, just because this may be new and unusual at your work, and
because people like to gossip. If you become heavily involved in
workplace activism, you may ultimately become That Angry Transgender
Library Staffer Who Ruins Everything.} (Roberto 2011, S. 127)
\end{quote}

Dass Personen mit einer solchen Rolle und dem Einsatz für die LGBTQIA+
Community auch in 2024 noch immer nicht ausschließlich positiv begegnet
und selten dafür gedankt wird, kommt also erschwerend hinzu. Am Ende
wurde aus dem scherzhaften \enquote{Dann können wir auch gleich ein
Netzwerk gründen und uns so wenigstens gegenseitig helfen} dennoch
Realität und die \emph{Queerbrarians} nahmen ihren Anfang.

Das erste Treffen mit rund 60 Personen fand virtuell am 21. November
2023 statt. So viele Menschen waren dem Aufruf über den
DACH-Discord-Server, TikTok und bibliothekarische Mailinglisten gefolgt.
Bei diesem Treffen wurden die Kommunikationskanäle Discord\footnote{Discord
  (\url{https://discord.com/}) ist eine Kommunikationsplattform, auf der
  sich Communities auf eigenen Servern in Sprach-, Video- und
  Text-Kanälen organisieren und austauschen können.} und
E-Mail\footnote{Wer nicht auf Discord ist, aber dennoch über Treffen
  informiert werden möchte, schreibt an
  \href{mailto:librarians@queerbrarians.de}{\nolinkurl{librarians@queerbrarians.de}}
  und bittet um Aufnahme auf den E-Mail-Verteiler.} festgelegt, weitere
Treffen geplant und Themen, Wünsche und Anliegen gesammelt. Alles, was
queere Bibliotheksmenschen bewegt, im Großen und im Kleinen. Am Ende
stand dennoch das Netzwerken im Zentrum: Wer bist du? Wo arbeitest du?
Wie geht es dir dort? Was sind deine Pronomen? Ist hier noch jemand
\href{https://lgbtqia.mywikis.wiki/wiki/Aroace}{aroace}? Nicht weniger
zentral war das Festhalten gewisser Regeln, die sowohl für die
virtuellen Treffen\footnote{\url{https://queerbrarians.de/naechste-termine/}}
als auch für Discord gelten (Tabelle 1).

\begin{table}[h]
\begin{tabular}{|ll|}
\hline
\multicolumn{2}{|l|}{\makecell[tl]{Bei den \emph{Queerbrarians} handelt es sich um einen Safe Space der queeren Community.\\ Damit sie auch ein Safe Space bleiben, gibt es ein paar Regeln für den Discord-Server\\ und die Treffen.}}                  \\ \hline
\multicolumn{1}{|l|}{1}  & Respektvoller und freundlicher Umgang miteinander.                                                                                                                     \\ \hline
\multicolumn{1}{|l|}{2}  & \makecell[tl]{Respektiert die Pronomen der anderen (Klick aufs Profilbild zeigt diese, ansonsten\\ nachfragen).}                                                                                \\ \hline
\multicolumn{1}{|l|}{3}  & \makecell[tl]{Diskriminierung aufgrund von Ethnie, Geschlecht, Geschlechtsidentität, sexueller\\ oder romantischer Orientierung, Religion, nationaler Herkunft, Alter, Abstammung\\ oder Behinderung wird hier nicht geduldet.} \\ \hline
\multicolumn{1}{|l|}{4}  & Inhalte angemessen halten..                                                                                                                                              \\ \hline
\multicolumn{1}{|l|}{5}  & \makecell[tl]{Es werden keine Themen, die während des Treffs oder auf Discord besprochen\\ werden, nach außen getragen, außer es wurde davor so abgemacht.}                                      \\ \hline
\multicolumn{1}{|l|}{6}  & \makecell[tl]{Niemand muss sich outen. Wenn du hier bist, wird angenommen,\\ dass du Teil der LGBTQIA+ Community bist und mehr muss keiner wissen.}                                 \\ \hline
\multicolumn{1}{|l|}{7}  & \href{https://queer-lexikon.net/2020/04/29/terf/}{TERF}s und allgemein transfeindliche Personen sind hier nicht geduldet.                                                                                                          \\ \hline
\multicolumn{1}{|l|}{8}  & \makecell[tl]{Das A in LGBTQIA+ steht nicht für \href{https://lgbtqia.mywikis.wiki/wiki/Ally}{Ally}, sondern für \href{https://lgbtqia.mywikis.wiki/wiki/Asexual}{Asexual}, \href{https://lgbtqia.mywikis.wiki/wiki/Aromantic}{Aromantic} und\\ \href{https://lgbtqia.mywikis.wiki/wiki/Agender}{Agender}. Ihr seid Teil der Community und damit Teil von uns.}                                 \\ \hline
\multicolumn{1}{|l|}{9}  & Kein Spammen und kein Trolling.                                                                                                                                               \\ \hline
\multicolumn{1}{|l|}{10} & Keine unpassenden Profile (Nutzendennamen, Avatare, Accounts und Status).                                                                                                   \\ \hline
\multicolumn{1}{|l|}{11} & \makecell[tl]{Bevor ihr Direktnachrichten schickt, bitte die entsprechende Person kurz fragen,\\ ob das gewünscht ist.}                                                                                    \\ \hline
\end{tabular}
\caption{Tabelle 1: Die Regeln der \textit{Queerbrarians} (Stand 20. Februar 2024).}
\end{table}

Auf Discord gibt es Bereiche zum Thema \emph{Queerbrarianship}, also
alles rund um \emph{Queerness} und Bibliotheken, aber auch allgemein zu
Interessen und zur Weiterentwicklung der \emph{Queerbrarians} selbst.
Außerdem werden dort, via E-Mail-Verteiler sowie auf der Webseite
(\url{https://queerbrarians.de/}) die nächsten virtuellen Treffen
angekündigt. Die Treffen bestehen derzeit aus einem thematischen Teil am
Anfang, wie zu queeren Medientipps oder dem Alltag als queere Person in
der eigenen Bibliothek, und einem offenen Teil für das Networking am
Ende. Ziel ist es zunächst, uns sowohl professionell, thematisch, aber
eben auch kollegial zu vernetzen und zu unterstützen --
Community-Building eben.

\section{Queere Themen}\label{queere-themen}

Beim ersten Treffen der \emph{Queerbrarians} wurden auf einer digitalen
Pinnwand Themen und Anliegen gesammelt, zu denen Austausch gewünscht
wird. Durch Upvotes konnten alle Anwesenden zudem signalisieren, welche
Wünsche anderer sie auch interessieren. Im Folgenden haben wir die
Ergebnisse zusammengetragen und in neun Themenblöcke aufgeteilt, wobei
einzelne Wünsche durchaus durch mehrere Themenblöcke fließen können. Wir
ergänzen diese zudem mit passender Literatur und weiteren sich daraus
ergebenden Aspekten. Dennoch sind die hier aufgegriffenen Themen keine
Liste mit Anspruch auf Vollständigkeit, sondern ein Abbild der Anliegen,
die für die \emph{Queerbrarians} eine besondere Relevanz haben.

\subsection{Queer(freundlich)es
Miteinander}\label{queerfreundliches-miteinander}

Das Thema mit den zweitmeisten Upvotes war das ausgesprochene Ziel, „die
Bibliothekswelt queerfreundlicher {[}zu{]} machen'' (25 Upvotes). An
dieser Stelle sollten alle direkt innehalten, denn das kann nur dann ein
Ziel sein, wenn es aktuell noch nicht so ist und diese Erkenntnis müssen
alle und insbesondere die nicht-queeren Mitglieder der
Bibliothekscommunity verinnerlichen, damit Veränderung möglich wird.
Dabei ist mit Bibliothekswelt alles gemeint: Mitarbeitende, Nutzende,
Angebote und Bestände.

Bibliotheken können keine Safe(r) Spaces für ihre Nutzenden sein, wenn
sie keine Safe(r) Spaces für ihre Mitarbeitenden sind. Das Thema
„Queerfeindlicher Arbeitsplatz'' ging einher mit den konkreten Fragen
„Wie gehe ich mit queerfeindlichen Kolleg*innen um?'' (12 Upvotes) und
„{[}W{]}as mache ich mit Deadnaming am Arbeitsplatz {[}...{]}?'' (7
Upvotes). Deadnaming bezeichnet dabei die Verwendung des Namens, den
eine trans oder
\href{https://queer-lexikon.net/2017/06/08/nichtbinaer/}{nicht-binäre}
Person bei der Geburt erhalten hat, dem sogenannten
\href{https://queer-lexikon.net/2019/12/23/deadname/}{Deadname},
anstelle des gewählten Namens (Sinclair-Palm \& Chokly 2023). Wie sehr
gerade Letzteres bewegt und auch dienstliche Beziehungen belasten kann,
lässt sich aus den Kommentaren zur Frage ablesen, die nach einem
möglichen Umgang mit falschen Anreden in dienstlichen E-Mails und den
Grenzen der Freundlichkeit nach mehrfachen Hinweisen fragen. Die Frage
danach, wie lange man freundlich bleiben muss, wenn mehrfach gesetzte
Grenzen überschritten werden, wie die Verwendung der richtigen Pronomen
und Anrede, schlägt tief in die Magengrube aller, die einer oder
mehreren marginalisierten Gruppen angehören und nicht selten Tone
Policing\footnote{Beim „Tone Policing'' wird der (angeblich
  unangemessene) Tonfall kritisiert, ohne sich mit dem eigentlichen
  Argument auseinanderzusetzen oder gar die Legitimität des Arguments
  mit Verweis auf den Tonfall explizit zurückgewiesen, nähere
  Erläuterungen siehe
  \url{https://geekfeminism.fandom.com/wiki/Tone_argument}.} erleben.

Problematisch ist auch die Tatsache, dass der Deadname rechtlich nicht
so einfach geändert werden kann, und somit noch häufig auf offiziellen
Dokumenten wie Arbeitsverträgen erscheint, und man gänzlich auf den
Respekt und die Empathie des Kollegiums und der Vorgesetzten angewiesen
ist. Personen, die keine Pronomen, Neopronomen\footnote{Neopronomen im
  Deutschen sind etwa „xier'' oder „dey'', siehe
  \url{https://www.nonbinary.ch/pronomen-anwendung/}.} oder wechselnde
Pronomen und Anreden verwenden, sind zusätzlich auf die Bereitschaft des
Kollegiums angewiesen, Neues zu lernen und flexibel zu sein. Ihre
einzigen anderen Alternativen sind,
\href{https://queer-lexikon.net/2020/04/29/misgendern/}{Misgendern}
hinzunehmen oder ungeouted zu bleiben und sich den Vorstellungen anderer
zu fügen. Dies kann eine schmerzhafte Erfahrung sein und betrifft unter
anderem, aber nicht ausschließlich, Menschen, die sich als nicht-binär,
\href{https://lgbtqia.fandom.com/wiki/Agender}{agender},
\href{https://lgbtqia.fandom.com/wiki/Genderqueer}{genderqueer} oder
\href{https://lgbtqia.fandom.com/wiki/Genderfluid}{genderfluid}
identifizieren (Thorne et al.~2020; Bradford 2020). Leitfäden zu
gendersensibler Sprache, die auch Geschlechtsidentitäten jenseits des
binären Schemas inkludieren, sind ein guter Anfang (Berufsverband
Information Bibliothek 2020; Keite 2024). Diese auch in die kollegiale
Kommunikation zu integrieren und zu leben, ist mancherorts leider noch
in weiter Ferne.

Dennoch drehen sich die Gedanken der \emph{Queerbrarians} nicht um
Negatives, sondern sie fragen konstruktiv, wie man \enquote{Kolleg*innen
sensibilisieren} kann \enquote{{[}f{]}ür queere Themen, Probleme, etc.}
(15 Upvotes). Dabei geht es vom Erkennen dieser kleinen Formen von
Alltagsdiskriminierung über das Mitdenken von nicht-cis und nicht-hetero
Perspektiven bis hin zum aktiven Einsatz für die LGBTQIA+ Community. Ein
Beispiel dafür sind \enquote{{[}u{]}nangenehme Harry-Potter-Diskussionen
bzw. Rechtfertigungen, warum ich keine HP-Veranstaltungen machen möchte}
(8 Upvotes). Für all jene, die nicht wissen, warum queere Menschen ein
durchaus schwieriges bis gänzlich ablehnendes Verhältnis zu Harry Potter
(HP) haben, ist es Zeit, das jetzt durch eine eigene Recherche
herauszufinden (Dias Correia 2023). Das selbst zu recherchieren und
nicht zu erwarten, es von queeren Personen auf Aufforderung erklärt zu
bekommen, ist Teil des Sensibilisierungsprozesses, der erste Schritt
\href{https://lgbtqia.mywikis.wiki/wiki/Ally}{Ally} zu werden, und
vermeidet das bereits diskutierte Overburdening. Sensibilisierung
bedeutet dabei nicht zwingend, eine solche Veranstaltung dann gar nicht
durchzuführen, sich aber der Problematik bewusst zu sein, offen zu
thematisieren, nicht nachzuhaken, wenn jemand die Beauftragung damit
ablehnt und nicht gerade eine der offen queeren Personen im Team dafür
einzuplanen.

Die Bibliothekscommunity muss darüber reden, wie sie ein
queerfreundliches Miteinander gestaltet, für Mitarbeitende und Nutzende.
Einen \enquote{Code of Conduct für Veranstaltungen} (3 Upvotes) in
unseren Bibliotheken zu etablieren, wie es beispielsweise bei Formaten
wie Jugend hackt\footnote{\emph{Jugend hackt} ist ein
  nicht-gewinnorientiertes Programm der gemeinnützigen Vereine Open
  Knowledge Foundation Deutschland e.V. und mediale pfade -- Verein für
  Medienbildung e.V. Mit dem Motto „Mit Code die Welt verbessern''
  richtet es sich an Jugendliche zwischen 12 und 18 Jahren. Für den Code
  of Conduct siehe \url{https://jugendhackt.org/code-of-conduct/}} der
Fall ist, kann ein Anfang sein. Aber auch auf unseren Konferenzen und
Treffen sowie in unseren Bibliotheken allgemein können Codes of Conduct
uns allen helfen, Unsicherheiten ab- und Sicherheit im Umgang
miteinander aufzubauen. Darüber hinaus sind klare Positionierungen ein
Werkzeug für Bibliotheken, um Mitgliedern der LGBTQIA+ Community zu
signalisieren, dass man zu den eigenen Worten steht. Es braucht
\enquote{Positionspapiere, wenn es mal wieder ernst wird} (9 Upvotes).
Dass es ernst werden kann, ist global und auch im deutschsprachigen Raum
wieder spürbar (Siggelkow \& Reveland 2023). Bibliotheken sollten daher
sichtbare und klare Verbündete sein für die Vielfalt von Identitäten und
Orientierungen. Gleichermaßen kann hierunter verstanden werden, dass
sich die \emph{Queerbrarians}, als Community queerer
Bibliotheksmenschen, durch Positionspapiere zu queeren Bibliotheksthemen
zu Wort melden wollen. Gemeinsam können wir uns eine Stimme geben.

\subsection{Queerness vor Ort}\label{queerness-vor-ort}

Der Ort, in dem das Miteinander gestaltet wird, ist dabei unweigerlich
mit dem Konzept des Safe(r) Spaces verbunden. Neben gendersensibler
Sprache und einem queerfreundlichen Miteinander braucht es daher auch
Räume, in denen Queerness mitgedacht wird. \enquote{Unisextoiletten in
Bibliotheken} (22 Upvotes) bleiben auch nach Jahren noch ein aktuelles,
wichtiges und bisher zu selten umgesetztes Thema.\footnote{Als
  Positivbeispiel möchten wir hier zumindest die All-Gender-Toilette in
  den Bibliotheken der Technischen Hochschule Köln und der Hochschule
  Neu-Ulm anführen. Abgerufen 20. Februar 2024, von
  \url{https://www.asta.th-koeln.de/ueber-uns/lgbt/} und
  \url{https://www.hnu.de/alle-news/detail/2023/11/9/erste-all-gender-toiletten-an-der-hochschule-neu-ulm}.}
So wie wir im kommunikativen Raum einen Safe(r) Space durch
gendersensible Sprache schaffen, in dem wir alle Identitäten
inkludieren, so können und müssen wir dies auch im physischen Raum tun.

Sichtbarkeit für queere Themen schaffen gerade Öffentliche Bibliotheken
durch den Einsatz von
\href{https://queer-lexikon.net/pride-flags/}{Pride Flags} während des
Pride Month oder dauerhaft zum Hervorheben queerer Literatur. Während
das durchaus Rainbow-Washing\footnote{\enquote{Rainbow-Washing}
  bezeichnet Strategien, die mit einer (angeblichen) Unterstützung der
  LGBTQIA+ Bewegung werben, um dadurch modern, fortschrittlich und
  tolerant zu wirken, ohne jedoch tatsächlich Maßnahmen umzusetzen.
  Siehe auch \url{https://thisisgendered.org/entry/rainbow-washing/}.}
sein kann (Fille 2022), wird dennoch signalisiert, dass queere Menschen
willkommen sind. Es kommt bei all dem nicht darauf an, von Anfang an
perfekt zu sein, sondern zu zeigen, dass das Thema reflektiert wird und
man offen ist für Feedback und Verbesserung.

\subsection{Queere IT}\label{queere-it}

Während wir im direkten Gespräch die Möglichkeit haben, unsere Worte und
Taten zu korrigieren, wie etwa beim Misgendern einer Person, so ist das
bei der Webseite oder in Formularen, wie dem Anmeldebogen für neue
Nutzende, nicht möglich (Frick \& Honold 2022). Dabei ließen sich
Anmeldebögen meist noch einfach ändern, im Gegensatz zu den darunter
liegenden IT-Systemen. Ein weiteres relevantes Thema ist daher der
Zusammenhang von \enquote{Queerness und IT} (12 Upvotes).

Ein Großteil der Prozesse in Bibliotheken wird mittlerweile digital in
vordefinierten Abläufen abgehandelt. Alles beginnt beim Anmeldeformular
und der Anmeldung von neuen Nutzenden im System. Beim Austausch über das
Thema kristallisierte sich ein einheitliches Bild heraus: An diesen
Stellen wird das Geschlecht abgefragt, \enquote{weil das System es
vorsieht}. Die Diskussion um weitere Gründe führte zu Statistiken, auf
deren Basis auch gezielter Literatur angeschafft und aufgestellt werden
soll. Dem entgegen steht aber der Wunsch, dass Bibliotheken veraltete
Geschlechterrollen nicht reproduzieren, sondern den Lesenden das bieten,
was sie lesen möchten, unabhängig von ihrer Geschlechtsidentität (Leyrer
2014).

Die Abfrage des Geschlechts anzupassen oder wegzulassen und bestehende
Formulare und Systeme zu ändern, macht diese queerfreundlicher, denn
gerade nicht-binäre und trans Menschen können durch solche Abfragen in
unangenehme Situationen gebracht werden. Anpassungen dieser Art
erfordern nicht selten größere Absprachen und manchmal einen rechtlichen
Anstoß. Doch selbst die inzwischen rechtlich vorgeschriebene Option
\emph{divers} wird diese Problematik nicht lösen, denn sie richtet sich
lediglich an \href{https://lgbtqia.mywikis.wiki/wiki/Intersex}{inter*}
Personen (Antidiskriminierungsstelle des Bundes o. J.). Auch hier wollen
die \emph{Queerbrarians} konstruktiv diskutieren, wie man Veränderungen
anstoßen und zumindest in Bibliotheken Lösungen finden kann.

Neben dem Umgang mit den Daten der Nutzenden gibt es aber auch bei den
Daten der Mitarbeitenden Diskussionen, wie und ob etablierte Systeme
umgestaltet werden können. Der Umgang mit \enquote{Vorgaben und
{[}das{]} Über{[}winden{]} von Schwierigkeiten in der Umsetzung von z.B.
E-Mail Signaturen und Ähnlichem} (11 Upvotes) ist dabei ein wichtiger
Aspekt. Gemeint ist damit unter anderem die freiwillige Einbindung von
Pronomen in die eigene Signatur. Ein Vorgehen, das theoretisch schnell
umgesetzt werden kann und dazu führt, dass das aktive Einbringen der
eigenen Pronomen normalisiert wird, da es den Fokus von jenen nimmt, die
das tun, um zu vermeiden, dass sie misgendert werden (Frick \& Honold
2022). Zudem hilft es auch der anderen Seite dabei, die gewünschten
Formulierungen zu nutzen. Leider wird nicht selten davon berichtet, dass
entsprechende Angaben in der Signatur nicht gern gesehen sind oder gar
untersagt werden.

\subsection{Queere Erschließung}\label{queere-erschlieuxdfung}

Auch im klassischen bibliothekarischen Arbeitsfeld der Erschließung muss
\emph{Queerness} mit- und weitergedacht werden. Das Thema
\enquote{Queere Erschließung - Wie verschlagworte ich queere Literatur?}
(17 Upvotes) kam nicht nur beim ersten Treffen der \emph{Queerbrarians}
auf, sondern sorgt für regelmäßigen Gesprächsstoff. So sollte bereits
diskutiert werden, ob das Schlagwort \emph{LGBTQIA+} ausreicht, wenn es
in einem Medium eigentlich konkret um Bisexualität und in einem anderen
um Aromantik geht, oder ob es überhaupt angebracht ist Medien als
\emph{divers} oder \emph{queer} zu verschlagworten (Brown 2020;
Drabinski 2013; Wilk \& Vincent 2018). Auch der GND-Schlagwort\-katalog
weist einige Mängel auf, wenn es um queere Verschlagwortung geht.
Begriffe wie \emph{cisgender} werden im Gegensatz zu \emph{transgender}
nicht geführt. \emph{Gender} existiert erst seit kurzem (Aleksander
2022). Die Existenz oder Nichtexistenz sowie die Vergabepraxis von
Schlagwörtern stehen dabei nicht erst seit kurzem in der Kritik (Sparber
2016).

Wie viel Tiefe und Breite wir in unseren Schlagwörtern und
Klassifikationen zulassen, hat eine Wirkung nach außen, aber auch nach
innen. Eine kontinuierliche Reflexion unter Berücksichtigung der
Perspektiven Betroffener und das kritische Hinterfragen und Verändern
bisheriger Praktiken (Hutchinson \& Nakatomi 2023) sind dabei ein
Schritt in die richtige Richtung. Der Homosaurus
(\url{https://homosaurus.org/}) und das Queer Metadata Collective
(\url{https://queermetadatacollective.org/}) sind zwei Beispiele für
entsprechende Initiativen. Daran schließen sich direkt Überlegungen zur
\enquote{Namensvergabe in der GND} (8 Upvotes) an. Ob und wenn ja, wie
dort der Deadname einer Person verzeichnet werden sollte, braucht eine
queere bibliothekarische Perspektive und eine bibliotheksethische
Diskussion über die Abwägung der Wünsche der Person, der
bibliothekarischen Praxis und den mitunter veralteten und ohne queere
Perspektiven verfassten Regelwerken. Auch die in der GND verwendeten
binär gegenderten Berufsbezeichnungen, wie Soziologe und Soziologin,
sind problematisch, wenn im Personennormsatz eine nicht-binäre oder
agender Person beschrieben werden soll (Bargmann 2022).

\subsection{Queerer Bestand}\label{queerer-bestand}

Die Bestände von Bibliotheken bilden menschliche Lebensrealitäten sowie
menschliche Bedürfnisse und menschliches Wissen in allen Facetten ab.
Das tun die verschiedenen Bibliothekstypen jeweils auf ihre ganz eigene
Art und Weise und jede Bibliothek für sich noch einmal ganz im
Besonderen. Vom strategischen Bestandsmanagement bis hin zu
individuellen Anschaffungsentscheidungen einzelner Personen können viele
Ebenen Einfluss auf die repräsentierten Inhalte und die jeweilige Breite
und Tiefe haben. Vermehrt versuchen auch externe Faktoren darauf
Einfluss zu nehmen. Besonders lebhaft kann das seit etwa 2021 in
Schulbibliotheken in den USA beobachtet werden, wo die sogenannten Book
Bans den Umfang von Zensur drastisch ausweiten (Orsborn 2022). Bisher
ist kein Rückgang der Entwicklung zu erkennen, im Gegenteil steigen die
Zahlen der gebannten und in Frage gestellten Medien weiter an. PEN
America beobachtet und dokumentiert das Geschehen und stellt fest, dass
es oft gerade jene Bücher betrifft, die bereits lange um einen Platz im
Regal kämpfen mussten: Bücher von BIPoC (Black, Indigenous, and People
of Color) oder Mitgliedern der LGBTQIA+ Community sowie Bücher, die ganz
unabhängig von den Verfassenden Rassismus, Sexualität, Geschlecht und
Geschichte thematisieren.\footnote{\url{https://pen.org/issue/book-bans/}}
Proteste von Eltern oder Initiativen, Verwaltungsentscheidungen oder
politischer Druck führen dazu, dass der Zugang zu diesen Büchern
eingeschränkt wird oder sie ganz aus den Schulbibliotheken verschwinden.
Dass auch im deutschsprachigen Raum Druck auf bibliothekarische Bestände
ausgeübt wird, ist nichts Neues (Laudenbach 2023, Mobile Beratung gegen
Rechtsextremismus Berlin 2023). Die \emph{Queerbrarians} möchten daher
das \enquote{Bannen von queeren Büchern in Amerika - (und Deutschland?
Hopefully not\ldots)} (8 Upvotes) thematisieren und im Auge behalten,
auch um die Bibliotheken dort zu unterstützen.

Eine allgemeine Diskussion über \enquote{Queere Themen/Charaktere in
Filmen, Serien, Bücher{[}n{]}} (2 Upvotes) kann und soll ebenfalls
stattfinden. Um eine adäquate Repräsentation aller
Geschlechtsidentitäten und Orientierungen zu gewährleisten, ist
stellenweise noch immer eine kreative Bestandsentwicklung gefragt. So
spielt Selfpublishing gerade für unterrepräsentierte Teile der LGBTQIA+
Community eine nicht zu vernachlässigende Rolle (Kennon 2021). Auch
\enquote{aktivistische Literatur und Verbände} (2 Upvotes) in
Bibliotheken ist ein Thema, das aufgegriffen werden soll. Die
\enquote{Fetischisierung von BL/GL (in Manga)} (6 Upvotes) soll
ebenfalls Thema werden. BL steht dabei für \emph{Boys Love} und GL für
\emph{Girls Love}. Es wurde weiter ausgeführt: \enquote{Die Auswahl von
nicht-fetischisierenden Büchern ist manchmal schwer, gerade im
Mainstream und auch weil es oft einfach gefühlt am meisten gelesen wird.
Ich versuche gerne aufzuklären, warum diese Bücher/dargestellten
Beziehungen problematisch sind, finde es aber oft wie ein Kämpfen gegen
Windmühlen. Habt ihr auch solche Probleme?} Die \emph{Queerbrarians}
wollen und sollen ein Ort für genau solche Fragen sein, die in den
eigenen Bibliotheken oft ungehört verhallen. Kein Wunder also, dass
\enquote{Kolleg*innen überzeugen{[},{]} queeren Bestand zu sichten
{[}und zu{]} erweitern} (7 Upvotes) auch zum Thema wird.

Bibliotheken können durch ihre Bestände \enquote{Queer sein mehr
normalisieren} (12 Upvotes) und damit einen geäußerten Wunsch der
\emph{Queerbrarians} realisieren. Dass es sogar schon (wenn auch nicht
immer bibliothekarische) Vorbilder für Bibliotheken, Sammlungen und
Bestände mit queerem Fokus gibt, möchten wir dabei nicht unter den Tisch
fallen lassen. Als Beispiele seien die Queer Bibliothek
(\url{https://queerbib.de/}), die Queerfeministische Bibliothek des
Allgemeinen Studierendenausschusses der FU Berlin
(\url{https://astafu.de/bibliothek}) und die Bibliothek des Schwulen
Museums in Berlin (\url{https://www.schwulesmuseum.de/bibliothek/})
genannt.

\subsection{Zugang zu queeren Themen}\label{zugang-zu-queeren-themen}

\begin{quote}
\enquote{Oftentimes, when an individual is discovering and exploring their
identity, they will search for mirror characters: examples of themselves
in media as a way to understand what it means to identify a particular
way.} (Allen 2022, S. 3).
\end{quote}

Gerade junge Menschen verbinden daher auch mit dem Lesen die Erfahrung,
sich in fiktiven Figuren wiederzufinden und sich mit ihnen zu
identifizieren. Sie auf ihren Reisen durch das Leben, schwierige
Situationen und die Identitätsfindung zu begleiten, kann ein integraler
Bestandteil des Aufwachsens und der eigenen Identitätsfindung sein.
Insbesondere für queere Jugendliche zeigen Studien \enquote{that LGBTIQ+
identity development processes are their primary motivators to read}
(Wexelbaum 2019, S. 115). Durch eine heute höhere Bandbreite und Dichte
an Diversität in allen Medien ist es auch für marginalisierte Gruppen
leichter geworden Geschichten zu finden, die sie widerspiegeln, während
sie ihre Identität, Beziehungen und die Welt um sich herum verstehen
lernen -- vorausgesetzt der Zugang zu den Medien wird nicht durch
Initiativen wie Book Bans eingeschränkt oder ganz verhindert.

Menschen, die sich der LGBTQIA+ Community zugehörig fühlen, erleben
vermehrt Mobbing, sexuelle Gewalt und psychische Probleme (Orsborn
2022). Umso wichtiger ist es, dass Bibliotheken als sichere, öffentliche
Räume fungieren und queeren Menschen die Möglichkeit geben, sich
geschützt zu informieren, auszutauschen und zu entfalten (Wright 2024).
Gerade für Menschen mit eher unterrepräsentierten Identitäten und
Orientierungen kann das essentiell sein:

\begin{quote}
\enquote{The validation and affirmation of
\href{https://lgbtqia.fandom.com/wiki/Asexual}{asexuality} as an
orientation and the equitable recognition of the full spectrum of
asexuality are particularly significant for questioning,
\href{https://lgbtqia.fandom.com/wiki/Asexual}{ace}, and
\href{https://lgbtqia.fandom.com/wiki/Asexual_spectrum}{acespec} young
readers seeking representation and who might not have encountered
inclusive and respectful stories about their experiences and
identities.} (Kennon 2021, S. 19; Verlinkungen von den Autor*innen
ergänzt)
\end{quote}

Die Stadtbibliothek Hannover hatte anlässlich des Christopher Street Day
(CSD) 2022 Infoflyer zum Thema
\href{https://trans.fandom.com/wiki/Binding}{Binding} und
\href{https://trans.fandom.com/wiki/Tucking}{Tucking} ausgelegt (Becker
2023). Die Empörung in den sozialen Medien darüber war groß \enquote{und
die erste Reflexreaktion war, die Flyer einfach wegzupacken}, so Tom
Becker, Leiter der Stadtbibliothek Hannover (Mobile Beratung gegen
Rechtsextremismus 2023, S. 37). Er resümiert stattdessen: \enquote{Hier
müssen wir resilienter werden -- auch bei Themen, die wir so nicht immer
direkt in der Breite der Mitarbeitendenschaft einordnen können.} (Mobile
Beratung gegen Rechtsextremismus 2023, S. 37) So geht Normalisierung des
Zugangs zu queeren und insbesondere gesundheitsrelevanten queeren
Themen. Bibliotheken können hier, gerade durch die Auslage solcher
Flyer, einen signifikanten Beitrag zur Gesundheitsbildung und Sicherheit
queerer Jugendlicher leisten. Die LGBTQIA+ Community hat, gewachsen aus
der traurigen Realität der Notwendigkeit, viel Erfahrung in der
eigenständigen Organisation und Bereitstellung von
Gesundheitsinformationen und rückt das kollektive Wissen und die
Erfahrungen ihrer Mitglieder ins Zentrum (Kitzie et al.~2023).
Initiativen Öffentlicher Bibliotheken sollten daher mit der Community
zusammenarbeiten, wie es die Stadtbibliothek Hannover getan hat, und bei
Gegenwind nicht nachgeben.

Hitzige Diskussionen und Demonstrationen löste auch die Veranstaltung
\enquote{Wir lesen euch die Welt, wie sie euch gefällt} der Münchner
Stadtbibliothek im Juni 2023 aus (Heudorfer 2023). Dabei handelte es
sich um eine Lesung, bei der eine
\href{https://queer-lexikon.net/2017/06/15/drag-queen/}{Dragqueen} und
ein \href{https://queer-lexikon.net/2017/06/15/drag-king/}{Dragking} aus
Kinderbüchern vorlasen und Aufklärungsarbeit leisteten. In den
Geschichten ging es um das Überwinden von Geschlechterschubladen
(Miebach 2023) und das Entdecken der eigenen freien Entfaltung
(Heizereder 2023). Diese Lesung war nicht die erste dieser Art in dieser
Bibliothek, jedoch die erste mit wahlkampfbedingten und medienwirksamen
Gegenstimmen aus allen Richtungen. So sprach die CSU von \enquote{woker
Frühsexualisierung} (Miebach 2023) und in der Fachzeitschrift des
Berufsverbands Information Bibliothek (BuB) wurde per Zuschrift in der
Kategorie \emph{Leserbrief} kommentiert, dass \enquote{Bibliotheken ihre
Programme nicht mit gesellschaftlichen Botschaften überfrachten} sollten
(Werner 2024, S. 21). Ackermann, der Direktor der Münchner
Stadtbibliothek, betonte das Gegenteil: \enquote{Wir brauchen Vorbilder, die
zeigen, es ist okay anders zu sein}' (Miebach 2023).

Um nicht den Eindruck zu erwecken, dass der Zugang zu queeren Themen nur
für Öffentliche Bibliotheken relevant ist, möchten wir betonen, dass
eine starke wissenschaftliche Bibliothek mit umfangreichen und diversen
Sammlungen, unterstützenden und hilfsbereiten Mitarbeitenden und
relevanten Angeboten in Kooperation und im Austausch mit queeren
Netzwerken auf dem Campus sehr dazu beitragen kann, den Campus für
Studierende aus der LGBTQIA+ Community integrativer zu gestalten (Clarke
2011, Wright 2024). Zudem brauchen auch angehende und bereits im
Berufsleben stehende Bibliotheksmenschen Zugang zu queeren Themen. Wie
oft diese in Berufs- und Hochschulen aufkommen und behandelt werden,
kann man bisher nur spekulieren. Die Wahl queerer Themen für
Abschlussarbeiten und deren Veröffentlichung kann ein Hinweis sein und
signalisiert nach außen zumindest die Offenheit in der Hochschullehre.
Das gleiche gilt für studentische Projekte (Berends et al.~2023, 2024).
Wir dürfen nicht unterschätzen, wie wichtig die Sichtbarkeit von
\emph{Queerness} in unserem Berufsfeld für Interessierte an Ausbildung
und Studium sein kann. Wo wir \emph{Queerness} nicht sehen, können wir
uns nicht sicher sein, dass unsere \emph{Queerness} willkommen ist. Aber
auch als ganz reguläres Thema in Ausbildung und Studium kann und sollte
\emph{Queerness} vorkommen. Mögliche Ansätze beschrieb Mehra bereits
2011. Neben formaler Repräsentation, wie durch Arbeitsgruppen und
Netzwerke, können offizielle Ansprechpersonen und Workshops zur
Sensibilisierung nicht nur Signale sein, sondern unsere Kollegien
sensibler oder gar bunter machen (Mehra 2011, Table 1).

\subsection{Out in Beruf, Ausbildung und
Studium}\label{out-in-beruf-ausbildung-und-studium}

Beim ersten Treffen der \emph{Queerbrarians} wurde an einer Stelle im
lockeren Gespräch gefragt, wer am Arbeitsplatz geoutet ist. Das Ergebnis
war sehr gemischt. An dieser Stelle wird gerne argumentiert, dass auch
cis und hetero Personen nicht im Beruf out sind oder ihre Identität und
Orientierung dort nicht vor sich hertragen. Wir leben jedoch in einer
cis- und heteronormativen Welt und vielen ist offenbar nicht bewusst,
dass der Kollege, der vom Wochenende mit seiner Frau und den Kindern
erzählt, nicht zwingend cis und hetero sein muss, sich aber der Kollege,
der vom Wochenende mit seinem Mann und den Kindern erzählt, automatisch
als zumindest nicht hetero outet, obwohl er nichts anderes tut als der
Kollege zuvor: von seinem Wochenende mit der Familie erzählen. Eine
nicht-binäre Person, die sich am Arbeitsplatz nicht outen will, wird
zwangsläufig misgendert, weil der vorherrschende deutsche Sprachgebrauch
noch immer binäre Formen präferiert, Menschen also vornehmlich männliche
oder weibliche Pronomen und Anreden nutzen und diese basierend auf ihrer
Wahrnehmung der Person auswählen. Falls die Person zudem einen gewählten
Namen hat, den sie jedoch nicht nennt, um sich nicht zu outen, kommt es
zum Deadnaming. Auch das Gegenteil kann der Fall sein: Wenn eine
nicht-binäre oder trans Person einen Namen gewählt hat, der nach Ansicht
anderer nicht zu dem von ihnen wahrgenommenen Geschlecht passt, ist
diese Person oft gezwungen, sich zu outen, ihre Namenswahl zu
rechtfertigen oder mitunter unangenehme Diskussionen zu diesem Thema zu
führen. Das ist vor allem dann der Fall, wenn der Deadname noch in
offiziellen Schreiben, der eigenen E-Mail-Adresse oder an anderen
Stellen auftaucht, da dieser rechtlich noch überall mit angegeben werden
muss und Verwaltung und IT keine Anpassung ohne rechtliche Änderung
vornehmen müssen.

\enquote{Out im Beruf, ja{[},{]} nein? Wie? Umgang mit Unverständnis und
Vorurteilen} (14 Upvotes) ist daher ein zentraler Aspekt für das
Wohlbefinden und die Unbefangenheit queerer Personen am Arbeitsplatz
(Riggle et al.~2017). Laut einer Studie aus dem Jahr 2020 gehen in
Deutschland 31\,\% der Befragten nicht offen mit ihrer \emph{Queerness}
im Kollegium um (Vries et al.~2020). An dieser Stelle muss anerkannt
werden, dass es dabei signifikante Unterschiede innerhalb der LGBTQIA+
Community gibt. Laut einer Studie der britischen Regierung aus dem Jahr
2018 sind 29\,\% der befragten cis Personen und 38\,\% der befragten trans
Personen bei niemandem auf der Arbeit geoutet (Government Equalities
Office 2018, S. 139 und S. 142). Diese Zahlen variieren stark mit der
sexuellen Orientierung. Während nur 18\,\% der sich als schwul oder
lesbisch identifizierenden cis Personen bei niemandem auf der Arbeit
geoutet sind, sind es 77\,\% der sich als asexuell identifizierenden cis
Personen. Bei trans Personen liegen diese Zahlen bei 27\,\% für sich als
schwul oder lesbisch identifizierende trans Personen und 57\,\% für sich
als asexuell identifizierende trans Personen. Mit Blick auf Schule,
Ausbildung und Studium zeigt sich, dass sowohl 10\,\% der befragten cis
als auch der befragten trans Personen nicht bei ihren Mitschüler*innen
oder Mitstudierenden geoutet sind (Government Equalities Office 2018,
S.111 und S. 113). Die Unterschiede in Abhängigkeit von der sexuellen
Orientierung sind in diesen Personenkreisen deutlich geringer, aber noch
immer gegeben. Für Deutschland gibt es bisher keine Studien mit einer
solchen Detailtiefe.

Wer außerhalb von Beruf, Ausbildung oder Studium geoutet ist, innerhalb
aber nicht, findet sich in einem konstanten Spannungsfeld und in
Habachtstellung wieder: \enquote{I was female-identified at work, and
some flavor of transgender almost everywhere else; as I've never really
been able to completely separate my personal and professional lives,
this was incredibly difficult to do.} (Roberto 2011, S. 124) Das kostet
Kraft und belastet die mentale Gesundheit.

Dem Wunsch nach und der Notwendigkeit von mehr Sichtbarkeit und
Repräsentation muss mit strukturellen und gemeinschaftlichen
Veränderungen begegnet werden. Diese Aufgabe darf nicht auf die
Mitglieder der LGBTQIA+ Community abgewälzt und erst recht nicht als
ihre Individualverantwortung definiert werden. Noch immer hat Offenheit
über die eigene \emph{Queerness} auch negative Konsequenzen.
\enquote{One student was told by a senior lecturer that talking about
their asexuality in their work would limit their career.} (Benoit \& de
Santos 2023, S. 13). Es ist genauso inakzeptabel, dass Menschen sich
nicht outen können, weil sie Repressionen befürchten, wie es
inakzeptabel ist, von Menschen zu erwarten, dass sie sich outen, um
\emph{Queerness} sichtbarer zu machen und zu normalisieren. Wir sprechen
hier von sehr persönlichen und individuellen Entscheidungen sowie
persönlicher Sicherheit und Sichtbarkeit, und alle Optionen sind legitim
und zu unterstützen.\footnote{Aus diesem Grund haben wir beschlossen,
  gesammelt auf die Angabe unserer Pronomen zu verzichten, um die
  Privatsphäre der einzelnen Beteiligten zu schützen.} Das bedeutet
auch, dass niemand eine andere Person outet.

Bibliotheken müssen eine Atmosphäre schaffen, in der sich alle
Mitglieder der LGBTQIA+ Community sicher genug fühlen, um frei zu
entscheiden, ob sie sich outen oder nicht. Mitarbeitende sollen sich
frei fühlen, ihre ganze Persönlichkeit zum Arbeitsplatz mitzubringen
(Wright 2024). In einer idealen Welt, wenn wir hier einmal träumen
dürfen, würde niemand etwas über die Identität und Orientierung einer
anderen Person annehmen und niemand beziehungsweise alle müssten sich
outen. Soweit ist unsere cis- und heteronormative Welt jedoch noch
nicht. Die \emph{Queerbrarians} wollen daher auch \enquote{Queer sein
mehr normalisieren} (12 Upvotes) indem sie zusammenstehen und so als
Gruppe und nicht notwendigerweise als Einzelpersonen sichtbar sind.

\subsection{Queere Personalentwicklung}\label{queere-personalentwicklung}

Für diejenigen, die auf der Arbeit geoutet sind, wird das Thema
\emph{Queerness} im Arbeitsleben schnell zur Herzenssache. Das führt in
vielen Fällen dazu, dass queere Themen und Weiterentwicklungsprozesse an
diesen Personen hängen bleiben und es schnell zum Overburdening kommt.
Damit das nicht so bleibt und die notwendigen Bemühungen auf viele
Schultern verteilt werden können, muss es mehr
Weiterbildungsmöglichkeiten und Aufklärungsarbeit zu Diversität geben,
unter die neben vielen anderen Themen auch das Thema \emph{Queerness}
fällt. Der Redebedarf zu \enquote{Kolleg*innen sensibilisieren -- Für
queere Themen, Probleme, etc.} (15 Upvotes) spiegelt diesen Wunsch
wider. Dem gegenüber steht das Problem, dass viele Mitarbeitende keinen
Bedarf sehen, sich in diesem Bereich fortzubilden (Mefebue 2016).
Dahinter steckt vielleicht keine böse Absicht. \enquote{{[}Der{]}
Arbeitsplatz {[}ist{]} überhaupt nicht queerfeindlich, aber weiter mega
heteronormativ} (19 Upvotes). Unwissenheit auf Seiten der nicht-queeren,
teilweise auch der queeren, Mitarbeitenden führt aber dazu, dass offene
Handlungsfelder nicht gesehen und daher nicht angegangen werden. An
dieser Stelle können weiterbildende Maßnahmen sowie eine Steigerung der
Attraktivität von Bibliotheken als Arbeitsplatz für queere Menschen
ansetzen.

Neben der persönlichen Weiterbildung von Mitarbeitenden muss aber auch
auf den institutionellen Ebenen Veränderung passieren, damit sich die
Situation langfristig ändern kann:

\begin{quote}
\enquote{We must put our money where our mouths are. We must have
leadership that is willing to engage in brave, difficult conversations
that interrogate the hiring practices of their organizations, as well as
how to retain talented people from underrepresented backgrounds}
(Stringer-Stanback \& Jackson 2023, S. 463).
\end{quote}

Eine Person berichtete, dass die eigene Bibliothek gerade daran
arbeitet, einen \enquote{Gleichstellungsplan} (6 Upvotes) aufzustellen.
Auf dem Discord-Server wurden Vorschläge und Ideen zu queeren Maßnahmen
gesammelt, die in diesen Plan mit aufgenommen werden könnten. Solche
Initiativen sind ein Grundstein von vielen für die zukunftsorientierte
Weiterentwicklung von Bibliotheken und ihren Mitarbeitenden, bei denen
alle mitgedacht und mitgenommen werden.

\subsection{Queeres Netzwerk}\label{queeres-netzwerk}

\begin{quote}
\enquote{In my working life as a library technician {[}...{]}, being
trans has led to hilarious and awkward conversations with colleagues,
moments of genuine connection, and exciting professional development.
I've also felt the impact and toll when transgender issues enter
workplace discussions. My experience of being a trans library technician
has been positive overall, but there is always fear.} (Nault 2023, S.
46)
\end{quote}

Im queeren Netzwerk möchten \emph{Queerbrarians} Positives und Negatives
miteinander teilen, sich unterstützen und nicht alleine lassen.
Gleichzeitig entstehen so Verbindungen und Synergien, die Neues schaffen
können, wie dieser Beitrag. Daher wundert es nicht, dass \enquote{Neue
Leute kennenlernen} (28 Upvotes) die meisten Upvotes hatte. Aber auch
andere Netzwerkaktivitäten kamen auf, wie gegenseitige
\enquote{Buchtipps, queere Romane, gute Bilderbücher, gendersensible
Aufklärungsbücher etc.} (11 Upvotes), das \enquote{Sammeln aller
wissenschaftlichen Arbeiten aus dem Netzwerk {[}...{]} zum Thema - z. B.
Artikel, Bachelor-Arbeiten, Abschlussarbeiten\ldots{}} (6 Upvotes) als
offene Zotero-Gruppe und eine Auseinandersetzung mit Unterschieden in
der Community selbst, wie dem \enquote{Age Gap untersch{[}iedlicher{]}
LGBTQIA/\href{https://queer-lexikon.net/2020/05/30/flint/}{FLINTA}-Generationen}
(7 Upvotes). Das alles waren Ideen, die beim ersten Treffen aufkamen und
durch das Netzwerk gerade angegangen werden. Bei all dem schwingt mit,
dass Veränderungen am besten im Team erreicht werden können und die
\emph{Queerbrarians} dafür einen möglichen Rahmen bieten.

\section{Queere Zukunft}\label{queere-zukunft}

Die Repräsentation von \emph{Queerness} ist in Bibliotheken im
deutschsprachigen Raum, egal ob vor oder hinter der Theke, in den
Beständen oder Katalogen, noch immer nicht ausreichend, um queere
Menschen zu unterstützen und ihnen genug Sicherheit zu geben, sich frei
zu bewegen und zu entfalten. Basierend auf den zusammengetragenen
Erfahrungen und Wünschen aus dem Kreis der \emph{Queerbrarians} und der
angeführten Literatur, identifizieren wir die noch nicht ausreichende
Sensibilität für diese Themen und queere Lebensrealitäten in der
Bibliothekscommunity als eine der Hauptursachen. Wir erkennen an, dass
es für einige noch immer ungewohnt ist nicht-cis und nicht-hetero
Perspektiven mitzudenken, gleichzeitig denken queere Menschen jedoch cis
und hetero Perspektiven immer mit und können nicht alleine dafür
verantwortlich sein, dass auch queere Lebensrealitäten in der
Bibliotheksarbeit abgebildet und unterstützt werden. Das ist ein
Gemeinschaftsauftrag, der nicht am Mangel an Bereitschaft bei
nicht-queeren Mitarbeitenden und Vorgesetzten scheitern darf.

Argumentationen, die das Thema als Randthema darstellen, verkennen den
bibliothekarischen Auftrag und die Tatsache, dass Inklusion immer
wichtig ist, selbst wenn sie nur eine einzige Person betrifft. Wenn
Bibliotheken sich allen Lebensrealitäten öffnen wollen, gehören auch
queere dazu, egal ob vor oder hinter der Theke. \emph{Queerness} darf
nicht aus dem Diskurs verdrängt werden, weder aufgrund fehlender
Sensibilisierung noch von rechten gesellschaftlichen Strömungen. Queere
Menschen sind Teil unserer Nutzenden und Teil der Bibliothekscommunity
und kein Bibliotheksmensch sollte mehr feststellen müssen \enquote{how
deeply alienating and dehumanizing it is to always be thinking about how
to better serve a community when it\textquotesingle s politically toxic
to even acknowledge that you and people like you are part of that
community.} (Baker 2023, S. 158) Deshalb würden wir an dieser Stelle
eigentlich gerne darauf verzichten mit Statistiken zu argumentieren,
aber da einige Menschen sie benötigen, verweisen wir darauf, dass sich
laut Ipsos-Umfrage zum LGBT+ Pride 2023 in der Schweiz 13\,\% und in
Deutschland 11\,\% der befragten Bevölkerung als Teil der LGBTQIA+
Community identifizieren (Ipsos 2023). Dass Bibliotheken also zur
Normalisierung queerer Lebensrealitäten beitragen sollten, kann nicht
wegargumentiert werden.

Die \emph{Queerbrarians} möchten aktiv daran mitarbeiten
auszuformulieren, wie queerfreundliche Bibliotheken für Mitarbeitende
und Nutzende aussehen können und wie das bibliothekspolitisch umsetzbar
ist, Fortbildungen und Vorträge gestalten und offene Treffen für alle
Bibliotheksmenschen, also auch nicht-queere, anbieten. Gleichzeitig
erkennen wir unseren eigenen Bias als derzeit vorwiegend weißes Netzwerk
an. Eine Überrepräsentation, der sich die gesamte Bibliothekscommunity
ebenfalls bewusst werden muss, um sie zu beseitigen. Intersektionale
Perspektiven brauchen mehr Raum und Sichtbarkeit. Wir brauchen mehr
Verständnis füreinander. Wir brauchen mehr Wissen über queere Themen.
Wir brauchen mehr Wissen über romantische und sexuelle Orientierungen.
\enquote{Wir brauchen mehr Wissen über geschlechtliche Vielfalt, mehr
Informationsmöglichkeiten und eine ehrliche Auseinandersetzung.} (Lieb
2023, S. 88)

\section{Referenzen}\label{referenzen}

Achterberg, B. (2023, 15.November). Jens Spahn distanziert sich vom
Queer-Begriff: Bin nicht queer, ich bin schwul. \emph{Neue Zürcher
Zeitung}. Abgerufen 20. Februar 2024, von
\url{https://www.nzz.ch/international/jens-spahn-distanziert-sich-vom-queer-begriff-bin-nicht-queer-ich-bin-schwul-ld.1765767}

Aleksander, K. (2022). Antrag zur Aufnahme des Sachbegriffs „Gender'' in
die Gemeinsame Normdatei (GND) der Deutschen Nationalbibliothek (DNB).
\emph{027.7 Zeitschrift Für Bibliothekskultur / Journal for Library
Culture}, \emph{9}(4). \url{https://doi.org/10.21428/1bfadeb6.c232c54d}

Allen, M. (2022). \enquote{In a Romantic Way, Not Just a Friend Way!}:
Exploring the Developmental Implications of Positive Depictions of
Bisexuality in Alice Oseman's Heartstopper. \emph{Journal of
Bisexuality}, 23(2), 197--228.
\url{https://doi.org/10.1080/15299716.2022.2153191}

Antidiskriminierungsstelle des Bundes. (o. J.). \emph{Frau -- Mann -
Divers: Die „Dritte Option'' und das Allgemeine Gleichbehandlungsgesetz
(AGG)}. Antidiskriminierungsstelle des Bundes. Abgerufen 20. Februar
2024, von
\url{https://www.antidiskriminierungsstelle.de/DE/ueber-diskriminierung/diskriminierungsmerkmale/geschlecht-und-geschlechtsidentitaet/dritte-option/Dritte_Option.html}

Baker, A. E. (2023). Holding onto Dreams. In K. K. Adolpho, S. G.
Krueger, \& K. McCracken (Hrsg.), \emph{Trans and gender diverse voices
in libraries} (S. 157--167). Library Juice Press.

Bargmann, M., Blumesberger, S., Gruber, A., Luef, E., \& Steltzer, R.
(2022). Sacherschließung geschlechtergerecht?! Rückblick auf den
Online-Workshop am 11. Mai 2022 und Aufruf zu gemeinsamen Aktivitäten.
\emph{Mitteilungen der Vereinigung Österreichischer Bibliothekarinnen
und Bibliothekare}, \emph{75}(2), Article 2.
\url{https://doi.org/10.31263/voebm.v75i2.7582}

Becker, T. (2023, 13.Juli). »Der Sinn von Politik ist Freiheit«.
\emph{BuB - Forum Bibliothek und Information}. Abgerufen 20. Februar
2024, von
\url{https://www.b-u-b.de/detail/der-sinn-von-politik-ist-freiheit}

Benoit, Y., \& de Santos, R. (2023). \emph{Ace in the UK Report.}
Stonewall. Abgerufen 20. Februar 2024, von
\url{https://www.stonewall.org.uk/system/files/ace_in_the_uk_report_2023.pdf}

Berends, A., Fabian, L. S., Fels, R., Lächelt, A.-L., Nguyen Thu, M.,
Reinhard, J., Schliemann, T., Schmidt, L., Steinike-Kuhn, J., Storch,
J., \& Waldorf, M. (2023). \emph{Diversität in Bibliotheken - Wie gut
sind Bibliotheken aufgestellt?} \url{https://doi.org/10.25968/OPUS-2933}

Berends, A., Fabian, L. S., Fels, R., Lächelt, A.-L., Nguyen Thu, M.,
Reinhard, J., Schliemann, T., Schmidt, L., Steinike-Kuhn, J., Storch,
J., \& Waldorf, M. (2024, 17. Januar). Diversität in Bibliotheken -- Wie
gut sind Bibliotheken aufgestellt? \emph{BuB - Forum Bibliothek und
Information}. Abgerufen 20. Februar 2024, von
\url{https://www.b-u-b.de/detail/diversitaet-in-bibliotheken-wie-gut-sind-bibliotheken-aufgestellt}

Berufsverband Information Bibliothek (2020). \emph{Leitfaden für
gendersensible Sprache und diskriminierungsfreie Kommunikation}.
Abgerufen 20. Februar 2024, von
\url{https://www.bib-info.de/fileadmin/public/Dokumente_und_Bilder/BIB-Standpunkte/LeitfadenSpracheBild_20201107.pdf}

Bradford, N. J., Rider, G. N., Catalpa, J. M., Morrow, Q. J., Berg, D.
R., Spencer, K. G., \& McGuire, J. K. (2020). Creating gender: A
thematic analysis of genderqueer narratives. In Joz, M., Nieder, T., \&
Bouman, W. (Hrsg.), \emph{Non-binary and Genderqueer Genders} (S.
37--50). Routledge.

Brown, A. (2020, 26. August). How Labeling Books as \enquote{Diverse}
Reinforces White Supremacy. Lee \& Low Blog. The Open Book Blog.
\url{https://blog.leeandlow.com/2020/08/26/how-labeling-books-as-diverse-reinforces-white-supremacy/}

Charta der Vielfalt (o. J.). \emph{Vielfaltsdimensionen - Die sieben
Dimensionen von Vielfalt.} Abgerufen 20. Februar 2024, von
\url{https://www.charta-der-vielfalt.de/fuer-organisationen/vielfaltsdimensionen/}

Clarke, K. L. (2011). LGBTIQ Users and Collections in Academic
Libraries. In E. Greenblatt (Hrsg.), \emph{Serving LGBTIQ library and
archives users: Essays on outreach, service, collections and access} (S.
81--93). McFarland \& Company, Inc., Publishers.

Drabinski, E. (2013). Queering the Catalog: Queer Theory and the
Politics of Correction. \emph{The Library Quarterly}, 83(2), 94--111.
\url{https://doi.org/10.1086/669547}

Dias Correira, J. (2023). Asking the Bigger Questions: The Problem with
LGBT+ Allyship in Libraries. In K. K. Adolpho, S. G. Krueger, \& K.
McCracken (Hrsg.), \emph{Trans and gender diverse voices in libraries}
(S. 449--453). Library Juice Press.

Diversity Arts Culture (o. J.). Queer. In \emph{Diversity Arts Culture}.
Abgerufen 3. Januar 2024, von
\url{https://diversity-arts-culture.berlin/woerterbuch/queer}

Elsherif, M. M., Middleton, S. L., Phan, J. M., Azevedo, F., Iley, B.
J., Grose-Hodge, M., Tyler, S. L., Kapp, S. K., Gourdon-Kanhukamwe, A.,
Grafton-Clarke, D., Yeung, S. K., Shaw, J. J., Hartmann, H., \&
Dokovova, M. (2022). Bridging Neurodiversity and Open Scholarship: How
Shared Values Can Guide Best Practices for Research Integrity, Social
Justice, and Principled Education. \emph{MetaArXiv}.
\url{https://doi.org/10.31222/osf.io/k7a9p}

Fille, R. (2022). \emph{Rainbowashing: Does it Impact Purchase
Intention?} {[}Honors Thesis, Bryant University{]}.
\url{https://digitalcommons.bryant.edu/honors_marketing/46}

Frick, C., \& Honold, C. (2022). Gendersensible Sprache im Kontakt mit
Bibliotheksnutzenden. \emph{027.7 Zeitschrift Für Bibliothekskultur /
Journal for Library Culture}, 9(4).
\url{https://doi.org/10.21428/1bfadeb6.58e38319}

Gerlach, S. (2023). \emph{How queer is the library (not)? - Die
bibliothekswissenschaftliche Rezeption von LGBTIQ*: ein Vergleich
zwischen Deutschland und den USA.} {[}Masterarbeit, Technische
Hochschule Köln{]}.
\url{https://nbn-resolving.org/urn:nbn:de:hbz:79pbc-opus-20933}

Glass, V. Q. (2022). Queering Relationships: Exploring Phenomena of
Asexual Identified Persons in Relationships. \emph{Contemporary Family
Therapy}, 44(4), 344--359.
\url{https://doi.org/10.1007/s10591-022-09650-9}

Government Equalities Office (2018). \emph{National LGBT Survey:
Research Report.} Government Equalities Office.
\url{https://assets.publishing.service.gov.uk/media/5b3b2d1eed915d33e245fbe3/LGBT-survey-research-report.pdf}

Heinze, J. L. (2021, 24. Januar). Fact Sheet on Injustice in the LGBTQ
community. National Sexual Violence Resource Center\emph{.}
\url{https://www.nsvrc.org/blogs/fact-sheet-injustice-lgbtq-community}

Heizereder, S. (2023). Der Twitter-Mob wütet. \emph{BuB - Forum
Bibliothek und Information}, 75(6), 257.

Heudorfer, K., Steinbauer, M. M., Schröter, A. M., \& Brack, G. (2023,
13 Juni). \emph{Drag-Lesung für Kinder: 500 Menschen für ein buntes
München}. BR24. Abgerufen 20. Februar 2024, von
\url{https://www.br.de/nachrichten/bayern/drag-lesung-fuer-kinder-500-menschen-fuer-ein-buntes-muenchen,Th3oG7z}

Hutchinson, J., \& Nakatomi, J. (2023). Improving Subject Description of
an LGBTQ+ Collection. \emph{Cataloging \& Classification
Quarterly},\emph{61}(3--4), 380--394 .
\url{https://doi.org/10.1080/01639374.2023.2229828}

Ipsos. (2023, 1. Juni). \emph{LGBT+ Pride 2023}. Abgerufen 20. Februar
2024, von
\url{https://www.ipsos.com/en/pride-month-2023-9-of-adults-identify-as-lgbt}

Keite, U. (2024). Gendersensible und diskriminierungsfreie Sprache. In
Berufsverband Information Bibliothek / Kommission für One-Person
Libraries, \emph{OPL-Checklisten}, Checkliste 47. Abgerufen 20. Februar
2024, von
\url{https://www.bib-info.de/fileadmin/public/Dokumente_und_Bilder/Komm_OPL/Checklisten/check47.pdf}

Kennon, P. (2021). Asexuality and the Potential of Young Adult
Literature for Disrupting Allonormativity. \emph{The International
Journal of Young Adult Literature}, 2(1), 1--24.
\url{https://doi.org/10.24877/IJYAL.41}

Kitzie, V., Vera, A.N., Lookingbill, V. and Wagner, T.L. (2024),
\enquote{What is a wave but 1000 drops working together?}: The role of
public libraries in addressing LGBTQIA+ health information disparities,
\emph{Journal of Documentation}, 80(2), 533--551.
\url{https://doi.org/10.1108/JD-06-2023-0122}

Laudenbach, P. (2023, 30. August). Rechte Angriffe auf Bibliotheken:
Bücher mit Messern zerschnitten - Kulturkampf. \emph{Süddeutsche
Zeitung.} Abgerufen 20. Februar 2024, von
\url{https://www.sueddeutsche.de/kultur/bibliotheken-rechte-angriffe-antisemitismus-buecher-zerschneiden-1.6177557}

Leyrer, K. (2014). Das Geschlecht spukt in der Stadtbibliothek: Ein
Aufruf für genderneutrale Bibliotheksangebote, \emph{LIBREAS Library
Ideas,} 25, 76--79. \url{https://doi.org/10.25595/401}

Lieb, S. (2023). \emph{Alle(s) Gender: Wie kommt das Geschlecht in den
Kopf?} (1. Auflage). Querverlag.

Log, A. (2022, 15. November). Answering questions on the label
\enquote{queer}. \emph{The Clock}.
\url{https://www.plymouth.edu/theclock/answering-questions-on-the-label-queer/}

Mefebue, A. B. (2016). Umgang mit sozialer Diversität in der
Bibliotheksarbeit -- eine empirische Untersuchung. In K. Futterlieb \&
J. Probstmeyer (Hrsg.), \emph{Diversity Management und interkulturelle
Arbeit in Bibliotheken} (S. 43--74). De Gruyter.
\url{https://doi.org/10.1515/9783110338980-005}

Mehra, B. (2011). Integrating LGBTIQ Representation Across the Library
and Information Science Curriculum: A Strategic Framework for
Student-Centered Interventions. In E. Greenblatt (Hrsg.), \emph{Serving
LGBTIQ library and archives users: Essays on outreach, service,
collections and access} (S. 298--309). McFarland \& Company, Inc.,
Publishers.

Miebach, E. (2023, 13. Juni). Eine Kinderbuchlesung im Wahlkampfstrudel.
\emph{ZDFheute}. Abgerufen 20. Februar 2024, von
\url{https://www.zdf.de/nachrichten/panorama/muenchen-dragqueen-lesung-diskussion-demo-100.html}

Minkov, M. (2021). Eine Pause von der Welt. \emph{Zeitgeister: Das
Kulturmagazin des Goethe-Instituts.} Abgerufen 20. Februar 2024, von
\url{https://www.goethe.de/prj/zei/de/art/22554555.html}

Mobile Beratung gegen Rechtsextremismus Berlin. (2023). \emph{Alles nur
leere Worte? - Zum Umngang mit dem Kulturkampf von rechts in
Bibliotheken.} Abgerufen 20. Februar 2024, von
\url{https://www.mbr-duesseldorf.de/fileadmin/content/medien/230715_MBR_Broschuere_Bibliotheken_online.pdf}

Nault, C. (2023). On Fear, Professionalism, and Being That Trans Guy
Library Technician. In K. K. Adolpho, S. G. Krueger, \& K. McCracken
(Hrsg.), \emph{Trans and gender diverse voices in libraries} (S.
45--51). Library Juice Press.

Orsborn, C. E. (2022). \emph{A Golden Age of Censorship: LGBTQ Young
Adult Literature in High School Libraries} {[}Electronic thesis or
dissertation, Ohio Dominican University{]}. OhioLINK Electronic Theses
and Dissertations Center.
\url{http://rave.ohiolink.edu/etdc/view?acc_num=oduhonors1669994581915957}

Riggle, E. D. B., Rostosky, S. S., Black, W. W., \& Rosenkrantz, D. E.
(2017). Outness, concealment, and authenticity: Associations with LGB
individuals' psychological distress and well-being. \emph{Psychology of
Sexual Orientation and Gender Diversity}, 4(1), 54--62.
\url{https://doi.org/10.1037/sgd0000202}

Roberto, K. R. (2011). Passing Tips and Pronoun Police: A Guide to
Transitioning at Your Local Library. In T. M. Nectoux (Hrsg.), \emph{Out
behind the desk: Workplace issues for LGBTQ librarians} (S. 121--127).
Library Juice Press.

Siggelkow, C., \& Reveland, P. (2023,17. Juli ).
\emph{Queerfeindlichkeit: Verstärkte Mobilisierung gegen queere
Menschen}. Tagesschau. Abgerufen 20. Februar 2024, von
\url{https://www.tagesschau.de/faktenfinder/kontext/queerfeindlichkeit-desinformation-100.html}

Sinclair-Palm, J., \& Chokly, K. (2023). \enquote*{It's a giant faux
pas}: Exploring young trans people's beliefs about deadnaming and the
term deadname. \emph{Journal of LGBT Youth}, 20(2), 370--389.
\url{https://doi.org/10.1080/19361653.2022.2076182}

Sparber, S. (2016). What's the frequency, Kenneth? -- Eine
(queer)feministische Kritik an Sexismen und Rassismen im
Schlagwortkatalog. \emph{Mitteilungen der Vereinigung Österreichischer
Bibliothekarinnen und Bibliothekare}, 69(2), 236--243.
\url{https://doi.org/10.31263/voebm.v69i2.1629}

Stringer-Stanback, K., \& Jackson, L. (2023). Remixing LIS Leadership:
Considering Gender-Variant BIPOC - Are we there yet? In K. K. Adolpho,
S. G. Krueger, \& K. McCracken (Hrsg.), \emph{Trans and gender diverse
voices in libraries} (S. 455--465). Library Juice Press.

The Roestone Collective (2014). Safe Space: Towards a
Reconceptualization. \emph{Antipode}, 46(5), 1346--1365.
\url{https://doi.org/10.1111/anti.12089}

Thorne, N., Yip, A. K.-T., Bouman, W. P., Marshall, E., \& Arcelus, J.
(2020). The terminology of identities between, outside and beyond the
gender binary -- A systematic review. In J. Motmans, T. O. Nieder, \& W.
P. Bouman (Hrsg.), \emph{Non-binary and Genderqueer Genders} (S. 20--36).
Routledge.

Timmo D. (2022). \emph{Wheel of Privilege and Power}. Center for
Teaching, Learning \& Mentoring.
\url{https://kb.wisc.edu/instructional-resources/page.php?id=119380}

Vries, L. D., Fischer, M., Kasprowski, D., Kroh, M., Kühne, S., Richter,
D., \& Zindel, Z. (2020). LGBTQI*-Menschen am Arbeitsmarkt: Hoch
gebildet und oftmals diskriminiert. \emph{DIW Wochenbericht}, 36,
620--627. \url{https://doi.org/10.18723/DIW_WB:2020-36-1}

Walters, J. (2023). Standing Out. In K. K. Adolpho, S. G. Krueger, \& K.
McCracken (Hrsg.), \emph{Trans and gender diverse voices in libraries}
(S. 175--180). Library Juice Press.

Werner, K. U. (2024). Gut gemeint ist nicht immer gut. \emph{BuB - Forum
Bibliothek und Information}, 76(1), 21.

Wexelbaum, R. (2019). The Reading Habits and Preferences of LGBTIQ+
Youth. \emph{The International Journal of Information, Diversity, \&
Inclusion (IJIDI)}, 3(1),
112--129.{[}\url{https://doi.org/10.33137/ijidi.v3i1.32270}{]}

Wilk, A., \& Vincent, J. (2018, 3. April). Respecting anonymity through
collection development. \emph{Open Shelf}. Abgerufen 20. Februar 2024,
von
\url{https://open-shelf.ca/180403-respecting-anonymity-through-collection-development/}

Wright, D. (2024, 23. Februar). \emph{Heartstopper! Sustaining the
Library-LGBTQ Love Affair.} Making collections accessible and diverse:
current approaches to audience engagement {[}Video{]}. YouTube.
\url{https://youtu.be/6t4yJGK4QGY?t=2282}

Zeuner, P., Buchert, C., Fischer, Y., Baumann, N., Frick, C., \&
Ramünke, S. (2024). Bibliotheken als Safe(r) Spaces für die LGBTQIA+
Community? Hands-on Lab auf der BiblioCon 2024. API Magazin, 5(2),
Article 2. \url{https://doi.org/10.15460/apimagazin.2024.5.2.209}

%autor
\begin{center}\rule{0.5\linewidth}{0.5pt}\end{center}

\textbf{Autor*innen}

Claudia Frick, Technische Hochschule Köln, Köln, Deutschland, claudia.frick@th-koeln.de, \url{https://orcid.org/0000-0002-5291-4301}

Philipp Zeuner, Bundesinstitut für Berufsbildung, Bonn, Deutschland, \url{https://orcid.org/0000-0002-1307-1145}

Caleb Buchert, Technische Hochschule Köln, Köln, Deutschland

Daniela Markus, Universitätsbibliothek Kiel, Christian-Albrechts-Universität zu Kiel, Kiel, Deutschland, \url{https://orcid.org/0009-0008-3514-9450}

Norma Fötsch, University Library, Radboud University, Nijmegen, The Netherlands

Yvonne Fischer, Stadtbibliothek Köln, Köln, Deutschland

Emma Wieseler, Technische Hochschule Köln, Köln, Deutschland

Sabrina Ramünke, Universitätsbibliothek Freie Universität Berlin, Berlin, Deutschland, \url{https://orcid.org/0000-0003-4091-7588}

Nik Baumann, Stadt- und Schulbücherei Gunzenhausen und Landesfachstelle für das öffentliche Bibliothekswesen Bayern - Nürnberg, Deutschland

\end{document}