\textbf{Kurzfassung}: Seit November 2023 gibt es die \textit{Queerbrarians}, ein 
deutschsprachiges Netzwerk queerer Bibliotheksmenschen. Anhand der beim 
ersten Treffen zusammengetragenen Themen legt dieser Beitrag neben den Beweggründen 
für die Entstehung des Netzwerks auch dessen Ideen und Vorstellungen einer 
queerfreundlicheren Bibliothekswelt dar. \textit{Queerness} ist in Bibliotheken im 
deutschsprachigen Raum ein noch immer unterrepräsentiertes Thema, egal ob vor 
oder hinter der Theke, im Bestand oder in den Katalogen. Das muss sich ändern. 
Die \textit{Queerbrarians} und dieser Beitrag möchten diese Veränderung anstoßen, erläutern, 
warum sie wichtig und wertvoll ist und für eine queere Perspektive sensibilisieren. 
Dass dabei (bibliotheks)politische Überlegungen ins Spiel kommen, ist beinahe 
unvermeidlich, denn queer sein ist (bibliotheks)politisch.