\documentclass[a4paper,
fontsize=11pt,
%headings=small,
oneside,
numbers=noperiodatend,
parskip=half-,
bibliography=totoc,
final
]{scrartcl}

\usepackage[babel]{csquotes}
\usepackage{synttree}
\usepackage{graphicx}
\setkeys{Gin}{width=.4\textwidth} %default pics size

\graphicspath{{./plots/}}
\usepackage[ngerman]{babel}
\usepackage[T1]{fontenc}
%\usepackage{amsmath}
\usepackage[utf8x]{inputenc}
\usepackage [hyphens]{url}
\usepackage{booktabs} 
\usepackage[left=2.4cm,right=2.4cm,top=2.3cm,bottom=2cm,includeheadfoot]{geometry}
\usepackage[labelformat=empty]{caption} % option 'labelformat=empty]' to surpress adding "Abbildung 1:" or "Figure 1" before each caption / use parameter '\captionsetup{labelformat=empty}' instead to change this for just one caption
\usepackage{eurosym}
\usepackage{multirow}
\usepackage[ngerman]{varioref}
\setcapindent{1em}
\renewcommand{\labelitemi}{--}
\usepackage{paralist}
\usepackage{pdfpages}
\usepackage{lscape}
\usepackage{float}
\usepackage{acronym}
\usepackage{eurosym}
\usepackage{longtable,lscape}
\usepackage{mathpazo}
\usepackage[normalem]{ulem} %emphasize weiterhin kursiv
\usepackage[flushmargin,ragged]{footmisc} % left align footnote
\usepackage{ccicons} 
\setcapindent{0pt} % no indentation in captions
\usepackage{xurl} % Breaks URLs

%%%% fancy LIBREAS URL color 
\usepackage{xcolor}
\definecolor{libreas}{RGB}{112,0,0}

\usepackage{listings}

\urlstyle{same}  % don't use monospace font for urls

\usepackage[fleqn]{amsmath}

%adjust fontsize for part

\usepackage{sectsty}
\partfont{\large}

%Das BibTeX-Zeichen mit \BibTeX setzen:
\def\symbol#1{\char #1\relax}
\def\bsl{{\tt\symbol{'134}}}
\def\BibTeX{{\rm B\kern-.05em{\sc i\kern-.025em b}\kern-.08em
    T\kern-.1667em\lower.7ex\hbox{E}\kern-.125emX}}

\usepackage{fancyhdr}
\fancyhf{}
\pagestyle{fancyplain}
\fancyhead[R]{\thepage}

% make sure bookmarks are created eventough sections are not numbered!
% uncommend if sections are numbered (bookmarks created by default)
\makeatletter
\renewcommand\@seccntformat[1]{}
\makeatother

% typo setup
\clubpenalty = 10000
\widowpenalty = 10000
\displaywidowpenalty = 10000

\usepackage{hyperxmp}
\usepackage[colorlinks, linkcolor=black,citecolor=black, urlcolor=libreas,
breaklinks= true,bookmarks=true,bookmarksopen=true]{hyperref}
\usepackage{breakurl}

%meta
%meta

\fancyhead[L]{S. Dunker\\ %author
LIBREAS. Library Ideas, 45 (2024). % journal, issue, volume.
\href{https://doi.org/10.18452/...}{\color{black}https://doi.org/10.18452/...}
{}} % doi 
\fancyhead[R]{\thepage} %page number
\fancyfoot[L] {\ccLogo \ccAttribution\ \href{https://creativecommons.org/licenses/by/4.0/}{\color{black}Creative Commons BY 4.0}}  %licence
\fancyfoot[R] {ISSN: 1860-7950}

\title{\LARGE{Eindrücke von einem Dienstag (dem ausleihstärksten Tag der Woche) aus der Gemeindebücherei Grasbrunn im Dezember 2023}}% title
\author{Sabine Duncker} % author

\setcounter{page}{1}

\hypersetup{%
      pdftitle={Eindrücke von einem Dienstag (dem ausleihstärksten Tag der Woche) aus der Gemeindebücherei Grasbrunn im Dezember 2023},
     pdfauthor={Sabine Duncker},
      pdfcopyright={CC BY 4.0 International},
      pdfsubject={LIBREAS. Library Ideas, 45 (2024).},
      pdfkeywords={Öffentliche Bibliothek, Gemeindebibliothek, Alltag},
      pdflicenseurl={https://creativecommons.org/licenses/by/4.0/},
      pdfurl={https://doi.org/10.18452/...},
      pdfdoi={10.18452/...},
      pdflang={de},
      pdfmetalang={de}
     }



\date{}
\begin{document}

\maketitle
\thispagestyle{fancyplain} 

%abstracts

%body
Zur Situation: Die Bücherei hat ca. 14.000 Medien und circa 800
angemeldete Leserinnen und Leser. Wir hatten zwei Mikrofone von Oliver
Bertram von grasbrunn-aktuell.de angebracht (eines an der Glasscheibe
eines Thekenarbeitsplatzes und eines im Kulturcafé, ein separater, durch
eine Tür getrennter Raum mit Kaffeemaschine zur Selbstbedienung,
Spielekonsole KUTI, Gesellschaftsspielen zum Spielen vor Ort und
Bilderausstellungen).

Die Tonaufzeichnungen im Einzelnen:

\hypertarget{an-der-theke}{%
\section{1. An der Theke:}\label{an-der-theke}}

\hypertarget{kundenfrage-nach-einem-bestimmten-buchtitel}{%
\subsection{1.1 Kundenfrage nach einem bestimmten
Buchtitel}\label{kundenfrage-nach-einem-bestimmten-buchtitel}}

\hypertarget{scannen-und-bondruck}{%
\subsection{1.2 Scannen und Bondruck}\label{scannen-und-bondruck}}

\hypertarget{leise-nebengeruxe4usche-leise-gespruxe4che}{%
\subsection{1.3 leise Nebengeräusche, leise
Gespräche}\label{leise-nebengeruxe4usche-leise-gespruxe4che}}

\hypertarget{lauter-geworden-spielende-kinder-einschreiten-der-bibliothekarin}{%
\subsection{1.4 lauter geworden, spielende Kinder: Einschreiten der
Bibliothekarin}\label{lauter-geworden-spielende-kinder-einschreiten-der-bibliothekarin}}

\hypertarget{huxf6flicher-austausch-mit-kundinnen-und-kunden}{%
\subsection{1.5 höflicher Austausch mit Kundinnen und
Kunden}\label{huxf6flicher-austausch-mit-kundinnen-und-kunden}}

\hypertarget{im-kulturcafuxe9}{%
\section{2. Im Kulturcafé:}\label{im-kulturcafuxe9}}

\hypertarget{lautes-spielen-am-kuti-und-absprachen-dazu}{%
\subsection{2.1 lautes Spielen am KUTI und Absprachen
dazu}\label{lautes-spielen-am-kuti-und-absprachen-dazu}}

\hypertarget{umruxfchren-in-einer-kaffeetasse}{%
\subsection{2.2 Umrühren in einer
Kaffeetasse}\label{umruxfchren-in-einer-kaffeetasse}}

\hypertarget{lauteres-spiel-steigert-sich}{%
\subsection{2.3 lauteres Spiel, steigert
sich}\label{lauteres-spiel-steigert-sich}}

\hypertarget{wuxfcrfeln-bei-einem-gesellschaftsspiel}{%
\subsection{2.4 Würfeln bei einem
Gesellschaftsspiel}\label{wuxfcrfeln-bei-einem-gesellschaftsspiel}}

Wir möchten mit diesen Aufnahmen dokumentieren, dass es in öffentlichen
Bibliotheken inzwischen um einiges lebhafter geworden ist, es aber
dennoch nicht an gegenseitiger Rücksichtnahme fehlt.

%autor
\begin{center}\rule{0.5\linewidth}{0.5pt}\end{center}

\textbf{Sabine Dunker}. Die gelernte Sortimentsbuchhändlerin und
Diplom-Bibliothekarin (FH) war zunächst in der kleinen evangelischen
Bücherei in Kamen (Westfalen) engagiert. Im Anschluss an die Ausbildung
war sie fünf Jahre in der Stadtbibliothek Essen beschäftigt. Nach
Erziehungszeit und Umzug nach Bayern arbeitete sie ein Jahr in der
Beratung der oberbayerischen Bibliotheken bei der Landesfachstelle für
das öffentliche Bibliothekswesen (im Münchener Team) mit. Seit 2009
leitet sie die Gemeindebücherei Grasbrunn und arbeitet dort zusammen mit
drei nebenamtlichen und einem Mitarbeiter der Lebenshilfewerkstatt sowie
fast fünfzig ehrenamtlichen Mitarbeitern und Mitarbeiterinnen.

\end{document}