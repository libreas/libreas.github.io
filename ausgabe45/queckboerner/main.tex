\documentclass[a4paper,
fontsize=11pt,
%headings=small,
oneside,
numbers=noperiodatend,
parskip=half-,
bibliography=totoc,
final
]{scrartcl}

\usepackage[babel]{csquotes}
\usepackage{synttree}
\usepackage{graphicx}
\setkeys{Gin}{width=.4\textwidth} %default pics size

\graphicspath{{./plots/}}
\usepackage[ngerman]{babel}
\usepackage[T1]{fontenc}
%\usepackage{amsmath}
\usepackage[utf8x]{inputenc}
\usepackage [hyphens]{url}
\usepackage{booktabs} 
\usepackage[left=2.4cm,right=2.4cm,top=2.3cm,bottom=2cm,includeheadfoot]{geometry}
\usepackage[labelformat=empty]{caption} % option 'labelformat=empty]' to surpress adding "Abbildung 1:" or "Figure 1" before each caption / use parameter '\captionsetup{labelformat=empty}' instead to change this for just one caption
\usepackage{eurosym}
\usepackage{multirow}
\usepackage[ngerman]{varioref}
\setcapindent{1em}
\renewcommand{\labelitemi}{--}
\usepackage{paralist}
\usepackage{pdfpages}
\usepackage{lscape}
\usepackage{float}
\usepackage{acronym}
\usepackage{eurosym}
\usepackage{longtable,lscape}
\usepackage{mathpazo}
\usepackage[normalem]{ulem} %emphasize weiterhin kursiv
\usepackage[flushmargin,ragged]{footmisc} % left align footnote
\usepackage{ccicons} 
\setcapindent{0pt} % no indentation in captions
\usepackage{xurl} % Breaks URLs

%%%% fancy LIBREAS URL color 
\usepackage{xcolor}
\definecolor{libreas}{RGB}{112,0,0}

\usepackage{listings}

\urlstyle{same}  % don't use monospace font for urls

\usepackage[fleqn]{amsmath}

%adjust fontsize for part

\usepackage{sectsty}
\partfont{\large}

%Das BibTeX-Zeichen mit \BibTeX setzen:
\def\symbol#1{\char #1\relax}
\def\bsl{{\tt\symbol{'134}}}
\def\BibTeX{{\rm B\kern-.05em{\sc i\kern-.025em b}\kern-.08em
    T\kern-.1667em\lower.7ex\hbox{E}\kern-.125emX}}

\usepackage{fancyhdr}
\fancyhf{}
\pagestyle{fancyplain}
\fancyhead[R]{\thepage}

% make sure bookmarks are created eventough sections are not numbered!
% uncommend if sections are numbered (bookmarks created by default)
\makeatletter
\renewcommand\@seccntformat[1]{}
\makeatother

% typo setup
\clubpenalty = 10000
\widowpenalty = 10000
\displaywidowpenalty = 10000

\usepackage{hyperxmp}
\usepackage[colorlinks, linkcolor=black,citecolor=black, urlcolor=libreas,
breaklinks= true,bookmarks=true,bookmarksopen=true]{hyperref}
\usepackage{breakurl}

%meta
%meta

\fancyhead[L]{B. Queckbörner\\ %author
LIBREAS. Library Ideas, 45 (2024). % journal, issue, volume.
\href{https://doi.org/10.18452/29138}{\color{black}https://doi.org/10.18452/29138}
{}} % doi 
\fancyhead[R]{\thepage} %page number
\fancyfoot[L] {\ccLogo \ccAttribution\ \href{https://creativecommons.org/licenses/by/4.0/}{\color{black}Creative Commons BY 4.0}}  %licence
\fancyfoot[R] {ISSN: 1860-7950}

\title{\LARGE{Vom \enquote{Sound-of-Silence} -- Anmerkungen zur Soundscape Bibliothek}}% title
\author{Boris Queckbörner} % author

\setcounter{page}{1}

\hypersetup{%
      pdftitle={Vom "Sound-of-Silence" -- Anmerkungen zur Soundscape Bibliothek},
     pdfauthor={Boris Queckbörner},
      pdfcopyright={CC BY 4.0 International},
      pdfsubject={LIBREAS. Library Ideas, 45 (2024).},
      pdfkeywords={Soundscape, theoretischer Raum, Lärm, Lautstärke, soziale Konstruktion, Soundlandkarte},
      pdflicenseurl={https://creativecommons.org/licenses/by/4.0/},
      pdfurl={https://doi.org/10.18452/29138},
      pdfdoi={10.18452/29138},
      pdflang={de},
      pdfmetalang={de}
     }



\date{}
\begin{document}

\maketitle
\thispagestyle{fancyplain} 

%abstracts
\begin{abstract}
\noindent
\textbf{Kurzfassung:} Der Beitrag stellt in aller Kürze das Konzept der
Soundscape vor, wie es vor allem von R. Murray Schafer (1977) entwickelt
worden ist. Ausgehend von Schafer wird in der Folge argumentiert, dass
es sich bei der Bibliothek um eine spezifische Ausprägung einer
Soundscape handelt, die im Kontrast zur gesellschaftlichen Umwelt als
klangliche Heterotopie (Hi-Fi-Soundscape) beschrieben werden kann. Diese
Hi-Fi-Umgebung trägt maßgeblich zur Entstehung einer lernförderlichen
Atmosphäre in Bibliotheken bei, die unter anderem mit Begriffen wie
\enquote{Bibliothekskonzentration} oder \enquote{Out-of-the-Box-Konzentration}
(Fansa, 2008) beschrieben worden ist und als Alleinstellungsmerkmal der
Einrichtung gilt. Das Soundscape-Konzept ermöglicht hier eine neue
analytische Perspektive auf diese Zusammenhänge, wodurch ein Mehrwert
unter anderem für die bibliothekarischen Diskussionen um den physischen
Lernort Bibliothek generiert werden kann. Zum Abschluss werden einige
potentielle Anschlussthemen skizziert.

\begin{center}\rule{0.5\linewidth}{0.5pt}\end{center}

\noindent
\textbf{Abstract:} This article briefly introduces the concept of the
soundscape as primarily conceptualized by R. Murray Schafer (1977).
Based on Schafer, it is argued that the library is a specific form of
soundscape that can be described as a sonic heterotopia
(Hi-Fi-Soundscape) in contrast to the social environment. This hi-fi
environment contributes significantly to an atmosphere conducive to
learning, which has been described with terms such as \enquote{library
concentration} (Bibliothekskonzentration) or \enquote{out-of-the-box
concentration} (Fansa, 2008) and is considered a unique selling point
of libraries. The soundscape concept enables a new analytical
perspective on these contexts, which can add value to discussions about
the physical learning location of the library as a whole. Finally, some
potential follow-up topics are outlined.
\end{abstract}

%body


\hypertarget{einleitung}{%
\section{Einleitung}\label{einleitung}}

„{[}E{]}s gibt so einen {[}\ldots{]} Geräuschkanon, der eine Bibliothek
auch ausmacht und attraktiv macht.''\footnote{Fansa (2008) 116.} So
äußert sich die Bibliotheksnutzerin einer wissenschaftlichen
Institutsbibliothek in einem der von Jonas Fansa durchgeführten und in
seiner Studie \enquote{Bibliotheksflirt} dokumentierten Interviews. Ein
zweiter befragter Nutzer beschreibt die „diffuse Geräuschkulisse'' in
der Bibliothek als lernförderlich, wenn er anmerkt: „Ich habe in
{[}\ldots{]} Bibliotheken durchaus schon die Erfahrung gemacht, dass
gerade -- {[}\ldots{]} wenn man so will -- das störende Rauschen, als
optisches und akustisches Rauschen, einen auch in einer Weise isoliert,
dass man in der Lage ist, sich zu konzentrieren.''\footnote{Fansa (2008)
  136.}

Diesem positiv besetzten, diffusen Hintergrundrauschen stehen oftmals
störende Geräusche in der Bibliothek gegenüber. Ob nun das Telefonieren
oder die Smartphonenutzung, das „Hacken'' auf der Tastatur, Musikhören
(mit und ohne Kopfhörer), Gespräche, Gruppenarbeit oder die Teilnahme an
Online-Veranstaltungen: Lärm -- so soll es Kurt Tucholsky gesagt haben
-- ist immer das Geräusch der Anderen und kann dadurch auch schnell zu
einem konfliktbehafteten Thema werden.\footnote{Zitiert nach Prayer
  (2003), S. 188. Zum Thema Lärm in Bibliotheken zuletzt unter anderem
  Hacker (2011); Niehoff (2017); Oestreich (2017); Thommen (2021);
  Yelinek und Bressler (2013); Aarts und Dijksterhuis (2003); Applegate
  (2009); Bell (2008); Franks und Asher (2014); Lange et\,al.~(2016);
  McCaffrey und Breen (2016). Jeweils mit weiterführender Literatur.}

Ein dritter Aspekt im Hinblick auf die Geräuschkulisse einer Bibliothek
ergibt sich aus dem bewussten Ausblenden nahezu sämtlicher Geräusche in
bestimmten Funktionsbereichen wie den Lesesälen. Die hier häufig
eingeforderte, absolute Stille kann für einen Teil der Nutzer*innen
konzentrationsfördernd wirken, von anderen aber gerade als unangenehm
empfunden werden. So erklärte Holger Schulze, Professor für Sound
Studies, in einem Gespräch mit Dirk Wissen dazu:

\begin{quote}
„Die eher sanften, unendlich kleinen und unvorhersehbar verknäulten
Klänge des Zeitungsraschelns erlebe ich als wohltuend. Leere Räume ohne
Geräusche, in dem jedes Kugelschreiberklicken oder Umblättern wie ein
Donnerhall verstärkt wird, solche Räume sind viel beklemmender und
belastender.''\footnote{Schulze und Wissen (2015).}
\end{quote}

In ganz ähnlicher Weise beschrieb die Historikerin Arlette Farge einmal
den Lesesaal eines Archivs:

\begin{quote}
„Die Stille eines Lesesaals ist gewaltsamer als jedes
Durcheinandergeschrei einer Schulklasse; wie in einer Kirchenandacht
schneidet sie gnadenlos das Blubbern und Gurgeln der Körper heraus,
isoliert es, macht sie ebenso aggressiv und bedrohlich wie
beängstigend.''\footnote{Farge (2018) 43.}
\end{quote}

Die positiv wie negativ erfahrenen Geräusche in der Bibliothek können
letztlich als Bestandteil eines eigentümlichen Sounds betrachtet werden,
der in der einen wie der anderen Weise eng mit der Wahrnehmung der
Institution verbunden ist. Gleichzeitig scheint die vorherrschende
Geräuschkulisse in Abhängigkeit von spezifischen Faktoren wie der
konkreten, individuellen, sinnlichen Wahrnehmung einer Person und ihrer
Erwartungshaltung gegenüber der Bibliothek und deren Funktionsbereiche
entweder lernförderliche oder lerneinschränkende Auswirkungen zu haben.
Obwohl daraus eine gewisse Bedeutung des Themas speziell für die
Diskussionen im Zusammenhang mit dem Bibliotheksbau, der
Lernraumentwicklung und speziell der Bibliothek als Lernort innerhalb
einer Hochschule abgeleitet werden kann, spielt es in den einschlägigen
Beiträgen bislang allenfalls eine marginale Rolle. Ausgenommen ist hier
freilich die Debatte um die zumeist praktischen Probleme im Umgang mit
Lärm und der entsprechenden Herrichtung der Akustik.\footnote{In Auswahl
  aus den jüngeren Beiträgen: Vergleiche Holländer et\,al. (2021);
  Günther et\,al.~(2019); Prill (2019); Eigenbrodt (2021); Werner
  (2021); Stang und Becker (2022); Stang (2016); siehe auch die
  Checkliste für den Bibliotheksbau der IFLA Library Buildings and
  Equipment Section (2016) 464.} In dieser verkürzten, pragmatischen
Sichtweise wird jedoch oft eine Behandlung des Phänomens in
theoretischer und konzeptioneller Hinsicht ausgeblendet.\footnote{Auf
  ein grundsätzliches Theoriedefizit im Zusammenhang der
  Lernortdiskussion hat Eigenbrodt bereits 2010 hingewiesen. Siehe
  Eigenbrodt (2010); freilich gab es in der Zwischenzeit Arbeiten, die
  diese Lücke füllen sollen. Allerdings ohne besondere Hervorhebung der
  Soundscape-Problematik. Siehe etwa Eigenbrodt (2021); Eigenbrodt und
  Stang (2014) sowie Anmerkung 6.} Dieses Theoriedefizit kann dazu
führen, dass man sich in den Debatten mit Unterhaltsträgern, in
Projektanträgen und in der Konkurrenz zu anderen Anbietern von
Lernräumen auf dem Campus wichtiger Alleinstellungsmerkmale des
Lernortes Bibliothek beraubt.

Das grundlegende Problem an dieser Stelle scheint darin zu bestehen,
dass aktuell kein Konzept zur Verfügung steht, das jenseits der
akustischen Herausforderung des Lärmproblems in der Lage ist, die
angedeutete Mehrdimensionalität in einer ganzheitlichen Perspektive zu
erfassen und für den bibliothekarischen Diskurs in der Praxis anwendbar
zu machen. Es fehlt mit anderen Worten an einem handhabbaren Werkzeug.
Die folgenden Ausführungen sollen vor diesem Hintergrund dazu dienen,
das Konzept der „Soundscape'', wie es in den sogenannten \emph{Sound
Studies} und in erster Linie von R. Murray Schafer formuliert worden
ist, vorzustellen und dessen potentielle Mehrwerte vor allem für den
Bereich des Lernortes Bibliothek zu skizzieren.\footnote{Einen Überblick
  zu den Sound Studies geben etwa Morat (2011); (2013); (2010); Meyer
  (2008); Schulze (2008); Hilmes (2005).} Der Beitrag versteht sich also
in erster Linie als Anregung für weitergehende Diskussionen zu diesem
Themenbereich.

\hypertarget{das-konzept-der-soundscape}{%
\section{Das Konzept der Soundscape}\label{das-konzept-der-soundscape}}

In seiner heutzutage zumeist genutzten Form geht der Begriff der
Soundscape zurück auf den kanadischen Komponisten und Klangforscher R.
Murray Schafer, der ihn in seiner zuerst 1977 erschienenen Studie „The
Tuning of the World'' in den wesentlichen Grundlagen
ausarbeitete.\footnote{Schafer (1977); Eine deutsche Neuübersetzung
  liegt ebenfalls vor. Siehe (2010).} Die Studie basiert dabei in großen
Teilen auf dem von ihm initiierten und von der UNESCO unterstützten
„World Soundscape Project'', das er mit seinem Forschungsteam Anfang der
1970er Jahre an der Simon-Fraser-Universität bei Vancouver ins Leben
gerufen hat. Aufgabe des Projektes sollte es sein, das akustische
Erscheinungsbild von Orten, Räumen, Landschaften und Situationen auf
Tonträgern festzuhalten, zu dokumentieren und in ihren Veränderungen
über Jahre hinweg zu verfolgen.\footnote{Einen ersten Überblick gibt die
  Website zum Projekt unter World Soundscape Project, URL:
  \url{https://www.sfu.ca/~truax/wsp.html} (27.4.2024).}

Der Begriff Soundscape ist ein Neologismus, der sich aus „Sound'' und
„Landscape'' zusammensetzt.\footnote{Eine entsprechende deutsche
  Übertragung wäre am ehesten „Klanglandschaft''. Da aber das englische
  „Sound'' einen größeren Bedeutungskontext aufweist, hat man in der
  deutschen Übersetzung von Schafers Werk den englischen Begriff
  beibehalten. Siehe die Erläuterung bei Schafer (2010) 14\,f.} In einem
Radiointerview Ende der 1990er Jahre beschrieb Schafer Soundscape mit
den Worten: „Die Landschaft mit den Ohren sehen.'' Eine etwas
ausführlichere Erläuterung lieferte Sabine Breitsameter in einem
einführenden Essay zur deutschen Neuübersetzung von Schafers Werk:

\begin{quote}
„Das Konzept Soundscape ist eine Hörgestalt, die in einer bestimmten
Wahrnehmungshaltung gründet: in der auditiven Aneignung der Gesamtheit
aller Schallereignisse eines Orts, Raums oder einer Landschaft, rundum
und vollständig, bis auf den leisesten Laut. Eine Soundscape ist also
die akustische Hülle, die den Menschen umgibt.''\footnote{Schafer (2010)
  15.}
\end{quote}

Ergänzend merkt Breitsameter an, dass es sich bei der Soundscape
gleichzeitig um eine Denkfigur handle, die das auditive Wahrnehmen
reformuliere, indem das „Rundum-Hören'' gegenüber einer frontalen
Rezeption gefordert würde. Die frontale Rezeption, wie sie in vielen
alltäglichen Situationen (stereofones Hören bei Radio, Fernseher,
Musikanlage; Bühnensituation oder Frontalunterricht) vorkomme, wird
dabei als Regelfall betrachtet.\footnote{Vergleiche Schafer (2010) 15.}

Ein wichtiger theoretischer Aspekt, der hier in den Begriffen
Wahrnehmung und Aneignung lediglich implizit thematisiert ist, bezieht
sich auf die Rolle der Hörenden. Diese sind in der vorliegenden
Vorstellung nicht mehr lediglich passive Empfänger*innen, sondern im
Sinne einer neueren kulturwissenschaftlichen Sichtweise der Wahrnehmung
und Aneignung vielmehr Akteur*innen und aktiv Einfluss nehmendes
Elemente ihrer Umwelt.\footnote{Zur Einführung in die Neuere
  Kulturwissenschaft siehe unter anderem Bachmann-Medick (2014); Jaeger
  (2004); Daniel (2006). Eine gute Einführung zum Konzept der Aneignung
  liefert zudem Füssel (2006).} „Wahrnehmen ist teilnehmen'', wie Sabine
Breitsameter schreibt.\footnote{Schafer (2010) 16.} Diese Sichtweise
verortet das Konzept Soundscape schließlich, wie viele andere
kulturwissenschaftlich und konstruktivistisch beeinflusste Konzepte der
jüngeren Vergangenheit, in einem Spannungsfeld zwischen ‚gemacht' und
‚machend'. In diesem Sinne hat Emily Thompson die Soundscape wie folgt
definiert:

\begin{quote}
„Like a landscape, a soundscape is simultaneously a physical environment
and a way of perceiving that environment; it is both a world and a
culture constructed to make sense of that world. The physical aspects of
a soundscape consist not only of the sounds themselves, the waves of
acoustical energy permeating the atmosphere in which people live, but
also the material objects that create, and sometimes destroy, those
sounds. {[}\ldots{]} A soundscape, like a landscape, ultimately has more
to do with civilization than with nature, and as such, it is constantly
under construction and always undergoing change.''\footnote{Thompson
  (2002) 1\,f.}
\end{quote}

Die Parallelen dieser Konzeptualisierung zu gegenwärtigen
raumtheoretischen Überlegungen sind offensichtlich und bieten
Anknüpfungspunkte für die Integration des Konzepts in die aktuellen
Lernraumdebatten. Für den Moment kann festgehalten werden, dass es sich
bei der Soundscape auf der einen Seite um die Gesamtheit aller
Schallereignisse eines Orts, Raums oder einer Landschaft handelt
einschließlich der materiellen Grundlagen, die zur Erzeugung, Leitung
oder Brechung vorhanden sind. Auf der anderen Seite steht freilich die
wahrgenommene Wirkung der konkreten Konfiguration einer Soundscape auf
das Individuum. Hier finden Prozesse der Aneignung und Interpretation
statt, die unter anderem durch Vorerfahrungen, Erwartungshaltungen,
Sozialisierung, kontextbezogene Verhaltensweisen und Normen beeinflusst
werden. Erst durch diese Leistung der Aneignung und Interpretation wird
die ‚an sich' bestehende Soundscape zur Realität einer Person. In genau
diesem Sinne werden dann aus den passiven Hörenden konkrete
Akteur*innen, welche die je ‚eigene' Soundscape im Moment des Hörens
gleichsam herstellen:

\begin{quote}
„Die Soundscape ist ein prägnantes Modell dafür, wie Raum durch den Akt
der Wahrnehmung geschaffen und durch den Hörer-Akteur performativ
erfahrbar gemacht wird. Dessen wahrnehmendes Handeln konstituiert den
Raum, der sich, ausgehend von seinen jeweiligen Akteuren, immer wieder
aufs Neue definieren kann. Dieser Raum ist, ebenso wie seine Grenzen,
fließend.''\footnote{Schafer (2010) 27.}
\end{quote}

\hypertarget{elemente-der-soundscape}{%
\section{Elemente der Soundscape}\label{elemente-der-soundscape}}

Zur konkreten Beschreibung einer Soundscape führt Schafer drei zentrale
Begriffe ein: 1.) Grundlaut, 2.) Signal, 3.) Lautmarke. Die Bedeutung
des Grundlauts entlehnt Schafer der Musiktheorie, in dem er den
Grundlaut als jenen Grundton definiert, den die anderen umspielen, der
aber nicht immer zwangsläufig auch bewusst gehört wird. Obwohl er
überhört werden könne, sei ein Ignorieren indes nicht möglich, weil das
Wahrnehmen von Grundlauten auch unwillkürlich zu Hörgewohnheiten führe.
Grundlaute seien allgegenwärtig und prägten daher die Stimmung und das
Verhalten von Menschen nachhaltig. Als Beispiel nennt Schafer den
Grundlaut einer Landschaft, der von ihrer Geografie, ihrem Klima, dem
Wind, den Wäldern, Wiesen, Vögeln, Insekten und Tieren hervorgebracht
werde.\footnote{Schafer (2010) 45\,f.}

Signale dagegen seien Vordergrundgeräusche, die bewusst gehört werden.
Da man im Grunde jedem Geräusch oder Klang bewusst zuhören kann, kann
letztlich alles zum Signal werden. Schafer merkt aber gleichzeitig an,
dass es spezifische Formen von Sound gibt, die explizit als Signal
wirken sollen. Dazu zählen zum Beispiel Warngeräusche wie Sirenen,
Hörner, Pfeifen oder Glocken.\footnote{Schafer (2010) 46.}

Abzuheben vom Signal ist noch einmal die Lautmarke. Darunter versteht
Schafer „Klänge und Geräusche einer Gemeinschaft, die einzigartig sind
oder Qualitäten aufweisen, aufgrund derer ihnen innerhalb der
Gemeinschaft eine besondere Bedeutung zukommt{[}.{]}``\footnote{Schafer
  (2010) 46.} Lautmarken seien geeignet, einer Gemeinschaft Identität zu
verleihen und sollten daher bestmöglich geschützt werden. Als ein
Beispiel für eine derartige Lautmarke wird die Glocke als sinn- und
identitätsstiftendes Element der christlichen Gemeinschaften genannt:

\begin{quote}
„Das markanteste akustische Signal der christlichen Gemeinschaft ist,
wie bereits erwähnt, die Kirchenglocke. Sie definiert die Gemeinde als
akustischen Raum, der durch die klangliche Reichweite der Glocke
abgegrenzt wird. {[}\ldots{]} Der Ruf der Glocke bringt die Gemeinde
zusammen, vereint sie im sozialen Sinne und führt den Menschen mit Gott
zusammen.''\footnote{Schafer (2010) 107; weitere Beispiele für den
  außergewöhnlichen Symbolgehalt von Glocken liefert Corbin (1995);
  siehe auch das Beispiel bei Corbin (1998) 124\,f.}
\end{quote}

Ein weiteres essentielles Element im Konzept der Soundscape ist für
Schafer die Unterscheidung zwischen einer „High Fidelity-'' (Hi-Fi) und
einer „Low Fidelity-Soundscape'' (Lo-Fi). Beide Begrifflichkeiten
beschreiben dabei das Verhältnis zwischen Signal und Rauschen, wobei
dieses in einer Hi-Fi-Umgebung günstiger sei, da sich die einzelnen
Laute nur selten überlappten und sich dadurch deutlich von den
Umgebungsgeräuschen abheben würden (hohe Klangtreue).\footnote{Dazu
  Schafer (2010) 91.} Demgegenüber würden in einer Lo-Fi-Soundscape die
einzelnen akustischen Signale von einer „übermäßig verdichteten
Anhäufung von Lauten''\footnote{Schafer (2010) 91.} überdeckt (geringe
Klangtreue). Die Folge sei, dass sich alle Laute durch- und miteinander
mischten. Damit man in diesem Kontext überhaupt noch einzelne Laute
wahrnehmen könne, müssten diese immer mehr verstärkt werden.\footnote{Schafer
  (2010) 92.}

Mit der Unterscheidung von Hi-Fi- und Lo-Fi-Soundscape verknüpft Schafer
in der Folge weitere Gegensatzpaare: So führt er eine räumliche
Dimension der Soundscape ein, indem er der Hi-Fi-Soundscape die
Möglichkeit zuschreibt, tief in die Ferne zu hören. Dadurch ergebe sich
eine deutlich wahrnehmbare akustische Perspektive von Vordergrund und
Hintergrund, die er in erster Linie auf dem Land verortet
sieht.\footnote{Schafer (2010) 91.} Im Gegensatz dazu gehe diese
Hörperspektive in der Lo-Fi-Umgebung der Stadt verloren:

\begin{quote}
„An der Straßenecke eines modernen Stadtzentrums gibt es keine Ferne;
dort gibt es nur unmittelbare Anwesenheit.''\footnote{Schafer (2010)
  91\,f.}
\end{quote}

Während dieser Gegensatz nachvollziehbar wirkt, wird sein zweites
Gegensatzpaar zuweilen deutlich kritischer betrachtet: So historisiert
Schafer den Gegensatz Hi-Fi / Lo-Fi, indem er die Entstehung einer
Lo-Fi-Klangumwelt an die Industrialisierung und die damit einhergehende
Urbanisierung knüpft. In dieser Dichotomie reproduziert sich bei Schafer
gleichsam der Gegensatz Natur / Kultur beziehungsweise Technik.
Beispiele für eine Zunahme der Geräusche und der Geräuschpegel seien der
Verbrennungsmotor, Industrie- und Haushaltsmaschinen, die Eisenbahn oder
der moderne Luftverkehr.\footnote{Siehe dazu Schafer (2010) 136--161;
  weitere Arbeiten, die in diese Richtung zielen, nennt Morat (2011)
  713--715; (2010) 4--7.} Einen vorläufigen Höhepunkt erreichte diese
Entwicklung nach Meinung Schafers mit der „Elektrischen Revolution'',
die unter anderem dazu geführt habe, dass ein Laut von seinem
ursprünglichen Entstehungskontext abgespalten und elektroakustisch
übertragen oder reproduziert werden konnte. Der Impetus von Schafers
Aktivitäten wird hier besonders deutlich, wenn er in diesem Zusammenhang
von einer „imperialistischen Klangüberschwemmung'' spricht und
resümiert, dass die Soundscape der Welt spätestens mit der Elektrischen
Revolution in einen anhaltenden Lo-Fi-Zustand abgeglitten
sei.\footnote{Vergleiche Schafer (2010) 162.}

Aufgrund der gesellschaftspolitischen Stoßrichtung wurde der ökologisch
geprägten Klangforschung mitunter eine normative \emph{bias}
unterstellt, die es im Rahmen einer Rezeption der Ansätze zu
berücksichtigen gilt.\footnote{Siehe dazu etwa den Forschungsüberblick
  bei Morat (2011) 712--715; (2013) 136--139.} Freilich muss den
Arbeiten von Schafer und anderen in diesem Zusammenhang zugutegehalten
werden, dass sie auf die gesellschaftlichen und gesundheitlichen
Auswirkungen einer Lärmbelastung und „akustischen Umweltverschmutzung''
aufmerksam gemacht haben. Gleichwohl erscheint gerade seine
Unterscheidung von Hi-Fi und Lo-Fi einen Ansatzpunkt zu liefern, der für
die bibliothekarischen Diskussionen rund um den Lernort weiterführend
ist.

\hypertarget{die-soundscape-bibliothek-als-klangliche-heterotopie}{%
\section{Die Soundscape Bibliothek als klangliche
Heterotopie}\label{die-soundscape-bibliothek-als-klangliche-heterotopie}}

Geht man von der Prämisse Schafers aus, wonach sich die moderne
Gesellschaft in einem anhaltenden Lo-Fi-Zustand befindet, dann kommt all
jenen Orten eine besondere Bedeutung zu, in denen dieser Zustand
ausgesetzt oder reduziert wird.\footnote{Christian Mikunda beschreibt
  derartige Orte unter anderem als „Oasen'', die „Verdünnung''
  auslösten, also den Prinzipien „langsamer, leiser, einfacher,
  entspannter, kühler'' folgten. Mikunda (2009) 231--247.} Im Folgenden
soll argumentiert werden, dass Bibliotheken solch einen Ort darstellen,
in dem sie der Lo-Fi-Umwelt der modernen Welt eine Art Hi-Fi-Umgebung
entgegensetzen. Sie bilden dadurch eine klangliche Heterotopie, die ein
charakteristisches Merkmal der Institution bildet und daher in den
Debatten um den Lernort Bibliothek entsprechend berücksichtigt werden
sollte. R. Murray Schafer selbst deutete bereits diese Stellung der
Bibliothek an, als er schrieb:

\begin{quote}
„Der Mensch erhielt sich Reservate der Stille in seinem Leben, um seinen
geistigen Stoffwechsel auszugleichen. Selbst in den Herzen der großen
Städte gab es die dunklen, stillen Gewölbe der Kirchen und Bibliotheken
{[}\ldots{]}.''\footnote{Schafer (2010) 408.}
\end{quote}

Michel Foucault hat für diese besonderen Orte innerhalb der Gesellschaft
das Konzept der Heterotopie formuliert.\footnote{Vergleiche Foucault
  (1992); (2021). Der Begriff ist ein Neologismus aus den Bezeichnungen
  „heteros'' (verschieden, anders) und „topos'' (Ort).} Foucault
versteht darunter einen „anderen Ort'', der aufgrund seiner
Beschaffenheit aus einem gegebenen gesellschaftlichen Zusammenhang
herausragt, weil er zum Beispiel bestehende Normen, gesellschaftliche
Praktiken oder selbst die ansonsten unhinterfragbaren Wahrheiten einer
Gemeinschaft transzendiert, bricht, aussetzt oder anderweitig verformt.
Es sind also Gegenorte, „in denen die alltäglichen Funktionen des
menschlichen Lebensraums außer Kraft gesetzt werden.''\footnote{Ruoff
  (2018) 193.} Im Gegensatz zur Utopie handelt es sich bei Heterotopien
nach Foucault um wirkliche und konkret lokalisierbare Orte. Dass
Bibliotheken als Heterotopie firmieren, ist dabei keine neue Erkenntnis.
Foucault selbst hat sie bereits als Beispiel genannt.\footnote{Siehe
  Foucault (2021) 16f; (1992) 43\,f.} Und auch in der Fachdiskussion
kommt es hin und wieder zu entsprechenden Herleitungen oder
Assoziationen.\footnote{Siehe etwa Knoche (2018) 116, allerdings ohne
  weitere Diskussion des Begriffs; Hobohm (2015); Gemmel und Vogt
  (2013); Clemens (2014); Stampfl (2019) 64\,f.} Die Fokussierung auf
den klanglichen Aspekt ihrer heterotopischen Eigenschaft ist indes etwas
Neues, das es zu plausibilisieren gilt.

Im Grunde kann die Eigenschaft als klangliche Heterotopie mindestens bis
zur Institutionalisierung der modernen Bibliotheken zurückverfolgt
werden. So hatte bereits Thomas Bodley (1544--1612), Namensgeber der
Bodleian Library, in den Statuten für die Nutzung der Bibliothek
festgehalten, dass die Nutzenden einen Eid zu schwören hatten. Darin
hieß es gleich zu Anfang:

\begin{quote}
„You shall Promise and Swear in the Presence of Almighty God, That
whensoever you shall repair to the Publick Library of this University,
you will conform your self to study with Modesty and
Silence''.\footnote{Bodley (1906) 96.}
\end{quote}

Und auch heutzutage sind Regelungen zum angemessenen Verhalten in
Bibliotheken in der Regel in den Benutzungsordnungen enthalten, darunter
häufig auch in der Form, dass die ruhige Arbeitsatmosphäre (primär im
Lesesaal) nicht gestört werden dürfe.\footnote{Siehe dazu etwa die
  Empfehlungen bei Hilpert et\,al.~(2014) 72.} Deutlicher tritt die
Bibliothek in ihrer Eigenschaft als klangliche Heterotopie allerdings
dann auf, wenn dieser Zustand gestört wird. So schrieb beispielsweise
Sallie Tisdale in einem kritischen Beitrag zur Lärmzunahme in
US-amerikanischen, öffentlichen Bibliotheken im Harper's Magazine vom
März 1997 über die Bibliotheken ihrer Jugend:

\begin{quote}
„This was a place set outside the ordinary day. Its silence --
outrageous, magic, unlike any other sound in my life -- was a
counterpoint to the interior noise in my crowded mind. It was the only
sacred space I knew, intimate and formal at once, hushed,
potent.''\footnote{Tisdale (1997) 65.}
\end{quote}

Steven Bell nahm später die Klage von Tisdale unter der Überschrift „The
Death of the Refuge'' in seine eigene Stellungnahme zum Lärmproblem in
Bibliotheken auf.\footnote{Bell (2008).}

Auch bei Jonas Fansa, in dessen Studie Bibliotheksflirt, wird der
heterotope Charakter der Bibliothek deutlich, wenn diese als etwas
beschrieben wird, in das man eintrete wie in „eine andere Welt''.
Folgerichtig beschreibt ein Nutzer den Eingangsbereich als
„Übergangszone zwischen der Außenwelt'' und der „Konzentrationswelt der
Bibliothek''.\footnote{Fansa (2008) 101.} Zur Erläuterung dieser
Konzentrationswelt kann folgende Aussage dienen:

\begin{quote}
„Es ist eine Atmosphäre, in der ich nicht ganz bewusst Konzentration
herstellen muss. Oder: Es ist ein Raum, in dem eben die Atmosphäre von
Konzentration schon gegeben ist.''\footnote{Fansa (2008) 121.}
\end{quote}

Fansa beschreibt dieses Phänomen abwechselnd mit einer spezifischen
„Bibliothekskonzentration'' oder mit einer
„out-of-the-Box-Konzentration'', die durch das atmosphärische Ensemble
der Bibliothek begünstigt werde und ein Alleinstellungsmerkmal der
Institution bilde.\footnote{Vergleiche Fansa (2008) 10, 36, 79. Er
  spricht sogar von einem Warenzeichen beziehungsweise einer
  Schutzmarke.}

In ganz ähnlicher Weise attestierte zuletzt André Schüller-Zwierlein in
seiner Studie „Fragilität des Zugangs'' den Bibliotheken eine „Muße- und
Konzentrationskompetenz''.\footnote{Siehe Schüller-Zwierlein (2022)
  379--387.} Die Voraussetzung dafür sei, dass Bibliotheken Räume
eigener Art darstellten, die das normale Leben mit seinen verschiedenen
Ablenkungen ausschließen und gerade deshalb für das konzentrierte
Arbeiten, Denken, Sinnieren prädestiniert seien. Charakteristisch für
diese Funktion von Bibliothek ist auch nach Schüller-Zwierlein die
vorherrschende Konfiguration der Soundscape:

\begin{quote}
„Die ausschließliche Anwesenheit Gleichgesinnter, die statische Stille
ihrer Bücherwände -- Bibliotheken sind der einzige Ort, wo wir
gemeinschaftlich still sind -- haben auch im elektronischen Zeitalter
die Funktion, Ablenkungen fernzuhalten.''\footnote{Schüller-Zwierlein
  (2022) 381.}
\end{quote}

Wichtig erscheint in diesem Zusammenhang, dass es sich bei dieser
‚gemeinschaftlichen Stille' nicht um die völlige Abwesenheit von
Geräuschen handelt. Diese Form von Stille wird laut Schafer vom Gros der
‚westlichen Bevölkerung' als eher negativ empfunden, weil der Mensch
sich nicht zuletzt über Klänge vergewissere, dass er nicht alleine
sei.\footnote{Genau dies spiegelt sich auch in den Interviews bei Fansa
  wider, wenn die verschiedenen Interviewpartner immer wieder betonen,
  dass es um „gemeinschaftliches Alleinsein'' ginge und die Anwesenheit
  anderer Personen stimulierend oder motivierend wirke. Siehe Fansa
  (2008) 32--36, 39 f, 43.} Die völlige Stille könne demnach in letzter
Instanz als Negation des Menschseins aufgefasst werden.\footnote{Schafer
  (2010) 411. Auch Niehoff bewertet die „akustische Leere'' negativ.
  Siehe Niehoff (2017) 31. Siehe dazu auch die Bemerkungen oben von
  Holger Schulze oder Arlette Farge zum Thema.} Dagegen deuten
Beschreibungen der Atmosphäre in Bibliotheken wie „knisternde
Ruhe''\footnote{Fansa (2008) 142.} oder „geräuschvolle
Stille''\footnote{Oestreich (2017).} eher darauf hin, dass es um die
Qualität der vorherrschenden Soundscape geht. Einen entsprechenden
Hinweis liefert etwa Sallie Tisdale, wenn sie ihre Vorstellung von
„Silence'' in der Bibliothek schildert:

\begin{quote}
„The silence I remember from my childhood library {[}\ldots{]} is the
thick, busy silence one sometimes finds in an operating room. It is
profoundly pleasing profoundly full. There used to be such silences in
many places, in open desert and in forests, in meadows and on
riverbanks, and something of this kind of silence was common, a century
or so ago, even in small towns, broken only by the unhurried sounds of
unhurried people. There is no such silence in the world now; in every
corner we live smothered by the shrill, growling, strident, piercing
racket of crowded, hurried lives. The street is noisy, stores and banks
and malls are noisy, classrooms are noisy, virtually every workplace is
noisy.''\footnote{Tisdale (1997) 74.}
\end{quote}

Auch Christine Niehoff sieht in der anzustrebenden „Stille'' der
Bibliothek eher eine Fülle niedriger und als beruhigend empfundener
Schallwellen, die sie im Konzept der Bibliothekskonzentration
verwirklicht sieht.\footnote{Vgl. Niehoff (2017) 31.} Sehr schön wird
der Unterschied zwischen einer lernförderlichen und störenden
Geräuschkulisse auch bei Lucia Hacker deutlich. In ihrer Arbeit zum
„Lärmort'' am Beispiel der Universitätbibliothek Erfurt hat sie unter
anderem die Geräuschpegel einzelner Bereiche erfasst und verglichen.
Eine interessante Beobachtung ergibt sich dabei aus dem Vergleich zweier
Bereiche mit annähernd gleichen Geräuschpegeln. So wurden für den
Garderobenbereich (55dB) und den sogenannten offenen Bereich (48dB), der
sich hinter dem Haupteingang befindet, nahezu gleiche Dezibelwerte
gemessen. In der Auswertung merkt die Autorin jedoch an:

\begin{quote}
„Für das eigene Empfinden waren diese Geräusche jedoch durchaus
unterschiedlich in ihrer ‚Qualität'. Während man die Geräuschkulisse im
Offenen Bereich eher als ein, teils durchaus anregendes
Hintergrundrumoren beschreiben könnte, ist es -- vor allem in den
Stoßzeiten -- im Garderobenbereich einfach nur \ldots{}
Lärm.''\footnote{Hacker (2011) 66. Auslassungen im Original.}
\end{quote}

Während es bislang an konkreten Möglichkeiten zur Beschreibung dieser
Phänomene fehlte, bietet das Konzept der Soundscape als klanglicher
Heterotopie ein mögliches sprachliches Instrumentarium an. Mit der
Unterscheidung einer Hi-Fi- zu einer Lo-Fi-Umgebung lassen sich die
Unterschiede zwischen störendem Lärm und anregendem Hintergrundrauschen
genauer fassen. Wichtig dabei ist, dass es nicht einfach darum geht, die
Lautstärke zu senken, sondern im Sinne des Hi-Fi-Konzeptes von Schafer
eine wohlgeordnete akustische Umgebung zu gestalten, in der die
übermäßig verdichtete Anhäufung von Lauten reduziert wird.\footnote{Dieser
  Gedanke korrespondiert auch mit einer Eigenschaft, die Foucault
  Heterotopien zuschreibt. Konkret heißt es bei ihm: „Oder man schafft
  einen anderen Raum, einen anderen wirklichen Raum, der so vollkommen,
  so sorgfältig, so wohlgeordnet ist wie der unsrige ungeordnet,
  mißraten und wirr ist.'' Foucault (1992) 45.} So würde eine positive
Immersionserfahrung wie die Bibliothekskonzentration möglich, die
stimulierend und lernförderlich ist -- ein „akustisches Rauschen,
{[}das{]} einen auch in einer Weise isoliert, dass man in der Lage ist,
sich zu konzentrieren.''\footnote{Fansa (2008) 136.}

\hypertarget{potentielle-anschlussthemen-fuxfcr-das-soundscape-konzept}{%
\section{Potentielle Anschlussthemen für das
Soundscape-Konzept}\label{potentielle-anschlussthemen-fuxfcr-das-soundscape-konzept}}

Bislang ist die Soundscape in ihrer Form als klangliche Heterotopie
lediglich in ihrer Beziehung zur ‚Außenwelt' dargestellt worden. Das ist
zwar entscheidend, wenn es um die Frage des Alleinstellungsmerkmals von
Bibliotheken geht. Allerdings darf dies nicht den Eindruck erwecken,
dass es nach innen eine einheitliche, monolithische Soundscape gebe. Im
Gegenteil nimmt ihr postulierter Doppelcharakter aus einer gegebenen
Situation und der letztlich wahrgenommenen akustischen Umwelt jüngere
Entwicklungen im Bereich des Lernortes auf. Das prägnanteste Beispiel
dafür sind die diversen Zonierungsprojekte und -initiativen, die in der
jüngeren Vergangenheit umgesetzt worden sind, um auf die sich
ausdifferenzierenden Lern- und Arbeitsszenarien zu reagieren.\footnote{Thommen
  (2021) 10\,f; Niehoff (2017) 31--39; Depping (2013) 103; Latimer
  (2007) 72\,f.} Dass die Betrachtung der jeweiligen Geräuschkulissen in
diesem Zusammenhang hilfreich sein kann, legt unter anderem die Studie
von Lucia Hacker nahe. Die von ihr erstellten Soundkarten sind ein
erster Schritt auf dem Weg zu einem empirisch gestützten Verständnis der
Soundscape Bibliothek. Gleichzeitig zeigen diese Karten, wie der
Bibliotheksraum in unterschiedliche akustische Räume zerfällt, die
jeweils eigenen Regeln folgen und dadurch auch spezifische
Geräuschkulissen erzeugen. Diese Räume bestehen im Sinne des
Soundscape-Ansatzes allerdings nicht einfach, sondern werden durch die
Praktiken der Anwesenden, also deren konkretes Lern- und
Arbeitsverhalten vor Ort, die Interaktion mit anderen Personen sowie dem
Mobiliar und technischen Equipment und auch der Bewegung der Körper im
Raum, immer wieder neu hergestellt und reproduziert. Aufgrund der
Vielzahl derartiger Räume und der damit verbundenen
Komplexitätssteigerung der jeweils geltenden Regeln und gewünschten
Verhaltensweisen (Aufhebung einer einheitlichen Norm), kann es sinnvoll
sein, über zusätzliche stabilisierende Maßnahmen dieser einzelnen Räume
nachzudenken.\footnote{Dieses Problem wird sehr schön geschildert bei
  Depping (2013).} Das könnten etwa Policies sein, die unter anderem die
konkreten Regeln und Normen für einen bestimmten Bereich wiedergeben,
idealerweise nicht nur als Pflicht für die Nutzer*innen, sondern auch
als Verpflichtung für die Bibliothek und deren
Mitarbeiter*innen.\footnote{Depping (2013); Niehoff (2017) 60; McCaffrey
  und Breen (2016) 784\,f, 788; generell zum Einsatz von Policies für
  die „Kundenkommunikation'' in Bibliotheken auch Georgy (2011).}

Die Erweiterung des Verständnisses der Soundscape durch empirische
Untersuchungen ist ein eigenes zukünftiges Betätigungsfeld. Die genauere
Bestimmung der vorherrschenden Grundlaute, der als positiv und negativ
empfundenen Signale und eventuell identitätsstiftender Lautmarken in
Verbindung mit systematischen Messungen können als Grundlage eines
fundierteren Lärmmanagements der Bibliotheken dienen. So gibt es zwar
bereits jetzt eine Reihe von bekannten Lärmquellen, die aber noch zu
selten kategorisiert und hierarchisiert werden. Auf der Grundlage von
Schafers Hi-Fi-/Lo-Fi-Unterscheidung kann es aber durchaus sinnvoll
sein, in der Problemanalyse und Problembearbeitung die störenden
Geräusche durch technische Gerätschaften wie Kopierer, Drucker,
Computer/Laptops oder Klimaanlagen von jenen durch Menschen verursachte
(Gespräche, Lachen, Bewegung im Raum et cetera) deutlicher zu
unterscheiden.\footnote{Siehe exemplarisch Schafers
  Klassifizierungsschema Schafer (2010) 234--240.} Nach Schafer weisen
nämlich gerade Maschinen und Geräte eine „künstlich-statische
Geräuschlinie'' auf und erzeugen damit in der Regel „Laute von geringem
Informationsgrad und hoher Redundanz''\footnote{Schafer (2010) 146.},
wodurch sie sich von den Menschen gemachten unterscheiden.

Abschließend sei auf einen weiteren Aspekt der Thematik verwiesen.
Konflikte um Lärm in Bibliotheken sind ein bekanntes und häufig
anzutreffendes Thema. Steven Bell wählt in seinem Beitrag zum „Noise
Management'' nicht ganz zu Unrecht eine drastische Sprache, um das
Problem zu beschreiben. So spricht er unter anderem von „the library's
battle royal over noise'' oder vom „battlefield'' und vergleicht die
Situation mit einem „Wild West frontier town saloon''.\footnote{Siehe
  Bell (2008).} Auch wenn andere Studien und Beiträge eine weniger
nachdrückliche Sprache benutzen, bestätigen sie doch alle den
grundsätzlichen Befund: Es wird lauter in Bibliotheken und das führt zu
Konflikten. Interessanterweise wird jedoch selten jenseits der
praktischen Probleme und Lösungsversuche nach einer symbolischen
Bedeutungsebene dieser Konflikte gefragt.\footnote{Vergleiche zum
  Symbolgehalt auch die Ausführungen bei Schafer (2010) 279--296.}

Monika Dommann zeigte in diesem Rahmen sehr schön, dass sich im Kampf um
Lärm oftmals grundsätzliche asymmetrische Gegensatzpaare\footnote{Klassisch
  dazu Koselleck (1989).} reproduzieren, weil Lärm am Ende nicht
gemessen werden kann, sondern dann entsteht, wenn Menschen Geräusche als
Störung wahrnehmen.\footnote{Siehe Dommann (2006).} Lärm ist somit
zuallererst ein soziales Konstrukt, das zwischen unterschiedlichen
Akteuren ausgehandelt wird.

So protestierten und agitierten etwa im ausgehenden 19. und frühen 20.
Jahrhundert Vertreter des Bürgertums gegen den zunehmenden Lärm auf den
Straßen der Städte. Lärm wird hier primär der Stadt, der Masse, den
Ungebildeten, den Unmündigen und Unnützen zugeschrieben, also den
„kreischenden Strassenverkäufern'', den „peitschenknallenden Knechten'',
den „trommelnden Kindern'' oder den „musizierenden
Almosensammlern''.\footnote{Dommann (2006) 133.} Bei Schopenhauer nimmt
das Ganze die Form eines Klassen- und Kulturkampfes an, wenn er
schreibt: „Lärm ist eine Waffe der Handarbeiter im Krieg mit den
Kopfarbeitern.''\footnote{Zitiert nach Dommann (2006) 134.} Und Peter
Bailey erinnert in einem Beitrag daran, dass die Entscheidung, wann
etwas als Lärm gilt, und wo und wann Ruhe zu herrschen habe, letztlich
eine Machtfrage sei. Das bewusste Unterlaufen solcher Ruhegebote
bedeutet somit auch stets einen Akt der Devianz oder Subversion und
damit eine Störung der bestehenden Ordnung.\footnote{Siehe Bailey
  (1996). Dies kann freilich auch bewusst instrumentalisiert werden, wie
  die Aktion im November 2023 zeigt, als über versteckte Bluetooth-Boxen
  verschiedene politische Botschaften in den Lesesaal der
  Staatsbibliothek zu Berlin ausgestrahlt worden sind.}

Angesichts dessen bekommen womöglich die Beschwerden älterer
Studierender über das zu laute und störende Verhalten „jüngerer
Studierender'', vor allem der Erstsemester, wie es etwa in der Studie
von Lucia Hacker dokumentiert ist, eine zusätzliche
Bedeutung.\footnote{Hacker (2011) 29.} Eklatant wird das Ganze, wenn ein
Interviewpartner von Hacker dazu erläutert, dass diese Studierenden
offenbar noch glaubten, der Wunsch nach Ruhe komme von „oben''. Ihnen
fehle also noch das Verständnis dafür, dass sich die (älteren)
Studierenden selbst durch das nichtangepasste Verhalten gestört fühlen
könnten. Entscheidend ist, dass er das Problem am Ende durch eine
fehlende Erfahrung beziehungsweise „Sozialisierung'' erklärt, die diese
Neuen erst selbst (durch-)machen müssten. Die störenden ‚Erstsemester'
erscheinen somit als unwissende, ungebildete Masse, die erst noch mit
den Spielregeln vertraut gemacht werden müsse. Noch deutlicher wird das
am Beispiel störender Teenagergruppen, die offenbar eine Art
Schnitzeljagd durch die Bibliothek machten:

\begin{quote}
„Dabei rasten sie zum Teil wie bei einer Schnitzeljagd durch die Etagen,
diskutierten in Vierergrüppchen lautstark über ihre Fortschritte,
lachten und alberten. Ein schönes Beispiel dafür, dass
‚Bibliothekskultur' etwas ist, das man erst erlernen muss. Eine große
und kontinuierliche Aufgabe für die ‚Teaching Library'.''\footnote{Hacker
  (2011) 91.}
\end{quote}

Die Frage nach der symbolischen Bedeutungsebene der Lärmprobleme kann
somit spezifische Identitätskonstruktionen offenbaren, die, Nutzer*innen
zum Bestandteil einer Bibliothekskultur machen und diese Identität von
anderen abgrenzt. Gleichzeitig werden Erwartungshaltungen kommuniziert,
die sich in diesem Beispiel konkret an die Bibliothek als
sozialdisziplinierende oder sozialisierende Institution richten. Weitere
Konfliktlagen sind in diesem Zusammenhang durchaus denkbar. Besonders
interessant könnten bewusst deviante Verhaltensweisen sein, die in
dieser Form eventuell auf Defizite oder Mängel in der Lernraumgestaltung
hindeuten. So können die Konflikte um den Lärm auch positiv gewendet und
zu einem Innovationsmotor werden.

Eine interessante Frageperspektive im Zusammenhang mit der
identifikatorischen Wirkung der Soundscape ergibt sich aus dem Vorbild
des Spotify-Kanals der UB Leipzig. Eingerichtet zu Pandemiezeiten, wirbt
die Bibliothek damit, den authentischen Lesesaal-Soundtrack zu den
Nutzer*innen nach Hause zu bringen. Das Angebot ist seit seiner
Einführung 2021 nach Angaben der Bibliothek gewachsen.\footnote{Vergleiche
  den Blog-Beitrag von Pichlmair (2023).} An dieser Stelle muss man sich
fragen, ob die spezifische Geräuschkulisse der Bibliothek bereits zu
einem eigenständigen Gut geworden ist, das sogar außerhalb des
physischen Ortes einen eindeutigen Wiedererkennungswert hat. Oder
funktioniert dies nur dann, wenn man als Nutzer*in bereits die
spezifische Immersionserfahrung des Arbeitens und Lernens im physischen
Ort gemacht hat?

\hypertarget{fazit}{%
\section{Fazit}\label{fazit}}

Das Konzept der Soundscape scheint geeignet, die bisherigen Erkenntnisse
zum Thema der „Wohlfühlatmosphäre'' beziehungsweise der
„Bibliothekskonzentration'' in Bibliotheken zu bereichern und zu
präzisieren, indem es den Fokus auf einen bislang vernachlässigten
Aspekt der Lernort-Debatte wirft. Es stellt die theoretische Grundierung
für ein in der Praxis zu beobachtendes Phänomen dar und hilft so, eine
Kontextualisierung vorzunehmen. Das Konzept der Soundscape avanciert
damit zu einem analytischen Werkzeug, das vielfältig eingesetzt werden
kann: als Frageperspektive für weitere Studien, als These zur
Plausibilisierung oder Falsifizierung, als Sammelbegriff für positive
und negative Effekte der genuinen Geräuschkulisse in Bibliotheken oder als
Anknüpfungspunkt für die Integration in andere Diskussionszusammenhänge
wie etwa Lernraumentwicklungsdebatten.

Der Blick auf die Soundscape und deren Funktion als klangliche
Heterotopie unterstreicht am Ende die große Bedeutung von Bibliotheken
als physische Orte. Bibliotheken bieten mit ihrem spezifischen „Sound of
Silence'' einen außergewöhnlichen und offensichtlich gesellschaftlich
wichtigen Service an und sollten dieses Alleinstellungsmerkmal schützen,
fördern und ausbauen.

\hypertarget{literatur}{%
\section{Literatur}\label{literatur}}

Aarts, Henk; Dijksterhuis, Ap (2003): The silence of the library:
Environment, situational norm, and social behavior. In: \emph{Journal of
Personality and Social Psychology}, 84 (1), 18--28. DOI:
\url{https://doi.org/10.1037/0022-3514.84.1.18}.

Applegate, Rachel (2009): The Library Is for Studying. In: \emph{The
Journal of Academic Librarianship}, 35 (4), 341--46. DOI:
\url{https://doi.org/10.1016/j.acalib.2009.04.004}.

Bachmann-Medick, Doris (2014): Cultural Turns. Neuorientierungen in den
Kulturwissenschaften. 5. Aufl. Reinbek: Rowohlt (Rowohlts Enzyklopädie:
55675).

Bailey, Peter (1996): Breaking the Sound Barrier: A Historian Listens to
Noise. In: \emph{Body \& Society}, 2 (2), 49--66. DOI:
\url{https://doi.org/10.1177/1357034X96002002003}.

Bell, Steven J. (2008): Stop Having Fun and Start Being Quiet: Noise
Management in the Academic Library. In: \emph{Library Issues}, 28 (4).

Bodley, Thomas (1906): The Life of Sir Thomas Bodley, written by
himself, together with the first draft of the statutes of the public
library at Oxon. Chicago: A. C. McClurg \& co.

Clemens, Manuel (2014): Review: Die Bibliothek als Heterotopie. In:
\emph{KulturPoetik}, 14 (1), 138--140. DOI:
\url{https://doi.org/10.13109/kult.2014.14.1.138}.

Corbin, Alain (1995): Die Sprache der Glocken. Ländliche Gefühlskultur
und symbolische Ordnung im Frankreich des 19. Jahrhunderts. Frankfurt am
Main: S. Fischer.

Corbin, Alain (1998): Zur Geschichte und Anthropologie der
Sinneswahrnehmung. In: \emph{Kultur \& Geschichte. Neue Einblicke in
eine alte Beziehung}, hg. von Christoph Conrad und Martina Kessel,
121--140. Stuttgart: Reclam (Universal-Bibliothek: 9638).

Daniel, Ute (2006): Kompendium Kulturgeschichte: Theorien, Praxis,
Schlüsselwörter. 5., durchges. und aktual. Aufl. Frankfurt am Main:
Suhrkamp (Suhrkamp-Taschenbuch Wissenschaft: 1523).

Depping, Ralf (2013): Können Bibliotheksbau und -ausstattung
verhaltenssteuernd wirken? Ein Beitrag zur Architekturpsychologie in
Bibliotheken. In: \emph{B.i.T.Online}, 16 (2), 103--114.
\url{https://www.b-i-t-online.de/heft/2013-02/fachbeitrag-depping.pdf}.

Dommann, Monika (2006): Antiphon: Zur Resonanz des Lärms in der
Geschichte. In: \emph{Historische Anthropologie}, 14 (1), 133--146. DOI:
\url{https://doi.org/10.7788/ha.2006.14.1.133}.

Eigenbrodt, Olaf (2010): Definition und Konzeption der
Hochschulbibliothek als Lernort. In: \emph{ABI-Technik}, 30 (4),
252--260. DOI: \url{https://doi.org/10.1515/ABITECH.2010.30.4.252}.

Eigenbrodt, Olaf (2021): Lernwelt Wissenschaftliche Bibliothek.
Pädagogische und raumtheoretische Facetten. Berlin\,; Boston: De Gruyter
(Lernwelten).

Eigenbrodt, Olaf; Stang, Richard (Hrsg.) (2014): Formierungen von
Wissensräumen. Optionen des Zugangs zu Information und Bildung. De
Gruyter.

Fansa, Jonas (2008): Bibliotheksflirt. Bibliothek als öffentlicher Raum.
1. Aufl. Bad Honnef: Bock + Herchen. DOI:
\url{https://doi.org/10.18452/13444}.

Farge, Arlette (2018): Der Geschmack des Archivs. 2. Aufl. Göttingen:
Wallerstein Verlag.

Foucault, Michel (1992): Andere Räume. In: \emph{Aisthesis. Wahrnehmung
heute oder Perspektiven einer anderen Ästhetik}, hg. von Karlheinz
Barck, Peter Gente, Heidi Paris und Stefan Richter, 34--46. Leipzig:
Reclam.

Foucault, Michel (2021): Die Heterotopien. Der utopische Körper. Zwei
Radiovorträge. 5. Aufl. Frankfurt am Main: Suhrkamp (Suhrkamp
Taschenbuch Wissenschaft: 2071).

Franks, Janet E.; Asher, Darla C. (2014): Noise Management in
Twenty-First Century Libraries: Case Studies of Four U.S. Academic
Institutions. In: \emph{New Review of Academic Librarianship}, 20 (3),
320--331. DOI: \url{https://doi.org/10.1080/13614533.2014.891528}.

Füssel, Marian (2006): Die Kunst der Schwachen. Zum Begriff der
„Aneignung'' in der Geschichtswissenschaft. In:
\emph{Sozial.Geschichte}, 21, 7--28.

Gemmel, Mirko; Vogt, Margrit (Hrsg.) (2013): Wissensräume. Bibliotheken
in der Literatur. Berlin: Ripperger \& Kremers.

Georgy, Ursula (2011): Benutzerordnungen als Marketinginstrument in
Bibliotheken. In: \emph{Bibliothek Forschung und Praxis}, 35 (1),
100--108. DOI: \url{https://doi.org/10.1515/bfup.2011.013}.

Günther, Dorit; Kirschbaum, Marc; Kruse, Rolf; Ladwig, Tina; Prill,
Anne; Stang, Richard; Wertz, Inka (2019): Zukunftsfähige
Lernraumgestaltung im digitalen Zeitalter. Thesen und Empfehlungen der
Ad-hoc Arbeitsgruppe Lernarchitekturen des Hochschulforum
Digitalisierung. Berlin: Hochschulforum Digitalisierung (Arbeitspapier:
44).

Hacker, Lucia (2011): „Lärmort'' Bibliothek? Der Lern- und
Kommunikationsort Bibliothek im Spannungsfeld unterschiedlicher
Nutzerbedürfnisse am Beispiel der Universitätsbibliothek Erfurt. Berlin
(Berliner Handreichungen zur Bibliotheks- und Informationswissenschaft:
310). DOI: \url{https://doi.org/10.18452/2052}.

Hilmes, Michele (2005): Is There a Field Called Sound Culture Studies?
And Does It Matter? In: \emph{American Quarterly}, 57 (1), 249--259.

Hilpert, Wilhelm; Gillitzer, Bertold; Kuttner, Sven; Schwarz, Stephan
(2014): Regeln für die Benutzung einer Bibliothek. In:
\emph{Benutzungsdienste in Bibliotheken: Bestands- und
Informationsvermittlung}, 67--76. Berlin\,; Boston: De Gruyter Saur
(Bibliotheks- und Informationspraxis: Band 52).

Hobohm, Hans-Christoph (2015): Vom Ort zum Akteur. Heterotopologie +
Akteur-Network-Theorie auf die Bibliothek bezogen. In: \emph{LIBREAS.
Library Ideas}, 28. Verfügbar unter
\url{http://libreas.eu/ausgabe28/06hobohm/}.

Holländer, Stephan; Sühl-Strohmenger, Wilfried; Syré, Ludger (2021):
Hochschulbibliotheken auf dem Weg zu Lernzentren. Beispiele aus
Deutschland, Österreich und der Schweiz. Wiesbaden: b.i.t.verlag Gmbh.

IFLA Library Buildings and Equipment Section (2016): Bibliotheksgebäude
auf dem Prüfstand. Kennzeichen, Betrieb und Evaluation -- ein
Fragenkatalog. In: \emph{Praxishandbuch Bibliotheksbau. Planung --
Gestaltung -- Betrieb}, hg. von Petra Hauke und Klaus Ulrich Werner,
459--471. Berlin\,; Boston: De Gruyter Saur.

Jaeger, Friedrich (Hrsg.) (2004): Handbuch der Kulturwissenschaften. 3
Bde. Stuttgart Weimar: Metzler.

Knoche, Michael (2018): Die Idee der Bibliothek und ihre Zukunft. 2.
Aufl. Göttingen: Wallstein Verlag.

Koselleck, Reinhart (1989): Zur historisch-politischen Semantik
asymmetrischer Gegenbegriffe. In: \emph{Vergangene Zukunft. Zur Semantik
geschichtlicher Zeiten}, hg. von Reinhart Koselleck, 211--259. Frankfurt
am Main: Suhrkamp (Suhrkamp-Taschenbuch Wissenschaft: 757).

Lange, Jessica; Miller-Nesbitt, Andrea; Severson, Sarah (2016): Reducing
noise in the academic library: the effectiveness of installing noise
meters. In: \emph{Library Hi Tech}, 34 (1), 45--63. DOI:
\url{https://doi.org/10.1108/LHT-04-2015-0034}.

Latimer, Karen (2007): Users and Public Space: What to Consider When
Planning Library Space. In: \emph{IFLA Library Building Guidelines:
Developments \& Reflections}, hg. von Karen Latimer und Hellen Niegaard,
68--82. München: Saur.

McCaffrey, Ciara; Breen, Michelle (2016): Quiet in the Library: An
Evidence-Based Approach to Improving the Student Experience. In:
\emph{portal: Libraries and the Academy}, 16 (4), 775--91. DOI:
\url{https://doi.org/10.1353/pla.2016.0052}.

Meyer, Petra Maria (Hrsg.) (2008): Acoustic turn. München: W. Fink.

Mikunda, Christian (2009): Warum wir uns Gefühle kaufen. Die 7
Hochgefühle und wie man sie weckt. Berlin: Econ.

Morat, Daniel (2010): Sound Studies -- Sound Histories. Zur Frage nach
dem Klang in der Geschichtswissenschaft und der Geschichte in der
Klangwissenschaft. In: \emph{Kunsttexte}, 4 (2010). DOI:
\url{https://doi.org/10.48633/ksttx.2010.4.87900}.

Morat, Daniel (2011): Zur Geschichte des Hörens. Ein Forschungsbericht.
In: \emph{Archiv für Sozialgeschichte}, 51, 695--716.

Morat, Daniel (2013): Zur Historizität des Hörens: Ansätze für eine
Geschichte auditiver Kulturen. In: \emph{Auditive Medienkulturen}, hg.
von Axel Volmar und Jens Schröter, 131--144. Bielefeld: transcript
Verlag.

Niehoff, Christine (2017): Der Stille auf der Spur. An exploration of
quiet study spaces in German and British university libraries.
Wiesbaden: Verlag: Dinges \& Frick (b.i.t. online Innovativ: Band 65).

Oestreich, Raimar (2017): Geräuschvolle Stille. Schallmaskierung in der
Bibliothek als akustischer Kompromiss für gegensätzliche
Nutzungsbedürfnisse. Berlin (Berliner Handreichungen zur Bibliotheks-
und Informationswissenschaft: 416). DOI:
\url{https://doi.org/10.18452/2159}.

Pichlmair, Veronika (2023): Authentischer Lesesaal-Soundtrack.
\emph{Blog der UB Leipzig}. Verfügbar unter
\url{https://blog.ub.uni-leipzig.de/authentischer-lesesaal-soundtrack/},
zugegriffen am 22.02.2024.

Prayer, Peter (2003): Vom Geräusch zum Lärm. Zur Geschichte des Hörens
im 19. und frühen 20. Jahrhundert. In: \emph{Sinne und Erfahrung in der
Geschichte}, hg. von Wolfram Aichinger, Franz X. Eder und Claudia
Leitner, 173--191. Innsbruck / Wien / München / Bozen: Studien-Verl
(Querschnitte: 13).

Prill, Anne (2019): Lernräume der Zukunft. Vier Praxisbeispiele zu
Lernraumgestaltung im digitalen Wandel. Berlin: Hochschulforum
Digitalisierung (Arbeitspapier: 45).

Ruoff, Michael (2018): Foucault-Lexikon. Entwicklung - Kernbegriffe -
Zusammenhänge. 4., aktual. und erw. Aufl. Paderborn: Wilhelm Fink.

Schafer, R. Murray (1977): The tuning of the world. 1st ed.~New York: A.
A. Knopf.

Schafer, R. Murray (2010): Die Ordnung der Klänge. Eine Kulturgeschichte
des Hörens. Mainz: Schott.

Schüller-Zwierlein, André (2022): Die Fragilität des Zugangs. Eine
Kritik der Informationsgesellschaft. Berlin\,; Boston: De Gruyter Saur
(Age of access? Grundfragen der Informationsgesellschaft: 14). DOI:
\url{https://doi.org/10.1515/9783110735796}.

Schulze, Holger (Hrsg.) (2008): Sound Studies: Traditionen, Methoden,
Desiderate. Eine Einführung. Bielefeld: transcript (Sound studies: 1).

Schulze, Holger; Wissen, Dirk (2015): Der Klang, das Geräusch, der gute
Ton. In: \emph{BuB - Forum Bibliothek und Information}, Wissen
fragt...?, , 67 (12), 744.
\url{https://nbn-resolving.org/urn:nbn:de:urmel-1ce6f2ac-92f4-4a5d-b843-dc42f89cf82b7-00304739-13}.

Stampfl, Nora S. (2019): Zwischen Realität und Virtualität. Zur
Verortung Öffentlicher Bibliotheken. In: \emph{Öffentliche Bibliothek
2030. Herausforderungen - Konzepte - Visionen}, hg. von Petra Hauke,
61--67. Bad Honnef: Bock + Herchen. DOI:
\url{https://doi.org/10.18452/19927}.

Stang, Richard (2016): Lernwelten im Wandel. Entwicklungen und
Anforderungen bei der Gestaltung zukünftiger Lernumgebungen. De Gruyter.
DOI: \url{https://doi.org/10.1515/9783110379471}.

Stang, Richard; Becker, Alexandra (Hrsg.) (2022): Lernwelt Hochschule
2030. Konzepte und Strategien für eine zukünftige Entwicklung. De
Gruyter. DOI: \url{https://doi.org/10.1515/9783110729221}.

Thommen, Rachel Noëmi (2021): Lärmmanagement an Deutschschweizer
Hochschulbibliotheken. Chur (Churer Schriften zur
Informationswissenschaft: 141).
\url{https://nbn-resolving.org/urn:nbn:de:101:1-2022042608353580974098}.

Thompson, Emily Ann (2002): The soundscape of modernity. Architectural
acoustics and the culture of listening in America, 1900-1933. Cambridge,
Mass: MIT Press.

Tisdale, Sallie (1997): Silence, Please. The public library as
entertainment center. In: \emph{Harper's Magazine}, 294 (März), 65--74.

Werner, Klaus Ulrich (Hrsg.) (2021): Bibliotheken als Orte kuratorischer
Praxis. Berlin\,; Boston: De Gruyter Saur (Bibliotheks- und
Informationspraxis: 67). DOI:
\url{https://doi.org/10.1515/9783110673722}.

World Soundscape Project. Verfügbar unter
\url{https://www.sfu.ca/~truax/wsp.html}, zugegriffen am 27.04.2024.

Yelinek, Kathryn; Bressler, Darla (2013): The Perfect Storm: A Review of
the Literature on Increased Noise Levels in Academic Libraries. In:
\emph{College \& Undergraduate Libraries}, 20 (1), 40--51. DOI:
\url{https://doi.org/10.1080/10691316.2013.761095}.

%autor
\begin{center}\rule{0.5\linewidth}{0.5pt}\end{center}

\textbf{Boris Queckbörner} hat Geschichte und Politikwissenschaft
studiert. Nach mehreren Jahren in der Forschung und Lehre im Bereich
Geschichte an den Universitäten Marburg und Kassel hat er in Berlin den
Master in Bibliotheks- und Informationswissenschaft erworben. Er hat in
verschiedenen Positionen an der Niedersächsischen Staats- und
Universitätsbibliothek in Göttingen gearbeitet und war dort zuletzt
stellvertretender Abteilungsleiter der Benutzungsabteilung. Aktuell ist
er Leiter der Hochschulbibliothek der Hochschule für Technik und
Wirtschaft in Dresden. Sein besonderes Interesse gilt Fragen der
Lernraumgestaltung sowie der Organisationsentwicklung.

ORCID: \url{https://orcid.org/0000-0003-4025-5692}

ROR: \url{https://ror.org/05q5pk319}

Kontakt:
\href{mailto:boris.queckboerner@htw-dresden.de}{\nolinkurl{boris.queckboerner@htw-dresden.de}}

\end{document}