\documentclass[a4paper,
fontsize=11pt,
%headings=small,
oneside,
numbers=noperiodatend,
parskip=half-,
bibliography=totoc,
final
]{scrartcl}

\usepackage[babel]{csquotes}
\usepackage{synttree}
\usepackage{graphicx}
\setkeys{Gin}{width=.4\textwidth} %default pics size

\graphicspath{{./plots/}}
\usepackage[ngerman]{babel}
\usepackage[T1]{fontenc}
%\usepackage{amsmath}
\usepackage[utf8x]{inputenc}
\usepackage [hyphens]{url}
\usepackage{booktabs} 
\usepackage[left=2.4cm,right=2.4cm,top=2.3cm,bottom=2cm,includeheadfoot]{geometry}
\usepackage[labelformat=empty]{caption} % option 'labelformat=empty]' to surpress adding "Abbildung 1:" or "Figure 1" before each caption / use parameter '\captionsetup{labelformat=empty}' instead to change this for just one caption
\usepackage{eurosym}
\usepackage{multirow}
\usepackage[ngerman]{varioref}
\setcapindent{1em}
\renewcommand{\labelitemi}{--}
\usepackage{paralist}
\usepackage{pdfpages}
\usepackage{lscape}
\usepackage{float}
\usepackage{acronym}
\usepackage{eurosym}
\usepackage{longtable,lscape}
\usepackage{mathpazo}
\usepackage[normalem]{ulem} %emphasize weiterhin kursiv
\usepackage[flushmargin,ragged]{footmisc} % left align footnote
\usepackage{ccicons} 
\setcapindent{0pt} % no indentation in captions
\usepackage{xurl} % Breaks URLs

%%%% fancy LIBREAS URL color 
\usepackage{xcolor}
\definecolor{libreas}{RGB}{112,0,0}

\usepackage{listings}

\urlstyle{same}  % don't use monospace font for urls

\usepackage[fleqn]{amsmath}

%adjust fontsize for part

\usepackage{sectsty}
\partfont{\large}

%Das BibTeX-Zeichen mit \BibTeX setzen:
\def\symbol#1{\char #1\relax}
\def\bsl{{\tt\symbol{'134}}}
\def\BibTeX{{\rm B\kern-.05em{\sc i\kern-.025em b}\kern-.08em
    T\kern-.1667em\lower.7ex\hbox{E}\kern-.125emX}}

\usepackage{fancyhdr}
\fancyhf{}
\pagestyle{fancyplain}
\fancyhead[R]{\thepage}

% make sure bookmarks are created eventough sections are not numbered!
% uncommend if sections are numbered (bookmarks created by default)
\makeatletter
\renewcommand\@seccntformat[1]{}
\makeatother

% typo setup
\clubpenalty = 10000
\widowpenalty = 10000
\displaywidowpenalty = 10000

\usepackage{hyperxmp}
\usepackage[colorlinks, linkcolor=black,citecolor=black, urlcolor=libreas,
breaklinks= true,bookmarks=true,bookmarksopen=true]{hyperref}
\usepackage{breakurl}

%meta
%meta

\fancyhead[L]{J. Bemme, C. Förster, J. Flade\\ %author
LIBREAS. Library Ideas, 45 (2024). % journal, issue, volume.
\href{https://doi.org/10.18452/...}{\color{black}https://doi.org/10.18452/...}
{}} % doi 
\fancyhead[R]{\thepage} %page number
\fancyfoot[L] {\ccLogo \ccAttribution\ \href{https://creativecommons.org/licenses/by/4.0/}{\color{black}Creative Commons BY 4.0}}  %licence
\fancyfoot[R] {ISSN: 1860-7950}

\title{\LARGE{The Sound of Gesprochene Wikisource}}% title
\author{Jens Bemme, Caroline Förster, Juliane Flade} % author

\setcounter{page}{1}

\hypersetup{%
      pdftitle={The Sound of Gesprochene Wikisource},
      pdfauthor={Jens Bemme, Caroline Förster, Juliane Flade},
      pdfcopyright={CC BY 4.0 International},
      pdfsubject={LIBREAS. Library Ideas, 45 (2024).},
      pdfkeywords={Bibliothek, Citizen Science, Gartenlaube, Vertonung, Wikisource, Vereinsarbeit},
      pdflicenseurl={https://creativecommons.org/licenses/by/4.0/},
	  pdfurl={https://doi.org/10.18452/...},
	  pdfdoi={10.18452/...},
      pdflang={de},
      pdfmetalang={de}
     }



\date{}
\begin{document}

\maketitle
\thispagestyle{fancyplain} 

%abstracts
\begin{abstract}
\noindent
\textbf{Zusammenfassung}: Digitale Sammlungen klingen erstmal nicht. Aber dann.

\enquote*{Spoken Wikisource -- German}\footnote{\url{https://commons.wikimedia.org/wiki/Category:Spoken_Wikisource_-_German}} enthält Töne von Joachim Ringelnatz,
gesprochene\footnote{\url{https://commons.wikimedia.org/wiki/Category:Joachim_Ringelnatz_spoken}}, Texte aus \enquote*{Die Gartenlaube}, auch \enquote*{spoken}\footnote{\url{https://commons.wikimedia.org/wiki/Category:Die_Gartenlaube_spoken}}, Stimmen und
Ideen. Der Klang der \enquote*{Gesprochenen Wikisource} ist vielschichtig
beziehungsweise kann und könnte er sehr vielseitig sein -- metaphorisch
als der Klang des Projekts \enquote*{Gesprochene Wikisource}, durch die Stimmen
der beteiligten Gemeinschaften oder der Räume, in denen diese Aufnahmen
entstehen; zum Beispiel Tonstudios, in denen Sprecher:innen sprechen.\footnote{Jens Bemme und Juliane Flade: Hat man da schon Töne? So funktioniert Gesprochene Wikisource, SLUB Open Science Lab, 20. Dezember 2023, \url{https://osl.hypotheses.org/9881}}

Gesprochen werden Textquellen aus Wikisource akustisch erfahrbar.
Wahrnehmbar wird dadurch nicht nur der Inhalt, sondern auch eine
Textinterpretation. Im Bibliothekskatalog gibt es diese Möglichkeit noch
nicht, die Quellen sind entweder als Text oder Audio verfügbar, hybrid
selten.

Beim Einsprechen und Hören ist ein Teil der Rezeption zu spüren:
Wie verstehe und interpretiere ich den Text einer Vorleser- oder
Sprecherin, die sich mit gesprochenem Volltext des Portals Wikisource
auseinandersetzt?! Dabei findet Interpretation statt, mehr als eine.

Wikisource gewinnt so als gesprochene digitale Sammlung weitere
Deutungs- und Metaebenen -- Nuancen, Zugänge und Freiheitsgrade,
Bedeutungen, Links. Dieser \enquote*{Sound of Gesprochene Wikisource} ist ein
Nachhall medialer Auseinandersetzungen mit \ldots{} Text in einer
Bibliothek. Die Autorinnen berichten, betonen und spielen dabei mit
Aspekten der \enquote*{Gesprochenen Wikisource} in historischen, bibliophilen und
modernen Linkzusammenhängen mit dem Podcaststudio der SLUB, mit den
offenen Kulturdaten des Dresdner Geschichtsvereins, mit \enquote*{DatenlaubeJam}
am Dienstag und Lesungen im Advent -- \enquote*{The Sound of Wikisource}
sozusagen, \enquote*{read, spoken and linked open}.\footnote{\url{https://de.wikiversity.org/wiki/DieDatenlaube/LIBREAS_The_Sound_of_Gesprochene_Wikisource}}

\begin{center}\rule{0.5\linewidth}{0.5pt}\end{center}
\end{abstract}

%body

\newpage
\hypertarget{einleitung}{%
\section{Einleitung}\label{einleitung}}

\begin{center}
\emph{Was meint ihr wohl, was eure Eltern treiben,}

\emph{Wenn ihr schlafen gehen müßt?}

\emph{Und sie angeblich noch Briefe schreiben.}

\emph{Ich kann's euch sagen: (...)}
\end{center}

Ja, was kann man abends alles machen? Lesen, Transkribieren, Editieren
und Edieren. Die Verszeilen \enquote*{An Berliner Kinder} von Joachim
Ringelnatz stehen im \enquote*{Kinder-Verwirr-Buch} von 1931. Das
Gedicht und das Buch wurden transkribiert und sind Teil der
deutschsprachigen Wikisource.\footnote{\url{https://de.wikisource.org/wiki/An_Berliner_Kinder}}
Man findet den Text vertont längst auf YouTube,\footnote{\url{https://youtu.be/nbWMbkPey-k}}
seit Weihnachten 2023 auch eingesprochen\footnote{\url{https://www.slub-dresden.de/besuchen/arbeitsplaetze-und-arbeitsraeume/podcaststudio}}
in Wikimedia Commons -- verknüpft in den Wikisourceseiten, in Wikidata
und eingebettet in Blogbeiträgen.\footnote{\url{https://osl.hypotheses.org/9881}}
So klingt 'Gesprochene Wikisource\textquotesingle{} für uns.

\hypertarget{sound-of-hackathon-editathon-dienstags}{%
\section{{[}{[}Sound of\ldots{]}{]} Hackathon, Editathon,
dienstags}\label{sound-of-hackathon-editathon-dienstags}}

Anfang 2022 begannen wir beim wöchentlichen \enquote*{DatenlaubeJam}
Schriften des Dresdner Geschichtsvereins aus den Jahrzehnten um 1900 für
Transkriptionen mit Wikisource ins Auge zu fassen.\footnote{\url{https://de.wikiversity.org/wiki/DieDatenlaube/Notizen/2022}}
Ein paar Aktive, andere Freunde des Vereins und die Geschäftsführerin
öffneten von da an dienstags morgens Browser und Webcams, um zu lernen
mit den Methoden der \enquote*{Datenlaube} erst ein
Mitgliederverzeichnis und dann reihenweise Vereinsmitteilungen und
Sonderveröffentlichungen in Wikisource\footnote{\url{https://de.wikisource.org/wiki/Dresdner_Geschichtsverein}}
neu zu publizieren.\footnote{\url{https://saxorum.hypotheses.org/10099}}

Hinzu kam das Podcaststudio in der Zentralbibliothek der Staats-,
Landes- und Universitätsbibliothek (SLUB), das 2022 eingerichtet wurde,
als neue Infrastruktur für digitale Töne. Erste Aufnahmen
selbstgewählter Texte entstanden dort an einem vorweihnachtlichen
Dienstagmorgen im Advent gemeinsam: Technikschulung, Gedichte von
Ringelnatz, kurze Artikel aus der \enquote*{Gartenlaube} -- eine kleine
Weihnachtsfeier.\footnote{\url{https://de.wikiversity.org/wiki/DieDatenlaube/Gesprochene_Wikisource\#Dezember_2022}}
Seitdem wächst die Liste fertiger Transkriptionen und damit der Fundus
umfangreicher Vereinsmitteilungen um 1900, kurzer und unterhaltender
\enquote*{Dresdner Geschichtsblätter}\footnote{\url{https://de.wikisource.org/wiki/Dresdner_Geschichtsbl\%C3\%A4tter}},
die bisher zwar gescannt und mit Optical Character Recognition (OCR) in
digitalen Sammlungen der SLUB aber gänzlich unbearbeitet zugänglich
waren. 2024 werden möglicherweise einige dieser -- unserer --
Geschichtsblätter eingesprochen und neu veröffentlicht, möglicherweise
wieder im Advent.

Wichtig dabei: Austausch, Spaß, Reflexion und Gruppendynamik. Von
\enquote*{Hackathon ist immer (dienstags)}\footnote{\url{https://de.wikiversity.org/wiki/DieDatenlaube/Notizen/GeNeMe_Abstrakt}},
einem inoffiziellen Leitspruch des \enquote*{DatenlaubeJams}, gibt es
längst Varianten: \enquote*{Editathon (auch) ist immer} und Hackathon
ist immer (öfter). Fast alle historischen \enquote*{Mittheilungen} des
Geschichtsvereins sind komplett in Wikisource bearbeitet. Die
Autor:innen verwiesen darin auf ihre weiteren Veröffentlichungen und auf
andere Referenzen, zum Beispiel in den \enquote*{Dresdner
Geschichtsblättern}. Diese Verweise können wir in Wikisource direkt
verlinken und in Wikidata nachweisen (WikiCite\footnote{\url{https://meta.wikimedia.org/wiki/WikiCite}}).
Band 1 der Geschichtsblätter ist fast komplett, der zweite wird in Kürze
gemeinsam korrigiert, Band 7 folgt.\footnote{\url{https://de.wikisource.org/wiki/Dresdner_Geschichtsverein\#Dresdner_Geschichtsbl\%C3\%A4tter}}

\hypertarget{vereins--und-beteiligungsprojekt}{%
\section{Vereins- und
Beteiligungsprojekt}\label{vereins--und-beteiligungsprojekt}}

Mit dem Seitenprojekt \enquote*{Gesprochene Wikisource} geht es
einerseits um die Inhalte und die Nutzung der historischen Texte. Basis
sind auf der anderen Seite die Textproduktion und -edition: die
Erstellung und das eigenhändige Bearbeiten -- also das Transkribieren --
von Vereinsmitteilungen, Aufsätzen und Artikeln als Aktivität des
Geschichtsvereins mit dem Team der \enquote*{Datenlaube} und der
Bibliothek.

Was motiviert uns, als Aktive, Mitglieder des Geschichtsvereins, Gründer
des Datenlaube-Pro\-jekts und Mitarbeiter:innen der SLUB? Die Motivation,
mit Wikisource ehrenamtlich tätig zu werden, speist sich aus eigenem
Interesse an Geschichte in Dresden, Anerkennung und dem Gefühl von
Selbstwirksamkeit in einer Gemeinschaft ähnlich \enquote*{tickender}
Menschen. Welche Themen wecken überhaupt Interesse? Der Dresdner
Geschichtsverein -- vielmehr der historische Vorgängerverein -- war als
Herausgeber geschichts(bürger)wissenschaftlicher Schriften äußerst
produktiv. Wir haben hier einen reichen Fundus von Textquellen, die für
Dresden Stadtgeschichte beschreiben. Die SLUB erprobt nebenbei Citizen
Science sowie Crowdsourcing und gewinnt weitere digitale Versionen plus
Katalogisate historischer Drucke.

Teilnehmer:innen am DatenlaubeJam suchen sich Texte selbst aus, die sie
bearbeiten, und sind dann meist begeistert. Diese Bearbeitung ist
zeitlich nicht an den \enquote*{DatenlaubeJam} gebunden und geschieht
unabhängig. Jede:r arbeitet weitgehend autonom in Wikisource und bei
Bedarf mit Wikimedia Commons und Wikidata.

Anerkennung bietet der DatenlaubeJam, die digitale morgendliche
Austauschrunde jeden Dienstag. Aber auch der Geschichtsverein, der
Ergebnisse dieser Arbeiten in \enquote*{Dresdner Heften}, in Vorträgen
und Artikeln aufgreift und publik macht.\footnote{\url{https://www.dresdner-geschichtsverein.de/dresdner_hefte.html}}

Selbstwirksamkeit entsteht hier aus erlernter digitaler
Methodenkompetenz und gegenseitiger Hilfe. Die Bedienoberfläche von
Wikisource für sich zu erschließen, dort immer sicherer und schneller zu
werden, ist Erfolg und Ansporn. Dabei erhalten alle Unterstützung von
erfahrenen Wikisourclern innerhalb und außerhalb der sich regelmäßig
treffenden Gruppe. Diese Selbstwirksamkeit ist ein gewichtiges Motiv für
diese Form bürgerwissenschaftlichen Arbeitens: Transkriptionen und
Citizen Science mit historischen Quellen der Geschichtswissenschaft,
deren bibliografischer Erschließung, Edition und schließlich Vertonung
für \enquote*{Gesprochene Wikisource}.

Solche selbst vertonten, gesprochenen historischen Quellen fügen dem
Bearbeiten und Erleben eine Stufe hinzu: Sprache und Wahrnehmung der
eigenen Stimme, verschiedene Ausdrucksweisen, Aufnahmetechnik und
gemeinsames Ausprobieren. Oder: Die Scheu, die eigene Stimme zu hören
und sich nicht als Profi zu fühlen, schreckt manche:n letztlich ab,
einen Wikisourcetext einzusprechen oder die Aufnahme dann auch zu
veröffentlichen. So oder so: Das gemeinsame digitale Treffen und
Ausprobieren sowie die technische Einführung im Podcaststudio prägen
indirekt auch die eigene Arbeit an den Quellen, Texten und
Linkzusammenhängen in Wikisource, Wikidata, Commons und in benachbarten
Webprojekten, Blogs und Open-Access-Publikationen wie LIBREAS. Der
Dresdner Geschichtsverein hat mit Wikisource ein Vereinsprojekt entdeckt
und entwickelt. \enquote*{Die Datenlaube} vermittelt digitale Methoden.
Die Bibliothek bietet Räume, Technik und Begleitung und gewinnt
Erfahrungen für die Nutzung des Podcaststudios. \enquote*{Sound}
entsteht dabei durch Kooperation.

\hypertarget{wikisource-inklusive-klang-mit-sprachen}{%
\section{Wikisource inklusive Klang mit
Sprachen}\label{wikisource-inklusive-klang-mit-sprachen}}

Die Arbeit an Texten in Wikisource und die Aufnahmen für
\enquote*{Gesprochene Wikisource} offenbaren längst weitere Facetten und
daran anknüpfend Ideen für Neues. Ein paar Beispiele zum Weiterdenken:

Die Texte werden gelesen, transkribiert und gesprochen nach dem
Zwei-Sinne-Prinzip barrierefreier Gestaltung zugänglicher.\footnote{\url{https://www.dguv.de/barrierefrei/grundlagen/anwendung/ergonomie/zwei-sinne/index.jsp}}

\begin{center}
\emph{(...) Ich kann's euch sagen: Da wird geküßt,}

\emph{Geraucht, getanzt, gesoffen, gefressen,}

\emph{Da schleichen verdächtige Gäste herbei. (...)}
\end{center}

Durch die offene Wikimedia-Infrastruktur und das Podcaststudio der
Bibliothek lässt sich Inklusion schnell und unkompliziert verwirklichen.
Barrierefreiheit kann so von jeder und jedem selbst umgesetzt werden.
Könnten wir den Ansatz ähnlich einfach auf Videospuren mit eingebetteter
Gebärdensprache übertragen? Wikimedia Commons speichert auch Videos.

\begin{center}
\emph{(...) Da wird jede Stufe der Unzucht durchmessen}

\emph{Bis zur Papagei-Sodomiterei.}

\emph{Da wird hasardiert um unsagbare Summen.}

\emph{Da dampft es von Opium und Kokain.}

\emph{Da wird gepaart, daß die Schädel brummen. (...)}
\end{center}

Multilingualität -- Vielsprachigkeit: Wikisource gibt es in vielen
Sprachversionen.\footnote{\url{https://wikisource.org/}} Wikidata bietet
multilinguale Metadaten für all diese Transkriptionen und deren
Tonaufnahmen. Wir können diese \enquote*{Spoken Wikisources} und
Podcaststudios in Bibliotheken zusammen als offene
Informationsinfrastruktur denken, in denen Tonspuren entstehen. Im
Idealfall sind das neue offene Bildungsressourcen mit soliden offenen
Metadaten, verknüpfbar, leicht einzubetten, einfach benutzbar und damit
nützlich auch an potentiell vielen anderen Stellen im offenen Web.

Verlinkendes Datendenken\footnote{\url{https://blog.slub-dresden.de/beitrag/2021/07/21/heimatforschung-mit-wikidata}}:
Wir können mit Tonaufnahmen transkribierter Wikisourcetexte und mit
deren offenen Metadaten in Wikidata \enquote*{Linked Open Data}-Methoden
neu beziehungsweise anders vermitteln. Wie verändert sich dann unsere
Idee der \enquote*{Digitalen Bibliothek}, wenn wir \enquote*{alles}
(immer öfter und systematischer) abtippen, transkribieren und/oder ein-
und aussprechen, vertonen, annotieren und offen erschließen?

\enquote*{The Sound of Gesprochene Wikisource} ist ein Resonanzraum: für
Menschen in Bibliotheken und deren offene Daten; für Ideen, was man mit
solchen Daten alles machen kann; für und mit Gemeinschaften, die neue
Medien erzeugen.

\begin{center}
\emph{(...) Ach schweigen wir lieber. --- Pfui Spinne, Berlin!}\footnote{Das vertonte Gedicht kann hier angehört werden: \url{https://commons.wikimedia.org/wiki/File:An\_Berliner\_Kinder.ogg}}
\end{center}

%autor
\begin{center}\rule{0.5\linewidth}{0.5pt}\end{center}

\textbf{Autor*innen}

Jens Bemme

Jens Bemme studierte Verkehrswirtschaft und Lateinamerikastudien. Heute interessiert er sich für Dorfbacköfen 
und historisches Radfahrerwissen um 1900 in der Oberlausitz und der Ostseeprovinzen. Mit der ‘Datenlaube’ und 
Christian Erlinger erschließt er Wikisource-Volltexte der Illustrierten ‘Die Gartenlaube’ offen in Wikidata. 
Als Mitarbeiter der SLUB Dresden begleitet Jens landeskundliche Citizen Science-Initiativen insbesondere mit 
den digitalen Werkzeugen und Gemeinschaften der Wikimedia-Bewegung. Mastodon: JensB@openbiblio.social

(\url{https://orcid.org/0000-0001-6860-0924})

Juliane Flade

Juliane Flade studierte an der TU Dresden Sprach-, Literatur- und Kulturwissenschaften. Ihr Schwerpunkt lag hierbei 
auf der Literatur der Gegenwart im ländlichen Raum. Vor ihrem Studium arbeitete sie als Logopädin. Aktuell ist sie 
in der SLUB Dresden als Projektmanagerin für Inklusion und Citizen Science tätig. Dabei ist ihr Ziel Quellen, Wissen 
und die Bibliothek zugänglicher zu machen. Ein Beispiel hierfür ist das Projekt Gesprochene Wikisource.

(\url{https://orcid.org/0000-0002-3249-7299})

Caroline Förster

Caroline Förster ist Geschäftsführerin des Dresdner Geschichtsverein e.V.


\end{document}