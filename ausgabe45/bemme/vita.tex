\begin{center}\rule{0.5\linewidth}{0.5pt}\end{center}

\textbf{Jens Bemme} (\url{https://orcid.org/0000-0001-6860-0924}) studierte Verkehrswirtschaft und Lateinamerikastudien. Heute interessiert er sich für Dorfbacköfen und historisches Radfahrerwissen um 1900 in der Oberlausitz und der Ostseeprovinzen. Mit der \enquote*{Datenlaube} und Christian Erlinger erschließt er Wikisource-Volltexte der Illustrierten \enquote*{Die Gartenlaube} offen in Wikidata. Als Mitarbeiter der SLUB Dresden begleitet Jens landeskundliche Citizen Science-Initiativen insbesondere mit den digitalen Werkzeugen und Gemeinschaften der Wikimedia-Bewegung. Mastodon: JensB@openbiblio.social

\textbf{Dr. Caroline Förster} ist seit 2021 Geschäftsführerin des Dresdner Geschichtsvereins und gibt in dieser Funktion die Dresdner Hefte heraus. Sie studierte Geschichte und Kommunikationswissenschaft an der TU Dresden und arbeitete festangestellt und freiberuflich im Bereich der Wissenschaftskommunikation. Die promovierte Historikerin engagiert sich in verschiedenen Ehrenämtern, darunter zum Beispiel bei Memorare Pacem e.\,V. – einem Verein, der sich seit 1990er Jahren mit den Fragen der Erinnerungskulturen in Dresden beschäftigt.

\textbf{Juliane Flade} (\url{https://orcid.org/0000-0002-3249-7299}) studierte an der TU Dresden Sprach-, Literatur- und Kulturwissenschaften. Ihr Schwerpunkt lag hierbei auf der Literatur der Gegenwart im ländlichen Raum. Vor ihrem Studium arbeitete sie als Logopädin. Aktuell ist sie in der SLUB Dresden als Projektmanagerin für Inklusion und Citizen Science tätig. Dabei ist ihr Ziel Quellen, Wissen und die Bibliothek zugänglicher zu machen. Ein Beispiel hierfür ist das Projekt Gesprochene Wikisource.