\textbf{Kurzfassung}: Anhand eines Beispiels zeigt der Artikel, welche
Konsequenzen ein soziologisch orientiertes Denken für die
Bibliothekspraxis haben könnte. Es wird eine These über die Nutzung
Öffentlicher Bibliotheken entwickelt, Evidenzen für diese gesammelt und
anschliessend ein konzeptionelles Modell erstellt, mit dem diese These
überprüft werden kann. Es wird postuliert, dass es sei unterschiedliche
Gruppen von Nutzer*innen gibt, eine bestandsbezogene und eine
raumbezogene. Zusätzlich wird diskutiert, welche Konsequenzen für die
Praxis Öffentlicher Bibliotheken gezogen werden könnten, wenn sich die
These als richtig herausstellt.

\begin{center}\rule{0.5\linewidth}{0.5pt}\end{center}

\textbf{Abstract}: Using an example, the article shows what consequences
sociologically-oriented thinking could have for library practice. It
develops a thesis about the use of public libraries, collects evidence
for it and creates a conceptual model with which this thesis can be
tested. It is postulated that there are different groups of users, one
collection-related and one space-related. In addition, it discusses
possible consequences for the practice of public libraries if the thesis
proves to be correct.
