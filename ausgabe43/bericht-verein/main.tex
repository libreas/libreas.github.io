\documentclass[a4paper,
fontsize=11pt,
%headings=small,
oneside,
numbers=noperiodatend,
parskip=half-,
bibliography=totoc,
final
]{scrartcl}

\usepackage[babel]{csquotes}
\usepackage{synttree}
\usepackage{graphicx}
\setkeys{Gin}{width=.4\textwidth} %default pics size

\graphicspath{{./plots/}}
\usepackage[ngerman]{babel}
\usepackage[T1]{fontenc}
%\usepackage{amsmath}
\usepackage[utf8x]{inputenc}
\usepackage [hyphens]{url}
\usepackage{booktabs} 
\usepackage[left=2.4cm,right=2.4cm,top=2.3cm,bottom=2cm,includeheadfoot]{geometry}
\usepackage[labelformat=empty]{caption} % option 'labelformat=empty]' to surpress adding "Abbildung 1:" or "Figure 1" before each caption / use parameter '\captionsetup{labelformat=empty}' instead to change this for just one caption
\usepackage{eurosym}
\usepackage{multirow}
\usepackage[ngerman]{varioref}
\setcapindent{1em}
\renewcommand{\labelitemi}{--}
\usepackage{paralist}
\usepackage{pdfpages}
\usepackage{lscape}
\usepackage{float}
\usepackage{acronym}
\usepackage{eurosym}
\usepackage{longtable,lscape}
\usepackage{mathpazo}
\usepackage[normalem]{ulem} %emphasize weiterhin kursiv
\usepackage[flushmargin,ragged]{footmisc} % left align footnote
\usepackage{ccicons} 
\setcapindent{0pt} % no indentation in captions

%%%% fancy LIBREAS URL color 
\usepackage{xcolor}
\definecolor{libreas}{RGB}{112,0,0}

\usepackage{listings}

\urlstyle{same}  % don't use monospace font for urls

\usepackage[fleqn]{amsmath}

%adjust fontsize for part

\usepackage{sectsty}
\partfont{\large}

%Das BibTeX-Zeichen mit \BibTeX setzen:
\def\symbol#1{\char #1\relax}
\def\bsl{{\tt\symbol{'134}}}
\def\BibTeX{{\rm B\kern-.05em{\sc i\kern-.025em b}\kern-.08em
    T\kern-.1667em\lower.7ex\hbox{E}\kern-.125emX}}

\usepackage{fancyhdr}
\fancyhf{}
\pagestyle{fancyplain}
\fancyhead[R]{\thepage}

% make sure bookmarks are created eventough sections are not numbered!
% uncommend if sections are numbered (bookmarks created by default)
\makeatletter
\renewcommand\@seccntformat[1]{}
\makeatother

% typo setup
\clubpenalty = 10000
\widowpenalty = 10000
\displaywidowpenalty = 10000

\usepackage{hyperxmp}
\usepackage[colorlinks, linkcolor=black,citecolor=black, urlcolor=libreas,
breaklinks= true,bookmarks=true,bookmarksopen=true]{hyperref}
\usepackage{breakurl}

%meta
\expandafter\def\expandafter\UrlBreaks\expandafter{\UrlBreaks%  save the current one
  \do\a\do\b\do\c\do\d\do\e\do\f\do\g\do\h\do\i\do\j%
  \do\k\do\l\do\m\do\n\do\o\do\p\do\q\do\r\do\s\do\t%
  \do\u\do\v\do\w\do\x\do\y\do\z\do\A\do\B\do\C\do\D%
  \do\E\do\F\do\G\do\H\do\I\do\J\do\K\do\L\do\M\do\N%
  \do\O\do\P\do\Q\do\R\do\S\do\T\do\U\do\V\do\W\do\X%
  \do\Y\do\Z}
%meta

\fancyhead[L]{Vorstand LIBREAS-Verein\\ %author
LIBREAS. Library Ideas, 43 (2023). % journal, issue, volume.
\href{https://doi.org/10.18452/...}{\color{black}https://doi.org/10.18452/...}
{}} % doi 
\fancyhead[R]{\thepage} %page number
\fancyfoot[L] {\ccLogo \ccAttribution\ \href{https://creativecommons.org/licenses/by/4.0/}{\color{black}Creative Commons BY 4.0}}  %licence
\fancyfoot[R] {ISSN: 1860-7950}

\title{\LARGE{In eigener Sache: Bericht über die Aktivitäten des LIBREAS-Vereins 2021/2022}}% title
\author{Vorstand LIBREAS-Verein} % author

\setcounter{page}{1}

\hypersetup{%
      pdftitle={In eigener Sache: Bericht über die Aktivitäten des LIBREAS-Vereins 2021/2022},
      pdfauthor={Vorstand LIBREAS-Verein},
      pdfcopyright={CC BY 4.0 International},
      pdfsubject={LIBREAS. Library Ideas, 43 (2023)},
      pdfkeywords={Bibliothekswissenschaft, Informationswissenschaft, Verein, Tätigkeitsbericht},
      pdflicenseurl={https://creativecommons.org/licenses/by/4.0/},
      pdfcontacturl={http://libreas.eu},
      baseurl={},
      pdflang={de},
      pdfmetalang={de},
      pdfdoi={10.18452/...},
      pdfurl={https://doi.org/10.18452/...}
     }



\date{}
\begin{document}

\maketitle
\thispagestyle{fancyplain} 

%abstracts

%body
\emph{Vorbemerkung: Die Mitgliederversammlung des LIBREAS-Vereins hat in
ihrer Sitzung am 29.11.2022 beschlossen, dass der Vereinsvorstand den
Bericht über die Vereinsaktivitäten im Sinne der Transparenz künftig
jeweils in der auf die Mitgliederversammlung folgenden Ausgabe der
LIBREAS.Library Ideas in gekürzter Form veröffentlicht.
Personenbeziehbare Daten werden dabei ausgelassen, sofern nicht die
ausdrückliche Zustimmung der betreffenden Person(en) vorliegt. Ebenso
werden Details ausgelassen, die das Vereinsvermögen betreffen. Sie
können durch Mitglieder des Vereins beim Vereinsvorstand jederzeit
erfragt werden beziehungsweise werden in den Protokollen der
Versammlungen aufgeschlüsselt und mit den Mitgliedern geteilt.}

\hypertarget{berichtszeitraum}{%
\subsection{Berichtszeitraum}\label{berichtszeitraum}}

Der Bericht bezieht sich auf den Zeitraum zwischen der
Mitgliederversammlung 2021 (06.11.2021) bis zur Mitgliederversammlung
2022 (29.11.2022).

\hypertarget{vorstand}{%
\subsection{Vorstand}\label{vorstand}}

Dem Vereinsvorstand gehörten im Berichtszeitraum Matti Stöhr
(Vorsitzender), Dr.~Karsten Schuldt (stellvertretender Vorsitzender),
Jana Rumler (Schriftleiterin), Dr.~Maxi Kindling (Finanzerin) und Ben
Kaden (Ressort LIBREAS.Library Ideas) an. Der Vorstand hat sich
regelmäßig getroffen und bei Bedarf virtuell ausgetauscht.

\hypertarget{mitglieder}{%
\subsection{Mitglieder}\label{mitglieder}}

Der LIBREAS-Verein hatte mit Stand 23.10.2022 52 Mitglieder. Davon waren
49 persönliche Mitglieder sowie drei Fördermitglieder.

\hypertarget{vereinsfinanzen}{%
\subsection{Vereinsfinanzen}\label{vereinsfinanzen}}

Die Einnahmen des LIBREAS-Vereins setzten sich im Haushaltsjahr
2021/2022 aus den Mitgliedsbeiträgen und Spenden sowie einer
Gebührenrückzahlung zusammen. Ausgaben wurden getätigt für das Hosting
der Webauftritte, Kontoführungsgebühren, der Servicepauschale für die
L4F-Website. Der größte Ausgabenposten war der Druck und Versand des
Jubiläumsbandes. Die Kasse wird jährlich geprüft und das Ergebnis im
Rahmen der Mitgliederversammlung berichtet. Es gab keine Beanstandungen.

\hypertarget{libreas-ausgabenredaktion}{%
\subsection{LIBREAS-Ausgaben/Redaktion}\label{libreas-ausgabenredaktion}}

Der Schwerpunkt der Vorstands- und der Vereinstätigkeit liegt in der
Redaktion der LIBREAS. Im Berichtszeitraum lagen die Ausgaben
\href{https://libreas.eu/ausgabe41/inhalt/}{\#41} \enquote{Big Scholarly
Data}, \href{https://libreas.eu/ausgabe42/inhalt/}{\#42} \enquote{Das
Leben, das Universum und der ganze Rest} und die Vorbereitung der
Ausgabe \#43 \enquote{Soziologie der Bibliothek}. Die Redaktion hat
erfreulicherweise inzwischen mehrere weitere feste Mitglieder
hinzugewinnen können.

\hypertarget{kommunikation}{%
\subsection{Kommunikation}\label{kommunikation}}

Die Social-Media-Kanäle -- insbesondere
\href{https://libreas.wordpress.com/}{LIBREAS.Weblog} und
\href{https://twitter.com/libreas}{LIBREAS.Twitter} -- wurden in
üblicher Weise bespielt, um insbesondere die Ausgaben inklusive Call for
Papers zu bewerben. Im Jahr 2021 wurde zusätzlich der
\href{https://www.instagram.com/libreas.libraryideas/}{LIBREAS.Instagram-Kanal}
durch das LIBREAS-Projektseminar (IBI der HU Berlin) eingerichtet.
Dieser erfreute sich wachsender Beliebtheit (derzeit über 250 Follower).
Seit November 2022 ist LIBREAS auch auf
\href{https://berlin.social/@libreas@openbiblio.social}{Mastodon}
verfügbar.

\hypertarget{praktikum-beim-libreas-verein}{%
\subsection{Praktikum beim
LIBREAS-Verein}\label{praktikum-beim-libreas-verein}}

Vom 20.09.2021 bis 31.3.2022 hat Philipp Falkenburg (Studierender der
Bibliothekswissenschaft an der FH Potsdam im Praxissemester) bei LIBREAS
ein Praktikum absolviert und sich in dieser Rolle in der Redaktions- und
Vereinsarbeit engagiert. Seine Tätigkeitsschwerpunkte waren: Die
Unterstützung bei der Erstellung des LIBREAS-Jubiläumsbandes, der
Mitgliederverwaltung sowie bei der Redaktionsarbeit. Er gab wichtige
Impulse zur Weiterentwicklung, beispielsweise zur strukturierten
Erfassung der Kolumne \enquote{Das liest die LIBREAS}
(\href{https://www.zotero.org/groups/4620604/libreas_dldl/}{DLDL}) mit
Zotero. Das Angebot eines Praktikums kann im Sinne einer nachhaltigen
Vermittlung von Kompetenzen zur Organisation von Publikations- und
Redaktionsworkflows unmittelbar als Realisierung des Vereinszwecks
gesehen werden.

\hypertarget{kooperation-mit-der-fachhochschule-potsdam}{%
\subsection{Kooperation mit der Fachhochschule
Potsdam}\label{kooperation-mit-der-fachhochschule-potsdam}}

Der Verein agierte für ein Seminar zum Thema \enquote{B12 Vermittlung
von Informationskompetenz} am Fachbereich Informationswissenschaften der
Fachhochschule Potsdam im Sommersemester 2022 als Praxispartner. Das
Projektteam entwickelte ein Konzept für mögliche Weiterbildungsangebote
des LIBREAS-Vereins.

\hypertarget{jubiluxe4umsband-15-jahre-libreas}{%
\subsection{Jubiläumsband 15 Jahre
LIBREAS}\label{jubiluxe4umsband-15-jahre-libreas}}

Der Band konnte in diesem Jahr dank des Engagements im Rahmen des
Praktikums (siehe oben) fertiggestellt werden. Er wurde im Frühsommer
2022 an alle Vereins- und Redaktionsmitglieder sowie die Autor*innen des
Jubiläumsbandes per Post versandt. Der Verein hat mehrere Rückmeldungen
mit herzlichem Dank erhalten. Einige wenige Exemplare konnten nicht
zugestellt werden. Die Ermittlung der aktuellen Postadressen einiger
Vereinsmitglieder wurde bis zur Mitgliederversammlung 2022 verfolgt.

\hypertarget{stipendien}{%
\subsection{Stipendien}\label{stipendien}}

Die Planung des Stipendiums war in diesem Berichtsjahr weiterhin
ausgesetzt. Die Idee eines Online-Seminars für wissenschaftliches
Schreiben wurde aufgrund mangelnder zeitlicher Kapazitäten verschoben.

\hypertarget{libraries4future}{%
\subsection{Libraries4Future}\label{libraries4future}}

Der Vereinsvorstand hat sich im Berichtszeitraum weiterhin mit dem
\href{https://www.netzwerk-gruene-bibliothek.de/}{Netzwerk Grüne
Bibliothek} ausgetauscht und sich bei der Verbreitung von Infos
vorwiegend über Twitter eingebracht. Außerdem unterstützt der Verein
auch in 2022 die Betreuung der
\href{https://libraries4future.org/}{Website} finanziell. Die Initiative
sucht nach wie vor nach Menschen, die sich aktiv einbringen können.
Unterstützung ist für \href{https://twitter.com/Libraries4F}{Twitter}
bzw. \href{https://climatejustice.global/@libraries4future}{Mastodon},
das \href{https://libraries4future.org/blog/}{Blog}, den
\href{https://www.youtube.com/channel/UC2ugDvNT713LndZuONDt8-A}{YouTube-Kanal},
aber auch bei der Beteiligung an Orga von Barcamps gesucht. L4F-Material
kann jederzeit bestellt werden.

\hypertarget{teilnahme-am-writing-sprint-arbeitsabluxe4ufe-und-workflows-des-projekts-scholar-led-plus}{%
\subsection{\texorpdfstring{Teilnahme am Writing Sprint
\enquote{Arbeitsabläufe und Workflows} des Projekts Scholar-led
Plus}{Teilnahme am Writing Sprint ``Arbeitsabläufe und Workflows'' des Projekts Scholar-led Plus}}\label{teilnahme-am-writing-sprint-arbeitsabluxe4ufe-und-workflows-des-projekts-scholar-led-plus}}

LIBREAS wurde vom
\href{https://www.hiig.de/project/scholar-led-plus/}{Projekt Scholar-led
Plus} des Alexander von Humboldt Instituts für Internet und Gesellschaft
(HIIG) für die Teilnahme an einem Writing Sprint angefragt. Als
Scholar-led- (beziehungsweise zunächst Student-led)-Journal der ersten
Stunde sagte die Redaktion gern zu. Ben Kaden vertrat LIBREAS bei der
Veranstaltung am 25.10.2022 und ergänzte Erfahrungen und Perspektiven
aus fast 18 Jahren LIBREAS. Konkret ging es um Workflow-Planungen für
Ausgaben und die Formatentwicklung von Scholar-Led-Titeln.

%autor

\end{document}