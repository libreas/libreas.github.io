\textbf{Kurzfassung}: Der Begriff Mediensozialisation verweist auf den
Sachverhalt, dass sich Prozesse der Vergesellschaftung und der
Persönlichkeitsentwicklung zunehmend in der Interaktion mit Medien
vollziehen. Die Auseinandersetzung mit Webangeboten, Büchern, TV und
anderen Medien trägt mit dazu bei, dass der einzelne Mensch
gesellschaftliche Strukturen, Normen et cetera verinnerlicht. Begreift
man Medien als wesentlichen Teil des Alltagslebens und damit verbunden
als Sozialisationsprozesse beeinflussend, so ergeben sich daraus
interessante Impulse für die bibliothekarische Praxis. Der vorliegende
Artikel zeigt schlaglichtartig auf, inwiefern die
Mediensozialisationsforschung dazu beitragen kann,
Bibliotheksdienstleistungen und -angebote weiterzuentwickeln.

\textbf{Abstract}: \enquote{Media socialization} points out that the
interaction with media is gaining in importance for the process of
socialization and for the development of personality. Dealing with the
web, books, television and other media contributes to internalizing
social structures, standards et cetera. If media is understood as an
indispensable part of daily life and, subsequently, as an important
factor for socialization, we can receive impulses for the library
praxis. This article highlights how the research on media socialization
could help librarians to develop or to refine their services and offers.
