\documentclass[a4paper,
fontsize=11pt,
%headings=small,
oneside,
numbers=noperiodatend,
parskip=half-,
bibliography=totoc,
final
]{scrartcl}

\usepackage[babel]{csquotes}
\usepackage{synttree}
\usepackage{graphicx}
\setkeys{Gin}{width=.4\textwidth} %default pics size

\graphicspath{{./plots/}}
\usepackage[ngerman]{babel}
\usepackage[T1]{fontenc}
%\usepackage{amsmath}
\usepackage[utf8x]{inputenc}
\usepackage [hyphens]{url}
\usepackage{booktabs} 
\usepackage[left=2.4cm,right=2.4cm,top=2.3cm,bottom=2cm,includeheadfoot]{geometry}
\usepackage[labelformat=empty]{caption} % option 'labelformat=empty]' to surpress adding "Abbildung 1:" or "Figure 1" before each caption / use parameter '\captionsetup{labelformat=empty}' instead to change this for just one caption
\usepackage{eurosym}
\usepackage{multirow}
\usepackage[ngerman]{varioref}
\setcapindent{1em}
\renewcommand{\labelitemi}{--}
\usepackage{paralist}
\usepackage{pdfpages}
\usepackage{lscape}
\usepackage{float}
\usepackage{acronym}
\usepackage{eurosym}
\usepackage{longtable,lscape}
\usepackage{mathpazo}
\usepackage[normalem]{ulem} %emphasize weiterhin kursiv
\usepackage[flushmargin,ragged]{footmisc} % left align footnote
\usepackage{ccicons} 
\setcapindent{0pt} % no indentation in captions

%%%% fancy LIBREAS URL color 
\usepackage{xcolor}
\definecolor{libreas}{RGB}{112,0,0}

\usepackage{listings}

\urlstyle{same}  % don't use monospace font for urls

\usepackage[fleqn]{amsmath}

%adjust fontsize for part

\usepackage{sectsty}
\partfont{\large}

%Das BibTeX-Zeichen mit \BibTeX setzen:
\def\symbol#1{\char #1\relax}
\def\bsl{{\tt\symbol{'134}}}
\def\BibTeX{{\rm B\kern-.05em{\sc i\kern-.025em b}\kern-.08em
    T\kern-.1667em\lower.7ex\hbox{E}\kern-.125emX}}

\usepackage{fancyhdr}
\fancyhf{}
\pagestyle{fancyplain}
\fancyhead[R]{\thepage}

% make sure bookmarks are created eventough sections are not numbered!
% uncommend if sections are numbered (bookmarks created by default)
\makeatletter
\renewcommand\@seccntformat[1]{}
\makeatother

% typo setup
\clubpenalty = 10000
\widowpenalty = 10000
\displaywidowpenalty = 10000

\usepackage{hyperxmp}
\usepackage[colorlinks, linkcolor=black,citecolor=black, urlcolor=libreas,
breaklinks= true,bookmarks=true,bookmarksopen=true]{hyperref}
\usepackage{breakurl}

%meta
%meta

\fancyhead[L]{A. Roth\\ %author
LIBREAS. Library Ideas, 43 (2023). % journal, issue, volume.
\href{https://doi.org/10.18452/27070}{\color{black}https://doi.org/10.18452/27070}
{}} % doi 
\fancyhead[R]{\thepage} %page number
\fancyfoot[L] {\ccLogo \ccAttribution\ \href{https://creativecommons.org/licenses/by/4.0/}{\color{black}Creative Commons BY 4.0}}  %licence
\fancyfoot[R] {ISSN: 1860-7950}

\title{\LARGE{Mediensoziologie}}% title
\author{Alexandra C. Roth} % author

\setcounter{page}{1}

\hypersetup{%
      pdftitle={Mediensoziologie},
      pdfauthor={Alexandra C. Roth},
      pdfcopyright={CC BY 4.0 International},
      pdfsubject={LIBREAS. Library Ideas, 43 (2023).},
      pdfkeywords={Mediensoziologie, Medien, Sozialisation},
      pdflicenseurl={https://creativecommons.org/licenses/by/4.0/},
      pdfurl={https://doi.org/10.18452/27070},
      pdfdoi={10.18452/27070},
      pdflang={de},
      pdfmetalang={de}
     }



\date{}
\begin{document}

\maketitle
\thispagestyle{fancyplain} 

%abstracts
\begin{abstract}
\noindent
\textbf{Kurzfassung}: Der Begriff Mediensozialisation verweist auf den
Sachverhalt, dass sich Prozesse der Vergesellschaftung und der
Persönlichkeitsentwicklung zunehmend in der Interaktion mit Medien
vollziehen. Die Auseinandersetzung mit Webangeboten, Büchern, TV und
anderen Medien trägt mit dazu bei, dass der einzelne Mensch
gesellschaftliche Strukturen, Normen et cetera verinnerlicht. Begreift
man Medien als wesentlichen Teil des Alltagslebens und damit verbunden
als Sozialisationsprozesse beeinflussend, so ergeben sich daraus
interessante Impulse für die bibliothekarische Praxis. Der vorliegende
Artikel zeigt schlaglichtartig auf, inwiefern die
Mediensozialisationsforschung dazu beitragen kann,
Bibliotheksdienstleistungen und -angebote weiterzuentwickeln.

\noindent \textbf{Abstract}: \enquote{Media socialization} points out that the
interaction with media is gaining in importance for the process of
socialization and for the development of personality. Dealing with the
web, books, television and other media contributes to internalizing
social structures, standards et cetera. If media is understood as an
indispensable part of daily life and, subsequently, as an important
factor for socialization, we can receive impulses for the library
praxis. This article highlights how the research on media socialization
could help librarians to develop or to refine their services and offers.
\end{abstract}

%body
\hypertarget{einfuxfchrung}{%
\section{Einführung}\label{einfuxfchrung}}

Medien zählen zu den Faktoren, die den Prozess der
Persönlichkeitsentwicklung eines Menschen beeinflussen können. Inwiefern
Medienumwelt und -handeln ein Individuum prägen, wird unter anderem im
Kontext der Mediensozialisationsforschung untersucht (Aufenanger 2008).
Für Bibliotheken als Einrichtungen, zu deren Kerngeschäft das
Verfügbarmachen von Büchern, Zeitschriften und seit geraumer Zeit auch
von Datenbanken sowie anderen digitalen Angeboten gehört, kann ein Blick
auf dieses Forschungsfeld daher sehr wertvoll sein. Dieser Beitrag
widmet sich dementsprechend den Begriffen der Sozialisation
beziehungsweise der Mediensozialisation und leitet daraus Denkanstöße
für die bibliothekarische Praxis ab.

\hypertarget{sozialisation}{%
\section{Sozialisation}\label{sozialisation}}

Mediensozialisation ist ein Kompositum aus den Begriffen
\enquote{Medien} und \enquote{Sozialisation}. Beide Termini sind in der
Alltagssprache gebräuchlich, müssen für die weiteren Ausführungen jedoch
genauer bestimmt werden.

Zunächst zum Begriff Medien: Darunter lassen sich in Anlehnung an Fromme
(2006, S. 115) Mittel beziehungsweise in einem engeren Sinne technische
Hilfsmittel zur Übertragung von Informationen oder Zeichen verstehen.
Ein Teilbereich dessen sind die sogenannten Neuen Medien. Diese
Bezeichnung wird speziell auf computerbasierte Technologien angewandt
(Fromme 2006, S. 116). Wenn im Folgenden von Medien die Rede ist, so
sind damit also nicht-menschliche Kommunikationsmittel gemeint wie zum
Beispiel Bücher, Zeitungen, TV, das Internet.

Der Begriff Sozialisation dürfte im Vergleich zu den Termini Medialität,
Medium et cetera einen kleineren Raum in der bibliothekarischen
Fachsprache einnehmen und wird daher an dieser Stelle ausführlicher
dargelegt. Er wurde ursprünglich durch den Soziologen Durkheim
eingeführt, um die Vergesellschaftung des einzelnen Menschen zu
beschreiben (Gudjons/Traub 2020, S.~160). Ganz grundsätzlich beziehen
sich Sozialisationsprozesse auf Vorgänge, im Rahmen derer sich der
einzelne Mensch soziale Standards, Verhaltensanforderungen und so weiter
aneignet, um gesellschafts- und handlungsfähig zu werden, sich also in
einen bestimmten Kontext einzugliedern (Bernhard 2018, S. 303 ff.; Borst
2016, S. 22; Hurrelmann 2011, S. 319). Schwerpunkte von
Sozialisationstheorien sind unter anderem die Einwirkungen der
Gesellschaft auf das Subjekt, die Effekte bestimmter Einflussfaktoren
wie auch Determinanten des menschlichen Verhaltens (Dörpinghaus/Uphoff
2019, S. 104). Es handelt sich hierbei um ein Forschungsfeld mit
multiplen methodischen und theoretischen Zugängen. Besonders stark sind
unter anderem die Bezüge zur Soziologie, so etwa zum Symbolischen
Interaktionismus oder zu Bourdieus Habitus-Konzept, zur Psychologie, das
heißt zur Entwicklungs-, Lern- und Verhaltenspsychologie, zur
Psychoanalyse et cetera, zur Philosophie sowie zur
Erziehungswissenschaft (Gudjons/Traub 2020, S.~162--175; Hurrelmann
2011, S. 319; Hurrelmann et al.~2015; Vogel 2019, S. 117).

Zu betonen ist, dass Sozialisation nicht ausschließlich, wie etwa
Erziehung, auf intentionale, zielgerichtete Einflussnahmen auf die
Persönlichkeitsentwicklung bezogen ist, sondern alle Faktoren umfasst,
die sich auf menschliches Wissen, Verhalten, Fühlen et cetera auswirken
-- seien sie geplant, ungeplant, bewusst oder unbewusst (Baacke 2007, S.
38; Hurrelmann 2011, S. 321; Löw/Geier 2014, S. 24 f.). Es handelt sich
hierbei um einen lebenslangen und interaktiven Prozess, das heißt, dass
der einzelne Mensch sich fortlaufend in Auseinandersetzung mit der
Umwelt Normen, Auffassungen et cetera aneignet (vergleiche
Dörpinghaus/Uphoff 2019, S. 96, 98; Löw/Geier 2014, S. 24). Hurrelmann
und Bauer (2015, S. 144 f.) nutzen in diesem Zusammenhang den Begriff
des \enquote{produktiv realitätsverarbeitenden Subjekts}, wodurch
verdeutlicht wird, dass Sozialisation zumindest nach heutigem
Verständnis kein reiner An- beziehungsweise Einpassungsprozess in einen
vorgegebenen Kontext darstellt. Vielmehr existiere eine dialektische
Beziehung zwischen der gesellschaftlich vermittelten Realität einerseits
und dem Individuum andererseits. Die \enquote{Verinnerlichung sozialer
Normen, Werte, Einstellungen, Geschmäcker und Haltungen}
(Dörpinghaus/Uphoff 2019, S. 98) vollzieht sich nicht zuletzt unter dem
Einfluss der sogenannten Sozialisationsinstanzen beziehungsweise
-kontexte, wie der Familie (Hurrelmann/Bauer 2015, S. 153--156). Wie
groß deren Wirkmacht ist, unterscheidet sich je nach Lebensabschnitt: So
sind im Rahmen der Primärsozialisation vor allem Familienmitglieder
prägend, während in der Sekundärsozialisation unter anderem pädagogisch
Tätige in Schulen oder Kindertagesstätten das Individuum beeinflussen
(Hurrelmann/Bauer 2015, S. 153 ff.). Außerschulische Einrichtungen für
Arbeit, Freizeit, Unterhaltung, Kultur et cetera gehören dabei zu den
tertiären Sozialisationsinstanzen (Dörpinghaus/Uphoff 2019, S. 101;
Hurrelmann/Bauer 2015, S. 156). Dazu lassen sich auch Bibliotheken
rechnen.

\hypertarget{mediensozialisation}{%
\section{Mediensozialisation}\label{mediensozialisation}}

Medien wiederum gelten als sekundäre, aber auch als tertiäre
Sozialisationsinstanzen (Dörpinghaus/Uphoff 2019, S. 101;
Hurrelmann/Bauer 2015, S. 156). Diese Einordnung wird allerdings
kritisch reflektiert. So gibt es Ansätze beziehungsweise Überlegungen,
Medien nicht als separate Sozialisationsinstanzen zu betrachten, sondern
sie in ihrer Verwobenheit mit weiteren sozialisierenden Kontexten wie
Familie zu fassen (Hoffmann/Wagner 2013, S.~3; Lange 2015, S.~537; Mikos
2010, S. 42; Spanhel 2013, S.~30 f.; Süss 2004, S.~26). Trotz dieser
unterschiedlichen Betrachtungsweisen bleibt festzuhalten: Mediale
Angebote scheinen für den Sozialisationsprozess eine gewichtige Rolle zu
spielen. Dies ist naheliegend, führt man sich erneut das Grundkonzept
von Sozialisation vor Augen: Das Selbst wird in der beziehungsweise
durch die Interaktion mit der Umwelt geprägt (Dörpinghaus/Uphoff 2019,
S. 96). Zur Umweltinteraktion gehören, nicht zuletzt bedingt durch den
technischen Fortschritt, in einem immer stärkeren Maße mediale Angebote.
Selbige sind durch ihre Omnipräsenz und die Veralltäglichung ihrer
Nutzung zu einem essenziellen Teil der sozialisierenden Lebenswelt
avanciert. In ihrer Wechselbeziehung mit Sozialisationsinstanzen wie
Schule, Familie et cetera verändern Medien also die Bedingungen von
Sozialisation (Mikos 2010, S. 42).

Sozialisationsimpulse, die von Medien ausgehen, sind dabei vielfältig.
Letztere \enquote{fungieren {[}\ldots{]} mitunter parasozial,
sozial-integrativ, sinngebend, wertschöpfend und identitätsstiftend}
(Hoffmann/Mikos 2010, S. 7). Dieses breite Wirkungsspektrum ist
angesichts der Funktionsvielfalt von Medien nicht verwunderlich: Selbige
werden zu Unterhaltungs-, ebenso wie zu Informations- und
Kommunikationszwecken genutzt, sie dienen als Wissensressourcen sowie
als Instrumente der Lebensbewältigung, gestalten soziale Beziehungen mit
und liefern Normen, Werte, Lebensmodelle (Hoffmann/Mikos 2010, S. 7;
Hoffmann/Wagner 2013, S. 4; Mikos 2010, S. 37). Medien kultivieren dabei
bestimmte, die Alltagswelt betreffende, Vorstellungen, so etwa zu
Geschlechterrollen (Bucher 2005, S. 43). Die Enkulturationsfunktion
medialer Angebote zeigt sich auch darin, dass Menschen über ihre
Präferenzen, etwa was bestimmte Inhalte oder Träger betrifft, ihre
Zugehörigkeit zu Kontexten definieren (Hoffmann/Mikos 2010, S. 7). Die
Motive für die Mediennutzung sind dabei divers und den Rezipient*innen
nicht immer bewusst (Hoffmann/ Kutscha 2010, S. 225).

Wie auch in der allgemeinen Sozialisationsforschung darf das Subjekt bei
all dem nicht als völlig passiv, das heißt den Medieneinflüssen hilflos
ausgeliefert verstanden werden. Vielmehr lässt sich, Süss folgend (2010,
S. 110--114), differenzieren zwischen Selbst- und Fremdsozialisation mit
beziehungsweise durch Medien. Im erstgenannten Fall würden die
Sozialisanden autonom über ihren Medienumgang entscheiden, also zum
Beispiel festlegen, welche Inhalte sie wann, an welchem Ort, mit welchem
Trägermedium et cetera nutzen. Grenzen der Selbstsozialisation sind laut
Süss (2010, S. 112) etwa dadurch gesetzt, dass der Zugriff auf mediale
Angebote unter anderem durch finanzielle Hürden eingeschränkt werden
kann oder auch dadurch, dass der Mediengebrauch zum Teil Kulturtechniken
wie Literalität voraussetzt, die erst mithilfe von erwachsenen
Bezugspersonen erlernt werden müssen. Damit ist der Bereich der
Fremdsozialisation angesprochen. Darunter lassen sich Bemühungen fassen,
den Medienumgang von zum Beispiel Heranwachsenden zu steuern, anzuleiten
oder bewusst zu gestalten, wie es etwa im Kontext der Medienerziehung
der Fall ist (Süss 2010, S. 110; 113). So bieten Erwachsene
beispielsweise den Zugang zu medialen Angeboten an oder verhindern
beziehungsweise erschweren diesen, wenn sie negative Auswirkungen auf
die Rezipient*innen befürchten (Süss 2004, S. 274). Die Mediennutzung
und damit auch mögliche Sozialisationseffekte sind also durch
Sozialisatoren, wie Erwachsene, Schule und so weiter, mitbestimmt (Süss
2004, S. 275).

Zusammenfassend lässt sich Mediensozialisation definieren
\begin{quote} 
\enquote{als
die aktive Auseinandersetzung von Menschen mit den Medienangeboten in
einer medial geprägten Welt, die selbst wieder Denken und Handeln
derselben beeinflussen kann {[}\ldots{]}. Die innere Realität beschreibt
dabei den Einfluss von Medien auf Persönlichkeitsmerkmale durch deren
sozialen Gebrauch und zugleich aber auch die Befähigung zur
Auseinandersetzung mit Medien. Die äußere Realität spiegelt die
Mediatisierungsprozesse wider, die eine Gesellschaft zu einem bestimmten
historischen Zeitpunkt durchläuft und mit der sich die Menschen in Form
von Aneignungsprozessen auseinandersetzen müssen} (Aufenanger 2020, S.
2).
\end{quote}

Inwiefern Medien Sozialisationsprozesse tangieren, hängt jedoch nicht
nur vom jeweiligen sozio-kulturellen Kontext ab. Auch Faktoren wie die
individuelle Wahrnehmungsfähigkeit oder die jeweilige Lebensphase sind
hier zu berücksichtigen. Diese bedingen unter anderem ästhetische
Präferenzen oder auch, wie empfänglich ein Mensch für mediale Einflüsse
ist, wie medial Erfahrenes verarbeitet wird, wie welche Medienarten mit
welcher Motivation genutzt werden et cetera (Hoffmann/Kutscha 2010; Rose
2013, S. 102--112).

Zu den zuvor angesprochenen Aspekten der Mediennutzung, -aneignung und
-wirkung existieren bereits zahlreiche Ansätze sowie Studien (Hoffmann
2010, S. 11; Rose 2013, S. 100). Nichtsdestotrotz gibt es bisher keine
umfassende Theorie der Mediensozialisation (Aufenanger 2008, S. 90;
Hoffmann/Mikos 2010, S. 9; Rose 2013, S. 100). Wie Hoffmann (2010, S.
15--22) darlegt, werden darüber hinaus in Sozialisationskonzepten Medien
nur bedingt berücksichtigt und umgekehrt vernachlässigen Medientheorien
Sozialisationsaspekte. Die Forschung im Bereich der Mediensozialisation
stellt Aspekte wie die individuelle Medienaneignung in den Fokus (Rose
2013, S. 99). Das bedingt die Nähe des Forschungsfeldes zu verschiedenen
Disziplinen. Dazu zählen etwa die Mediensoziologie, -psychologie,
-pädagogik und -wissenschaft (Aufenanger 2020, S. 2). Bereits vorhandene
Theorien, Studien und Konzepte werden beispielsweise bei Aufenanger
(2008; 2020, S. 3--6) prägnant beschrieben und differenziert.
Anschlussfähig an die jüngere Sozialisationsforschung sind dabei vor
allem Ansätze, die nicht monokausale Einflüsse medialer Angebote
unterstellen, sondern Zugänge, die von einer komplexen
Subjekt-Medium-Wechselbeziehung ausgehen, den Medienumgang als Teil des
sozialen Handelns begreifen und in einem lebensweltlichen Kontext
betrachten (Aufenanger 2008, S. 88; Niesyto 2010, S. 48).

\hypertarget{mediensozialisation-und-bibliotheken}{%
\section{Mediensozialisation und
Bibliotheken}\label{mediensozialisation-und-bibliotheken}}

Die vorangegangenen Ausführungen zur Mediensozialisation im Allgemeinen
lassen bereits Impulse für die bibliothekarische Arbeit erkennen. Ohne
Anspruch auf Vollständigkeit werden einige davon nun näher erläutert.

So ist zunächst hervorzuheben, dass Mediensozialisation nicht nur die
Wirkmacht von Medien fokussiert, sondern auch jene Fähigkeiten, die
erforderlich sind, um sich in einer mediatisierten Gesellschaft
behaupten und daran partizipieren zu können (Aufenanger 2020; Süss 2004,
S. 65). Dies verweist auf den Bereich der Medienkompetenz
beziehungsweise auf die Frage, wie sich diese stärken lässt.
Bibliotheken haben dabei ein großes Potenzial, die Entwicklung von
Fähigkeiten und Fertigkeiten gerade auch im Umgang mit Neuen Medien zu
unterstützen, sei es durch gezielte pädagogische Maßnahmen oder durch
das Bereitstellen von Ressourcen wie Lernräume oder Geräte zum
Ausleihen. Diese Rolle wird auch seitens der Interessenverbände
unterstrichen, so etwa vom Deutschen Bibliotheksverband (2016). Wichtig
ist hierbei, dass es nicht um das Bewahren vor potenziell negativen
Medienwirkungen gehen kann, sondern den Sozialisanden Selbststeuerung
zuerkannt werden muss (Süss 2010, S. 128). Letztere wiederum kann dann
erfolgreich sein, wenn es Bezugspersonen oder -instanzen wie
Bibliotheken gelingt, zu einem kritisch-reflexiven Medienumgang
anzuregen.

Wie eben angedeutet, können Bibliotheken Mediensozialisationsprozesse
allein dadurch positiv unterstützen, dass sie einen niedrigschwelligen,
das heißt vor allem einen kostenneutralen oder -günstigen Zugang zu
medialen Angeboten eröffnen. Bemerkenswert ist dieser Aspekt vor allem,
wenn man sich vor Augen führt, dass Medien im Lebensverlauf an Relevanz
gewinnen, wohingegen andere Sozialisatoren an Bedeutung verlieren (Süss
2004, S. 287). Für den Bestandsaufbau bedeutet dies, dass eine möglichst
breite Medienpalette angeboten werden sollte, um auch solche Gruppen von
Nutzer*innen anzusprechen, die weniger auf Printmedien fokussieren (Rose
2013, S. 425). Gerade im Bereich der Lesesozialisation, der eng mit der
Mediensozialisation verbunden ist, lässt sich, so Rose (2013, S. 425;
428; 433; 448), häufig noch eine starke Konzentration auf das
Trägermedium Buch feststellen. Jedoch erfolge das Lesen inzwischen
medienkonvergent. Damit einhergehend sei es auch Aufgabe der
Bibliotheken, beim Erschließen von multimodalen Texten zu unterstützen.
Analog dazu weist Ziegenhagen (1995, S. 135 f.) auf die Bedeutung von
Bibliotheken als Orte hin, an denen man Medienpluralität erfahren kann.
Wichtig sei, die jeweils spezifischen Qualitäten divergenter Medien
aufzuzeigen. Dabei müssten sich die vermittelnden Bibliothekar*innen
allerdings bewusst sein, dass die Mediensozialisation Heranwachsender
unter anderen Bedingungen stattfindet als die eigene. Die sogenannten
\enquote{Digital Natives} (Prensky 2001) wachsen mit neuen Technologien
auf und erleben selbige von früher Kindheit an als einen
selbstverständlichen Teil ihrer Lebenswelt.

Bibliotheken können durch ihre Angebote Mediensozialisationsprozesse zu
einem gewissen Grad positiv beeinflussen. Dabei sollte aber nicht
vergessen werden, dass das Gros der Sozialisationsimpulse in jenen
Kontexten verortet ist, in denen sich ein Mensch primär bewegt. Bei
Heranwachsenden spielt unter anderem, wie bereits angedeutet, das
familiäre Umfeld eine wichtige Rolle. Hier werden nicht nur
Mediennutzungsmuster grundgelegt, sondern auch die Genese von zum
Beispiel ästhetischen Präferenzen hängt davon ab, welche
Anregungspotentiale und Ressourcen Bezugspersonen zur Verfügung stellen
(vergleiche Hoffmann/Kutscha 2010, S. 221; 232). Familienmitglieder sind
zugleich auch \enquote{Vorbilder im Medienumgang, {[}\ldots{]} Partner
in gemeinsamen Medienaktivitäten und {[}\ldots{]} Gesprächspartner in
der Verarbeitung von Medienerfahrungen} (Süss 2004, S. 276). Das
Potenzial von Bibliotheken liegt hier darin, Multiplikator*innen im
Sozialisierungsprozess zu unterstützen, also jene Anregungs-,
Erfahrungs- und Erlebnisräume zu bieten, die manchen beispielsweise aus
finanziellen Gründen andernfalls verschlossen blieben. Dabei ist mit
Niesyto (2010, S. 57) darauf aufmerksam zu machen, dass soziokulturelle
Unterschiede im Medienumgang zunächst einmal nur auf für divergente
Muster oder Präferenzen, aber nicht zwangsläufig auf Formen von
Benachteiligung hinweisen. Letztere wären vor allem dann erkennbar, wenn
ein aktives und reflexives Medienhandeln durch mangelnde Ressourcen
verunmöglicht würde.

\hypertarget{fazit}{%
\section{Fazit}\label{fazit}}

Sozialisationstheorien und speziell die Mediensozialisationsforschung
können den Blick auf die bibliothekarische Praxis erweitern: Sie laden
dazu ein, darüber nachzudenken, wie sich Menschen in der Interaktion mit
ihrer Umwelt beziehungsweise mit einer zunehmend mediatisierten
Gesellschaft entwickeln. Die theoretisch-empirisch fundierten
Erkenntnisse zu den Faktoren, die Sozialisationskontexte,
Mediennutzungsmuster et cetera beeinflussen, können Bibliotheken dabei
helfen, zu verstehen, was sie selbst für gelingende
Sozialisationsprozesse beitragen können.

\hypertarget{literatur-und-quellen}{%
\section{Literatur und Quellen}\label{literatur-und-quellen}}

Aufenanger, Stefan: Mediensozialisation. In: Sander, Uwe; Gross,
Friederike von; Hugger, Kai-Uwe (Hrsg.): Handbuch Medienpädagogik.
Wiesbaden: VS Verlag für Sozialwissenschaften, 2008, S. 87--92. DOI:
\url{https://doi.org/10.1007/978-3-531-91158-8} (15.01.23).

Aufenanger, Stefan: Mediensozialisation. In: Sander, Uwe; Gross,
Friederike von; Hugger, Kai-Uwe (Hrsg.): Handbuch Medienpädagogik.
Living reference work. Wiesbaden: Springer VS, 2020. DOI:
\url{https://doi.org/10.1007/978-3-658-25090-4} (15.01.23).

Baacke, Dieter: Medienpädagogik. Tübingen: Max Niemeyer Verlag, 2007.
Grundlagen der Medienkommunikation, 1.

Bernhard, Armin: Sozialisationstheorie und Pädagogik. In: Bernhard,
Armin; Rothermel, Lutz; Rühle, Manuel (Hrsg.): Handbuch Kritische
Pädagogik: Eine Einführung in die Erziehungs- und Bildungswissenschaft.
Neuausg. Weinheim: Beltz Juventa, 2018, S. 302--318.

Bernhard, Armin; Rothermel, Lutz; Rühle, Manuel (Hrsg.): Handbuch
Kritische Pädagogik: Eine Einführung in die Erziehungs- und
Bildungswissenschaft. Neuausg. Weinheim: Beltz Juventa, 2018.

Borst, Eva: Theorie der Bildung: Eine Einführung. 4. überarb. Aufl.
Baltmannsweiler: Schneider Verlag Hohengehren, 2016. Pädagogik und
Politik, 2.

Bucher, Priska: Leseverhalten und Leseförderung: Zur Rolle von Schule,
Familie und Bibliothek im Medienalltag Heranwachsender. 2. Aufl. Zürich:
Verlag Pestalozzianum, 2005. Zürich: Univ., Diss., 2004.

Deutscher Bibliotheksverband e. V. (dbv): Bibliotheken vermitteln
Schlüsselqualifikationen für die digitale Gesellschaft: Stellungnahme
des Deutschen Bibliotheksverbandes (dbv) zum KMK-Strategiepapier
\enquote{Bildung in der digitalen Welt}. 2016. URL:
\url{https://www.bibliotheksverband.de/sites/default/files/2020-11/2016_07_15_dbv_Stellungnahme_KMK_Strategie_digitale_Bildung.pdf}
(15.01.23).

Dörpinghaus, Andreas; Uphoff, Ina Katharina: Grundbegriffe der
Pädagogik. 5. Aufl. Darmstadt: WBG, 2019. Einführung
Erziehungswissenschaft.

Fromme, Johannes: Mediensozialisation im Zeitalter der Neuen Medien:
Aufgaben und Perspektiven der erziehungswissenschaftlichen
Medienforschung. In: Wiedemann, Dieter; Volkmer, Ingrid (Hrsg.): Schöne
neue Medienwelten? Konzepte und Visionen für eine Medienpädagogik der
Zukunft. Bielefeld: GMK, 2006. Schriften zur Medienpädagogik, 38, S.
110--124.

Gudjons, Herbert; Traub, Silke: Pädagogisches Grundwissen: Überblick --
Kompendium -- Studienbuch. 13. aktual. Aufl. Bad Heilbrunn: Verlag
Julius Klinkhardt, 2020. UTB, 3092. DOI:
\url{https://doi.org/10.36198/9783838555232} (15.01.23).

Hoffmann, Dagmar: Plädoyer für eine integrative
Mediensozialisationstheorie. In: Hoffmann, Dagmar; Mikos, Lothar
(Hrsg.): Mediensozialisationstheorien: Modelle und Ansätze in der
Diskussion. 2. überarb. und erweit. Aufl. Wiesbaden: VS Verlag für
Sozialwissenschaften, 2010, S. 11--26. DOI:
\url{https://doi.org/10.1007/978-3-531-92249-2} (15.01.23).

Hoffmann, Dagmar; Kutscha, Annika: Medienbiografien: Konsequenzen
medialen Handelns, ästhetischer Präferenzen und Erfahrungen. In:
Hoffmann, Dagmar; Mikos, Lothar (Hrsg.): Mediensozialisationstheorien:
Modelle und Ansätze in der Diskussion. 2. überarb. und erweit. Aufl.
Wiesbaden: VS Verlag für Sozialwissenschaften, 2010, S. 221--243. DOI:
\url{https://doi.org/10.1007/978-3-531-92249-2} (15.01.23).

Hoffmann, Dagmar; Mikos, Lothar (Hrsg.): Mediensozialisationstheorien:
Modelle und Ansätze in der Diskussion. 2. überarb. und erweit. Aufl.
Wiesbaden: VS Verlag für Sozialwissenschaften, 2010. DOI:
\url{https://doi.org/10.1007/978-3-531-92249-2} (15.01.23).

Hoffmann, Dagmar; Mikos, Lothar: Warum dieses Buch? Einige einführende
Anmerkungen. In: Hoffmann, Dagmar; Mikos, Lothar (Hrsg.):
Mediensozialisationstheorien: Modelle und Ansätze in der Diskussion. 2.
überarb. und erweit. Aufl. Wiesbaden: VS Verlag für
Sozialwissenschaften, 2010, S. 7--10. DOI:
\url{https://doi.org/10.1007/978-3-531-92249-2} (15.01.23).

Hoffmann, Dagmar; Wagner, Ulrike: Editorial: Aufwachsen in komplexen
Medienwelten -- Neue Medientechnologien und erweiterte Medienensembles
in der Sozialisation von Kindern und Jugendlichen. In: Medien +
Erziehung. Jg. 57, H. 6. München: kopaed, 2013, S. 3--8.

Hurrelmann, Klaus: Sozialisation. In: Mertens, Gerhard; Frost, Ursula;
Böhm, Winfried; Koch, Lutz; Ladenthin, Volker (Hrsg.): Allgemeine
Erziehungswissenschaft I: Handbuch der Erziehungswissenschaft 1.
Paderborn: Ferdinand Schönigh, 2011. UTB, 8455, S. 319--329. DOI:
\url{https://doi.org/10.36198/9783838584553} (15.01.23).

Hurrelmann, Klaus; Bauer, Ullrich: Das Modell des produktiv
realitätsverarbeitenden Subjekts. In: Hurrelmann, Klaus; Bauer, Ullrich;
Grundmann, Matthias; Walper, Sabine (Hrsg.): Handbuch
Sozialisationsforschung. 8. überarb. Aufl. Weinheim: Beltz Verlag, 2015.
Pädagogik, S. 144--161.

Hurrelmann, Klaus; Bauer, Ullrich; Grundmann, Matthias; Walper, Sabine
(Hrsg.): Handbuch Sozialisationsforschung. 8. überarb. Aufl. Weinheim:
Beltz Verlag, 2015. Pädagogik.

Lange, Andreas: Sozialisation in der mediatisierten Gesellschaft. In:
Hurrelmann, Klaus; Bauer, Ullrich; Grundmann, Matthias; Walper, Sabine
(Hrsg.): Handbuch Sozialisationsforschung. 8. überarb. Aufl. Weinheim:
Beltz Verlag, 2015. Pädagogik, S. 537--556.

Löw, Martina; Geier, Thomas: Einführung in die Soziologie der Bildung
und Erziehung. 3. überarb. und erweit. Aufl. Opladen: Verlag Barbara
Budrich, 2014. Einführungstexte Erziehungswissenschaft. UTB, 8243. DOI:
\url{https://doi.org/10.36198/9783838584942} (15.01.23).

Mertens, Gerhard; Frost, Ursula; Böhm, Winfried; Koch, Lutz; Ladenthin,
Volker (Hrsg.): Allgemeine Erziehungswissenschaft I: Handbuch der
Erziehungswissenschaft 1. Paderborn: Ferdinand Schönigh, 2011. UTB,
8455. DOI: \url{https://doi.org/10.36198/9783838584553} (15.01.23).

Mikos, Lothar: Mediensozialisation als Irrweg: Zur Integration von
medialer und sozialer Kommunikation aus der Sozialisationsperspektive.
In: Hoffmann, Dagmar; Mikos, Lothar (Hrsg.):
Mediensozialisationstheorien: Modelle und Ansätze in der Diskussion. 2.
überarb. und erweit. Aufl. Wiesbaden: VS Verlag für
Sozialwissenschaften, 2010, S. 27--46. DOI:
\url{https://doi.org/10.1007/978-3-531-92249-2} (15.01.23).

Niesyto, Horst: Kritische Anmerkungen zu Theorien der Mediennutzung und
-sozialisation. In: Hoffmann, Dagmar; Mikos, Lothar (Hrsg.):
Mediensozialisationstheorien: Modelle und Ansätze in der Diskussion. 2.
überarb. und erweit. Aufl. Wiesbaden: VS Verlag für
Sozialwissenschaften, 2010, S. 47--66. DOI:
\url{https://doi.org/10.1007/978-3-531-92249-2} (15.01.23).

Prensky, Marc: Digital Natives, Digital Immigrants Part 1. In: On the
Horizon. Vol. 9, No.~5, Bradford: Emerald, 2001, S. 1--6. DOI:
\url{https://doi.org/10.1108/10748120110424816} (15.01.23).

Rose, Stefanie: Bibliothek -- Medien -- Lesen: Von der Buchausleihe zur
Leseförderung: Lesedidaktische Kompetenzen von Bibliotheken im Selbst-
und Fremdbild: Eine empirische Studie zu Angebot und Nachfrage
außerschulischer Leseförderung in Öffentlichen Bibliotheken unter
lesedidaktischer Perspektive. Dortmund: Univ., Diss., 2013. DOI:
\url{https://doi.org/10.17877/DE290R-11530} (16.01.23).

Rosebrock, Cornelia (Hrsg.): Lesen im Medienzeitalter: Biographische und
historische Aspekte literarischer Sozialisation. Weinheim: Juventa
Verlag, 1995.

Sander, Uwe; Gross, Friederike von; Hugger, Kai-Uwe (Hrsg.): Handbuch
Medienpädagogik. Wiesbaden: VS Verlag für Sozialwissenschaften, 2008.
DOI: \url{https://doi.org/10.1007/978-3-531-91158-8} (15.01.23).

Sander, Uwe; Gross, Friederike von; Hugger, Kai-Uwe (Hrsg.): Handbuch
Medienpädagogik. Living reference work. Wiesbaden: Springer VS, 2020.
DOI: \url{https://doi.org/10.1007/978-3-658-25090-4} (15.01.23).

Spanhel, Dieter: Sozialisation in mediatisierten Lebenswelten: Grundzüge
eines theoretischen Bezugsrahmens. In: Medien + Erziehung. Jg. 57, H. 6.
München: kopaed, 2013, S. 30--43.

Süss, Daniel: Mediensozialisation von Heranwachsenden: Dimensionen --
Konstanten -- Wandel. Wiesbaden: VS Verlag für Sozialwissenschaften,
2004. Zürich: Univ., Habil., 2004.

Süss, Daniel: Mediensozialisation zwischen gesellschaftlicher
Entwicklung und Identitätskonstruktion. In: Hoffmann, Dagmar; Mikos,
Lothar (Hrsg.): Mediensozialisationstheorien: Modelle und Ansätze in der
Diskussion. 2. überarb. und erweit. Aufl. Wiesbaden: VS Verlag für
Sozialwissenschaften, 2010, S. 109--130. DOI:
\url{https://doi.org/10.1007/978-3-531-92249-2} (15.01.23).

Vogel, Peter: Grundbegriffe der Erziehungs- und Bildungswissenschaft.
Opladen: Verlag Barbara Budrich, 2019. Einführung in die Erziehungs- und
Bildungswissenschaft, 2. UTB, 5271.

Wiedemann, Dieter; Volkmer, Ingrid (Hrsg.): Schöne neue Medienwelten?
Konzepte und Visionen für eine Medienpädagogik der Zukunft. Bielefeld:
GMK, 2006. Schriften zur Medienpädagogik, 38.

Ziegenhagen, Beate: Öffentliche Bibliotheken und Schulbibliotheken: Ohne
Einfluß auf die Mediensozialisation von Kindern und Jugendlichen? In:
Rosebrock, Cornelia (Hrsg.): Lesen im Medienzeitalter: Biographische und
historische Aspekte literarischer Sozialisation. Weinheim: Juventa
Verlag, 1995, S. 127--136.

%autor
\begin{center}\rule{0.5\linewidth}{0.5pt}\end{center}

\textbf{Alexandra C. Roth} ist Diplom-Bibliothekarin und hat einen Masterabschluss in \enquote{Medien und Bildung}. Sie ist seit 2009 bei der Münchner Stadtbibliothek beschäftigt. Derzeit absolviert sie ein Promotionsstudium am Institut für Bibliotheks- und Informationswissenschaft der Humboldt-Universität zu Berlin.

\end{document}