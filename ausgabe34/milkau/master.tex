\documentclass[a4paper,
fontsize=11pt,
%headings=small,
oneside,
numbers=noperiodatend,
parskip=half-,
bibliography=totoc,
final
]{scrartcl}

\usepackage{synttree}
\usepackage{graphicx}
\setkeys{Gin}{width=.4\textwidth} %default pics size

\graphicspath{{./plots/}}
\usepackage[greek,ngerman]{babel}
\usepackage[T1]{fontenc}
%\usepackage{amsmath}
\usepackage[utf8x]{inputenc}
\usepackage [hyphens]{url}
\usepackage{booktabs} 
\usepackage[left=2.4cm,right=2.4cm,top=2.3cm,bottom=2cm,includeheadfoot]{geometry}
\usepackage{eurosym}
\usepackage{multirow}
\usepackage[ngerman]{varioref}
\setcapindent{1em}
\renewcommand{\labelitemi}{--}
\usepackage{paralist}
\usepackage{pdfpages}
\usepackage{lscape}
\usepackage{float}
\usepackage{acronym}
\usepackage{eurosym}
\usepackage[babel]{csquotes}
\usepackage{longtable,lscape}
\usepackage{mathpazo}
\usepackage[normalem]{ulem} %emphasize weiterhin kursiv
\usepackage[flushmargin,ragged]{footmisc} % left align footnote
\usepackage{ccicons} 

%%%% fancy LIBREAS URL color 
\usepackage{xcolor}
\definecolor{libreas}{RGB}{112,0,0}

\usepackage{listings}

\urlstyle{same}  % don't use monospace font for urls

\usepackage[fleqn]{amsmath}

%adjust fontsize for part

\usepackage{sectsty}
\partfont{\large}

%Das BibTeX-Zeichen mit \BibTeX setzen:
\def\symbol#1{\char #1\relax}
\def\bsl{{\tt\symbol{'134}}}
\def\BibTeX{{\rm B\kern-.05em{\sc i\kern-.025em b}\kern-.08em
    T\kern-.1667em\lower.7ex\hbox{E}\kern-.125emX}}

\usepackage{fancyhdr}
\fancyhf{}
\pagestyle{fancyplain}
\fancyhead[R]{\thepage}

% make sure bookmarks are created eventough sections are not numbered!
% uncommend if sections are numbered (bookmarks created by default)
\makeatletter
\renewcommand\@seccntformat[1]{}
\makeatother


\usepackage{hyperxmp}
\usepackage[colorlinks, linkcolor=black,citecolor=black, urlcolor=libreas,
breaklinks= true,bookmarks=true,bookmarksopen=true]{hyperref}
%URLs hart brechen
\makeatletter 
\g@addto@macro\UrlBreaks{ 
  \do\a\do\b\do\c\do\d\do\e\do\f\do\g\do\h\do\i\do\j 
  \do\k\do\l\do\m\do\n\do\o\do\p\do\q\do\r\do\s\do\t 
  \do\u\do\v\do\w\do\x\do\y\do\z\do\&\do\1\do\2\do\3 
  \do\4\do\5\do\6\do\7\do\8\do\9\do\0} 
% \def\do@url@hyp{\do\-} 
\makeatother 

%meta
%meta

\fancyhead[L]{F. Milkau \\ %author
LIBREAS. Library Ideas, 34 (2018). % journal, issue, volume.
\href{http://nbn-resolving.de/}
{}} % urn 
% recommended use
%\href{http://nbn-resolving.de/}{\color{black}{urn:nbn:de...}}
\fancyhead[R]{\thepage} %page number
\fancyfoot[L] {\ccpd\ \href{https://creativecommons.org/publicdomain/mark/1.0/}{\color{black}Creative Commons Public Domain Mark 1.0}}  %licence
\fancyfoot[R] {ISSN: 1860-7950}

\title{\LARGE{Zur Einführung [Handbuch der Bibliothekswissenschaft, Erster Band, 1931]}}% title
\author{Fritz Milkau} % author

\setcounter{page}{1}

\hypersetup{%
      pdftitle={Zur Einführung [Handbuch der Bibliothekswissenschaft, Erster Band, 1931]},
      pdfauthor={Fritz Milkau},
      pdfcopyright={CC Public Domain Mark 1.0},
      pdfsubject={LIBREAS. Library Ideas, 34 (2018).},
      pdfkeywords={Bibliothekswissenschaft, Geschichte, Selbstverständnis, Berufsbild},
      pdflicenseurl={https://creativecommons.org/publicdomain/mark/1.0/},
      pdfcontacturl={http://libreas.eu},
      baseurl={http://libreas.eu},
      pdflang={de},
      pdfmetalang={de}
     }



\date{}
\begin{document}

\maketitle
\thispagestyle{fancyplain} 

%abstracts
\begin{abstract}
\noindent Erstabdruck: Fritz Milkau (Hrsg.): Handbuch der Bibliothekswissenschaft,
Erster Band. Leipzig: Otto Harrassowitz, 1931:V--XI
\end{abstract}

%body

SCHON beim Titel wird manch einer stocken: Bibliothekswissenschaft? Gibt
es denn so etwas? Ja, das \emph{Handbuch der Bibliothekswissenschaft}
muß doch wohl glauben, daß es so etwas gibt, und wenn es auch nicht
gerade darauf ausgeht, den Zweifler zu bekehren, so ist es doch sicher,
ihn davon zu überzeugen, daß es zum mindesten praktische Gründe gibt,
die die gewählte Benennung seines Gegenstandes rechtfertigen. Eine etwas
wunderliche Situation, wenn man bedenkt, daß das immer noch gelegentlich
in seiner Existenzberechtigung angefochtene Wort nun seit einem
Jahrhundert und länger durch unsere Literatur läuft.\footnote{Sehr
  eingehend hat sich zuletzt mit der Frage beschäftigt \textsc{Georg
  Leidinger} in seinem auf der 24. Tagung des Vereins deutscher
  Bibliothekare gehaltenen Vortrag \emph{Was ist
  Bibliothekswissenschaft}? (ZfB XLV 1928 S. 440--454.)}

Geprägt hat es \textsc{Martin Schrettinger}, ein Mann aus der Zucht des
h. Benedikt. Also eine Provenienz, die sich sehen lassen kann. Als
Bibliothekar seines Klosters (Weißenohe bei Nürnberg) in den Geschmack
der Bücher gekommen, durch die Säkularisation entwurzelt und dann in der
Hofbibliothek zu München vor gewaltige, plötzlich eingebrochene
Büchermassen gestellt, die nach einem festen Ordnungsplan schreien,
rafft er Anno 1808 mit dem Mut der Jugend, nicht mehr als 35 Jahre
zählend, seine bibliothekarischen Erfahrungen und Anschauungen zusammen
und gibt dem auf eigene Kosten gedruckten Buch frisch und unbefangen den
Titel \emph{Versuch eines vollständigen Lehrbuchs der
Bibliothek-Wissenschaft}. Und als er 1834, jetzt Unterbibliothekar und
Hofkaplan des Königs, sein Buch in starker Verkürzung von neuem ausgehen
läßt, da zeigt der Titel bereits die zuversichtliche Fassung
\emph{Handbuch der Bibliothek-Wissenschaft}, diesmal sogar mit dem
interessanten, bisher kaum beachteten Zusatz: \emph{Auch als}
\emph{Leitfaden zu Vorlesungen über die Bibliothek-Wissenschaft zu
gebrauchen}. Daß er sich der Neuheit seines Vorgehens und wohl auch der
Neuheit seiner Benennung bewußt gewesen ist, daran läßt er selbst keinen
Zweifel, wenn er in der Vorrede zum ersten Bande sich für fest überzeugt
erklärt, \emph{mit gegenwärtigem Lehrbuche der}
\emph{Bibliothek-Wissenschaft eine ganz neue Bahn gebrochen zu haben},
oder wenn er in dem erst 1829 erschienenen zweiten Band der ersten
Ausgabe (S. 3f.) die Bibliothek-Wissenschaft ein bis jetzt noch zartes
Pflänzchen nennt, das der Boreas einer unfreundlichen Kritik in seinem
Wachstum hindern könne. An die Anfechtbarkeit seiner Namengebung aber
hat er ebenso zweifellos nicht gedacht. Wie sollte er auch, da man das
Wort \emph{Wissenschaft} damals viel freier brauchte als heute und jedes
Wissen von einer einzelnen Sache so bezeichnete? Auch fiel es damals
kaum jemand ein, daran Anstoß zu nehmen: \textsc{Friedrich Adolf Ebert}
braucht die neue Prägung wie etwas Selbstverständliches, nicht etwa in
Anführungsstrichen; der Professor und Unterbibliothekar in Kiel
\textsc{W. Ratjen}, der \textsc{Christian Molbechs} Buch \emph{Om
offentlige Bibliotheker, Bibliothekarer og det man} \emph{har kaldet
Bibliotheksvidenskab} ins Deutsche zu übertragen hat, gibt seiner
Übersetzung den Titel \emph{Über Bibliothekswissenschaft} (1833); der
Ilmenauer Diakonus \textsc{Johann August Friedrich Schmidt} läßt 1840 in
Weimar ein \emph{Handbuch der Bibliothekswissenschaft} erscheinen; die
in demselben Jahr begründeten bibliothekarischen Zeitschriften
\emph{Naumanns Serapeum} und \emph{Petzholdts Anzeiger}, nebenbei die
ersten ihrer Art in der Welt, nennen beide auf dem Titelblatt als ihr
Arbeitsgebiet die \emph{Bibliothekswissenschaft}; \textsc{Edmund
Zoller}, der spätere Stuttgarter Hofbibliothekar, veröffentlicht, ganze
23 Jahre alt, 1846 seine viel beachtete \emph{Bibliothekwissenschaft im
Umrisse}, und der von der Kritik hart mitgenommene Buchhändler und
Bibliothekssekretär des Germanischen Museums in Nürnberg \textsc{Johann
Georg} \textsc{Seizinger} schließlich 1863 seine \emph{Theorie und
Praxis der Bibliothekswissenschaft}.

Also \emph{Bibliothekswissenschaft} allerwegen. Erst jener um die Mitte
des neunzehnten Jahrhunderts einsetzende tiefgreifende Wandel in den
Zielen und Methoden der Forschung ist es, der mit seiner strengeren
Auffassung des Begriffs der Wissenschaft die so lange kaum beanstandete
Benennung \textsc{Schrettingers} in Mißkredit bringt. Und so schnell
wirkt sich diese Umstellung aus, daß das nach dem Verschwinden von
Naumanns Serapeum und kurz vor dem Ende von Petzholdts Anzeiger 1884
begründete neue Bibliotheksorgan sich bereits Zentralblatt für
\emph{Bibliothekswesen} nennt und der 1886 in Göttingen für die
Interessen der Bibliothek errichtete Lehrstuhl beileibe nicht die
Bestimmung für \emph{Bibliothekswissenschaft} erhält, sondern schüchtern
und unlogisch zugleich als Professur für
\emph{Bibliothekshilfswissenschaften} signiert wird. Aber des Wandels
ist auch hier kein Ende. War die Reaktion gegen die Sorglosigkeit in der
Verwendung des Wortes \emph{Wissenschaft} zu streng gewesen, so hat sie
heute einer kaum beschränkten Weitherzigkeit Platz gemacht, vielleicht
aus der Erkenntnis heraus, daß das, was das eigentliche Wesen der
wissenschaftlichen Arbeit ausmacht, die Ehrfurcht vor der Wahrheit und
die innere Freiheit, nicht an bestimmte Objekte gebunden ist, sondern in
jeder Sphäre menschlicher Tätigkeit bewährt werden kann. So sehen wir
denn auch heute die \emph{Bibliothekswissenschaft} selbst in die
amtliche Sprache aufgenommen und die Namengebung \textsc{Martin
Schrettingers}, dem die Bibliotheksgeschichte bisher auch sonst manches
schuldig geblieben ist,\footnote{\textsc{Adolf Hilsenbeck}, \emph{Martin
  Schrettinger und die Aufstellung in der Kgl. Hof- u. Staatsbibliothek
  in München}. Vortrag auf der 15. Tagung des Vereins deutscher
  Bibliothekare (ZfB XXXI 1914 S. 427--433).} wieder zu Ehren gebracht.
Aber es soll nicht verschwiegen werden, daß auch heute noch solche
gefunden werden, die den Kopf schütteln. Das Handbuch, das sei
ausdrücklich gesagt, will es ihnen nicht verwehren.

So viel -- schon zu viel -- über das Wort. Und wie steht es nun um die
Sache? Was wollte \textsc{Schrettinger} mit dem neuen Namen sagen? Und
was haben wir heute darunter zu verstehen? Zur Beantwortung der ersten
Frage wird am besten der Erfinder selbst herangezogen, aber nicht mit
der prätentiös philosophisch aufgezogenen Definition, die zu Anfang
seines Lehrbuchs zu lesen ist.\footnote{Heft I S. 16:
  Bibliothek-Wissenschaft ist der auf feste Grundsätze systematisch
  gebaute und auf einen obersten Grundsatz zurückgeführte Inbegriff
  aller zur zweckmäßigen Einrichtung einer Bibliothek erforderlichen
  Lehrsätze.} Er kann sich gottlob auch verständlich ausdrücken, wie in
der Vorrede desselben Heftes, wo er (S. IV\,f.) seinen Vorgängern den
Vorwurf macht, sie hätten sich in ihren Schriften hauptsächlich damit
beschäftigt, die Bibliothekare über den verschiedenen Wert der Bücher zu
belehren und ihnen \emph{Klugheits-Regeln in Betreff ihres}
\emph{Ankaufes} an die Hand zu geben, die Einrichtung einer Bibliothek
selbst aber, gleichsam als eine unbedeutende Nebensache, nur obenhin
berührt, und wo er dann also fortfährt: \emph{Die Idee eines
Bibliothekärs setzt die Bücherkunde voraus, da es lächerlich wäre,
jemanden zum Bibliothekäre zu machen, der nicht schon vorläufig
Literator wäre; aber nicht jeder Literator ist darum auch schon zum
Bibliothekäre geeignet. Bücher sammeln kann zwar der Erstere; aber aus
den gesammelten Büchern eine brauchbare Bibliothek zu bilden ist das
wesentliche Geschäft des Letztern, und dazu gehören praktische
Kenntnisse, die ein ganz eigenes Studium erfordern, und welche ich unter
dem Begriffe Bibliothek-Wissenschaft zusammengefaßt habe}. Nach dem
Vorangegangenen ist man gründlich überrascht. Aber so ist es in der Tat:
Nur selten und auch dann nur ganz leicht wird in dem Buche eine Frage
gestreift, die über die Praxis hinausgeht, und das Neue -- das
Bahnbrechende nennt es der Verfasser -- ist nichts anderes als die den
Vorgängern gegenüber freilich sehr viel eindringlichere Behandlung aller
bei der Einrichtung und Verwaltung der Bibliothek erforderlichen
Arbeiten und die straffere Formulierung der nötigen Anweisungen.
Immerhin verdienstvoll genug, um ihm die großen Worte nachzusehen.

Schwieriger ist die zweite Frage. Man hat, wie wir gesehen haben,
\textsc{Schrettingers} Benennung bis in die Mitte des neunzehnten
Jahrhunderts festgehalten, ist aber auch, wenn man von dem Ilmenauer
Diakonus, bei dem sich Ansätze zu einer Weiterbildung finden, absieht,
über seine Begrenzung des Begriffs nicht hinausgekommen, auch der
vortreffliche \textsc{Arnim Graesel} nicht, an dessen Hand ganze
Generationen Bibliotheksbeflissener ihre ersten Schritte getan haben und
dessen Name, nicht in den deutschen Bibliotheken allein, noch lange in
Dankbarkeit genannt werden wird.\footnote{\emph{Grundzüge der
  Bibliothekslehre} (Leipzig 1890; Italienische Übersetzung von
  \textsc{A. Capha} 1893; Französische Bearbeitung von \textsc{Jules
  Laude} 1897). --- 2. Aufl. u. d. T. \emph{Handbuch der
  Bibliothekslehre} (Leipzig 1902).} Die Zeit war noch nicht gekommen.
Erst die Erweckung des Bibliothekars durch die Schaffung der
bibliothekarischen Laufbahn bereitet die Kräfte, mit denen die alte
Engigkeit der Auffassung vom Wesen der Bibliothek und den Aufgaben des
Bibliothekars überwunden wird. Der gewaltig beschleunigte
Arbeitsrhythmus des neuen Deutschlands, der alle Kreise erfaßt und
überall Anforderung wie Leistung ins Ungeahnte steigert, macht vor der
Bibliothek nicht Halt. Was ein Jahrhundert lang tauben Ohren gepredigt
ist, das wird jetzt Gegenstand ernstester Prüfung: ist es am Ende nicht
doch richtig, daß der Bibliothekar gewisser besonderer, aus der Eigenart
seines Berufs sich ergebender Kenntnisse bedarf? Aller Völker und aller
Zeiten Sprachen tauchen vor ihm auf, gehen durch seine Hände. Wäre es da
nicht seine Pflicht, von ihren Zusammenhängen etwas mehr zu wissen als
der Durchschnittsgebildete? Muß er, der Traditionsgebundene, dem
Überkommenen gegenüber nahezu Machtlose nicht wenigstens dessen
Geschichte kennen, um sich das unentbehrliche Mindestmaß von Freiheit zu
retten? Sollte er, dessen berufliches Leben sich um das Buch dreht,
nicht eine Vorstellung von dem Weg haben, den es durch die Jahrtausende
gemacht hat, von seiner Form und deren Wandel und von der Zier seines
Kleides, von dem Wunder des Alphabets und der Entwicklung der Schrift,
von der Ausbildung und dem stürmischen Siegeslauf der fünfundzwanzig
Bleisoldaten Gutenbergs über die Erde, von dem Gold- und Farbenglanz der
Miniatur, den feinen Reizen der Illustration?

Und noch weitere Überlegungen melden sich, die Wünschbarkeit wenigstens
eines Überblicks über diese Dinge im Besitze des Bibliothekars zu
steigern. Anders als Archiv und Museum braucht die Bibliothek, deren
Interessenkreis das ganze Leben umspannt, Arbeiter verschiedenster
Vorbildung, vom Theologen bis zum Techniker, Landwirt und Kaufmann, und
andererseits verlangt es die Eigenart der bibliothekarischen Arbeit, daß
diese von Hause aus einander fremden Kräfte jetzt einander -- das Wort
im eigentlichsten Sinne genommen -- in die Hand arbeiten. Sollte es da
nicht gut und der Einheitlichkeit der Arbeit förderlich sein, wenn diese
verschiedenen Geister, die zunächst nur das Räderwerk des Dienstes
zusammenhält, nun auch noch durch die Gemeinsamkeit eines wenn auch
begrenzten, so doch eigenartigen geistigen Besitzes einander näher
gebracht würden? Und, das Problem von einer anderen Seite gesehen: es
gibt keinen gelehrten Beruf, der seine Angehörigen in so hohem Grade der
Einrostung und Verstaubung aussetzt wie der des Bibliothekars, bei dem
auch die großen Leistungen nur durch Überwindung unendlicher Kleinarbeit
erreicht werden, einer Kleinarbeit, die zwar gelehrte Kenntnisse aller
Art erfordert, aber nur ausnahmsweise zu jener Anspannung der geistigen
Kräfte zwingt, die deren eigentliche Nährmutter ist. Da gibt es für den
Bibliothekar nur ein \textgreek{άντιφάρμαχον}, und das ist die wissenschaftliche
Arbeit. Wohl dem, dem es gelingt, sie von der Universität ins Leben
hinüberzuretten. Wie aber wird der Mediziner, der Physiker, der
Chemiker, wie werden alle die das anfangen, deren Arbeit an Beobachtung
und Experiment gebunden ist? Bieten sich da nicht ungerufen jene
besonderen Wissensgebiete dar, mit deren Objekten schon der tägliche
Dienst den Bibliothekar ständig zusammenbringt? Jene Wissensgebiete, für
die das Lehrprogramm der Universität in der Regel keinen Raum hat und
die daher selbst dem historisch oder philologisch vorgebildeten
Bibliothekar in seiner Studienzeit kaum begegnet sind? Die Geschichte
des Buchs, die Entwicklung der Schrift und des Drucks, die Geschichte
der Buchmalerei und der Buchillustration, des Bucheinbands und des
Buchhandels, die Bibliographie und die Gelehrtengeschichte und
schließlich, das Ganze im Zusammenhang überschauend, die Geschichte der
Bibliotheken -- alle diese Disziplinen finden sich nur ausnahmsweise in
den Vorlesungsverzeichnissen unserer Universitäten, und wenn ihrer keine
daran denken darf, sich als selbständig und gleichberechtigt neben die
alten Fakultätswissenschaften zu stellen, so ist doch andererseits kaum
eine darunter, der selbst die strengste Observanz, soweit Probleme und
Arbeitsmethoden maßgebend sind, den Eintritt in den geheiligten Bezirk
der Wissenschaft verwehren möchte. Fallen sie also nicht auf die
natürlichste Weise von der Welt der Bibliothek zu und sind sie nicht
bereits zu einem guten Teil in der Bibliothek beheimatet?\footnote{\textsc{Erich
  v. Rath}, \emph{Die Forschungsaufgaben der Bibliotheken}
  (Forschungsinstitute hrsg. von \textsc{Ludolph Brauer} u. a. I S.
  136--147, Hamburg 1930). --- \textsc{Fritz Milkau}.
  \emph{Bibliothekswesen} (Aus fünfzig Jahren deutscher Wissenschaft:
  Festschrift f. \textsc{Friedrich Schmidt-ott}, 1930 S. 41\,ff.).}

Das ungefähr mögen die Erwägungen sein, von denen die
Unterrichtsverwaltungen jetzt ausgehen,\footnote{Preußen mit dem Erlaß
  vom 15. Dez. 1893 voran, die anderen Länder wenig später sich
  anschließend.} wenn sie bei der Schaffung der bibliothekarischen
Laufbahn für die Zulassung zur Ausbildung nicht mehr verlangen als
Staatsexamen und Promotion, die endgültige Übernahme in den Staatsdienst
aber von dem in einer Prüfung zu erbringenden Nachweis einer gewissen
Vertrautheit mit eben jenen besonderen Wissensgebieten abhängig machen.
Nicht als hätten sie nun -- das muß doch der Sicherheit Wegen noch
ausdrücklich gesagt werden -- lauter Spezialisten auf dem Gebiete des
Buch- und Bibliothekswesens heranzüchten oder gar solche Anwärter, die
ihrem Studienfach treu geblieben waren, davon abtrünnig machen wollen.
Daran hat tatsächlich niemand gedacht, und nur, weil man vom
Selbstverständlichen nicht spricht, ist in den Prüfungsordnungen kein
Wort darüber zu finden, daß die heute glücklich durchgedrungene
Auffassung von der Aufgabe der Bibliothek durch die neuen Ordnungen
nicht im geringsten geändert ist und daß der Hauptakzent der Ausbildung
nach wie vor auf die Erziehung zur rückhaltlosen Hingabe an den Dienst
zu legen ist. Wenn man also hier von einer Lücke in den neuen Ordnungen
nicht sprechen kann, so darf man vielleicht einen Mangel darin sehen,
daß sie es ganz und gar dem Anwärter überlassen, wie er, ohnehin durch
den neuen Dienst ungewöhnlich in Anspruch genommen, sich nebenher die
verlangten Kenntnisse aneignen soll. Denn hier findet er keine festen
Grenzen, keine \emph{ad hoc} verfaßten Kompendien, keine Tradition, und
an eine Orientierung an der Hand der großen oder größeren
Veröffentlichungen ohne Lehrmeister läßt schon der Mangel an Zeit nicht
denken. Zweifellos ein gewisser Notstand. Wie man versucht hat, ihn zu
überwinden, wie zuerst die großen und dann auch kleinere Bibliotheken
unter Heranziehung der Beamten für den Unterricht systematisch
aufgebaute Kurse einführten,\footnote{Hier München voran 1905.} wie dann
die Unterrichtsverwaltungen mit Lehraufträgen nachhalfen und wie
schließlich Preußen mit der Errichtung des BIBLIOTHEKSWISSENSCHAFTLICHEN
INSTITUTS bei der Universität Berlin (1928) -- neben sechs
Bibliothekaren der Staatsbibliothek sind dort ebensoviel Professoren der
Universität um die theoretische Ausbildung der Anwärter bemüht -- eine
Form gefunden hat, die sich hoffentlich bewähren wird: das alles soll im
zweiten Bande ausführlicher behandelt werden. Hier muß es genügen, den
Zusammenhang mit dem vorliegenden Werk festzustellen: eben jene Not des
Nachwuchses ist es, die den lange gehegten Plan eines Handbuchs der
Bibliothekswissenschaft zur Reife gebracht hat.

Sieht man allein auf den Titel, so hat dies Handbuch also Vorgänger
genug; es hat keinen, wenn man den Inhalt, Anlage und Absicht ins Auge
faßt, wenigstens nicht in der überreichen Fülle der deutschen
Bibliotheksliteratur, und auch das Ausland bietet nur zwei Beispiele
gleichartiger Unternehmungen.\footnote{Handbog i Bibliotekskundskab udg.
  af \textsc{Svend Dahl}, 3. Udg. Bd I--II (København 1924--1930).
  Erscheint gleichzeitig in schwedischer Bearbeitung. -- Československé
  Knihovnictví. Red. Zdeněk V. \textsc{Tobolka} (V Praze 1925).} Alle
die älteren Veröffentlichungen, die mit dem Anspruch auftreten, das
Ganze des Bibliothekswesens zu umfassen, beschränken sich auf eine mehr
oder weniger kritische und nur ausnahmsweise auch leise historisch
gefärbte Behandlung der bibliothekarischen Praxis. Das vorliegende
Handbuch will mehr. Und zwar will es nicht allein, was sich von selbst
versteht, der Tatsache Rechnung tragen, daß die deutschen Bibliotheken
und nicht viel anders die ausländischen in den letzten vier bis fünf
Jahrzehnten mit ihrem enormen Aufstieg zum Teil auch Ziele und
Arbeitsmethoden geändert haben. Darüber hinaus will es einmal
grundsätzlich jede Erscheinung bei der Wurzel fassen, um sie durch
geschichtliche Beleuchtung ihrer Entwicklung verständlich zu machen, und
weiter will es, was die Hauptsache ist und was es am stärksten von der
älteren Literatur unterscheidet, alle die besonderen theoretischen
Kenntnisse zur Darstellung bringen, deren ein richtig konstruierter
Bibliothekar in seinem Beruf nicht entraten kann und deren Aneignung
allein der dienstlichen Praxis zu überlassen die bisherigen Erfahrungen
keineswegs empfehlen.

Fern von aller Prätension, Pionierarbeit zu leisten, will sich das
Handbuch doch nicht daran genug sein lassen, lediglich den gegenwärtigen
Stand unseres Wissens rein objektiv zur Anschauung zu bringen; vielmehr
hofft es, auch durch eigene Kritik und Vorlegung eigner
Forschungsergebnisse seiner Sache zu dienen. Entsprechend jedoch der ihm
zugrunde liegenden Absicht will es sich überall auf das Wesentliche und
Notwendige, auf die führende Linie und die entscheidenden Punkte
beschränken, nirgends eine Erschöpfung des Stoffes anstreben und auch in
den Literaturangaben jeden Überfluß vermeiden. Andererseits aber will es
auch nichts versäumen, was der Leser braucht, um sich mühelos über die
Herkunft des Mitgeteilten und über den weiterführenden Weg zu
unterrichten -- das alles jedoch, und darauf ist das größte Gewicht
gelegt worden, ohne die Lesbarkeit zu gefährden. Und um auch die
Einheitlichkeit des Ganzen leidlich zu sichern, wurde die Bearbeitung
der einzelnen Gegenstände ausschließlich Bibliothekaren in die Hände
gelegt.

Daß der Ausgangspunkt immer und überall die deutsche Bibliothek ist, das
versteht sich von selbst, wie es andererseits bei der grundsätzlichen
Einstellung der deutschen Wissenschaft ebenso selbstverständlich ist,
daß die Verhältnisse des Auslands sorgfältig berücksichtigt worden sind.

Nach dem Vorstehenden braucht kaum gesagt zu werden, daß es zunächst die
wissenschaftliche Bibliothek ist, auf die das Handbuch abgestellt ist.
Der jungen, so schnell aufgeblühten Schwester, der Volksbibliothek, etwa
nebenher durch Herausstellung ihrer Eigenheiten bei den einzelnen
Abschnitten gerecht zu werden, hat sich als unmöglich erwiesen, so daß
ihr eine zusammenhängende Sonderbehandlung am Schluß eingeräumt werden
mußte. Von einem Register ist abgesehen worden, Weil es durch das mit
dem zweiten Band folgende Gesamtregister überflüssig werden würde.

Das wäre alles, was zur Einführung in das Verständnis des Buches zu
sagen wäre, und so bleibt dem Herausgeber nichts anderes zu tun übrig,
als seiner Dankbarkeit Ausdruck zu geben gegen alle, die ihm geholfen
haben, das Werk zu gestalten wie es hier vorliegt. Herzlichsten Dank
also dem Präsidenten der Notgemeinschaft der Deutschen Wissenschaft
Exzellenz \textsc{Schmidt-Ott}, der als überzeugter alter Freund der
Bibliotheken ermutigend seine Hand auftat, als der Plan noch kaum feste
Gestalt gewonnen hatte, und mit ihm seinem treuen Berater Geheimrat
\textsc{Siegismund}, der die Arbeit helfend und ratend durch alle
Stadien begleitete, und endlich als dem Dritten im Bunde der Förderer,
dem Generaldirektor der Staatsbibliothek Geheimrat \textsc{Krüss}, der
dem Herausgeber in vorbildlicher Gastlichkeit Arbeitsbedingungen
geschaffen hat, wie sie freundlicher nicht gewünscht werden können. Wie
sehr weiter die Mitarbeiter und der Verleger, Herr \textsc{Hans
Harrassowitz}, ihn verpflichtet haben, das zeigt der heute der
Öffentlichkeit übergebene Band deutlicher als Worte es darzutun
vermöchten. Für den Leser aber möchte der Herausgeber sich das Gebet
CASSIODORS, des Schutzpatrons der Bibliotheken und der Bibliothekare zu
eigen machen, nicht ohne den \emph{profectus} auch auf die Güter dieser
Welt auszudehnen.

BERLIN, den 5. März 1931

\emph{Fritz Milkau}

%autor
\begin{center}\rule{0.5\linewidth}{\linethickness}\end{center}

\textbf{Fritz Milkau}, Bibliothek in Königberg, Bonn und Berlin, später
Bibliotheksdirektor Universitätsbibliothek Greifswald und
Universitätsbibliothek Breslau, Generaldirektor der Preußischen
Staatsbibliothek zu Berlin, ab 1928 Honorarprofessor
Bibliothekswissenschaft (Friedrichs-Wilhelm-Universität Berlin), erster
Direktor des Instituts für Bibliothekswissenschaft an dieser
Universität. († 1934)

\end{document}
