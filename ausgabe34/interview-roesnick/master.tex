\documentclass[a4paper,
fontsize=11pt,
%headings=small,
oneside,
numbers=noperiodatend,
parskip=half-,
bibliography=totoc,
final
]{scrartcl}

\usepackage{synttree}
\usepackage{graphicx}
\setkeys{Gin}{width=.4\textwidth} %default pics size

\graphicspath{{./plots/}}
\usepackage[ngerman]{babel}
\usepackage[T1]{fontenc}
%\usepackage{amsmath}
\usepackage[utf8x]{inputenc}
\usepackage [hyphens]{url}
\usepackage{booktabs} 
\usepackage[left=2.4cm,right=2.4cm,top=2.3cm,bottom=2cm,includeheadfoot]{geometry}
\usepackage{eurosym}
\usepackage{multirow}
\usepackage[ngerman]{varioref}
\setcapindent{1em}
\renewcommand{\labelitemi}{--}
\usepackage{paralist}
\usepackage{pdfpages}
\usepackage{lscape}
\usepackage{float}
\usepackage{acronym}
\usepackage{eurosym}
\usepackage[babel]{csquotes}
\usepackage{longtable,lscape}
\usepackage{mathpazo}
\usepackage[normalem]{ulem} %emphasize weiterhin kursiv
\usepackage[flushmargin,ragged]{footmisc} % left align footnote
\usepackage{ccicons} 

%%%% fancy LIBREAS URL color 
\usepackage{xcolor}
\definecolor{libreas}{RGB}{112,0,0}

\usepackage{listings}

\urlstyle{same}  % don't use monospace font for urls

\usepackage[fleqn]{amsmath}

%adjust fontsize for part

\usepackage{sectsty}
\partfont{\large}

%Das BibTeX-Zeichen mit \BibTeX setzen:
\def\symbol#1{\char #1\relax}
\def\bsl{{\tt\symbol{'134}}}
\def\BibTeX{{\rm B\kern-.05em{\sc i\kern-.025em b}\kern-.08em
    T\kern-.1667em\lower.7ex\hbox{E}\kern-.125emX}}

\usepackage{fancyhdr}
\fancyhf{}
\pagestyle{fancyplain}
\fancyhead[R]{\thepage}

% make sure bookmarks are created eventough sections are not numbered!
% uncommend if sections are numbered (bookmarks created by default)
\makeatletter
\renewcommand\@seccntformat[1]{}
\makeatother


\usepackage{hyperxmp}
\usepackage[colorlinks, linkcolor=black,citecolor=black, urlcolor=libreas,
breaklinks= true,bookmarks=true,bookmarksopen=true]{hyperref}
%URLs hart brechen
\makeatletter 
\g@addto@macro\UrlBreaks{ 
  \do\a\do\b\do\c\do\d\do\e\do\f\do\g\do\h\do\i\do\j 
  \do\k\do\l\do\m\do\n\do\o\do\p\do\q\do\r\do\s\do\t 
  \do\u\do\v\do\w\do\x\do\y\do\z\do\&\do\1\do\2\do\3 
  \do\4\do\5\do\6\do\7\do\8\do\9\do\0} 
% \def\do@url@hyp{\do\-} 
\makeatother 

%meta
%meta

\fancyhead[L]{Th. Roesnick, F. Härting, M. Brauer\\ %author
LIBREAS. Library Ideas, 34 (2018). % journal, issue, volume.
\href{http://nbn-resolving.de/}
{}} % urn 
% recommended use
%\href{http://nbn-resolving.de/}{\color{black}{urn:nbn:de...}}
\fancyhead[R]{\thepage} %page number
\fancyfoot[L] {\ccLogo \ccAttribution\ \href{https://creativecommons.org/licenses/by/3.0/}{\color{black}Creative Commons BY 3.0}}  %licence
\fancyfoot[R] {ISSN: 1860-7950}

\title{\LARGE{Interview mit \\ Vivien Petras und Elke Greifeneder}}% title
\author{Thomas Roesnick (Interviewer) \\ Felicitas Härting \& Miriam Brauer (Transkription und Korrektur)} % author

\setcounter{page}{1}

\hypersetup{%
      pdftitle={Interview mit Vivien Petras und Elke Greifeneder},
      pdfauthor={Thomas Roesnick (Interviewer), Felicitas Härting \& Miriam Brauer (Transkription und Korrektur)},
      pdfcopyright={CC BY 3.0 Unported},
      pdfsubject={LIBREAS. Library Ideas, 34 (2018).},
      pdfkeywords={Bibliothekswissenschaft, Institut für Bibliotheks- und Informationswissenschaft, Humboldt-Universität zu Berlin, Projektseminar, Buchprojekt, Studium},
      pdflicenseurl={https://creativecommons.org/licenses/by/3.0/},
      pdfcontacturl={http://libreas.eu},
      baseurl={http://libreas.eu},
      pdflang={de},
      pdfmetalang={de}
     }



\date{}
\begin{document}

\maketitle
\thispagestyle{fancyplain} 

%abstracts

%body
\textbf{TR: Sie verkörpern eine neue Generation am IBI: Als
geschäftsführende Institutsdirektorin, Frau Petras, und stellvertretende
geschäftsführende Institutsdirektorin, Frau Greifeneder. Jede Generation
hat einen Anfang und deshalb ist die erste Frage: Wie kam es zu der
Entscheidung \enquote{Bibliothekswissenschaft} am IBI zu studieren?}

\textbf{VP:} Meine erste Entscheidung war, dass ich in Berlin studieren
wollte. Außerdem habe ich

Bibliotheken schon immer gemocht, sie zählten zu meinen
Lieblingsaufenthaltsorten in der Kindheit.

Jedoch wusste ich nicht genau, was ich werden wollte -- auch zu meinen
Abiturzeiten nicht. Blauäugig wie ich war, nahm ich mir also eine
Broschüre mit allen Studienfächern, die man in Berlin studieren konnte
und ging mit einem Bleistift die Liste durch: Möchte ich nicht, möchte
ich vielleicht\ldots{} Dann habe ich mir ein paar Studienfächer
herausgesucht und \enquote{Bibliothekswissenschaft} mit dem Schwerpunkt
Dokumentationswesen, was ich letztendlich studiert habe, ist mir dabei
ins Auge gesprungen. Mit meinem Vater fuhr ich schließlich zu einer
Informationsveranstaltung, sowohl für Bibliothekswissenschaft als auch
für Kunstgeschichte --- weil ich dieses Fach auch studieren wollte, fürs
Herz. Und um alles abzusichern, habe ich auch noch
Betriebswirtschaftslehre studiert. Wir waren also bei dieser
Informationsveranstaltung, früh im Jahr 1995, hier im Gebäude des IBI.
Und eine Dozentin, Frau Iris Schwarz, bei der ich später auch Lehre
hatte, stellte uns etwas ganz Neues vor: das Internet. Sie erklärte, wie
man darin recherchieren könne und da dachte ich mir: Das ist es, was ich
einmal machen möchte. Und so war die Entscheidung getroffen. Ich habe
dann tatsächlich Bibliothekswissenschaft mit Schwerpunkt
Dokumentationswesen, BWL und Kunstgeschichte studiert und so alle meine
Herzenswünsche vereint.

\textbf{EG:} Bei mir war es ganz anders. Ich wollte eigentlich überhaupt
nicht nach Berlin. Ursprünglich komme ich aus Schwäbisch Hall, das ist
eine Kleinstadt im Süden Deutschlands. In der zwölften Klasse waren wir
in Berlin auf einer Klassenfahrt und ich fand es furchtbar. Mir war
diese Stadt zu groß und zu laut. Aber auch ich habe Bibliotheken gemocht
und mich dort unglaublich viel aufgehalten. Sogar ein Schülerpraktikum
habe ich in einer Bibliothek gemacht. Jedoch haben mich dort auch
relativ früh verschiedene Sachen angefangen zu stören
\emph{{[}lacht{]}}. Ich werde nie vergessen, wie die Bibliothekarin dort
um fünf vor sechs mit einem Krächzen durch die Bibliothek rief:
\enquote{Wir schließen in fünf Minuten! Bitte gehen Sie jetzt!!} Das war
der Moment, in dem ich gesagt habe: Das möchte ich gerne anders machen!
Ich bin dann nach Stuttgart an die Fachhochschule zum Informationstag
gegangen. Stuttgart war ja nah dran an meiner Heimat und ich konnte mir
damals gar nicht vorstellen, irgendwo anders hinzugehen. Bei diesem
Informationstag habe ich viel Interessantes gehört. Ich habe aber durch
die Unterhaltungen mit den Anwesenden schnell festgestellt, dass ich mit
dem damaligen Diplom-Studiengang dort nicht das machen konnte, was ich
wollte -- nämlich etwas zu verändern in Bibliotheken. Also habe ich
geschaut, wo ich das studieren kann und festgestellt: \enquote{Mist! Das
wäre in Berlin.} Schließlich bin ich mit meinem Bruder nach Berlin
gefahren, zur Studienberatung ans IBI gegangen und habe gesagt, dass ich
hier gerne studieren und im Zweitfach Französisch belegen würde. Mir
wurde dann gesagt: \enquote{Also, ob Sie hier richtig sind, das weiß ich
nicht.} Ich solle lieber an die Fachhochschule gehen. Geblieben bin ich
aber trotzdem und habe es nie bereut.

\textbf{TR:} \textbf{Und an welchem Punkt während Ihrer Studienzeit am
IBI wussten Sie, dass Sie eine wissenschaftliche Karriere einschlagen
wollen?}

\textbf{EG:} Ich wusste es bis zu dem Moment, als ich hier am IBI eine
wissenschaftliche Mitarbeiterstelle angetreten habe, nicht. Ich habe
ganz normal studiert und dachte auch während der Praktika, dass ich
danach in eine Bibliothek gehen würde, um in der Benutzerabteilung oder
ähnliches zu arbeiten. Mein Traum war es immer, die Benutzerabteilung
der ZLB (Zentral- und Landesbibliothek Berlin) zu leiten. Dann war es
aber so, dass gegen Ende meines Studiums eine wissenschaftliche
Mitarbeiterstelle bei Michael Seadle ausgeschrieben wurde und er mich
fragte, ob das nicht etwas für mich wäre. Und da dachte ich mir, ja,
eigentlich schon. Ich habe dann erst auf dieser Stelle festgestellt,
dass mir das sehr viel Spaß macht und dass das etwas ist, das mir liegt.
Während meiner Arbeit habe ich mich dann fragen müssen, ob ich die
wissenschaftliche Karriere verfolgen oder zurück ins \enquote{reale}
Leben möchte. Ich habe dann relativ schnell festgestellt, dass ich eine
Professur anstreben möchte. Dementsprechend bin ich dann auch meine
Karriereplanung angegangen. Aber davor war es, wie vieles in
wissenschaftlichen Karrieren, eher Zufall.

\textbf{VP:} Ich wusste auch mit dreißig Jahren noch nicht, was ich
werden wollte. Bei mir war es so, dass ich nach einer Exkursion nach
Hamburg beschlossen habe, dass ich im Spiegel-Archiv arbeiten möchte.
Das fand ich nämlich toll, weil die Leute, die dort Informationsarbeit
gemacht haben, den Journalisten gleichgestellt waren. Und ich wollte ja
Informationsarbeit machen. Das war der ursprüngliche Plan. Ich bin dann
im achten Semester ins Ausland gegangen, nach Berkeley, Kalifornien, und
habe da ein Jahr lang vor mich hin studiert. Das war zur Zeit des
Dotcom-Booms. Es war fantastisch, eine ganz andere Welt. Im Sommer habe
ich dann in diesem Institut gefragt, ob sie nicht einen Job für mich
hätten und habe dort bei einem Forschungsprojekt mitgearbeitet. Und das
fand ich großartig. Am Ende des Sommers bin ich mit meinen Kollegen aus
diesem Team Mittagessen gegangen und während des Essens haben sie mich
gefragt: \enquote{Kannst du dir vorstellen, einen Doktor zu machen?} Und
ich: \enquote{Naja, darüber habe ich noch nie nachgedacht.} Darauf
sagten sie: \enquote{Also, wenn du dir das vorstellen könntest, wir
hätten dich gerne hier und würden dich dabei unterstützen.} Und wenn
Berkeley sagt, dass sie dich gerne als Doktorandin nehmen würden, dann
äußert man kein \enquote{nein}. Ein Jahr später war ich in Berkeley als
Doktorandin. Es war dann so, dass die eine Hälfte meiner Kommilitonen
irgendwo ins Silicon Valley gegangen ist oder gehen wollte und die
andere Hälfte akademisch arbeiten wollte. Ich hingegen ging zurück nach
Deutschland und wusste immer noch nicht genau, was ich tun möchte. Ich
wusste, dass ich Forschung machen wollte, aber nicht, ob es an einer
Universität oder irgendwo anders sein sollte. Nach meiner Promotion habe
ich dann tatsächlich bei einer Suchmaschinenfirma, nicht bei Google
\emph{{[}lacht{]},} ein Bewerbungsgespräch geführt, mich aber dann
stattdessen für ein Forschungsinstitut entschieden.

\textbf{TR:} \textbf{Und aus welchem Grund haben Sie sich nach dem
Studium am IBI entschieden, sich auch hier für eine Professur zu
bewerben?}

\textbf{VP:} Die Entscheidung war zunächst einmal aus Amerika zurück
nach Deutschland zu kehren. Ich habe dann in Bonn bei der GESIS
(Leibniz-Institut für Sozialwissenschaften) gearbeitet und war dort
stellvertretende Leiterin einer Abteilung, die
informationswissenschaftliche Forschung und Entwicklung gemacht hat.
Während dieser Zeit hielt ich jedoch den Kontakt zum IBI und hatte auch
den neuen Direktor, Michael Seadle, kennengelernt und einen Vortrag über
meine Dissertation gehalten. Als dann die Juniorprofessur für
\enquote{Information Retrieval} ausgeschrieben wurde, wusste ich:
\enquote{Die suchen mich!} und bewarb mich darauf. Außerdem wollte ich
gerne zurück nach Berlin, um meiner Familie wieder näher zu sein, die in
Ostdeutschland wohnt. So war es sowohl örtlich als auch fachlich perfekt
für mich. Und es war schön, wieder an ein informationswissenschaftliches
Institut zurückzukommen.

\textbf{EG:} Am IBI hatte ich nicht meine erste Professur. Vorher war
ich in Kopenhagen und hatte dort eine \emph{Assistant Professor
Position} für \enquote{Information Science}, weil ich zu meiner Zeit als
wissenschaftliche Mitarbeiterin gesehen habe, dass mich das
interessiert. Also habe ich mich direkt nach meiner Promotion dort
beworben und die Stelle auch bekommen. Das Schöne an einer Professur
ist, dass man mehr Freiheiten genießt als in der freien Wirtschaft. Man
steht in regem Kontakt zu den Studierenden, deren Fragen ganz anders
sind als jene auf einer Fachkonferenz. Wir lernen unglaublich viel
voneinander und das finde ich sehr spannend. Außerdem beforscht man
nicht nur seine eigenen Themen, sondern gibt diese auch weiter, was mir
Spaß macht. Es gibt nicht viele Stellen, wo man so arbeiten kann. Auf
dem diesjährigen Bibliothekartag habe ich mehrere Absolventen getroffen
und zu hören, wo diese jetzt alle arbeiten, war wirklich toll.

\textbf{TR: Im Wintersemester 2018/19 feiert das IBI seinen neunzigsten
Geburtstag. Wie schätzen Sie die Relevanz unseres Faches in der Zukunft
ein?}

\textbf{EG:} Diese Frage finde ich schwierig, weil ich es lieber so
formulieren würde: Wie würde ich es mir wünschen? Weil ich fest davon
überzeugt bin, dass unser Fach momentan vielleicht {das} relevanteste
ist, das es gibt. Denn wir sind in der Zeit der Digitalisierung
angekommen, in der es nicht mehr um die Entwicklung der Technik geht,
sondern um die Frage, wie die Technik auf die Gesellschaft einwirkt.

Die Forschung zur Wirkung ist schon seit einigen Jahren im Kommen und
wird in den nächsten Jahren noch stärker werden. Wir müssen uns nur
anschauen, wie es mit den Themen \enquote{Industrie 4.0} und
\enquote{Smart Living} wird. Und da wir in einem Fach arbeiten, welches
sich seit vielen, vielen Jahren damit beschäftigt, welche Auswirkungen
Information auf die Gesellschaft hat, welche Rollen Bibliotheken und
andere informationsverarbeitende Einrichtungen spielen, da glaube ich,
dass wir eine hohe Relevanz haben werden. Aber ich sage auch, dass das
viel Arbeit ist. Denn wir sind nicht so bekannt und müssen unsere
Relevanz erst einmal stärker nach außen tragen.

\textbf{TR:} \textbf{Kann das IBI in diesem Zusammenhang eine Rolle
einnehmen?}

\textbf{VP:} Absolut! Wer, wenn nicht wir mit dieser einzigartigen
Kombination zwischen dem Bibliotheksbereich und anderen
informationsverarbeitenden Einrichtungen könnte das mitgestalten? Welche
Rolle wir einnehmen können? Wir {sollten} eine zentrale Rolle einnehmen,
aber ob wir das {können}, mit unserer Größe, ist eine andere Frage. Aber
dass wir da mitspielen, strategisch, politisch und wissenschaftliche
Aussagen treffen, weil wir, die Studierenden und Forschenden uns mit dem
Themengebiet auseinandersetzen, da sehe ich unglaublich viel Potential.
Und wir tun unseren Beitrag, um Botschafter des Instituts zu sein -- das
macht jeder Kollege und jede Kollegin.

\textbf{EG:} Ja, wir sind viel, viel kleiner als andere Einrichtungen
und können personell viele Sachen nicht leisten, aber wir werden
Strategien entwickeln und sind auch schon dabei. Es tut sich viel und wo
wir früher nicht wahrgenommen wurden, ist jetzt Interesse und
Unterstützung vorhanden.

\textbf{TR: Der Direkt-Masterstudiengang des Instituts wurde kürzlich
von \enquote{Bibliotheks- und Informationswissenschaft} in
\enquote{Information Science} umbenannt. Welchen Stellenwert hat der
Bibliotheksbereich in der zukünftigen Ausrichtung des IBIs?}

\textbf{VP:} Für mich einen ganz großen! Denn dieses
Alleinstellungsmerkmal, das wir haben und mit dem wir uns historisch und
wissenschaftlich bedingt eine wichtige Position herausgearbeitet haben,
das sollten wir nicht verlieren. Auch Bibliotheken haben viele
Forschungsbereiche, viele Dinge, die beforscht werden können und
sollten. Nichtsdestotrotz werden Sie eine Diversifizierung am IBI
feststellen. Das sieht man auch an den Studiengängen. Wir haben ja nicht
nur den einen Bachelor (BA) und den einen Master (MA), sondern mehrere.
Der Bibliotheksbereich im MA befindet sich stärker im Fernstudiengang
und in einem neuen Studiengang, den wir gerade mit der FH Potsdam
aufbauen. \enquote{Digitales Datenmanagement} ist auch für
Forschungsdatenmanager in Bibliotheken und an anderen Einrichtungen
gedacht. Wie Elke Greifeneder in ihrem Vortrag auf dem Bibliothekartag
2018 gesagt hat: \enquote{Nur weil da nicht Bibliothek drauf steht,
heißt das nicht, dass da nicht Bibliothek drin ist!} Ich vertrete eine
expansive Auffassung des Begriffes der Informationslandschaft und für
mich bedeutet \enquote{Information} {immer}, dass da auch Bibliothek mit
drin ist. Als besondere Ansprechpartner sind Bibliotheken und andere
Kulturerbeinstitutionen am IBI wichtig und sollen es auch bleiben.

\textbf{EG:} Das kann ich so nur unterstützen. Wenn man sich den
Masterstudiengang anschaut, ist da viel für Bibliotheken enthalten:
Digitalisierung, User Experience Design, Datenbanken\ldots{} Das sind
alles relevante Inhalte für Bibliotheken und Kenntnisse, die man auf
diesem Niveau in der Bibliothek braucht. Um mit einem Masterabschluss
auf einer E13-Stelle oder höher eingestellt zu werden, muss man ja auch
bestimmte Aufgaben übernehmen. Und wenn wir uns angucken, was das für
Aufgaben sind, dann sind diese eben nicht mehr {nur}
bibliotheksspezifisch. Es sind Kenntnisse und Kompetenzen, die man
genauso in Museen und Archiven oder Firmen wie IBM, Amazon und Google
braucht. Es sind die gleichen Kompetenzen. Deshalb haben wir die
Entscheidung getroffen, die Studiengänge auszuweiten. Denn wenn wir nur
die ganz konkreten Anwendungen, die ein Fachreferent braucht,
vermitteln, bilden wir am Bedarf vorbei aus. Das muss man leider so
sagen. Bibliotheken sind Informationssysteme, der größte Bereich, den
wir haben. Aber es gibt eben auch noch mehr, wir machen eben nicht nur
das.

\textbf{TR: Auch bei der Ausrichtung der Lehrstühle gab es in den
letzten Jahren einige Veränderungen. Welchen Stellenwert wird das Thema
\enquote{Öffentliche Bibliotheken} künftig bei den Lehrinhalten des IBI
einnehmen?}

\textbf{VP:} Das ist eine interessante Frage, denn es kommt darauf an,
wie Sie das Thema \enquote{Öffentliche Bibliothek} (ÖB) betrachten.
Meine Antwort: Wenn da Bibliothek drauf steht, ist auch ÖB drin. Die
Spartentrennung, die wir in den 1990er Jahren aufgehoben haben, die
wollen wir hier nicht wieder einführen. Dass Wissenschaftliche
Bibliotheken (WBs) und ÖBs verschiedene Themen haben und unterschiedlich
fokussiert sind, stimmt sicherlich. Jetzt ist die Frage, inwiefern
können wir mit unseren Lehrinhalten auch wichtige Inhalte der ÖBs
vermitteln? Und da gebe ich Ihnen einen kleinen Ausblick: Im nächsten
Semester wird es im Bachelorstudiengang einen Kurs geben, der
\enquote{Soziale und interkulturelle Bibliotheksarbeit} heißt. Das ist
interessant für ÖBs, aber eben auch für WBs. Es wird vermutlich keinen
Kurs geben, der \enquote{Öffentliche Bibliotheken} heißt, denn es gibt
ja auch keinen, der \enquote{Wissenschaftliche Bibliotheken} heißt. Man
muss also detaillierter und präziser in die Inhalte der Module
hineinschauen. Die meisten Absolventen gehen in WBs und die Inhalte der
Kurse zielen auf Themengebiete ab, die vermehrt in WBs interessieren
werden.

Aber heißt das, dass wir für ÖBs nicht ausbilden? Das würde ich so nicht
sagen.

\textbf{TR: Nicht nur der Name \enquote{Information Science} weist
darauf hin: Viele Masterveranstaltungen werden mittlerweile in
englischer Sprache angeboten und auch im Bachelorstudium ist ein großer
Teil der Lektüre englischsprachig. Sind sehr gute Englischkenntnisse in
unserem Fach eine Pflichtvoraussetzung und wird sich das in Zukunft noch
weiter intensivieren?}

\textbf{VP:} Es ist die momentane Wissenschaftssprache und dadurch ist
die aktuellste Literatur zum Fach englischsprachig. In einigen Jahren
sind die Übersetzungsprogramme vielleicht so gut, dass man alles durch
den \emph{Translator} schicken kann und dann hat sich diese Problematik
auch erledigt. Andererseits muss man die Texte dennoch auf Englisch
verfassen können.

\textbf{EG:} Englischkenntnisse sind wichtig und unabdingbar. Es ist
{die} Arbeitssprache. Man muss mit guten Kenntnissen herkommen, sonst
wird man die Artikeltexte nicht verstehen können. Aber man muss und soll
auch während des Studiums ganz viel dazulernen, dafür ist das Studium
da. Es ist in Zukunft einfach ein weiterer Teil des Studiums, sich diese
Kompetenz anzueignen.

\textbf{TR:} \textbf{Brauchen moderne Studiengänge wie
\enquote{Bibliotheks- und Informationswissenschaft}, \enquote{INFOMIT}
und \enquote{Information Science} eigentlich auch ein modernes
Institutsgebäude oder gehört das Gebäude in der Dorotheenstraße 26
mittlerweile untrennbar zum IBI?}

\textbf{VP:} Ich liebe dieses Gebäude, ich liebe die \enquote{Harry
Potter}-Treppe, ich liebe die großen Hallen --- und ich habe genügend
Fantasie, um mir das \enquote{schön} vorzustellen \emph{{[}lacht{]}.} Es
ist aber nicht untrennbar mit dem IBI verbunden, denn das IBI ist
schließlich erst 1994/95 hier eingezogen. Und die Frage ist: Brauchen
wir überhaupt ein {Gebäude}? Ich denke, wir bräuchten viel mehr moderne
Informationstechnologie. Mit dem \emph{iLab} haben wir ein exzellentes
Beispiel, wie es funktionieren kann und hoffentlich ab dem nächsten Jahr
auch eine geupdatete Infrastruktur. Also gerne das alte Gebäude, aber
bitte mit sehr moderner Infrastruktur. Eingepfercht zwischen zwei
Bibliotheken --- symbolischer kann unser Standort doch eigentlich nicht
sein. Und nah am Hauptgebäude, dem Zentrum allen Geschehens.

\textbf{EG:} Hinzuzufügen ist nur: Was gibt es denn Cooleres und
Moderneres als in einem Gebäude zu sitzen, in dem Steven Spielberg 2014
einen Film gedreht hat?! Ein Gebäude, in dem Tom Hanks schon die Türen
geöffnet hat? (Anmerkung des Interviewers: Das IBI-Gebäude war
Schauplatz für den Film \enquote{Bridge of Spies}.)

\textbf{TR:} \textbf{Vor kurzem haben Sie, Frau Prof.~Petras, die
Leitung des IBI von Herrn Prof.~Seadle übernommen. Was möchten Sie aus
der Vergangenheit weiterführen und was wollen Sie am IBI ändern?}

\textbf{VP:} Ich würde gerne weiterführen, dass wir ein offenes,
studierendenfreundliches Institut bleiben, an dem es Spaß macht zu
studieren. Dass wir weiterhin einen innovativen Raum für Mitarbeiter und
Mitarbeiterinnen anbieten, in dem man Dinge ausprobieren kann, sowohl
didaktisch als auch wissenschaftlich. Ich würde die Identität, die wir
hier haben, gerne weiter stärken. Diese Trennung, in den Köpfen der
Studierenden, zwischen den Studiengängen abschaffen, sodass sich alle
gleichermaßen zugehörig fühlen. Und zu dieser Identität gehört für mich
auch die Repräsentation nach außen.

Ein Wunsch von mir wäre, dass wenn jemand, zum Beispiel von dieser
Universität, das Wort \enquote{Digitalisierung} sagt, damit sofort das
IBI verbunden wird. Dass bestimmte Themengebiete einfach mit dem
Institut identifiziert werden. Und mein Anspruch ist dann, dass das
nicht nur an der Universität, sondern in der Stadt und darüber hinaus
auch so ist. Durch die Mitgliedschaft in der \emph{iSchool-}Vereinigung
sind wir schon in den informationswissenschaftlichen Instituten bekannt.
Wir haben gute Beziehungen zu

Bibliotheken und die möchte ich weiter stärken. Ich möchte außerdem,
dass wir stärker als Forschungs- und Beratungszweig von Bibliotheken
gesehen werden. Wissenschaftlich sind wir gut aufgestellt, was uns
jedoch fehlt, sind gute Kontakte zur Industrie und in den
Wirtschaftsbereichen. Daran wollen wir beide arbeiten.

\textbf{TR:} \textbf{Was wünschen Sie dem Institut zum 90. Geburtstag?}

\textbf{EG:} Ich würde mir wünschen, dass zu diesem Anlass ganz viele
Leute zusammenkommen, die vielleicht seit Jahren nicht mehr hier waren.
Von denen jeder seine ganz eigene Art der Beziehung zum Institut hat und
sieht, was daraus geworden ist und das mit uns feiert. Und dann
natürlich weitere tolle 90 Jahre.

\textbf{VP:} Eine schöne Party! Und hoffen wir, dass es so weiter geht.
Dieses Institut hatte eine sehr bewegte Geschichte mit sehr vielen
interessanten Verwicklungen, sodass es spannend sein wird zu sehen, wie
das Leben hier in dreißig oder auch schon in zehn Jahren aussehen wird.
Ich wünsche mir, dass es ein Ort der Innovation und Kollegialität,
zwischen Mitarbeitern und Studierenden, bleibt und dass wir weiterhin
anpassungsfähig, flexibel und dynamisch bleiben -- sonst schaffen wir es
in dieser Wissenschaftslandschaft und in diesem speziellen Kontext gar
nicht.

%autor
\begin{center}\rule{0.5\linewidth}{\linethickness}\end{center}

\textbf{Prof.~Vivien Petras, PhD} ist geschäftsführende
Institutsdirektorin und hat den Lehrstuhl für Information Retrieval am
Institut für Bibliotheks- und Informationswissenschaft seit 2009 inne.
Ihre Forschungsschwerpunkte sind die Evaluation von
Informationssystemen, mehrsprachige Aspekte des Information Retrievals,
Informationssysteme für das Kulturerbe und die Wissensorganisation.

\textbf{Prof.~Dr.~Elke Greifeneder} ist Juniorprofessorin und
stellvertretende geschäftsführende Direktorin am IBIund hat den
Lehrstuhl für Information Behavior am Institut für Bibliotheks- und
Informationswissenschaft seit 2014 inne. Ihre Forschungsschwerpunkte
sind das Informationsverhalten von Menschen in der Interaktion mit
Technik, mit Schwerpunkten in der Methodenforschung, Validitätsforschung
und Studien in natürlichen Nutzerkontexten.

\textbf{Thomas Roesnick} studiert aktuell im sechsten Bachelorsemester
„Bibliotheks- und Informationswissenschaft`` am Institut. Momentan
schreibt er an seiner Bachelorarbeit und würde danach gerne an der
Humboldt-Universität den Master in „Deutsche Literatur`` belegen.

\textbf{Felicitas Härting} studiert zurzeit im zweiten Bachelorsemester
„Bibliotheks- und Informationswissenschaft`` am Institut. 2013 hat sie
die Ausbildung zur „Fachangestellten für Medien- und
Informationsdienste`` in der Stadtbibliothek Steglitz-Zehlendorf
absolviert und arbeitet seitdem auch dort. Nach ihrem Bachelorabschluss
würde sie gerne in einer Kinder- und Jugendbibliothek arbeiten, das
Referat für Jugendliteratur betreuen und Veranstaltungen für und mit
Kindern auf die Beine stellen.

\textbf{Miriam Brauer} studiert aktuell im vierten Bachelorsemester
Bibliotheks- und Informationswissenschaft am Institut. Zudem ist sie als
studentische Mitarbeiterin am Lehrstuhl für Information Processing and
Analytics tätig.

\end{document}
