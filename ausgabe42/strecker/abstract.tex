\textbf{Kurzfassung}: Verschiedene Stakeholder fordern eine bessere
Verfügbarkeit von Forschungsdaten. Der Erfolg dieser Initiativen hängt
wesentlich von einer guten Auffindbarkeit der publizierten Datensätze
ab, weshalb Dataset Retrieval an Bedeutung gewinnt. Dataset Retrieval
ist eine Sonderform von Information Retrieval, die sich mit dem
Auffinden von Datensätzen befasst. Dieser Beitrag fasst aktuelle
Forschungsergebnisse über das Informationsverhalten von Datensuchenden
zusammen. Anschließend werden beispielhaft zwei Suchdienste
verschiedener Ausrichtung vorgestellt und verglichen. Um darzulegen, wie
diese Dienste ineinandergreifen, werden inhaltliche Überschneidungen von
Datenbeständen genutzt, um den Metadatenaustausch zu analysieren.

\textbf{Abstract}: Various stakeholders are calling for better
availability of research data. The success of these initiatives depends
largely on good discoverability of published datasets, which is why
Dataset Retrieval is gaining in importance. Dataset Retrieval is a
special form of Information Retrieval that is concerned with finding
datasets. This paper summarizes recent research on the information
behavior of data users. Subsequently, two search services with different
objectives are presented and compared. In order to show how these
services interconnect, overlaps in content are used to analyze metadata
exchange between them.
