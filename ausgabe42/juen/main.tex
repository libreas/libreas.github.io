\documentclass[a4paper,
fontsize=11pt,
%headings=small,
oneside,
numbers=noperiodatend,
parskip=half-,
bibliography=totoc,
final
]{scrartcl}

\usepackage[babel]{csquotes}
\usepackage{synttree}
\usepackage{graphicx}
\setkeys{Gin}{width=.4\textwidth} %default pics size

\graphicspath{{./plots/}}
\usepackage[ngerman]{babel}
\usepackage[T1]{fontenc}
%\usepackage{amsmath}
\usepackage[utf8x]{inputenc}
\usepackage [hyphens]{url}
\usepackage{booktabs} 
\usepackage[left=2.4cm,right=2.4cm,top=2.3cm,bottom=2cm,includeheadfoot]{geometry}
\usepackage[labelformat=empty]{caption} % option 'labelformat=empty]' to surpress adding "Abbildung 1:" or "Figure 1" before each caption / use parameter '\captionsetup{labelformat=empty}' instead to change this for just one caption
\usepackage{eurosym}
\usepackage{multirow}
\usepackage[ngerman]{varioref}
\setcapindent{1em}
\renewcommand{\labelitemi}{--}
\usepackage{paralist}
\usepackage{pdfpages}
\usepackage{lscape}
\usepackage{float}
\usepackage{acronym}
\usepackage{eurosym}
\usepackage{longtable,lscape}
\usepackage{mathpazo}
\usepackage[normalem]{ulem} %emphasize weiterhin kursiv
\usepackage[flushmargin,ragged]{footmisc} % left align footnote
\usepackage{ccicons} 
\setcapindent{0pt} % no indentation in captions

%%%% fancy LIBREAS URL color 
\usepackage{xcolor}
\definecolor{libreas}{RGB}{112,0,0}

\usepackage{listings}

\urlstyle{same}  % don't use monospace font for urls

\usepackage[fleqn]{amsmath}

%adjust fontsize for part

\usepackage{sectsty}
\partfont{\large}

%Das BibTeX-Zeichen mit \BibTeX setzen:
\def\symbol#1{\char #1\relax}
\def\bsl{{\tt\symbol{'134}}}
\def\BibTeX{{\rm B\kern-.05em{\sc i\kern-.025em b}\kern-.08em
    T\kern-.1667em\lower.7ex\hbox{E}\kern-.125emX}}

\usepackage{fancyhdr}
\fancyhf{}
\pagestyle{fancyplain}
\fancyhead[R]{\thepage}

% make sure bookmarks are created eventough sections are not numbered!
% uncommend if sections are numbered (bookmarks created by default)
\makeatletter
\renewcommand\@seccntformat[1]{}
\makeatother

% typo setup
\clubpenalty = 10000
\widowpenalty = 10000
\displaywidowpenalty = 10000

\usepackage{hyperxmp}
\usepackage[colorlinks, linkcolor=black,citecolor=black, urlcolor=libreas,
breaklinks= true,bookmarks=true,bookmarksopen=true]{hyperref}
\usepackage{breakurl}

%meta
%meta

\fancyhead[L]{S. Juen\\ %author
LIBREAS. Library Ideas, 42 (2022). % journal, issue, volume.
\href{http://nbn-resolving.de/}
{}} % urn 
% recommended use
%\href{http://nbn-resolving.de/}{\color{black}{urn:nbn:de...}}
\fancyhead[R]{\thepage} %page number
\fancyfoot[L] {\ccLogo \ccAttribution\ \href{https://creativecommons.org/licenses/by/4.0/}{\color{black}Creative Commons BY 4.0}}  %licence
\fancyfoot[R] {ISSN: 1860-7950}

\title{\LARGE{Sicherheitspersonal in Bibliotheken – Eine qualitative Untersuchung zur Rolle des Wachschutzes in Bibliotheken Deutschlands}}% title
\author{Sara Juen} % author

\setcounter{page}{1}

\hypersetup{%
      pdftitle={Sicherheitspersonal in Bibliotheken – Eine qualitative Untersuchung zur Rolle des Wachschutzes in Bibliotheken Deutschlands},
      pdfauthor={Sara Juen},
      pdfcopyright={CC BY 4.0 International},
      pdfsubject={LIBREAS. Library Ideas, 42 (2022).},
      pdfkeywords={Bibliothek, Sicherheit, Wachdienst, Sicherheitsdienst, Interview, Zusammenarbeit, library, safety, security guards, survey, cooperation},
      pdflicenseurl={https://creativecommons.org/licenses/by/4.0/},
      pdfcontacturl={http://libreas.eu},
      baseurl={http://libreas.eu},
      pdflang={de},
      pdfmetalang={de}
     }



\date{}
\begin{document}

\maketitle
\thispagestyle{fancyplain} 

%abstracts
\begin{abstract}
\noindent
\textbf{Kurzfassung}: Wachdienste in deutschen Bibliotheken sind keine
Seltenheit mehr und scheinen sich als Partner*innen im Alltag etabliert
zu haben. Doch was genau macht das Sicherheitspersonal in Bibliotheken
und warum braucht es sie überhaupt? Der folgende Artikel präsentiert die
Ergebnisse einer Interview-Studie, welche zum Ziel hatte herauszufinden,
wie Sicherheitspersonal in den Arbeitsort Bibliothek eingebunden ist.
Dazu wurden Interviews mit Personen geführt, welche in ihrer Bibliothek
für den Wachschutz zuständig sind. Es stellte sich heraus, dass es
zwischen den Bibliotheken Gemeinsamkeiten bezüglich der Anforderungen
und Herausforderungen mit den Wachdiensten gibt, genauso aber auch
unterschiedliche Herangehensweisen und bibliotheksspezifische
Bedürfnisse. Diese Arbeit hatte zum einen das Ziel, eine Lücke in der
aktuellen Forschung zu schliessen und zum anderen die Aufmerksamkeit auf
eine Praxis zu lenken, die in deutschen Bibliotheken immer alltäglicher
zu werden scheint.

\begin{center}\rule{0.5\linewidth}{0.5pt}\end{center}

\noindent\textbf{Abstract}: Security guards in German libraries are no longer a
rarity and seem to have established themselves as partners in the daily
business. But what exactly security staff do in libraries and why do
they need them at all? The following article presents the results of an
interview study that aimed to find out how security staff are integrated
into the library as their workplace. For this purpose, interviews were
conducted with people who are responsible for security in their library.
It turned out that there are similarities between the libraries
regarding the requirements and challenges with the security services,
but also different approaches and library-specific needs. This work
aimed on the one hand to fill a gap in current research and on the other
hand to draw attention to a practice that seems to become more and more
commonplace in German libraries.
\end{abstract}

%body
Dieser Artikel\footnote{\textbf{Editorische Notiz}

  In der vorliegenden Arbeit wird geschlechtersensible Sprache
  verwendet. Es werden geschlechtsneutrale Bezeichnungen oder der
  Asteriskus (Besucher*innen) eingesetzt. Eine mögliche Störung des
  Leseflusses wird für die inklusive Repräsentation aller
  Geschlechtsidentitäten hingenommen. Zitate werden unverändert
  verwendet.} beruht auf meiner Masterarbeit, die ich im Fach
Information Science am Institut für Bibliotheks- und
Informationswissenschaft an der Humboldt Universität zu Berlin
geschrieben und im Juni 2022 erfolgreich verteidigt habe. Der
Artikelschwerpunkt liegt auf den Ergebnissen der Untersuchung, andere
Teile der Arbeit wurden stark gekürzt oder weggelassen.

\hypertarget{einleitung}{%
\section{1 Einleitung}\label{einleitung}}

Sicherheitspersonal scheint in Bibliotheken immer alltäglicher zu
werden. Sie kontrollieren Taschen, gehen durch die Räume und schauen
nach dem Rechten, sie schlichten Konflikte, helfen Nutzenden und sind
Ansprechpersonen, wenn das Bibliothekspersonal längst im Feierabend ist.
Manchmal arbeiten sie im Hintergrund und sind kaum wahrnehmbar, an
anderen Orten stehen sie direkt an der Tür und sind die ersten Personen,
die vom Bibliothekspublikum gesehen werden. Sie arbeiten in öffentlichen
und wissenschaftlichen Bibliotheken und sind aus vielen Einrichtungen
nicht mehr wegzudenken. Verschiedene Untersuchungen bestätigen eine
Zunahme von Sicherheitspersonal in öffentlichen und teil-öffentlichen
Räumen (Frevel, 2012, S. 599; Kaufmann, 2017, S. 17).

Warum ist das so? Wozu brauchen Bibliotheken Wachschutz? Wie sind
Sicherheitsdienste in Bibliotheken integriert? Wachdienst und die
Offenheit einer Bibliothek ist das nicht ein Widerspruch? Wie sieht die
Zusammenarbeit von Bibliotheks- und Sicherheitspersonal im Alltag aus?
Diese und ähnliche Fragen der Autorin führten zu der Forschungsfrage:
\emph{Wie ist der Wachschutz in den Arbeitsort Bibliothek eingebunden?}

Um die Forschungsfrage beantworten zu können, wurde die induktive
Methode des qualitativen, teilstrukturierten Leitfadeninterviews
gewählt. In die Analyse flossen die Daten von acht Interviews ein, die
mit neun leitenden Angestellten von öffentlichen sowie
wissenschaftlichen Bibliotheken aus ganz Deutschland geführt wurden. Die
Gespräche offenbarten unterschiedliche individuelle Erfahrungen, doch
gab es auch Themen, die alle Interviewten beschäftigen. Es zeigte sich,
dass die Präsenz von Wachdiensten positive Effekte in Bibliotheken
hervorruft. Allerdings wurde auch deutlich, dass die Zusammenarbeit der
verschiedenen Berufsgruppen Herausforderungen birgt und
Verständnisbereitschaft von beiden Seiten erfordert.

Für die vorliegende Arbeit wurde eine umfangreiche Recherche in
verschiedenen Themenfeldern durchgeführt, auf die im nächsten Kapitel
eingegangen wird (Kapitel 2). Die Methodik und Vorgehensweise der
Datenerhebung und -analyse wird in diesem Artikel kurz und
zusammenfassend vorgestellt (Kapitel 3). Das Hauptaugenmerk liegt auf
den Ergebnissen der Untersuchung, welche mit Zitaten der Interviewten
vervollständigt werden, um die Erfahrungen der Teilnehmenden gebührend
zu berücksichtigen (Kapitel 4). Anschließend folgt die Beantwortung der
Forschungsfrage und es wird auf die Limitationen der Arbeit eingegangen
(Kapitel 5), gefolgt vom Fazit mit einem Ausblick auf mögliche
Problemstellungen und Forschungsvorhaben die zukünftig bearbeitet werden
könnten (Kapitel 6).

\hypertarget{stand-der-forschung}{%
\section{2 Stand der Forschung}\label{stand-der-forschung}}

Um einen möglichst umfassenden Überblick zu dem spezifischen Thema
\emph{Sicherheitsdienste in Bibliotheken} zu erhalten, wurden im Vorfeld
mehrere thematische Recherchen durchgeführt. Zu der konkreten Thematik
wurde kaum Forschungsliteratur gefunden. Somit wurde die Recherche
breiter angelegt und nach Literatur zum Themenkomplex \emph{Sicherheit
in Bibliotheken} gesucht. Hier ergaben sich mehr Treffer, allerdings
beschäftigt sich ein Großteil der Titel eher mit den technischen
Sicherheitsaspekten einer Bibliothek. Die Ergebnisse dieser beiden
Recherchen werden zusammengefasst vorgestellt. Aufgrund der begrenzten
Funde wurde in einem dritten Schritt entschieden, noch die allgemeinere
Thematik \emph{Sicherheit im öffentlichen Raum} hinzuzuziehen und auch
diese auf den Schwerpunkt Sicherheitsdienste zu untersuchen. Die
Ergebnisse dieser Recherche werden als erstes vorgestellt.

\hypertarget{sicherheit-im-uxf6ffentlichen-raum}{%
\subsection{2.1 Sicherheit im öffentlichen
Raum}\label{sicherheit-im-uxf6ffentlichen-raum}}

Das Themenfeld des öffentlichen Raumes wird in verschiedenen Disziplinen
untersucht und diskutiert, dazu gehören unter anderem die Architektur,
Stadtplanung, Soziologie beziehungsweise die Stadtsoziologie.

Anke Schröder (2021) stellt fest, dass der öffentliche Raum zwei Ebenen
hat, einerseits der konkrete physische Raum und andererseits der soziale
Raum (Schröder, 2021, S. 62). Sie sieht das Prinzip des subjektiven
Handelns in öffentlichen Räumen durch die sich verbreitende
Privatisierung in Gefahr. Vermehrte Konflikte im öffentlichen Raum haben
die Nutzung dessen verändert. Die Relevanz von Sicherheit habe
zugenommen. Schröder sieht die Auseinandersetzungen im öffentlichen Raum
als einen Ausdruck aktueller gesellschaftlicher Herausforderungen
(Schröder, 2021, S. 65).

Bernhard Frevel (2012) konstatiert, dass seit den 1990er Jahre die
\enquote{Kriminalitätsfurcht} in der Bevölkerung angestiegen ist und
\enquote{ein Nachlassen des Sicherheitsempfinden} stattgefunden hat,
dass Frevel ebenso auf eine Zunahme von Konflikten im öffentlichen Raum
zurückführt (Frevel, 2012, S. 593). Des Weiteren stellt der Autor die
Erweiterung der \enquote{sicherheitsorientierten Kontrollkultur} fest
(Frevel, 2012, S. 607), die auf lokaler Ebene sichtbar wird, unter
anderem durch den Ausbau von Videoüberwachung oder durch die vermehrte
Anwesenheit privater Sicherheitsunternehmen (Frevel, 2012, S. 608). Auch
Stefan Kaufmann (2017) stellt in seinem Beitrag \emph{Das Themenfeld
\enquote{Zivile Sicherheit}} fest, dass es einen \enquote{Aufstieg der
Sicherheitsdienstleister} gegeben hat und führt dies auf die
Privatisierung von Infrastrukturen zurück (Kaufmann, 2017, S. 14).

Das Phänomen der unklaren Grenzen und Ausweitung von Überwachung durch
private Wachdienste untersuchen auch Elisa Saarikkomäki und Anne
Alvesalo-Kuusi (2019). Sie stellen einen weitgreifenden Wandel fest, vom
Monopol der staatlichen Polizei zu einem pluralistischen System, in dem
auch private Firmen Sicherheitsdienste anbieten können (Saarikkomäki \&
Alvesalo-Kuusi, 2019, S. 129). Laut Saarikkomäki und Alvesalo-Kuusi hat
die Kommerzialisierung der Städte erhöhte Überwachung hervorgerufen. Die
trage dazu bei, dass marginalisierte Gruppen im öffentlichen Raum
stärker überwacht werden, da ihre Anwesenheit unerwünscht zu sein
scheine. Davon seien insbesondere Jugendliche betroffen (Saarikkomäki et
al., 2019, S. 130). Auch eine Untersuchung aus Australien, in der
ebenfalls die Interaktion zwischen Jugendlichen und Sicherheitsdiensten
analysiert wurde (Morey, 1999), kommt zu dem Ergebnis, dass Jugendliche
in öffentlichen Räumen mehr überwacht werden als andere Gruppen. Eine
weitere Studie kommt zu dem Schluss, dass Sicherheitspersonal in der
Lage ist, an öffentlichen Orten Kontrolle auszuüben, obwohl ihnen die
rechtliche Autorität dafür fehlt (Kammersgaard, 2021).

\hypertarget{sicherheit-und-wachschutz-in-bibliotheken}{%
\subsection{2.2 Sicherheit und Wachschutz in
Bibliotheken}\label{sicherheit-und-wachschutz-in-bibliotheken}}

Das Thema \emph{Sicherheit in Bibliotheken} wird von vielen
Publikationen hauptsächlich von der technischen Seite her behandelt. Es
geht also um Katastrophen-, Brand- und Diebstahlschutz sowie
Bestandssicherung (Dohrmann et al., 2009; Jopp, 1991; McGinty, 2008;
Robertson, 2014). Die Thematik des Sicherheitsdienstes wird in
Fachbeiträgen teilweise angesprochen. Die Recherchen legen nahe, dass
dieses Thema in der englischen Diskussion weiterverbreiteter ist, als in
der deutschen.

In mehreren Artikeln wird festgestellt, dass sich die Bibliotheksnutzung
und die Bedürfnisse der Besuchenden stark verändert haben, was nach
Meinung der Autor*innen auch dazu führt, dass neue Sicherheitsaspekte
hinzugekommen sind (McGuin, 2010; Reed, 2007; Trapskin, 2008). Steve
Albrecht (2015) sieht die Zusammenarbeit mit der Polizei als
entscheidend für die Sicherheit in der Bibliothek und ihre Anwesenheit
als nützliches und langfristiges Abschreckungsmittel gegen
\enquote{problematische Besucher*innen} (Albrecht, 2015, S. 117).
Privates Sicherheitspersonal wird als notwendig gesehen, allerdings
geben die Autoren zu bedenken, dass passendes und gut ausgebildetes
Wachdienstpersonal schwer zu finden sei (Albrecht, 2015, S. 77; Graham,
2011, S. 72--74). In amerikanischen Beiträgen wird des Öfteren
empfohlen, eng mit der universitätseigenen Campuspolizei oder der
staatlichen Polizei zusammenzuarbeiten (Lashley, 2007, S. 196; McGuin,
2010, S. 110).

Dieser Auffassung widerspricht der Artikel von Ben Robinson, der 2019
unter dem Titel \emph{No holds barred: Policing and Security in the
public library} erschienen ist\emph{.} Robinson kritisiert, dass zu
enger Zusammenarbeit mit Sicherheitsdienst und Polizei geraten wird und
dass in nordamerikanischen Bibliotheken immer öfter permanente
Polizeiposten eingerichtet werden. In diesem Szenario werde nicht
bedacht, dass sich ein solcher Zusammenschluss negativ auf bestimmte
Personengruppen auswirken kann. Untersuchungen außerhalb der
Bibliotheks- und Informationswissenschaft haben gezeigt, dass
Wachschutz- und Polizeipräsenz insbesondere bei Schwarzen, Indigenen,
People of Color und anderen marginalisierten Gruppen negative
psychologische Folgen haben kann (Robinson, 2019, o.\,S.). Der Autor
schlägt vor, die Zusammenarbeit mit anderen Spezialist*innen wie
Sozialarbeiter*innen oder Psycholog*innen in Betracht zu ziehen.
Robinson stellt fest, dass die vorhandene Literatur zu
Sicherheitsaspekten in Bibliotheken oftmals von Autor*innen verfasst
wird, die einen Wachschutz- oder Polizeihintergrund haben und selten aus
dem Bibliotheksumfeld stammen (Robinson, 2019, o.\,S.). Außerdem
konstatiert er, dass es in der Bibliotheks- und Informationswissenschaft
kaum Literatur darüber gibt, wie sich Wachdienst- und Polizeiaufsicht in
Bibliotheken auf marginalisierte Menschen auswirken kann. Robinson kommt
zu dem Schluss, dass sich das Bibliothekswesen und insbesondere die
Bibliotheks- und Informationswissenschaft interdisziplinär und kritisch
mit der Rolle von Sicherheitspersonal und Polizei in Bibliotheken
auseinandersetzen sollte, damit Bibliotheken weiterhin ein offener und
sicherer Ort für alle Besuchenden darstellen (Robinson, 2019, o.\,S.).

Im deutschsprachigen Raum scheint die Zusammenarbeit mit der Polizei
nicht in dem Ausmaß stattzufinden wie in den USA. Allerdings gibt es
auch hier teilweise Tendenzen zu Empfehlungen in diese Richtung. Martin
Eichhorn (2015) empfiehlt Bibliotheken, sich mit der örtlichen Polizei
vertraut zu machen und diese bei Problemen auch um Präsenz vor Ort zu
bitten (Eichhorn, 2015, S. 137--140). Der Autor stellt fest, dass
Wachdienstpersonal in großen Öffentlichen Bibliotheken inzwischen
verbreitet ist und durch das gestiegene Sicherheitsbedürfnis der
Nutzenden deren Akzeptanz erhöht hat (Eichhorn, 2015, S. 136). In einem
Interview für die Zeitschrift \emph{BuB Forum Bibliothek und
Information} (2016) erläutert der Autor ein Jahr später, dass er nicht
damit rechnet, dass sich die Anzahl von Bibliotheken mit
Sicherheitsdienst erhöhen wird. Er weist darauf hin, dass die
Wachdienstmitarbeitenden oft schlecht ausgebildet seien und es von
Einzelpersonen abhänge, ob der Service einen Nutzen für die Bibliothek
sei oder nicht (Eichhorn, 2016, S. 331).

Ulrike Verch (2006) kritisiert, im Zusammenhang mit der Sonntagsöffnung
von Bibliotheken, den Einsatz von Sicherheitsdienstleistern, da das
Outsourcing von Personal zwar finanzielle Vorteile bringen kann, die
\enquote{Einflussmöglichkeiten der Bibliotheksleitung} allerdings
eingeschränkt werden (Verch, 2006, S. 110). Sie sieht es als
problematisch an, wenn das Sicherheitspersonal allein in der Bibliothek
tätig ist, da sie den Besucher*innen keine Fachauskünfte geben können
und sich dies eventuell nachteilig auf die Beziehung mit Nutzenden
auswirken könnte. Sie empfiehlt, einen Wachdienst \enquote{allenfalls
zusätzlich oder ausschließlich für sicherheitsrelevante und technische
Aufgaben} zu engagieren (Verch, 2006, S. 110). Im Zusammenhang mit
Sonntagsöffnungen oder allgemein erweiterten Öffnungszeiten von
Bibliotheken wird der Wachschutz auch in anderen Artikeln erwähnt, meist
als Alternativbesetzung für das Bibliothekspersonal (Christensen, 2017,
S. 249; Emskötter, 2021, S. 377; Vorberg, 2009, S. 61).

In einem Vortrag mit dem Titel \emph{Vom Nachtwächter zum
Lernortmanager? -- Neue Herausforderungen für das Qualitätsmanagement
von Wachdiensten in wissenschaftlichen Bibliotheken} stellt Rolf Duden
auf dem Bibliothekartag in Nürnberg 2015 neue Aufgabenfelder für
Sicherheitspersonal vor und stellt sich der Frage, ob und wie
Wachdienste diesen Anforderungen gerecht werden können. Anhand des
Beispiels der Staats- und Universitätsbibliothek Hamburg definiert er
die neuen Aufgaben für die Wachschutzmitarbeitenden. Er konstatiert,
dass vermehrt \enquote{kommunikative, intellektuelle und technische
Fähigkeiten gefordert} sind, und dass sich die Wachschutzmitarbeitenden
mit den Sicherheitskonzepten und dem gewünschten Umgang mit Nutzenden in
der Bibliothek identifizieren müssen (Duden, 2015, S. 7). Außerdem
stellt er mögliche Schwierigkeiten fest, die im Zusammenhang der neuen
Aufgabenfelder entstehen können. Unter anderem werden die zu hohen
Anforderungen erwähnt und/oder dass die Kapazitäten für den zusätzlichen
Betreuungsaufwand in den Bibliotheken nicht vorhanden sein könnten
(Duden, 2015, S. 17).

Was bei der Recherche und der Sichtung der vorhandenen
Forschungsliteratur auffällt, ist, dass sie oft Lösungsansätze für
Probleme liefert, die genauere Beleuchtung der Ursachen allerdings
häufig versäumt wird. Weiter fällt auf, dass bei dem Thema
\emph{Sicherheit und Wachschutzpersonal in Bibliotheken} der oft sehr
technisch behandelte Aspekt der Sicherheit meistens mehr Aufmerksamkeit
erhält als die menschliche Seite dieser Thematik. Was gänzlich zu fehlen
scheint, sind Untersuchungen und Berichte, welche die Herausforderungen
und Vorteile des gemeinsamen Alltags von Bibliotheks- und
Sicherheitsdienstpersonal beleuchten. Diese Lücke versucht die
vorliegende Arbeit zu schließen.

\hypertarget{exkurs-bibliothek-und-wachschutz-in-der-deutschen-presse}{%
\subsection{\texorpdfstring{2.3 Exkurs: \emph{Bibliothek und Wachschutz}
in der deutschen
Presse}{2.3 Exkurs: Bibliothek und Wachschutz in der deutschen Presse}}\label{exkurs-bibliothek-und-wachschutz-in-der-deutschen-presse}}

Im folgenden Kapitel werden einige Artikel der letzten Jahre
vorgestellt, welche in der deutschen Tagespresse im Zusammenhang mit
Sicherheitspersonal und Bibliotheken veröffentlicht wurden. Diese
Zusammenfassung soll dabei helfen, die Herausforderungen von
Bibliotheken genauer zu beleuchten und erläutern, welche Ereignisse dazu
führen können, dass sich Bibliotheken für Wachschutz entscheiden.

2015 forderten die Mitarbeitenden der Helen-Nathan-Bibliothek in Berlin
Neukölln einen Wachdienst für die Bibliothek, da in der Bibliothek
unzumutbare Arbeitsbedingungen herrschten. Es werden Vorkommnissen wie
\enquote{Sex auf der Toilette, Brandstiftung, Beleidigungen und
unflätiges Benehmen} (Vieth-Entus, 2015, o.\,S.) beschrieben. Des
Weiteren werde die Bibliothek für Drogengeschäfte genutzt und vor allem
die weiblichen Mitarbeitenden werden von Jugendlichen, die diese nicht
als Respektspersonen anerkennen, bedroht und sexistisch beleidigt
(Vieth-Entus, 2015, o. S.). Aufgrund dieser Lage wurde entschieden,
einen Sicherheitsdienst in der Bibliothek zu engagieren.

Der Westfälische Anzeiger berichtete im März 2017 in zwei Artikeln über
den Einsatz eines Sicherheitsdienstmitarbeiters in der Zentralbibliothek
Hamm. Eine Gruppe Jugendliche und junge Erwachsene fielen durch
überhöhte Lautstärke, Pöbeleien und Drohungen auf. Ein Mitarbeiter eines
externen Sicherheitsdienstes solle während der ganzen Öffnungszeiten
Präsenz zeigen und die Lage beruhigen (Westfälischer Anzeiger, 2017a, o.
S.). Der zweite Artikel berichtet, dass die Anwesenheit des
Wachdienstmitarbeiters dazu geführt hat, dass die störende Gruppe ein
angemessenes Verhalten angenommen hat und die Bibliothek weiterhin
besucht (Westfälische Anzeiger, 2017b, o. S.).

Ein weiterer Fall, der Aufmerksamkeit in der Presse erlangte, waren zwei
vermutlich rechte Attacken auf den Bibliotheksbestand der
Bezirksbibliothek Tempelhof-Schöneberg in Berlin. rbb24 (2021)
berichtete das erste Mal im August 2021 darüber, dass Mitarbeitende der
Bibliothek zerstörte Bücher entdeckten, welche sich \enquote{inhaltlich
kritisch mit rechten gesellschaftlichen Tendenzen
auseinander{[}setzen{]}} (Russew, 2021, o. S.). Außerdem waren
Biografien aus dem Sozialismus betroffen. Der Leiter der Bibliothek
erzählt im Artikel von früheren Vorfällen, wie Vandalismus mit rechten
Botschaften oder Flyern mit rechtem Gedankengut, die in der Bibliothek
platziert wurden. Weiter beschreibt er, dass der aktuelle Angriff eine
\enquote{neue Qualität} aufweise, weil die Bücher gezielt ausgesucht und
zerschnitten wurden (Russew, 2021, o. S.). Im September 2021, berichtete
rbb24 erneut von mutwillig zerstörten Büchern, welche dasselbe
Themenspektrum behandelten wie die Exemplare bei dem Vorfall einen Monat
zuvor. Aufgrund der Wiederholung der Vorfälle entschied sich die
Bibliothek dazu, einen Wachdienst zu engagieren (rbb24, 2021, o. S.).

\hypertarget{forschungsdesign-und-methodisches-vorgehen}{%
\section{3 Forschungsdesign und methodisches
Vorgehen}\label{forschungsdesign-und-methodisches-vorgehen}}

Die Aufgabe der vorliegenden Arbeit ist es herauszufinden, wie der
Wachdienst in den Arbeitsort Bibliothek eingebunden ist. Aufgrund des
Literatur- und Forschungsmangels zu diesem spezifischen Thema (Kapitel
2), bietet es sich an, mit Menschen zu sprechen, die mit dem Gegenstand
vertraut sind. Da sich die Zusammenarbeit mit Sicherheitspersonal von
Bibliothek zu Bibliothek unterscheidet und sich dieses Verhältnis am
besten von involvierten Personen beschreiben lässt, wurde ein
qualitativer Forschungsansatz gewählt. Wie Cornelia Helfferich (2011)
betont, ist der Auftrag der qualitativen Forschung das Verstehen
subjektiver Ansichten und es wird mit \enquote{sprachlichen Äußerungen}
als Gegenstand gearbeitet (Helfferich, 2011, S. 21).

Um sich dem Themenkomplex \emph{Sicherheitsdienst in Bibliotheken} zu
nähern, ist es unabdingbar, mit Personen zu sprechen, welche praktische
Erfahrungen mit dieser Berufsgruppe im bibliothekarischen Arbeitsumfeld
haben. Somit wurden für diese Arbeit Interviews mit leitenden
Angestellten, die in direktem Kontakt zum Sicherheitspersonal stehen, in
öffentlichen sowie wissenschaftlichen Bibliotheken geführt. Um einen
möglichst großen Spielraum für individuelle Nachfragen zu haben und
trotzdem auf vorbereitete Fragen zurückgreifen zu können, wurde die
Methode des teilstrukturierten Leitfadeninterviews gewählt. Die
\enquote{Kernideen des teilstrukturierten Interviews sind die
Orientierung an einem Leitfaden und der gleichzeitig flexible, aber
durchgängig reflektierte Umgang mit dem Leitfaden} (Werner, 2013, S.
130). Der Interviewleitfaden wurde nach dem SPSS-Prinzip von Cornelia
Helfferich (2011) entwickelt. Diese Methode beinhaltet vier Schritte,
das \textbf{S}ammeln, \textbf{P}rüfen, \textbf{S}ortieren und
\textbf{S}ubsumieren von Fragen (Helfferich, 2011, S. 182--185).

Aufgrund der COVID-19-Pandemie war klar, dass die Interviews für die
vorliegende Arbeit nicht vor Ort geführt werden können. Es wurde
entschieden, die Interviews digital mit Hilfe der Videokonferenzsoftware
Zoom durchzuführen. Über die Mailingliste \emph{Forumoeb} und die
beruflichen Kontakte der Autorin wurden geeignete Personen für die
Interviews gesucht. Es wurden auch Anstrengungen unternommen,
Sicherheitsfirmen zu kontaktieren und um ein Gespräch zu bitten.

Es kamen acht Interviews zustande, an denen neun Personen beteiligt
waren. Ein Interview wurde mit zwei Personen geführt. Die Interviewten
sind alle zuständig für den Wachdienst ihrer Bibliothek. Die Stichprobe
enthält vier Öffentliche Bibliotheken, von denen alle Stadtbibliotheken
mit unterschiedlich großem Einzugsgebiet sind. Eine der Öffentlichen
Bibliotheken ist Teil einer größeren Einrichtung. Vier weitere
Bibliotheken sind wissenschaftliche Bibliotheken und mit einer
Universität verbunden. Die untersuchten Bibliotheken, als Arbeitsort der
Wachdienste, sind über ganz Deutschland verteilt. Es konnten leider
keine Mitarbeitenden eines Sicherheitsdienstes für die Untersuchung
gewonnen werden.

Die Interviews wurden Anfang November 2021 durchgeführt. Aufgrund einer
Absage wurde am 15.12.21 ein weiteres Interview geführt. Die Interviews
dauerten 30--45 Minuten und wurden aufgezeichnet. Anschließend wurden
die Interviews mit Hilfe der Textanalysesoftware MAXQDA2020
transkribiert und anhand der \emph{Thematic Analysis} nach Virginia
Braun und Victoria Clarke (2006) analysiert. Diese Methode zeichnet sich
dadurch aus, dass sie dabei hilft, Muster und Themen innerhalb der Daten
zu identifizieren (Braun \& Clarke, 2006, S. 79).

Für diesen Artikel wurden die Namen der Teilnehmenden, der Einrichtung
und der Orte anonymisiert.

\hypertarget{ergebnispruxe4sentation}{%
\section{4 Ergebnispräsentation}\label{ergebnispruxe4sentation}}

Wie der Wachschutz in den Arbeitsort Bibliothek eingebunden ist, möchte
die vorliegende Untersuchung klären. Aus der Analyse der Interviewdaten
ergaben sich folgende neun Kategorien, welche die Beantwortung der
Forschungsfrage unterstützen:

\begin{itemize}
\item
  Eckdaten/Informationen über die Bibliothek
\item
  Gründe für/gegen Wachschutz
\item
  Die Aufgaben des Wachschutzes
\item
  Die Einweisung des Wachschutzes
\item
  Die Stellung des Wachschutzes in der Bibliothek
\item
  Die Außenwirkung des Wachschutzes
\item
  Positive Erfahrungen
\item
  Negative Erfahrungen
\item
  Probleme/Verbesserungswünsche
\end{itemize}

Im Folgenden werden die Ergebnisse anhand dieser Kategorien dargestellt.
Im Sinne der qualitativen Forschung erfolgt erst die Analyse und
Präsentation der einzelnen Erfahrungen, um anschließend Vergleiche
vorzunehmen und zusammenfassend zu erläutern (Flick et al., 2019, S.
23). Es werden die Kenntnisse einzelner Personen sowie Gemeinsamkeiten
der untersuchten Bibliotheken vorgestellt.

Die Interviews verdeutlichen, dass der Wachschutz in Bibliotheken
vielseitige Aufgaben hat und sich die Intensität der Zusammenarbeit mit
dem Bibliothekspersonal von Einrichtung zu Einrichtung unterscheiden
kann. Alle Interviewteilnehmenden sehen einen Vorteil darin,
Sicherheitspersonal als zusätzliche Fachkräfte in der Bibliothek zu
beschäftigen. Allerdings wird diese Entwicklung auch differenziert und
nicht ohne Skepsis betrachtet. Die Zusammenarbeit beinhaltet durchaus
Herausforderungen, wie die Interviews gezeigt haben.

\enquote{Eigentlich ist die Erfahrung gut. Also es kostet viel Arbeit,
es kostet viele Nerven, es ist immer wieder dranbleiben, immer wieder
erklären, immer wieder sagen wir möchten es so haben, nein nicht so, ja
wir stehen dafür da, nein es dürfen eigentlich im Normalfall alle in die
Bibliothek, wir heißen alle willkommen. Das ist {[}\ldots{]} eine
Never-ending Story, aber im Großen und Ganzen {[}...{]} könnte ich mir
den Betrieb des Hauses nur schlecht vorstellen ohne die Bewachung}\\
(Person8, 2021, Pos. 72).

\hypertarget{eckdateninformationen-uxfcber-die-bibliotheken}{%
\subsection{4.1 Eckdaten/Informationen über die
Bibliotheken}\label{eckdateninformationen-uxfcber-die-bibliotheken}}

Während des Arbeitsprozesses wurden neun Themenbereiche identifiziert,
welche die Interviews prägten. Ein Themenbereich, die Kategorie
\emph{Eckdaten/Informationen über die Bibliotheken} liefert
Kontextinformationen zu den einzelnen Teilnehmenden und ihrem
Arbeitsumfeld. Dieser Themenbereich hilft nur indirekt dabei, die
Forschungsfrage zu beantworten, ist aber trotzdem wichtig, um die
Erfahrungen der Teilnehmenden einordnen zu können.

An den Interviews nahmen vier wissenschaftliche und vier öffentliche
Bibliotheken teil. In der Tendenz zeigt sich, dass wissenschaftliche
Bibliotheken von Eröffnung an Sicherheitspersonal beschäftigen, im
Gegensatz zu den Öffentlichen Bibliotheken, welche oft erst zu einem
späteren Zeitpunkt Wachschutz engagierten.

Oft liegen die Bibliotheken in der Nähe des Bahnhofs. Es wird mehrfach
erwähnt, dass es dadurch in der Bibliothek Probleme geben kann, zum
Beispiel im Zusammenhang mit Drogen (Person2, 2021, Pos. 26; Person3,
2021, Pos. 64; Person5, 2021, Pos. 104).

In der Hälfte der Interviews wurde erwähnt, dass die Architektur des
Bibliotheksgebäudes unübersichtlich gestaltet ist, zum Beispiel durch
Zwischenebenen und/oder verwinkelte Raumstruktur. Dies betrifft sowohl
Alt- wie auch Neubauten (Person1, 2021, Pos. 14; Person6, 2021, Pos. 34;
Person8, 2021, Pos. 12; Person9, 2021, Pos. 17).

Außerdem wurden in den Gesprächen verschiedene Bibliothekskonzepte
angesprochen, welche in dem besprochenen Haus umgesetzt werden oder
wünschenswert wären. In allen Interviews wird zum Ausdruck gebracht,
dass die Bibliothek ein Ort für alle/offener Ort sein soll. Die
Bibliothek als Lernort sowie dritter Ort wird erwähnt (Person4, 2021,
Pos. 66 \& 74) ebenso wie das Konzept der personalfreien Bibliothek
\emph{Open Library} (Person2, 2021, Pos. 20 \& 22) und die
24-Stunden-Bibliothek (Person6, 2021, Pos. 14 \& 18).

\hypertarget{gruxfcnde-fuxfcrgegen-wachschutz}{%
\subsection{4.2 Gründe für/gegen
Wachschutz}\label{gruxfcnde-fuxfcrgegen-wachschutz}}

Die Notwendigkeit, Sicherheitspersonal in Bibliotheken einzusetzen, wird
unterschiedlich begründet. Auch die Bedenken, die einen solchen Schritt
begleiten, sind verschieden. Alle untersuchten Bibliotheken haben eine
externe Firma mit der Aufgabe des Wachschutzes betraut und die Aufträge
werden regelmäßig neu ausgeschrieben, wie im Öffentlichen Dienst üblich.
Hieraus ergaben sich zwei Positionen, die von verschiedenen Interviewten
geteilt werden. Es wird positiv gesehen, dass durch die externe Firma
eine gewisse Ausfallsicherheit gewährleistet wird, da es der
Sicherheitsfirma obliegt, sich zum Beispiel bei Krankheit von
Mitarbeitenden um Ersatz zu kümmern. Außerdem wären die Bibliotheken
nicht in der Lage, so viele Stellen selbst zu besetzen (Person6, 2021,
Pos. 102; Person8, 2021, Pos. 28; Person9, 2021, Pos. 49).

Andererseits entstehe durch die Ausschreibungen und dem damit
verbundenen Wechsel von Firmen ein hoher Arbeitsaufwand für die
Bibliotheken. Aufgrund der sich regelmäßig wiederholenden Einweisungen
für neue Mitarbeitende, ist ein stetiger Dialog darüber, was die
Bibliotheken vom Sicherheitspersonal erwarten, welche Aufgaben sie
übernehmen sollen und welches Verhalten und Auftreten gewünscht ist,
nötig (Person4, 2021, Pos. 36; Person5, 2021, Pos. 124; Person8, 2021,
Pos. 72; Person9, 2021, Pos. 75).

Viele Gründe, die für einen Wachdienst sprechen, beziehen sich auf die
Unterstützung in der Bibliothek. Die wissenschaftlichen Bibliotheken
kontrollieren oft strenger, welche Gegenstände mit in die Bibliothek
genommen werden dürfen. Für diese Eingangskontrolle benötige die
Bibliothek die Hilfe des Wachdienstes (Person1, 2021, Pos. 10). Der
Wachschutz wurde bei einigen Bibliotheken zum Thema, als die
Öffnungszeiten verlängert werden sollten (Person1, 2021, Pos. 10;
Person2, 2021, Pos. 20; Person6, 2021, Pos. 14; Person9, 2021, Pos. 11).

\begin{flushright}
\enquote{Es wurde dann mehr in den Nullerjahren mit der Verlängerung,
mit den zunehmenden Verlängerungen der Öffnungszeiten, dass wir, ich
glaub 2006 haben wir sonnabends länger geöffnet und Sonntagsöffnung
eingeführt. Wir haben dann die Öffnungszeiten vor fünf oder sechs Jahren
bis 24 Uhr verlängert und diese Randzeiten wurden immer mehr mit dem
Wachdienst bewacht} \linebreak(Person9, 2021, Pos. 11).
\end{flushright}

Während der Analyse fiel auf, dass einige Bibliotheken
Sicherheitspersonal einstellen, um die erweiterten Öffnungszeiten ohne
Bibliothekspersonal abdecken zu können, den Wachdienst aber während der
Zeiten mit Personal nicht benötigen (Person2, 2021, Pos. 22; Person6,
2021, Pos. 18). Andere Bibliotheken wiederum benötigten den Wachdienst
für die gesamten Öffnungszeiten (Person1, 2021, Pos. 10; Person3, 2021,
Pos. 16; Person4, 2021, Pos. 66; Person5, 2021, Pos. 30; Person8, 2021,
Pos. 16; Person9, 2021, Pos. 11).

Die Unterstützung und Entlastung des Bibliothekspersonals durch
Sicherheitspersonal war ein wiederkehrendes Thema in den Gesprächen. So
werden potenzielle Konfliktsituationen oft von den Mitarbeitenden des
Wachdienstes geklärt, dies bedeute weniger Stress für die
Bibliotheksangestellten (Person2, 2021, Pos. 52; Person4, 2021, Pos. 30;
Person8, 2021, Pos. 70).

Es wurde in mehreren Interviews erwähnt, dass die Bibliothek zu einer
der meistbesuchten Kultureinrichtungen der Stadt gehört und ein
beliebter Aufenthaltsort ist, weswegen sich unterschiedliches Publikum
in sehr großer Anzahl in den Bibliotheken aufhält (Person4, 2021, Pos.
6; Person5, 2021, Pos. 16; Person8, 2021, Pos. 90 \& 92). Um den
Interessenausgleich der verschiedenen Gruppen und eine Grundordnung zu
gewährleisten, wurde ein Wachdienst eingestellt. Bei den Öffentlichen
Bibliotheken war, bis auf eine, der Sicherheitsdienst nicht von Anfang
an miteingeplant, sondern wurde aufgrund von Veränderungen zu einem
späteren Zeitpunkt engagiert.

Obwohl alle Interviewten die Notwendigkeit dafür sehen,
Sicherheitspersonal in ihren Bibliotheken zu beschäftigen, bestehen oder
bestanden auch Zweifel, ob eine Bibliothek mit Wachschutz überhaupt
vorstellbar ist oder ob dies nicht der Idee einer Bibliothek als offenes
Haus widerspricht (Person2, 2021, Pos. 32; Person4, 2021, Pos. 28).

\hypertarget{die-aufgaben-des-wachschutzes}{%
\subsection{4.3 Die Aufgaben des
Wachschutzes}\label{die-aufgaben-des-wachschutzes}}

Die Aufgaben des Sicherheitspersonals in Bibliotheken sind vielseitig.
Sie reichen von der Publikumsaufsicht bis hin zu der Überwachung von
Brandschutzanlagen. Es sind viele einzelne Pflichten, denen die
Mitarbeitenden des Wachdienstes nachkommen müssen. Die Hauptaufgaben in
den Bibliotheken sind bei allen untersuchten Einrichtungen in etwa
gleich.

In den folgenden Beschreibungen wird nicht auf die corona-spezifischen
Aufgaben eingegangen, da diese einer Ausnahmesituation geschuldet sind
und nicht zu den vorgesehenen Pflichten des Wachdienstes gehören. Nur so
viel, alle Interviewten bestätigten, dass das Sicherheitspersonal dafür
zuständig war, die Eingangskontrolle, die das Überprüfen von
Zertifikaten, Tests und Masken beinhaltete, durchzuführen.

In allen untersuchten wissenschaftlichen Bibliotheken und in einer
Öffentlichen Bibliothek ist das Sicherheitspersonal für die Eingangs-
und/oder Ausgangskontrolle zuständig. Diese ist unterschiedlich stark
ausgeprägt. Der Wachschutz kontrolliert in diesen Bibliotheken, aufgrund
der Hausordnung, welche Gegenstände in die Bibliothek oder Teile der
Bibliothek mitgenommen werden (Person1, 2021, Pos. 12; Person3, 2021,
Pos. 30; Person9, 2021, Pos. 15). Außerdem greift der Wachdienst ein,
wenn die Buchsicherungsanlage anschlägt (Person6, 2021, Pos. 42;
Person8, 2021, Pos. 16; Person9, 2021, Pos. 19), betreut die
Schließfächer (Person1, 2021, Pos. 12; Person3, 2021, Pos. 30), hilft
bei Problemen mit den Ausleih- und/oder Rückgabeautomaten (Person1,
2021, Pos. 40; Person8, 2021, Pos. 16) und ist bei Erstfragen
ansprechbar (Person9, 2021, Pos. 15).

Dies bildet einen starken Kontrast zu den analysierten Öffentlichen
Bibliotheken, dort wird darauf geachtet, dass sich der Wachdienst
\emph{nicht} im Eingangsbereich befindet und wenn doch, dann im
Hintergrund (Person2, 2021, Pos. 32; Person4, 2021, Pos. 28; Person5,
2021, Pos. 90; Person8, 2021, Pos. 34). Eine interviewte Person
beschrieb es so: \enquote{{[}A{]}uf keinen Fall wollten wir die unten an
der Tür stehen haben, weil es für mich ganz wichtig ist, dass der Zugang
zur Bibliothek frei ist} (Person4, 2021, Pos. 28).

Darauf zu achten, dass die Hausordnung eingehalten wird, ist in allen
untersuchten Bibliotheken die Aufgabe des Sicherheitspersonals, genauso
wie regelmäßige Rundgänge durch das Haus, auf den manchmal sehr großen
und unübersichtlichen Flächen der Bibliotheken, um gegebenenfalls
Besucher*innen auf die geltenden Regeln aufmerksam zu machen und den
Besuchenden ein Sicherheitsgefühl zu vermitteln.

Außerdem ist das Sicherheitspersonal dafür zuständig, dass Diebstähle,
Vandalismus und andere Arten von Straftaten verhindert werden (Person1,
2021, Pos. 12; Person2, 2021, Pos. 50; Person5, 2021, Pos. 118) und
Situationen, die zu eskalieren drohen, entschärft werden. Auch bei
bereits bestehenden Konfliktsituationen soll der Wachdienst eingreifen
(Person1, 2021, Pos. 12; Person3, 2021, Pos. 30; Person6, 2021, Pos. 42;
Person9, 2021, Pos. 47).

Auf die Frage, wen oder was der Wachdienst beschützt, stand der
Personenschutz im Mittelpunkt der Antworten. Zum einen schützt der
Wachschutz die Mitarbeitenden der Bibliothek, wenn diese in eine
Konfliktsituation mit Nutzenden geraten oder von diesen bedrängt werden
(Person1, 2021, Pos. 28; Person3, 2021, Pos. 72; Person4, 2021, Pos.
118; Person8, 2021, Pos. 64), zum anderen schützt das
Sicherheitspersonal aber auch die Nutzenden voreinander, falls diese in
Konflikte geraten (Person1, 2021, Pos. 28; Person3, 2021, Pos. 72;
Person5, 2021, Pos. 118; Person7, 2021, Pos. 80).

Der Gebäude- und Bestandsschutz wurde auch als Aufgaben des
Sicherheitspersonals genannt. Wie vielfältig die Aufgaben im
Gebäudeschutz sind, hängt davon ab, wie sehr der Wachschutz in die
Betreuung der technischen Anlagen eingebunden ist. Der Bestandsschutz
hat insgesamt eine eher untergeordnete Rolle. Der Wachdienst ist auch
dafür zuständig, Diebstähle zu verhindern, aber wenn es um die
Schutzfunktion geht, werden die Medien nicht als Priorität gesehen
(Person1, 2021, Pos. 28; Person4, 2021, Pos. 72; Person7, 2021, Pos. 80
Person8, 2021, Pos. 64).

Zusammenfassend kann gesagt werden, dass der Wachdienst in Bibliotheken
zum einen Aufgaben übernimmt, um das Bibliothekspersonal zu entlasten
und dafür zu sorgen, dass die Nutzenden ihren unterschiedlichen
Interessen nachgehen können, ohne andere dabei zu stören oder gegen die
Hausordnung zu verstoßen. Andererseits kümmert sich der Wachschutz um
die Gebäudesicherheit und die damit anfallenden Aufgaben.

\hypertarget{die-einweisung-des-wachschutzes}{%
\subsection{4.4 Die Einweisung des
Wachschutzes}\label{die-einweisung-des-wachschutzes}}

Die Einweisung oder Einarbeitung der Wachdienstmitarbeitenden läuft in
den Bibliotheken unterschiedlich ab. Sie reichen von sehr umfangreichen
Konzepten, welche beinhalten, dass viele Gespräche geführt werden,
ausführliche Unterlagen vorhanden sind, und dass in der Ausschreibung
schon festgehalten wird, wie viele Stunden für die Einarbeitung
berechnet werden sollen (Person9, Pos. 25), bis hin zu Bibliotheken, die
wenig bis keinen Einfluss darauf haben, wie das Sicherheitspersonal
eingewiesen wird (Person1, 2021, Pos. 14; Person3, 2021, Pos. 46).

\begin{flushright}
\enquote{Hmm, das ist so ein bisschen ein Knackpunkt. Es ist bei uns ja
nicht oft so, dass wir einen Wechsel haben und wenn, informiert sich der
Wachschutz eigentlich gegenseitig. Dass sie sich selbst einweisen, also,
dass der Kollege, der die Schicht übernimmt, eigentlich einen Tag vorher
kommt und von dem Stammpersonal eingewiesen wird. Da haben wir
eigentlich nicht wirklich was mit zu tun, {[}\ldots{]} wir werden
eigentlich in der Regel nur aktiv, wenn was schiefläuft, dass wir sagen,
wir nehmen die nochmal zur Seite} \linebreak(Person1, 2021, Pos. 14).
\end{flushright}

Es gibt unterschiedliche Ebenen bei der Einweisung von
Sicherheitspersonal. Zum einen gibt es die Besprechungen und
Vereinbarungen mit den Vorgesetzten in der Wachschutzfirma und zum
anderen die Einarbeitung der Mitarbeitenden vor Ort. Bei den
wissenschaftlichen Bibliotheken, die in dieser Arbeit untersucht wurden,
handelt es sich um Bibliotheken, die an eine Universität angeschlossen
sind. In den meisten Fällen ist der Wachdienst für die gesamte
Universität zuständig und nicht nur für die Aufsicht in der Bibliothek.
Dies bedeutet, dass die Bibliotheken keinen großen Einfluss oder
Mitspracherecht im Auswahlverfahren haben (Person1, 2021, Pos. 10;
Person3, 2021, Pos. 40; Person6, 2021, Pos. 58 \& 60).

Alle Interviewten sind sich einig, dass viele und fortlaufende Gespräche
mit dem Sicherheitspersonal wichtig sind, um diesem den gewünschten
Umgang in einer Bibliothek näherzubringen. Es wird Wert darauf gelegt,
dass die Wachdienstmitarbeitenden \enquote{den Geist des Hauses
verstehen} (Person4, 2021, Pos. 38) und es wird in aktiver Arbeit
versucht den \enquote{Servicegedanken} (Person8, 2021, Pos. 32) und das
\enquote{Selbstverständnis} (Person5, 2021, Pos. 52) der Bibliothek
weiterzugeben.

Diese Bemühungen rühren auch daher, dass viele der Interviewten
festgestellt haben, dass es nicht immer einfach ist, die gewünschten
Verhaltensweisen oder das Selbstbild einer Bibliothek, Personen zu
erklären, die ohne den Hintergrund einer bibliothekarischen Ausbildung
in einer solchen Einrichtung arbeiten. Dies kann zu Verständnis- und
Umsetzungsproblemen, die Aufgaben betreffend, führen.

\begin{flushright}
\enquote{{[}W{]}eil diese Wachleute ja mit einem ganz anderen
Hintergrund kommen, die haben\ldots, die bewachen ja meistens ganz
andere Objekte als eine Bibliothek. Also, es ist ja ein Unterschied, ob
jemand Wachschutz macht vor einer Diskothek oder bei einer Messe oder
der Eine, der vorher hier im Bahnhof eingesetzt {[}wurde{]}. Das ist ja
eine ganz andere Art, mit den Menschen umzugehen und es ist schon immer
wieder eine Aufgabe, dem klarzumachen, so, das ist unser
Selbstverständnis, das hat ja ein Stück weit auch mit Toleranz zu tun.
{[}\ldots{]} {[}W{]}ir wissen, dass unser Publikum heterogen ist, und
dass wir das akzeptieren und das muss man denen schon auch immer wieder
klar machen} \linebreak(Person5, 2021, Pos. 52).
\end{flushright}

Außerdem wird die Wichtigkeit betont, dass auch die Ansprechperson in
der Sicherheitsfirma eine Vorstellung vom Selbstbild der Bibliothek hat
und was sie von den Mitarbeitenden erwartet (Person2, 2021, Pos. 56;
Person8, 2021, Pos. 74).

Des Weiteren wird das Thema \emph{Soft Skills}, also Umgangsformen,
Ausdrucksweise und allgemeines Verhalten von Sicherheitspersonal
gegenüber den Nutzenden, mehrfach angesprochen (Person2, 2021, Pos. 41;
Person3, 2021, Pos. 54; Person8, 2021, Pos. 42). Die Interviewten
erwarten von den Wachdienstmitarbeitenden, dass sie freundlich,
respektvoll und offen auf die Bibliotheksbesucher*innen zugehen. Die
Umsetzung dieser Erwartungen wird teilweise als nicht ganz einfach
angesehen.

Die Untersuchung hat gezeigt, dass sich die Einweisungen von
Sicherheitspersonal in den einzelnen Bibliotheken stark unterscheiden,
allerdings scheinen sie bei allen einen hohen Arbeitsaufwand zu
verursachen. Gemeinsam haben sie, dass Wert auf gute Umgangsformen bei
den Mitarbeitenden des Wachdienstes gelegt wird und es scheint ein
Bedürfnis zu geben, diesen auch die Vision und die Werte der jeweiligen
Bibliothek näher zu bringen.

\hypertarget{die-stellung-des-wachschutzes-in-der-bibliothek}{%
\subsection{4.5 Die Stellung des Wachschutzes in der
Bibliothek}\label{die-stellung-des-wachschutzes-in-der-bibliothek}}

\begin{flushright}
\enquote{Ja, das wäre schon schön, wenn man das schaffen würde, ein Team
zu bilden, was so auf der gleichen Grundlage arbeitet, sozusagen mit dem
gleichen Berufsethos. Mit der gleichen Vorstellung davon, wie unser
Verhältnis zu unseren Kunden sein soll. Ja, so ein, ich weiß nicht, ich
würde jetzt sagen, so einen gemeinsamen Spirit einfach {[}\ldots{]}.Das
wäre schön, wenn daran die Security auch Teil hätte}
\linebreak
(Person5, 2021, Pos. 148).
\end{flushright}

Die Tatsache, dass der Wachdienst in allen Bibliotheken über eine
externe Firma beschäf-\linebreak tigt wird, ist ein Thema, welches nicht nur den
Einstellungs- und Einweisungsprozess beeinflusst, sondern auch das
Zusammengehörigkeitsgefühl und die Zusammenarbeit mit den
Bibliotheksmitarbeitenden zu prägen scheint. Die Personen arbeiten zwar
alle im und für dasselbe Haus, haben aber verschiedene
Aufgabenschwerpunkte und unterschiedliche Arbeitgeber*innen. Die
Bibliotheken gehen unterschiedlich damit um. Die einen versuchen das
Sicherheitspersonal mehr ins Team einzubeziehen, andere bevorzugen die
klare Trennung.

Wie die Bibliotheken dies handhaben, hängt auch davon ab, wie sehr sich
die verschiedenen Arbeitsbereiche im Alltag überschneiden. In einigen
Bibliotheken ist der Wachdienst zum Beispiel nur zu den Uhrzeiten ohne
Bibliothekspersonal da (Person2, 2021, Pos. 66; Person7, 2021, Pos. 22)
oder es gibt eine räumliche Trennung der Arbeitsplätze, die es
erschwert, dass die Wachschutz- und Bibliotheksmitarbeitenden sich als
Team verstehen (Person3, 2021, Pos. 24; Person9, 2021, Pos. 35).

Bei anderen Bibliotheken wiederum ist das Sicherheitspersonal mehr in
die Abläufe der Bibliothek eingebunden und es kommt vor, dass
Mitarbeitende des Wachdienstes, die schon länger in der Bibliothek
arbeiten auch bibliotheksspezifische Aufgaben übernehmen. Zum Beispiel
helfen sie Nutzenden mit den verschiedenen Automaten (Person1, 2021,
Pos. 40; Person8, 2021, Pos. 16) oder beantworten einfache Fragen in
Bezug auf die Bibliothek (Person7, 2021, Pos. 20; Person9, 2021, Pos.
15).

Das Verhältnis zwischen den Mitarbeitenden der Bibliothek und des
Sicherheitsdienstes scheint nicht immer einfach zu sein. Obwohl alle
Interviewten bestätigten, dass sie um ein kollegiales Miteinander bemüht
sind, scheint es im Alltag gewisse Herausforderungen zu geben. Der
Wachschutz bedeutet eine Entlastung für das Bibliothekspersonal, dadurch
muss aber darauf geachtet werden, dass die Bibliotheksmitarbeitenden
unliebsame Aufgaben nicht nur an den Wachdienst abgeben, sondern die
Aufgaben auch weiterhin selbst ausführen (Person4, 2021, Pos. 30;
Person9, 2021, Pos. 89).

\begin{flushright}
\enquote{{[}F{]}ür uns ist es wirklich eine Entlastung, weil man auch
die Konflikte auslagert. Also, wenn ich jetzt eine Gruppe Jugendliche
zur Ordnung rufen muss, das ist natürlich unangenehm und macht Stress
und es ist einfacher, wenn ich den Sicherheitsdienst bitten kann, da mal
vorbei zu gehen {[}...{]}. Und das ist, finde ich, aber auch wieder
ambivalent, weil ich sage, aus meinem Laden schmeiße ich die Leute raus.
Also wir haben da die Verantwortung, wir können das auch nicht komplett
delegieren an den Sicherheitsdienst. {[}\ldots{]} Aber die Versuchung
ist groß, dass man alles auf den Sicherheitsdienst dann abwälzt, was
irgendwie unangenehm ist oder mit der Durchsetzung der Regeln zu tun
hat. Da muss man ein bisschen drauf achten} \linebreak(Person4, 2021, Pos. 30).
\end{flushright}

In den Gesprächen machte es den Anschein, als ob teilweise bewusst
zwischen Bibliotheks- und Sicherheitsdienstmitarbeitenden Unterschiede
gemacht werden. Diese Unterschiede scheinen eher von den Angestellten
selbst auszugehen und weniger vom Publikum, welches oft alle Personen
als Mitarbeitende der Bibliothek wahrnimmt (Person9, 2021, Pos. 43). Es
werden aber auch kollegiale Verhältnisse geschildert. Wichtig ist es den
Interviewten auch, dass alle Mitarbeitenden in der Bibliothek sich
bewusst sind, dass man ein gemeinsames Ziel verfolgt, und dass ein
Vertrauensverhältnis geschaffen wird (Person2, Pos. 90; Person4, 2021,
Pos. 92; Person5, 2021, Pos. 148). Es wird in den Interviews mehrfach
erwähnt, dass die Fluktuation bei den Sicherheitsfirmen und der damit
verbundene Wechsel der Wachdienstmitarbeitenden in den Bibliotheken den
Aufbau eines guten Verhältnisses erschwert (Person2, 2021, Pos. 90;
Person3, 2021, Pos. 56; Person4, 2021, Pos. 60; Person8, 2021, Pos. 74).

Zusammenfassend kann gesagt werden, dass das Sicherheitspersonal in den
Bibliotheken verschieden stark in das Team der Bibliotheksmitarbeitenden
eingebunden wird. Die Untersuchung scheint darauf hinzudeuten, dass die
unterschiedlichen Arbeitgeber*innen es dem Personal, trotz der Arbeit in
und für dieselbe Einrichtung, erschweren, sich als gemeinsames Team zu
sehen. Nichtsdestotrotz sind die Bibliotheken um kollegiale und
vertrauensvolle Arbeitsverhältnisse bemüht.

\hypertarget{die-auuxdfenwirkung-des-wachschutzes}{%
\subsection{4.6 Die Außenwirkung des
Wachschutzes}\label{die-auuxdfenwirkung-des-wachschutzes}}

Ein Umstand, der das Bibliothekspersonal von
Sicherheitsdienstmitarbeitenden nach außen hin unterscheidet, ist, dass
die Angestellten des Wachdienstes meistens Berufskleidung tragen. Dies
kann ein Anzug sein oder ein T-Shirt, das mit einem Logo oder einem
Schriftzug auf die Firma hinweist, für die sie arbeiten (Person4, 2021,
Pos. 28; Person5, 2021, Pos. 62; Person7, 2021, Pos. 87; Person8, 2021,
Pos. 60; Person9, 2021, Pos. 37). Abgesehen davon, dass sie dieser
Umstand von dem Bibliothekspersonal abhebt, scheint es auch dazu zu
führen, dass sie eine gewisse Autorität ausstrahlen, die anderen
Mitarbeitenden zu fehlen scheint (Person1, 2021, Pos. 26; Person2, 2021,
Pos. 54; Person4, 2021, Pos. 28). Hierzu zählt auch, dass das
Sicherheitspersonal ein anderes Auftreten hat; dies wird auf ihre
Ausbildung zurückgeführt (Person5, 2021, Pos. 62).

Ein sehr wichtiger Aspekt, der die Außenwirkung von
Wachdienstmitarbeitenden in der Bibliothek betrifft, ist das
Sicherheitsgefühl, das den Besucher*innen, aber auch dem
Bibliothekspersonal vermittelt wird. Dieser Umstand wurde in allen
Interviews angesprochen. Es wurden verschiedene Begebenheiten oder
Situationen genannt, in denen sich die Menschen in der Bibliothek
sicherer fühlen, wenn Sicherheitspersonal anwesend ist. Durch die
Hausrundgänge der Wachschutzmitarbeitenden wird den Nutzenden das Gefühl
vermittelt, dass sich jemand um ihre Sicherheit kümmert (Person3, 2021,
Pos. 24; Person4, 2021, Pos. 72; Person9, 2021, Pos. 67). Mehrfach
genannt wurde auch, dass sich die Besucher*innen gerade in den Abend-
und Nachtstunden sicherer fühlen, wenn Sicherheitspersonal präsent ist
(Person2, 2021, Pos. 74; Person3, 2021, Pos. 72; Person9, 2021, Pos.
67). Durch die Anwesenheit des Sicherheitspersonals haben sowohl das
Personal wie auch das Publikum Ansprechpersonen, falls es zu
unangemessenem Verhalten oder kritischen Situationen kommt (Person3,
2021, Pos. 72; Person4, 2021, Pos. 42 \& 72; Person5, 2021, Pos. 144).

Des Weiteren sind die Wachdienstmitarbeitenden in einigen Bibliotheken
die ersten Kontaktpunkte und somit die ersten Personen, die von
Besucher*innen gesehen werden (Person6, 2021, Pos. 90; Person1, 2021,
Pos. 34; Person3, 2021, Pos. 40; Person8; 2021, Pos. 16; Person9, 2021,
Pos. 15). Diese Tatsache wird nicht unkritisch gesehen. Die Interviewten
erzählen auch, dass das Sicherheitspersonal das \enquote{Gesicht der
Bibliothek} (Person1, 2021, Pos. 12) ist und die Bibliothek nach außen
repräsentiert (Person4, 2021, Pos. 96). Aus diesem Grund wünschen sich
die Bibliotheksleitungen, dass alle Mitarbeitenden den Besucher*innen
respektvoll, freundlich und offen begegnen (Person4, 2021, Pos. 98).

Außerdem waren sich die meisten befragten Personen einig, dass das
Sicherheitspersonal einige Menschen davon abhalten könnte, in die
Bibliothek zu kommen. Zum einen könnte dies Personen betreffen, die
\enquote{unter Autorität schon mal gelitten} (Person1, 2021, Pos. 26)
haben oder Menschen, die einen Fluchthintergrund haben, \enquote{weil
die einfach eine andere Erfahrung gemacht haben mit solchen Personen}
(Person8, 2021, Pos. 54). Zum anderen können auch Leute abgehalten
werden, die \enquote{skurrile Absichten} (Person2, 2021, Pos. 74) haben
oder in der Bibliothek, aufgrund ihres Verhaltens, nicht gerne gesehen
werden (Person5, 2021, Pos. 100). In einigen Fällen wurde vermutet, dass
eventuell Menschen ohne Wohnsitz durch Wachdienstpersonal abgeschreckt
sein könnten, dies bestätigte sich aber im Laufe der Zeit nicht
(Person6, 2021, Pos. 72; Person8, 2021, Pos. 54; Person9, 2021, Pos.
41).

\hypertarget{positive-erfahrungen}{%
\subsection{4.7 Positive Erfahrungen}\label{positive-erfahrungen}}

Mehr als die Hälfte der Interviewten berichten, dass sie positive
Erfahrungen damit gemacht haben, wenn die gleichen Personen vom
Wachdienst über eine längere Zeit in der Bibliothek arbeiten, weil alle
Mitarbeitenden aufeinander abgestimmt sind und verinnerlicht wird, was
die Bibliothek will (Person4, 2021, Pos. 32). Außerdem wissen und können
Sicherheitsdienstmitarbeitende, die länger in derselben Einrichtung
sind, mehr und werden besser vom Bibliothekpersonal akzeptiert (Person3,
Pos. 40; Person7, 2021, Pos. 100). Mehrfach wurde erwähnt, dass sich
langjährige Mitarbeitende besser mit der Bibliothek identifizieren und
sich dies positiv auf die (gemeinsame) Arbeit auswirkt (Person2, 2021,
Pos. 62; Person3, 2021, Pos. 48; Person5, 2021, Pos. 72).

Interviewte berichteten weiterhin, dass es Sicherheitspersonal gab,
welches nach erneuten Ausschreibungen zu den neuen Arbeitgeber*innen
wechselte, um weiter in der Bibliothek arbeiten zu können. Dies wurde
von Seiten der Bibliothek positiv wahrgenommen (Person3, 2021, Pos. 40
\& 50; Person9, 2021, Pos. 71). Außerdem wurde es mehrheitlich positiv
gewertet, wenn Wachdienstmitarbeitende sich einbringen, Vorschläge für
Verbesserungen machen oder Unterstützung anbieten (Person1, 2021, Pos.
34; Person2, 2021, Pos. 60; Person3, 2021, Pos. 48 \& 80).

\begin{flushright}
\enquote{Aber im Großen und Ganzen jetzt hier für unsere Zweigbibliothek
zu sprechen, haben wir total Glück {[}\ldots{]}, da das eigentlich ein
festes Team ist von zwei Kollegen, die sich hier abwechseln und die
beiden Kollegen sind einfach super. Also, das sieht an anderen
Standorten glaub ich ein bisschen anders aus, da haben wir hier schon,
glaub ich, ziemliches Glück, weil wir eben diese zwei festen Kollegen
haben, die auch schon über Jahre da sind und die die Bibliothek kennen}
\linebreak(Person1, 2021, Pos. 12).
\end{flushright}

\hypertarget{negative-erfahrungen}{%
\subsection{4.8 Negative Erfahrungen}\label{negative-erfahrungen}}

Auch negative Erfahrungen mit Sicherheitspersonal wurden während der
Interviews geschildert. In diesem Abschnitt soll auf Ereignisse
eingegangen werden, die hauptsächlich einzelne Mitarbeitende betrafen.
Allgemeine Herausforderungen werden im nächsten Kapitel (4.9) näher
beschrieben.

Es gab Berichte darüber, dass einzelne Wachdienstmitarbeitende nicht die
nötige Distanz gegenüber Besucher*innen oder Mitarbeitenden aufwiesen.
Dies äußerte sich in zwei Fällen so, dass Sicherheitsdienstmitarbeiter
sich unangemessen und aufdringlich gegenüber Frauen verhielten (Person1,
2021, Pos.16 \& 32). Dies hatte zur Folge, dass sie nicht mehr in der
Bibliothek eingesetzt wurden.

Drei Interviewte beschreiben die fehlende Distanz von
Wachschutzmitarbeitenden gegenüber Nutzenden der Bibliothek als Problem,
weil es zu einer \enquote{Verbrüderung} (Person9, 2021, Pos. 85)
zwischen Sicherheitspersonal und Besucher*innen kam oder in einem
anderen Fall ließ sich ein Mitarbeiter aus Mangel an professioneller
Distanz leicht von Jugendlichen provozieren (Person8, 2021, Pos.
54--58). Dann gab es noch den Fall, dass eine Person eine
\enquote{Vater-, Freundrolle} einnahm und sich \enquote{mit den
Jugendlichen auf den Fußboden gesetzt und das erstmal ausdiskutiert} hat
(Person4, 2021, Pos. 34).

Des Weiteren wurde beschrieben, dass das Wachdienstpersonal aus Sicht
der Bibliothek überreagiert hat, wie dieses Beispiel verdeutlicht:

\begin{flushright}
\enquote{Es ist auch so, dass oft dann irgendwie, wenn es Rangeleien
gibt, haben wir die Erfahrung gemacht, dass die Personen dann irgendwie
auf den Boden gedrückt werden, also die Besucher, die dann irgendwie
Stress gemacht haben {[}\ldots{]}, dann sagen wir auch, lieber die
Polizei holen und, ja, das macht halt immer so ein Bild nach außen}
\linebreak(Person8, 2021, Pos. 36).
\end{flushright}

\hypertarget{problemeverbesserungswuxfcnsche}{%
\subsection{4.9
Probleme/Verbesserungswünsche}\label{problemeverbesserungswuxfcnsche}}

\begin{flushright}
\enquote{Ich habe auch festgestellt, dass so Sicherheitsdienstleute, die
sind ja auf eine bestimmte Art ausgebildet und die kennen nur falsch
oder richtig. Bibliotheken kennen ganz viele Grauzonen} \linebreak(Person4, 2021,
Pos. 30).
\end{flushright}

Das einleitende Zitat fasst die Herausforderung gut zusammen, die bei
der Zusammenarbeit zwischen Bibliotheken und Wachdienst herrscht. Im
Folgenden wird geschildert, welche Probleme die Interviewten sehen,
welche Veränderungen sie sich wünschen und welche Lösungsansätze schon
angestoßen wurden. Diese Kategorie dient dazu, genauer auf die möglichen
Vor- und Nachteile von Sicherheitsdiensten in Bibliotheken einzugehen
und das Arbeitsverhältnis genauer zu beleuchten.

Der Ausschreibungsprozess, dem alle Bibliotheken als öffentliche
Einrichtungen unterworfen sind, birgt einige Herausforderungen. Zum
einen die Anforderungen betreffend, welche der zukünftige Wachdienst
erfüllen sollte und zum anderen, wie viel und ob die Bibliotheken
überhaupt Einfluss auf diesen Prozess haben. Es wird mehrheitlich als
eine Schwierigkeit betrachtet, dass die Arbeit des Sicherheitsdienstes
immer wieder neu ausgeschrieben werden muss, wird aber auch als eine
Notwendigkeit innerhalb des Öffentlichen Dienstes angesehen. Etablierte
Abläufe, eine gute Zusammenarbeit oder ein gut funktionierendes Team
muss stets von neuem aufgebaut werden. Mehrere Bibliotheken berichteten,
dass sie schon froh gewesen wären, wenn es keine neue Besetzung hätte
geben müssen, da sie mit dem bestehenden Team sehr zufrieden waren
(Person3, 2021, Pos. 40; Person5, 2021, Pos. 138; Person8, 2021, Pos.
74).

Die Fluktuation innerhalb der Sicherheitsfirmen beschäftigt die Mehrheit
der interviewten Personen. Es wurde außerdem darauf hingewiesen, dass es
im Sicherheitsbereich einen Fachkräftemangel gibt. Dieser Umstand
erschwert es den Wachdienstfirmen qualifiziertes Personal zu finden und
zu halten (Person3, 2021, Pos. 40; Person7, 2021, Pos. 100; Person9,
2021, Pos. 53).

Die Kommunikation stellt sich auf zwei Ebenen als eine Schwierigkeit
heraus, mit der sich die meisten Interviewten konfrontiert sehen. Die
eine Ebene ist auf die Tatsache zurückzuführen, dass die
Wachdienstmitarbeitenden über eine externe Firma angestellt sind und
somit nicht in die Kommunikationskanäle der Einrichtungen eingebunden
sind. Dies bedeutet, dass zum Beispiel wichtige Informationen zu
Änderungen, etwa während der COVID-19-Pandemie, nicht automatisch auch
an das Sicherheitspersonal gehen und daran gedacht werden muss, diese
gesondert zu informieren (Person1, 2021, Pos. 34; Person3, 2021, Pos.
46; Person5, 2021, Pos. 82).

Die zweite Ebene, auf der sich die Kommunikation schwierig gestalten
kann, ist, zwischen den Sicherheitsdienstmitarbeitenden, den
Angestellten und/oder den Besucher*innen der Bibliothek. Die
Bibliotheken wünschen sich von den Wachdienstmitarbeitenden gute
Umgangsformen, und dass sie sich angemessen ausdrücken, damit sie gerade
auch in Konfliktsituationen deeskalierend eingreifen können. Diese
gewünschten Umgangsformen und die Sprachgewandtheit sind allerdings
nicht immer vorhanden (Person1, 2021, Pos. 12 \& 18; Person8, 2021, Pos.
40). Dies wird auf die verschiedenen Ausbildungs- und allgemeine
Bildungshintergründe zurückgeführt sowie auf die unterschiedlichen
Umgebungen, in denen Sicherheitspersonal in der Regel arbeitet (Person1,
2021, Pos. 18 \& 40; Person4, 2021, Pos. 94; Person5, 2021, Pos. 56;
Person9, 2021, Pos. 27).

Die Wünsche und Erwartungen der Bibliotheken an die Sicherheitsdienste
werden durchaus auch kritisch reflektiert:

\begin{flushright}
\enquote{Die Frage ist halt ein bisschen, wo die Grenze ist, was man
erwarten kann, das ist halt so die andere Frage. Es sind ja letztendlich
dann doch nur Leute, die einen Monat eine Ausbildung gemacht haben und
wenn dann kommunikative Fähigkeiten erwartet werden, wie nach einem
abgeschlossenen Hochschulstudium, das kann man dann eben auch nicht
erwarten} \linebreak(Person9, 2021, Pos. 29).
\end{flushright}

Auf beiden Ebenen der Kommunikation sehen die Verantwortlichen in den
Bibliotheken Ver\-bes\-se\-rungs- und Handlungsbedarf. So gibt es zum Beispiel
die Idee, ein Dokument mit Standardformulierungen aufzusetzen und dieses
zur Unterstützung an die Wachdienstmitarbeitenden auszugeben (Person1,
2021, Pos. 18) oder sehr detaillierte Checklisten zu formulieren, um
\enquote{die Bibliothekspolitik auch dem Wachpersonal nahe zu bringen}
(Person9, 2021, Pos. 25). Darüber hinaus soll auch die Kommunikation
miteinander gestärkt werden, indem \enquote{gute Kommunikationswege
entwickelt} werden und \enquote{dass man mitdenkt auch für die andere
Gruppe} (Person2, 2021, Pos. 92).

Eine interviewte Person erzählte von einem ganz eigenen Lösungsweg, den
die Bibliothek eingeschlagen hat, angesichts des Problems mit den vielen
Wechseln und den Kommunikationsproblemen. Die Idee ist, eine*n
Bibliotheksmitarbeiter*in mit der Zuständigkeit für die Einarbeitung und
Betreuung des Wachdienstpersonals zu beauftragen. Diese Person soll neue
Sicherheitsdienstmitarbeitende einweisen, sich mit den technischen
Aspekten auskennen und die Ansprechperson im Alltag sein. So würde eine
dauerhafte Schnittstelle zwischen dem externen Wachschutz und der
Bibliothek geschaffen werden, die unabhängig von den Ausschreibungen und
dem Personalwechsel in die Einrichtung integriert wäre (Person3, 2021,
Pos. 68, 70, 80 \& 82).

\hypertarget{beantwortung-der-forschungsfrage}{%
\section{5 Beantwortung der
Forschungsfrage}\label{beantwortung-der-forschungsfrage}}

Die Forschungsfrage, die mit der vorliegenden Arbeit beantwortet werden
soll, lautet: \emph{Wie ist der Wachschutz in den Arbeitsort Bibliothek
eingebunden?} Von den vorliegenden Daten und deren Analyse ausgehend,
kann diese Frage nicht allgemeingültig beantwortet werden. Die
Untersuchung hat gezeigt, dass die Rolle des Wachdienstes in der
Bibliothek durchaus individuelle Züge der einzelnen Einrichtungen
aufweist und auf deren Bedürfnisse abgestimmt ist. Trotzdem gibt es auch
Gemeinsamkeiten, die die untersuchten Bibliotheken teilen, wie die
Auswertung gezeigt hat. Seien dies die Hauptaufgaben des
Sicherheitspersonals, Interessenausgleich schaffen, Sicherheitsgefühl
vermitteln und die Hausordnung durchsetzen, nur um einige zu nennen,
oder die alltäglichen Herausforderungen, welche das Ausschreibungssystem
im Öffentlichen Dienst mit sich bringt. Außerdem die Erfahrung, dass die
Einarbeitung und Betreuung mit viel Aufwand verbunden ist.

Auf die Tätigkeiten des Wachdienstes bezogen wurde festgestellt, dass
die Aufgabenbereiche größtenteils klar von denen der
Bibliotheksmitarbeitenden getrennt werden können. Allerdings gibt es
auch hier Aufgaben, die sich überschneiden oder die von beiden Parteien
ausgeführt werden sollen, wie zum Beispiel Nutzende auf Regelverstöße
hinzuweisen. Auf die Zusammenarbeit zwischen den Mitarbeitenden bezogen,
stellte sich heraus, dass eine gewisse, manchmal auch gewünschte,
Distanz vorhanden zu sein scheint. Teilweise wird auch
Zwischenmenschlich ein Unterschied gemacht, ob die Person der Bibliothek
angehört oder der Gruppe der externen Dienstleister*innen. Die
unterschiedlichen beruflichen und teils persönlichen
(Bildungs-)Hintergründe werden bei Kommunikations- und
Umgangsdifferenzen vermutet.

\hypertarget{limitationen}{%
\subsection{5.1 Limitationen}\label{limitationen}}

Der qualitative Ansatz der vorliegenden Arbeit erlaubte es, einzelne,
subjektive Erfahrungen genau zu untersuchen. Aus diesem und dem Grund
der begrenzten Anzahl von Teilnehmenden, erhebt die Untersuchung keinen
Anspruch auf Vollständigkeit oder Repräsentativität. Ursprünglich
sollten auch Wachdienstmitarbeitende für diese Untersuchung interviewt
werden. Die Anfragen an Sicherheitsfirmen blieben jedoch unbeantwortet.
Eine weitere Überlegung war es, eine Umfrage unter den
Bibliotheksbesucher*innen zu machen, allerdings wurde diese Idee, auf
Grund von fehlenden Ressourcen und zeitlichen Grenzen, wieder verworfen.
Außerdem hatte sich die Präsenz des Wachdienstes in vielen Bibliotheken
durch die COVID-19-Pandemie stark verändert. So stehen die
Sicherheitsdienstmitarbeitenden während der Pandemie wegen der
Einlassbeschränkungen größtenteils direkt am Eingang der Bibliothek
(auch bei Bibliotheken, die regulär keinen Wachdienst beschäftigen) und
kontrollieren die Besucher*innen. Die Nutzenden losgelöst von dieser
aktuellen Entwicklung zum Thema Wachschutz zu befragen, wäre unmöglich
gewesen und hätte somit nicht dem Ziel der Arbeit gedient. Deshalb
besteht eine Limitierung der präsentierten Arbeit darin, dass nur eine
Perspektive auf das Thema berücksichtigt werden konnte. Nichtsdestotrotz
konnte mit den verfügbaren Daten die Forschungsfrage beantwortet und ein
detailreicher Einblick in die Thematik gewährt werden.

\hypertarget{fazit-und-ausblick}{%
\section{6 Fazit und Ausblick}\label{fazit-und-ausblick}}

Das Ziel der vorliegenden Arbeit war es, die Forschungsfrage zu
beantworten und eine bestehende Lücke in der Forschung, zumindest in
Teilen, zu schließen. Die Arbeit untersuchte, wie Sicherheitspersonal in
den Arbeitsort Bibliothek eingebunden ist. Dafür wurde die qualitative
Methode des teilstrukturierten Leitfadeninterviews gewählt und mit einer
umfassenden Recherche zu der Thematik flankiert. Für die Untersuchung
wurden acht Interviews durchgeführt, welche die Erfahrungen von neun
Bibliotheksmitarbeitenden in leitenden Positionen abbilden.

Die durchgeführte Untersuchung hat gezeigt, dass die Eingliederung von
Sicherheitspersonal in Bibliotheken zum einen unterschiedlich stark
angestrebt wird und zum anderen gewisse Herausforderungen mit sich
bringt. Der Wachdienst wird als notwendige Komponente gesehen, um die
Bibliothek in dem gewünschten Umfang betreiben und ein möglichst
ungestörtes Miteinander unterschiedlicher Interessengruppen
gewährleisten zu können. Der Umstand, dass das Wachschutzpersonal über
eine externe Firma beschäftigt wird, hat einen finanziellen und
personellen Vorteil, da die Sicherheitsfirmen meistens über einen großen
Pool an Mitarbeitenden verfügen, die bei Ausfällen sofort eingesetzt
werden können. Allerdings birgt die Zusammenarbeit mit Externen auch
einige Herausforderungen, wie die Untersuchung gezeigt hat. Regelmäßige
Ausschreibungen, wie sie im Öffentlichen Dienst bei externen
Dienstleistungen üblich sind, machen es den Bibliotheken schwer, ein
langfristiges Vertrauensverhältnis zu dem Personal der Sicherheitsfirmen
aufzubauen. Die Fluktuation, die in vielen dieser Firmen herrscht, wird
hier ebenfalls zum Problem. So müssen in den Bibliotheken viele
Arbeitsstunden für regelmäßig neu anfallende Einweisungen, wiederholte
Konzeptvermittlung und Betreuung des Wachdienstpersonals investiert
werden. Außerdem scheint es so, dass die unterschiedlichen
(Aus-)Bildungshintergründe der Bibliotheks- und
Sicherheitsmitarbeitenden zu Kommunikationsproblemen führen können.
Darüber hinaus gibt es offenbar teilweise auch ein unterschiedliches
Verständnis hinsichtlich der Arbeitsausführung, welches wiederum
Verständigungsschwierigkeiten und Unverständnis hervorbringen kann.

Trotz der großen Herausforderungen, die die Zusammenarbeit von
Bibliotheken und Sicherheitsdiensten begleiten, sind die interviewten
Personen grundsätzlich mit deren Einsatz zufrieden und bemühen sich um
ein gutes Betriebsklima. Wie in den Gesprächen festzustellen war, gibt
es von Seiten der Bibliotheken verschiedene Bestrebungen, die
Zusammenarbeit zu verbessern und zu erleichtern. Sei dies indem in der
Ausschreibung Ansprüche und Arbeitsfelder genau definiert werden, die
Einarbeitung ausgeweitet wird oder auch die Bibliotheksmitarbeitenden
bestärkt werden, das Sicherheitspersonal als Verbündete mit demselben
Ziel zu sehen.

Aufgrund des noch kaum erforschten Themas \emph{Wachschutz in
Bibliotheken} gerade im deutschsprachigen Raum, ergeben sich
verschiedene Ansätze, um diese Thematik weiter und eingehender zu
untersuchen. Die vorliegende Arbeit konnte die Perspektive von leitenden
Angestellten, welche für das Sicherheitspersonal in ihrer Bibliothek
zuständig sind, analysieren und vorstellen. Weitere Perspektiven, die
Aufschluss über die Rolle und den Einfluss von Wachschutz in
Bibliotheken geben könnten, wären Bibliotheksmitarbeitende, die im
Alltag mit und neben dem Wachdienstpersonal arbeiten. Außerdem die
Angestellten von Sicherheitsdiensten, die in Bibliotheken tätig sind und
auch die Besucher*innen von Bibliotheken mit Wachschutz. Diese
unterschiedlichen Perspektiven würden dabei helfen, einen umfassenden
Einblick in die Thematik zu gewähren, um daraus beispielsweise
Handlungsempfehlungen und neue Formen der alltäglichen Zusammenarbeit zu
entwickeln. Des Weiteren wäre es vorstellbar, dass sich durch die
Analyse der verschiedenen Perspektiven weitere Bedürfnisse und eventuell
auch Ängste/Vorbehalte herausarbeiten ließen. Diese Erkenntnisse könnten
zum Beispiel von Bibliotheksverbänden dazu genutzt werden, Leitfäden
oder auch Schulungen für die Zusammenarbeit mit Sicherheitspersonal in
Bibliotheken zu entwickeln.

Ein weiterer Aspekt, der in diesem Zusammenhang untersucht werden
könnte, sind die Auswirkungen auf Arbeitsbedingungen, Betriebsklima
sowie die Zusammenarbeit der verschiedenen Parteien, welche durch das
Outsourcing von Dienstleistungen entstehen.

In interdisziplinärer Zusammenarbeit mit anderen
Wissenschaftsdisziplinen, wie der Stadtsoziologie, den
Sozialwissenschaften oder auch der sozialen Arbeit, könnten Ursachen und
Herausforderungen tiefergehend auf gesellschaftliche Phänomene hin
untersucht werden. Im besten Falle könnten Lösungsansätze erarbeitet
werden, um das Zusammenspiel von Bibliothekspublikum jeden Alters mit
diversen Interessen und Bedürfnissen, Bibliotheksmitarbeitenden sowie
Sicherheitspersonal in Zukunft verständnisvoller und ausgewogener zu
gestalten.

\hypertarget{literaturnachweise}{%
\section{7 Literaturnachweise}\label{literaturnachweise}}

Albrecht, S. (2015). \emph{Library Security: Better Communication, Safer
Facilities.} American Library Association.
\url{https://ebookcentral.proquest.com/lib/huberlin-ebooks/detail.action?docID=2070053}

Christensen, A. (2017). Benutzungsdienste zwischen Automation, Bau und
Technik. \emph{ABI-Technik}, 2017-11-27, Vol.37 (3), 249--255.
\url{https://doi.org/10.1515/abitech-2017-0056}

Braun, V. \& Clarke, V. (2006). Using thematic analysis in psychology.
\emph{Qualitative Research in Psychology}, 3(2), 77--101.
\url{https://doi.org/10.1191/1478088706qp063oa}

Dohrmann, A. \& Siegel, A. (2009). Sicherheit durch Prävention. Die
Konferenz Nationaler Kultureinrichtungen (KNK) entwickelt einen
Handlungsleitfaden für Museen, Archive und Bibliotheken.
\emph{Zeitschrift für Bibliothekswesen und Bibliographie.} Vol.
56(3--4), 200--207. \url{http://dx.doi.org/10.3196/18642950095634102}

Duden, R. (2015, 28. Mai). \emph{Vom Nachtwächter zum Lernortmanager? --
Neue Herausforderungen für das Qualitätsmanagement von Wachdiensten in
wissenschaftlichen Bibliotheken.} 104. Deutscher Bibliothekartag,
Nürnberg, Deutschland.
\url{https://opus4.kobv.de/opus4-bib-info/frontdoor/index/index/docId/1727}

Eichhorn, M. (2015). \emph{Konflikt- und Gefahrensituation in
Bibliotheken. Ein Leitfaden für die Praxis.} De Gruyter Saur.

Eichhorn, M. (2016). \enquote{Konflikt- und Gewaltsituationen lassen
sich im Team am besten lösen} -- Experte Martin Eichhorn gibt Tipps zur
Deeskalation / Konsequentes Handeln ist wichtig. \emph{BuB Forum
Bibliothek und Information}, 68, 06/2016, 330--333.
\url{https://opus4.kobv.de/opus4-bib-info/files/2707/BuB_2016_06_330_333.pdf}

\sloppy
Emskötter, I. (2021). \enquote{So'n Tag für Hamburg}. Zweijähriges
Pilotprojekt zur Sonntagsöffnung der Zentralbibliothek der Bücherhallen.
\emph{BuB Forum Bibliothek und Information}, 73, 07/2021, 377.
\url{https://zs.thulb.uni-jena.de/servlets/MCRFileNodeServlet/jportal_derivate_00300983/BUB_2021_07_377.pdf}

Flick, U., von Kardorff, E. \& Steinke, I. (2019). Was ist qualitative
Forschung? Einleitung und Überblick. In U. Flick, E. von Kardorff \& I.
Steinke (Hrsg.), \emph{Qualitative Forschung. Ein Handbuch} (S. 13--29).
Rowohlt Taschenbuch Verlag.

Frevel, B. (2012). Kriminalität und lokale Sicherheit. In E. Frank
(Hrsg.), \emph{Handbuch Stadtsoziologie} (S. 593--611). Springer VS.
\url{https://link.springer.com/book/10.1007/978-3-531-94112-7}

Graham, W. (2012). \emph{The Black Belt Librarian: Real-World Safety \&
Security.} American Library Association.

Helfferich, C. (2011). \emph{Die Qualität qualitativer Daten. Manual für
die Durchführung qualitativer Interviews.} VS Verlag.

Jopp, R. K. (Hrsg.). (1991). \emph{Sicherheit in Bibliotheken:
Raumsicherung - Buchsicherung, Arbeitsplatz - Brandschutz; Referate
einer Fortbildungsveranstaltung der Baukommission des DBI am 2./3.
Oktober 1985 in Duisburg.} Deutsches Bibliotheksinstitut.

Kaufmann, S. (2017). Das Themenfeld \enquote{Zivile Sicherheit}. In Gusy
C., Kugelmann D. \& Würtenberger T. (Hrsg), \emph{Rechtshandbuch Zivile}
\emph{Sicherheit} (S. 3--22). Springer.
\url{https://link.springer.com/book/10.1007/978-3-662-53289-8?page=1\#toc}

Kammersgaard, T. (2021). Private security guards policing public space:
using soft power in place of legal authority, \emph{Policing and
Society}, 31:2, 117--130.
\url{https://doi.org/10.1080/10439463.2019.1688811}

Lashley, E. L. (2008). Library Safety and Security -- Campus/Community
Police Collaboration. \emph{Library \& Archival Security}, 21:2,
195--201. DOI: 10.1080/01960070802202027

McGinty, J. (2008). Enhancing Building Security: Design Considerations,
\emph{Library \& Archival Security}, 21:2, 115--127. DOI:
10.1080/01960070802201474

McGuin, H. (2010). The Evolution of Security at Sims Memorial Library: A
Case Study\emph{, Library \& Archival Security}, 23:2. 105--115, DOI:
10.1080/01960075.2010.495328

Morey, M. (1999). Whose space is it anyway? A study of young
people\textquotesingle s interaction with security guards in New South
Wales, \emph{Australian Social Work}, 52:3, 51--56.
\url{https://doi.org/10.1080/03124079908414136}

Osiewacz, F. (2017a, 11. März). \emph{Provokation und Pöbelei in Hammer
Zentralbibliothek.} Der Westfälische Anzeiger.
\url{https://www.wa.de/hamm/zentralbibliothek-hamm-erhaelt-sicherheitsdienst-weil-jugendliche-poebeln-provozieren-7626004.html}

Osiewacz, F. (2017b, 22. März). \emph{Ruhe zwischen den Regalen. Nach
Randalen: Einsatz eines Sicherheitsdienstes in Zentralbibliothek zeigt
Wirkung.} Der Westfälische Anzeiger. (Eine Kopie des Artikels aus der
Printausgabe wurde von der Zentralbibliothek Hamm zur Verfügung
gestellt)

rbb24 (2021, 16. September). \emph{Wachschutz eingesetzt. Tempelhofer
Bibliothek wird erneut Ziel von mutmaßlich rechtem Angriff.} rbb24.
\url{https://www.rbb24.de/panorama/beitrag/2021/09/bibliothek-tempelhof-schoeneberg-buecher-zersoert.html}
(letzter Aufruf Juni 2022)

Reed, C. (2008). The Correct Mindset. \emph{Library \& Archival
Security}, 21:2, 59--67. \url{https://doi.org/10.1080/01960070802201334}

Robertson, G. (2014). \emph{Disaster Planning for Libraries. Process and
Guidelines}. Chandos Publishing.

Robinson, B. (2019). No holds barred: Policing and Security in the
Public Library. \emph{In the Library with the lead pipe.}
\url{http://www.inthelibrarywiththeleadpipe.org/2019/no-holds-barred/}

Russew, G. (2021, 12. August). \emph{Tempelhof-Schöneberg.
Bezirksbibliothek wird Ziel mutmaßlich rechtsgerichteter Attacke.}
rbb24.
\url{https://www.rbb24.de/panorama/beitrag/2021/08/berlin-tempelhof-zerstoerte-buecher-bibliothek-mutmasslich-rechte-attacke.html}
(letzter Aufruf Juni 2022)

Saarikkomäki, E., \& Alvesalo-Kuusi, A. (2020). Ethnic Minority Youths'
Encounters With Private Security Guards: Unwelcome in the City Space.
\emph{Journal of Contemporary Criminal Justice,} 36(1), 128--143.
\url{https://doi.org/10.1177/1043986219890205}

Schröder, A. (2021). Neue kriminalpräventive Konzepte für die Sicherheit
im öffentlichen Raum. In H. Lange, C. Kromberg \& A. Rau (Hrsg.),
\emph{Urbane Sicherheit. Migration und der Wandel kommunaler
Sicherheitspolitik} (S. 55--84). Springer VS.
\url{https://link.springer.com/book/10.1007/978-3-658-34398-9}

Trapskin, B. (2008). A Changing of the Guard: Emerging Trends in Public
Library Security. \emph{Library \& Archival Security}, 21:2, 69--76.
\url{http://doi.org/10.1080/01960070802201359}

Verch U. (2006). \emph{Sonntags in die Bibliothek! Die Wiederbelebung
des Bibliothekssonntags in Deutschland}. Logos Verlag Berlin.

Vieth-Entus, S. (2015a, 08. Februar). \emph{Polizeieinsatz am
Bücherregal. Drogenhandel in Neuköllner Bibliothek.} Der Tagesspiegel.
\url{https://www.tagesspiegel.de/berlin/polizeieinsatz-am-buecherregal-drogenhandel-in-neukoellner-bibliothek/11341912.html}

Vorberg, M. (2009). »Law around-the-clock« Die 24-Stunden-Öffnung der
Hengeler Mueller-Bibliothek der Bucerius Law School in Hamburg.
\emph{BuB Forum Bibliothek und Information}, 61, 01/2009, 4445.

Werner, P. (2013). Qualitative Befragungen. In K. Umlauf, S.
Fühles-Ubach \& M. Seadle (Hrsg.), \emph{Handbuch Methoden der
Bibliotheks- und Informationswissenschaft. Bibliotheks-,
Benutzerforschung, Informationsanalyse} (S. 128--151)\emph{.} De Gruyter
Saur.

%autor
\begin{center}\rule{0.5\linewidth}{0.5pt}\end{center}

\textbf{Sara Juen}, (M.A. Information Science) ist wissenschaftliche
Mitarbeiterin am Institut für Gender \& Diversity an der OST -
Ostschweizer Fachhochschule in St. Gallen und Redaktionsmitglied der
LIBREAS. Library Ideas. ORCID: https://orcid.org/0000-0003-0725-8592

\end{document}