\textbf{Kurzfassung}: Wachdienste in deutschen Bibliotheken sind keine
Seltenheit mehr und scheinen sich als Partner*innen im Alltag etabliert
zu haben. Doch was genau macht das Sicherheitspersonal in Bibliotheken
und warum braucht es sie überhaupt? Der folgende Artikel präsentiert die
Ergebnisse einer Interview-Studie, welche zum Ziel hatte herauszufinden,
wie Sicherheitspersonal in den Arbeitsort Bibliothek eingebunden ist.
Dazu wurden Interviews mit Personen geführt, welche in ihrer Bibliothek
für den Wachschutz zuständig sind. Es stellte sich heraus, dass es
zwischen den Bibliotheken Gemeinsamkeiten bezüglich der Anforderungen
und Herausforderungen mit den Wachdiensten gibt, genauso aber auch
unterschiedliche Herangehensweisen und bibliotheksspezifische
Bedürfnisse. Diese Arbeit hatte zum einen das Ziel, eine Lücke in der
aktuellen Forschung zu schliessen und zum anderen die Aufmerksamkeit auf
eine Praxis zu lenken, die in deutschen Bibliotheken immer alltäglicher
zu werden scheint.

\begin{center}\rule{0.5\linewidth}{0.5pt}\end{center}

\textbf{Abstract}: Security guards in German libraries are no longer a
rarity and seem to have established themselves as partners in the daily
business. But what exactly security staff do in libraries and why do
they need them at all? The following article presents the results of an
interview study that aimed to find out how security staff are integrated
into the library as their workplace. For this purpose, interviews were
conducted with people who are responsible for security in their library.
It turned out that there are similarities between the libraries
regarding the requirements and challenges with the security services,
but also different approaches and library-specific needs. This work
aimed on the one hand to fill a gap in current research and on the other
hand to draw attention to a practice that seems to become more and more
commonplace in German libraries.
