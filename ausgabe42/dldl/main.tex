\documentclass[a4paper,
fontsize=11pt,
%headings=small,
oneside,
numbers=noperiodatend,
parskip=half-,
bibliography=totoc,
final
]{scrartcl}

\usepackage[babel]{csquotes}
\usepackage{synttree}
\usepackage{graphicx}
\setkeys{Gin}{width=.4\textwidth} %default pics size

\graphicspath{{./plots/}}
\usepackage[ngerman]{babel}
\usepackage[T1]{fontenc}
%\usepackage{amsmath}
\usepackage[utf8x]{inputenc}
\usepackage [hyphens]{url}
\usepackage{booktabs} 
\usepackage[left=2.4cm,right=2.4cm,top=2.3cm,bottom=2cm,includeheadfoot]{geometry}
\usepackage[labelformat=empty]{caption} % option 'labelformat=empty]' to surpress adding "Abbildung 1:" or "Figure 1" before each caption / use parameter '\captionsetup{labelformat=empty}' instead to change this for just one caption
\usepackage{eurosym}
\usepackage{multirow}
\usepackage[ngerman]{varioref}
\setcapindent{1em}
\renewcommand{\labelitemi}{--}
\usepackage{paralist}
\usepackage{pdfpages}
\usepackage{lscape}
\usepackage{float}
\usepackage{acronym}
\usepackage{eurosym}
\usepackage{longtable,lscape}
\usepackage{mathpazo}
\usepackage[normalem]{ulem} %emphasize weiterhin kursiv
\usepackage[flushmargin,ragged]{footmisc} % left align footnote
\usepackage{ccicons} 
\setcapindent{0pt} % no indentation in captions

%%%% fancy LIBREAS URL color 
\usepackage{xcolor}
\definecolor{libreas}{RGB}{112,0,0}

\usepackage{listings}

\urlstyle{same}  % don't use monospace font for urls

\usepackage[fleqn]{amsmath}

%adjust fontsize for part

\usepackage{sectsty}
\partfont{\large}

%Das BibTeX-Zeichen mit \BibTeX setzen:
\def\symbol#1{\char #1\relax}
\def\bsl{{\tt\symbol{'134}}}
\def\BibTeX{{\rm B\kern-.05em{\sc i\kern-.025em b}\kern-.08em
    T\kern-.1667em\lower.7ex\hbox{E}\kern-.125emX}}

\usepackage{fancyhdr}
\fancyhf{}
\pagestyle{fancyplain}
\fancyhead[R]{\thepage}

% make sure bookmarks are created eventough sections are not numbered!
% uncommend if sections are numbered (bookmarks created by default)
\makeatletter
\renewcommand\@seccntformat[1]{}
\makeatother

% typo setup
\clubpenalty = 10000
\widowpenalty = 10000
\displaywidowpenalty = 10000

\usepackage{hyperxmp}
\usepackage[colorlinks, linkcolor=black,citecolor=black, urlcolor=libreas,
breaklinks= true,bookmarks=true,bookmarksopen=true]{hyperref}
\usepackage{breakurl}

%meta
\expandafter\def\expandafter\UrlBreaks\expandafter{\UrlBreaks%  save the current one
  \do\a\do\b\do\c\do\d\do\e\do\f\do\g\do\h\do\i\do\j%
  \do\k\do\l\do\m\do\n\do\o\do\p\do\q\do\r\do\s\do\t%
  \do\u\do\v\do\w\do\x\do\y\do\z\do\A\do\B\do\C\do\D%
  \do\E\do\F\do\G\do\H\do\I\do\J\do\K\do\L\do\M\do\N%
  \do\O\do\P\do\Q\do\R\do\S\do\T\do\U\do\V\do\W\do\X%
  \do\Y\do\Z}
%meta

\fancyhead[L]{Redaktion LIBREAS\\ %author
LIBREAS. Library Ideas, 42 (2022). % journal, issue, volume.
%\href{https://doi.org/10.18452/xxx}{\color{black}https://doi.org/10.18452/xxx}
{}} % doi 
\fancyhead[R]{\thepage} %page number
\fancyfoot[L] {\ccLogo \ccAttribution\ \href{https://creativecommons.org/licenses/by/4.0/}{\color{black}Creative Commons BY 4.0}}  %licence
\fancyfoot[R] {ISSN: 1860-7950}

\title{\LARGE{Das liest die LIBREAS, Nummer \#11 (Herbst bis Winter 2022)}}% title
\author{Redaktion LIBREAS} % author

\setcounter{page}{1}

\hypersetup{%
      pdftitle={Das liest die LIBREAS, Nummer \#11 (Herbst bis Winter 2022)},
      pdfauthor={Redaktion LIBREAS},
      pdfcopyright={CC BY 4.0 International},
      pdfsubject={LIBREAS. Library Ideas, 41 (2022)},
      pdfkeywords={Literaturübersicht, Bibliothekswissenschaft, Informationswissenschaft, Bibliothekswesen, Rezension, literature overview, library science, information science, library sector, review},
      pdflicenseurl={https://creativecommons.org/licenses/by/4.0/},
      pdfcontacturl={http://libreas.eu},
      baseurl={},
      pdflang={de},
      pdfmetalang={de},
      pdfdoi={},
      pdfurl={}
     }



\date{}
\begin{document}

\maketitle
\thispagestyle{fancyplain} 

%abstracts

%body
Beiträge von Eva Bunge (eb), Ben Kaden (bk), Karsten Schuldt (ks),
Michaela Voigt (mv)

\hypertarget{zur-kolumne}{%
\section{1. Zur Kolumne}\label{zur-kolumne}}

Ziel dieser Kolumne ist es, eine Übersicht über die in der letzten Zeit
erschienene bibliothekarische, informations- und
bibliothekswissenschaftliche sowie für diesen Bereich interessante
Literatur zu geben. Enthalten sind Beiträge, die der LIBREAS-Redaktion
oder anderen Beitragenden als relevant erschienen.

Themenvielfalt sowie ein Nebeneinander von wissenschaftlichen und
nicht-wissenschaftlichen Ansätzen wird angestrebt und auch in der Form
sollen traditionelle Publikationen ebenso erwähnt werden wie
Blogbeiträge oder Videos beziehungsweise TV-Beiträge.

Gerne gesehen sind Hinweise auf erschienene Literatur oder Beiträge in
anderen Formaten. Diese bitte an die Redaktion richten. (Siehe
\href{http://libreas.eu/about/}{Impressum}, Mailkontakt für diese
Kolumne ist
\href{mailto:zeitschriftenschau@libreas.eu}{\nolinkurl{zeitschriftenschau@libreas.eu}}.)
Die Koordination der Kolumne liegt bei Karsten Schuldt, verantwortlich
für die Inhalte sind die jeweiligen Beitragenden. Die Kolumne
unterstützt den Vereinszweck des LIBREAS-Vereins zur Förderung der
bibliotheks- und informationswissenschaftlichen Kommunikation.

LIBREAS liest gern und viel Open-Access-Veröffentlichungen. Wenn sich
Beiträge dennoch hinter eine Bezahlschranke verbergen, werden diese
durch \enquote{{[}Paywall{]}} gekennzeichnet. Zwar macht das Plugin
\href{http://unpaywall.org/}{Unpaywall} das Finden von legalen
Open-Access-Versionen sehr viel einfacher. Als Service an der
Leserschaft verlinken wir OA-Versionen, die wir vorab finden konnten,
jedoch auch direkt. Für alle Beiträge, die dann immer noch nicht frei
zugänglich sind, empfiehlt die Redaktion Werkzeuge wie den
\href{https://openaccessbutton.org/}{Open Access Button} oder
\href{https://core.ac.uk/services/discovery/}{CORE} zu nutzen oder auf
Twitter mit
\href{https://twitter.com/hashtag/icanhazpdf?src=hash}{\#icanhazpdf} um
Hilfe bei der legalen Dokumentenbeschaffung zu bitten.

Die bibliographischen Daten der besprochenen Beiträge aller Ausgaben
dieser Kolumne finden sich in der öffentlich zugänglichen Zotero-Gruppe:
\url{https://www.zotero.org/groups/4620604/libreas_dldl/library}.

\hypertarget{artikel-und-zeitschriftenausgaben}{%
\section{2. Artikel und
Zeitschriftenausgaben}\label{artikel-und-zeitschriftenausgaben}}

\hypertarget{vermischte-themen}{%
\subsection{2.1 Vermischte Themen}\label{vermischte-themen}}

Murphy, Julie A. ; LaCombe, Kent (2022). \emph{Recapturing Misplaced
Opportunities in Academia: The Problematic Privatization of Library
Services and Holding}. In: Serials Review 47 (2022), 3--4, 252--256,
\url{https://doi.org/10.1080/00987913.2022.2044578} {[}Paywall{]}

Dieser Text ist eine Polemik über den Zustand der Privatisierung in und
um das Bibliothekswesen (in den USA). Es wird kritisiert, dass die
Bibliotheken zum Markt von Privatunternehmen geworden seien und dass ein
-- nicht explizit so genanntes -- neoliberales Management dazu geführt
hat, dass öffentliches Geld und öffentliche Ressourcen (also vor allem
aus Steuermitteln bezahlte) nicht im Dienst von Nutzer*innen oder der
Öffentlichkeit stehen, sondern vor allem Profit für Firmen liefern.
Vieles könnte stattdessen von Bibliotheken selbst angeboten oder
betrieben werden, insbesondere im Bereich der Bibliothekstechnologie.

Man lernt inhaltlich nichts Neues in diesem Text. Die Klagen sind
berechtigt, die Situation ist bekannt. Aber gleichzeitig scheint es beim
Lesen, dass es vielleicht wirklich notwendig wäre, immer wieder einmal
darauf hinzuweisen, dass es nicht so sein (und bleiben) muss, wie es
ist, sondern schon einmal anders war und deshalb (vor allem, aber nicht
nur, durch politische Entscheidungen) wieder anders werden kann.

Dass dieser Text dann selbst nicht frei verfügbar ist, sondern hinter
einer Paywall steckt, ist ironisch. Immerhin, so könnte man es
interpretieren, weigerten sich die Autor*innen, Geld für eine APC
auszugeben -- was nur hiesse, wieder für einPrivatunternehmen Profit zu
generieren. (ks)

\begin{center}\rule{0.5\linewidth}{0.5pt}\end{center}

Divall, Pip ; James, Cathryn ; Heaton, Michael (2022). \emph{UK survey
demonstrates a wide range of impacts attributable to clinical library
services}. In: Health Information Library Journal, (2022) 39: 116--131,
\url{https://doi.org/10.1111/hir.12389} {[}Paywall{]}, {[}OA-Version:
\url{http://hdl.handle.net/20.500.12904/14864}{]}

Die Bibliotheken in Spitälern und anderen Gesundheitseinrichtungen in
Grossbritannien haben eine Basis, die mit denen im DACH-Raum nicht
vergleichbar ist. Dadurch, dass es einen National Health Service gibt,
der zentrale Infrastrukturen entwickelt und zum Beispiel auch Lizenzen
für das gesamte System übernimmt, sind sie etablierter und besser
ausgestattet, als dies im deutschen Sprachraum allgemein der Fall ist.
In Grossbritannien sind voll ausgestattete Bibliotheken, die
Dienstleistungen für das medizinische Personal übernehmen --
beispielsweise Recherchen -- Normalität. Insofern sind die Ergebnisse
dieser Studie nicht übertragbar. (Wenn, dann eher in Länder mit
ähnlichen nationalen Gesundheitsstrukturen wie zum Beispiel Kanada.)

Interessant ist die Studie vielleicht, weil sie Potentiale aufzeigt, wie
Bibliotheken arbeiten und wirken könnten. Aber vor allem, weil die
Methodik übernommen und für andere Bibliotheken, auch im DACH-Raum, die
ähnliche Dienstleistungen für spezialisierte Professionen anbieten --
man kann hier zum Beispiel an forschungsnahe Dienstleistungen in
Hochschulen oder, im juristischen Bereich, Dienstleistungen für Gerichte
und Kanzleien denken.

Methodisch wurde ein Fragebogen entworfen und Interviews vorbereitet,
die dann während sechs Monaten von Bibliothekar*innen in den Spitälern
im ganzen Land durchgeführt wurden, um Nutzer*innen, die Services in
Anspruch genommen hatten, zu befragen. Es ging dabei darum zu verstehen,
wer, wofür und mit welchem Erfolg die Services nutzte und das auf der
Ebene ganz Grossbritanniens. Selbstverständlich war es nur eine Auswahl
an Nutzer*innen, die so erreicht wurde -- vor allem die mit positiven
Erfahrungen und Haltungen zur Bibliothek. Gleichzeitig nahmen nicht alle
Spitalbibliotheken und ähnliche Einrichtungen Grossbritanniens an dieser
Studie teil -- zum einen gibt es keine Liste von ihnen, insoweit sind
einige gewiss übersehen worden, zum anderen haben einige schon eigene
Systeme des Qualitätsmanagements installiert und versprachen sich von
der Studie vermutlich keinen zusätzlichen Gewinn. Und dennoch ergibt
sich durch die Menge an Daten, die gesammelt wurde, hier ein besseres
Bild von der Bibliotheksnutzung, als es vorher vorhanden war.

Die Ergebnisse selbst zeichnen ein sehr positives Bild und deuten auch
auf eine steigende Professionalisierung der betreffenden Bibliotheken
hin. Services wie Recherchen und Beschaffung von Informationen werden
sowohl von Ärzt*innen als auch anderem medizinischen Personal genutzt,
sowohl für eigene Forschungen als auch für die medizinische Praxis. Sie
werden sehr positiv bewertet, auch wenn der Effekt nicht einfach benannt
werden kann. Es gibt aber Hinweise, dass durch die Informationen, welche
Bibliotheken zur Verfügung stellten, das Personal besser in der Lage
war, Diagnosen vorzunehmen und Patient*innen zu beraten. Dies hätte in
vielen Fällen Folgekosten, die sonst durch unzureichende Analysen und
weitergehende Krankheitsverläufe entstanden wären, verhindert. Zudem
würde es mehr und mehr normal, dass Bibliothekar*innen in den Artikeln
der Ärzt*innen als Co-Autor*innen genannt werden, was die Autor*innen
dieser Studie als Hinweis darauf deuten, dass deren Kompetenz als
relevante Mitarbeiter*innen immer mehr wahrgenommen wird. (ks)

\begin{center}\rule{0.5\linewidth}{0.5pt}\end{center}

Curry, Claire ; Robbins, Sarah ; Schilling, Amanda ; Tweedy, B. N.
(2022). \emph{Recruiting, Hiring, \& On-Boarding Non-MLS Liaison
Librarians: A Case Study}. In: Library Leadership \& Management 36
(2022) 1, \url{https://doi.org/10.5860/llm.v36i1.7490}

Die Bibliotheken der University of Oklahoma hatten Probleme, zwei
Stellen für Liason Librarians (in diesem Fall einigermassen vergleichbar
mit Fachreferent*innen mit Lehrverantwortung) zu besetzen. Deshalb
änderten sie ihre Ausschreibungen und ihre Strategie, wie sie neue
Kolleg*innen in den Bibliotheken «on-boarden». Der Text beschreibt
diesen Prozess etwas vollmundig als «Case Study», kann aber vor allem
als Anregung dafür verstanden werden, warum es sich lohnt, solche
Ausweitungen vorzunehmen und nach Personal zu suchen, dass nicht direkt
eine «klassische Bibliotheksausbildung» durchlaufen hat, als auch was
dies an Mehrarbeit und Vorteilen mit sich bringt. Beispielsweise wurde
die Ausschreibung so verändert, dass sie nicht nur für Personen mit
einer bibliothekarischen Ausbildung verständlich war, es wurden klare
Aussagen über die Vorteile einer Bibliothekskarriere -- im Gegensatz zu
einer in der Forschung -- gemacht, gleichzeitig wurden in der
Ausschreibung, den Bewerbungsgesprächen und dem späteren on-boarding
klare Vorgaben dazu gemacht, was von dem neuen Personal erwartet wird.
Dies alles war erfolgreich, die Stellen sind besetzt und die Stellung
der Bibliotheken in der Universität -- so deuten die Autor*innen
zumindest an -- verbessert.

Selbstverständlich ist der Text auf die Realitäten in den USA bezogen,
aber das Problem, qualifiziertes Personal zu finden, stellt sich
bekanntlich auch in Bibliotheken im DACH-Raum immer mehr und wird sich
mit den demographischen Veränderungen in Zukunft nur verstärken. Dieser
Text bietet gute Ansatzpunkte, um die Anwerbungsprozesse von
Bibliotheken zu überdenken. (ks)

\begin{center}\rule{0.5\linewidth}{0.5pt}\end{center}

Macfarlane, Isla H. (2022). Visitors visiting books: visitors' books at
the Library of Innerpeffray. In: Studies in Travel Writing, 25 (2022) 3,
315--333, \url{https://doi.org/10.1080/13645145.2022.2057387}

In diesem Artikel wird vor allem das Gästebuch einer schottischen
Bibliothek, die 1680 als erste öffentlich zugängliche Bibliothek in
Schottland gegründet wurde (vom Third Lord Madertie, David Drummond)
behandelt. Die Bibliothek wurde einige Jahrzehnte eher als Club denn als
Bibliothek geführt, aber ab Ende des 19. Jahrhunderts als mehr oder
minder professionelle Bibliothek. Auch heute kann sie noch besucht
werden.

Interessant ist, dass die Bibliothek auch spätestens ab Ende des 19.
Jahrhunderts zu einer touristischen Attraktion wurde, die zum Beispiel
in mehreren Reiseführern erwähnt wird. Seit dieser Zeit wurde ein
Gästebuch geführt, welches sich vollständig erhalten hat. (Die Autorin
des Artikels erwähnt, dass dies selten der Fall ist, weil viele
Gästebücher zerschnitten wurden, um die Autographen berühmter
Persönlichkeiten zu sammeln.) Der Text geht darauf ein, wer sich alles
in diesem Gästebuch verewigt hat. Es ist eine Studie aus dem Bereich der
Tourismusforschung, die einfach eine Bibliothek zum Thema hat. Die
Fragen, die an die Quelle (also das Gästebuch) gestellt werden, sind
deshalb verständlicherweise auch nicht bibliothekswissenschaftlich. (ks)

\begin{center}\rule{0.5\linewidth}{0.5pt}\end{center}

Samara, Afroditi ; Garoufallou, Emmanouel (2022). \emph{Organizational
Change and Librarians' Attitudes in Special Libraries}. In:
International Information \& Library Review {[}Latest Articles{]},
\url{https://doi.org/10.1080/10572317.2022.2100216} {[}Paywall{]}

In einer Umfrage unter Bibliothekar*innen aus griechischen
Spezialbibliotheken zeigen die Autor*innen, dass es eine grundsätzlich
positive Haltung zu Veränderungen gibt. Das Personal stellt sich diesen
nicht entgegen und hat auch selten Bedenken. Diese positive Haltung kann
noch unterstützt werden, wenn klar ist, warum Veränderungen stattfinden
und wenn dann diese Veränderungen auch gut gemanagt werden. Dem Bild
eines grundsätzlich struktur-konservativen Personals in Bibliotheken
entsprechen zumindest die befragten Kolleg*innen nicht. (ks)

\begin{center}\rule{0.5\linewidth}{0.5pt}\end{center}

\pagebreak
Schlembach, Mary C. ; Chrzastowski, Tina (2022). \emph{A Pioneer in
Chemical Literature: Librarian Marion E. Sparks.} In: Bulletin for the
History of Chemistry 47 (2022) 2: 215--221,
\url{http://acshist.scs.illinois.edu/bulletin/bull22-vol47-2.php}
{[}Paywall{]} {[}OA-Version: \url{https://hdl.handle.net/2142/114377}{]}

In diesem Artikel wird das Leben und Wirken einer wissenschaftlichen
Bibliothekarin in den Vereinigten Staaten des frühen 20. Jahrhundert
nachgezeichnet: Marion E. Sparks Karriere war durch enge Kontakte zu
Forschung und Studierenden gekennzeichnet sowie durch für die damalige
Zeit innovative Angebote und Dienstleistungen. Bevor sie jedoch als
Dozentin und Autorin von Aufsätzen und Lehrbüchern anerkannt wurde,
musste sie einige Hürden überwinden. Im Artikel sind beispielsweise
einige Empfehlungsschreiben für Sparks abgedruckt, die ihr gute
berufliche Kompetenz bescheinigen, jedoch auch bemerken, dass andere
Absolventinnen desselben Jahrgangs wesentlich hübscher und charmanter
seien und die Bibliothek des potentiellen Arbeitgebers besser schmücken
würden. Der Artikel gibt insgesamt einen sehr interessanten,
exemplarischen Einblick in das berufliche Leben früher
Bibliothekarinnen. (eb)

\begin{center}\rule{0.5\linewidth}{0.5pt}\end{center}

Snow, Jackie: \emph{Code of Conduct}. In: Wired. 30.09. (Sep.~2022), S.
122--123 {[}gedruckt{]}

Die September-Ausgabe 2022 von WIRED stellt kurz das Projekt
Wampum.Codes der Künstlerin Amelia Winger-Bearskin vor. Ihr geht es
darum, ethische Aspekte unmittelbar in Softwarecode zu explizieren. Im
Prinzip ist der Ansatz, dass die Entwickler*innen eindeutig beschreiben,
wie der von ihnen geschriebene Code aus informationsethischer Sicht
genutzt und nicht genutzt werden soll. Mit wampum.codes hat sie ein
Modell und ein Workshop-Programm entworfen, mit dem ethische
Abhängigkeiten (ethical software dependencies) ausgedrückt werden.
Inspiriert ist diese von einer indigenden, dezentralen Logik ethischen
Verhaltens. (bk)

\begin{center}\rule{0.5\linewidth}{0.5pt}\end{center}

Fortier, Alexandre ; Pretty, Heather J. ; Scott, Daniel B. (2022).
\emph{Assessing the Readiness for and Knowledge of BIBFRAME in Canadian
Libraries}. In: Cataloging \& Classification Quarterly {[}Latest
Articles{]}, \url{https://doi.org/10.1080/01639374.2022.2119456}

In der Umfrage, über welche die Autor*innen hier berichten, wurden
Bibliothekar*innen aller Bibliothekstypen in Kanada dazu befragt, welche
Kenntnisse sie über BIBFRAME haben und wie sie den bevorstehenden
Übergang in der Katalogisierungspraxis hin zu BIBFRAME und Linked Data
planen. Die Ergebnisse sind eher ernüchternd: Je kleiner die
Bibliotheken, umso weniger ist BIBFRAME überhaupt bekannt. Auch ist das
Wissen in allen Bibliothekstypen ausser Wissenschaftlichen Bibliotheken
kaum vorhanden. Selbst die Bibliothekar*innen, welche BIBFRAME an sich
kennen, können konkrete Fragen zum Standard selten richtig beantworten.
Das kanadische Bibliothekswesen ist also kaum darauf vorbereitet, die
Katalogisierungspraxis zu ändern. (Allerdings gibt es bei den
Bibliothekar*innen, die antworten, auch kaum Bedenken gegen eine
Veränderung.) Die Ergebnisse sind eine Erinnerung daran, dass die
Standardisierungsarbeit, die in Kommissionen und Arbeitsgruppen
geleistet wird, erst in der breiten Praxis etabliert werden muss, um
einen Einfluss zu generieren.

Relevant ist der erste Teil des Artikels, in welchem die Autor*innen
nicht nur darstellen, warum es für Bibliotheken relevant wäre, die
Katalogisierungspraxis so zu verändern, dass Bibliothekskataloge zum
Teil des Semantic Web werden, sondern auch darstellen, was sich am
Standard BIBFRAME in den letzten Jahren grundlegend entwickelt hat. Für
alle, die sich schnell à jour stellen wollen, sei dieser Teil zur
Lektüre empfohlen. (ks)

\hypertarget{covid-19-und-die-bibliotheken-fuenfte-welle}{%
\subsection{2.2 COVID-19 und die Bibliotheken, Fünfte
Welle}\label{covid-19-und-die-bibliotheken-fuenfte-welle}}

Guernsey, Lisa ; Prescott, Sabia ; Park, Claire (2022). \emph{A Pandemic
Snapshot: Libraries' Digital Shifts and Disparities to Overcome}. In:
Public Library Quarterly (Latest Articles),
\url{https://doi.org/10.1080/01616846.2022.2073783} {[}Paywall{]}

Den relevanten Teil dieses Artikels stellt die Auswertung einer Umfrage
unter rund 2.600 US-Amerikaner*innen darüber, wie sie die elektronischen
Angebote ihrer jeweiligen Public Library während der ersten Monate der
COVID-19 Pandemie wahrnahmen und nutzten, dar. Leider geht aus dem Text
nicht ganz hervor, wie die Teilnehmenden zu dieser -- von September bis
Oktober 2020 durchgeführten -- Umfrage ausgewählt wurden. Es wird nur
berichtet, dass sie einen Durchschnitt der US-amerikanischen Bevölkerung
darstellen würden und dass sie selber die Wahl trafen, an der Umfrage
teilzunehmen.

Zudem ist die Auswertung eingefasst in einen langen Abschnitt zur
Bedeutung von Public Libraries im Allgemeinen, zu einem ausgewählten
Angebot einer Bibliothek und einem abschliessenden Kapitel, in welchem
Bibliotheken Hinweise gegeben werden, wie sie in Zukunft mehr Menschen,
die von elektronischen Angeboten ausgeschlossen sind, erreichen könnten
-- allerdings, ohne das sich diese Hinweise irgendwie aus den
Ergebnissen selber ergeben würden.

Und dennoch sind die Ergebnisse der Auswertung selber, die mit Vorsicht
interpretiert und im US-amerikanischen Kontext verortet werden müssen,
interessant. So zeigte sich zum Beispiel, dass es grundsätzlich einen
positiven Bezug zu Public Libraries und auch deren elektronischen
Angeboten gab, aber dass nur rund 25\,\% der Befragten sagen konnten, ob
ihre jeweilige Bibliotheken während der ersten Monate der Pandemie
spezifische Angebote aufgebaut hatten. Es zeigte sich bei denen, die
solche Angebote dann nutzten, dass die Pandemie ein Treiber war -- elf
Prozent mehr nutzten diese Angebote erstmals in der Pandemie und von
diesen gaben rund zwei Drittel auch explizit an, dies wegen der Pandemie
zu tun. Grundsätzlich wurden diese Angebote auch positiv eingeschätzt,
aber oft sei es schwer, in ihnen zu recherchieren und das jeweils
Gesuchte zu finden. Zudem gab es einige Variablen -- Einkommen, die
Bibliothek als Hauptzugang zum Internet oder nicht, Alter, Ethnizität --
welche Unterschiede in der Nutzung bedingten. Meistens so, wie man sich
dies denken könnte, aber zum Beispiel zeigte sich auch, dass Personen,
die die Bibliothek sonst als Hauptzugang zum Internet nutzen, die
elektronischen Angebote vor allem für Bildung und Arbeit benutzten,
während andere Gruppen sie eher für Freizeit und Unterhaltung
benötigten. (ks)

\begin{center}\rule{0.5\linewidth}{0.5pt}\end{center}

\pagebreak
Oyelude, Adetoun Adebisi ; Ebijuwa, Adefunke Sarah ; Ahmad, Hauwa Sani ;
Abba; Mabruka Abubakar ; Nongo, Celina Jummai (2022). \emph{Perception
of Librarians on COVID-19 Information and Sensitization: Challenges and
Change Agenda}. In: IJoL -- International Journal of Librarianship 7
(2022) 1: 79--98, \url{https://doi.org/10.23974/ijol.2022.vol7.1.233}

In dieser Studie wurden, per halbstrukturierter Interviews, 13
Bibliothekar*innen aus verschiedenen afrikanischen Ländern dazu befragt,
wie sie die COVID-19 Pandemie und die Reaktionen ihrer
Trägereinrichtungen sowie der jeweiligen nationalen Politik erlebten.
Vier Länder (Botswana, Ghana, Kenia und Uganda) sind mit jeweils einer
Person vertreten, Nigeria mit neun. Die Auswahl erfolgte über
persönliche Kontakte der Autor*innen.

In den Antworten zeigt sich, dass die Pandemie nicht viel anders
wahrgenommen wurde, als in anderen Ländern, nur immer wieder auf Basis
der jeweiligen infrastrukturellen Möglichkeiten. Bibliothekar*innen auf
allen Ebenen versuchten, aktiv mit der Situation umzugehen. Sie
versuchten, sich selbst mit den jeweils geltenden, politisch gesetzten
Regeln auseinanderzusetzen. Und sie versuchten, ihre Nutzer*innen auf
der einen Seite möglichst gut über die Pandemie zu informieren, auf der
anderen Seite möglichst Zugang zu Medien zu verschaffen. Gleichzeitig
fühlten sie sich stellenweise überfordert.

Praktisch übereinstimmend denken sie, eine bessere und weitreichendere
Sensibilisierung der gesamten Gesellschaft für Gesundheitsthemen wäre
eine zukünftige Anforderung, die sich aus den (bisherigen) Erfahrungen
mit der Pandemie ergeben hat. (ks)

\begin{center}\rule{0.5\linewidth}{0.5pt}\end{center}

Watson, Alex P. (2022). \emph{Pandemic Chat: A Comparison of
Pandemic-Era and Pre-Pandemic Online Chat Questions at the University of
Mississippi Libraries}. In. Internet Reference Services Quarterly
{[}Latest Articles{]},
\url{https://doi.org/10.1080/10875301.2022.2117757} {[}Paywall{]}

Der Autor untersucht hier mithilfe beschreibender statistischer
Methoden, ob sich während der Pandemie (bei ihm bis Frühling 2022
gezählt) Veränderungen bei der Nutzung der Chat-Auskunft der im Titel
genannten Universitätsbibliotheken -- an welchen er arbeitet -- zur Zeit
vor der Pandemie ergeben haben. Dies ist möglich, da die gesamten Chats
gespeichert werden und somit als Daten vorliegen. Entgegen der
Erwartung, die man haben könnte, zeigte sich praktisch keine
Veränderung, weder bei der Uhrzeit, in welcher von Studierenden und
Forschenden Chat-Fragen gestellt werden, noch bei der Länge der
eigentlichen Chats (in Minuten). Auch die Auswertung der genutzten
Wörter zeigte keine grosse Bewegung. Grundsätzlich hat die Pandemie also
zumindest auf dieses Chat-Angebot, welches schon vor der Pandemie
etabliert war, und seine Nutzung keinen feststellbaren Einfluss gehabt.
(ks)

\pagebreak

\hypertarget{research-data-management}{%
\subsection{2.3 Research Data
Management}\label{research-data-management}}

Mexhid Ferati, Arben Hajra, Fidan Limani, Vladimir Radevski:
\emph{Research data repository requirements: A case study from
universities in North Macedonia}. In: International Journal of Knowledge
Content Development \& Technology. {[}Online First{]} 15. April 2022,
\url{http://www.ijkcdt.net/xml/32722/32722.pdf}

Die Autor*innen untersuchten die Praxis im Umgang mit Forschungsdaten in
institutionellen Repositorien an Einrichtungen in Nordmazedonien. In der
Umfrage wurden 110 Personen an insgesamt drei Universitäten des Landes
(die South East European University in Tetovo, die Ss. Cyril and
Methodius University of Skopje sowie die Goce Delčev University of Štip)
befragt. Die Anforderungsanalyse ergab sechs Funktionen von Repositorien
für den Umgang mit Forschungsdaten: (1) Metadaten und Dokumentation, (2)
Verbreitung, Teilen und Sichtbarmachen von Daten, (3) das
Zugangsmanagement zu den Daten, (4) Datenspeicherung, (5) Datensicherung
und schließlich (6) Archivierung. Das sind wenig überraschend keine
neuen Einsichten. In der Aufschlüsselung und Begründung können sie aber
Einrichtungen in Nordmadzedonien und angesicht der Universalität der
Anforderungen auch darüber hinaus als Orientierung dienen. (bk)

\begin{center}\rule{0.5\linewidth}{0.5pt}\end{center}

Pollock, Danielle ; Yan, An ; Parker, Michelle ; Allard, Suzie
(2022)\emph{. The Role of Data in an Emerging Research Community:
Environmental Health Research as an Exemplar}. In: International Journal
of Digital Curation, \url{https://doi.org/10.2218/ijdc.v16i1.653}

In dieser Studie versuchen die Autor*innen nachzuvollziehen, wie in
einer jungen, datenbasierten Wissenschaft -- Environmental Health
Research -- Forschungsdaten für Publikationen genutzt werden. Dabei wird
praktisch der Weg zurückverfolgt von Publikationen, welche explizit
angeben, welche Datensätze sie nutzen und dann in Interviews Autor*innen
dieser Papers befragt (insgesamt fünf, die allerdings dann über
Erfahrungen aus ihrer gesamten Forschungspraxis und nicht nur mit den
untersuchten Artikeln berichten). Dies ergibt einen, wenn auch etwas
kurzen, Einblick in die Datennutzungspraxis.

Was für Bibliotheken, gerade solchen, die Ressourcen in Services rund um
das Forschungsdatenmanagement investieren, an dieser Studie relevant
ist, ist die Erkenntnis -- die ein wenig von den bekannten Diskursen um
die wachsende Bedeutung von Forschungsdaten und guter wissenschaftlicher
Praxis überdeckt wird, die auch von den Autor*innen stark reproduziert
wird --, wie low-level dies in der Praxis ist. Die befragten Forschenden
und deren Mit-Forschende nutzen Daten aus verschiedenen Quellen und das
auch erfolgreich. Aber diese Praxis ist relativ pragmatisch: Zur
Bearbeitung und Speicherung werden vor allem solche Angebote wie Google
Docs genutzt, Forschungsgruppen werden vor allem so zusammengestellt,
dass Forschende zur Mitarbeit eingeladen werden, die schon irgendwie
persönlich miteinander bekannt sind oder, manchmal, durch schon bekannte
Publikationen. Es gibt in dieser Wissenschaft eine Kooperationspraxis,
aber am Ende ist es die*der Lead Researcher, welche*r vor allem die
Arbeit managt und bestimmt. Das ist weniger innovativ, kollaborativ oder
an Kriterien von Open Science orientiert, als dies manchmal den Eindruck
macht. Und die Forschenden greifen auch nicht gross auf Services von
Bibliotheken und anderen Einrichtungen zurück, solange sie Zugang zu
Daten haben, die sie benötigen. In gewisser Weise erdet der Artikel die
Praxis des Forschungsdatenmanagement. (ks)

\begin{center}\rule{0.5\linewidth}{0.5pt}\end{center}

Cheung, Melissa ; Cooper, Alexandra ; Dearborn, Dylanne ; Hill,
Elizabeth ; Johnson, Erin ; Mitchell, Marjorie ; Thompson, Kristi
(2022). \emph{Practices Before Policy: Research Data Management
Behaviours in Canada}. In: Partnership: The Canadian Journal of Library
and Information Practice and Research 17 (2022) 1,
\url{https://doi.org/10.21083/partnership.v17i1.6779}

In Vorbereitung darauf, dass die drei grossen Fördereinrichtungen für
wissenschaftliche Forschung in Kanada eine gemeinsame
Forschungsdaten-Policy erlassen würden -- was 2021 passiert ist und dies
mit in solchen Policies jetzt schon üblichen Anforderungen wie
Forschungsdaten\-management-Pläne -- führte ein Konsortium von 20
kanadischen Universitäten eine Umfrage unter den Forschenden und zum
Teil auch Promovierenden und Masterstudierenden durch. Sie wollten --
wie dies in solchen Umfragen üblich ist -- vor allem wissen, wie deren
Forschungsdaten\-management-Praxen aussehen und welche Unterstützung sie
sich wünschen würden. Jede Universität führte die Umfrage eigenständig
durch, dass heisst auch zu unterschiedlichen Zeitpunkten und mit anderen
Strategien der Rekrutierung von Teilnehmenden. Dies passiert zwischen
2015 und 2019. Dreizehn der Einrichtungen lieferten Daten zurück, die in
diesem Artikel für eine übergreifende Auswertung genutzt werden.

Interessant daran ist vor allem, dass der Blick auf eine ganze Anzahl
von unterschiedlichen Einrichtungen geworfen wird und nicht nur, wie
sonst oft, auf nur eine. Hervorzuheben ist auch, dass die Autor*innen
ihre Daten, Umfrageinstrumente und so weiter transparent im
umfangreichen Anhang mit publizierten.

Ansonsten sind die Ergebnisse aber wenig überraschend, sondern decken
sich mit vielen ähnlichen Umfragen: Die Forschungsdatenmanagement-Praxis
der Forschenden ist wenig offen und wenig nachhaltig (der Grossteil der
Daten wird nicht geteilt und stattdessen auf eigenen Rechnern
gespeichert). Es gibt Unterschiede vor allem zwischen Disziplinen und
den Karrierestufen, aber offenbar nicht gross zwischen den Einrichtungen
selber. Die Befragten geben ein Interesse daran an, in Zukunft Daten
offener teilen zu wollen und gleichzeitig dafür Beratung beziehungsweise
-- für Studierende -- Workshops zu besuchen. Die Autor*innen
interpretieren dies dahingehend, dass es dieses Interesse von
Forschenden in Zukunft wirklich geben würde und damit wohl Bibliotheken
Aufgaben finden würden. Aber das ist selbstverständlich eine gewagte
Interpretation. (Eine, die allerdings auch oft am Ende solcher Umfragen
gemacht wird.) (ks)

\hypertarget{dekolonisierung-und-bibliotheken}{%
\subsection{2.4 Dekolonisierung und
Bibliotheken}\label{dekolonisierung-und-bibliotheken}}

Marsh, Frances (2022). \emph{Unsettling information literacy: Exploring
critical approaches with academic researchers for decolonising the
university}. In: Journal of Information Literacy 16 (2022) 1: 4--29,
\url{https://doi.org/10.11645/16.1.3136}

Dekolonisierung ist ein aktuell in vielen Zusammenhängen benutzter
Begriff (auch im letzten Jahr in einem Schwerpunkt der LIBREAS), aber
einer, der schwer in die Praxis umzusetzen ist. Die Autorin dieses
Textes versucht, für Informationskompetenzarbeit von Wissenschaftlichen
Bibliotheken zu klären, wie eine dekoloniale Praxis in Bibliotheken
aussehen kann. Dabei möchte sie über mehr Leselisten oder diversere
Bestände in Bibliotheken -- so wichtig diese auch sind -- hinausgehen.

Methodisch führte sie Interviews mit fünf Forschenden durch, die
einerseits mit Bibliotheken im Bereich Informationskompetenz
zusammenarbeiten und andererseits, wie die Autorin, an Dekolonisierung
interessiert sind. Das alles ist eher als Startpunkt für Überlegungen
gedacht, nicht als (zum Beispiel) Praxisleitfaden. Und es ist -- wieder
einmal ohne dass dies extra thematisiert wird -- im US-amerikanischen
Kontext verortet.

Dennoch zeigen sich interessante Punkte. So verstehen alle Befragten
Dekolonisierung als einen kontinuierlichen Prozess des Lernens /
Ent-lernens, nicht als ein fertig definierbares Ziel. Sie stimmen auch
darin überein, dass die Arbeit, marginalisierte Literatur und
Standpunkte in Informationskompetenzschulungen, Bibliotheksbestände und
Leselisten zu integrieren, nur ein Anfang sei, aber einer, der an sich
schon schwer ist. Dabei thematisieren sie, warum es nicht einfach ist,
diverse Stimmen zu finden und plädieren dafür, dies als Anlass zu
nehmen, darüber nachzudenken, was warum als relevante und für
Universitäten akzeptable Literatur gilt -- beispielsweise welche Formen
an Publikationen akzeptabel sind und welche nicht -- und wie dies
geändert werden kann. Es geht ihnen darum, immer wieder Wege zu suchen,
um «Lücken» zu finden -- beispielsweise im Kanon, im System der
Wissensproduktion von Universitäten, im Verständnis davon, was
«richtiges» Wissen ist -- und diese Lücken zu thematisieren. Sie sollen
zum Ausgangspunkt von Unterricht und Nachdenken werden. Der Text ist --
so will ihn die Autorin auch verstanden wissen -- als Anregung für
weitere Diskussionen gedacht. (ks)

\begin{center}\rule{0.5\linewidth}{0.5pt}\end{center}

Frederick, Donna Ellen (2022). \emph{Libraries, decolonization and the
data deluge.} In: Library Hi Tech News, 39 (2022) 7: 1--12,
\url{https://doi.org/10.1108/LHTN-05-2022-0074} {[}Paywall{]}

In dieser, für diese Zeitschrift recht umfangreichen, Kolumne diskutiert
die Autorin -- zumindest ehemals Bibliothekarin -- aus einer kanadischen
Perspektive die Frage, ob und wie Bibliotheken \enquote{dekolonisiert}
werden können. Die zahlreichen Beispiele beziehen sich dann auch immer
auf den kanadischen und US-amerikanischen Kontext. Sie startet mit der
Entschuldigung des Papstes für die Verbrechen, die in den katholischen
\enquote{Residential Schools} in Kanada an Kindern aus First Nations
begangen wurden und mit der Kritik an dieser. Davon ausgehend führt sie
darin ein, was Kolonisierung und Dekolonisierung grundsätzlich bedeuten
und argumentiert dafür, zu versuchen, dies auch aus der Sicht der
Kolonisierten zu verstehen. Anschliessend diskutiert sie verschiedene
Punkte der bibliothekarischen Arbeit (Bestandsmanagement,
Katalogisierung, grundsätzliche bibliothekarische Überzeugungen von der
Notwendigkeit des Zugangs zu Wissen, die Position von Bibliotheken
gegenüber First Nations) sowie die Möglichkeiten, in diesen aktiv zu
werden. Am Schluss schildert sie auch ein Programm, welches sie
besuchte, um sich mit dem Leben und der Wissensproduktion von First
Nations direkt vertraut zu machen (diese werden in Kanada offenbar als
Fortbildung angeboten). Einerseits tut sie dies, um zu schildern, wie
selbst in diesem Programm von einer Teilnehmerin kolonial geprägte
Verhaltens- und Denkweisen reproduziert wurden. Andererseits aber auch,
wie das Programm es ihr im Nachhinein schwierig macht, Materialien von
First Nations adäquat zu katalogisieren, weil die Systeme, die
Bibliotheken zur Katalogisierung nutzen, überhaupt nicht darauf
ausgelegt sind.

Die Kolumne, gerade für Leser*innen ausserhalb Kanadas, zeigt vor allem,
wie komplex und \enquote{unfertig} das Thema bislang ist. (ks)

\hypertarget{gesellschaft-bibliotheken-und-kritik}{%
\subsection{2.5 Gesellschaft, Bibliotheken und
Kritik}\label{gesellschaft-bibliotheken-und-kritik}}

Shuva, Nafiz Zaman (2022). \emph{\enquote{Everybody Thinks Public
Libraries Have Only Books}: Public Library Usage and Settlement of
Bangladeshi Immigrants in Canada}. In: Public Library Quarterly (Latest
Articles), \url{https://doi.org/10.1080/01616846.2022.2074244}
{[}Paywall{]}

Der Autor dieser Studie behauptet, dass es bislang wenig Forschung dazu
geben würde, wie Migrant*innen Öffentliche Bibliotheken nutzen würden,
führt dann aber selbst eine ganze Reihe betreffender Untersuchungen an.
Anschliessend ergänzt er diese mit Daten, die er über eine spezifische
Gruppe von Migrat*innen -- die im Titel genannten Bangladeshi Immgrants
-- und deren Nutzung von Bibliotheken in Southern Ontario -- das unter
anderem die Metropolen Toronto und Ottawa umfasst -- mittels Interviews
und einer Umfrage erhoben hat. Er ist selber «Insider» in dieser Gruppe.
Die Daten wurden schon vor der Covid-19 Pandemie gesammelt, sagen also
nichts über die spezifische Situation während dieser aus. Zudem heben
sich die meisten der von Shuva Befragten dadurch hervor, dass sie schon
mit einem Hochschulabschluss nach Kanada einreisten. (In anderen Ländern
würden sie deshalb wohl eher «Expats» genannt.)

Was Shuva zeigen kann, ist, dass Public Libraries einen recht guten Ruf
bei diesen Personen geniessen. Sie nutzten sie vor allem kurz nach der
Einreise, sowohl als Ort, um das Internet zu nutzen, als auch um
Informationen über Kanada zu finden. Sie boten einen der Orte, welche
die Integration in die kanadische Gesellschaft erleichterten. Allerdings
wurden von den Personen viele Angebote, die gerade in den Bibliotheken
der Metropolen gemacht werden, um explizit diese Integration zu
unterstützen -- beispielsweise Kontakte zu anderen Organisationen -- gar
nicht benutzt. Und, nachdem die Migrant*innen in ihrer neuen Heimat
«angekommen» waren, sank im Normalfall auch ihre Bibliotheksnutzung. Sie
benutzten sie später zumeist vor allem für ihre Kinder. Der Autor
schliesst aus seinen Daten unter anderem, dass es eine wichtige Aufgabe
für die Bibliotheken wäre, mehr Outreach zu machen, also die eigenen
Angebote zu bewerben. (ks)

\begin{center}\rule{0.5\linewidth}{0.5pt}\end{center}

De Agostini, Michelle (2022). \emph{Locked Up Libraries: A Critique of
Canadian Prison Library Policy}. In: Journal of Radical Librarianship 8
(2022): 1--24,
\url{https://www.journal.radicallibrarianship.org/index.php/journal/article/view/69}

Die Autorin war bis vor Kurzem Bibliothekarin in einem kanadischen
Gefängnis und übt in diesem Text kenntnisreich Kritik an den
Bibliotheken in den Justizvollzugsanstalten Kanadas. Es ist einer dieser
Texte, die wohl vor allem dann entstehen können, wenn Menschen eine
bestimmte Karriere verlassen und sich keine Sorgen mehr darum machen
müssen, wem sie \enquote{auf die Füsse treten}, aber gleichzeitig noch
gut im Thema bewandert sind. Die Kritik ist mehrschichtig. Als
Hauptproblem macht sie aus, dass diese Bibliotheken eingebunden sind in
ein System von Strafe, Bewachung und struktureller Gewalt. Sie würden
selber als Strafe oder Belohnung in diesem System Gefängnis genutzt,
dabei müssten sie eigentlich ein Recht für alle Strafgefangenen
darstellen. De Agostini betont, dass die Strafe schon durch das
Einsperren von Menschen vollzogen ist. Es wäre nicht die Aufgabe des
Gefängnissystems oder der Gefängnisbibliotheken, diese Strafe zu
verstärken oder aber selber zu erziehen. Vielmehr stehe allen Menschen
der Zugang zu Informationen zu und der Staat, welche die Strafe des
Einsperrens vornimmt, hätte die Verpflichtung, diesen freien Zugang zu
gewährleisten, was am Besten über Bibliotheken funktionieren würde, die
professionell geführt, ausreichend ausgestattet und an den Angeboten der
Public Libraries orientiert sein müssten.

Das dem grundsätzlich nicht so ist, wäre der Grund für weitere Probleme:
Die Gefängnisbibliotheken wären unzureichend ausgestattet, die
Gefängnisverwaltung wäre sich oft nicht bewusst, was die Bibliothek sein
müsste und könnte. Auch wenn versucht würde, Bibliotheken in
Gefängnissen anzubieten, würde sich eher auf eine paternalistische
Tradition berufen, die die Gefangenen mittels Literatur erziehen will,
anstatt ihnen ihr Recht auf freie Information zuzugestehen.

De Agostini zeigt in ihrem Text aber auch, dass ihre Kritik Teil einer
Tradition ist: Es gibt, zumindest für Kanada, seit Jahrzehnten Studien,
Berichte und Policy Papers, welche die Situation kritisieren und
Veränderungen vorschlagen oder einfordern. Aber die gleiche Kritik, die
gleichen Forderungen wiederholen -- offenbar ändert sich wenig. (Die
Autorin weisst aber darauf hin, dass elektronische Medien neue Probleme
darstellen, weil diese in Gefängnissen oft gar nicht angeboten werden,
da die Gefangenen keinen Zugang zu den dafür notwendigen Geräten haben.)
Insoweit erstaunt es etwas, dass sich am Ende des Textes wieder
Vorschläge finden, wie die Situation zu ändern wäre. (ks)

\begin{center}\rule{0.5\linewidth}{0.5pt}\end{center}

Price, Apryl C. (2022). \emph{Barriers to an inclusive academic library
collection}. In: Collection and Curation 41 (2022) 3: 97--100,
\url{https://doi.org/10.1108/CC-05-2021-0018} {[}Paywall{]}

Bibliotheken streben mehr und mehr an, einen Bestand anzubieten, der
auch Diversität repräsentiert, aber gleichzeitig scheint es kompliziert
zu sein, dieses Ziel zu erreichen. In diesem kurzen Text denkt die
Autorin über Gründe im Bereich der wissenschaftlichen Literatur nach.
Sie identifiziert als Barrieren (a) das Angebot von Verlagen, welches
selber nicht divers ist und für das auch oft gar keine Daten existieren,
mit denen man überhaupt Entscheidungen im Bezug auf Diversität treffen
könnte, (b) impliziten Bias von Bibliothekar*innen, welche
Entscheidungen über die Bestandsauswahl treffen, (c) zu wenig
Arbeitszeit, die für solche Entscheidungen vorhanden ist und (d) zu
wenig freien, also noch nicht durch Lizenz- und andere Verträge
gebundenen Etat. Sie ruft trotzdem dazu auf, dass Bibliotheken ihr
Bestes versuchen sollen, um auch den Bestand diverser zu machen,
erinnert dabei an den Weg hin zu mehr Inklusion und Diversität, der
historisch seit den 1960ern schon zurückgelegt wurde und argumentiert,
dass darauf gedrungen werden muss, dass sich die Verlagsindustrie
ändert. (ks)

\pagebreak 

\hypertarget{monographien-und-buchkapitel}{%
\section{3. Monographien und
Buchkapitel}\label{monographien-und-buchkapitel}}

\hypertarget{vermischte-themen-1}{%
\subsection{3.1 Vermischte Themen}\label{vermischte-themen-1}}

Studding, Amy (edit.) (2022). \emph{Data-driven Decisions: A Practical
Toolkit for Library and Information Professionals.} London: facet
publishing, 2022 {[}gedruckt{]}

Das Erstaunlichste an diesem Buch ist für den Rezensenten, dass es
offenbar einen Bedarf für dieses gibt -- die Herausgeberin berichtet am
Anfang, dass sie hier in der ersten Hälfte des Buches ein Toolkit
ausbaut, welches sie schon mehrfach präsentiert und angewandt hat.
Anders als der Titel vermuten lassen könnte, geht es nicht um grosse
Datenmengen oder spezifische statistische Analysen, die im
Bibliotheksmanagement eingesetzt werden sollen. Vielmehr geht es darum,
recht einfache Auswertungen von vorhandenen Daten, beispielsweise aus
einem Bibliothekssystem, oder relativ leicht zu erhebende Daten, wie
regelmässige Zählungen der anwesenden Nutzer*innen in einer Bibliothek,
für bibliotheksspezifische Entscheidungen einzusetzen.

Das Buch führt -- in einem teilweise patronisierenden Ton -- durch die
Planung von Datensammlungen und Auswertungen sowie die möglichen
Entscheidungen, die mit den Analysen getroffen werden können. Andere
Formen der Nutzung von Daten, vor allem zum Marketing gegenüber
verschiedenen Stakeholdern, werden ebenso angesprochen. Das wird aber
kaum komplex. Die Daten werden in den gegebenen Beispielen immer nur in
Excel-Tabellen eingetragen und ausgewertet, nie zum Beispiel weiterer
statistischer Analysen unterzogen oder gar zur Bildung von statistischen
Modellen verwendet. Bei den Projekten, die vorgestellt werden, wird
immer wieder betont, dass man die Datenerhebung vorhergehend planen und
testen soll, sowie sich Gedanken machen muss, welche Daten man wirklich
benötigt, um bestimmte Fragen zu beantworten. Das ist alles nicht
falsch, aber man fragt sich immer wieder, ob das nicht auch schon so
allgemeine Praxis in Bibliotheken ist.

Im zweiten Teil stellen Autor*innen verschiedene Bibliotheken vor, in
denen zum Beispiel beim Bestandsmanagement auch kontinuierlich die
Auswertung von Daten integriert wird. Auch bei diesen Beispielen findet
sich nichts, was grundsätzlich falsch wäre, aber doch ebenso nichts, was
als grosse Neuheit angesehen werden kann. (ks)

\begin{center}\rule{0.5\linewidth}{0.5pt}\end{center}

Browndorf, Megan ; Pappas, Erin ; Ararys, Anna (edit.) (2021). \emph{The
Collector and the Collected: Decolonizing Area Studies Librarianship}.
Sacramento : Library Juice Press, 2021 {[}gedruckt{]}

Dieses Buch ist ein weiteres, das sich mit der Frage auseinandersetzt,
ob und wie das Bibliothekswesen dekolonisiert werden kann. Es geht,
zumindest vom Anspruch her, um einen spezifischen Bereich, nämlich
diejenigen Bibliotheken, welche die Forschung zu bestimmten
\enquote{Weltregionen} unterstützen. In der Einleitung gehen die
Herausgeber*innen darauf ein, dass ein solches Denken in Weltregionen
eine koloniale Geschichte hat, dass aber die betreffende Forschung
selber (und damit auch die Bibliotheken) vor allem im Kalten Krieg
strukturell anwuchs, als es insbesondere in den USA ein politisches und
militärisches Interesse an Wissen über bestimmte Regionen (zum Beispiel
\enquote{ganz Südostasien}) gab. Das Buch selber folgt diesen
Feststellungen aber nur teilweise.

Grundsätzlich sind die Beiträge thematisch und inhaltlich sehr
unterschiedlich. Sie lassen sich auch nicht alle dem eigentlichen
Themenfeld des Buches zuordnen. Beispielsweise gibt es eine Reflektion
darüber, ob und wie Dekolonisierung des Wissenschaftlichen
Bibliothekswesens im kanadischen Kontext überhaupt möglich ist. Es gibt
einen Artikel, welcher die Entwicklung der \enquote{Area Studies}
diskutiert, aber ohne gross auf das Bibliothekswesen einzugehen. Es
werden einzelne Bestände -- beispielsweise die türkisch-sprachige
Sammlung in der British Library -- und deren Entstehung vorgestellt.
Zudem findet sich eine Darstellung der Arbeit der Universitätsbibliothek
in Guam und deren Versuche, innerhalb der dort vorhandenen Strukturen
Literatur aus Mikronesien selber und in den Sprachen Mikronesiens zu
katalogisieren. Nicht zuletzt sind viele der Texte sehr lokal verankert.
Das mag teilweise Anspruch sein (es wird mehrfach betont, dass
Dekolonisierung auch hiesse, lokales Wissen und Formen der
Wissensproduktion zu präferieren), aber es macht einige der Texte schwer
verständlich. Beispielsweise wird mehrfach eine kurze Kritik der Praxis
der \enquote{Land Acknowledgements} angerissen -- einer in Australien,
Aotearoa Neuseeland und Kanada verbreiteten Praxis, die man erst einmal
kennen muss, um die Kritik zu verstehen.

Das Buch lässt den Eindruck zurück, dass auch diejenigen Kolleg*innen,
die sich aktiv mit der Frage der Dekolonisierung von Bibliotheken
auseinandersetzen, weiterhin vor allem auf der Suche danach sind, zu
bestimmen, was dies im Bibliothekswesen überhaupt genau bedeutet.
Auffällig ist auch, dass sich die meisten Beiträge auf Sammlungen
beziehen, die vor Jahrzehnten oder gar am Ende des 19. Jahrhunderts
begonnen wurden -- aber solche Sammlungen sind ja nur in einer gewissen
Anzahl von Bibliotheken zu finden. Und nicht zuletzt ist auch nicht
immer ganz klar, ob es um die Dekolonisierung der Bibliotheken oder der
Wissenschaft, der sie ja zuarbeiten, geht. (ks)

\begin{center}\rule{0.5\linewidth}{0.5pt}\end{center}

DeVoe, Lauren ; Duff, Sara (edit) (2022). \emph{Zines in Libraries:
Selecting, Purchasing, and Processing}. Chicago: ALA-Editions, 2022
{[}gedruckt{]}

Zines, also selbstpublizierte Hefte und ähnliche Publikationen, die
ausserhalb der handelsüblichen Publikationskanäle vor allem von
Einzelpersonen in Fancommunities, politischen Communities oder aus
persönlichem Interesse herausgegeben werden, sind im US-amerikanischen
Bibliothekswesen seit einigen Jahren ein recht oft besprochenes Medium.
Vor einigen Jahren ging es dabei zumeist darum, zu diskutieren, ob und
wenn ja, in welcher Form, diese Medien in den Bestand von Bibliotheken
gehörten. Es wurde argumentiert, dass sie einen besonderen Zugang zur
jeweiligen Community bieten würden und damit deshalb sowohl in
Wissenschaftliche Bibliotheken (vor allem als Sammlungsobjekt) als auch
Öffentlichen und Schulbibliotheken aufgenommen werden sollten.
Herausgestellt wurde oft, dass Zines von Personen veröffentlicht werden,
die keinen anderen Zugang zum Publikationsmarkt haben und somit die
Integration in den Bibliotheksbestand auch einen Beitrag dazu liefern
würde, deren Stimmen hörbar zu machen. Die damals, zu Beginn dieser
Diskussionen, publizierten Texte führten oft in das Thema ein, aber
lieferten wenig konkretes Material für die eigentliche Arbeit von
Bibliotheken.

Mittlerweile scheinen Zines zu einem im US-amerikanischen
Bibliothekswesen etablierten Medientyp geworden zu sein. Eine wachsende
Anzahl von Bibliotheken hat sie integriert und zum Teil eigene
Positionen als \enquote{Zine Librarian} geschaffen. Auf der jährlichen
Konferenz des US-amerikanischen Bibliotheksverbandes ALA gibt es
regelmässig eine \enquote{Zine Pavilion}
(\url{https://zinepavilion.tumblr.com/}) genannte Unterkonferenz nur zu
diesem Medientyp. Zudem sind \enquote{Zine Workshops}, in denen
gemeinsam Zines produziert werden, zu einem Veranstaltungsangebot
verschiedener Bibliotheken geworden.

Diese Professionalisierung ist auch \enquote{Zines in Libraries}
anzumerken. Viele Bücher, welche wie dieses im Verlag der ALA
erscheinen, führen nur kurz ins jeweilige Thema ein und liefern
anschliessend einige, eher oberflächliche Beispiele aus Bibliotheken.
\enquote{Zines in Libraries} hingegen geht tatsächlich näher an die
Bibliothekspraxis heran. Die meisten Texte stellen Berichte direkt aus
der Praxis dar, welche ganz konkrete Erfahrungen und Tipps vermitteln.
Hervorzuheben sind der Text von Lauren DeVoe über den konkreten
Erwerbungsprozess von Zines (die zumeist auf unkonventionellen Wegen
gekauft werden, beispielsweise auf \enquote{Zine Fest} genannten Messen)
und deren Integration in den normalen bibliothekarischen Geschäftsgang
sowie der Text von Jeremy Brett über Herausforderungen bei der
langfristigen Aufbewahrung von Zines (die beispielsweise zumeist auf
schlechtem Papier produziert werden). Zudem finden sich Beispiele für
die Integration von Zines in den Bestand Öffentlicher Bibliotheken und
von Schulbibliotheken. (ks)

\begin{center}\rule{0.5\linewidth}{0.5pt}\end{center}

Kempf, Charlotte (2020). \emph{Différences partagées.} Buchwissenschaft
\emph{et Histoire du livre en Allemagne et en France.} In: Sordet, Yann
(redact.) Histoire et civilisation du livre: Revue internationale XVI.
Genève: Librairie Droz, 2020: 99--111, URL:
\url{https://revues.droz.org/index.php/HCL/article/view/2020_16_99-111}
{[}Paywall{]}

Der Schwerpunkt der 2020er Ausgabe dieses buchwissenschaftlichen
Jahrbuchs ist der Einfluss des französischen Buchwissenschaftlers
Henri-Jean Martin auf die dortige Forschung zur Geschichte des Buches
und des Lesens. Im Artikel von Charlotte Kempf werden aber die
unterschiedlichen Entwicklungen der Buchwissenschaft (beziehungsweise
der histoire du livre) in Deutschland und Frankreich besprochen. Auf der
einen Seite ist dies eine konzise Zusammenfassung der Entwicklung der
deutschen Wissenschaft, zum Beispiel der Etablierung von Lehrstühlen an
Hochschulen (wobei die zuletzt angekündigte Schliessung beziehungsweise
der Umbau des Lehrstuhls an der Universität Leipzig selbstverständlich
noch nicht Thema des Beitrags ist) und der wichtigsten Debatten. Die
Entwicklung in Frankreich wird auch skizziert, allerdings ist dies,
schon weil sie in anderen Beiträgen des Jahrbuchs genauer besprochen
wird, recht kurz gehalten. Auf der anderen Seite zeigt der Beitrag, wie
unabhängig voneinander diese Entwicklungen stattfanden. Während in
Frankreich die Arbeiten Henri-Jean Martins, dem diese Ausgabe von
\emph{Historie et civilisation du livre} ja gewidmet ist, prägend für
die histoire du livre waren -- inklusive seiner Fragestellungen zum
Zusammenhang von Abbildung, Text, Layout und Typographie (\enquote{mise
en page}), der Materialität des Gesamtobjekts Buch (\enquote{mise en
livre}) und der Geschichte des Lesens als konkreter Aktivität (also
weniger, was gelesen wurde, sondern wie genau der Vorgang des Lesens
stattfand) sowie seinem Fokus auf die Nutzung von Archiven und
Archivmaterialien --, wurden diese in der deutschen Buchwissenschaft
kaum beachtet und gänzlich andere Schwerpunkte gesetzt. Nur punktuell
kam es überhaupt zu Kooperationen, aber kaum zur gegenseitigen
inhaltlichen Befruchtung.

Die Autorin ist für dieses Thema gut positioniert, da sie in einem
deutsch-französischen Doktorierendenkolleg ihre Promotion zum Thema
\enquote{Die deutschen Erstdrucker im französischsprachigen Raum bis
1500} erarbeitete (deren monographische Publikation unter anderem in
dieser Kolumne, Ausgabe \#9, vorgestellt wurde\footnote{Vergleiche
  Redaktion LIBREAS, "Das liest die LIBREAS, Nummer \#9 (Herbst / Winter
  2021)". LIBREAS. Library Ideas, 40 (2021).
  \url{https://libreas.eu/ausgabe40/dldl/}}) und damit als eine der
wenigen Personen direkten Einblick sowohl in die deutsche als auch die
französische Buchwissenschaft hat. (ks)

\hypertarget{bibliotheksgeschichte}{%
\subsection{3.2 Bibliotheksgeschichte}\label{bibliotheksgeschichte}}

Korotin, Ilse ; Stumpf-Fischer, Edith (Hrsg.) (2019).
\emph{Bibliothekarinnen in und aus Österreich: Der Weg zur beruflichen
Gleichstellung}. (biografiA: Neue Ergebnisse der
Frauenbiografieforschung, 25). Wien: Praesens Verlag, 2019
{[}gedruckt{]}

Das vorliegende Buch gehört zu einem Teilprojekt der Forschungen zur
Datenbank biografia
(\href{http://www.biografia.at/}{http://www.biografia.at}), in welcher
Biographien zu Frauen aus Österreich zusammengetragen werden. Die
Datenbank sowie dazugehörige Projekte und Publikationen, sind in der
Tradition der Frauengeschichtsschreibung verortet, die es sich zum Ziel
gemacht hat, das Leben und damit auch den Einfluss von Frauen sichtbar
zu machen. 2010 bis 2014 wurde in einem Teilprojekt die Erarbeitung von
Biographien von Bibliothekarinnen finanziert. Sie sind heute als Teil
der weiterhin betriebenen und ständig ergänzten Datenbank zu finden.

Der vorliegende Band ist im Rahmen des Teilprojektes entstanden. In der
zweiten Hälfte dieses fast 800 Seiten starken Buches (im A5-Format)
finden sich dann auch ausgewählte Biographien -- je nachdem, was über
die betreffenden Bibliothekarinnen zu finden war, mal sehr kurz, mal
mehrere Seiten lang. Wie die Auswahl dazu getroffen wurde, ist nicht
klar. Aber sie vermitteln zum einen, dass Frauen den Bibliotheksberuf
prägen und geprägt haben -- und zwar in verschiedensten Bereichen und
schon über eine lange Zeit.

In der ersten Hälfte des Buches sind Forschungen, die offenbar im Rahmen
des Teilprojektes durchgeführt und in zwei Workshops präsentiert wurden,
publiziert. Sie versuchen, den Bogen über die gesamte Zeit des
österreichischen Bibliothekswesens zu spannen. In den meisten Texten ist
dies explizit verbunden mit der Darstellung der Biographien einzelner
Bibliothekarinnen. In anderen kommen diese nur in kurzen Erwähnungen
vor, hier ist nicht immer klar, warum sie überhaupt in den Band
aufgenommen wurden. Der Band beginnt mit Klosterbibliotheken und endet
mit einer Geschichte der bibliothekarischen Ausbildung in Österreich im
20. und 21. Jahrhundert, bevor er mit einer Arbeit über das Bild der
Bibliothekarin in der österreichischen Literatur abschliesst. Es gibt
dabei einen Schwerpunkt auf dem frühen 20. Jahrhundert -- von der
Jahrhundertwende bis zum Ende des Nationalsozialismus. Gerade die
«Arbeiterbibliotheken» werden mehrfach thematisiert -- und die Stadt
Wien, die fast omnipräsent ist. Dies wird mit den Interessen der
Autor*innen und der Quellenlage zu tun haben. Aber gerade die Geschichte
der Bibliotheken ausserhalb Wiens, insbesondere ausserhalb von
Hochschulen, ist so (wieder einmal) unterbeleuchtet.

Die Anzahl der Beiträge ist zu gross, um sie getrennt zu besprechen. Wie
immer bei Sammelbänden sind sie inhaltlich und qualitativ
unterschiedlich, auch werden bestimmte Sachverhalte mehrfach
dargestellt. Aber im Ganzen ist das Buch einerseits eine erstaunlich
umfangreiche, tiefgehende Auseinandersetzung sowohl mit den Frauen im
österreichischen Bibliothekswesen als auch mit der Entwicklung dieses
Bibliothekswesens. Die inhaltlichen Schwerpunkte der Beiträge zeigen
zuerst, wie vielgestaltig das Bibliothekswesen ist und vor allem zu
Beginn des 20. Jahrhunderts war. Gleichzeitig zeigen sie, dass auch in
Österreich -- wie im restlichen DACH-Raum -- der Zugang von Frauen zum
Bibliothekswesen durch sexistische Strukturen bis in die 1970er Jahre
massiv beschränkt war und sie, wenn überhaupt, auf die Arbeit in
Öffentlichen Bibliotheken (und auch dort oft unter männlichen
Bibliotheksleitern) verwiesen wurden. Gleichwohl, dies zeigen die
Biographien, waren und sind Frauen aktiv dabei, das Bibliothekswesen zu
gestalten. Veränderungen wurden vor allem möglich durch das Engagement
von Frauen in wichtigen politischen Positionen, insbesondere als
Ministerinnen. (ks)

\begin{center}\rule{0.5\linewidth}{0.5pt}\end{center}

Marcum, Deanna ; Schonfeld, Roger C. (2021). \emph{Along Came Google: A
History of Library Digitization.} Princeton ; Oxford: Princeton
University Press, 2021 {[}gedruckt{]}

Die Massendigitalisierungen von Bibliotheksbeständen, welche von Google
ab 2004 im Zusammenhang mit dem damaligen Projekt \enquote{Google Books}
vorgenommen wurden, und ihre Auswirkungen auf Bibliotheken, sind fraglos
ein interessantes Thema für einen historischen Rückblick. (Auch wenn der
Rezensent beim Lesen etwas erschreckt darüber war, jetzt so alt zu sein,
um selber \enquote{Bibliotheksgeschichte} miterlebt zu haben.) Schon,
weil das Projekt sein Ziel eigentlich nicht erreichte. Die Autor*innen
sind (als hoch angesiedelte Vertreter*innen der Stiftung Ithaka S+R) gut
im US-amerikanischen Bibliothekswesen verankert und haben durch ihre
Kontakte offenbar auch Einblick in interne Dokumente von Bibliotheken
nehmen sowie Interviews mit relevanten Akteur*innen durchführen können.

Was dem Buch allerdings grundsätzlich fehlt, ist ein gewisser Abstand
zum Thema. Es ist offensichtlich, dass die Autor*innen ihre eigene
Meinung zum Projekt hatten und haben -- sie sehen es weiterhin als
notwendige Entwicklung hin zu einer \enquote{digitalen Weltbibliothek}
an, gleichzeitig wollen sie eine Veränderung im Bibliothekswesen hin zur
Kooperation und Zusammenarbeit mit Firmen wie Google. Alle, die ihrer
Meinung nach in diese Richtung arbeiteten, nennen sie durchweg
\enquote{Dreamers} und \enquote{Innovators}. Das restliche
US-amerikanische Bibliothekswesen wird implizit als rückwärtsgewandt und
strukturell unveränderlich gekennzeichnet. Zudem ist nicht ganz klar, an
wen sich das Buch richtet. Es ist im Stil einer
populärwissenschaftlichen Publikation geschrieben, mit recht wenigen
Quellen, dafür aber vielen Stellen, an denen beschrieben wird, was
einzelne Akteur*innen angeblich in einer spezifischen Situation fühlten
oder dachten. Grundsätzlich wird sich sehr auf einzelne Personen
konzentriert und auch unterstellt, dass Entscheidungen von Bibliotheken
gerade wegen persönlicher oder institutioneller Befindlichkeiten
getroffen wurden. Aber ob sich damit an eine breite Öffentlichkeit, an
das US-amerikanische Bibliothekswesen oder an spezifische Personen
gerichtet wird, ist nicht klar.

Ausserdem ist das Buch geprägt von der Position der Autor*innen selber,
ohne dass dies reflektiert wird: Der Fokus ist allein auf die USA
gerichtet -- zusammenfassend wird auch von \enquote{kanadischen
Bibliotheken} gesprochen, aber damit ist eigentlich nur eine, die der
University of Toronto, gemeint. Der Rest der Welt kommt nur in einem
Abschnitt vor, in welchem die Europeana als Ergebnis von nationalen
Ängsten insbesondere französischer Politiker*innen beschrieben wird.
Zudem fehlt erstaunlicherweise die Position von Google (oder deren
Konkurrenz, die ebenso erwähnt wird) fast vollständig. Warum entschied
sich die Firma für das Projekt? Wie liefen die dazugehörigen
Entscheidungsprozesse ab? All das wird eher aus offiziellen
Verlautbarungen übernommen. Und, nicht zuletzt, setzt das Buch
bestimmtes Wissen einfach voraus: Zwar wird zum Beispiel erklärt, wie
und wobei Bibliotheken in den USA seit Jahren kooperieren, was
impliziert, dass sich das Buch nicht an das Bibliothekswesen -- dass das
ja weiss -- richtet. Aber gleichzeitig wird vorausgesetzt, dass bekannt
ist, was im Vergleich steht, welcher am Ende des Projektes im Rahmen
eines Gerichtsverfahrens zwischen Google, Verleger*innen- und
Autor*innen-Verbänden sowie Bibliotheken geschlossen wurde. Dieser wird
zwar in einem ganzen Kapitel thematisiert, aber dessen Inhalt nicht
einmal zusammengefasst.

Kurzum: Das Buch geht chronologisch die Entwicklung des Projektes durch
und fokussiert sich dabei auf das US-amerikanische Bibliothekswesen.
Aber es ist keine historische Abhandlung, sondern eher ein persönliche
Erzählung und Bewertung der Autor*innen selber. Das hat seinen Wert,
aber es ist nur eine, sehr spezifische, Seite der Geschichte. (ks)

\begin{center}\rule{0.5\linewidth}{0.5pt}\end{center}

Kronenfeld, Michael R. ; Kronenfeld, Jennie Jacobs (2021). \emph{A
History of Medical Libraries and Medical Librarianship: From John Shaw
Billings to the Digital Era}. (Medical Library Association Books).
Lamham, Boulder, New York, London: Rowman \& Littlefield, 2021
{[}gedruckt{]}

Vorneweg: Das Buch ist leider nicht so spannend, wie es vom Thema her
hätte sein können. Das liegt an der Form, wie hier Geschichte erzählt
wird -- als eine Aneinanderreihung von Ereignissen, die alle praktisch
in direkter Linie zur heutigen Situation führen, ohne grosse
Abzweigungen. Und zudem in weiten Teilen als Geschichte
unternehmungsstarker Einzelpersonen, meist in führenden Positionen in
Verbänden, Spitälern, Bibliotheken oder Ministerien. Die Autor*innen
scheinen den letzten Punkt selber zu bemerken und erwähnen immer wieder,
dass die eigentliche Arbeit von zahllosen, heute praktisch namenlosen
Bibliothekar*innen gemacht wurde. Aber trotzdem durchbrechen sie den
Stil ihrer Erzählung von \enquote*{grossen} Einzelpersonen nicht.

Zudem, als Einschränkung, geht es nicht um das gesamte medizinische
Bibliothekswesen, sondern um das in den USA und Teilen Kanadas (vor
allem in Ontario). Die Medical Library Association, in deren Buchreihe
diese Publikation erschien, organisiert genau diese Bibliotheken. Es ist
auch eine Geschichte des Verbandes selber.

Die Geschichte hätte interessanter sein können, weil das medizinische
Bibliothekswesen bekanntlich eine sehr eigene Stellung gegenüber anderen
Bibliotheken hat, nicht nur in Nordamerika, sondern auch zum Beispiel im
DACH-Raum. Es hat ein eigenes Professionsverständnis, das sich explizit
an den Entwicklungen in Medizin und Spitälern orientiert. Und es baut
schneller und weitflächiger Angebote auf, die dann in anderen Teilen des
Bibliothekswesen entweder erst später oder auch nie etabliert werden. Es
ist bei vielen Entwicklungen, beispielsweise der Integration von
Informationstechnologie, oft schneller als andere Bibliotheken. Und es
hat auch immer eigene Strukturen, die zwar nicht gegen die anderen
Bibliotheksverbände stehen, aber doch nicht direkt integriert werden. So
ist die genannte Association nicht Teil der ALA, aber auch im DACH-Raum
ist die Arbeitsgemeinschaft für Medizinisches Bibliothekswesen nicht
Teil anderer Bibliotheksverbände. Es wäre interessant gewesen, in dieser
Geschichte zu erfahren, warum das so ist und auch, was das für
Konsequenzen auf die Entwicklung dieses Teils des Bibliothekswesens hat.

Was das Buch hingegen ist, ist eine chronologische Aufzählung von
Entwicklungen. Die Kontextualisierung, die gegeben wird, ist oft die
Entwicklung in Spitälern, aber auch hier ohne Diskussion. Es scheint so,
als hätte sich alles so entwickeln müssen, wie es jetzt ist.
Unterbrechungen auf dem Weg dorthin werden als kurzfristige Barrieren
dargestellt, die dann auch schnell überwunden werden. Was man lernt,
ist, dass das medizinische Bibliothekswesen sich als hoch professionell
versteht, vor allem auf ständige Weiterentwicklung ausgerichtet ist und
dass die betreffenden Bibliotheken vor allem untereinander eng
kooperieren. (ks)

\begin{center}\rule{0.5\linewidth}{0.5pt}\end{center}

Purdy, Jessica G. (2022). \emph{\enquote*{For the Edification of the
Common People}: Humphrey Chetham's Parish Libraries}. In: Oates,
Rosamund ; Purdy, Jessica G. (edit.). Communities of Print: Books and
Their Readers in Early Modern Europe. (Library of the Written Word, 99 ;
The Handpress World, 79). Leiden ; Boston: Brill, 2022: 79--96
{[}gedruckt{]}

Im September 1653 verstarb Humphrey Chetham, Händler aus Manchester. In
seinem Testament bestimmte er, dass ein Teil seines Erbes für die
Einrichtung von fünf Kirchgemeindebibliotheken genutzt werden sollte.
(Alle diese Gemeinden hatten etwas mit dem Leben Chethams zu tun.) Von
diesen fünf haben erstaunlicherweise zwei bis heute überlebt, wenn auch
an anderen Orten. Der Beitrag beschreibt sowohl, wie die Bibliotheken
zusammengesetzt wurden, als auch was über die Nutzung noch bekannt ist.

Die Bibliotheken waren grundsätzlich jeweils in einem Schrank
untergebracht, in welchem die betreffenden Bücher an Ketten gesichert
wurden. Die Schränke konnten aufgeklappt und das geöffnete Brett als
Buchablage zum Lesen genutzt werden. (Im Beitrag ist die eine
Bibliothek, welche genau in diesem Zustand überliefert ist, auch
abgebildet.) Gedacht waren die Bücher vor allem für die Gläubigen
selber, wobei Chetham Protestant war, aber innerhalb des Protestantismus
einen Kompromiss zwischen den verschiedenen Richtungen anstrebte.
Deswegen setzte er als Verwalter auch drei seiner Freunde ein, die alle
je einer anderen Denomination angehörten. Dies schlug sich im Bestand
und den Lesespuren, die noch vorhanden sind, nieder. Die Bibliotheken
bestanden praktisch nur aus geistlichen Werken, aber wenigen Polemiken
oder Auseinandersetzungen mit anderen Denominationen. Sie wurden vor
allem genutzt, um herauszufinden, wie man persönlich ein
\enquote{gottgefälliges Leben} führen konnte. Die Autorin erinnert
darin, dass im späten Mittelalter und der frühen Neuzeit im Allgemeinen
nicht einfach mit dem Ziel der Freizeitgestaltung gelesen wurde, sondern
immer mit einem Ziel -- das Lesen sollte Antworten hervorbringen. (ks)

\begin{center}\rule{0.5\linewidth}{0.5pt}\end{center}

Williams, Kelsey Jackson ; Stevenson, Jane ; Zachs, William (2022).
\emph{A History and Catalogue of the Lindsay Library, 1570-1792: The
Story of} \emph{\enquote*{some bonie litle bookes}}. (Library of the
Written World, 103; The Handpress World, 82). Leiden: Brill, 2022
{[}gedruckt{]}

Die \enquote{Lindsay Library}, um die es in diesem Buch geht, war die
Bibliothek einer adligen Familie aus Schottland. Grosse Teile davon
wurden 1792 von einem Mitglied der Familie auf einer Auktion verkauft.
Im Buch wird nun die Geschichte dieser Bibliothek rekonstruiert,
gleichzeitig die bibliographischen Angaben zu den über Tausend Büchern,
von denen bekannt ist, dass sie zu ihr gehörten, aus verschiedenen
Listen zusammengetragen und weitere Dokumente zur Bibliothek
präsentiert. Das Buch ist durchgehend und mit zahlreichen Bildern von
Büchern und deren Details illustriert.

Es steckt in diesem Band eine erstaunliche Fleissarbeit, die laut
Autor*innen über zehn Jahre in Anspruch nahm. Im Textteil wird nicht nur
die Suche und das Auffinden von Listen und des Katalogs der Auktion von
1792 ausführlich geschildert, sondern zusätzlich das Leben der Personen,
die zur Bibliothek beitrugen, im Einzelnen dargestellt und, wenn
möglich, ihre Buchkäufe rekonstruiert. Zudem wird der Bestand der
Bibliothek inhaltlich ausgewertet. Gerade bei den Biographien bewegt
sich das Buch in Richtung einer adligen Familiengeschichte (zu der dann
auch noch der aktuelle Earl of Crawford und Earl of Balcarres, das
jetzige Familienoberhaupt, mit einem persönlichen Vorwort beiträgt).
Teilweise, aber viel weniger, geht der Text auch darauf ein, was uns die
Buchkäufe und Biographien über den Buchhandel in Schottland und Europa
des 16. bis 18. Jahrhunderts erzählen.

Ganz am Ende des Textteils betonen die Autor*innen dann auch, dass mit
ihrem Werk dem Bild vom \enquote{intellektuell rückständigen Schottland}
der Zeit vor der Union mit England entgegengetreten werden soll. Das ist
hilfreich, weil es zumindest einen Hinweis gibt, warum jemand eine
solche intensive, kleinteilige Arbeit überhaupt unternimmt und warum
jemand eine Familiengeschichte (die, notabene, schon in einem anderen
Buch ausführlich dargestellt wurde) als Bibliotheksgeschichte vorlegt.
Das Buch scheint ein wenig von einem Verständnis von Wissenschaft als
möglichst vollständige Sammlung geprägt zu sein, welches eher ins 19.
Jahrhundert als ins 21. passen würde. Es ist beeindruckend, aber auch
ein wenig sehr aus der Zeit gefallen. (ks)

\begin{center}\rule{0.5\linewidth}{0.5pt}\end{center}

Duncan, Dennis (2021). \emph{Index, A History of the: A Bookish
Adventure}. Dublin : Allen Lane, 2021 {[}gedruckt{]}

Dieses Buch soll hier kurz erwähnt werden, um darauf hinzuweisen, dass
es für die Bibliotheksgeschichte selber leider nicht so viel enthält,
wie man wegen seines Titels vermuten könnte. Das heisst nicht, dass es
ein uninteressantes Buch wäre. Aber der Fokus liegt tatsächlich auf dem
Index, welcher sich (heute, aber früher nicht, wie man im Buch lernen
kann) am Ende wissenschaftlicher Monographien befindet. Bibliotheken und
Bibliothekskataloge kommen vor, aber nur als ein kleiner Teil dieser
Geschichte.

Der Autor forscht und unterrichtet (englische) Literaturwissenschaft am
University College London und dies prägt das Buch. Es ist auf der einen
Seite eine kontinuierliche Geschichte vom europäischen Mittelalter (mit
einigen Rückgriffen auf die ägyptische und europäische Antike) bis
heute, aber eine mit Fokus auf Quellen in Latein und Englisch sowie
solchen direkt aus England. Es ist auch ein literaturwissenschaftliches
Werk, welches die eigene Kontinuität immer wieder einmal verlässt, um
literarische Beispiele aus gänzlich anderen Zeiten anzuführen. Woran das
Buch erinnert, ist, dass der Index wie auch das Alphabet als
Ordnungssysteme selber Werkzeuge sind, die eine historische Entwicklung
hinter sich haben und deren Nutzung immer davon bestimmt war, wozu sie
überhaupt genutzt werden sollten. Sie sind das Ergebnis von
Entwicklungen, die immer weiter gehen. Dies gilt auch, wie der Autor
mehrfach betont, für die Nutzung des Alphabets als Ordnungsinstrument.
Für ihn als Literaturwissenschaftler eventuell selbstverständlich ist,
dass dies aber auch für das Lesen an sich gilt: Warum, von wem und wozu
gelesen wird, wandelt sich mit der Zeit und damit dann auch, wie gelesen
wird. Er erwähnt es, um den Zusammenhang zwischen dieser Entwicklung und
dem Index herzustellen.

Etwas enttäuschend ist, dass er am Ende -- entgegen dieser
Historisierung des Index als sind immer weiter entwickelndes Werkzeug --
auf automatisierte Indexe eingeht, nur um dann ohne grosse
Überzeugungskraft zu postulieren, dass Technik die Arbeit von
menschlichen Indexier*innen nie ersetzen können wird. (ks)

\begin{center}\rule{0.5\linewidth}{0.5pt}\end{center}

Federhofer, Marie-Theres ; Meyer, Sabine (Hg.). \emph{Mit dem Buch in
der Hand: Beiträge zur deutsch-skandinavischen Buch- und
Bibliotheksgeschichte / A Book in Hand: German-Scandinavian Book and
Library History}. (Berliner Beiträge zur Skandinavistik, 31) Berlin:
Nordeuropa-Institut der Humboldt-Universität zu Berlin, 2021
{[}gedruckt{]}

Der Titel des Sammelbandes verspricht ein grösseres Themenspektrum, als
die Beiträge dann selber liefern. Zum einen geht es vor allem um einen
spezifischen Zeitraum, das 17. bis 19. Jahrhundert, und in den Texten,
die sich mit Bibliotheken beschäftigen, immer um Bibliotheken der
skandinavischen Königsfamilien. Nur im letzten, der aber auch sonst
thematisch eine Ausnahme darstellt, geht es um pan-skandinavische
Vereinigungen im deutschsprachigen Raum im 19. und frühen 20.
Jahrhundert und deren Bibliotheken. An sich sind die Fragestellungen,
wie das bei einer Publikation aus der Skandinavistik zu erwarten ist,
nicht auf bibliotheks- und buchgeschichtliche Themen bezogen, sondern
auf solche aus der Skandinavistik selber. Zudem beschäftigen sich einige
Texte mit Manuskripten oder literaturgeschichtlichen Fragestellungen,
also nicht mit Bibliotheken an sich.

Das heisst nicht, dass für die Bibliotheksgeschichte nichts aus diesen
Beiträgen zu lernen wäre. Vor allem, dass der Adel Bibliotheken für
verschiedene Zwecke nutzte, darunter stark den der Repräsentation. Die
Königsfamilie, insbesondere die regierenden Monarch*innen, wollten und
mussten sich als gebildet und später auch als aufgeklärt darstellen. Sie
mussten zeigen, dass sie mit den relevanten Werken ihrer Zeit vertraut
waren. Und sie mussten dies jeweils mit grösseren Sammlungen als die
anderen Adligen im eigenen Land (welche in einem sich selbst
verstärkenden Kreis gleichzeitig selber immer grössere Sammlungen
anlegten, weil sie dem Königshof nacheiferten) tun.

Aber das war nicht die einzige Nutzungsform: Bibliotheken wurden auch
tatsächlich zur Bildung und Unterhaltung, sowie in einem Fall für
Interessen an Esoterik genutzt. Während die Beiträge oft auf den Inhalt
der verschiedenen Sammlungen selbst eingehen und zum Beispiel Kataloge
auswerten, ist eine spannende Erkenntnis, wie sich im 18. und 19.
Jahrhundert Staat und Königshof trennten -- also nicht mehr alles
Eigentum des Staates gleichzeitig als Eigentum der Herrschenden gesehen
und verwaltet wurde -- und sich dies auch auf die Bibliotheken
auswirkte. Sie wurden mehr und mehr voneinander getrennt, so dass Ende
des 19. Jahrhunderts die Königliche Bibliothek zur Nationalbibliothek
werden konnte, während das Königshaus weiter eigene Bibliotheken hatte.
Ausserdem interessant sind die Bibliothekare der Adelsbibliotheken, die
immer wieder erwähnt werden. Diese waren nie für diese Arbeit
ausgebildet, sondern immer Forschende oder Autoren, welche zusätzliche
Aufgaben in der Bibliothek erhielten. Zahlreiche von ihnen erstellten
dann unter anderem Kataloge und Systematiken, welche heute oft erst die
Forschungen ermöglichen, die in diesem Buch präsentiert werden. (ks)

%\begin{center}\rule{0.5\linewidth}{0.5pt}\end{center}
\pagebreak
Kaufmann, Thomas (2022). \emph{Die Druckmacher: Wie die Generation
Luther die erste Medienrevolution entfesselte}. München: C. H. Beck,
2022 {[}gedruckt{]}

Die offensichtliche Grundthese dieses Buches ist, dass die
Medienrevolution, die in Europa mit dem Buchdruck einherging, verglichen
werden kann mit der Medienrevolution durch Digitale
Kommunikationsmedien, welche wir aktuell durchlaufen. Aber beim Lesen
scheint dies eher ein nachträglich in den Text eingefügtes Thema zu
sein, welches besser als mögliche Fragestellung in Einleitung und Epilog
verblieben wäre. In der jetzigen Form kommt es vor allem in Form von
Anachronismen wie der ständigen Verwendung der Bezeichnung von
\enquote{Fake News} für Texte aus dem 16. und frühen 17. Jahrhundert
daher. Prägend ist auch der ständig bemühte Begriff \enquote{Print
Natives} als Analogon zu \enquote{Digital Natives}. Man hat oft den
Eindruck, dass es eher eine Idee des Verlages war, um den Titel für die
breite Öffentlichkeit interessanter zu machen. (Wobei eine Koinzidenz
darin besteht, dass der Autor selber als eine der Neuerungen von Luther
und anderen Autoren -- und ganz wenigen Autorinnen -- im 16. Jahrhundert
hervorhebt, dass diese sich immer mehr an eine ständig breiter werdende
Öffentlichkeit richteten und nicht nur an eine akademische; so wie
dieses Buch, das auch akademische Forschungen in einen eher für den
Massenmarkt geschriebenen Text übersetzt.)

Was das Buch eigentlich ist, wenn man von den eher gesuchten
Verbindungen zur heutigen Zeit absieht, ist eine Erzählung der
Verbindung von Buchdruck und Reformation in Europa, inklusive seiner
Auswirkungen auf Literaturformen, Nutzungsweisen von Büchern und
kulturellen sowie rechtlichen Entwicklungen. Das ist alles -- auch für
Personen wie den Rezensenten, die sich nicht besonders mit dieser Zeit
beschäftigen -- nicht neu, sondern hier noch einmal zusammenfassend
dargestellt. Dabei verliert sich der Autor streckenweise in der
Darstellung von Auseinandersetzungen zwischen den Reformatoren selber,
allerdings in einer Weise, welche diese teilweise vor allem als
persönliche Auseinandersetzungen erscheinen lässt. Wirklich theologische
Fragen, die ja Triebfedern der reformatorischen Bewegungen waren, werden
praktisch nicht angesprochen. Die Druckgeschichte erscheint dabei immer
als am Rand miterzählt. Nur im letzten der vier Kapitel wird explizit
auf die Veränderungen in der Nutzung von gedruckten Medien in der
zweiten Hälfte des 16. Jahrhunderts eingegangen. Das ist alles wenig
überzeugend: Irgendwie ist klar, dass die Reformation (und
Gegenreformation) ohne den Buchdruck nicht so zustande gekommen wäre,
wie es dann passierte, und auch, dass die Reformatoren sich des Drucks
und seiner Möglichkeiten bedienten. Aber wie genau der Zusammenhang ist,
wer hier zum Beispiel wen prägte, wird nicht klar. Ebenso wirft das
genannte letzte Kapitel auch Fragen auf. Wenn beispielsweise erwähnt
wird, dass sich durch den Druck von Büchern neue Formen der
Selbstbildung verbreiteten, aber im Buch selber vor allem auf die
Reformatoren geblickt wurde, deutet sich an, dass nebenher noch weitere
relevante gesellschaftliche Entwicklungen stattgefunden haben, die durch
den Fokus des Buches auf die Reformatoren verdeckt wurden.

Das Buch ist eher ein Refresher, inklusive vieler Abbildung von
Wiegendrucken, als ein neuer Beitrag zum Thema. (ks)

\begin{center}\rule{0.5\linewidth}{0.5pt}\end{center}

\pagebreak
Chapman, Wayne K. (2022). \emph{``Something that I read in a book: W.B.
Yeats's Annotations at the National Library of Ireland. Volume I:
Reading Notes}. Clemson: Clemson University Press, 2022 {[}gedruckt{]}

Chapman, Wayne K. (2022). \emph{``Something that I read in a book: W.B.
Yeats's Annotations at the National Library of Ireland. Volume II: Yeats
Writings}. Clemson: Clemson University Press, 2022 {[}gedruckt{]}

Roberts, Geoffrey (2022). \emph{Stalin's Library: A Dictator and his
Books}. New Haven ; London: Yale University Press, 2022 {[}gedruckt{]}

West, Mark I. (2022). \emph{Theodore Roosevelt and His Library at
Sagamore Hill}. Lanham ; Boulder ; New York ; London: Rowman \&
Littlefield, 2022 {[}gedruckt{]}

Die drei hier zusammengefasst vorgestellten Werke -- eines in zwei
Bänden -- beschäftigen sich alle mit den Büchern, welche von je einer
geschichtlich relevanten Person benutzt wurden. Alle sind im gleichen
Jahr und in der gleichen Sprache erschienen. Ansonsten sind sie aber
sehr unterschiedlich, sowohl vom eigentlichen Fokus als auch von der
Länge, dem Inhalt und wohl auch der gedachten Verwendungsweise. Der
Rezensent erhielt sie aber zufälligerweise alle am gleichen Tag per
Fernleihe und stellte wohl deshalb einen gewissen Zusammenhang zwischen
ihnen her. Was sie -- um das vorwegzugreifen -- zeigen, ist, wie
unterschiedlich dieses Versprechen, ein Buch über die Bücher vorzulegen,
welche eine \enquote{berühmte} Person im Laufe seines Lebens gelesen und
versammelt hat, eingelöst werden kann.

Das sowohl materiell (160 Seiten) als auch inhaltlich dünnste Werk
dieser drei stellt jenes über die Bücher Theodore Roosevelts und seiner
Familie dar, welche sich heute im \enquote{Sagamore Hill} -- dem Haus,
in welchem Roosevelt den grössten Teil seines Lebens ansässig war, und
das heute eine Gedenkstätte darstellt, die besucht werden kann --
befinden. Es liest sich eher wie eine unkritische Broschüre, die im
Museumsshop der Gedenkstätte verkauft wird, denn wie eine inhaltlich
tiefgehende Arbeit. Der Autor stellt die Lesebiographie Roosevelts dar,
aber in einer vollkommen unkritischen Form. Roosevelt hätte als junges
Kind schon gelesen und von dort an immer, auch während seiner
Ausbildung, seiner politischen Karriere, seiner Präsidentschaft und dann
weiter bis an sein Lebensende. Zudem hätte er das Lesen in seiner
Familie gefördert. Der Text besteht auch praktisch nur aus
Nacherzählungen biographischer Zeugnisse. Beendet wird das Buch sogar
mit einem kurzen Essay Roosevelts darüber, wie wichtig es sei, Bücher zu
lesen. Es ist eine Geschichte ohne Brüche und ohne, dass aus ihr etwas
zu lernen wäre. Den längsten Teil des Buches stellen aber gar nicht
diese Texte dar, sondern eine unkommentierte und auch nicht inhaltlich
erschlossene Liste der Bücher, welche heute in Sagamore Hill stehen.
(Das sind nicht alle, die einmal vorhanden waren, da seine letzte
Ehefrau nach dem Tod Roosevelts weiter dort lebte und der Bestand sich
deshalb im Laufe der Zeit veränderte. Zudem wurden nach ihrem Tod Bücher
von der weiteren Familie an sich genommen. Und selbstverständlich sind
es auch nicht alle, die Roosevelt zeitlebens gelesen hat.) Diese,
alphabetisch nach den Namen der Autor*innen geordnete Liste wurde nicht
einmal vom Autor des Buches selbst erstellt, sondern vom National Park
Service, welcher heute die Gedenkstätte unterhält.

Hingegen ist das zweibändige Werk zu W.B. Yeats -- bekannt gleichzeitig
als einer der Begründer der modernen irischen Literatur, aber auch
Politiker im dann unabhängigen Irland mit totalitären Tendenzen und
zumindest Interesse am Faschismus -- eindeutig das Ergebnis jahrelanger,
detailversessener Arbeit. Der Autor selber gibt an, an diesem zehn Jahre
lang tätig gewesen zu sein. Die Bibliothek Yeats befindet sich heute in
der National Library of Ireland in Dublin. Abgesehen von kurzen
Einleitungen, welche die folgenden Listen kontextualisieren und einen
Überblick zum Leben Yeats, mit einem Fokus auf seine literarischen
Werke, geben, bestehen beide Bände aus minutiösen Auflistungen aller
Kommentare, Korrekturen und Unterstreichungen, welche Yeats in diesen
Büchern hinterlassen hat. Diese Hinzufügungen sind in den Listen
allesamt transkribiert, teilweise auch weiter formal beschrieben,
beispielsweise ob eine Durchstreichung aus einer, zwei oder gar noch
mehr Linien besteht. Unterbrochen sind die Listen nur manchmal von
Abbildungen der Kommentare selber, wobei nicht klar wird, warum jeweils
diese für die Abbildung ausgesucht wurden. Im ersten Band finden sich
Anmerkungen aus Büchern anderer Autor*innen, die Yeast kommentiert hat,
im zweiten Band Anmerkungen in seinen eigenen Büchern oder solchen, die
er mit herausgegeben hat. Beide Bände vermitteln den Eindruck
pedantischer Arbeit und hinterlassen die Frage, für wen und wozu diese
Arbeit überhaupt geleistet wurde. Der Autor postuliert in der Einleitung
im ersten Band, dass es unwahrscheinlich sei, dass die National Library
je diese Werke vollständig digitalisieren würde, wie sie es mit denen
von James Joyce getan hat. Deshalb seien diese Listen Material für die
Yeats-Forschung. Abgesehen davon, dass das schwer einzusehen ist, warum
die National Library diese Digitalisierung nicht beizeiten in Angriff
nehmen sollte -- immerhin hat sie selber ähnliche Projekte durchgeführt
und Yeats gilt für die Literaturgeschichte, insbesondere die Irische,
nicht weniger relevant als Joyce --, ist dies wohl tatsächlich die eine,
sehr enge Zielgruppe dieses sehr spezifischen Werkes: Forschende, welche
es als Grundlage ihrer selber spezifischen Forschung benötigen.

Das dritte Buch über \enquote{Stalins Bibliothek} hingegen ist ein weit
zugänglicheres Werk. Der Titel ist allerdings etwas irreführend. Die
konkrete Bibliothek Stalins -- oder, wie im Buch zu lernen ist, die
verschiedenen Bibliotheken, die er im Kreml und in seinen beiden Dachas
unterhielt -- ist tatsächlich Thema und erhält auch ein eigenes Kapitel.
Aber ansonsten wird der Begriff Bibliothek hier ausgedehnt auf alle
Bücher, Broschüren, Artikel, Reden, die Stalin gelesen oder verfasst
hat, auf Publikationsprojekte, Redaktionsarbeit und Eingriffe Stalins in
die Publikationen anderer sowie seine Interventionen in den Literatur-
und Kulturbetrieb der Sowjetunion. Der Autor ist Historiker mit einem
Fokus auf die Geschichte der Sowjetunion und hat auch schon verschiedene
Biographien Stalins publiziert. Das vorliegende Buch ist eine weitere
dieser Stalin-Biographien, aber jetzt mit einem Fokus auf Bücher,
Bibliotheken und das Lesen. Allerdings ist der Inhalt sehr breit
aufgestellt und beinhaltet viele weitere Informationen, die teilweise
weniger mit diesem Fokus und mehr mit Stalins Leben selber zu tun haben.
Eine Grundprämisse des Autors ist, dass Stalin zwar ein Diktator war,
der den \enquote{grossen Terror} orchestriert hat, aber gleichzeitig
auch ein ernsthafter Intellektueller, der intellektuelle
Auseinandersetzungen mit Texten und Ideen als Basis von revolutionärer
Politik und Aktion angesehen hätte. Das führt nicht unbedingt dazu, dass
Stalin entschuldigt wird, aber doch drängt sich teilweise der Eindruck
auf, als würde der Autor Stalin in gewisser Weise bewundern. Zumindest
widerspricht er auch explizit Versuchen, Stalin psychologisch zu
verstehen. Eventuell ist dieses Bild in den anderen Werken des Autors
differenzierter, aber zumindest in diesem erscheint Stalin als hart
arbeitender, viel lesender und an vielen Themen ernsthaft interessierter
Intellektueller. Dies sei nicht überraschend, weil die russische
Sozialdemokratie, aus der Stalin stammte, von der Vorstellung geprägt
war, dass ihr Marxismus ein wissenschaftlicher sei, der auch eine
ständige wissenschaftliche Auseinandersetzung -- in Verbindung mit
politischer Praxis -- bedürfe. Das Lesen sei Teil der politische Praxis
gewesen und Stalin -- aber auch andere Bolschewist*innen -- hätten dies
ernst genommen.

Im Buch ist zum Beispiel zu lernen, welche Werke in Stalins Bibliotheken
standen, inklusive einer von ihm erstellten Systematik und auch den
Namen einiger der Bibliothekarinnen, die für ihn arbeiteten (und zum
Teil dem \enquote{grossen Terror} zum Opfer fielen). Wert legt der Autor
darauf, zu unterstreichen, dass Stalin auch viele Arbeiten von Opfern
seiner Politik gelesen hätte -- mehrfach stellt er heraus, dass Trotzki
von Stalin in seiner Bibliothek eine eigene Systematikstelle erhalten
hat -- sowie \enquote{bürgerliche}. Zudem wird die Lesebiographie des
jungen Stalin -- der zum Beispiel mehrfach gemassregelt wurde, weil er
in den Schulen, die er in Gori und Tiflis besuchte und die von der
orthodoxen Kirche betrieben wurden, Bücher aus Leihbibliotheken und
radikalen Buchläden lass -- ebenso dargestellt, wie Debatten darum,
welche Broschüren und Texte er tatsächlich verfasst hat. Aber es finden
sich auch Kapitel dazu, welche militärische Literatur Stalin während des
Zweiten Weltkrieges gelesen hat oder in welche intellektuellen Debatten
er eingegriffen hat. Einen grossen Fokus legt der Autor auch darauf
darzustellen, dass Stalin kontinuierlich gelesen und dabei sowohl
Anmerkungen in zahlreichen Büchern hinterlassen habe, als auch zahllose
Texte editiert hätte. Im längsten Kapitel seiner Biographie geht er
Anmerkungen in einigen dieser Bücher durch und kontextualisiert sie.
Allerdings merkt man in diesem Kapitel auch, dass der Autor einige
akademische Auseinandersetzungen austragen will. Mehrfach stellt er dar,
wie andere Forschende bestimmte Anmerkungen in Stalins Büchern
interpretiert haben, nur um dann zu sagen, dass die Anmerkungen gar
nicht von Stalin stammen, sondern beispielsweise von seiner Tochter
Svetlana oder \enquote{aus unbekannter Hand}.

Dargestellt ist auch, was mit den Büchern Stalins geschah. Nach seinem
Tod wurden diese zusammengehalten, da es einen Plan gab, in seiner
Moskauer Dacha ein Museum zu errichten. Mit dem politischen Wandel in
den folgenden Jahren wurde dieser Plan zunichte gemacht. Die Bücher
wurden an das damalige Institut für Marx-, Engels- und Lenin-Forschung
übereignet. Später wurden von diesem die Bücher, welche eindeutig zu
Stalins Bibliothek gehörten, weil sie mit einem Ex-Libris ausgestattet
waren oder aber Anmerkungen enthielten, zusammengehalten und in einem
Zettelkatalog erschlossen. Diese konnte der Autor -- allerdings
selbstverständlich vor dem 2022 von Russland begonnen Krieg gegen die
Ukraine -- für seine Forschung nutzen. Der Rest der Bibliothek wurde
verstreut, inklusive der gesamten Belletristik. Grundsätzlich scheint
diese Biographie von der kleinteiligen Arbeit eines Historikers geprägt,
der sich teilweise in Kleinigkeiten verliert, diese manchmal mehrfach
wiederholt und auch akademische Kämpfe ausfechten zu müssen glaubt. Es
ist nicht schwer zu lesen, aber eine durchgreifendere Redaktion hätte
ihm gut getan.

Was in dem Buch auch zu lernen ist -- und Bibliothekar*innen in ihrem
schlechten Bild von Stalin wieder bestätigen wird --, ist, dass Stalin
recht oft Bücher aus Moskauer Bibliotheken borgte, aber auch oft nicht
zurückgab und gleichzeitig in Büchern aus diesen Bibliotheken genauso
Anmerkungen hinterliess, wie in seinen eigenen. (ks)

\hypertarget{social-media}{%
\section{4. Social Media}\label{social-media}}

Grallert, Till: \emph{Die UB richtet einen Scholarly Makerspace ein}.
In: Future e-Research Support in the Humanities. Wissenschaftsblog zum
DFG-Projekt FuReSH II an der Universitätsbibliothek. 31.03. 2022.
\url{https://blogs.hu-berlin.de/furesh/2022/03/31/ankundigung-scholarly-makerspace/}

Die Universitätsbibliothek der Humboldt-Universität zu Berlin baut seit
dem Frühjahr 2022 im Rahmen eines DFG-Projektes einen
\enquote{prototypischen Scholarly Makerspace} auf. In seinem Beitrag
erläutert Till Grallert die Elemente und den Ansatz eines solchen
Angebots. Im Mittelpunkt steht die digitale Teilhabe, hier bezogen auf
\enquote{Lehrende und Forschende der Humboldt-Universität in allen
Phasen ihrer wissenschaftlichen Karrieren}. Im Scholarly Makerspace
haben sie die Möglichkeit, sich forschungspraktisch und zugleich
reflexiv mit digitaler geisteswissenschaftlicher Forschung
beziehungsweise Digital-Humanities-Ansätzen vor allem aus den
Perspektiven Tool Literacy und Datenkritik zu befassen. Der Ansatz der
Scholarly Makerspace setzt explizit auf Community-Effekte und nicht auf
klassische Schulungsansätze. Besonders die unmittelbare Einbettung von
Digital-Humanities-Forschung in die Bibliothek, also eine Art
Invertierung der Embedded Librarianship, ist der interessante
bibliothekswissenschaftliche Innovationspunkt. (bk)

\begin{center}\rule{0.5\linewidth}{0.5pt}\end{center}

Müller, Henrik (2022): \emph{Der MDPI-Verlag -- Wolf im Schafspelz?} In:
Laborjournal Blog, 13.06.2022,
\url{https://www.laborjournal.de/rubric/hintergrund/hg/hg_22_06_03.php}

Immer und immer wieder steht die Frage im Raum: Ist MDPI ein Raubverlag,
ein predatory publisher? Henrik Müller gibt in dem Beitrag im Blog des
Laborjournals einen nuancierten Überblick über die Diskussion. Müller
zeichnet die Entwicklung des Verlages, dem am stärksten wachsenden
Wissenschaftsverlag der letzten Jahre, nach und gibt kritischen Stimmen
Gehör, wobei sich zu allem auch entkräftende Erläuterungen aufzeigen
lassen. Im Fazit gibt es (vermutlich wenig überraschend) keine klare
Antwort -- aber Leser*innen haben ein besseres Verständnis für die
Vielfalt der Aspekte (quantitatives Wachstum, Bedeutung von Impact
Factor und auffällige Zitationsmuster, Arbeitsabläufe und (technische)
Verlagsplattform, Ablauf der wissenschaftlichen Qualitätssicherung und
Bearbeitungsgeschwindigkeit, finanzielle Transparenz), unter denen eine
solche Frage beleuchtet werden sollte. (mv)

\hypertarget{konferenzen-konferenzberichte}{%
\section{5. Konferenzen,
Konferenzberichte}\label{konferenzen-konferenzberichte}}

Lauer, Gerhard (2022): Vom Maxwell'schen Modell zum Science Tracking.
Entwicklungen des wissenschaftlichen Publizierens und seine Folgen.
Keynote bei der Veranstaltung \emph{Digitales Publizieren und die
Qualitätsfrage: Wege für Open Access in den Geisteswissenschaften},
30.--31.3.2022. Aufzeichnung (ca. 53 min):
\url{https://duepublico2.uni-due.de/receive/duepublico_mods_00076116},
Miro-Board zum Vortrag: \url{https://miro.com/app/board/uXjVOppsmsQ=/}

Gerald Lauer (Professor für Buchwissenschaft an der Johannes
Gutenberg-Universität Mainz) gibt in diesem Vortrag einen sehr breiten
und dichten Überblick über die Entwicklung des wissenschaftlichen
Publizierens: Von der Entstehung der ersten Zeitschriften über die
Entwicklung der wissenschaftlichen Verlagslandschaft (mit Schwerpunkt
auf dem Geschäftsmodell von Robert Maxwell, welches die Grundsteine
legte für den heutigen Oligopolmarkt) bis hin zur Etablierung der uns
heute bekannten Konzerne, die auch wissenschaftliche Publikationen
verlegen, aber inzwischen einen Schwerpunkt in Data Analytics haben.
Lauer thematisiert die Folgen von Publikationszwang,
Reputationsmechanismen ebenso wie von Kommerzialisierung und Big Data.
Absolute Sehempfehlung! (mv)

\hypertarget{populuxe4re-medien-zeitungen-radio-tv-etc.}{%
\section{6. Populäre Medien (Zeitungen, Radio, TV
etc.)}\label{populuxe4re-medien-zeitungen-radio-tv-etc.}}

N.N. (2022): \emph{The Guardian view on Oslo's Future Library: hope in
practice.} In: The Guardian / guardian.com, 21.06.2022.
\url{https://www.theguardian.com/commentisfree/2022/jun/21/the-guardian-view-on-oslos-future-library-hope-in-practice}
{[}Paywall{]}

Die Redaktion des GUARDIAN kommentiert aktuelle Entwicklungen des
norwegischen \emph{Future Library Project} (\emph{Framtidsbiblioteket}).
Das Projekt ist ein das Thema Bibliothek und kulturelle Überlieferung
umspielendes Kunstprojekt der schottischen Künstlerin Katie Paterson.
Für den Guardian stechen die Langfristperspektive und der
Nachhaltigkeitsanspruch des Projektes heraus. Die Grundidee des
Projektes ist, dass für den Zeitraum von 2014 bis 2114 jedes Jahr ein
Manuskript in einem neu gebauten Sonderbereich der \emph{Deichman
bibliotek} in Oslo in einem \enquote{Silent Room} platziert und
ausgestellt wird. Das Lesen der Manuskripte wird erst ab dem Jahr 2114
möglich sein. Für den Neubau nutzte man das Holz von Bäumen, die für
einen Hain, den Future Library Forest, gefällt wurden, auf dem seit 2014
begleitend zur Bibliothek 1000 Norwegische Fichten gezogen werden. Die
erfolgreiche Durchführung des 100-Jahre-Kunstwerks ist mittlerweile auch
vertraglich mit der Stadt Oslo abgesichert. Für den Guardian sind
folgende Aspekte \enquote{visionär}: Das Projekt ist auf Diversität und
nach außen gerichtet. Damit stelle es einen Gegenpunkt zur Diagnose
einer vermeintlichen gesellschaftlichen Perspektivverengung und
zunehmenden Ab- und Eingrenzung dar. Die \enquote{delayed gratification}
wird als Gegenmodell zur Ungeduld einer von Social Media dominierten
Kommunikation gesehen. Durch die Verbindung von Literatur und Kunst mit
Aspekten wie nachhaltiger Forstwirtschaft, öffentlicher Architektur und
zivilgesellschaftlichen Engagement setze sie auf eine begrüßenswerte
Interdisziplinarität. All das klingt einerseits hochsympathisch und
andererseits wie ein Strohhalm, an dem sich eine aktuelle
Kulturverzweiflung und Überforderung dankbar festhält. Genau genommen
ist das Projekt eine spezifische Variation der \enquote{Slow
culture}-Ansätze, die seit ein paar Jahrzehnten in unterschiedlichsten
Formen auftauchen. Die Redaktion schreibt von \enquote{hope in
practice}. Es gibt sicher auch keinen Grund, übermäßig kritisch auf die
Sache zu schauen. Aber zugleich auch wenig Anlass, eine sehr
privilegierte Umsetzung der alten Zeitkapselidee als singuläres
Hoffungszeichen über den grünen Setzling zu loben. (bk)

\begin{center}\rule{0.5\linewidth}{0.5pt}\end{center}

Stenning, Philip (2022): \emph{Academic prestige}. In: The Economist.
October 15th-21st 2022. Letters. S, 18,
\url{https://www.economist.com/letters/2022/10/13/letters-to-the-editor}
{[}Paywall{]}

In einem Leser*innenbrief zu einem Artikel über Bias beim Peer Review
benennt der*die Einsender*in ein aus eigener Sicht entscheidendes
Problem der wissenschaftlichen Qualitätssicherung. Der Erfahrung nach
sind weniger die Reviewenden selbst die Quelle einer möglichen
Voreingenommenheit, sondern die Herausgebenden, die eine spezifische und
offenbar bisweilen verkürzte Perspektive auf die zu ihrer Zeitschrift
thematisch passenden Inhalte haben. Da sie zugleich jeweils die
Reviewenden für die doppelt anonymisierte Begutachtung auswählen,
entscheiden sie sich oft für Reviewende, die ihre Perspektive eher
teilen. Als Steuerungsmaßnahme empfiehlt Stenning einerseits, diesen
Faktor bei der Wahl der Herausgebenden zu berücksichtigen und zweitens,
dass es ein \enquote{editorial advisory board} gibt, in dem gezielt eine
erweiterte Meinungsvielfalt zum Forschungsfeld vertreten ist. (bk)

%\begin{center}\rule{0.5\linewidth}{0.5pt}\end{center}

Lohmer, Andreas; Blöß, Louise (2022): \emph{Öffentliche Bibliotheken.
Nur sieben mit eigenem Beinamen}. In: katapult-mv.de.
\url{https://katapult-mv.de/artikel/nur-sieben-bibliotheken-mit-beinamen}
{[}Paywall{]}

Katapult-MV hat ausgezählt, wie viele der öffentlichen Bibliotheken in
Mecklenburg-Vorpommern einen Bei- oder Ehrennamen tragen. Es sind sieben
von insgesamt 82 Bibliotheken. Gewürdigt werden: Hans Fallada
(Greifswald und Feldberger Seenlandschaft), Johann Christoph Adelung
(Anklam), Maxim Gorki (Heringsdorf), Uwe Johnson (Güstrow), Käthe Miete
(Ahrenshoop), Ludwig Reinhard (Boizenburg). (bk)

\begin{center}\rule{0.5\linewidth}{0.5pt}\end{center}

Staretzek, Juliane (2022): \emph{Roboter holen Bibliothekspreis nach
Schkeuditz}. In: Leipziger Volkszeitung. 15./16.10.2022, S. 23.
{[}gedruckt{]}

Im Regionalteil auf der Seite \enquote{Rund um Leipzig} stellt Juliane
Staretzek die Stadtbibliothek Schkeuditz und ihre Leiterin, Stefanie
Maiwald, kurz vor. Anlass ist die Auszeichnung der Bibliothek mit dem
mit 10.000 Euro dotierten Sächsischen Bibliothekspreises. Das Porträt
liefert beispielhaft ein Stimmungsbild für aktuelle Entwicklungen in
öffentlichen Bibliotheken. Dabei sticht der Ansatz heraus, die
Bibliothek, mit dem Angebot kleiner Lego-Roboter gezielt als Ort auch
der natur- und technikwissenschaftlichen Bildung zu gestalten. (bk)

\begin{center}\rule{0.5\linewidth}{0.5pt}\end{center}

Wang, Vivian: \emph{In China, Living Not \enquote*{With Covid,} but With
\enquote*{Zero Covid}.} In: New York Times. Oct.~2, 2022, Section A,
Page 14.
\url{https://www.nytimes.com/2022/10/01/world/asia/china-covid-zero.html}

Im \enquote{China Dispatch} der New York Times berichtet Vivian Wang
über die Alltagswirkungen der Umsetzungen der Zero-Covid-Strategie in
China und erwähnt dabei unter anderem, dass es in der Guangzhou Library
Desinfektionsanlagen für Bibliotheksbücher gibt, die aussehen wie
\enquote{High-Tech-Kühlgeräte}. Die Information wird mit einer
Fotografie der \enquote{Book Sterilizing Machine} illustriert. Auf
diesem Bild ist ein Informationsschild erkennbar, das eine alle zwei
Stunden stattfindende Desinfektion der Geräte verkündet und die
Nutzenden der Bibliothek auffordert, mindestens einen Meter Abstand
voneinander zu halten. (bk)

\hypertarget{abschlussarbeiten}{%
\section{7. Abschlussarbeiten}\label{abschlussarbeiten}}

{[}Diesmal keine Beiträge{]}

\hypertarget{weitere-medien}{%
\section{8. Weitere Medien}\label{weitere-medien}}

rsp (2022): \emph{AGB bekommt Anbau. An der Blüchestraße entsteht bis
Ende des Jahres eine temporäre Bibliothekserweiterung}. In: KIEZ UND
KNEIPE. Mai 2022 (18. Jahrgang), S. 1, 3,
\url{https://archiv.kiezundkneipe.de/2022/2022-05.pdf}

Die Kreuzberger Nachbarschaftszeitung KIEZ UND KNEIPE informiert über
aktuelle Bauvorhaben bei der Amerika-Gedenkbibliothek. Während die
eigentliche Erweiterungsbebauung erst noch ausgeschrieben wird, entsteht
seit Mai 2022 ein so genannter \enquote{Tempobau}, also eine
eingeschossige Zwischenerweiterung um 800 Quadratmeter, die der sehr
stark genutzten Bibliothek mit neuen Arbeitsräumen und
Veranstaltungsbereichen eine räumliche Entlastung verschaffen soll. (bk)

\begin{center}\rule{0.5\linewidth}{0.5pt}\end{center}

pm (2022): \emph{Das BiboBike lädt zum Lesen ein. Fahrbare Leseinsel mit
Büchern, Hörbüchern -- und Hängematten}. In: KIEZ UND KNEIPE. Mai 2022
(18. Jahrgang), S. 14,
\url{https://archiv.kiezundkneipe.de/2022/2022-05.pdf}

Die Stadtbibliothek Friedrichshain-Kreuzberg verfügt über ein extra für
diesen Zweck angefertigtes Bibliotheksfahrrad (\enquote{BiboBike}), das
seit Ende April durch die Parkanlagen des Berliner Bezirks unterwegs
ist. Auf der den kurzen Artikel begleitenden Fotografie zeigt sich auch
die Bezirksbürgermeisterin Clara Herrmann sichtlich sehr erfreut über
das Angebot. (bk)

\begin{center}\rule{0.5\linewidth}{0.5pt}\end{center}

r. (2022): \emph{Über die Digitalisierung alter Thurgauer
Ansichtskarten}. In: Schaffhauser Nachrichten. Lokalteil
Stein/Diessenhofen. 10.06.2022, S. 20 {[}gedruckt{]}

Das Thurgauer Staatsarchiv in Frauenfeld (Schweiz) digitalisiert seit
2016 etwa 25.000 Ansichtskarten mit lokalen Bezug und stellt diese in
einer digitalen Sammlung unter
\url{https://archives-quickaccess.ch/search/statg/ansichtskarten}
bereit. Eine Herausforderung bei der Erschließung ist die Datierung, so
Martina Rohrbach, Leiterin der Abteilung Bestandserhaltung im
Staatsarchiv. Das Urheberrecht des Datenbestandes wird zudem sehr eng
und eher nachnutzungshinderlich interpretiert: \enquote{Nur Aufnahmen
von Privatpersonen, die seit mehr als 70 Jahren tot sind, seien nicht
mehr geschützt {[}...{]}.} Im Archiv finden sich je nach Objekt
unterschiedliche Angaben zum Urheberrechtsschutz. Hochauflösende
Digitalisate lassen sich aber jeweils gegen Gebühr beim Staatsarchiv
bestellen.(bk)

\begin{center}\rule{0.5\linewidth}{0.5pt}\end{center}

Günter de Bruyn: \emph{Zum Thema: Lesen}. In: Situation 66. 20 Jahre
Mitteldeutscher Verlag Halle (Saale) Verlag für neue deutsche Literatur
1966. Halle (Saale): Mitteldeutscher Verlag, 1966. S.139--141
{[}gedruckt{]}

Für einen Sonderband zum Verlagsjubiläum des Mitteldeutschen Verlags
veröffentlichte der Autor und Bibliothekswissenschaftler Günter de Bruyn
eine Reflexion über die Praxis des Lesens, in der er auch auf eines der
zentralen Bibliotheksklischees nicht nur dieser Zeit anspielt. Gemeint
ist die landläufige Meinung, Bibliothekare würden \enquote{immerfort
lesen}. Er weist darauf hin, dass dies nicht so ist, da Bibliothekare
(bei ihm nur Maskulinum) das Lesen als ihre eigentliche
\enquote{Hauptarbeit} nur außerhalb der Arbeitszeit erledigen können.
Zudem würde man schlecht bezahlt, wobei das wiederum vielleicht mit der
Vorstellung zusammenhängt, dass man als Bibliotheksmitarbeitende
\enquote{nichts zu tun {[}braucht{]}}. Interessant ist, dass de Bruyn
die Brücke zur Ausbildung schlägt. Bei der \enquote{Aufnahmeprüfung für
die Bibliothekarschule} mussten die Antragstellenden offenbar ihre
Berufswahl begründen. Betonte die Begründung den Aspekt der Zuneigung
zum Buch, \enquote{wurden die Gesichter so traurig, daß man aus Mitleid
für die prüfenden Damen die Sorge um die eigene Zukunft vergaß.}
Gewünscht war nämlich, die \enquote{Liebe zum Menschen} und das Streben,
die Menschen mittels Bücher zu besseren ihrer Art zu machen. (bk)

\begin{center}\rule{0.5\linewidth}{0.5pt}\end{center}

Ukeles, Raquel ; Finkelman, Yoel (2021). \emph{NLI USA Signature
Speakers Series. Curators Corner: Building the World\textquotesingle s
Greatest Judaica Collection.} National Library of Israel, \url{https://youtu.be/40DvGzP4DGM} (1h 12min)

In diesem Videovortrag präsentieren zwei Kurator*innen der National
Library of Israel deren Arbeit. Interessant ist dabei, dass sich die
Bibliothek nicht nur als Nationalbibliothek des Staates Israel versteht,
sondern auch als Bibliothek des Judentums im Allgemeinen. Die
historischen Gründe dafür werden kurz dargestellt, aber interessant für
Bibliotheken ist wohl vor allem, wie diese Arbeit, das Schrifttum der
jüdischen Diaspora zu sammeln, bewerkstelligt wird. Sicherlich ist ein
Vortrag zu kurz, um alle Fragen ausgiebig zu diskutieren --
beispielsweise, wer eigentlich alles zu dieser Diaspora gehört --, aber
es geht hier um Kooperationen mit Institutionen, Sammler*innen und
Spender*innen im globalen Rahmen und um Dilemmata, die bei dieser Arbeit
auftauchen. Gleichzeitig wird die Bedeutung der Digitalisierung --
sowohl als Möglichkeit, Medien zugänglich zu machen, als auch als
Herausforderung, digitale Medien zu sammeln -- angesprochen. Zu guter
Letzt wird vor allem in der Fragerunde betont, dass es die Aufgabe der
Bibliothek ist, Medien für die \enquote{gesamte Community} zur Verfügung
zu stellen und nicht nur für ausgewählte Forschende. Es wird explizit
zum Besuch des neuen Gebäudes der Bibliothek eingeladen, welches zum
Zeitpunkt des Vortrags 2022 eröffnen sollte, aber jetzt erst 2023
fertiggestellt sein wird. (ks)

\begin{center}\rule{0.5\linewidth}{0.5pt}\end{center}

Waldron, Angela (2022). \emph{Northeast Harbor Library and the
Farnsworth Present A Lecture: Women in American Book Cover Design.}
Farnsworth Art Museum, \url{https://youtu.be/Rvw6L_A0Ptk} (1h 10min)

In der im Titel genannten Ausstellung \enquote{Women in American Book
Cover Design} im Farnsworth Art Museum geht es um Buchcover, die ab den
1880er Jahren bis zum Beginn des Ersten Weltkrieges vor allem in Boston
(dem damaligen Zentrum der US-amerikanischen Buchindustrie) von einer
Anzahl von Frauen gestaltet wurden. Sie stammen aus der Sammlung der
Museumsbibliothek. Die Kuratorin stellt vor allem die Bücher selber vor
und gibt jeweils kurze biographische Skizzen zu den Gestalterinnen.
Gleichzeitig ordnet sie die Cover in ihre Zeit ein: Frauen ergriffen
damals mit dem Buchdesign eine der wenigen Möglichkeiten, selbstbestimmt
beruflich und künstlerisch tätig zu sein. Gleichzeitig wurde von den
Verlagen immer mehr Wert auf die Gestaltung ihrer Ausgaben gelegt, aber
noch nicht zur Massenproduktion übergegangen. Die Bücher und damit auch
die Cover, die in der Ausstellung gezeigt werden, waren noch Luxus, wenn
auch einer, der für immer mehr Menschen immer erreichbarer wurde.

Auch wenn die Kuratorin in der Präsentation auf Unterschiede zwischen
den Covern eingeht, ist doch sichtbar -- wie sie auch am Beginn kurz
erwähnt --, dass alle diese Designerinnen von den miteinander verwandten
Kunstbewegungen Arts and Crafts, Jugendstil und Art Nouveau geprägt
waren. Beispielsweise überwiegen Blumenmotive. (ks)

%autor

\end{document}