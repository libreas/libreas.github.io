\documentclass[a4paper,
fontsize=11pt,
%headings=small,
oneside,
numbers=noperiodatend,
parskip=half-,
bibliography=totoc,
final
]{scrartcl}

\usepackage{synttree}
\usepackage{graphicx}
\setkeys{Gin}{width=.4\textwidth} %default pics size

\graphicspath{{./plots/}}
\usepackage[ngerman]{babel}
\usepackage[T1]{fontenc}
%\usepackage{amsmath}
\usepackage[utf8x]{inputenc}
\usepackage [hyphens]{url}
\usepackage{booktabs} 
\usepackage[left=2.4cm,right=2.4cm,top=2.3cm,bottom=2cm,includeheadfoot]{geometry}
\usepackage{eurosym}
\usepackage{multirow}
\usepackage[ngerman]{varioref}
\setcapindent{1em}
\renewcommand{\labelitemi}{--}
\usepackage{paralist}
\usepackage{pdfpages}
\usepackage{lscape}
\usepackage{float}
\usepackage{acronym}
\usepackage{eurosym}
\usepackage[babel]{csquotes}
\usepackage{longtable,lscape}
\usepackage{mathpazo}
\usepackage[normalem]{ulem} %emphasize weiterhin kursiv
\usepackage[flushmargin,ragged]{footmisc} % left align footnote
\usepackage{ccicons} 
\setcapindent{0pt} % no indentation in captions

%%%% fancy LIBREAS URL color 
\usepackage{xcolor}
\definecolor{libreas}{RGB}{112,0,0}

\usepackage{listings}

\urlstyle{same}  % don't use monospace font for urls

\usepackage[fleqn]{amsmath}

%adjust fontsize for part

\usepackage{sectsty}
\partfont{\large}

%Das BibTeX-Zeichen mit \BibTeX setzen:
\def\symbol#1{\char #1\relax}
\def\bsl{{\tt\symbol{'134}}}
\def\BibTeX{{\rm B\kern-.05em{\sc i\kern-.025em b}\kern-.08em
    T\kern-.1667em\lower.7ex\hbox{E}\kern-.125emX}}

\usepackage{fancyhdr}
\fancyhf{}
\pagestyle{fancyplain}
\fancyhead[R]{\thepage}

% make sure bookmarks are created eventough sections are not numbered!
% uncommend if sections are numbered (bookmarks created by default)
\makeatletter
\renewcommand\@seccntformat[1]{}
\makeatother


\usepackage{hyperxmp}
\usepackage[colorlinks, linkcolor=black,citecolor=black, urlcolor=libreas,
breaklinks= true,bookmarks=true,bookmarksopen=true]{hyperref}
\usepackage{breakurl}

%meta
%meta

\fancyhead[L]{I. Roschmann-Steltenkamp\\ %author
LIBREAS. Library Ideas, 35 (2019). % journal, issue, volume.
\href{http://nbn-resolving.de/}
{}} % urn 
% recommended use
%\href{http://nbn-resolving.de/}{\color{black}{urn:nbn:de...}}
\fancyhead[R]{\thepage} %page number
\fancyfoot[L] {\ccLogo \ccAttribution\ \href{https://creativecommons.org/licenses/by/4.0/}{\color{black}Creative Commons BY 4.0}}  %licence
\fancyfoot[R] {ISSN: 1860-7950}

\title{\LARGE{Neutralität in Bibliotheken -- ein Werkstattbericht}}% title
\author{Irmela Roschmann-Steltenkamp} % author

\setcounter{page}{1}

\hypersetup{%
      pdftitle={Neutralität in Bibliotheken -- ein Werkstattbericht},
      pdfauthor={Irmela Roschmann-Steltenkamp},
      pdfcopyright={CC BY 4.0 International},
      pdfsubject={LIBREAS. Library Ideas, 35 (2019).},
      pdfkeywords={Neue Rechte, Bibliothekspraxis, Berlin},
      pdflicenseurl={https://creativecommons.org/licenses/by/4.0/},
      pdfcontacturl={http://libreas.eu},
      baseurl={http://libreas.eu},
      pdflang={de},
      pdfmetalang={de}
     }



\date{}
\begin{document}

\maketitle
\thispagestyle{fancyplain} 

%abstracts
\begin{abstract}
\noindent \textbf{Kurzfassung}: Der Beitrag ist ein Werkstattbericht der
Bibliothek des Zentrums für Antisemitismusforschung, deren regulärer
Bestand zum Teil aus rechter Literatur besteht und der Umgang damit ein
tagtäglicher ist. Nach einer kurzen Einführung entwickelt der Aufsatz
anhand der Schritte Erwerbung, Katalogisierung sowie Standort --
Zugänglichkeit -- Nutzung den Arbeitsalltag mit rechter Literatur.

\begin{center}\rule{0.5\linewidth}{\linethickness}\end{center}

\noindent \textbf{Abstract}: The article is a workshop report of the library of
the Center of Research on Antisemitism. The regular collection consists
partly of right-wing literature. After a short introduction, the essay
depicts the use of this literature on a daily basis along the steps
``acquisition'', ``cataloguing'' and ``location-accessibility''.
\end{abstract}

%body
Einen Artikel über Neutralität und den Umgang mit rechter Literatur in
Bibliotheken zu schreiben, stellt mich vor einige Herausforderungen: Ich
bin seit Oktober 2016 in der Bibliothek des Zentrums für
Antisemitismusforschung (ZfA) an der TU Berlin tätig. Davor habe ich 23
Jahre lang die Bibliothek der Stiftung Topographie des Terrors geleitet.
Im ZfA habe ich täglich mit rechter Literatur zu tun: Ich bestelle Titel
aus rechten Verlagen, abonniere Zeitschriften aus dem rechten und zum
Teil auch rechtsextremen Spektrum und muss mich viel mit Literatur von
verurteilten Holocaustleugnern auseinandersetzen. Die Forschungsfelder
des ZfA benötigen diese Literatur; die Studierenden unseres
Studienganges \enquote{Interdisziplinäre Antisemitismusforschung}
brauchen die Medien für ihre Referate und Hausarbeiten. Was kann ich
also aus meiner Perspektive, die fast schon von einem Alltagsumgang mit
rechten Medien gekennzeichnet ist, zu dem Thema dieser LIBREAS-Ausgabe
schreiben? Hinweise zu geben, wie andere Bibliotheken, für die rechte
Literatur die Ausnahme ist, mit entsprechender Literatur umzugehen
haben, maße ich mir nicht an. So erscheint mir ein Werkstattbericht
meiner täglichen Arbeit im ZfA und aus meiner Erfahrung der langjährigen
Tätigkeit in der Bibliothek der Stiftung Topographie des Terrors weitaus
sinnvoller.

Um einigermaßen angemessen über rechte Literatur und Neutralität
schreiben zu können, möchte ich zuerst beide Begriffe definieren:

Unter \enquote{rechter} Literatur verstehe ich in meinem Beitrag Medien,
die aus dem rechtsradikalen, rechtsextremen sowie dem
rechtsintellektuellen Spektrum stammen. Die Zuordnung mache ich an
bestimmten Verlagen, Autor*innen und Themen fest. Beispielhafte Verlage
sind hier etwa der Druffel-Verlag, Grabert (heute Hohenrain), Nation
Europa, Antaios, Junge Freiheit, Kopp, Deutsche Stimme, Castle Hill
Publishers. Bestimmte Autoren sind beispielsweise Germar Rudolf, Horst
Mahler, Götz Kubitschek oder Thilo Sarrazin. Themen, die das rechte
Spektrum abbilden, sind unter anderem Antisemitismus, NS-Verherrlichung
oder -Leugnung, Verschwörungstheorien und Hetze gegen Flüchtlinge. Wie
überall gibt es aber selbstverständlich auch bei der Abgrenzung rechter
von anderer Literatur Grauzonen und unscharfe Bereiche: Sind etwa Akif
Pirinçis Katzenkrimis rechts, weil er mittlerweile eindeutig rechte
Positionen bezieht?\footnote{Siehe hierzu auch die Diskussion im
  BuB-Heft 04/2018.} Ist die Biene Maja der antisemitischen Literatur
zuzurechnen, weil ihr Verfasser Waldemar Bonsels ein bekennender
Antisemit war? Hier muss meines Erachtens genau hingesehen und
diskutiert werden. Eine einfache, einheitliche Lösung gibt es nicht.

Was ist Neutralität? Wikipedia leitet vom Eintrag \enquote{Neutral} zum
Eintrag \enquote{Objektivität} weiter und sagt dazu:
\enquote{Objektivität {[}\ldots{}{]} bezeichnet die Unabhängigkeit der
Beurteilung oder Beschreibung einer Sache, eines Ereignisses oder eines
Sachverhalts vom Beobachter beziehungsweise vom Subjekt. Die Möglichkeit
eines neutralen Standpunktes, der absolute Objektivität ermöglicht, wird
verneint. {[}\ldots{}{]} Da man davon ausgeht, dass jede Sichtweise
subjektiv ist, werden wissenschaftlich verwertbare Ergebnisse an
bestimmten, anerkannten Methoden und Standards des Forschens gemessen.}
\footnote{\enquote{Neutral} siehe
  \url{https://de.wikipedia.org/wiki/Neutral} (letzter Zugriff
  20.2.2019); \enquote{Objektivität (als neutraler und unabhängiger
  Standpunkt)} siehe
  \url{https://de.wikipedia.org/wiki/Objektivit\%C3\%A4t} (letzter
  Zugriff 20.2.2019).} Von staatlich finanzierten Öffentlichen
Bibliotheken wird Neutralität und Objektivität erwartet. Ihre Aufgabe
ist es, ihren Leser*innen Literatur des gesamten Spektrums zur Verfügung
zu stellen und somit Informationsfreiheit und das Recht auf freie
Meinungsäußerung zu fördern. Privat finanzierte Bibliotheken sind den
Inhalten ihrer Geldgeber verpflichtet und gehalten, in deren Sinne
Literatur zu sammeln und bereitzustellen. Neutralität steht hierbei
nicht im Vordergrund. Wissenschaftliche (Spezial-)Bibliotheken wiederum
sind der wissenschaftlichen Neutralität verpflichtet und der
thematischen Ausrichtung ihres Instituts. Das Zentrum für
Antisemitismusforschung sammelt dementsprechend möglichst umfassend
Literatur zum Thema Antisemitismus und verwandten Bereichen wie
Rassismus, Rechtsextremismus/-radikalismus, Verschwörungstheorien,
Nationalsozialismus und Holocaust. Selbstverständlich ist das ZfA der
wissenschaftlichen Objektivität mit den genannten Standards und Methoden
verpflichtet, es kann jedoch selbst nicht neutral sein. Basis der
Forschung des ZfA ist die Darstellung des Antisemitismus in all seinen
Facetten und Ausprägungen mit dem Ziel, ihn zu bekämpfen. Die Arbeit des
ZfA soll in die politische Bildung einfließen, die Demokratie stärken
und rechte Einstellungen verringern oder im besten Falle verhindern.
Eine Ausrichtung des ZfA nach den Vorstellungen der AfD wäre demzufolge
undenkbar. Ich verfasse also einen Beitrag über Neutralität in
Bibliotheken aus Sicht einer Bibliothek, die eben gerade nicht neutral
sein soll.

Im Folgenden richte ich meinen Werkstattbericht am Weg der Bücher
innerhalb der ZfA-Bibliothek aus: Erwerbung, Katalogisierung, Standorte
und Zugänglichkeit, Nutzung. Am Ende schließe ich meinen Beitrag mit
persönlichen Eindrücken.

\hypertarget{erwerbung}{%
\section{Erwerbung}\label{erwerbung}}

Wie eingangs erwähnt, ist rechte Literatur im Bestand der ZfA-Bibliothek
selbstverständlich, ihr Anteil macht geschätzte 15--20 \% des Bestands
aus. Verschiedene Zeitschriften/Zeitungen sind abonniert, darüber hinaus
werden regelmäßig Bücher erworben. Mit den abonnierten Zeitschriften
soll das aktuelle rechte Spektrum möglichst umfassend abgedeckt sein.
Darüber hinaus verfügt das ZfA über einen umfangreichen Bestand alter
rechter Periodika, die mittlerweile nicht mehr erscheinen. Rechte Bücher
erwerbe ich entweder nach Hinweisen der Mitarbeiter*innen und
Studierenden des Instituts oder aufgrund eigener Recherchen. Aktuelle
Literatur wird vorgehalten, ebenso Literatur von Holocaustleugnern aus
den 1980er und 1990er Jahren sowie Literatur der \enquote{alten Rechten}
direkt nach 1945. Hinzu kommt selbstverständlich Sekundärliteratur zu
rechten Positionen, Politiker*innen, Autor*innen et cetera.\\
Diese rechten Periodika und Bücher finden ihren Weg ins Zentrum für
Antisemitismusforschung auf unterschiedliche Weise: Die meisten
Periodika bestelle ich über einen Zwischenhändler, damit das ZfA nicht
als Abonnent verzeichnet ist. Einige Zeitschriftenverlage möchten jedoch
genau wissen, wer der Besteller ist und verweigern deshalb die
Auslieferung über eine zwischengeschaltete Instanz. Diese
Zeitungen/Zeitschriften sind mit der offiziellen Adresse des ZfA
abonniert. Bücher bestelle ich wie auch sonst über unsere Buchhandlung.
Durch unsere Bestellungen ist die Buchhandlung bei einigen rechten
Verlagen schon so bekannt, dass sie Werbung und Broschüren von den
einschlägigen Verlagen erhält. Diese kostenlosen Broschüren (zum
Beispiel Rechtfertigungsschriften von Horst Mahler) gibt die
Buchhandlung an uns weiter -- graue Literatur, die unverzichtbar für uns
ist.

An mich wird sehr oft die Frage gerichtet, ob es zu rechtfertigen ist,
mit unseren Bestellungen die rechten Verlage finanziell zu
(unter-)stützen und ob wir damit nicht auch zu deren Überleben
beitragen. Meine Antwort ist, dass der Vorwurf einerseits natürlich
gerechtfertigt ist, unsere Bibliothek aber andererseits zum Ziel hat,
eben diese rechte Literatur zu sammeln und den Forschenden am Institut
für die Aufklärung über Antisemitismus zur Verfügung zu stellen. Andere
Erwerbungswege als Kauf sind nicht realistisch. Aber natürlich finde ich
es nicht angenehm, am Ende des Jahres zum Beispiel in der
\enquote{Deutsche{[}n{]} Stimme}\footnote{Die \enquote{Deutsche Stimme}
  ist das Organ der NPD.} einen Brief an die Abonnent*innen zu finden,
in dem sich die Herausgeber für die zahlreiche und effektive
Unterstützung der Zeitung bedanken. Ein Zwiespalt, mit dem ich leben
muss, wenn ich in der Bibliothek des ZfA arbeite.

\hypertarget{katalogisierung}{%
\section{Katalogisierung}\label{katalogisierung}}

Alle Titel, die wir erwerben, werden im Katalog der
TU-Universitätsbibliothek katalogisiert und sind somit weltweit
verzeichnet. Alle Zeitschriften sind in der ZDB nachgewiesen. Es gibt
keine \enquote{versteckten} Bestände. Nicht selten hat die
ZfA-Bibliothek Alleinbesitz oder ist eine von zwei oder drei
Bibliotheken, die den Titel besitzen. Das zeigt, wie wichtig unsere
Erwerbungspraxis für diejenigen ist, die zu den Themen Antisemitismus,
Rassismus und ähnlichem arbeiten.

\hypertarget{standorte-zuguxe4nglichkeit-nutzung}{%
\section{Standorte -- Zugänglichkeit --
Nutzung}\label{standorte-zuguxe4nglichkeit-nutzung}}

Die Bibliothek des ZfA ist eine Forschungs- und Wissenschaftliche
Spezialbibliothek, die (fast) ausnahmslos Nutzer*innen über 18 Jahre
hat. In der Bibliothek forschen die Mitarbeiter*innen des Instituts, die
Studierenden des institutseigenen Masterstudiengangs sowie an den Themen
der Bibliothek Interessierte, die nicht der TU Berlin angehören
(müssen). Der Großteil der Bestände (circa 35.000 Bände) steht im
Freihandbereich und kann ohne Beschränkungen genutzt werden. Hier stehen
auch die rechten Bücher und Zeitschriften. Aus Platz- und
konservatorischen Gründen stehen die Medien mit Erscheinungsjahr bis
1945 (circa 5.000 Bände) in einem gesonderten Raum, die
Zeitschriftenhefte vor 2012 im Keller in Kompaktregalen. Die Bibliothek
ist eine Präsenzbibliothek, die allen, die dort lesen und arbeiten
wollen, offensteht. Hinter dieser Entscheidung, alle Bücher in Freihand
zu stellen, die in Absprache mit den Mitarbeiter*innen des ZfA getroffen
wurde, stand der Wunsch der Wissenschaftler*innen, auch die rechte
Literatur ohne Hürden nutzen und direkt am Regal auf sie zugreifen zu
können. Dieser Wunsch ist verständlich und nachvollziehbar, birgt aber
rechtliche Gefahrenpotentiale. Bibliotheken sind verpflichtet,
volksverhetzende, rassistische und andere strafrechtlich bedingte
Literatur nicht an unter 18-jährige Nutzer*innen zugänglich zu machen,
sie also gesondert aufzubewahren.\footnote{Siehe hierzu die grundlegende
  Bachelorarbeit von Rebecca Behnk: Behnk, Rebecca:
  Nationalsozialistische Schriften -- freier Zugang oder Barrieren? :
  rechtliche Vorgaben und praktische Umsetzung am Beispiel von Berliner
  Spezialbibliotheken. Berlin: Simon Verlag für Bibliothekswissen, 2013.}
Holocaustleugnende Literatur fällt unter diese Rubrik. Eigentlich
müssten wir sie aus dem Freihandbestand entfernen, tun dies aber nicht
mit dem Argument, dass wir eine wissenschaftliche Forschungsbibliothek
sind, die nur von über 18-jährigen Personen genutzt wird. Da in meinen
bisherigen zweieinhalb Jahren in der Bibliothek des ZfA maximal zwei
Personen unter 18 Jahren (mit ihren Eltern) die Bibliothek genutzt
haben, werde ich den gegenwärtigen Zugang zur strafrechtlich relevanten
Literatur nicht ändern.

Ganz anders ist der Zugang zu rechter Literatur in der Bibliothek der
Topographie des Terrors gestaltet -- hier finden sich allerdings auch
andere Nutzer*innengruppen, die eine andere Zugänglichkeit dringend
notwendig machen. Die Zielsetzung der Topographie ist keine
wissenschaftliche, sondern eine bildungspolitische, die sich an ein
anderes Publikum als das im ZfA richtet. In der Topographie des Terrors
wird die rechte Literatur\footnote{Auch in der Topographie wird über
  rechte Literatur anhand von Verlagen, Autoren und Inhalten
  entschieden.} in einer eigenen Systematikgruppe zusammengefasst und im
Magazin der Bibliothek aufgestellt. Alle diese Titel sind katalogisiert
und somit suchbar. Vor der Ausgabe dieser Bestände müssen die
Nutzer*innen eine Versicherung unterschreiben, dass sie diese für
wissenschaftliche oder persönliche Forschungen (zum Beispiel
Familienforschung) benötigen. Auch in der Topographie des Terrors wurde
über dieses Verfahren in einer Runde aller Mitarbeitenden entschieden.
Ausschlaggebend für die Entscheidung waren zwei wichtige Faktoren:
Nutzergruppen der Topographie-Bibliothek sind zu einem großen Teil
Schüler*innen, die an Seminaren der Stiftung teilnehmen und im Rahmen
dieser Seminare eigenständig in der Bibliothek recherchieren und
arbeiten. Da der Großteil dieser Schüler*innen unter 18 Jahre alt ist,
MUSS der strafrechtsrelevante Teil des Bestandes außerhalb des
Freihandbereichs stehen und darf nicht offen zugänglich sein. Darüber
hinaus kommen sehr viele Besucher*innen der Ausstellung mit persönlichen
Fragen in die Bibliothek. Oft sind es Familienangehörige ehemaliger
Opfer der Nationalsozialisten, die nach Todesort und -zeitpunkt ihrer
Familienmitglieder forschen. Diese Fragen sind überwiegend hoch
emotional und müssen mit viel Feingefühl beantwortet werden. Deshalb
soll gerade dieser Nutzergruppe nicht zugemutet werden, rechte Literatur
ganz normal im Freihandbestand zu finden.

Die Bibliothek des ZfA ist eine Bibliothek der TU Berlin und damit der
Universitätsbibliothek der TU zugehörig. Alle ZfA-Bestände sind über
ALMA im Primo-Katalog der UB verzeichnet und somit offen für
Fernleih-Anfragen. Auch rechte Titel aus der ZfA-Bibliothek werden über
Fernleihe angefragt. Ich entscheide über jede Anfrage individuell, je
nach Zustand und Umfang des bestellten Titels. Hat das ZfA Alleinbesitz
und ist der Titel darüber hinaus nicht mehr zu erwerben, lehne ich eine
Fernleihe meist ab, da mir die Gefahr des Verlustes durch den Postweg zu
groß erscheint. Nach inhaltlichen Kriterien verweigere ich eine
Fernleihanfrage jedoch niemals. Ich kenne nur die anfragende Bibliothek,
nicht jedoch die letztendlichen Nutzer*innen -- aber nach meinem
beruflichen Verständnis steht es mir nicht zu, zu entscheiden, wer
welches Buch lesen darf und welches nicht. Dass die Anfrage von einer
Bibliothek kommt, reicht mir als Aussage einer wissenschaftlichen
Nutzung des angefragten Titels aus. Nutzer*innen mit einer rechten
politischen Einstellung werden mit Sicherheit ihre Literatur nicht beim
Zentrum für Antisemitismusforschung bestellen, sondern ihre eigenen
Bezugswege haben und nutzen.

Ebenso verhält es sich mit Nutzer*innen vor Ort in der Bibliothek des
ZfA: Es ist nicht meine Aufgabe, nach ihrer politischen Gesinnung zu
fragen und daraufhin zu entscheiden, ob sie in der Bibliothek arbeiten
dürfen oder nicht. Unsere Bibliothek steht allen Personen offen, die
darin arbeiten möchten.

\hypertarget{persuxf6nliche-eindruxfccke}{%
\section{Persönliche
Eindrücke}\label{persuxf6nliche-eindruxfccke}}

Ich möchte meinen Beitrag mit persönlichen Eindrücken beenden. Nach
langjähriger Tätigkeit in der Bibliothek der Stiftung Topographie des
Terrors habe ich unter anderem die Stelle gewechselt, weil mir die
Konzentration auf das Thema Nationalsozialismus, Täter und Holocaust zu
schwer wurde. Einige werden nun -- vielleicht zu Recht -- sagen, dass
ein Wechsel zum Zentrum für Antisemitismusforschung thematisch dann
nicht gerade ideal war. Das Thema Antisemitismus ist jedoch breiter und
vielfältiger als Nationalsozialismus und umfasst viele Aspekte. Worauf
ich allerdings tatsächlich nicht vorbereitet war, ist der hohe Anteil
rechter Literatur in der Bibliothek und eben diese
Selbstverständlichkeit, mit der sie im ZfA erworben wird. Mir persönlich
geht es damit nicht immer gut. Ich bekomme leicht den schiefen Eindruck,
dass es sehr viel rechte Literatur gibt und sie prozentual höher ist als
nicht-rechte Literatur. Verstärkt wird dieser Eindruck zusätzlich durch
die hohe Präsenz der AfD und rechter Positionen in der gegenwärtigen
Politik und Berichterstattung, die ich verfolge, um in der Bibliothek
auch inhaltlich auf dem aktuellen Stand zu sein. Ich muss mir dann
bewusst andere Felder suchen, die mich in der Realität erden und die
schönen Dinge des Lebens sehen lassen.

Vor dem Hintergrund der gegenwärtigen erschreckenden politischen
Entwicklungen in Deutschland und weltweit begrüße ich es sehr, dass sich
die aktuelle LIBREAS-Ausgabe mit rechter Literatur in Bibliotheken und
deren Positionierung in der Gesellschaft beschäftigt. Das unterstreicht,
dass im Bibliothekswesen neben den mittlerweile überwiegend technischen
Fragen die Bibliotheksethik weiterhin eine Rolle spielen muss.

%autor
\begin{center}\rule{0.5\linewidth}{\linethickness}\end{center}

\textbf{Irmela Roschmann-Steltenkamp} ist seit Oktober 2016
Bibliotheksleiterin im Zentrum für Antisemitismusforschung, 1994--2016
Bibliotheksleiterin in der Stiftung Topographie des Terrors. Studium der
Germanistik und Europäischen Ethnologie in Göttingen. 1995--1997
berufsbegleitendes Fernstudium am IBI der HU Berlin.

\end{document}
