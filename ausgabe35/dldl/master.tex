\documentclass[a4paper,
fontsize=11pt,
%headings=small,
oneside,
numbers=noperiodatend,
parskip=half-,
bibliography=totoc,
final
]{scrartcl}

\usepackage{synttree}
\usepackage{graphicx}
\setkeys{Gin}{width=.4\textwidth} %default pics size

\graphicspath{{./plots/}}
\usepackage[ngerman]{babel}
\usepackage[T1]{fontenc}
%\usepackage{amsmath}
\usepackage[utf8x]{inputenc}
\usepackage [hyphens]{url}
\usepackage{booktabs} 
\usepackage[left=2.4cm,right=2.4cm,top=2.3cm,bottom=2cm,includeheadfoot]{geometry}
\usepackage{eurosym}
\usepackage{multirow}
\usepackage[ngerman]{varioref}
\setcapindent{1em}
\renewcommand{\labelitemi}{--}
\usepackage{paralist}
\usepackage{pdfpages}
\usepackage{lscape}
\usepackage{float}
\usepackage{acronym}
\usepackage{eurosym}
\usepackage[babel]{csquotes}
\usepackage{longtable,lscape}
\usepackage{mathpazo}
\usepackage[normalem]{ulem} %emphasize weiterhin kursiv
\usepackage[flushmargin,ragged]{footmisc} % left align footnote
\usepackage{ccicons} 
\setcapindent{0pt} % no indentation in captions

%%%% fancy LIBREAS URL color 
\usepackage{xcolor}
\definecolor{libreas}{RGB}{112,0,0}

\usepackage{listings}

\urlstyle{same}  % don't use monospace font for urls

\usepackage[fleqn]{amsmath}

%adjust fontsize for part

\usepackage{sectsty}
\partfont{\large}

%Das BibTeX-Zeichen mit \BibTeX setzen:
\def\symbol#1{\char #1\relax}
\def\bsl{{\tt\symbol{'134}}}
\def\BibTeX{{\rm B\kern-.05em{\sc i\kern-.025em b}\kern-.08em
    T\kern-.1667em\lower.7ex\hbox{E}\kern-.125emX}}

\usepackage{fancyhdr}
\fancyhf{}
\pagestyle{fancyplain}
\fancyhead[R]{\thepage}

% make sure bookmarks are created eventough sections are not numbered!
% uncommend if sections are numbered (bookmarks created by default)
\makeatletter
\renewcommand\@seccntformat[1]{}
\makeatother


\usepackage{hyperxmp}
\usepackage[colorlinks, linkcolor=black,citecolor=black, urlcolor=libreas,
breaklinks= true,bookmarks=true,bookmarksopen=true]{hyperref}
\usepackage{breakurl}

%meta
\expandafter\def\expandafter\UrlBreaks\expandafter{\UrlBreaks%  save the current one
  \do\a\do\b\do\c\do\d\do\e\do\f\do\g\do\h\do\i\do\j%
  \do\k\do\l\do\m\do\n\do\o\do\p\do\q\do\r\do\s\do\t%
  \do\u\do\v\do\w\do\x\do\y\do\z\do\A\do\B\do\C\do\D%
  \do\E\do\F\do\G\do\H\do\I\do\J\do\K\do\L\do\M\do\N%
  \do\O\do\P\do\Q\do\R\do\S\do\T\do\U\do\V\do\W\do\X%
  \do\Y\do\Z}
%meta


%meta

\fancyhead[L]{Redaktion LIBREAS \\ %author
LIBREAS. Library Ideas, 35 (2019). % journal, issue, volume.
\href{http://nbn-resolving.de/}
{}} % urn 
% recommended use
%\href{http://nbn-resolving.de/}{\color{black}{urn:nbn:de...}}
\fancyhead[R]{\thepage} %page number
\fancyfoot[L] {\ccLogo \ccAttribution\ \href{https://creativecommons.org/licenses/by/4.0/}{\color{black}Creative Commons BY 4.0}}  %licence
\fancyfoot[R] {ISSN: 1860-7950}

\title{\LARGE{Das liest die LIBREAS, Nummer \#4 (Winter 2018-Frühling 2019)}} % title
\author{Redaktion LIBREAS} % author

\setcounter{page}{1}

\hypersetup{%
      pdftitle={Das liest die LIBREAS, Nr. \#4 (Winter 2018-Frühling 2019)},
      pdfauthor={Redaktion LIBREAS},
      pdfcopyright={CC BY 4.0 International},
      pdfsubject={LIBREAS. Library Ideas, 35 (2019).},
      pdfkeywords={Open Access},
      pdflicenseurl={https://creativecommons.org/licenses/by/4.0/},
      pdfcontacturl={http://libreas.eu},
      baseurl={http://libreas.eu},
      pdflang={de},
      pdfmetalang={de}
     }



\date{}
\begin{document}

\maketitle
\thispagestyle{fancyplain} 

%abstracts

%body
Beiträge von Eva Bunge (eb), Ben Kaden (bk) Karsten Schuldt (ks),
Michaela Voigt (mv)

\hypertarget{zur-kolumne}{%
\section*{1. Zur Kolumne}\label{zur-kolumne}}

Das Ziel dieser Kolumne ist, eine Übersicht über die in der letzten Zeit
erschienene bibliothekarische, informations- und
bibliothekswissenschaftliche sowie für diesen Bereich interessante
Literatur zu geben. Enthalten sind Beiträge, die der LIBREAS-Redaktion
oder anderen Beitragenden als relevant erschienen.

Themenvielfalt sowie ein Nebeneinander von wissenschaftlichen und
nicht-wissenschaftlichen Ansätzen wird angestrebt. Auch in der Form
sollen traditionelle Publikationen ebenso erwähnt werden wie
Blogbeiträge oder Videos beziehungsweise TV-Beiträge.

Gern gesehen sind Hinweise auf erschienene Literatur oder Beiträge in
anderen Formaten. Die Redaktion freut sich über entsprechende Hinweise
(siehe \url{http://libreas.eu/about/}, Mailkontakt für diese Kolumne ist
\href{mailto:zeitschriftenschau@libreas.eu}{\nolinkurl{zeitschriftenschau@libreas.eu}}).
Die Koordination der Kolumne liegt bei Karsten Schuldt. Verantwortlich
für die Inhalte sind die jeweiligen Beitragenden. Die Kolumne
unterstützt den Vereinszweck des LIBREAS-Vereins zur Förderung der
bibliotheks- und informationswissenschaftlichen Kommunikation.

LIBREAS liest gern und viel Open-Access-Veröffentlichungen. Wenn sich
Beiträge doch einmal hinter eine Bezahlschranke verbergen, werden diese
durch \enquote{{[}Paywall{]}} gekennzeichnet. Zwar macht das Plugin
Unpaywall (\url{http://unpaywall.org/}) das Finden von legalen
Open-Access-Versionen sehr viel einfacher. Als Service an der
Leserschaft verlinken wir OA-Versionen, die wir vorab finden konnten,
jedoch nach Möglichkeit auch direkt. Für alle Beiträge, die nicht frei
zugänglich sind, empfiehlt die Redaktion Werkzeuge wie den Open Access
Button (\url{https://openaccessbutton.org/}) zu nutzen oder auf Twitter
mit \#icanhazpdf (\url{https://twitter.com/hashtag/icanhazpdf?src=hash})
um Hilfe bei der legalen Dokumentenbeschaffung zu bitten.

\hypertarget{artikel-und-zeitschriftenausgaben}{%
\section*{2. Artikel und
Zeitschriftenausgaben}\label{artikel-und-zeitschriftenausgaben}}

In Österreich kam in den letzten Jahren eine Diskussion darüber auf,
welche Auswirkungen prekäre Arbeitsverhältnisse in Bibliotheken für das
betroffene Personal haben und wie verbreitet diese eigentlich sind. Eine
Arbeitsgruppe beim Präsidium der Vereinigung Österreichischer
Bibliothekarinnen und Bibliothekare untersuchte diese Frage. Ute Wödl
stellt deren Ergebnisse vor: Basis ist eine Umfrage, an der im Mai 2017
immerhin 616 KollegInnen teilgenommen haben. Erkennbar sei, dass der
Grossteil dieser eine universitäre Ausbildung abgeschlossen hätten,
obgleich die Stellen mit dieser Voraussetzung zurückgehen. Der Einstieg
ins Berufsfeld erfolgte fast immer über befristete Stellen, rund 30\,\%
des Personals in österreichischen Bibliotheken sei aktuell auf solchen
Stellen eingestellt. Zufrieden sind die KollegInnen im allgemeinen mit
den konkreten bibliothekarischen Aufgaben, aber unzufrieden mit
allgemeiner \enquote{Unterforderung, mangelnde{[}r{]} Führungsqualität
und schlechte{[}m{]} Betriebsklima sowie Überforderung durch
Arbeitsüberlastung und zu hohe Erwartungshaltung} {[}Wödl 2018: 445{]}.
Schlechte Bezahlung und die unsicheren Arbeitsverhältnisse tragen auch,
aber weniger, zur Unzufriedenheit bei. Die Arbeitsgruppe nennt die
Ergebnisse \enquote{nicht so besorgniserregend} {[}Wödl 2018: 448{]},
gibt aber trotzdem Hinweise, sie zu verbessern und regt gleichzeitig an,
die Untersuchung in einigen Jahren zu widerholen. Positiv zu erwähnen
ist auch, dass sich das Präsidium des Bibliotheksverbandes dieser Fragen
des Personals angenommen hat. Dies könnte für die Verbände in anderen
Ländern ein Vorbild sein. (Wödl, Ute (2018). Prekäre Arbeitsverhältnisse
im Bibliothekswesen. In: \emph{Mitteilungen der VÖB} 71 (2018) 3/4,
433-450, \url{https://doi.org/10.31263/voebm.v71i3-4.2164}) ( (ks)

Einen Überblick zu verschiedenen Strategien, mittels der Definition
spezifischer Positionen und Aufgabenfelder für das Bibliothekspersonal
auf die Herausforderungen für Wissenschaftliche Bibliotheken zu
reagieren, versuchen Masanori Koizumi und Michael Majewski Widdersheim
durch eine Clusteranalyse der Job-Titel und -Beschreibungen von 60
US-amerikanischen research libraries. Sie identifizieren sieben
verschieden Strategien, die sich unter anderem darin unterscheiden, wie
viele spezialisierte BibliothekarInnen in einer Einrichtung angestellt
sind. Gleichzeitig wird in ihrer Studie sichtbar, dass einerseits die
eher generische Position \enquote{SystembibliothekarIn} am meisten
verbreitet ist, andererseits aber für bestimmte Aufgaben (unter anderem
Repositorien, digital librarians und electronic resource librarians)
feine Differenzierungen vorgenommen werden. (Koizumi, Masanori ;
Majewski Widdersheim, Michael (2018). Specialities and strategies in
academic libraries: a cluster analysis approach. In: \emph{Library
Management}, 40 (2018), 1/2, 40--58,
\url{https://doi.org/10.1108/LM-10-2017-0114}) {[}Paywall{]} (ks)

Hochschulbibliotheken haben im letzten Jahrzehnt verstärkt ihre Räume
umgebaut, um auf erwartete Veränderungen der Bedarfe von Studierenden zu
reagieren. Es wurde angenommen -- und auch räumlich umgesetzt --, dass
mehr Arbeitsplätze benötigt würden, die einerseits flexibel zu nutzen
sein und andererseits unterschiedliche Formen von Gruppenarbeiten
ermöglichen sollen. So entstanden Lernlandschaften und
Gruppenarbeitsräume mit unterschiedlichen Ausstattungsvarianten, die
heutige Hochschulbibliotheken prägen. Gleichzeitig führten in den
letzten Jahren Bibliotheken Umfragen und andere lokale Studien dazu
durch, wie diese Räume tatsächlich genutzt werden und wie Studierenden
heute tatsächlich lernen. Die Ergebnisse dieser Studien gleichen sich:
Die erwarteten, grossen Veränderungen im Lernen sind nicht eingetreten.
Weiterhin nutzen Studierende die Bibliothek hauptsächlich zum
individuellen Lernen (das auch in Gruppen, also individuell
nebeneinander). Gruppenarbeiten kommen vor, dafür werden die
Gruppenarbeitsräume auch genutzt, aber nicht so massiv, wie es erwartet
wurde. Studierende arbeiten offenbar in Gruppen, wenn dies ihr Auftrag
aus Lehrveranstaltungen ist. Aber die Vorstellung, dass sie auch andere
Lernaktivitäten in Gruppensettings organisieren würden (was teilweise
als \enquote{soziales Lernen} vorhergesagt wurde), hat sich nicht
bestätigt. Ebenso finden andere Lernformen als die individuelle Arbeit
statt, aber auch hier eher weniger als erwartet. Es liegt also nicht
daran, dass die Studierenden diese Formen von Lernaktivitäten nicht
kennen würde, sondern eher daran, dass sie eher individuell arbeiten
wollen. Für die Texas State University erschien zuletzt eine weitere
dieser Studien, die durch eine Befragung zu Lernaktivitäten ihrer
Studierenden wieder zu diesem Ergebnis gelangte. Der Text zeichnet sich
durch eine sehr tiefe Aufschlüsselung der Daten aus, selbstverständlich
sind diese in ihrer Spezifik auch von den lokalen Umständen geprägt;
grundsätzlich stellt sich aber die Frage, ob noch weitere dieser
Befragungen notwendig sind oder ob Hochschulbibliotheken nicht anfangen
sollten, diese Frage als geklärt anzusehen und auf dieser Basis die
Weiterentwicklung ihrer Räume zu planen. Das ändert nichts daran, dass
grundsätzlich immer zu wenig Lernarbeitsplätze vorhanden sind. (Hegde,
Asha L. ; Boucher, Tricia M. ; Lavelle, Allison D. (2018). How Do You
Work? Understanding User Needs for Responsive Study Space Design.
\emph{College \& Research Libraries} 79 (2018) 7,
\url{https://doi.org/10.5860/crl.79.7.895}) (ks)

Klassifikation und Katalogisierung im Bibliothekswesen, wie sie im
DACH-Raum etabliert sind, sind fest in der europäischen Denktradition
verankert; sie sind nicht kontextlos (und so auch nicht ohne Geschichte
oder Ausschlüsse, also nicht \enquote{objektiv}). Für die
Bibliothekswissenschaft und -praxis wäre dieser Fakt ein Ansatz, um über
die Grundlagen der Klassifikation nachzudenken und diese zu verbessern;
auch wenn das bislang selten getan wird. Ein Ansatz, um dieses
Verankertsein in einer Denktradition konkret festzumachen, ist immer die
Beschäftigung mit Alternativen, möglichen und tatsächlich existierenden.
Eine Reihe von First Nations in Kanada und den USA nutzen mit der Brain
Deer Classification eine solche Alternative, welche die
Klassifikationspraxis von Bibliotheken dieser First Nations in Einklang
mit ihren Denktraditionen bringen soll. Die Klassifikation geht nicht
von der Einteilung von Wissen aus, sondern der Situierung von Wissen im
gelebten Alltag, räumlichen und zeitlichen Verbindungen.
Selbstverständlich ist das Ziel dieser Klassifikation, die Bibliotheken
für ihre Community nutzbar zu machen -- und nicht etwa, anderen Kulturen
die Grenzen der eigenen Denktraditionen aufzuzeigen. Aber das ist einer
der Effekte. Annie Bosum und Ashley Dunne stellen die Nutzung dieser
Klassifikation in der Bibliothek des Aanischaaukamikw Cree Cultural
Institute (Oujé-Bougoumou, Quebec, Canada) vor. (Bosum, Annie ; Dunne,
Ashley (2017). Implementing the Brian Deer Classification Scheme for
Aanischaaukamikw Cree Cultural Institute. \emph{Collection Management}
42 (2017) 3/4, 280--293,
\url{https://doi.org/10.1080/01462679.2017.1340858}) {[}Paywall{]} (ks)

Kate Dohe, Digital Programs \& Initiatives Manager in den Bibliotheken
der University of Maryland, reflektiert darüber, wie und wieso digitale
Bibliotheksangebote und -abteilungen -- auch solche, die selber
programmierend beitragen und nicht einfach Software von Firmen kaufen --
weiterhin gesellschaftliche Machtstrukturen reproduzieren, sowohl auf
inter-bibliothekarischen Level als auch lokal. Sie reflektiert darüber,
was eigentlich wirklich als Arbeit geleistet wird und was hingegen als
Arbeit wertgeschätzt oder nicht wertgeschätzt wird -- und thematisiert,
dass dies alles nicht so sein müsste, wie es ist. Insbesondere zeigt
sie, dass die Probleme strukturell sind -- also, das Geld, Geschlecht,
andere Differenzverhältnisse auch dann wirken, wenn es sich um
IT-Personal handelt, welches nicht unbedingt die \enquote{Silicon Valley
Bro-Kultur} lebt, sondern aus persönlichen Gründen in Bibliotheken
arbeitet. Sehr richtig ist ihre Anmerkung, dass sie es müde ist, das
alles immer und immer wieder erklären zu müssen. Ebenso richtig die
Anmerkung, dass interpersonelles Verhalten, welches diese Strukturen
abbauen helfen könnte, genauso erlernt werden kann, wie zum Beispiel das
Programmieren -- und eben nicht eine unveränderliche, persönliche
Charaktereigenschaft darstellt (die \enquote{zufälligerweise} immer
wieder Frauen haben sollen.) Ein wichtiger Text. (Dohe, Kate (2019).
Care, Code, and Digital Libraries: Embracing Critical Practice in
Digital Library Communities. In: \emph{In the Library with the Lead
Pipe} 20.02.2019,
\url{http://inthelibrarywiththeleadpipe.org/2019/digital-libraries-critical-practice-in-communities/})
(ks)

Die großen Entwicklungslinien um die DEAL-Verhandlungen und Plan S
lassen mitunter den Eindruck entstehen, dass Grünes Open Access ein
Auslaufmodell ist. Regine Tobias vom KIT macht in ihrem Artikel -- eine
Ausarbeitung ihres Vortrags beim Bibliothekstag 2018 -- deutlich, dass
dem Grünen Weg insbesondere mit Blick auf institutionelle, regionale
oder nationale Zielvorgaben für OA-Quoten weiterhin eine wichtige Rolle
zukommt: Sie gewährt einen tiefen Einblick in die Praxis, in Bezug auf
Vorüberlegungen und Policyentscheidungen ebenso wie in Fragen der
Integration im eigenen Forschungsinformationssystem mit
OA-Repository-Funktion. Vor allem Policyentscheidungen scheinen
maßgeblich für die erfolgreiche und flächendeckende Umsetzung: Regine
Tobias beschreibt klar, für welche Aspekte, für die Recht ausgelegt
sowie Risiko, Aufwand und Nutzen abgewägt werden müssen, am KIT in
Rücksprache mit der Rechtsabteilung Entscheidungen getroffen wurden.
Damit ist der Beitrag auch ein klares Statement zu einem pragmatischen,
zielgerichteten Umgang mit dem gesetzlich verankerten
Zweitveröffentlichungsrecht nach § 38 (4) UrhG. (Artikel: Tobias, Regine
(2018). Optimierung der Workflows für Zweitveröffentlichungen -- der
\enquote{Grüne Weg} am Karlsruher Institut für Technologie (KIT).
\emph{o-bib. Das offene Bibliotheksjournal}, 5 (2018), 4.
\url{https://doi.org/10.5282/o-bib/2018h4s71-83}; Vortrag: Tobias,
Regine (2018). \enquote{Die Quote kommt -- Einwerbung von
Open-Access-Publikationen durch nutzernahe Workflows im Repository}.
107. Deutscher Bibliothekartag, Berlin, 2018.
\url{https://nbn-resolving.org/urn:nbn:de:0290-opus4-35961}) (mv)

\hypertarget{uxf6ffentliche-bibliotheken-nutzung-und-alltag}{%
\paragraph{Öffentliche Bibliotheken: Nutzung und
Alltag}\label{uxf6ffentliche-bibliotheken-nutzung-und-alltag}}

Öffentliche Bibliotheken werden oft mit einbezogen, wenn
Stadtentwicklung geplant wird. Ihnen werden dann von der Politik
Aufgaben zugeschrieben, welche über die eigentliche Bibliotheksarbeit
hinausgehen: Sie sollen den Stadtteil beleben, helfen, die Wirtschaft
voranzutreiben, das Image der Stadt verändern oder ähnliches. In vielen
Fällen profitieren Bibliotheken davon, indem ihr Etat erhöht,
Personalstellen geschaffen oder gar neue Bibliotheksgebäude errichtet
werden. Geelong, eine Stadt rund 35 Kilometer von Melbourne entfernt,
versucht auf Strukturveränderungen (unter anderem dem Zusammenbruch der
örtlichen Wollindustrie, Rückzug der Automobilindustrie inklusive
Zulieferbetriebe) zu reagieren, indem sie sich zur \enquote{Smart City}
entwickelt, hier verstanden als Hub für Start-Ups, Kultur und Bildung,
ausserhalb, aber in Verbindung zur Metropole Melbourne. In diese
offizielle Entwicklungsstrategie ist die Public Library der Stadt
integriert worden. 2015 wurde unter anderem ein neues Bibliotheksgebäude
in zeitgenössischer Architektur am Hauptbahnhof eröffnet, inklusive der
heute bei solchen Gebäuden zu erwartenden Flächen für flexible Nutzung,
kreatives Arbeit, Business-Center und technischen Infrastruktur. Leorke,
Wyatt und McQuire untersuchten nun, wie sich dieser Anspruch, als Hub
für die Strukturveränderung zu dienen, auf die tägliche Arbeit der
Bibliothek auswirkt. Grundsätzliches Ergebnis ist, dass die Bibliothek
zwischen dem Reagieren auf den offiziellen Diskurs und den tatsächlichen
Bedürfnissen ihrer Nutzerinnen und Nutzer balanciert. Während einerseits
Angebote gemacht und Diskurse bedient werden, welche vorgeblich den
Strukturwandel unterstützen, ist andererseits die tägliche Arbeit nicht
davon geprägt. Der Grossteil der Arbeit besteht -- da hier ein Interesse
besteht -- in der Arbeit mit dem Buch, der Ausleihe und dem Anbieten
eines offenen Ortes. Real hat der offizielle Diskurs kaum Einfluss auf
die Realität der bibliothekarischen Arbeit (oder der Interessen der
NutzerInnen). Relevant ist aber, dass die neue Bibliothek zum negativen
Symbol für den Stadtumbau wurde, als zwei Zweigstellen in anderen Teilen
der Stadt geschlossen und von Teilen der Bevölkerung dafür die Strategie
der Stadt, Dienste im Stadtzentrum zu konzentrieren, verantwortlich
gemacht wurden. (Leorke, Dale ; Wyatt, Danielle ; McQuire, Scott (2018).
\enquote{More than just a library}: Public libraries in the
\enquote{smart city}. In: \emph{City, Culture and Society} 15 (2018),
37--44, \url{https://doi.org/10.1016/j.ccs.2018.05.002}) {[}Paywall{]}
(ks)

Wie Menschen, die neu in einem Land ankommen, die Öffentlichen
Bibliotheken nutzen, ist ein Thema, welches in den letzten Jahren
mehrfach anhand von Untersuchungen der Situation einzelner Bibliotheken
einer Handvoll von Ländern (Skandinavien, Kanada, Australien -- Länder,
die sich explizit als Einwanderungsgesellschaften begreifen) bearbeitet
wurde. Der Artikel \enquote{Settling in: how newcomers use a public
library} von Shepherd, Petrillo \& Wilson ist die Darstellung einer
weiteren dieser Untersuchungen, diesmal über die Bibliothek von Surrey,
einer Grossstadt nahe Vancouver (Kanada). Sie verorten ihre Ergebnisse
in denen vorhergehender Studien (die grundsätzlich zu ähnlichen
Ergebnissen kamen) und können zeigen, dass zumindest für diese
Gesellschaften immer wieder das gleiche gilt: Öffentliche Bibliotheken
können eine Einrichtung und ein Ort sein, welcher aktiv das Ankommen in
der neuen Gesellschaft und die Integration fördert, wenn sie denn
genutzt werden. Neu ankommende Menschen, insbesondere die, welche nicht
in einer der Landessprachen fluent sind, nutzen die Öffentlichen
Bibliotheken -- wenn sie sie nutzen -- durchschnittlich länger als
Einheimische, nutzen Services wie Veranstaltungen mehr, borgen aber
grundsätzlich ähnlich viel aus. Gerade in den ersten Jahren borgen sie
mehr Sachmedien aus, insbesondere über die kulturellen und sozialen
Eigenheiten ihrer neuen Heimat. Mehrsprachige Bestände erkennen sie
positiv als Integrationsmassnahme an, finden sie aber oft veraltet,
zahlenmässig zu gering und auch zu oberflächlich. Sie vertrauen der
Bibliothek als Einrichtung und dem Bibliothekspersonal mehr als anderen
Einrichtungen. Die Herausforderung für Bibliotheken ist vor allem, sie
zu den ersten Besuchen zu überzeugen und dann eine einladende Atmosphäre
zu generieren. Alle Neuankommenden bringen aus ihrer vorherigen
Gesellschaft Bilder über Öffentliche Bibliotheken mit, die nicht mit den
Bibliotheken in ihrer neuen Heimat übereinstimmen müssen -- aber das
muss erst einmal von ihnen bemerkt werden. (Shepherd, John ; Petrillo,
Larissa ; Wilson, Allan (2018). Settling in: How newcomers use a public
library. In: \emph{Library Management} 39 (2018) 8/9, 583--596,
\url{https://doi.org/10.1108/LM-01-2018-0001}) {[}Paywall{]} (ks)

Eine Auswirkung der zunehmenden sozialen Spaltung, gerade in den
Städten, ist die Zunahme von Obdachlosigkeit, auch im DACH-Raum. Auch
wenn gerne darauf verwiesen wird, dass diese in den USA, Kanada,
Grossbritannien sichtbarer und stärker wäre, hiesse es nur die Augen zu
verschliessen, würde man die Zunahme im deutschen Sprachraum ignorieren.
Menschen ohne festen Wohnsitz nutzen Bibliotheken und werden in Zukunft
verstärkt Bibliotheken nutzen -- aus unterschiedlichen Gründen.
Gleichzeitig gibt es -- zum Glück -- Einrichtungen, welche Menschen in
diesen Lagen helfen wollen, ihr Leben sicherer zu leben (und diese Phase
auch wieder zu verlassen): Obdachlosenheime, Vereine, Anlaufstellen,
soziale Einrichtungen. Gut möglich, dass sie sich auch in Europa wieder
stärker etablieren. Deshalb ist die Studie von Mark A. Giesler
(Michigan) darüber, wie solche Einrichtungen Bibliotheken wahrnehmen und
welche Kontakte sie mit ihnen haben, auch über die USA hinaus
instruktiv. Giesler führte mit insgesamt 32 SozialarbeiterInnen,
Angestellten in Homeless Shelters und ähnlichen Einrichtungen
Fokusgruppen zu dieser Frage durch. Hauptergebnis waren: (1) Auch dieses
Personal weiss wenig darüber, was Menschen ohne festen Wohnsitz machen,
wenn sie ihre Einrichtungen verlassen. (2) Sie wissen, dass Bibliotheken
im Leben ihrer KlientInnen eine Rolle spielen und \enquote{shelter
functions} übernehmen -- weil sie offen zugänglich und sicher sind.
Dabei wissen sie auch, dass ein Teil ihre KlientInnen in Konflikte
gerät, unter anderem, da ihnen die Regeln und der Sinn der Bibliothek
nicht bekannt seien und sie diese vor allem als Aufenthaltsraum
wahrnehmen -- was aber auch in Sheltern manchmal Probleme macht. (3)
Trotzdem haben die Meisten keine formalen Kontakte mit den Öffentlichen
Bibliotheken ihrer Gegend. Sie weisen ihre KlientInnen auf Angebote von
Bibliotheken hin; sie hören meistens von Bibliotheken nur wegen
Problemen. (4) Sie können sich aber fast alle vorstellen, gute
Arbeitskontakte mit den Bibliotheken zu haben. Grundsätzlich sind sie
dafür offen, es fehlt zumeist an Zeit und schon etablierten Kontakten,
oft geht diese Möglichkeit einfach unter der anderen Arbeit unter. Die
Kontakte müssen irgendwie hergestellt werden, dann liessen sich die
Angebote von Bibliotheken auch besser für Menschen ohne festen Wohnsitz
nutzen und gleichzeitig Konflikte vermeiden. Damit muss jemand anfangen.
(Giesler, Mark A. (2019). The Collaboration Between Homeless Shelters
and Public Libraries in Addressing Homelessness: A Multiple Case Study.
In: \emph{Journal of Library Administration} 59 (2019) 1, 18--44,
\url{https://doi.org/10.1080/01930826.2018.1549405}) {[}Paywall{]} (ks)

Die Kritik von Zhang und Chawner an anderen Arbeiten zum Themenbereich
Obdachlose / Menschen ohne festen Wohnsitz und Bibliotheken ist, dass
die Stimmen dieser Menschen kaum gehört werden. In ihrer Studie wollten
sie dies ändern und führten in Auckland (Neuseeland) Interviews mit
solchen Personen durch, die regelmässig die dortige Central City Library
nutzen. Grundsätzlich stellen sie fest, dass Bibliotheken im
anglo-amerikanischen Raum dazu übergegangen sind, Menschen ohne festen
Wohnsitz nicht mehr als \enquote{schwierige NutzerInnen} oder gar Gefahr
zu sehen, sondern als Menschen, die den gleichen Anspruch auf
Informationen und Nutzung von Bibliotheken haben wie alle anderen auch
und deshalb zum Beispiel ermöglicht haben, Bibliotheksausweise auf
Adressen von homeless shelters auszustellen. Dies hat auch die
Wahrnehmung der Bibliotheken durch diese Menschen selber verändert.
Wichtig ist für viele, eine Routine aufrecht zu erhalten und Räume zu
haben, in denen sie sich sicher und willkommen fühlen. Zumindest die
Central Library bietet dies, auch -- wie von den interviewten Personen
hervorgehoben wird -- durch das Personal. (Um dies zu erreichen musste
die Bibliothek aber einige Fortbildungen durchführen.) Wichtig sind
Angebote wie ein Movie Club (montags) und ein Buchclub, der zwar für
Menschen ohne festen Wohnsitz eingerichtet wurde, aber allen offen
steht. Relevant ist, dass diese Angebote verlässlich regelmässig
stattfinden und die von Obdachlosigkeit Betroffenen in ihnen mitsprechen
können. Dies erhöhte mit der Zeit das Vertrauen in die Institution
Bibliothek. Entgegen der Vorstellung von \enquote{schwierigen
NutzerInnen} sind sich Menschen ohne festen Wohnsitz sehr wohl bewusst,
was die Aufgaben und Möglichkeiten von Bibliotheken sind und nutzen sie
auch so. Nur sehr wenige schlafen zum Beispiel in der Bibliothek, und
wenn, dann auch nur sehr kurz. (Zhang, Hao ; Chawner, Brenda (2018).
Homeless (rough sleepers) perspectives on public libraries: a case
study. In: \emph{Global Knowledge, Memory and Communication} 67 (2018)
4/5, 276--296, \url{https://doi.org/10.1108/GKMC-11-2017-0093})
{[}Paywall{]} (ks)

Einen instruktiven Einblick in seine Arbeit als Sozialarbeiter in der
Öffentlichen Bibliotheken in Georgetown, einer konservativ geprägten
texanischen Kleinstadt in der Nähe von Austin, gibt Patrick Lloyd. Er
postuliert, dass Stellen wie seine in den letzten Jahren in den USA,
Kanada und auch darüber hinaus in grosser Zahl geschaffen wurden, dass
aber bislang Literatur und Forschung zu diesem Bereich fehlt. Zudem
beschreibt er die spezifischen Herausforderungen in einer wachsenden
Stadt, in der ein Grossteil der Bevölkerung (und Politik) sich weiterhin
in einer ländlichen Gemeinde leben sieht, in welcher alles perfekt sei
und Dinge wie Obdachlosigkeit oder häusliche Gewalt nicht existieren
würden. Die Bibliothek versteht er als \enquote{protective factor} für
Menschen in schwierigen Situationen, er insistiert auch darauf, dass es
nicht seine Aufgabe als Sozialarbeiter ist, die Probleme der Bibliothek
mit diesen Menschen zu lösen, sondern Menschen dabei zu unterstützen,
dahin zu kommen, ihre Bedürfnisse selbstständig erfüllen zu können. Das
wirkt auch beruhigend auf die Bibliothek, aber ist nicht das Hauptziel.
Weiterhin beschreibt er, wie er die Kultur innerhalb der Bibliothek
langsam ändert, hin zur Bewertung von Verhaltensweisen und nicht von
Menschen, auch hin zu einem pro-aktiven Umgang mit Problemen durch das
Personal und damit einhergehend einem besseren Verständnis für die
Probleme und Lebenslagen von Menschen in sozialen Schwierigkeiten durch
dieses Personal. Für den deutschen Sprachraum sind weniger die konkreten
Fakten aus Georgetown interessant als die grundsätzlichen Überlegungen.
Gleichzeitig würde man sich solche Texte aus hiesigen Bibliotheken auch
wünschen. {[}Patrick Lloyd (2019). The Public Library as a Protective
Factor: An Introduction to Library Social Work. In: \emph{Public Library
Quarterly}, \url{https://doi.org/10.1080/01616846.2019.1581872}{]}
{[}Paywall{]} (ks)

\hypertarget{brexit-edition-zur-krise-der-bibliotheken-in-grossbritannien}{%
\paragraph{Brexit-Edition: Zur Krise der Bibliotheken in
Grossbritannien}\label{brexit-edition-zur-krise-der-bibliotheken-in-grossbritannien}}

Neoliberale Politik und Austerität als Grundprinzip der nationalen und
regionalen Politik in Grossbritannien (sowohl unter Labour, Coalition
oder Tories) haben unter anderen zu einer Situation geführt, in der die
dortigen Öffentlichen Bibliotheken sich heute ständig unter realem Druck
sehen. Regelmässigen Schliessungen von Bibliotheken stehen
Aufforderungen gegenüber, trotzdem ständig neue und innovative Angebote
zu machen. Durch den Rückzug des Staates von vielen seiner
Steuerungsfunktionen hat sich heute als politische Strategie etabliert,
in eher kurzfristigen Initiativen Bibliotheken und andere Einrichtungen
zur Entwicklung solcher innovativen Angebote zu animieren, in der
Erwartungen, dass diese nach der jeweiligen Förderungslaufzeit
eigenständig weitergeführt werden, wenn sie erfolgreich sind. Wilson
untersucht den Effekt einer solchen Initiative, der \enquote{Libraries
Development Initiative} des \enquote{Arts Council England}, welche 2012
bis 2013 lief und 143 Öffentliche Bibliotheken erreichte (neben anderen
Einrichtungen). Die Evaluation zeigt, gegen den Strich gelesen, (a) dass
im Ergebnis wenig Innovatives oder Neuartiges Bestand hatte, sondern vor
allem solche Angebote erfolgreich waren, die an die als
\enquote{traditionell} beschriebene Arbeit von Bibliotheken anschlossen
(unter anderem \enquote{Cinema in Libraries}, \enquote{Arts live in
libraries}), (b) dass der Effekt, dass Arbeit durch Kooperation
effektiver wird, oft durch die zusätzliche Arbeit, diese Kooperationen
aufrechtzuerhalten, wieder aufgehoben wird, (c) dass vieles, was als
Erfolg solcher Initiativen geschildert wird, Rhetorik ist, nicht
wirkliche Entwicklung. (Wilson, Kerry (2018). Collaborative leadership
in public library service development. In: \emph{Library Management} 39
(2018) 8/9, 518--529, \url{https://doi.org/10.1108/LM-08-2017-0084})
{[}Paywall{]} {[}OA-Version:
\url{http://researchonline.ljmu.ac.uk/id/eprint/7758}{]} (ks)

In einer Anzahl von Fokusgruppen (acht in zwei Phasen, durchgeführt 2015
und 2016) erhoben Appleton et al., wieso NutzerInnen Öffentlicher
Bibliotheken (in Grossbritannien) diese Bibliotheken positiv
einschätzen. Alle Fokusgruppen wurden in Zeiten von Krisen für die
jeweils lokalen Öffentlichen Bibliotheken (drohende Schliessung, Etat-
oder Personalkürzungen) durchgeführt. Die Annahme war, dass sich in
diesen Zeiten mehr Personen Gedanken um die Vorteile und Aufgaben von
Bibliotheken machen. (Gleichzeitig sagt es auch etwas über den Zustand
Öffentlicher Bibliotheken in Grossbritannien, dass so viel lokale
\enquote{Krisen} zu finden waren.) Gemeinsam waren die Befragten der
Meinung, dass es einen \enquote{epistemischen Wert} der Bibliotheken
gäbe, der als Kern und Versprechen zu erhalten wäre. Sie wurden als safe
spaces, als einladend (\enquote{welcoming}) und auch als Orte der
lokalen Community beschrieben. Wie genau dies gemeint war und wie es zum
Beispiel im Zusammenhang steht mit anderen Ansprüchen (Demokratisierung,
Integration), die mit Bibliotheken verbunden wurden, darüber gab es
weniger Konsens. Die Fokusgruppen zeigten auch, dass für die NutzerInnen
das gedruckte Buch im Mittelpunkt der Bibliotheken stand, neben Raum und
Personal. Zu lernen ist aus dieser Studie, wie sehr sich die
Vorstellungen der Nutzenden von denen der in der bibliothekarischen
Literatur verbreiteten unterscheiden können, obwohl sie eine sehr
positive Meinung von Bibliotheken haben. (Appleton, Leo ; Hall, Hazle ;
Duff, Alistair S. ; Raeside, Robert (2018). UK public library roles and
value: A focus group. In: \emph{Journal of Librarianship and Information
Science} 50 (2018) 3: 275--283,
\url{https://doi.org/10.1177/0961000618769987}) {[}Paywall{]}
{[}OA-Version:
\url{https://www.napier.ac.uk/~/media/worktribe/output-1031576/uk-public-library-roles-and-value-a-focus-group-analysis.pdf}{]}
(ks)

In einem meinungsstarken Beitrag stellt der Bibliotheks- und
Buchhandelsberater Tim Coates dar, wieso sich seiner Meinung nach die
Öffentlichen Bibliotheken in Grossbritannien -- bei Coates im Vergleich
zu denen in den USA und Australien gestellt -- in einer massiven Krise
befinden. Regelmässig werden Bibliotheken geschlossen, Gebäude
zerfallen, Bestände veralten -- obwohl gleichzeitig immer wieder
Kampagnen für Bibliotheken durchgeführt werden. Seiner Meinung nach kann
diese Krise nicht einfach aus politischen Entwicklungen heraus erklärt
werden, auch wenn diese eine Rolle spielen. \enquote{Austerität} als
politisches Konzept gäbe es nicht nur in Grossbritannien, sondern auch
in den USA und Australien; die wirkliche Krise sei in dieser Form aber
nur in Grossbritannien zu beobachten. Er sieht das Hauptproblem darin,
dass die Öffentlichen Bibliotheken in Grossbritannien und ihre Verbände
ohne wirklichen Grund ständig versuchen, sich dem jeweiligen politischen
Diskurs anzupassen: Bibliotheken wurden unter \enquote{New Labour} zu
sozialen Orten erklärt, später zu Innovationszentren oder Orten, wo
Menschen ihre ökonomischen Potentiale entfalten sollten. Aber diese
Versuche seien Schlagworte gewesen, die kaum mit der Realität in den
Bibliotheken zu tun hätten. Durch diese Versuche, die jeweils aktuellen
Schlagworte aufzugreifen, hätten die Bibliotheken ihre eigentlich
relevante Funktion -- nämlich Literatur und Lesen -- vergessen. Gardens
meint, dass die Öffentlichkeit die ständigen Schlagwort-Wechsel nicht
nachvollziehen könnte und auch nicht, warum Bibliotheken nicht einfach
sagen, dass sie Geld für Medien (genauer Bücher), Gebäude und Betrieb
bräuchten. Mit solchen Aussagen könnte die Öffentlichkeit viel mehr
anfangen. Die Öffentlichkeit würde viel eher unterstützen, wenn die
Bibliotheken über Literatur und Lesen reden und das auch in der Praxis
umsetzen würden. Er stellt dies, wie gesagt, auch den Situationen in
Australien und den USA -- in denen er tätig ist -- gegenüber, wo
Bibliotheken zwar auch Schlagworte der Politik aufgreifen, aber diese
die eigene Arbeit viel weniger beeinflussen lassen. In den USA -- so
endet der Text -- würden Bibliothekarinnen und Bibliothekare trotz allen
innovativen Strategien und Projekten bei der ALA-Konferenz weiter massiv
bei Lesungen von AutorInnen zuhören und mit diesen das Gespräch suchen
-- in Grossbritannien nicht. Der Verlust an literarischem Interesse, so
die Grundaussage bei Coates, ginge einher mit der sich verstärkenden
Krise der britischen Bibliotheken; während der Erhalt des literarischen
Interesses zur Resilienz der Bibliotheken in den USA und Australien
beitragen würden. (Coates, Tim (2018). On the Closure of English Public
Libraries. In: \emph{Public Library Quarterly}, 38 (2018), 1, 3--18.
\url{https://doi.org/10.1080/01616846.2018.1538765}) {[}Paywall{]} (ks)

Auch wenn der Neoliberalismus britischer Prägung gerne mit dem Slogan
\enquote{There is no alternative} (Margaret Thatcher) illustriert wird,
gibt es selbstverständlich doch eine ganze Reihe von Versuchen, durch
die direkte und indirekte Kritik des Bestehenden Alternativen zu finden;
je schlechter die Verhältnisse werden umso mehr. Das Buch
\enquote{Against Creativity} von Oli Mould ist eines dieser Versuche.
Ansatzpunkt seiner Kritik ist die spezifische Form, mit der in der
aktuellen Ausprägung des Kapitalismus in Grossbritannien
\enquote{Kreativität} als Rohstoff und Lösung für gesellschaftliche
Probleme produziert, gefördert und ausgenutzt wird. Kreativität sei
seiner utopischen und gesellschaftsverändernden Funktion beraubt worden
und könne praktisch nur noch als Mittel zur Produktion von (ökonomisch
auszunutzendem) Mehrwert verstanden werden, also zur Produktion des
immer Gleichen -- noch mehr Produkten und noch mehr Gentrifizierung. Das
ist letztliche eine neo-marxistische Analyse (aber die hat gerade in
Grossbritannien Konjunktur). Relevant für Bibliotheken ist, dass Mould
sie explizit erwähnt. Er argumentiert, dass sie zusammen mit Museen als
Einrichtungen perfekt geeignet gewesen seien, um neoliberale Formen von
Kreativität zu befördern. Als Einrichtungen, die im Zentrum des
\enquote{Kreativitäts}-Diskurses standen und einige Jahre zum Beispiel
massiv dafür eingesetzt wurden, um einen \enquote{kultur-getriebenen
Stadtumbau} zu befördern (unter anderem durch Neubauten, die mit ihrer
Architektur beeindrucken sollten), wurde ihnen bald der Etat massiv
gekürzt und sie angehalten, sich durch Einbettung in die lokale
Community \enquote{neu zu erfinden} und sich gleichzeitig über diese
\enquote{Einbindung} zu finanzieren. Die Neu-Interpretation von
Bibliotheken als \enquote{Community-Hubs} -- in der bibliothekarischen
Literatur eher als zukunftsweisendes Konzept gehandelt -- interpretiert
er unter diesen Umständen als Zerstörung der gesellschaftsverändernden
Möglichkeiten von Bibliotheken: \enquote{{[}...{]} being forced into
this change because of a lack of alternative {[}...{]} creates an
environment of enforced entrepreneurialism and corporatization.
{[}...{]} {[}Libraries and museums{]} have been forced to compete in the
new global landscape of creative industrial activity because they are
traditional places of knowledge rather than because they are designed to
be competitive and commercial. {[}...{]} Libraries can diversify to be
social service centres; museums can hold evening classes; leisure
centres can host school PE lessons. But it has primed them for
appropriation because their ability to act as engines of non-capitalist
knowledge and social practices has been eroded and, in some cases,
completely destroyed, only for them to be resurrected as another agent
of capitalist, competitive and entrepreneurial versions of creativity.}
(Mould, Oli (2018). \emph{Against Creativity}. London ; New York :
Verso, 2018: 99f.) (ks)

Alles könnte auch anders sein mit den Bibliotheken in Grossbritannien.
Um diesen Abschnitt mit einer positiven Note abzuschliessen, hier der
Verweis auf einen Artikel von Michelle Johansen, welche die soziale
Mobilität von Bibliotheksleitern in London in der viktorianischen Epoche
untersuchte. Dabei konzentrierte sie sich auf die \enquote{Boom-Zeit}
von 1887 bis 1906, in welcher alleine in London mehr als 100 Öffentliche
Bibliotheken eröffnet wurden -- was auch damit zu tun hatte, dass diese
als Teil einer modernen, aufstrebenden, sich selbst verbessernden
Gesellschaft angesehen wurden. Diese Bibliotheken wurden fast immer von
Männern geleitet, die sich aus der \enquote{working class} in diese
Positionen emporgearbeitet hatten. Johansen rekonstruierte eine Reihe
von Biografien dieser Leiter und kommt zu dem Schluss, dass sie sich --
ganz im viktorianischen Zeitgeist -- durch Selbstbildung und harte
Arbeit auf diese Positionen hervor gearbeitet hätten, ohne grosse
Unterstützung durch andere. Sie tendierten dazu, ihre Herkunft zu
verschleiern -- weil eine Ablehnung der \enquote{working class} ebenso
zum viktorianischen Zeitgeist gehörte wie die Vorstellung vom Aufstieg
durch Selbstbildung --, gleichzeitig hätten sie versucht, \enquote{ihre}
Bibliotheken so zu gestalten, dass auch andere einen ähnlichen sozialen
Aufstieg absolvieren konnten. Auch das ist möglich: Die Gesellschaft
muss es nur wollen. (Johansen, Michelle (2019). \enquote{The Supposed
Paradise of Pen and Ink}: Self-education and Social Mobility in the
London Public Library (1880--1930). In: \emph{Cultural and Social
History}, \url{https://doi.org/10.1080/14780038.2019.1574047})
{[}Paywall{]} (ks)

\hypertarget{monographien-und-buchkapitel}{%
\section*{3. Monographien und
Buchkapitel}\label{monographien-und-buchkapitel}}

Das Hauptargument in der auch als Monographie erschienenen Dissertation
von Angelika Merk ist, dass sich die Medien im 15. Jahrhundert nicht in
dem recht einfachen Entwicklungsmodell von per Hand reproduzierten
Texten zum Druck mit beweglichen Lettern entwickelt hätten. Diese
Vorstellung, dass es immer eine Hauptmedienform gäbe, die dann von der
nächsten abgelöst wird, liesse sich nicht halten. Vielmehr sei gerade
das 15. Jahrhundert (hier vor allem im süddeutsch-schweizerischen und
niederländischen Raum) eine Zeit der Medieninnovationen gewesen, in dem
verschiedene Prozesse und Entwicklungen nebeneinander abliefen. Merk
untersucht die in diesem Jahrhundert publizierten Blockbücher -- bei
denen Bild und Text in Holz geschnitten und von diesem Holzschnitt
gedruckt wurden -- auf ihre Verbreitung, Nutzungsformen und Inhalte hin
und kommt so zum genannten Ergebnis: Blockbücher waren keine Vorform des
Drucks mit beweglichen Lettern, sondern existierten für einige
Jahrzehnte als eigene Medienform. Über Teile beschäftigt sich die
Dissertation mit anderen Fragen, zum Beispiel wird bei der näheren
Untersuchung zweier Blockbücher lange auf spätmittelalterliche Theologie
und Astronomie eingegangen, auch ist nicht klar, ob es für das
Hauptargument wirklich eines Rückgriffs auf Niklas Luhmann bedurft
hätte. Aber es ist selbstverständlich ein Ergebnis, welches auch heute
verbreitete Ideen um Medienwandel und Ablösung von Leitmedienformen in
einem anderen Licht erscheinen lässt: Wenn der Medienwandel zum Druck
mit beweglichen Lettern schon nicht so eindeutig und folgerichtig war,
warum sollte das heute, bei aktuellen Formen des Medienwandels, anders
sein? (Angelika Merk: \emph{Blockbücher des 15. Jahrhunderts. Artefakte
des frühen Buchdrucks}. Berlin ; Boston: De Gruyter, 2018) (ks)

Vorbehalte gegenüber Open Access sind in vielen Disziplinen noch weit
verbreitet, beruhen jedoch oft auf anekdotenhaftem Wissen, Hörensagen
und daraus resultierenden Missverständnissen. Cirasella und
Thistlethwaite nehmen in ihrem Buchkapitel diese Vorbehalte gegenüber
Open Access genauer unter die Lupe: Sie gruppieren die von
Promovierenden in Bezug auf die Open-Access-Publikation ihrer
Dissertation geäußerten Ängste in sechs Bereiche, die sie auf ihre
Validität untersuchen und in denen sie verbreitete Missverständnisse
identifizieren und aufklären. Den Autorinnen ist klar anzumerken, dass
sie Befürworterinnen der Open-Access-Bewegung sind -- trotzdem nehmen
sie die geäußerten Ängste und Vorbehalte ernst und wischen sie nicht
einfach vom Tisch. Damit werden Sorgen der Forschenden nachvollziehbar
dargestellt und eine guter Überblick über Ressourcen gegeben, die
herangezogen werden können, um die genannten Missverständnisse
aufzuklären. Der Text hat einen klaren Fokus auf die USA, sodass nicht
alle Punkte auf Deutschland übertragen werden können. (Cirasella, J. \&
Thistlethwaite, P. (2017). Open Access and the Graduate Author. A
Dissertation Anxiety Manual. In K. L. Smith \& K. A. Dickson (Eds.),
Open access and the future of scholarly communication: Implementation (
pp.~203-224). Lanham, MD: Rowman \& Littlefield) {[}Paywall{]}
{[}OA-Version: \url{https://academicworks.cuny.edu/gc_pubs/286/}{]} (eb)

\hypertarget{social-media}{%
\section*{4. Social Media}\label{social-media}}

Was wäre eine Fortsetzung der Kolumne \enquote{Das liest die LIBREAS}
ohne Verweis auf Twitter-Hashtags mit Fokus auf Unterhaltung? Im Januar
twitterten Einrichtungen unter \#GreatestHits Antworten auf häufige bzw.
stereotype Fragen -- so auch Bibliotheken. (\#GreatestHits eingeschränkt
auf \enquote{Bibliothek} siehe
\url{https://twitter.com/search?f=tweets\&q=\%23GreatestHits\%20bibliothek\&src=typd})
(mv)

Molly Poremski (@flyingnuns) hoffte auf die Weisheit der Massen und
fragte auf Twitter \enquote{Librarians: what is the best and/or weirdest
reference question you've ever recieved?}, um Praxisbeispiele für einen
Recherchekurs zu sammeln. Sie wurde nicht enttäuscht. Ob für die
bibliothekarische Ausbildung oder die Vorbereitung eigener Kurse für
Nutzer*innen -- wer selbst auf der Suche nach teils instruktiven, teils
absurden, teils humorvollen Beispielfragen ist, wird hier sicher
Anregungen finden können; bis Ende Februar 2019 gab es über 420
Antworten zu dem Tweet.
(\url{https://twitter.com/flyingnuns/status/1093217141069922304}) (mv)

Im Dezember 2018 erging das BGH-Urteil im sogenannten
Reiss-Engelhorn-Fall. Gegenstand war die Auseinandersetzung zwischen
Wikimedia beziehungsweise einem Wikipedianer und dem Mannheimer
Reiss-Engelhorn-Museum, welches gegen die Onlineveröffentlichung von
Fotografien gemeinfreier Werke geklagt hatte. Vor Verkündung des Urteils
hatte Thomas Hartmann in einem Gastbeitrag bei netzpolitk.org bereits
die Ergebnisse der mündlichen Verhandlung vorgestellt und enttäuscht
zusammengefasst: \enquote{Entsteht durch bloße Digitalisierung eines
gemeinfreien Werks ein neues Schutzrecht am Foto oder Scan? Darüber
streiten die Reiss-Engelhorn-Museen seit drei Jahren mit der Wikimedia
Foundation vor Gericht. In der mündlichen Verhandlung beim
Bundesgerichtshof hatten Freunde der Gemeinfreiheit wenig zu lachen.} Am
20.12.18 berichtete Torsten Kleinz bei Heise über das ergangene Urteil,
welches für Verfechter*innen der Gemeinfreiheit eine Hiobsbotschaft
darstellt. Der Volltext des Urteils ist online zugänglich, woran Ellen
Euler (@EllenEUler) auf Twitter erinnert. (Thomas Hartmann: Urheberrecht
abgelaufen, trotzdem abgemahnt? Wikimedia kämpft vor Gericht für
Gemeinfreiheit
\url{https://netzpolitik.org/2018/urheberrecht-abgelaufen-trotzdem-abgemahnt-wikimedia-kaempft-vor-gericht-fuer-gemeinfreiheit/\#comments},
netzpolitik.org, 1.11.2018; Torsten Kleinz: Bundesgerichtshof: Museen
dürfen gemeinfreie Bilder wegsperren,
\url{https://www.heise.de/newsticker/meldung/Bundesgerichtshof-Museen-duerfen-gemeinfreie-Bilder-wegsperren-4257238.html},
heise.de, 20.12.2018; Volltext des BGH-Urteils siehe
\url{https://juris.bundesgerichtshof.de/cgi-bin/rechtsprechung/document.py?Gericht=bgh\&Art=en\&Datum=Aktuell\&Seite=13\&nr=92142\&pos=419\&anz=558}
-- via
\url{https://mobile.twitter.com/EllenEuler/status/1093502403024953344}.)
(mv)

\hypertarget{konferenzen-konferenzberichte}{%
\section*{5. Konferenzen,
Konferenzberichte}\label{konferenzen-konferenzberichte}}

Sie suchen einen kurzen, englischsprachigen Vortrag zu den Grundlagen
von wissenschaftlichen Publizieren und Open Access, den Sie
Wissenschaftler*innen empfehlen können? Oder Sie suchen Anregungen, wie
Sie selbst diese Grundlagen einmal anders vermitteln können? Dann
empfehlen wir den Vortrag von Claudia Frick (@FuzzyLeapfrog) beim 35.
Chaos Communication Congress (35C3) -- ein wunderbares Beispiel für
Storytelling und zielgruppenspezifische Ansprache von
Wissenschaftler*innen. (Vortrag: Claudia Frick: Locked up science --
Tearing down paywalls in scholarly communication, 35C3, ca. 40 min,
\url{https://media.ccc.de/v/35c3-9599-locked_up_science}. Folien:
Claudia Frick. (2018). Locked up science -- Tearing down paywalls in
scholarly communication. Zenodo.
\url{http://doi.org/10.5281/zenodo.1495601}.) (mv)

\hypertarget{populuxe4re-medien-zeitungen-radio-tv-etc.}{%
\section*{6. Populäre Medien (Zeitungen, Radio, TV
etc.)}\label{populuxe4re-medien-zeitungen-radio-tv-etc.}}

In einem Artikel zum \enquote{30. Geburtstag des World Wide Web}
verweist die NZZ auf eine aktuelle Idee von Tim Berners-Lee zur
Re-Dezentralisierung der Struktur. Das Mittel dazu folgt dem
Grundgedanken des Hypertextes und wird als \enquote{Social Linked Data}
(beziehungsweise Solid) bezeichnet. Dabei werden nutzer*innengenerierte
beziehungsweise nutzer*innenspezifische (= social) Inhalte und Daten auf
Plattformen namens \enquote{Pods} (Personal Online Data Stores)
datenschutzgerecht und individuell transparent verwaltet. Dritte
Anwendungen, die diese Daten nutzen wollen, können dies nur, wenn diese
Nutzung von den individuellen Dateninhaber*innen authentifiziert wird.
(Roberto Simanowski: Traue keinem unter 30! Zum Geburtstag des World
Wide Web und wie sein Gründer es retten will. In: Neue Zürcher Zeitung /
nzz.ch, 02.04.2019
\url{https://www.nzz.ch/feuilleton/30-jahre-world-wide-web-ist-es-noch-zu-retten-ld.1471216})
(bk)

Im Journal der Künste der Berliner Akademie der Künster erläutert der
Archivar Haiko Hübner die Bedeutung von Normdaten und Linked Data. Er
illustriert dies an Entwicklungen im Akademiearchiv, das seit April 2018
ausschließlich GND-Daten für die Ansetzung von Personennamen akzeptiert.
Im Gegenzug wurde ein Tool implementiert, das die Erfassung von noch
nicht in der GND vorhandenen Namen normgerecht ermöglicht und diese per
Webformular an die GND überträgt. Die Akademie unterstützt damit als
Kooperationspartnerin die Bestrebungen, die GND für Kultureinrichtungen
auch außerhalb des Bibliothekswesens aktiv zu öffnen. Zugleich betont
der Autor die Rolle von Normdaten für ein Semantic Web, das im
Akademiearchiv über den Anschluss an Portale wie die Deutsche Digitale
Bibliothek oder Kalliope realisiert werden soll. Zudem ist geplant, so
genannte \enquote{Beacon-Files} in die Datenstruktur der
Akademiebestände einzubinden. (Haiko Hübner: Why use data standards? In:
Journal der Künste, 08, December 2018. Special Issue: The Archives.
S.81) (bk)

Ein Editorial im NEJM, dem New England Journal of Medicine, hat für
Furore gesorgt: Charlotte Haug, sie ist internationale Korrespondentin
des NEJM und hat für diese Tätigkeit ein Angestelltenverhältnis inne,
setzt sich kritisch mit Open Access und insbesondere Plan S\footnote{Siehe
  hierzu auch Eintrag zu Plan S in der Rubrik \enquote{8. Weitere
  Medien} in dieser Ausgabe von \enquote{Das liest die LIBREAS}.}
auseinander. Sie hinterfragt, was die Open-Access-Bewegung de facto an
Verbesserung erreicht hat und ob sie den ursprünglichen
(Selbst-)Ansprüchen gerecht wird. (Spoiler: Charlotte Haug negiert dies
und fasst zusammen \enquote{Open access to research articles is a goal
that both scientists and the public will support. But eliminating
subscription-based publication models without having alternatives in
place that can reliably produce independently vetted, cautiously
presented, high-quality content might have serious unintended
consequences for the integrity of the scientific literature.}) Steven
Salzberg, Professor an der Johns Hopkins University, greift dieses
Editorial auf -- er ist nicht der Einzige -- und seziert Haugs
Positionen beziehungsweise Argumente, die seines Erachtens zum einem
großen Teil aus Falschaussagen und Scheinargumenten bestehen. (Haug, C.
J. (2019). No Free Lunch --- What Price Plan S for Scientific
Publishing? \emph{New England Journal of Medicine}, 380 (2018), 12,
1181--1185. \url{https://doi.org/10.1056/nejmms1900864}. Steven
Salzberg: Highly Profitable Medical Journal Says Open Access Publishing
Has Failed. Right. In: \emph{Forbes}, 1.4.2019,
\url{https://www.forbes.com/sites/stevensalzberg/2019/04/01/nejm-says-open-access-publishing-has-failed-right/\#}.)
(mv)

In der Arte-Reihe \enquote{Baukunst} wird die Bibliothek der
US-amerikanischen Phillips Exeter Academy vorgestellt, ein Neubau von
1971 nach Plänen des Architekten Louis I. Kahn. Die Bibliothek zeichnet
sich unter anderem durch ein \enquote{Raum-im-Raum}-Konzept aus; statt
Gemeinschaftsarbeitsplätzen gibt es für jede*n Leser*in eine Art Carrel.
(Copans, Richard (Regie) (2015): Die Bibliothek von Exeter von Louis
Kahn. 27 Min. In Arte-Mediathek verfügbar bis 25.06.2019:
\url{https://www.arte.tv/de/videos/061747-000-A/die-bibliothek-von-exeter-von-louis-kahn/}.)
(mv)

\hypertarget{abschlussarbeiten}{%
\section*{7. Abschlussarbeiten}\label{abschlussarbeiten}}

Ronald Snijder hat sich in seiner Dissertation dem Themenfeld
Open-Access-Monographien widmet. Er geht unter anderem der Frage nach,
wie sich die freie Verfügbarkeit von Monographien auf Printabsätze
auswirkt. Die Universität Leiden hat dieser Doktorarbeit sogar eine
eigene Pressemitteilung gewidmet und stellt dabei besonders heraus: Open
Access hat kaum Auswirkungen auf den Absatz von Printexemplaren -- weder
positiv noch negativ. (Snijder, Ronald (2019): The deliverance of open
access books : examining usage and dissemination. Dissertation.
Universität Leiden. \url{http://hdl.handle.net/1887/68465}.; Universität
Leiden: Open access books attract many more readers and slightly more
citations, Pressemitteilung 28.01.2019,
\url{https://www.universiteitleiden.nl/en/news/2019/01/open-access-books-attract-many-more-readers-and-slightly-more-citations}.
(mv)

Melanie Janßen hat sich in ihrer Bachelorarbeit im Fachgebiet
Informationswissenschaften der FH Potsdam einer qualitative Untersuchung
von Forschungsdatenrepositorien gewidmet. Sie ist der Frage
nachgegangen, ob Repositorien audiovisuelle Forschungsdaten (und dabei
insbesondere der Typ Video) im Vergleich zu anderen Arten von
Forschungsdaten besonders behandeln. Um es vorweg zu nehmen: Nein, für
die Repositorien sind formal alle Forschungsdaten gleich. (Janßen,
Melanie (2019). Vergleich und Analyse von Forschungsdatenrepositorien:
Exemplarische Untersuchung des Umgangs mit Forschungsdaten unter
besonderer Betrachtung der Ressource Video. Bachelorarbeit.
Fachhochschule Potsdam.
\url{https://nbn-resolving.org/urn:nbn:de:kobv:525-23302}). (mv)

\hypertarget{weitere-medien}{%
\section*{8. Weitere Medien}\label{weitere-medien}}

Aaron Tay arbeitet in einem Blogpost seinen Vortrag beim OCLC Asia
Pacific Regional Conference Meeting 2018 auf und diskutiert, welche
Entwicklungen der letzten Jahre und Themen seines Erachtens die Rolle
von und Aufgaben in (Wissenschaftlichen) Bibliotheken maßgeblich
beeinflussen werden. Welchen Einfluss hat die Etablierung von offenen
Zugang zu Forschungsergebnissen auf die Informationsvermittlung? Zu
welchen Veränderungen führen Entwicklungen im Bereich Künstliche
Intelligenz beziehungsweise Maschinelles Lernen in Bibliotheken und für
Nutzter*innen von Bibliotheken? Wie ändert sich das Verhältnis von
Verlagen, Bibliotheken und Forschenden? (Aaron Tay: Thinking of the
future -- a summary of my thoughts as of 2018,
\url{https://musingsaboutlibrarianship.blogspot.com/2018/12/thinking-of-future-summary-of-my.html},
23.12.2018) (mv)

Eine DOI identifiziert Publikationen oder Forschungsdaten eindeutig,
eine ORCiD die einzelne Forscherin. Aber wir disambiguieren wir
Forschungseinrichtungen? Maria Gould stellt in ihrem Blogpost Research
Organization Registry (ROR) vor -- eine Community-gestützte Initiative,
die das ambitionierte Ziel hat, ein offenes und interoperables
Identifier-System für Forschungseinrichtungen weltweit zu bieten. (Maria
Gould: Hear us ROR! Announcing our first prototype and next steps,
\url{https://ror.org/blog/2019-02-10-announcing-first-ror-prototype/},
10.02.2019) (mv)

Im September 2018 haben verschiedene europäische Forschungsförderer --
DFG und BMBF haben bisher lediglich Sympathie bekundet, sich der
Coalition S aber nicht angeschlossen -- ihren Willen zur (stärkeren)
Verankerung von Open Access in der Förderpolitik kundgetan: Die
Grundsätze wurden im \enquote{Plan S}
(\url{https://www.coalition-s.org/10-principles/}) definiert; einige
Zeit später wurden die \enquote{Implementation Guidelines}
(\url{https://www.coalition-s.org/implementation/}), welche die zehn
Punkte des Plan S näher spezifizieren, veröffentlicht und um Feedback
bis Anfang Februar 2019 gebeten. Die Rückmeldungen gingen zahlreich ein;
laut einer Pressemitteilung der verantwortlichen \enquote{CoalitionS}
vom 20.02.2019 waren es 600 an der Zahl. Lisa Hinchliffe hat in einem
Blogpost bei Scholarly Kitchen einen ersten Versuch unternommen, die
Schwerpunkte und Argumentationslinien zu sortieren und zusammenzufassen.
Eine ausführlichere, dafür unkommentierte Übersicht über öffentliche
Stellungnahmen von Fachgesellschaften, Verlagen und anderen
Interessengruppen liefert das Office of Scholarly Communication im Blog
der Universität Cambridge (Lisa Hinchliffe: "Taking Stock of the
Feedback on Plan S Implementation Guidance,
\url{https://scholarlykitchen.sspnet.org/2019/02/11/with-thousand-of-pages-of-feedback-on-the-plans-s-implementation-guidance-what-themes-emerged-that-might-guide-next-steps/},
11.02.2019. Office of Scholarly Communication: Plan S -- links,
commentary and news items,
\url{https://unlockingresearch-blog.lib.cam.ac.uk/?p=2433}, 10.02.2019.)
(mv)

%autor

\end{document}
