\documentclass[a4paper,
fontsize=11pt,
%headings=small,
oneside,
numbers=noperiodatend,
parskip=half-,
bibliography=totoc,
final
]{scrartcl}

\usepackage{synttree}
\usepackage{graphicx}
\setkeys{Gin}{width=.4\textwidth} %default pics size

\graphicspath{{./plots/}}
\usepackage[english]{babel}
\usepackage[T1]{fontenc}
%\usepackage{amsmath}
\usepackage[utf8x]{inputenc}
\usepackage [hyphens]{url}
\usepackage{booktabs} 
\usepackage[left=2.4cm,right=2.4cm,top=2.3cm,bottom=2cm,includeheadfoot]{geometry}
\usepackage{eurosym}
\usepackage{multirow}
\usepackage[english]{varioref}
\setcapindent{1em}
\renewcommand{\labelitemi}{--}
\usepackage{paralist}
\usepackage{pdfpages}
\usepackage{lscape}
\usepackage{float}
\usepackage{acronym}
\usepackage{eurosym}
\usepackage[babel]{csquotes}
\usepackage{longtable,lscape}
\usepackage{mathpazo}
\usepackage[normalem]{ulem} %emphasize weiterhin kursiv
\usepackage[flushmargin,ragged]{footmisc} % left align footnote
\usepackage{ccicons} 

%%%% fancy LIBREAS URL color 
\usepackage{xcolor}
\definecolor{libreas}{RGB}{112,0,0}

\usepackage{listings}

\urlstyle{same}  % don't use monospace font for urls

\usepackage[fleqn]{amsmath}

%adjust fontsize for part

\usepackage{sectsty}
\partfont{\large}

%Das BibTeX-Zeichen mit \BibTeX setzen:
\def\symbol#1{\char #1\relax}
\def\bsl{{\tt\symbol{'134}}}
\def\BibTeX{{\rm B\kern-.05em{\sc i\kern-.025em b}\kern-.08em
    T\kern-.1667em\lower.7ex\hbox{E}\kern-.125emX}}

\usepackage{fancyhdr}
\fancyhf{}
\pagestyle{fancyplain}
\fancyhead[R]{\thepage}

% make sure bookmarks are created eventough sections are not numbered!
% uncommend if sections are numbered (bookmarks created by default)
\makeatletter
\renewcommand\@seccntformat[1]{}
\makeatother


\usepackage{hyperxmp}
\usepackage[colorlinks, linkcolor=black,citecolor=black, urlcolor=libreas,
breaklinks= true,bookmarks=true,bookmarksopen=true]{hyperref}

%meta
%meta

\fancyhead[L]{B. Mittermaier \\ %author
LIBREAS. Library Ideas, 32 (2017). % journal, issue, volume.
\href{http://nbn-resolving.de/}
{}} % urn 
% recommended use
%\href{http://nbn-resolving.de/}{\color{black}{urn:nbn:de...}}
\fancyhead[R]{\thepage} %page number
\fancyfoot[L] {\ccLogo \ccAttribution\ \href{https://creativecommons.org/licenses/by/3.0/}{\color{black}Creative Commons BY 3.0}}  %licence
\fancyfoot[R] {ISSN: 1860-7950}

\title{\LARGE{From the DEAL engine room -- an interview with Bernhard Mittermaier}} % title
\author{Bernhard Mittermaier} % author

\setcounter{page}{1}

\hypersetup{%
      pdftitle={From the DEAL engine room -- an interview with Bernhard Mittermaier},
      pdfauthor={Bernhard Mittermaier},
      pdfcopyright={CC BY 3.0 Unported},
      pdfsubject={LIBREAS. Library Ideas, 32 (2017).},
      pdfkeywords={Open Access, DEAL, Transformation, Elsevier, Springer Nature, Wiley},
      pdflicenseurl={https://creativecommons.org/licenses/by/3.0/},
      pdfcontacturl={http://libreas.eu},
      baseurl={http://libreas.eu},
      pdflang={en},
      pdfmetalang={en}
     }



\date{}
\begin{document}

\maketitle
\thispagestyle{fancyplain} 

%abstracts

%body
A few years ago it seemed far out of reach (at least for OA advocates
and maybe even for the heads of acquisition) and for more than two years
now something has been on everybody's lips: Whether management in higher
education, scholars, librarians, or science journalists --
\enquote{DEAL} is repeatedly associated with a project that aims at
closing a national deal for scholarly publications, first and foremost
with the three top dog publishers Elsevier, Springer~Nature and Wiley,
and -- later on -- maybe even further academic publishers.

What's special about that? The deal shall not only include access to
scholarly journals, with one price tag and transparent pricing. It shall
also include an open access component for all \enquote{German} articles.
In other words, authors affiliated with institutions that are part of
the DEAL consortium shall be able to publish their articles open access.
Publish and Read (PAR), one deal at one transparent price. It would mean
a huge step for the open access transformation of the journal market.

The (tough) negotiations with Elsevier have in particular been in the
national and international spot. Since the beginning of 2017 as many as
76 German research or higher education institutions did not have a
license agreement with the publisher. Scholars from these institutions
cannot access Elsevier journals directly.\footnote{In the meantime
  Elsevier has restored access for most institutions, see
  \url{http://www.sciencemag.org/news/2017/02/elsevier-journals-are-back-online-60-german-institutions-had-lost-access}}
Some feared scholars would riot, this fear turned out to be
unsubstantiated, though.\footnote{See e.g.
  \url{http://www.nature.com/news/german-scientists-regain-access-to-elsevier-journals-1.21482},
  \url{https://www.helmholtz.de/en/current_topics/press_releases/artikel/artikeldetail/helmholtz_zentren_kuendigen_die_vertraege_mit_elsevier/}
  or
  \url{https://www.berliner-zeitung.de/berlin/hohe-preise-berliner-universitaeten-kuendigen-vertrag-mit-grossem-wissenschaftsverlag-27926974}}
As of end of October 2017, 109 institutions have announced not to renew
individual agreements with Elsevier when the year closes. Thus, from
January 2018 on 185 institutions\footnote{Since beginning of 2017 there
  has been 76 institutions without license agreement: 30 universities,
  16 universities of applied science, 27 research institutions as well
  as 3 state libraries. In total, 109 institutions with subscriptions
  ending by December 2017 have announced not to renew their individual
  license agreements, whereof 29 universities, 57 universities of
  applied science and 23 research institutions. See
  \url{https://www.projekt-deal.de/vertragskundigungen-elsevier-2017/}
  (as of 12.11.2017)} will either be part of a national DEAL consortium,
or Elsevier journals cannot not be accessed.

Reports on the DEAL project have been numerous, on the project's aims as
well as the negotiation progress (or rather negotiation
deadlock).\footnote{See press review on the project website
  \url{https://www.projekt-deal.de/press-review/}} Our primary interest
was different though. The endeavor is of high relevance for the current
and future publishing landscape in German academia. How can it be
accomplished? Which wheels have to mesh to complete the project
successfully? What do the negotiations require from people and
institutions involved? We have interviewed Bernhard Mittermaier, member
of the DEAL negotiating team. The interview was led and translated by
Michaela Voigt and Maxi Kindling.

\hypertarget{chronology-and-team}{%
\section*{Chronology and team}\label{chronology-and-team}}

\emph{\textbf{LIBREAS}: When was the idea for project DEAL born? Is
there a point in time that can be considered as starting point?}

\textbf{BM}: In summer 2013 the rector of the Universität Leipzig
approached the German Rectors' Conference
(\enquote{Hochschulrektorenkonferenz} or \enquote{HRK} in short) and
suggested to negotiate licensing agreements with the major journal
publishers on a national level.\footnote{\url{http://www.tagesspiegel.de/wissen/teure-fachzeitschriften-nationallizenzen-fuer-uni-bibliotheken-gefordert/8624114.html}}
The HRK in turn approached the Alliance of Science Organisations in
Germany that finally commissioned an already existing working group
(\enquote{AG Lizenzen}) to investigate the matter. The expert report
should assess under which conditions the portfolio of major academic
publishers could be part of a national license agreement. Anne Lipp
(German Research Foundation), Hildegard Schäffler (Bavarian State
Library) and myself were then jointly heading the working group AG
Lizenzen and took on the matter on behalf of the working group. Building
on this expertise the Alliance of Science Organisations in Germany set
up a project team -- a group that, with a few additions, exists to this
very day.\footnote{\url{https://www.projekt-deal.de/about-deal/}} The
project team was later joined by the steering committee and, in 2016,
the negotiating group. Furthermore, two full-time positions have been
funded since 2015.

\emph{\textbf{LIBREAS}: That sounds rather straightforward. With some
insight in the landscape of libraries and higher education in Germany,
one can suspect this to be rather difficult. What did it take to turn
the idea into a project?}

\textbf{BM}: Two things are essential -- many institutions, whether
higher education sector and other research organisations, want central
negotiations and want them to be led in a completely new context, and
they are prepared to conduct these negotiations with the necessary
severity. It is equally important to have support from all areas --
management, scholars and libraries.

\emph{\textbf{LIBREAS:} The team is quite heterogeneous. Coincidence or
intention?}

\textbf{BM}: The team as such consists of three groups -- the actual
project team, the negotiating group and the project steering
committee.\footnote{See figure \enquote{Project structure} at
  \url{https://www.projekt-deal.de/about-deal/}} Care was taken to
ensure that libraries and academics, the various disciplines and various
types of institutions were adequately represented in all three groups.
The widest representation of all perspectives is given in the project
steering committee. The project team consists of librarians only,
whereas academics dominate the negotiation team. To use a nautical
analogy: The engine room is staffed with librarians, the ship's bridge
is staffed with academics.

\emph{\textbf{LIBREAS:} And you are certainly also exchanging ideas with
colleagues from abroad?}

\textbf{BM}: Well, we are talking with others, during conferences for
example. But the negotiations are conducted separately.

\emph{\textbf{LIBREAS:} There are two project positions -- \enquote{only
two} or \enquote{two after all}, whichever way you prefer. Do all others
contribute to DEAL as part of their official duties? How much time do
you invest in the project, for example?}

\textbf{BM}: DEAL massively ties up resources, this holds true for both
members of the project and the negotiation team. Personally, I probably
invest at least 20 hours a week.

\hypertarget{the-publishers}{%
\section*{The publishers}\label{the-publishers}}

\emph{\textbf{LIBREAS}: How was the DEAL initiative taken up by
publishers?}

\textbf{BM}: The publishers initially assured themselves that the DEAL
negotiations were indeed mandated by the German research institutions.
Consequently, they were open for negotiations.

\emph{\textbf{LIBREAS:} How many persons are involved on the publisher's
side, can you give a rough estimate?}

\textbf{BM}: During negotiation talks, five people are present on the
publisher's side on average -- at least two hierarchical levels from
sales, as well as dedicated experts, also in the field of Open Access.
We can not estimate how many persons are involved behind the scenes.

\emph{\textbf{LIBREAS}: Press reports suggest that both parties pull no
punches. How would you describe the appearance of the publishers -- both
during the negotiating sessions and in public?}

\textbf{BM}: Overall, the appearance is correct and professional. Still,
neither party gives something away for free. For official announcements
we have come to good terms with both Wiley and Springer Nature: We
coordinate what information is released to institutions and retailers;
in the meantime there were even joint press releases. Details are
released very discreetly. This does not satisfy the understandable need
for information of the public. But it helps to continue negotiations
without major disruptions.

Looking at Elsevier, the situation is more difficult: Although we agreed
not to release details publicly, Elsevier gradually shied away -- when
communicating with individual institutions at first, and little by
little even when communicating with journal editors. Meanwhile, the
present offer is fairly open -- some \enquote{dirty details} are missing
though and, above all, financial aspects are left in the dark.

\emph{\textbf{LIBREAS}: In general, how is the project received by the
(national and international) publishing market? Are you in contact with
other publishers already?}

\textbf{BM}: Other publishers are following the project closely. It is
understandable that deals with the three largest publishers in Germany,
which account for more than half of the market, would have an impact on
other publishers in Germany as well as on publishers' international
business. At the beginning there was an antitrust complaint of the
Börsenverein, the interest group of German publishers. The lobby
expressed concerns that by joining the DEAL consortium the entire budget
would be consumed and libraries would not have the funds for other
publishers. This concern was apparently based on the assumption that one
would have to spend more on a DEAL contract than before. The German
Federal Cartel Office (Bundeskartellamt) did not take up on this path,
but apparently some publishers submit to the propaganda of their own
association: they now want to conduct own DEAL negotiations in order to
get a (supposedly larger) piece of the cake. For reasons of capacity the
DEAL negotiating team currently can not take up negotiations with other
publishers. However, there are some negotiations under the umbrella of
the proposal "Open Access Transformation Contracts" of the German
Research Foundation DFG. They are negotiated by individual institutions,
similar to the DFG funded \enquote{Alliance licenses}.

\hypertarget{getting-prepared-and-negotiations}{%
\section*{Getting prepared and
negotiations}\label{getting-prepared-and-negotiations}}

\emph{\textbf{LIBREAS:} How is the DEAL office organised?}

\textbf{BM}: There is no actual DEAL office. The Alliance of Science
Organisations in Germany funds two project positions: One position has a
focus on data collection and analysis, located at the Max Planck Digital
Library in Munich, one position has a focus on public relations, located
at the University Library Freiburg. Furthermore, we are supported by the
office of the regional consortium Baden-Wuerttemberg, which is also
located at the University Library Freiburg.

\emph{\textbf{LIBREAS:} The DEAL team is preparing for a new negotiating
session. How does that look like? Is there a flood of emails, strategy
papers...?}

\textbf{BM}: Initially, there were many meetings, face to face and
videoconferencing, for example to set the DEAL negotiating goals. In the
meantime, the preparation mainly takes place via email and before the
actual negotiating session, in the form of preliminary meetings. The
negotiating team is now very well established.

\emph{\textbf{LIBREAS:} How do negotiations come about, who sets the
dates?}

\textbf{BM}: The office of Professor Hippler proposes dates, which are
then coordinated in the negotiating group -- we doodle, to put it
briefly. We always try to gather as many academics as possible and at
least one librarian. The result will then be coordinated with the
publishers. It's easy to imagine that finding a date is not easy, also
because based in the Netherlands, Great Britain and even the USA are
involved on the publishers' side.

\emph{\textbf{LIBREAS:} How should one envision the atmosphere in the
negotiating room?}

\textbf{BM}: The sessions last between two and four hours and usually
take place in the rooms of the German Rectors' Conference in Bonn or
Berlin. Professor Hippler, President of the German Rectors' Conference
HRK, is the main negotiator for the DEAL team. As already mentioned, the
negotiation team consists of academics and librarians, the team is very
well established. Sure, the mood is tense more often than it is
resolved, most of the time words are not minced. But there has always
been a handshake -- even at the farewells.

\emph{\textbf{LIBREAS:} How common is the term "SciHub" during such a
session?}

\textbf{BM:} In the meantime, only rarely. The publishers know that
ultimately we are interested in signing a deal -- and not in choosing
between a DEAL contract and Sci-Hub. Conversely, if it is claimed that a
research institution could not do without the publisher's journals, we
point out that the experience of Elsevier dropouts teaches something
different: various legal ways of alternative document delivery are used
to ensure the literature supply at the individual institutions.

\hypertarget{criticism}{%
\section*{Criticism}\label{criticism}}

\emph{\textbf{LIBREAS:} You negotiate a DEAL with publishers that have a
focus on the traditional publishing model. What does this actually have
to do with Open Access? }

\textbf{BM}: A lot, by now. Sure, the initial focus was negotiating
subscription deals. But all parties realized very quickly that this
could not be the end of the road. Apart from access to all of the
publisher's journals for all participating institutions, the
negotiations also aim at unleashing articles by academics of the
participating institutions. That is to ensure that if an academic from a
participating institution is corresponding author, the article is
published Open Access and under CC BY, an Open Access compliant license.
In addition, the costs incurred should be fair and forward-thinking --
and they should be based on the number of articles. In other words, for
all subscription journals of the publishers concerned the DEAL project
would put participating institutions almost in the same position as if
it were open access journals: own articles are Gold OA and are published
under CC BY. All other articles can be accessed, but not used under the
terms of a Creative Commons license. No further costs are incurred for
publishing Gold Open Access. The same goes for all gold open access
journals -- they would be included in the DEAL contract with the
according publisher, authors would not have to pay additional article
processing charges.

\emph{\textbf{LIBREAS:} OA advocates criticize that DEAL leads to a
further commercialization of the scholarly publication market and gives
preference to the "big ones" -- while grassroots initiatives and newly
founded OA publishers, especially in the field of OA monographs, are
under constant financial pressure. What do you think about this?}

\textbf{BM}: DEAL wants to make a contribution to the Open Access
transformation of scholarly publishing. It is true that the actual DEAL
negotiations are limited to these three publishers. However, there are
also discussions with other publishers in connection with the call for
proposals "Open Access Transformation Contracts" by the German Research
Foundation DFG. All publishers who are ready to embark on a journey to
transformation are invited to participate. Apart from that, DEAL does
not presume to intervene in the freedom of research and teaching
guaranteed by the constitution. Also, we do not want to dictate scholars
where to publish, but we want to make sure the publishing options that
scholars can select freely from meet the needs of the scholarly system.
In our view, this means that they should become open access if they are
not already.

\hypertarget{outlook}{%
\section*{Outlook}\label{outlook}}

\emph{\textbf{LIBREAS}: When will we have reached a DEAL?}

\textbf{BM:} Talks with Springer Nature and Wiley are well on their way.
When it became apparent in September 2017 that the agreement could not
be concluded by the end of the year, a transitional solution had to be
found in order to avoid a contractless status from January 2018
on.\footnote{\url{https://www.projekt-deal.de/vertragskundigungen-elsevier-2017/}}
Such a transitional solution was possible with both publishers and
agreed upon via email, in telephone conferences and at the recent
Frankfurt book fair. Due to the good atmosphere during these talks and
the publishers' willingness to sign up for the DEAL route, which has now
become contractually evident, I am optimistic that a deal can be
reached, probably in the first or second quarter of 2018.

With Elsevier the situation is very different. Although the negotiations
have lasted almost a year longer, they are less advanced than those with
Springer Nature and Wiley. Elsevier has not yet accepted the approach of
exclusively paying for publishing. Also, the mutual setup is much more
confronting: institutions have not extended their expiring contracts. At
the beginning of 2017, this involved almost 70 institutions, and now the
number is up to 180 and more. At the beginning of October, the first
resignations of editors were handed over to Elsevier and announced
publicly.\footnote{\url{https://www.projekt-deal.de/herausgeber_elsevier/}}
Elsevier, on the other hand, is reaching out to individual institutions
-- even though there are no negotiations to be held. Editors are
contacted and invited to "Editors Dinners", and contact with rectorates
and ministries is sought. The alleged purpose is to break up the lines
on the side of DEAL -- so far without success. One can only hope that
the transitional solutions with Wiley and Springer Nature, which have
met great interest in the public, trigger some action on Elsevier's
side. If this does not happen, there will be further escalation: more
institutions will terminate agreements, editors will resign at regular
intervals. Eventually DEAL will announce Elsevier's latest offer to the
institutions, including financial details. At the latest when deals with
Wiley and Springer Nature are concluded, Elsevier will have to put their
cards on the table. If there is still no progress to be seen, one must
assume that Elsevier would rather forego sales in Germany than to
question their business model. But even that would be very risky for the
publisher: After all, it is a large field trial to the question of
whether you can live without Elsevier journals.

\emph{\textbf{LIBREAS}: From the DEAL team's perspective -- what is
desirable?}

\textbf{BM:} That involved institutions remain calm. Rarely, if ever,
has there been such an attempt. It receives international attention, not
to say admiration. And it has the best chance of success.

\emph{\textbf{LIBREAS}: Dear Mr.~Mittermaier, thank you very much for
your time!}

%autor
\begin{center}\rule{0.5\linewidth}{\linethickness}\end{center}

\textbf{Dr.~Bernhard Mittermaier} holds degrees in Chemistry (Diplom),
Library and Information Science (M.A.) and Analytical Chemistry (Ph.D.).
He is Head of the Central Library of Forschungszentrum Jülich. Apart
from being a member of the DEAL negotiating team and the National Open
Access Contact OA2020-DE he is part of the steering committee ``Zukunft
der Digitalen Informationsversorgung'' (``Future of the Digital
Information Supply'') of the Alliance of Science Organisations in
Germany. (ORCID: \url{http://orcid.org/0000-0002-3412-6168})

\textbf{Michaela Voigt} is member of the Open Access Team at Technische
Universität Berlin and editorial board member of LIBREAS. Library Ideas.
(ORCID: \url{http://orcid.org/0000-0001-9486-3189})

\textbf{Maxi Kindling} is researcher and lecturer at Berlin School of
Library and Information Science (Humboldt-Universität zu Berlin). She is
co-founder and co-editor of LIBREAS. Library Ideas. (ORCID:
\url{http://orcid.org/0000-0002-0167-0466}

\end{document}
