\documentclass[a4paper,
fontsize=11pt,
%headings=small,
oneside,
numbers=noperiodatend,
parskip=half-,
bibliography=totoc,
final
]{scrartcl}

\usepackage{synttree}
\usepackage{graphicx}
\setkeys{Gin}{width=.4\textwidth} %default pics size

\graphicspath{{./plots/}}
\usepackage[ngerman]{babel}
\usepackage[T1]{fontenc}
%\usepackage{amsmath}
\usepackage[utf8x]{inputenc}
\usepackage [hyphens]{url}
\usepackage{booktabs} 
\usepackage[left=2.4cm,right=2.4cm,top=2.3cm,bottom=2cm,includeheadfoot]{geometry}
\usepackage{eurosym}
\usepackage{multirow}
\usepackage[ngerman]{varioref}
\setcapindent{1em}
\renewcommand{\labelitemi}{--}
\usepackage{paralist}
\usepackage{pdfpages}
\usepackage{lscape}
\usepackage{float}
\usepackage{acronym}
\usepackage{eurosym}
\usepackage[babel]{csquotes}
\usepackage{longtable,lscape}
\usepackage{mathpazo}
\usepackage[normalem]{ulem} %emphasize weiterhin kursiv
\usepackage[flushmargin,ragged]{footmisc} % left align footnote
\usepackage{ccicons} 

%%%% fancy LIBREAS URL color 
\usepackage{xcolor}
\definecolor{libreas}{RGB}{112,0,0}

\usepackage{listings}

\urlstyle{same}  % don't use monospace font for urls

\usepackage[fleqn]{amsmath}

%adjust fontsize for part

\usepackage{sectsty}
\partfont{\large}

%Das BibTeX-Zeichen mit \BibTeX setzen:
\def\symbol#1{\char #1\relax}
\def\bsl{{\tt\symbol{'134}}}
\def\BibTeX{{\rm B\kern-.05em{\sc i\kern-.025em b}\kern-.08em
    T\kern-.1667em\lower.7ex\hbox{E}\kern-.125emX}}

\usepackage{fancyhdr}
\fancyhf{}
\pagestyle{fancyplain}
\fancyhead[R]{\thepage}

% make sure bookmarks are created eventough sections are not numbered!
% uncommend if sections are numbered (bookmarks created by default)
\makeatletter
\renewcommand\@seccntformat[1]{}
\makeatother


\usepackage{hyperxmp}
\usepackage[colorlinks, linkcolor=black,citecolor=black, urlcolor=libreas,
breaklinks= true,bookmarks=true,bookmarksopen=true]{hyperref}
%URLs hart brechen
\makeatletter 
\g@addto@macro\UrlBreaks{ 
  \do\a\do\b\do\c\do\d\do\e\do\f\do\g\do\h\do\i\do\j 
  \do\k\do\l\do\m\do\n\do\o\do\p\do\q\do\r\do\s\do\t 
  \do\u\do\v\do\w\do\x\do\y\do\z\do\&\do\1\do\2\do\3 
  \do\4\do\5\do\6\do\7\do\8\do\9\do\0} 
% \def\do@url@hyp{\do\-} 
\makeatother 

%meta
%meta

\fancyhead[L]{V. Seer \\ %author
LIBREAS. Library Ideas, 32 (2017). % journal, issue, volume.
\href{http://nbn-resolving.de/}
{}} % urn 
% recommended use
%\href{http://nbn-resolving.de/}{\color{black}{urn:nbn:de...}}
\fancyhead[R]{\thepage} %page number
\fancyfoot[L] {\ccLogo \ccAttribution\ \href{https://creativecommons.org/licenses/by/3.0/}{\color{black}Creative Commons BY 3.0}}  %licence
\fancyfoot[R] {ISSN: 1860-7950}

\title{\LARGE{Von der Schattenbibliothek zum Forschungskorpus. Ein Gespräch über Sci-Hub und die Folgen für die Wissenschaft}} % title
\author{Veil Seer} % author

\setcounter{page}{1}

\hypersetup{%
      pdftitle={Von der Schattenbibliothek zum Forschungskorpus. Ein Gespräch über Sci-Hub und die Folgen für die Wissenschaft},
      pdfauthor={Veil Seer},
      pdfcopyright={CC BY 3.0 Unported},
      pdfsubject={LIBREAS. Library Ideas, 32 (2017).},
      pdfkeywords={Open Access, Sci-Hub, Schattenbibliothek, Piraterie},
      pdflicenseurl={https://creativecommons.org/licenses/by/3.0/},
      pdfcontacturl={http://libreas.eu},
      baseurl={http://libreas.eu},
      pdflang={de},
      pdfmetalang={de}
     }



\date{}
\begin{document}

\maketitle
\thispagestyle{fancyplain} 

%abstracts

%body
\hypertarget{vorbemerkung}{%
\section*{Vorbemerkung}\label{vorbemerkung}}

Die so genannten Schattenbibliotheken sind vermutlich die radikalste
Form, um Beschränkungen des Zugangs zu wissenschaftlicher Literatur zu
umgehen. Sie nutzen die Möglichkeiten digitaler Medien und Netze und
vermitteln konsequent, niedrigschwellig und umfangreich den Zugang zu
digitalen Kopien von Monographien und Aufsätzen. Obschon offensichtlich
in vielen Fällen urheberrechtswidrig, erfreuen sich diese Plattformen
einer regen Nachfrage und zwar naturgemäß vor allem aus dem Bereich der
Wissenschaft. Es scheint, als würden die Schattenbibliotheken
Versorgungslücken schließen, die auch vergleichsweise gut ausgestattete
Hochschulbibliotheken haben. Einer weiterer Aspekt könnte sein, dass die
sehr einfache Benutzbarkeit verbunden mit der Aussicht, ein Paper
garantiert herunterladen zu können, den Umweg über VPN-Einwahl und
Bibliotheks-Login unattraktiv macht. Werden Schattenbibliotheken somit
zur Konkurrenz für Hochschulbibliotheken? Und mit welcher Motivation
setzen sich die Betreiber dieser Plattformen einer möglichen
Strafverfolgung, die im Kontext der jüngsten Urteile nun
wahrscheinlicher wird, aus? Im Juni diesen Jahres bekam der
Wissenschaftsverlag Elsevier, der Gewinnmargen von 37\% erzielt (siehe
Holcombe 2015), von einem US-Gericht knapp 13 Millionen Euro
Schadensersatz für die nicht genehmigte Verbreitung von 100
wissenschaftlichen Artikeln (siehe Scherschel 2017 und Strecker 2017)
zugesprochen. Die Fachgesellschaft American Chemical Society (ACS)
erhielt in einem zweiten Urteil den Anspruch auf knapp 4,1 Millionen
Euro Schadenersatz für 32 zu Unrecht kopierte und verbreitete Werke
(Siehe Ernesto 2017). Neben der zunehmenden Kriminalisierung erfährt
Sci-Hub jedoch auch großen Zuspruch, der sich neben der starken Nutzung
auch in Solidaritätsbekundungen manifestiert (siehe Barok, D. et al
2015).

LIBREAS hatte die Gelegenheit, sich mit einer an der Plattform Sci-Hub
beteiligten Person zu unterhalten. Das Interview und die damit
verbundenen informationsethischen Fragestellungen führten dabei
redaktionsintern zu einigen Diskussionen. Auch stand die Frage im Raum,
ob der Text überhaupt veröffentlicht werden sollte. Sci-Hub kann jedoch
als mittlerweile sehr großer potentieller Forschungskorpus sowie
aufgrund seiner starken Nutzung als wissenschaftliches Instrument trotz
seines rechtlichen Status nicht ignoriert werden. Wir haben versucht,
einzelne Behauptungen im Text mit Quellen zu belegen. Naturgemäß sind
jedoch nicht sämtliche Angaben verifizierbar. Daher haben wir uns
entschlossen größtenteils möglichst direkt die Aussagen des
Interviewpartners wiederzugeben.

Alle redaktionellen Anmerkungen sind durch eckige Klammern
gekennzeichnet.

Das Gespräch für LIBREAS führten Linda Freyberg und Ben Kaden.

\hypertarget{interview}{%
\section*{Interview}\label{interview}}

\emph{\textbf{LIBREAS}: Wie kam Alexandra Elbakyan eigentlich dazu,
Sci-Hub ins Leben zu rufen?}

Soweit bekannt war Alexandra Elbakyan, {[}die von Nature 2016 in die Top
10 der wichtigsten Personen der Wissenschaft gewählt wurde (siehe Van
Noorden 2016){]}, am \emph{Freiburg Institute of Advanced Studies} und
beschäftigte sich dort mit Wissenschaftskommunikation, genauer gesagt
mit jener der frühen Neuzeit. Sie näherte sich dem Thema also
wissenschaftshistorisch und nicht als die subversive Hackerin, als die
sie in der Berichterstattung oftmals dargestellt wird. Als sie dann nach
Kasachstan zurückgekehrt war, stellte sie fest, dass sie auf die für
ihre Forschungen und ihre Dissertation notwendigen Quellen keinen
Zugriff hatte. Dieses Zugangsproblem ist die Ausgangsmotivation für die
Gründung von Sci-Hub. Im Prinzip besteht natürlich bei fast allen
Wissenschaftsverlagen die Möglichkeit, den Zugriff auf Inhalte online zu
erwerben. Aber Durchschnittspreise von 20--30 Euro pro Aufsatz sind
nicht nur für Promovierende kaum finanzierbar. Besonders in Regionen, in
denen die wissenschaftliche Informationsversorgung ohnehin schwach
ausgeprägt ist, ergeben sich daraus erhebliche Problem für die
Forschung, die auf relevante und aktuelle Literatur angewiesen ist.

Mittlerweile geht es bei Sci-Hub um Daten im Petabyte-Bereich, die
Alexandra sicherlich nicht alleine in ihrer Studentenwohnung verwaltet,
was die Fixierung auf ihre Person nicht rechtfertigt. Es steht natürlich
sehr viel Entwickler- und Hardwarekapazität hinter Sci-Hub. Die
Infrastruktur wird zum Beispiel von einer großen Stiftung unterstützt.
Und es gibt viele freiwillige Mitarbeiter wie mich. Diese werden jetzt
zunehmend kriminalisiert. Vor ein paar Tagen erst wurde in Australien
ein Kollege verhaftet. {[}Dies wäre dann im Kontext der in der
Vorbemerkung genannten Urteile die erste Festnahme einer an Sci-Hub
mitarbeitenden Person.{]}

\emph{\textbf{LIBREAS}: Wie arbeitet Sci-Hub?}

Der Schlüssel bei Sci-Hub sind die DOIs. Darüber werden die Paper
identifiziert, abgeholt und ausgeben. Gibt man eine solche in die
Suchmaske ein, wird eine Abfrage generiert, bei der wissenschaftliche
Bibliotheken oder auch große Landesbibliotheken als Proxy funktionieren.
{[}Diese Einrichtungen zum Beispiel Cambridge University werden dann in
der Fußzeile auf dem Volltext-PDF ausgewiesen, so wie es bei einem
inneruniversitären Zugriff der Fall ist.{]} Sci-Hub nutzt also die
Zugänge von Bibliotheken und vermittelt daher prinzipiell lizenzierte
Inhalte. Nur eben an die Allgemeinheit. Das funktioniert natürlich nur,
weil es in diesen Einrichtungen Leute gibt, die mit Sci-Hub kooperieren,
nämlich Bibliothekar*innen und technische Mitarbeiter*innen. {[}Dies
wird auch von der Gründerin so wiedergegeben: \enquote{According to
founder Alexandra Elbakyan, the website uses donated library credentials
of contributors to circumvent publishers' paywalls and thus downloads
large parts of their collections} (Greshake, 2017). Auf der UWM-Website
wird dies einerseits auch so wiedergegeben, aber daneben auch das
Erhalten der Zugangsdaten durch Phising behauptet: \enquote{Sci-Hub
appears to obtain institutional credentials from both people who support
the program and willingly surrender their account information to
Sci-Hub; and phishing victims who unwittingly give Sci-Hub access to
their accounts.} (\url{http://guides.library.uwm.edu/scihub}){]}

Dahinter stehen durchaus wissenschaftsethische Motivationen, sicher auch
die Frustration, dass sich an der Monopolsituation im wissenschaftlichen
Publikationswesen so wenig ändert.

\emph{\textbf{LIBREAS}: Gibt es Entwicklungspläne für das Angebot?}

Traditionell besorgt Sci-Hub die Paper nur auf Anfrage, also auf Bedarf.
Daran soll sich perspektivisch etwas dahingehend ändern, dass Paper
direkt über die DOI bei Crossref geharvestet und vorgehalten werden. Im
Prinzip sollten aber bereits jetzt über das alte Modell potentiell
98\,\% aller wissenschaftlichen Publikationen mit einer Crossref-DOI
verfügbar sein.

Während LibGen, die ja hauptsächlich Monographien zur Verfügung stellen,
meines Wissens nach nun ihr Angebot um Patente und fiktionale Literatur
erweitern möchten, beginnt Sci-Hub auch systematisch Monographien zu
sammeln. In absehbarer Zeit möchte Sci-Hub, die zur Zeit ca. 37
Millionen Bücher, die über Google Books verfügbar sind in einem einzigen
Volltext-Repositorium außerhalb der Kontrolle von Google verfügbar
machen. Dabei sollen die technischen Besonderheiten der
länderspezifischen Google Book Suche genutzt werden.

\emph{\textbf{LIBREAS}: Was sagt Google dazu?}

Das fragen sich viele. Ich denke Google weiß um diese technische
Besonderheit. Ansonsten würden sie es ändern. Also ich gehe davon aus,
dass sie es wissen.

\emph{\textbf{LIBREAS}: Was fällt Dir am Echo in der Öffentlichkeit zu
Sci-Hub auf?}

Die journalistische Berichterstattung konzentriert sich ja häufig auf
das Hacker- und Piraten-Narrativ, bei dem Paper \enquote{erbeutet}
werden. Das wird Sci-Hub aber wenig gerecht. Denn das Hacken spielt
eigentlich gar keine Rolle. Wie beschrieben vermittelt Sci-Hub zumindest
lokale legale Zugänge über die Bibliotheken und Rechenzentren.

\emph{\textbf{Libreas}: Warum nutzen Leute Sci-Hub und auch andere
Schattenbibliotheken, obwohl man ja weiß, dass das Angebot rechtlich
eigentlich nicht zulässig ist?}

Schattenbibliotheken bedienen ganz klassisch eine Lücke zwischen
Nachfrage, Angebot und Zugangsmöglichkeit. Denkt man an die Tradition
des Samisdat, so sieht man eine Motivation: Publikationen, die es gab,
aber die man nicht offiziell bekommen konnte, wurden unter der Hand
vervielfältigt und weiter gegeben. Eine ähnliche Logik motiviert
zumindest in Osteuropa viele Akteure hinter digitalen
Schattenbibliotheken. Das sieht man auch an Angeboten wie LibGen. Es
werden also Versorgungsengpässe aufgefangen.

Bei Sci-Hub kommt noch etwas anderes hinzu. Viele Nutzer*innen rufen
auch Publikationen ab, die sie über ihre Heimateinrichtungen abrufen
könnten, da diese Lizenzen erworben hat. Das ist eher eine Sache der
Usability. Bei Sci-Hub bekommt man ein PDF direkt nach Eingabe der DOI.
Die kann man leicht aus den Referenzen eines anderen Papers
herauskopieren oder sogar abtippen. In zwei Schritten hat man die
Quelle. Bei Bibliothekskatalogen und Verlagsdatenbanken muss man in der
Regel längere Wege gehen. Arbeitet man zum Beispiel im Home-Office, muss
man sich zumeist erst umständlich einwählen bzw. identifizieren.

Die Alternative bei der Auffindbarkeit wäre Google. Der Vorteil von
Sci-Hub ist allerdings, dass man sicher sein kann, so gut wie jede
Publikation zu bekommen und nicht doch am Ende vor einer Bezahlschranke
zu landen.

Insofern überrascht wenig, dass die intensivste Nutzung des Angebots
nach Russland aus den USA und aus Deutschland zu verzeichnen ist, also
prinzipiell reichen und lizenzstarken Ländern mit einer sehr
umfangreichen Wissenschaft. Bequemlichkeit ist hier eindeutig als
Motivation festzustellen. Und der Bedarf nach vielen Quellen.

Erstaunlich ist auch, wie gering die ethischen Bedenken der Forschenden
sind. Sie gewichten den Zugang zu wissenschaftlichen Materialien
deutlich höher als die Verwertungsinteressen der Intermediäre. Man kann
es auch so sagen: Es wird sich kaum ein Wissenschaftler finden, der
nicht über die Wissenschaftsverlage klagt. Allerdings sind viele
Forschende auf Selbstschutz bedacht. Sie müssen also zwischen der
Versuchung des schnellen und einfachen Zugriffs und den möglichen
Konsequenzen abwägen. Da aber bisher keine Sanktionen gegen
Sci-Hub-Nutzer*innen bekannt sind, halten sich die Skrupel doch in
Grenzen. Es gibt gerade bei Sci-Hub eine große Toleranz, vermutlich weil
hier wirklich ein Bedarf direkt aufgegriffen wird.

\emph{\textbf{LIBREAS}: Wie können oder sollten die Wissenschaftsverlage
darauf reagieren?}

Sicherlich nicht mit Strafverfolgung. Im Prinzip müssen sie einsehen,
dass die Zugangskontrolle zu digitalen Publikationen kein tragfähiges
Geschäftsmodell sein kann. Perspektiven sehe ich in Mehrwertdiensten und
besonders im Information Retrieval. Bei einem stetig wachsendem
Publikationsaufkommen ist die Auffindbarkeit der jeweils relevanten
Texte für viele eine Herausforderung. Es verwundert daher nicht, dass
zum Beispiel Elsevier verstärkt informationswissenschaftliche Kompetenz
einkauft. Ich kann mir gut vorstellen, dass wir früher oder später eine
Art Wissenschafts-Spotify sehen werden, das dann auch von individuellen
Wissenschaftler*innen direkt und nicht mehr unbedingt über Bibliotheken
lizenziert wird.

Ich gehe auch davon aus, dass der einzelne Artikel nicht in jedem Fall
die relevante Bezugsgröße sein wird. Interessanter sind ja Fragen des
Korpus und der Korpusanalyse: Wie kann ich Fragen an hunderte oder
tausende Artikel stellen?

Generell gehe ich davon aus, dass zukünftige Geschäftsmodelle vorwiegend
über Mehrwertdienste und auf individuelle Bedürfnisse zugeschnittene
Angebote laufen werden. Wahrscheinlich wird das auch bei Sci-Hub eine
Rolle spielen. Das Angebot ist bislang sehr begrenzt. Es gibt nur eine
Abfrage über die DOI. Der nächste Schritt dürfte eine Suchmöglichkeit
über Metadaten sein. Und dann kann man weiter denken in Richtung
semantische Erschließung oder beispielsweise ein Mapping der Daten von
Sci-Hub und des Microsoft Academic Graph.

\emph{\textbf{LIBREAS}: Sci-Hub würde dann also zu aktuellen Trends in
der digitalen Wissenschaft aufschließen?}

Ja. Man muss beobachten was die kommerziellen Akteure tun und überlegen,
was Sci-Hub oder andere Schattenbibliotheken in diesen
bibliothekswissenschaftlichen Entwicklungslinien machen. Die Folgen
werden für viele Disziplinen auch methodologisch spürbar. Es gibt diesen
Trend zur Korpusforschung, aus dem sich neue Annäherungen an klassische
hermeneutische Fragen entwickeln. So kann man das Interesse, das man
traditionell in der Formulierung \enquote{Was will der Autor sagen?}
fasst, über verknüpfte Korpora sehr komplex bedienen. Es ist nun
möglich, die Originaltexte als Forschungsobjekte mit einer Unmenge an
Kontext anzureichern und sie darin zu positionieren. So kann man darüber
forschen, wie Leser*innen Werke zu einem bestimmten Zeitpunkt
interpretieren konnten -- also auch hypothetische Interpretationen
vornehmen. Voraussetzung ist, dass man die Korpora hat. Da geht es nicht
mehr nur um wissenschaftliche Publikationen sondern auch um Korpora wie
Zeitungsarchive.

Im Ergebnis würden wir ein ganz neue Art von Wissenschaft sehen, mit
einer veränderten Argumentationspraxis, neuen Entdeckungs- und
Deutungslogiken. Die Frage ist dann, wie sich die Darstellungsformen und
die Durchsetzbarkeit von Geltungsansprüchen entwickelt. Ich gehe davon
aus, dass diese Art Wissenschaft weniger narrativistisch geprägt sein
wird.

\emph{\textbf{LIBREAS}: Aber benötigen wir dann nicht auch eine neue
Kritik dieser digitalen, post-narrativistischen Wissenschaft? Und werden
die Infrastrukturen auch aus Gründen der Komplexitätsreduktion weniger
Literatur vermitteln und mehr ein, wenn man so will,
Assoziationsmanagement betreiben müssen, nicht zuletzt mit dem Ziel
einer Komplexitätsreduktion?}

Auf jedem Fall. Ich sehe dahingehend aber auch viel Potential für
semantische Technologien. Das ist ein zentraler nächster Schritt. Wir
müssen, möglichst automatisiert, daran arbeiten, dass wir die
PDF-Strukturen auslesen und auflösen können. Dafür gibt es schon
Analysesoftware. Diese funktioniert für STEM-Publikationen recht gut, da
diese traditionell sehr konventionalisiert und formalisiert sind. Bei
anderen Textsorten ist es noch sehr schwierig.

\emph{\textbf{LIBREAS}: Wenn Sci-Hub oder auch Spotify-artige Formen der
Zugangsvermittlung den Bestand an wissenschaftlichen Publikationen
verfügbar machen, vielleicht sogar angereichert mit elaborierten
Retrieval-Formen, geraten wissenschaftliche Bibliotheken möglicherweise
unter einen großen Legitimationsdruck. Man benötigte sie ja nicht mehr
für die wissenschaftliche Literaturversorgung. Was bleibt für sie als
Ausblick?}

Das ist wirklich eine interessante Frage. Es ist durchaus vorstellbar,
dass Sci-Hub in absehbarer Zeit auch den gesamten Google-Books-Korpus
vermittelt, wobei es Google selbst {[}wie oben genannt{]} mittlerweile
relativ gleichgültig zu sein scheint, was mit diesem Bestand geschieht.
Über eine verteilte Struktur lassen sich die Volltexte vergleichsweise
leicht harvesten. Wenn Bibliotheken darauf reagieren wollen, müssen sie
vermutlich vor allem erst einmal genauer wissen, was ihre Zielgruppe,
nämlich die Wissenschaftler*innen wollen. Welche Erwartungen sind zu
bedienen? Wie kuratieren Wissenschaftler*innen ihre eigenen Bibliotheken
und Literatursammlung? Die Bibliothekswissenschaft kann dies
ethnografisch untersuchen. Mit diesem Wissen können Bibliotheken
beispielsweise Dienste für personalisierte Sammlungen und individuelle
Forschungskorpora entwickeln. Sicher sind auch Angebote wie
Cloud-Dienste sinnvoll. Ob man noch Bibliothekskataloge braucht, ist
dagegen fraglich. Sie werden schon jetzt wenig genutzt. Wenn man sich
von der Idee des \enquote{Bestands} gelöst hat, sind sie eigentlich
obsolet. Spannender wären dann Erschließungsmöglichkeiten, die auf
Serendipity abzielen. Und natürlich auch semantische Erschließungen.

Das beste Zukunftsszenerio wäre jedoch dasjenige, in dem Angebote wie
Sci-Hub nicht mehr gebraucht werden und Sci-Hub wird es auch nur genau
so lange geben.

\emph{\textbf{LIBREAS:} Wir danken Dir für das Gespräch!}

\hypertarget{quellen}{%
\section*{Quellen}\label{quellen}}

Barok, D. et al.~In Solidarität mit Library Genesis und Sci-Hub.
Translated by Paul Feigelfeld. 30.11.2015.
\url{http://custodians.online/german.html}.

Bohannon, J. Who's downloading pirated papers? Everyone. 2016.
\url{http://www.sciencemag.org/news/2016/04/whos-downloading-pirated-papers-everyone}

Ernesto. US Court Grants ISPs and Search Engine Blockade of Sci-Hub.
06.11.2017.\url{https://torrentfreak.com/us-court-grants-isps-and-search-engine-blockade-of-sci-hub-171106/}

Greshake B. Looking into Pandora's Box: The Content of Sci-Hub and its
Usage. F1000Research 2017, 6:541 (doi: 10.12688/f1000research.11366.1)

Holcombe, A. Alex Holcombe's Blog. Scholarly publisher profit update.
21.05.2015.
\url{https://alexholcombe.wordpress.com/2015/05/21/scholarly-publisher-profit-update/}

Scherschel, F. Guerilla-Forschungsbibliothek: US-Gericht fällt hartes
Urteil gegen Sci-Hub. 08.11.2017.
\url{https://www.heise.de/newsticker/meldung/Guerilla-Forschungsbibliothek-US-Gericht-faellt-hartes-Urteil-gegen-Sci-Hub-3884267.html}

Strecker, D. Schattenbibliotheken: Ein Krisensymptom der Wissenschaft.
11.08.2017.
\url{https://irights.info/artikel/schattenbibliotheken-ein-krisensymptom-der-wissenschaft/28663}

UWM Libraries. \url{http://guides.library.uwm.edu/scihub}

Van Noorden, R. ALEXANDRA ELBAKYAN: Paper pirate. The founder of an
illegal hub for paywalled papers has attracted litigation and acclaim.
19.12.2016.
\url{http://www.nature.com/news/nature-s-10-1.21157\#/elbakyan}

%autor
\begin{center}\rule{0.5\linewidth}{\linethickness}\end{center}

\textbf{Veil Seer}, Pseudonym.

\textbf{Linda Freyberg}, Doktorandin am Promotionskolleg Wissenskulturen
/ Digitale Medien der Leuphana Universität Lüneburg, Stipendiatin im
Rahmen des Professorinnenprogrammes am Urban Complexity Lab (FH Potsdam)
und Redakteurin der LIBREAS.Library Ideas. ORCID:
\url{https://orcid.org/0000-0002-4620-7571}.

\textbf{Ben Kaden} ist wissenschaftlicher Mitarbeiter an der
Universitätsbibliothek der Hum\-boldt-Uni\-ver\-si\-tät zu Berlin. Er ist
Mitbegründer und -herausgeber von LIBREAS. Library Ideas.

\end{document}
