Ende der 1990er waren die Bibliotheken der Stadt Hume bei Melbourne,
Australien, unter den ersten, die Lernort, Bildungskooperationen und
Stadtentwicklung in einem Konzept und zum Wohle der Bürger gedacht
haben. Heute ist Hume unter den „Lernenden Städten'' der UNESCO oder den
lokalen Bildungsnetzwerken nicht mehr allein. Solche lokalen Netzwerke
wollen die Stärken regionaler Bildungsakteure bündeln, um den
Bedürftigsten der Bevölkerung durch Lebenslanges Lernen eine neue
Perspektive zu geben. In Hume hat das funktioniert, auch wenn es ein
langer, herausfordernder Weg war. Jene Herausforderungen werden im
deutschsprachigen Raum für non-formelle Bildungsanbieter, wie
Bibliotheken, kaum zusammenhängend diskutiert. Das Beispiel Hume soll
sechs praktische Erfolgsfaktoren aufzeigen, um diese Lücke zu füllen.
