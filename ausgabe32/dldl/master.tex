\documentclass[a4paper,
fontsize=11pt,
%headings=small,
oneside,
numbers=noperiodatend,
parskip=half-,
bibliography=totoc,
final
]{scrartcl}

\usepackage{synttree}
\usepackage{graphicx}
\setkeys{Gin}{width=.4\textwidth} %default pics size

\graphicspath{{./plots/}}
\usepackage[ngerman]{babel}
\usepackage[T1]{fontenc}
%\usepackage{amsmath}
\usepackage[utf8x]{inputenc}
\usepackage [hyphens]{url}
\usepackage{booktabs} 
\usepackage[left=2.4cm,right=2.4cm,top=2.3cm,bottom=2cm,includeheadfoot]{geometry}
\usepackage{eurosym}
\usepackage{multirow}
\usepackage[ngerman]{varioref}
\setcapindent{1em}
\renewcommand{\labelitemi}{--}
\usepackage{paralist}
\usepackage{pdfpages}
\usepackage{lscape}
\usepackage{float}
\usepackage{acronym}
\usepackage{eurosym}
\usepackage[babel]{csquotes}
\usepackage{longtable,lscape}
\usepackage{mathpazo}
\usepackage[normalem]{ulem} %emphasize weiterhin kursiv
\usepackage[flushmargin,ragged]{footmisc} % left align footnote
\usepackage{ccicons} 

%%%% fancy LIBREAS URL color 
\usepackage{xcolor}
\definecolor{libreas}{RGB}{112,0,0}

\usepackage{listings}

\urlstyle{same}  % don't use monospace font for urls

\usepackage[fleqn]{amsmath}

%adjust fontsize for part

\usepackage{sectsty}
\partfont{\large}

%Das BibTeX-Zeichen mit \BibTeX setzen:
\def\symbol#1{\char #1\relax}
\def\bsl{{\tt\symbol{'134}}}
\def\BibTeX{{\rm B\kern-.05em{\sc i\kern-.025em b}\kern-.08em
    T\kern-.1667em\lower.7ex\hbox{E}\kern-.125emX}}

\usepackage{fancyhdr}
\fancyhf{}
\pagestyle{fancyplain}
\fancyhead[R]{\thepage}

% make sure bookmarks are created eventough sections are not numbered!
% uncommend if sections are numbered (bookmarks created by default)
\makeatletter
\renewcommand\@seccntformat[1]{}
\makeatother


\usepackage{hyperxmp}
\usepackage[colorlinks, linkcolor=black,citecolor=black, urlcolor=libreas,
breaklinks= true,bookmarks=true,bookmarksopen=true]{hyperref}
%URLs hart brechen
\makeatletter 
\g@addto@macro\UrlBreaks{ 
  \do\a\do\b\do\c\do\d\do\e\do\f\do\g\do\h\do\i\do\j 
  \do\k\do\l\do\m\do\n\do\o\do\p\do\q\do\r\do\s\do\t 
  \do\u\do\v\do\w\do\x\do\y\do\z\do\&\do\1\do\2\do\3 
  \do\4\do\5\do\6\do\7\do\8\do\9\do\0} 
% \def\do@url@hyp{\do\-} 
\makeatother 

%meta
%meta

\fancyhead[L]{Redaktion LIBREAS \\ %author
LIBREAS. Library Ideas, 32 (2017). % journal, issue, volume.
\href{http://nbn-resolving.de/}
{}} % urn 
% recommended use
%\href{http://nbn-resolving.de/}{\color{black}{urn:nbn:de...}}
\fancyhead[R]{\thepage} %page number
\fancyfoot[L] {\ccLogo \ccAttribution\ \href{https://creativecommons.org/licenses/by/3.0/}{\color{black}Creative Commons BY 3.0}}  %licence
\fancyfoot[R] {ISSN: 1860-7950}

\title{\LARGE{Das liest die LIBREAS, Nummer \#1 (Sommer / Herbst 2017)}} % title
\author{Redaktion LIBREAS} % author

\setcounter{page}{1}

\hypersetup{%
      pdftitle={Das liest die LIBREAS, Nummer \#1 (Sommer / Herbst 2017)},
      pdfauthor={Redaktion LIBREAS},
      pdfcopyright={CC BY 3.0 Unported},
      pdfsubject={LIBREAS. Library Ideas, 32 (2017).},
      pdfkeywords={Open Access},
      pdflicenseurl={https://creativecommons.org/licenses/by/3.0/},
      pdfcontacturl={http://libreas.eu},
      baseurl={http://libreas.eu},
      pdflang={de},
      pdfmetalang={de}
     }



\date{}
\begin{document}

\maketitle
\thispagestyle{fancyplain} 

%abstracts

%body
Beiträge von Ben Kaden (bk), Karsten Schuldt (ks), Alexander Struck
(as), Michaela Voigt (mv)

\hypertarget{zur-kolumne}{%
\section*{Zur Kolumne}\label{zur-kolumne}}

Ziel dieser Kolumne ist es, eine Übersicht zu geben über die in der
letzten Zeit erschienene bibliothekarische, informations- und
bibliothekswissenschaftliche sowie für diesen Bereich interessante
Literatur. Enthalten sind Beiträge, die der LIBREAS-Redaktion oder
anderen Beitragenden als relevant erschienen.

Themenvielfalt sowie ein Nebeneinander von wissenschaftlichen und
nicht-wissenschaftlichen Ansätzen wird angestrebt und auch in der Form
sollen traditionelle Publikationen ebenso erwähnt werden wie
Blogbeiträge oder Videos beziehungsweise TV-Beiträge.

Gerne gesehen sind Hinweise auf erschienene Literatur oder Beiträge in
anderen Formaten, diese bitte an die Redaktion richten. (Siehe
Impressum: \url{http://libreas.eu/about/}), Mailkontakt für diese
Kolumne ist
\href{mailto:zeitschriftenschau@libreas.eu}{\nolinkurl{zeitschriftenschau@libreas.eu}}.)
Die Koordination der Kolumne liegt bei Karsten Schuldt, verantwortlich
für die Inhalte sind die jeweiligen Beitragenden. Die Kolumne
unterstützt den Vereinszweck des LIBREAS-Vereins zur Förderung der
bibliotheks- und informationswissenschaftlichen Kommunikation.

\hypertarget{artikel-und-zeitschriftenausgaben}{%
\section*{Artikel und
Zeitschriftenausgaben}\label{artikel-und-zeitschriftenausgaben}}

Der Schwerpunkt \enquote{Metadata} der \emph{Mitteilungen der
Vereinigung Österreichischer Bibliothekarinnen \& Bibliothekare} (70
(2017) 2, \url{https://ojs.univie.ac.at/index.php/voebm/issue/view/178}
vereinigt die Vorträge eines Workshops im Rahmen des
Infrastrukturprojektes e-infrastructures. In diesen werden einige
Projekte aus Österreich vorgestellt, aber vor allem noch einmal
grundsätzliche Überlegungen zum Metadatenmanagement an
Hochschulbibliotheken dargelegt. (ks)

Das Sammeln, Erschliessen und Anbieten von Medien in Bibliotheken und
Archiven wird allgemein als ethisch problemlos verstanden, obwohl es das
selbstverständlich auch nicht ist. In Sammlungen sind Machtstrukturen
und Geschichte eingeschrieben, Kataloge und Infrastruktur sind nicht
objektiv. Solche Aussagen sind manchmal nur schwer fassbar, der
Schwerpunkt der \emph{Collection Development} \enquote{Sharing Knowledge
and Smashing Stereotypes: Representing Native American, First Nation,
and Indigenous Realities in Library Collections} (42 (2017) 3-4,
\url{http://www.tandfonline.com/toc/wcol20/42/3-4?nav=tocList}
{[}größtenteils Paywall{]}), welcher sich auf Sammlungen über (weniger,
das ist die Kritik, mit) American Natives / First Nations in den USA und
Kanada konzentriert, macht diese Strukturen greifbarer. In diesen
Sammlung treffen nicht nur koloniale Geschichte und Forderungen nach
De-Kolonialisierung, sondern auch unterschiedliche Wissensansprüche (zum
Beispiel der wissenschaftliche Drang, alles darzustellen im Gegensatz
zum Anspruch, bestimmtes Wissen über Rituale abgeschirmt zu halten)
aufeinander. (ks)

Eine bemerkenswerte ethnologische Studie zur Praxis der
Vorleseveranstaltungen in (schwedischen) Öffentlichen Bibliotheken
erschien in der \emph{New Review of Children's Literature and
Librarianship}. (Åse Hedemark: Telling Tales. An Observational Study of
Storytelling for Children in Swedish Public Libraries. In: \emph{New
Review of Children's Literature and Librarianship} 23 (2017) 02,
106-125, \url{https://doi.org/10.1080/13614541.2017.1367574}
{[}Paywall{]}) Grundsätzlich versuchen die Bibliothekarinnen und
Bibliothekare in diesen Veranstaltungen ein vorgängig festgelegtes
Programm durchzuziehen, welches auch mit vielen, oft zu vielen,
pädagogischen Zielen verknüpft sei. Gleichzeitig gibt es von ihnen kaum
ernsthafte Kommunikation mit den Kindern. Es werden viele rhetorische
Fragen gestellt, zumeist zum vorgelesenen Text und den gezeigten
Bildern, aber auf andere Anregungen und Verhaltensweise der Kinder wird
nicht eingegangen. Die Kinder werden, so die Autorin, selten als
Partizipierende wahrgenommen; auch wenn sie dies einfordern. (ks)

In \enquote{Studying the Night Shift} (Schwieder, David ; Spears, Lauras
I.: Studying the Night Shift: A Multi-method Analysis of Overnight
Academic Library Users. In: \emph{Evidence Based Library and Information
Practice} 12 (2017) 3, \url{https://doi.org/10.18438/B8BM1F}) berichten
David Schwieder und Laura I. Spears von einer Untersuchung der
Interessen von Studierenden, welche an einer US-amerikanischen
Universitätsbibliothek die 24/7-Öffnungszeiten nutzen, um die Nächte
durchzuarbeiten. Im Ergebnis bestätigen Sie, was auch in anderen Ländern
und Bibliotheken vermutet wird: Dass zumindest Studierende vor allem
Interesse am Raum Bibliothek und der gebotenen Infrastruktur haben, aber
kaum an bibliothekarischer Beratung oder am Kontakt mit dem
Bibliothekspersonal. (ks)

Ein überraschend negatives Bild zeichnet Kaetrena Davis Kendrick (Davis
Kendrick, Kaetrena: The Low Morals Experience of Academic Librarians: A
Phenomenological Study. In: \emph{Journal of Library Administration} 57
(2017), 846-878, \url{https://doi.org/10.1080/01930826.2017.1368325}
{[}Paywall{]}) in einer Untersuchung über die Wissenschaftlichen
Bibliotheken (in den USA) als Arbeitsplatz, an dem eine Arbeitskultur
herrscht, in denen das Personal beleidigt, unterbewertet oder auch
psychologisch beschädigt wird. Davis Kendrick beschreibt dabei, auf der
Basis von Interviews, wie diese negative Kultur auf das Personal wirkt.
(ks)

Wissenschaftliche Bibliotheken streben immer wieder an, direkt mit
Forschenden in Kontakt zu kommen, um diese zu beraten und direkt bei
ihrer Arbeit unterstützen zu können. Ein Ausdruck davon sind
\enquote{Library Liaison Programs}, die mit großem Personalaufwand
betrieben werden. Laura Banfield und Jo-Anne Petropoulos bieten in einem
Artikel, der eigentlich eine spezifische Bibliothek beschreiben soll,
eine umfassende Übersicht zu verschiedenen Modellen dieser Liaison
Programme. (Banfield, Laura ; Petropoulos, Jo-Anne: Re-visioning a
Library Liaison Program in Light of External Forces and Internal
Pressures. In: \emph{Journal of Library Administration} 57 (2017),
827-845, \url{https://doi.org/10.1080/01930826.2017.1367250}
{[}Paywall{]}) (ks)

In einer recht umfangreichen Studie (n=1000) untersuchten Dirk
Lewandowski, Friederike Kerkmann, Sandra Rümmele und Sebastian Sünkler
die Kompetenz deutscher Internetnutzer, Anzeigen auf Google als solche
zu erkennen und von den eigentlichen Suchergebnissen zu unterscheiden.
Dies gelang nur einem sehr geringen Teil von Proband*innen vollständig.
Darüber hinaus ermittelte die Forschungsgruppe, dass bei den
Proband*innen nur ein begrenztes Wissen zum Geschäftsmodell der
Suchmaschine vorliegt. Die Autor*innen schlussfolgern, dass die
Kennzeichnung der Anzeigen auf Google nicht ausreichend ist.

(Lewandowski, Dirk ; Kerkmann, Friederike; Rümmele, Sandra: An empirical
investigation on search engine ad disclosure. In: \emph{JASIST}, Early
View. 10.1002/asi.23963, \url{https://doi.org/10.1002/asi.23963}
{[}Paywall{]}
{[}OA-Version{]}(\url{https://arxiv.org/abs/1710.08389}){]} (bk)

\enquote{Die Datenstelle ermöglicht den Übergang von der Spekulation zur
faktenbasierten Handlung.} -- so fasst Bernhard Mittermaier die
Zielsetzung des \enquote{Nationale{[}n{]} Kontaktpunkt Open Access}
(NOAK) zusammen. Das Interview führte Konstanze Söllner und thematisiert
neben der Entstehungsgeschichte die künftigen Schwerpunkte und Aufgaben
des NOAK. (Mittermaier, Bernhard: Datenarbeit und \enquote{Nationaler
Kontaktpunkt Open Access} -- ein Interview mit Dr.~Bernhard Mittermaier.
In: \emph{ABI Technik} 37 (2017) 4, S. 293--296.
{[}\url{https://doi.org/10.1515/abitech-2017-0062}) (mv)

Andreas Ledl hat verschiedene Open-Access-Zeitschriften (unter anderem
aus dem deutschsprachigen LIS-Bereich) untersucht und gibt einen
Überblick darüber, welche Plattformen und Werkzeuge diese Journale für
Hosting, Workflowmanagement und Satz nutzen. Es wird deutlich, dass OJS
zwar der Platzhirsch auf dem Markt der Open-Source-Software für
Zeitschriftenhosting ist, aber bei weitem nicht die einzige in der
Praxis verwendet Lösung. Für Interessierte, die selbst Workflows für
Zeitschriften aufbauen bzw. optimieren wollen, liefert der Beitrag
zahlreiche praktische Hinweise -- in Bezug auf Plattform, Hosting, Satz
sowie Indexierung. (Ledl, Andreas: Software, Server, Suchmaschine --
Technische Kriterien der Gründung und des Betriebs von (Diamond) Open
Access-Zeitschriften. In: \emph{ABI Technik} 37 (2017) 4, S. 30--38.
\url{https://doi.org/10.1515/abitech-2017-0005} {[}Paywall{]}
{[}OA-Version: (\url{http://edoc.unibas.ch/54846/}){]}) (mv)

\hypertarget{monographien}{%
\section*{Monographien}\label{monographien}}

In einer Abschlussarbeit, die letztlich auch als Monographie in der
Reihe Leipziger Arbeiten zur Bibliotheks- und Informationswissenschaft
publiziert wurde, stellt Christian Schmidt anhand einer Umfrage klar,
dass der Tausch von gedruckten Schriften für Wissenschaftliche
Spezialbibliotheken weiterhin Relevanz hat, zumeist zwar mit abnehmender
Tendenz, teilweise aber auch zunehmend. Der digitale Wandel sei nur ein
Grund, dass sich der Schriftentausch ändere. Budget- und Personalabbau
seien ebenso relevant. (Schmidt, Christian: \emph{Schriftentausch und
Digitaler Wandel: Eine empirische Untersuchung am Beispiel
wissenschaftlicher Spezialbibliotheken}. Berlin: BibSpider, 2017) (ks)

Offensichtlich für eine breite Öffentlichkeit, und nicht die
Fachcommunity, geschrieben, ist die an Illustrationen reiche Monographie
zur Geschichte der Bibliothek des Deutschen Museums in München, die ihr
Leiter Helmut Hilz im Verlag des Museums vorgelegt hat. (Hilz, Helmut:
\emph{Die Bibliothek des Deutschen Museums.} München: Deutsches Museum
Verlag, 2017) Das im Coffee Table Book Format gehaltene Buch gibt eine
-- teilweise sehr langatmige, weil auf den Akten von Sitzungen und
Verwaltungsvorgängen basierende -- Übersicht zur Entwicklung der
Bibliothek, zu ausgewählten Beständen und im Bestand enthaltenen Werken.
Der Gründungszeit wird weit mehr Platz eingeräumt als den späteren
Jahrzehnten bis heute. Dennoch ein sehr schönes Buch. (ks)

Aus einer selbstorganisierten Lerngruppe von Bibliothekarinnen und
Bibliothekaren ging eine Monographie zur Nutzung von Autoethnographie in
der Bibliothekspraxis und -forschung hervor. (Deitering, Anne-Marie ;
Schroeder, Robert ; Stoddart, Richard (Hrsg.): \emph{The Self as
Subject: Autoethnographic Research into Identity, Culture, and Academic
Librarianship}. Chicago: Association of College and Research Libraries,
2017) Das Buch umfasst, neben einer Einführung und einer Reflektion am
Ende, 15 Kapitel, die jeweils autoethnographische Forschungen in
(US-amerikanischen und kanadischen) Wissenschaftlichen Bibliotheken
präsentieren. Obwohl nicht alle Themen auf Bibliotheken im DACH-Raum zu
übertragen sind (was auch der Methode widersprechen würde), zeigt dieses
Sammlung die Stärken solcher selbst-reflektiven Forschungen auf.
Nachahmenswert ist aber vor allem der Ansatz, dass Kolleginnen und
Kollegen selbsttätig Lerngruppen organisieren und in ihnen gemeinsam
Lernprojekte durchführen -- um bessere Menschen und Bibliothekarinnen,
Bibliothekare zu werden. (ks)

\hypertarget{social-media}{%
\section*{Social Media}\label{social-media}}

Elsevier hat bepress gekauft, den Hersteller der Repositorienplattform
Digital Commons, welche vor allem in Nordamerika für den Betrieb
institutioneller Repositorien eingesetzt wird. Die Bibliothek der
University of Pennsylvania hat angekündigt, die Partnerschaft mit
bepress zu beenden und berichtet nun über den Umstieg zu einer neuen
Repositorienlösung im eigens gestarteten Blog und Twitter-Account:
\url{https://beprexit.wordpress.com/} und
\url{https://twitter.com/beprexit/} (mv)

Hinweis von @oa\_intact:
\url{https://twitter.com/oa_intact/status/922790669013389312}: Welches
Land hat den höchsten Gold-OA-Anteil weltweit? Brasilien führt vor
Serbien und Pakistan; Nigeria ist in den Top 10. Weitere überraschende
und weniger überraschende Zahlen finden sich in einer im Oktober 2017
veröffentlichten Studie,
\url{http://nbn-resolving.de/urn:nbn:de:0070-pub-29128079}. (mv)

Was schreibt man auf die letzte Folie einer Vortragspräsentation? Eine
klare Meinung zu dieser Frage hat @AndreasZeller
\url{https://twitter.com/AndreasZeller/status/926116900651896835}:
\enquote{Why final "Thank you" slides drive me nuts}
\url{https://andreas-zeller.blogspot.de/2013/10/summarizing-your-presentation-with.html}
(mv)

Ablehnung eines Artikels in einer Elsevier-Zeitschrift durch den Journal
Editor nach 39 Minuten, das erlebte @ManuelPorcar1
\url{https://twitter.com/ManuelPorcar1/status/928251836406083584} und
vermutet dahinter ein Geschäftsmodell. (mv)

Clone Wars sind nur noch einen Wimpernschlag entfernt: Boston Dynamics
zeigt die Fähigkeiten ihrer aktuellen (Kriegs-)Roboterentwicklung
\url{https://twitter.com/mrmedina/status/931291808394440706}, die zu
informationsethischer Diskussion einladen. (as)

Die Nutzung von Standardbibliotheken bei häufig auftretenden
Programmierproblemen empfiehlt sich. Aber einige solcher
Standardbibliotheken enthalten auch unerwartete Features -- so enthält
etwa libxml2 auch einen FTP-Client, berichtet @moyix
\url{https://twitter.com/moyix/status/932324519427141635}. Jedes Feature
erhöht die Komplexität, was IT-Sicherheit schwerer handhabbar macht.
(as)

Rein in die Organisation und von innen aufmischen? SpringerNature geht
wohl an die Börse, voraussichtlich im Sommer 2018 -- das berichtet
Christian Gutknecht im Gemeinschaftsblog wisspub.net
(\url{https://wisspub.net/2017/11/28/springernature-vor-boersengang/}).
Voraussichtlich wird ein beachtlicher Anteil der Aktien frei an der
Börse zu handeln sein. (mv)

\hypertarget{konferenzen-konferenzberichte}{%
\section*{Konferenzen,
Konferenzberichte}\label{konferenzen-konferenzberichte}}

{[}Diesmal keine Hinweise.{]}

\hypertarget{populuxe4re-medien-zeitungen-radio-tv-etc.}{%
\section*{Populäre Medien (Zeitungen, Radio, TV
etc.)}\label{populuxe4re-medien-zeitungen-radio-tv-etc.}}

Anhand der Ausstellung \enquote{Frank Lloyd Wright at 150: Unpacking the
Archive} im Guggenheim-Museum in New York (12. Juni-01. Oktober 2017)
reflektiert Julian Rose grundsätzliche Aspekte, des Umgangs mit
Archivalien im Museumszusammenhang. Er stellt dabei zunächst (mit Walter
Benjamin) die grundlegende Spannung zwischen Ordnung und Unordnung
heraus und betont, dass Museen bei der Auswahl von Archivialien für die
Präsentation unvermeidlich eine bestimmte Form von Geschichte
konstruieren sowie die prinzipielle Unabschließbarkeit eines Archivs und
seiner Erschließung. Neuigkeit, und damit die Freude am Entdecken, sind
verführerisch, als Auswahlkriterien aber häufig zu eng. Er plädiert
vielmehr für die grundlegende Verschiebung der Archivarbeit von der
Auswahl (\emph{Selection}) zur Deutung (\emph{Analysis}), \enquote{from
material to method}. Nicht die Entdeckung des Unbekannten muss
vorrangiges Ziel sein sondern das Entdecken (beziehungsweise neue
Betrachten) des Bekannten. Anhand des Wright-Archivs und dem
Guggenheim-Museum selbst betrachtet er schließlich Gebäude selbst als
dynamische Medien, in denen sich soziale und politische Effekte sammeln
und lesen lassen. Gebautes wird, wie die Archivalien zu ihnen
verdeutlichen, \enquote{geschrieben}. Während Julian Rose
\enquote{material-to-method}-Verschiebung gut in das Selbstverständnis
der Digital Humanities passen dürfte, ist der Ansatz des Gebäudes
sozio-politisches Konstrukt unmittelbar für das Forschungsfeld
Bibliotheksbau und -architektur relevant. (Julian Rose: Archive Fever.
In: \emph{ARTFORUM}, October 2017, S. 85f.) (bk)

2005 wurde die neue gemeinsame Universitätsbibliothek der Technischen
Universität Berlin und der Universität der Künste Berlin eröffnet. Der
Neubau konnte dank einer Spende des Volkswagen-Konzerns fertiggestellt
werden; die Bibliothek trug fortan dessen Namen. Ein Antrag der Partei
DIE LINKE wirft nun die Diskussion auf, ob die Bibliothek umbenannt
werden sollte. (Thomas Loy: Linke will Volkswagen-Bibliothek umbenennen.
In: \emph{Tagesspiegel},
\url{http://www.tagesspiegel.de/berlin/charlottenburg-wilmersdorf-linke-will-volkswagen-bibliothek-umbenennen/20602480.html},
So 19.11.2017) (mv)

\hypertarget{weitere-medien}{%
\section*{Weitere Medien}\label{weitere-medien}}

Relevant für das Management elektronischer Medien in Bibliotheken ist
selbstverständlich die Veröffentlichung der Version 5 des COUNTER Code
of Practice:
\url{https://www.projectcounter.org/release-5-code-practice/} Die neue
Version versucht die unterschiedlichen Daten zu den unterschiedlichen
Formen elektronischer Medien in möglichst wenigen, dafür flexiblen
Formularen abzubilden. (ks)

\hypertarget{rezensionen}{%
\section*{Rezensionen}\label{rezensionen}}

Frustriert von einer wahrgenommen sinkenden Qualität des
wissenschaftlichen Publizierens und generell einer
\enquote{Kommodifizierung der Autorschaft} spielt der Autor in seiner
Kolumne mit der Idee einer fixen Beschränkung der Gesamtwortzahl pro
Wissenschaftler*in. (Brian C. Martinson: Give reasearchers a lifetime
word limit. In: Nature, 19. October 2017. Volume 550 Number 7676.
\url{https://doi.org/10.1038/550303a} ) Er schildert zahlreiche Vorteile
und wenige Nachteile dieses Ansatzes und übersieht erstaunlicherweise
das totalitäre und dystopische Potential einer solchen Verordnung. (bk /
ausführlicher:
\url{http://libreas.tumblr.com/post/166779373311/word-limits})

Besprechung zur Festschrift für Rafael Capurro. Die Autorin würdigt den
Bibliotheks- und Informationswissenschaftler, umreißt anhand des
Sammelbandes dessen Forschungsfelder und seinen Einfluss und arbeitet
heraus, dass der Band als einführendes Buch in sein Werk vermutlich zu
komplex ist. (Kristene Unsworth (2017): Book Review: Information
Cultures in the Digital Age: A Festschrift in Honor of Rafael Capurro.
Matthew Kelly and Jared Bielby. Wiesbaden, Germany: Springer VS, 2016.
479 pp. \$129.00 (Paperback). (ISBN 978-3-658-14679-5). In: JASIST.
Early View. 17 October 2017. \url{https://doi.org/10.1002/asi.23891}
{[}Paywall{]}) (bk / ausführlicher:
\url{http://libreas.tumblr.com/post/166747888776/libreas-lektueren})

\hypertarget{debatten}{%
\section*{Debatten}\label{debatten}}

Birger Hjørland: Does informetrics need a theory? A rejoinder to
professor Anthony Van Raan. Letter to the Editor. In: JASIST Volume 68,
Issue 12 December 2017 Page 2846.
\url{https://doi.org/10.1002/asi.23964} {[}Paywall{]}

Birger Hjørland annotiert eine Rezension von Anthony van Raan und weist
auf eine mögliche Fehleinschätzung des Rezensenten hin, die besagte,
dass die Informetrie ohne wissenschaftstheoretisches Fundament auskommen
würde. Auslöser war eine Kritik Hjørlands am Web of Science, das als
System der Wissensorganisation laut Hjørland die Sichtbarkeit von
Inhalten und damit Wissen aktiv beeinflusst. Hjørland erhält diese
Position einer fehlenden Neutralität des Web of Science aufrecht,
betont, dass unterschiedliche Ansätze der Informationswissenschaft
verschiedene theoretische Grundlagen haben und plädiert seinerseits für
eine explizite Verständigung über diese theoretischen Grundlagen. (bk)

\hypertarget{vermischtes}{%
\section*{Vermischtes}\label{vermischtes}}

Die kurze, mit zwei Fotografien des Pressefotografen Heinz Schönfeld
illustrierte Redaktionsmeldung, informiert darüber, dass der Neubau der
Stadtbezirksbibliothek Prenzlauer Berg in der Greifswalder Straße 87 am
07. April 1981 eröffnet wurde und montags bis freitags von 10-19 Uhr und
samstags von 09-12 Uhr geöffnet sein wird. Welche Ausnahme
\enquote{außer mittwochs} darstellt, führt die Meldung nicht aus. (ND:
Ein Literaturtreff in der Greifswalder Straße. In: Neues Deutschland,
Mi. 8. April 1981, S.8) (bk)

In der Rubrik Berliner Chronik meldet die Redaktion der Berliner Zeitung
kurz, dass Dr.~Joris Vorstius, Direktor der Katalogabteilung der
Öffentlichen Wissenschaftlichen Bibliothek Berlin, einen Lehrauftrag für
Bibliothekswissenschaft an der Philosophischen Fakultät der
Humboldt-Universität erhielt. (Berliner Zeitung: Lehrauftrag für
Bibliothekswissenschaft. In: Berliner Zeitung, Fr. 14. März 1947, S. 4)
(bk)

%autor

\end{document}
