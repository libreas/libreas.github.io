\documentclass[a4paper,
fontsize=11pt,
%headings=small,
oneside,
numbers=noperiodatend,
parskip=half-,
bibliography=totoc,
final
]{scrartcl}

\usepackage{synttree}
\usepackage{graphicx}
\setkeys{Gin}{width=.4\textwidth} %default pics size

\graphicspath{{./plots/}}
\usepackage[ngerman]{babel}
\usepackage[T1]{fontenc}
%\usepackage{amsmath}
\usepackage[utf8x]{inputenc}
\usepackage [hyphens]{url}
\usepackage{booktabs} 
\usepackage[left=2.4cm,right=2.4cm,top=2.3cm,bottom=2cm,includeheadfoot]{geometry}
\usepackage{eurosym}
\usepackage{multirow}
\usepackage[ngerman]{varioref}
\setcapindent{1em}
\renewcommand{\labelitemi}{--}
\usepackage{paralist}
\usepackage{pdfpages}
\usepackage{lscape}
\usepackage{float}
\usepackage{acronym}
\usepackage{eurosym}
\usepackage[babel]{csquotes}
\usepackage{longtable,lscape}
\usepackage{mathpazo}
\usepackage[normalem]{ulem} %emphasize weiterhin kursiv
\usepackage[flushmargin,ragged]{footmisc} % left align footnote
\usepackage{ccicons} 

%%%% fancy LIBREAS URL color 
\usepackage{xcolor}
\definecolor{libreas}{RGB}{112,0,0}

\usepackage{listings}

\urlstyle{same}  % don't use monospace font for urls

\usepackage[fleqn]{amsmath}

%adjust fontsize for part

\usepackage{sectsty}
\partfont{\large}

%Das BibTeX-Zeichen mit \BibTeX setzen:
\def\symbol#1{\char #1\relax}
\def\bsl{{\tt\symbol{'134}}}
\def\BibTeX{{\rm B\kern-.05em{\sc i\kern-.025em b}\kern-.08em
    T\kern-.1667em\lower.7ex\hbox{E}\kern-.125emX}}

\usepackage{fancyhdr}
\fancyhf{}
\pagestyle{fancyplain}
\fancyhead[R]{\thepage}

% make sure bookmarks are created eventough sections are not numbered!
% uncommend if sections are numbered (bookmarks created by default)
\makeatletter
\renewcommand\@seccntformat[1]{}
\makeatother


\usepackage{hyperxmp}
\usepackage[colorlinks, linkcolor=black,citecolor=black, urlcolor=libreas,
breaklinks= true,bookmarks=true,bookmarksopen=true]{hyperref}
%URLs hart brechen
\makeatletter 
\g@addto@macro\UrlBreaks{ 
  \do\a\do\b\do\c\do\d\do\e\do\f\do\g\do\h\do\i\do\j 
  \do\k\do\l\do\m\do\n\do\o\do\p\do\q\do\r\do\s\do\t 
  \do\u\do\v\do\w\do\x\do\y\do\z\do\&\do\1\do\2\do\3 
  \do\4\do\5\do\6\do\7\do\8\do\9\do\0} 
% \def\do@url@hyp{\do\-} 
\makeatother 

%meta
%meta

\fancyhead[L]{Ch. Gutknecht \\ %author
LIBREAS. Library Ideas, 32 (2017). % journal, issue, volume.
\href{http://nbn-resolving.de/}
{}} % urn 
% recommended use
%\href{http://nbn-resolving.de/}{\color{black}{urn:nbn:de...}}
\fancyhead[R]{\thepage} %page number
\fancyfoot[L] {\ccLogo \ccAttribution\ \href{https://creativecommons.org/licenses/by/3.0/}{\color{black}Creative Commons BY 3.0}}  %licence
\fancyfoot[R] {ISSN: 1860-7950}

\title{\LARGE{Zur Situation von Open Access in der Schweiz:\\ Interview mit Christian Gutknecht}} % title
\author{Christian Gutknecht} % author

\setcounter{page}{1}

\hypersetup{%
      pdftitle={Zur Situation von Open Access in der Schweiz:\\ Interview mit Christian Gutknecht},
      pdfauthor={Christian Gutknecht},
      pdfcopyright={CC BY 3.0 Unported},
      pdfsubject={LIBREAS. Library Ideas, 32 (2017).},
      pdfkeywords={Open Access, OA Gold, Transformation, Schweiz},
      pdflicenseurl={https://creativecommons.org/licenses/by/3.0/},
      pdfcontacturl={http://libreas.eu},
      baseurl={http://libreas.eu},
      pdflang={de},
      pdfmetalang={de}
     }



\date{}
\begin{document}

\maketitle
\thispagestyle{fancyplain} 

%abstracts

%body
Fragen für LIBREAS. Library Ideas: Karsten Schuldt (KS), Michaela Voigt
(MV)

\emph{\textbf{MV}: In der OA-Szene, und vielleicht auch darüber hinaus,
bist du bekannt (und gefeiert) als der, der den Kampf um
Preistransparenz für Zeitschriftendeals mit den Schweizer Universitäten
aufgenommen hat. Für alle, die dich nicht kennen: Was machst du gerade
beruflich? Was sind deine Arbeitsschwerpunkte?}

\textbf{CG:} Ich arbeite beim Schweizer Nationalfonds, dem SNF, in einem
Team, das sich um die Konfiguration und Weiterentwicklung der
Informationssysteme der Forschungsförderung kümmert. Häufig geht es
darum, Anforderungen innerhalb der Geschäftsstelle zu koordinieren und
zusammen mit den Entwicklern auf eine sinnvolle Art in den Systemen
umzusetzen. Open Access ist dabei nur ein kleiner Teil des
Themenspektrums meiner Arbeit, aber natürlich einer, wo ich mich stark
begeistern kann.

\emph{\textbf{KS:} Wie bereits erwähnt, du bist bekannt für deine
Auseinandersetzungen mit den Universitäten in der Schweiz über die Daten
zu deren Verträgen mit den Wissenschaftsverlagen. Kannst du kurz
schildern, was dein Ziel bei diesen Auseinandersetzungen war? Hast du es
erreicht? Würdest du ähnliche Auseinandersetzungen in anderen Ländern
anregen?}

\textbf{CG:} Die Idee die Schweizer Hochschulen offiziell zu ihren
Ausgaben anzufragen, wuchs mit der Frustration, die ich zunehmend in
meiner damaligen Arbeit an der Universität Bern empfand. Auf der einen
Seite forderten wir Forschende gemäss der verabschiedeten OA-Policy dazu
auf, nicht in klassischen Subskriptionszeitschriften zu publizieren,
sondern OA-Zeitschriften zu wählen. Die dafür nötigen Gelder wollten
oder konnten die Universitätsbibliotheken dann aber nicht oder nur
beschränkt zur Verfügung stellen. Gleichzeitig bezahlte man Jahr für
Jahr nahezu unbestritten immer höhere Subskriptionskosten für genau die
Verlage, welche sich offenkundig gegen OA positionierten. Fast niemand
in der Bibliotheksszene schien dies überhaupt als Widerspruch
wahrzunehmen.

Im Kontakt mit Forschenden zeigte sich auch, dass diese häufig keine
Vorstellung hatten, zu welchen Bedingungen die Bibliothek Literatur
beschaffte, so dass der absurde Status Quo von ihnen häufig als nicht
problematisch empfunden wurde. Noch während meiner Anstellung an der
Universitätsbibliothek Bern erstellte ich deshalb zusammen mit einer
Kollegin von der Erwerbungsabteilung eine Statistik zu den Ausgaben der
Universität an Elsevier und veröffentliche sie -- mit Zustimmung der
Abteilungsleiterin -- auf dem öffentlichen OA-Blog der Bibliothek.
Fortan verwies ich bei meiner Arbeit zur Sensibilisierung zu OA gerne
auf diese Zahlen, was durchaus einige erstaunte Reaktionen auslöste.

Einige Monate später, als ich bereits nicht mehr bei der Universität
Bern arbeitete, wurde die Leiterin der Geschäftsstelle des Schweizer
Konsortiums auf den Blogbeitrag aufmerksam und schrieb mir in einem
erzürnten E-Mail, dass ich diese Zahlen weder haben noch veröffentlichen
dürfte, da es Vertragsbruch sei und die Position gegenüber Elsevier
schwächen würde. Anstatt mich in einer langen E-Mail zu erklären,
entschied ich mich, mit privaten Anfragen an alle Schweizer
Hochschulbibliotheken zu deren Ausgaben an Elsevier, Wiley und Springer
zu überprüfen, ob die Praxis von Geheimhaltungsklauseln tatsächlich mit
dem Öffentlichkeitsgesetz vereinbar ist. Damals musste gerade die
Universität Zürich ihren Sponsoring-Vertrag mit der UBS
{[}Schweizerische Grossbank, MV/KS{]} offenlegen, weshalb ich auch auf
das Öffentlichkeitsgesetz aufmerksam wurde.

Bis auf die Universität der italienischen Schweiz wollte anfangs keine
Bibliothek ihre Ausgaben offenlegen. Viele Universitäten und deren
Bibliotheken argumentierten, dass für sie das Öffentlichkeitsprinzip
nicht gälte oder gerade die Ausgaben an die Verlage als
Geschäftsgeheimnis davon ausgenommen wären. In teils langen und
kostspieligen Rekursverfahren {[}Beschwerdeverfahren MV/KS{]} konnte ich
das mehrheitlich widerlegen. In bisher sieben von acht Fällen kamen
unabhängige Rekursinstanzen zum Schluss, dass Geheimhaltungsklauseln
zwischen Bibliothek und Verlag das Öffentlichkeitsprinzip nicht
grundsätzlich aushebeln können und die Zahlen offenzulegen sind. Nur bei
der Universität Basel entschieden die Gerichte anders. Zwei Verfahren im
Kanton St.~Gallen laufen noch.

Trotz dieser ersten juristischen Erfolge ist jedoch bei einigen
Institutionen immer noch kein Umdenken zu erkennen. Weitergehende
Transparenz wird nun neu mit absurden Gebührenforderungen verhindert. So
verlangen die ETH-Zürich und die Universitäten Zürich und St.~Gallen nun
über 11'000 Franken für ihre Ausgaben an die grössten 13 Verlage,
natürlich wissend, dass dies für mich als Privatperson nicht zu stemmen
ist. Wahrscheinlich werde ich da juristisch nicht weiterkommen, da die
Gesetze solche Gebührenforderungen noch zulassen. Doch ich bin stark
gewillt, für eine Änderung dieser Gesetze zu einzustehen, da mein Fall
die Absurdität dieser Gebührenforderungen besonders gut aufzeigt. Hier
gibt es also noch viel zu tun.

Allerdings bin ich optimistisch, dass es mir bereits gelungen ist, auf
die Diskrepanz zwischen Anspruch und Tun von wissenschaftlichen
Bibliotheken aufmerksam zu machen. Ein Indiz dazu ist sicherlich auch
die Anfrage einer Nationalrätin an den Schweizer Bundesrat, zu der Höhe
der Subskriptionsausgaben und den geplanten Massnahmen im Bereich
OA.\footnote{\url{https://www.parlament.ch/de/ratsbetrieb/suche-curia-vista/geschaeft?AffairId=20163930}}

International scheint sich zudem auch langsam ein grösseres Bewusstsein
für die Wichtigkeit von Transparenz zu bilden. So fordert nun auch LIBER
in ihren fünf Prinzipien für Verlagsverhandlungen\footnote{\url{http://libereurope.eu/blog/2017/09/07/open-access-five-principles-negotiations-publishers/}}
ganz explizit den Verzicht auf Non-Disclosure-Agreements und gar die
ganze Offenlegung von Verlagsagreements.

Dies ist eine nicht zu unterschätzende Entwicklung. Zu denken, es geht
auch ohne Transparenz, da ja BibliotheksdirektorInnen sich informell
untereinander absprechen, ist reichlich naiv. Die Dokumente zum Konflikt
der ETH Zürich und der EPF Lausanne von 2010 bis 2012 über die Schweizer
Elsevier-Lizenz,\footnote{\url{https://wisspub.net/2016/01/14/der-eth-bereich-und-elsevier-teil-1/}}
zeigen enthüllend, wie wenig Informationen Bibliotheken wirklich für
gute Verhandlungen zu Verfügung haben und auf welche Weise
Non-Disclosure-Agreements problematisch sind.

Ich finde es zudem sehr irritierend, dass wir zwar seit über zehn Jahren
zu OA debattieren, aber nach wie vor wenig solide Daten zu den heutigen
Ausgaben haben, auf die wir uns bei der Debatte stützen können. Von
daher wäre es natürlich begrüssenswert, wenn auch in anderen Ländern
Transparenzinitiativen gestartet werden.

\emph{\textbf{KS:} Du hast sehr viel Hoffnung auf diese Analyse der
Finanzströme in der Wissenschaftskommunkation in der Schweiz gelegt. Die
ist jetzt seit einiger Zeit öffentlich.}\footnote{\emph{Cambridge
  Economic Policy Associates Ltd (2016/2017): Financial Flows in Swiss
  Publishing. Final Report,} \url{https://doi.org/10.5281/zenodo.240896}}\emph{Was
sind deiner Meinung nach die wichtigsten Ergebnisse dieser Studie?}

\textbf{CG:} Dass in der Schweiz etwa 70 Millionen Franken für
Subskriptionen pro Jahr ausgegeben werden und nur ein Zehntel davon für
OA. Die Studie liefert somit indirekt die Lösung zu OA. Will man OA,
müssen die Hochschulen einfach das Geld aus den Subskriptionen nehmen
und zu OA verlagern. Es wäre eigentlich ganz einfach. Wie die Zahlen aus
einer begleitenden Analyse des Schweizer
Publikationsverhaltens\footnote{\url{https://doi.org/10.5281/zenodo.167381}}
zeigen, wird eine solche Verlagerung für die Schweiz im Bereich der
Zeitschriften mit grossen Kosteneinsparungen verbunden sein.

\emph{\textbf{KS:} Hat die Studie Einfluss gehabt auf die
Wissenschaftskommunikation in der Schweiz? Oder auf die
Förderungspolitik des SNF?}

\textbf{CG:} Ich denke schon, dass die erstmalige Ermittlung dieser 70
Millionen Franken Subskriptionsausgaben an wichtigen
wissenschaftspolitischen Stellen das Problembewusstsein erhöht hat. 70
Millionen Franken sind im ganzen nationalen Wissenschaftsbetrieb zwar
nicht viel, aber auch nicht so wenig, dass man sie weiterhin ohne
Beachtung unter den Tisch kehren kann.

Was die weiteren Resultate der Studie betrifft, dürfte der Einfluss
beschränkt geblieben sein. Leider ist die Studie zu sehr in einem Geist
geschrieben, der mehr die bisherigen hohen Ausgaben und Passivität in
der Schweiz zu rechtfertigen versucht, als ernstzunehmende Lösungswege
aufzeigt. So verpasste es die Studie gerade dort in die Tiefe zu gehen,
wo es wirklich spannend geworden wäre. So wurden zum Beispiel die 70
Millionen Franken Subskriptionskosten nicht weiter nach Hochschule oder
nach Verlag aufgeschlüsselt. Diese Information wäre aber fundamental, um
die offene Frage nach den finanziellen \enquote{Gewinnern} und
\enquote{Verlierern} der Transformation zu beantworten. In allen
Transformationsszenarien wurde zudem davon ausgegangen, dass keine
Journals abbestellt werden, bevor nicht die restliche Welt OA
implementiert hat. Auf das Durchrechnen eines Transformationsszenarios
mit Offsetting wurde gar verzichtet, weil man dies aufgrund eines
falschen Verständnisses von Offsetting als nicht lohnend abgeschrieben
hatte.

Auf die Förderpolitik des SNF hatte die Studie meines Wissens keinen
direkten Einfluss. Der SNF verlangt weiterhin Green OA und fördert Gold
OA. Natürlich könnte man sich angesichts der bereits hohen Ausgaben bei
den Hochschulen fragen, ob es sinnvoll ist, dass der SNF zusätzliches
Geld für Gold OA ins System hineinwirft und somit Hochschulen weniger
Anreiz haben, ihr Geld aus den Subskriptionen für die Finanzierung von
Gold OA umzuschichten. Allerdings sind die Gold-OA-Beiträge beim SNF
gesamthaft noch auf einem verhältnismässig kleinen Niveau,\footnote{Gutknecht,
  Christian, Graf, Regula, Kissling, Ingrid, Krämer, Daniel, Milzow,
  Katrin, Würth, Stéphanie, \& Zimmermann, Thomas. (2016). Open Access
  to Publications: SNSF monitoring report 2013 - 2015. Zenodo.
  \url{http://doi.org/10.5281/zenodo.584131}} so dass die politische
Frage, wer letztlich für Gold OA bezahlen soll (SNF oder Hochschulen)
bisher noch unbeantwortet ist.

\emph{\textbf{KS:} Was hättest du dir an Wirkungen von der Studie
versprochen, welche (noch) nicht eingetreten sind?}

\textbf{CG:} Eine zügigere Umschichtung der Subskriptionsgelder nach OA.
Leider wird das immer noch verschleppt.

\emph{\textbf{KS:} Du hast dich mit Bezug auf diese Studie öffentlich
dahingehend geäussert, dass sich die Transformation der
Wissenschaftskommunikation der Schweiz zu OA auch finanziell lohnen
würde. Und doch ist das bislang nicht eingetreten. Nach all den Jahren
an Engagement: Siehst du noch einen Weg, dass das jemals passieren wird?
Was steht dem wirklich im Weg?}

\textbf{CG:} Dass OA kommen wird, bin ich mir ganz sicher. Die Frage ist
bloss wie schnell. Bei den Einsparungen kommt es vor allem darauf an,
wie konsequent OA und tiefere Preise eingefordert werden.
SCOAP\textsuperscript{3} zeigt, dass ein Durchschnittspreis von
1'000~Euro pro Artikel möglich ist. Die Niederländer haben beim Springer
Offsetting Agreement auch 1'400~Euro pro Artikel erreicht. Wenn man sich
in diesem Rahmen weiterbewegt, sind Einsparungen für die Schweiz sehr
realistisch. Allerdings zeigt der Vergleich des Schwedischen und
Niederländischen Springer Offsetting Agreements,\footnote{\url{https://wisspub.net/2017/10/04/schweden-springer-und-iop-offsetting}}
dass tiefe Preise hart erkämpft werden müssen. Schweden zahlt pro
OA-Artikel doppelt so viel wie die Niederlande. Ich vermute, das kommt
daher, dass Schweden nicht mit einer \enquote{Alles oder
nichts}-Forderung gegenüber Springer aufgetreten ist.

Für mich ist zudem die sonderbare Partnerschaft von Wiley und
Hindawi\footnote{\url{https://wisspub.net/2016/06/18/wiley-hindawi-partnerschaft}}
ein starkes Indiz, dass ein klassischer Verlag wie Wiley auch damit
rechnet, dass die Preise in einem kommenden OA-Markt mit mehr Wettbewerb
stark unter Druck kommen werden, beziehungsweise die Gewinnmargen nicht
mehr einfach so gegeben sind.

Noch unklar ist, in welchem Masse die Transformation für andere Länder
finanzierbar ist. Die \enquote{Pay-It-Forward}-Studie\footnote{\url{http://icis.ucdavis.edu/wp-content/uploads/2016/07/UC-Pay-It-Forward-Final-Report.rev_.7.18.16.pdf}}
hat beispielsweise aufgezeigt, dass die heutigen Subskriptionsausgaben
von grossen nordamerikanischen Universitäten wie British Columbia,
Harvard oder University of California nicht ausreichen werden, um ihren
Publikationsoutput zu den heutigen OA-Artikelkosten vollständig zu
finanzieren. Allerdings bin ich recht optimistisch, dass auch da
national oder gegebenenfalls auch international Lösungen gefunden
werden. Denn man darf ja nicht vergessen, dass das heutige System
inklusive Profite der Verlage ja auch von jemanden bezahlt wird.

\emph{\textbf{KS:} In der OA-Szene wird die Schweiz nicht nur wegen
deines Engagements, sondern auch wegen der relativ neuen
Open-Access-Strategie wahrgenommen.}\footnote{Swissuniversites; Swiss
  National Science Foundation (2017): Nationale Open-Access-Strategie
  für die Schweiz,
  \url{https://www.swissuniversities.ch/fileadmin/swissuniversities/Dokumente/Hochschulpolitik/Open_Access/Open_Access__strategy_final_DE.pdf}}
\emph{Als Wissenschaftler an einer kleinen Fachhochschule in der Schweiz
würde mich interessieren, was mich davon erreichen sollte. Was würdest
du aus deiner Position im SNF erwarten, dass sich für mich -- und andere
Forschende -- ändern wird? Oder ist die Strategie nicht doch eher ein
kaum wirksames Papier geblieben?}

\textbf{CG:} Bislang ist es in der Tat nur ein Papier. Aber immerhin
eins mit einem verbindlichen Ziel: OA bis 2024. Und ich gehe stark davon
aus, dass man dies Ziel auch erreichen will.

Ich bin ehrlich gesagt auch gespannt, wie Swissuniversities das Ziel OA
bis 2024 erreichen will. Der Aktionsplan ist ja zurzeit noch in der
Ausarbeitung und ich bin da nicht involviert.

Meine Vermutung ist, dass man wohl zuerst versuchen wird Green OA zu
stärken, indem nun alle Hochschulen entsprechende OA-Policies und
Ablageorte haben. Dann dürfte von Oben der Druck auf die Forschenden
erhöht werden, ihre Publikationen gemäss dieser Policies auf ein
Repository zu stellen. So lässt sich der OA-Anteil durchaus etwas
erhöhen. Doch Green OA hat seine Grenzen: Denn die Verlage bestimmen,
wann, wo und welche Version hochgeladen werden darf. Verlage können nach
Belieben ihre Self-Archiving Policies ändern oder so kompliziert machen,
dass Forschende schlicht überfordert sind (siehe beispielsweise den
Autorenvertrag von ACS\footnote{\url{http://pubs.acs.org/page/copyright/journals/index.html}}).
Dem kann Gegensteuer gegeben werden, indem ein
Zweitveröffentlichungsrecht eingeführt wird. Allerdings sind die Chancen
dazu in der Schweiz klein, da die aktuelle Urheberrechtsrevision wegen
zu stark divergierenden Interessen blockiert ist. Alternativ dazu können
Hochschulen prüfen, ob ihnen nicht schon sowieso die Rechte an den
Erzeugnissen ihrer Angestellten gehören. Demnach kann eine Hochschule
bestimmen, dass ihre Angehörige gar nicht erst das exklusive
Verbreitungsrecht dem Verlag übertragen können, und somit das Recht zu
Green OA automatisch behalten. Dieser Ansatz, auch das \enquote{Harvard
Modell} genannt, erlebt zurzeit in UK unter dem Titel \enquote{UK
Scholarly Communications Licence}\footnote{\url{http://ukscl.ac.uk/}}
eine Renaissance und scheint mir sehr vielversprechend.

Hoffentlich früher als später werden die Hochschulen merken, dass es
viel effizienter ist, die OA-Stellung nicht den Forschenden aufzubürden,
sondern von den Verlagen direkt zu verlangen. Man zahlt diesen ja
sowieso 70 Millionen Franken pro Jahr und man hätte mit diesem Geld den
Hebel um die sofortige OA-Publikation von Schweizer Forschenden zu
fordern.

\emph{\textbf{KS:} Mit deinem Überblickswissen: Wo, im Vergleich zu
anderen Staaten, steht denn die Schweiz im Bezug auf OA heute,
realistisch gesprochen?}

\textbf{CG:} Da stellt sich die Frage, was man vergleichen will. Leider
gibt es noch keinen direkten Ländervergleich der OA-Anteile (Green und
Gold), der dann zusätzlich nach Mitteleinsatz gewichtet ist. Beachtet
man nur den Gold-OA-Anteil in Web of Science liegen Länder wie
Brasilien, Serbien, Pakistan, Kroatien, Kolumbien oder Chile mit einem
Gold-OA-Anteil von über 27\,\% weit vor der Schweiz mit
12\,\%.\footnote{Wohlgemuth, Michael; Rimmert, Christine, \& Taubert,
  Niels Christian (2017): \emph{Publikationen in
  Gold-Open-Access-Journalen auf globaler und europäischer Ebene sowie
  in Forschungsorganisationen}. Bielefeld: Universität Bielefeld;
  \url{http://nbn-resolving.de/urn:nbn:de:0070-pub-29128079}}

Wenn man die Schweiz mit Ländern vergleicht, welche eine ähnliche
Wissenschaftskultur aufweisen, würde ich sagen, dass wir im unteren
Mittelfeld sind. Es gibt einige Schweizer Repositories, die einen
vergleichsweise grossen OA-Anteil aufweisen. Und bei Gold OA mischt man
auch im Mittelfeld mit, da Forschende grundsätzlich finanziell so gut
ausgestattet sind, dass das mehrheitlich Fehlen von institutionellen
Publikationsfonds sich nicht negativ auf das Publizieren in
kostenpflichtigen OA-Journals auswirkt.

Allerdings, wer die Entwicklungen in UK seit dem Finch Report von 2012
oder den Strategiewechsel in den Niederlanden seit 2014 verfolgt, wird
feststellen, wie man in der Schweiz trotz Wissen um diese Entwicklungen
und trotz enorm viel Geld wenig vorwärts gekommen ist.

\emph{\textbf{MV:} Auf wisspub.net bloggst du regelmäßig zu Verträgen
mit OA-Verlagen beziehungsweise Verträgen mit OA-Komponenten -- in der
Schweiz und im europäischen Ausland. Was sind deiner Meinung nach die
wichtigsten Entwicklungen in Sachen OA-Verträge der letzten drei Jahre?}

\textbf{CG:} Es lohnt sich das Beispiel Niederlande anzuschauen, wo die
Verträge mit OA-Komponenten inzwischen ja öffentlich sind.\footnote{\url{http://www.vsnu.nl/en_GB/public-access-request}}
Man hat in den Niederlanden sehr früh gemerkt, dass Green OA nicht
skaliert und hat das auch sehr offen und ehrlich kommuniziert. Unter dem
Druck eines Politikers, der genau verstanden hat, um was es bei Open
Access geht, haben die Universitäten den Hebel Subskriptionszahlungen
weltweit erstmals für ihre OA-Forderung eingesetzt und dabei mit
Ausnahme von Elsevier und Oxford University Press beachtliche Erfolge
erzielt. Dabei wurde nur ganz wenig mehr bezahlt als bisher.

\emph{\textbf{MV:} Der Trend geht zu Offsetting-Deals -- das heisst
gemeinsamen Verträgen für das OA- und Subskriptionsgeschäft. Wie
bewertest du diese Entwicklung?}

\textbf{CG:} Nun, ich halte Offsetting-Agreements als den realistischen
Weg, OA zu erreichen, da er sowohl für Bibliotheken als auch für Verlage
funktioniert. Selbstverständlich wäre es mir viel lieber, wenn weltweit
Bibliotheken gleich alle Subskriptionen kündigen und mit dem
freigewordenen Geld verlagsunabhängig nur reines Gold OA unterstützen
würden. Doch ich sehe beim besten Willen nicht genügend Unterstützung
für einen solch radikalen Bruch.

\emph{\textbf{MV:} Wie sieht ein \enquote{guter} Vertrag aus -- das
heisst welche Anforderungen sollte er erfüllen, damit er als
wissenschaftsfreundlich gelten kann?}

\textbf{CG:} Er sollte transparent sein. Non-Disclosure-Agreements
sollten nicht akzeptiert werden. Des Weiteren sollte der Verlag als
Dienstleister in die Pflicht genommen werden. Insbesondere die
vollständige Ausgabe von Metadaten (Lizenzinformationen,
Literaturreferenzen, Förderinformationen, Peer Review Informationen,
ORCID etc.) an die DOI-Agentur Crossref sollte eingefordert werden. Auch
wichtig ist natürlich ein vernünftiger Preis, der sich am geleisteten
Service des Verlages orientiert.

\emph{\textbf{MV:} Wie schätzt du DEAL ein?}

\textbf{CG:} Sehr positiv. Auch wenn noch keine Resultate zur
Beurteilung vorliegen, scheint mir der beispielslose Elsevier-Boykott
schon ein grosser Sieg für Open Access zu sein. Denn die ganze Welt
sieht nun, dass die Wissenschaftscommunity auch ohne Big Deal mit
Elsevier irgendwie funktioniert und der befürchte Aufstand der
Wissenschaftler ausgeblieben ist.

Ich bin zudem enorm überrascht, wie es deutsche Institutionen geschafft
haben, die inneren Gräben zu überwinden, um gegenüber Elsevier, Springer
Nature und Wiley geschlossen mit einer klaren Botschaft aufzutreten. Für
mich ist klar, dass, wenn die deutschen Institutionen lange genug
durchhalten, sie als Gewinner vom Platz gehen. Elsevier kann es sich
nicht leisten, seine Aktionäre über mehrere Jahre mit massiv weniger
Einnahmen aus Deutschland zu vertrösten.

\emph{\textbf{MV:} Nicht alle OA-Aktivist*innen befürworten Deals mit
Verlagen, deren Geschäftsmodell aktuell noch vom Subskriptionsgeschäft
dominiert wird. Es steht die Befürchtung im Raum, dass dies zu einer
weiteren Marktkonzentration führen könnte und reine OA-Verlage
beziehungsweise Kleinverlage aussen vor bleiben. Wie stehst du dazu?}

\textbf{CG:} Wie gesagt, ich befürworte Offsetting-Deals mit den
Verlagen bloss, weil ich bei den Hochschulbibliotheken noch keine
Bereitschaft erkenne, einen koordinierten radikalen Schnitt bei den
Subskriptionen zu machen. Die Frage ist vielerorts leider nicht
\enquote{Finanzierung von Gold OA oder Offsetting-Deals} sondern
vielmehr nur \enquote{Subskriptionen oder überhaupt Offsetting-Deals}.

Das Argument, dass bei Offsetting-Verträgen OA-Verlage benachteiligt
werden, ist natürlich nicht von der Hand zu weisen. Von daher sollte
neben Offsetting-Deals auch immer Gold-OA-Funding zu Verfügung gestellt
werden, das nicht schon durch bestimmte Verlage gebunden wird. Dies ist
beispielsweise in den Niederlanden nur ungenügend geschehen.

Die befürchtete Marktkonzentration scheint mir in Hinblick auf eine
reine OA-Welt, in der es einen Wettbewerb um Beiträge von AutorInnen
geben wird, gar willkommen zu sein. Im wissenschaftlichen Publizieren,
insbesondere bei den technischen Belangen, gibt es noch viele nicht
ausgeschöpfte Skaleneffekte. Als kleines Beispiel ist die Initiative for
Open Citations (I4OC) zu nennen, bei der Verlage ihre Referenzen
standardisiert und offen an Crossref rausgeben. Im August 2017 gab es
einen Aufruf,\footnote{\url{https://elifesciences.org/inside-elife/b17ec36f/an-open-letter-to-stakeholders-of-the-initiative-for-open-citations}}
dass man zwar schon viele Referenzen von grossen Verlagen offen und
nachnutzbar bei Crossref gemeldet hat, aber bisher die Unterstützung
durch etwa 3'000 Kleinverlage fehlt. Wohlgemerkt, Elsevier macht zurzeit
als grösste Ausnahme bei I4OC auch nicht mit. Allerdings ist das bei
Elsevier mehr eine Frage des Geschäftsmodells und nicht, wie bei den
kleineren Verlagen, eine Frage der technischen Kapazitäten.

\emph{\textbf{MV:} Gibt es deiner Einschätzung nach Vertragsmodelle, die
wir als Bibliotheks- und OA-Community nach bisherigen Erkenntnissen
nicht weiter verfolgen sollten? Warum?}

\textbf{CG:} Ja, wir sollten endlich weg vom obsoleten
Subskriptionsmodell, hin zu irgendeinem Gold-OA-Modell, bei dem sich der
Preis an der effektiv geleisteten Arbeit des Verlags orientiert.

\emph{\textbf{MV:} Transparenz über die Vertragskonditionen und de facto
Kosten ist ein key factor auf dem Weg zu 100\% OA. Du hast sogar
Universitäten verklagt, damit sie diese Transparenz über ihre
Zeitschriftendeals mit den großen Verlagen herstellen. Doch leider sind
noch immer zu wenige Verträge öffentlich. Was hält öffentliche
Einrichtungen und insbesondere Bibliotheken davon ab, ihre Verträge zu
veröffentlichen?}

\textbf{CG:} Um präzis zu sein, habe ich die Hochschulbibliotheken
bislang bloss um die Offenlegung der konkreten Ausgaben angefragt, aber
noch nicht um die Verträge.

Das meist gebrauchte Argument gegen Transparenz bei den Ausgaben war die
mögliche Reaktion der Verlage. Bei der Offenlegung der Zahlen würden
Vertraulichkeitsklauseln verletzt, und Schadenersatzklagen, Einstellung
des Zugangs oder künftig schlechtere Konditionen wären die Folge. Auch
argumentierten die Hochschulbibliotheken, dass es sich bei ihren
Ausgaben um Geschäftsgeheimnisse der Verlage handelt, deren Offenlegung
den Verlagen substanziellen Schaden zuführen würde.

Lediglich die Universität Basel schaffte es, mit dieser initialen
Argumentation die sachunkundigen Gerichte zu vereinnahmen. In anderen
Kantonen und auf Bundesebene hielten die zuständigen Rekursinstanzen die
negativen Konsequenzen für die öffentlichen Institutionen für nicht
wahrscheinlich, beziehungsweise wertete die Transparenz zu den Ausgaben
höher als das Interesse der Verlage an Geheimhaltung.

Und selbstverständlich ist inzwischen klar, dass dort, wo Zahlen
offengelegt wurden, es nicht zu den befürchteten Reaktionen der Verlage
gekommen ist. Im Gegenteil, es gab sogar die Situation, wo eine
Bibliotheksdirektorin mir gegenüber behauptete, sie dürfe die Zahlungen
nicht offenlegen, da sonst schlimmes passieren würde. Als sie dann von
der Rekursinstanz angehalten wurde, dazu eine Stellungnahme von den
Verlagen einzuholen, haben einerseits viele Verlage nicht einmal
geantwortet, während dessen Taylor \& Francis sogar explizit die
Erlaubnis zur Veröffentlichung erteilt hat.

Der anhaltende Widerstand einiger öffentlicher Institutionen in der
Schweiz, die Zahlungen oder gar die Verträge im Sinne von Open Data nun
proaktiv zu veröffentlichen, ist für mich sachlich nicht mehr
nachvollziehbar. In Finnland haben eine ähnliche Anfrage und ein
einziger Gerichtsentscheid dazu geführt, dass das Bildungsministerium
die Zahlungen von sämtlichen finnischen Hochschulen an eine grosse
Anzahl Verlage der letzten Jahre proaktiv gesammelt und veröffentlicht
hat.

Ich vermute, die Gründe liegen eher im menschlichen Bereich. Nicht alle
können sich eingestehen, dass sie sich im Kampf gegen Transparenz und
teils gegen meine Person in etwas verrannt haben, das langfristig nicht
besonders erstrebenswert oder zweckdienlich war. Verantwortlich sein und
Verantwortung übernehmen sind halt häufig zweierlei Dinge.

\emph{\textbf{MV:} In vielen -- aber nicht allen -- Fällen helfen
Informationsfreiheitsgesetze, um prinzipiell an die Vertragsdaten zu
kommen. Aber selbst dann kann es ein steiniger Weg sein, bis die
Verträge de facto öffentlich sind. Wäre es denkbar, über
Vergaberichtlinien oder sonstige Verwaltungsvorschriften (in der Schweiz
und/oder Europa...) dafür zu sorgen, dass Verträge immer veröffentlicht
werden müssen? Und zwar direkt nach Abschluss?}

\textbf{CG:} Das öffentliche Beschaffungswesen müsste im
Bibliothekswesen tatsächlich mehr Beachtung finden. Wie ich relativ spät
nach meinen Anfragen herausgefunden habe, gibt es in der Schweiz bereits
die gesetzliche Regelung, wonach Aufträge beziehungsweise Einkäufe ab
230'000 Franken öffentlich deklariert werden müssen. Nur wenige
Bibliotheken sind dieser Deklarationspflicht je nachgekommen, was im
Nachgang zu meinen Anfragen durchaus bei der internen und externen
Revision bei einigen Hochschulen Anlass zur Korrektur gegeben haben
dürfte.

Zudem hat das Schweizer Parlament bereits beschlossen, das
Bundesinstitutionen künftig ihre Einkäufe ab 50'000 Franken einmal pro
Jahr maschinenlesbar publizieren müssen. Diese Regelung soll mit dem
revidierten Beschaffungsgesetz des Bundes kommen.

Allerdings zeigt gerade diese Revision, dass Transparenz nicht
automatisch ein Selbstläufer ist. Gemäss dem aktuellen Entwurf des
revidierten Beschaffungsgesetzes sollen nämlich sämtliche Dokumente in
Verbindung mit Beschaffungsverfahren des Bundes grundsätzlich dem
Geltungsbereich des Öffentlichkeitsgesetzes entzogen werden.\footnote{\url{https://www.admin.ch/gov/de/start/dokumentation/medienmitteilungen.msg-id-65657.html}}
Es scheint eine Tendenz der Verwaltung zu geben, das
Transparenzbedürfnis der Politik und der Bevölkerung permanent zu
hintertreiben.

Aber unabhängig vom Beschaffungswesen scheint mir die Lösung im
OA-Bereich bereits skizziert. Mit Open APC\footnote{\url{https://www.intact-project.org/openapc/}}
gibt es bereits eine Bottom-up-Initiative, welche APC-Zahlungen erfasst
und bei der viele Institutionen bereit sind, sich zu beteiligen. Sprich:
Je schneller wir vom Subskriptions-Modell wegkommen, umso schneller
haben wir auch da mehr Transparenz über die De-facto anfallenden Kosten
für wissenschaftliche Publikationen.

%autor
\begin{center}\rule{0.5\linewidth}{\linethickness}\end{center}

\textbf{Christian Gutknecht} arbeitet beim Schweizer Nationalfonds und
schreibt im Gruppenblog \url{http://www.wisspub.net}. (ORCID:
\url{http://orcid.org/0000-0002-7265-1692}). Die im Interview
geäusserten Aussagen geben seine persönliche Sicht wider und sind nicht
mit dem SNF abgestimmt.

\textbf{Karsten Schuldt}, Wissenschaftlicher Mitarbeiter Schweizerisches
Institut für Informationswissenschaft, HTW Chur. Redakteur LIBREAS.
Library Ideas.

\textbf{Michaela Voigt}, Open-Access-Team der TU Berlin, Redakteurin
LIBREAS. Library Ideas. (ORCID:
\url{http://orcid.org/0000-0001-9486-3189})

\end{document}
