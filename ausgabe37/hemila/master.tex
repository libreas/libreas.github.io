\documentclass[a4paper,
fontsize=11pt,
%headings=small,
oneside,
numbers=noperiodatend,
parskip=half-,
bibliography=totoc,
final
]{scrartcl}

\usepackage[babel]{csquotes}
\usepackage{synttree}
\usepackage{graphicx}
\setkeys{Gin}{width=.4\textwidth} %default pics size

\graphicspath{{./plots/}}
\usepackage[ngerman]{babel}
\usepackage[T1]{fontenc}
%\usepackage{amsmath}
\usepackage[utf8x]{inputenc}
\usepackage [hyphens]{url}
\usepackage{booktabs} 
\usepackage[left=2.4cm,right=2.4cm,top=2.3cm,bottom=2cm,includeheadfoot]{geometry}
\usepackage{eurosym}
\usepackage{multirow}
\usepackage[ngerman]{varioref}
\setcapindent{1em}
\renewcommand{\labelitemi}{--}
\usepackage{paralist}
\usepackage{pdfpages}
\usepackage{lscape}
\usepackage{float}
\usepackage{acronym}
\usepackage{eurosym}
\usepackage{longtable,lscape}
\usepackage{mathpazo}
\usepackage[normalem]{ulem} %emphasize weiterhin kursiv
\usepackage[flushmargin,ragged]{footmisc} % left align footnote
\usepackage{ccicons} 
\setcapindent{0pt} % no indentation in captions

%%%% fancy LIBREAS URL color 
\usepackage{xcolor}
\definecolor{libreas}{RGB}{112,0,0}

\usepackage{listings}

\urlstyle{same}  % don't use monospace font for urls

\usepackage[fleqn]{amsmath}

%adjust fontsize for part

\usepackage{sectsty}
\partfont{\large}

%Das BibTeX-Zeichen mit \BibTeX setzen:
\def\symbol#1{\char #1\relax}
\def\bsl{{\tt\symbol{'134}}}
\def\BibTeX{{\rm B\kern-.05em{\sc i\kern-.025em b}\kern-.08em
    T\kern-.1667em\lower.7ex\hbox{E}\kern-.125emX}}

\usepackage{fancyhdr}
\fancyhf{}
\pagestyle{fancyplain}
\fancyhead[R]{\thepage}

% make sure bookmarks are created eventough sections are not numbered!
% uncommend if sections are numbered (bookmarks created by default)
\makeatletter
\renewcommand\@seccntformat[1]{}
\makeatother

% typo setup
\clubpenalty = 10000
\widowpenalty = 10000
\displaywidowpenalty = 10000

\usepackage{hyperxmp}
\usepackage[colorlinks, linkcolor=black,citecolor=black, urlcolor=libreas,
breaklinks= true,bookmarks=true,bookmarksopen=true]{hyperref}
\usepackage{breakurl}

%meta
%meta

\fancyhead[L]{M. Hemila\\ %author
LIBREAS. Library Ideas, 37 (2020). % journal, issue, volume.
\href{http://nbn-resolving.de/}
{}} % urn 
% recommended use
%\href{http://nbn-resolving.de/}{\color{black}{urn:nbn:de...}}
\fancyhead[R]{\thepage} %page number
\fancyfoot[L] {\ccLogo \ccAttribution\ \href{https://creativecommons.org/licenses/by/4.0/}{\color{black}Creative Commons BY 4.0}}  %licence
\fancyfoot[R] {ISSN: 1860-7950}

\title{\LARGE{Forschung an öffentlichen Bibliotheken: Chancen und Problemstellungen illustriert am Beispiel von Sprachlernabteilungen}}% title
\author{Mahmoud Hemila} % author

\setcounter{page}{1}

\hypersetup{%
      pdftitle={Forschung an öffentlichen Bibliotheken: Chancen und Problemstellungen illustriert am Beispiel von Sprachlernabteilungen},
      pdfauthor={Mahmoud Hemila},
      pdfcopyright={CC BY 4.0 International},
      pdfsubject={LIBREAS. Library Ideas, 37 (2020).},
      pdfkeywords={Schweiz, Öffentliche Bibliotheken, Fremdsprachen},
      pdflicenseurl={https://creativecommons.org/licenses/by/4.0/},
      pdfcontacturl={http://libreas.eu},
      baseurl={http://libreas.eu},
      pdflang={de},
      pdfmetalang={de}
     }



\date{}
\begin{document}

\maketitle
\thispagestyle{fancyplain} 

%abstracts

%body
Im Rahmen meiner Bachelorarbeit an der Hochschule für Technik und
Wirtschaft (HTW) Chur -- heute Fachhochschule Graubünden -- führte ich
im Jahr 2019 eine Untersuchung von Sprachlernabteilungen in öffentlichen
Bibliotheken von vier Schweizer Städten durch. Am Beispiel der
untersuchten Sprachlernabteilungen zeigen sich sehr deutlich die
Chancen, aber auch Mängel und Problemstellungen des Miteinanders
zwischen den öffentlichen Bibliotheken und Lehre und Forschung an
Universitäten und Fachhochschulen. Diese Arbeit befindet sich aktuell im
Publikationsprozess und wird unter dem Titel «Nutzung der
Sprachlernabteilungen in öffentlichen Bibliotheken» veröffentlicht. Die
durch diese Forschungszusammenarbeit gewonnenen Erkenntnisse werden im
folgenden Artikel reflektiert und vorgestellt.

\hypertarget{forschungsluxfccke-bei-einem-populuxe4ren-bibliotheksangebot}{%
\section{Forschungslücke bei einem populären
Bibliotheksangebot}\label{forschungsluxfccke-bei-einem-populuxe4ren-bibliotheksangebot}}

Zahlreiche öffentliche Bibliotheken in der Schweiz verfügen über
Sprachlernabteilungen beziehungsweise Bibliotheksbestände zum Erlernen
von Sprachen. In ihrem Bestreben als öffentliche Bibliotheken, ihren
Benutzerinnen und Benutzern den Zugang zu gedruckter und/oder
gespeicherter Information zu bieten und ihrer Weiterbildung,
Leseförderung und Unterhaltung zu dienen (Büchereiverband Österreichs
o.\,J.), investieren die Verantwortlichen der Bibliotheken viel
Leidenschaft und zahlreiche Ressourcen in den Erwerb und die Pflege
eines geeigneten Bestands sowie den Aufbau sinnvoller ergänzender
Angebote.

Die Evaluation des Aufwandes und der Ergebnisse dieser Bemühungen bleibt
jedoch bis heute eine offene Frage, für die sich für den gesamten
deutschsprachigen Raum in der bisherigen Forschungslandschaft keine
Antwort findet. Es herrscht ein akuter Mangel an Fachliteratur. Dies
erstaunt insofern, als dass hier in nicht unwesentlichem Ausmass
öffentliche Gelder investiert werden -- gemäss Erfahrungen im Rahmen der
durchgeführten Studie allem Anschein nach ohne jegliche systematische
Analyse, ob, von wem und wie oft die Angebote denn genutzt werden und
wie effizient das Lernen mit den bereitgestellten Angeboten ist.

Es zeigt sich eine klare Lücke zwischen der Praxis, welche
Sprachlernabteilungen trotz fehlender Wirksamkeits- und
Interessensnachweise unbeirrt finanziell fördert und pflegt, und der
Forschung, welche diesem Phänomen bislang keinerlei Aufmerksamkeit
gewidmet hat und die öffentlichen Bibliotheken daher auch nicht mit den
dringend erforderlichen Erkenntnissen zu Nutzen und
Optimierungsmöglichkeiten zu versorgen vermag.

\hypertarget{die-zusammenarbeit-mit-den-bibliotheken}{%
\section{Die Zusammenarbeit mit den
Bibliotheken}\label{die-zusammenarbeit-mit-den-bibliotheken}}

Mit der Mission, meinen Beitrag zur Schliessung dieser Lücke zu leisten,
begann ich mit meiner Arbeit. Ich kontaktierte vier Bibliotheken in der
deutschsprachigen Schweiz mit der Bitte, dieses Forschungsanliegen zu
unterstützen. Alle vier sagten postwendend zu, weshalb davon ausgegangen
werden kann, dass die Unterstützungsbereitschaft der Praxis für
wissenschaftliche Forschung insgesamt hoch ist.

Die Motive für die Zusagen dürften jedoch unterschiedlich gewesen sein,
was spätestens bei der Anwendung der gewonnenen Ergebnisse einen höchst
entscheidenden Faktor darstellt. Manche Bibliotheken sagten zu aus
aufrichtigem Interesse an wissenschaftlichem Feedback zu diesem Thema
sowie um einen Vergleich ihrer Angebote zu anderen Bibliotheken zu
erhalten. Andere dagegen erweckten eher den Eindruck, dass sie
ausschliesslich deshalb teilnahmen, um mich als Forscher bei meinem
Vorhaben zu unterstützen. Dabei handelt es sich um ein Phänomen, welches
wohl in verschiedenen Forschungsfeldern auftreten dürfte. Womöglich
zeigt es sich jedoch bei Bibliotheken, die sich vermutlich in ihrer
tiefsten Natur als Dienstleisterinnen von Forschung und Bildung
verstehen, noch verstärkt.

In der Situation als externe, nicht mit den Organisationsstrukturen
vertraute Person und überdies noch in gewisser Weise als «Bittsteller»,
wie es bei Abschlussarbeiten häufig vorkommen dürfte, war es mir nicht
möglich, zielgerichtet geeignete Ansprechpartner zu verlangen. Meine
Anfrage habe ich an die für die Abteilung zuständigen Personen und,
sofern diese nicht bekannt waren, an allgemeine «Info»-Mailadressen
geschickt. Die Bibliotheken wiesen mir daraufhin aus ihrer Sicht
geeignete Ansprechpersonen zu. Infolge der Unterschiede in der
betrieblichen Organisation waren dies bei drei der vier Bibliotheken die
für die Abteilung und deren Bestand zuständige Person -- unter Umständen
mit verschiedenen Stellenbezeichnungen und -fokussen --, in einem Fall
jedoch eine übergeordnete Führungsperson. Erfahrung, Wissen, Interesse
und Fokus dieser Personen unterschieden sich stark, was eine
standardisierte, generalisierbare Forschung erschwert.

Auch beim inhaltlichen Umfang der Unterstützung zeigten sich grosse
Unterschiede bei den verschiedenen beitragenden Bibliotheken, sowohl im
«Können» als auch im «Wollen». Die Auskunftspersonen mit
unterschiedlichem Erfahrungsschatz führten natürlich zu
unterschiedlicher Qualität und Tiefe bei den im Rahmen der Studie
durchgeführten Interviews -- wenngleich ausnahmslos alle im Rahmen ihrer
Möglichkeiten zweifellos sehr offen und unterstützend Auskunft gaben.
Darüber hinaus verfügten die Bibliotheken auch sonst über sehr
unterschiedliche Datenressourcen, welche sie teilweise zu teilen bereit
waren oder dies in manchen Fällen aber auch verweigerten. Nur eine
einzelne Bibliothek vermochte systematische Ausleihdaten zur Abteilung
zur Verfügung zu stellen. Andere verweigerten dies aus Gründen der
Vertraulichkeit oder verfügen nach eigener Aussage auf Abteilungsebene
nicht über solches Datenmaterial. Somit konnte für solche
aufschlussreichen quantitativen statistischen Auswertungen leider nur
das Material einer einzigen Bibliothek genutzt werden, was in sich nicht
repräsentativ ist. Insgesamt gab es ein geringes Bewusstsein für die
Notwendigkeit von sorgfältigen und systematischen Datenerhebungen und
-auswertungen.

Die zur Studie gehörende passive, teilstrukturierte Beobachtung nach
Skibar (2017, S. 1--2) bewilligten alle Bibliotheken ohne Vorbehalte,
sie hatten lediglich unterschiedliche Anforderungen an die
Transparentmachung gegenüber ihren KundInnen. Keine der Bibliotheken war
bereit, die im Rahmen der Studie durchgeführte Nutzungsumfrage bei ihren
BesucherInnen aktiv zu unterstützen. Teilweise war das Auslegen von
Flugblättern erlaubt, bei anderen Bibliotheken nicht einmal dies. Alle
Bibliotheken lehnten ab, ihrerseits NutzerInnen anzusprechen und auf die
Studie hinzuweisen. Sie erlaubten dem Forscher zwar, Personen vor Ort
selbst direkt anzusprechen. Dieses Vorgehen ohne offizielle
Unterstützung der Bibliothek führte allerdings zu einer gewissen Skepsis
der NutzerInnen im Hinblick auf die Seriosität der Studie. Infolgedessen
weigerten sich viele, an der Studie teilzunehmen. Es existierte gemäss
Auskunft der AnsprechpartnerInnen nur eine einzige andere
Nutzungserhebung bei KundInnen öffentlicher Bibliotheken, und dies nur
zu einer bestimmten Bibliothek. Dabei handelte es sich um eine durch die
Stadtbibliothek durchgeführte repräsentative Umfrage. Leider konnte mir
die Bibliothek darauf keinen Zugriff gewähren.

\hypertarget{die-uxf6ffentliche-bibliothek-im-spannungsfeld-der-politik}{%
\section{Die öffentliche Bibliothek im Spannungsfeld der
Politik}\label{die-uxf6ffentliche-bibliothek-im-spannungsfeld-der-politik}}

Die geschilderte Entscheidung der teilnehmenden öffentlichen
Bibliotheken, die Akquise von Umfrageteilnehmenden nicht aktiv zu
unterstützen, ist aber absolut verständlich mit Blick auf ihre
Eigenschaft als öffentliche Institution. Aufgrund ihres Status als
öffentliche Bibliotheken stehen sie im Fokus der Politik, müssen mit
deren Entwicklungen und Spannungen umgehen und repräsentieren zudem
Staat, Kanton und/oder Gemeinde. Das schafft das Vertrauen ihrer
NutzerInnen beziehungsweise KundInnen, bringt jedoch deshalb auch eine
grosse Verantwortung mit sich. Sie verfügen daher nicht über Freiheiten,
welche beispielsweise Unternehmen in der Privatwirtschaft für sich in
Anspruch nehmen können. Jede Zu- oder Absage, jede unterstützende
Massnahme oder Umsetzung von Forschung muss gewissen Grundsätzen an
Neutralität und Gleichbehandlung, aber auch der allgemeinen Stimmung in
der Bevölkerung folgen. Gerät sie aus irgendwelchen Gründen in die
Kritik drohen Proteste aus der Bevölkerung und unter Umständen strenge
Massnahmen und Forderungen seitens unter Druck geratener Politiker. In
gewisser Weise haben öffentliche Bibliotheken daher eingeschränktere
Freiheiten als dies bei privatrechtlichen Institutionen der Fall sein
dürfte, auch bei der Zusammenarbeit mit ForscherInnen. Auch bei Themen
wie Vertraulichkeit und Datenschutz dürften die Konsequenzen bei
Verfehlungen -- seien diese tatsächlich oder nur von der Öffentlichkeit
so wahrgenommen -- aufgrund des öffentlichen Auftrags direkter und
spürbarer sein als bei unabhängigen Unternehmen.

Ein weiterer Hinderungsgrund zur Unterstützung und vor allem Umsetzung
von Forschung könnte an einer weiteren Eigenheit des in der Schweiz als
«Service Public» bekannten Auftrags liegen: Die Gewährleistung eines
Angebots «für alle». Im Personenverkehr äussert sich dies darin, dass
auch schwach bewohnte Gebiete, deren Erschliessung sich zahlenmässig
eigentlich nicht rechnet, angebunden werden müssen. In der Praxis des
Bibliothekswesens dürfte es wohl bedeuten, dass das Budget sehr häufig
gewünschter beziehungsweise benötigter Bücher, zum Beispiel
Wörterbüchern Tigrinisch-Deutsch im Sprachlernalltag, trotz
entsprechender Forschungsergebnisse zugunsten seltener benötigter
Sprachen zurückgestellt werden muss. Warum also Forschung fördern, die
man aufgrund von Widersprüchen mit dem öffentlichen Auftrag im Alltag
ohnehin nicht anwenden kann?

Andererseits wäre es sicherlich zu einfach zu behaupten, dass die
Bibliotheken den Wendungen der Politik machtlos ausgeliefert sind. In
einer immer mehr von Daten gesteuerten Welt liegt es an ihnen, der
Politik mithilfe der entsprechenden Datenbasis auch im Sinne eines
Bottom-up-Ansatzes selbstbewusst deutlich zu machen, was von der
Bevölkerung gebraucht und gewünscht wird und wie die Prioritäten zu
setzen sind. Allzu oft werden die Top-down-Forderungen und
Entscheidungen von PolitikerInnen schliesslich aufgrund subjektiv
wahrgenommener Bevölkerungsstimmungen oder besonders eindringlicher
Presseschlagzeilen geleitet. Dazu benötigen die Bibliotheken jedoch
Ressourcen, welche nach meinem Eindruck zunächst aufgebaut werden
müssen. Es braucht Expertise, Zukunftsorientierung und Optimierungswille
bei Mitarbeitenden, geeignete elektronische Tools zur Erfassung der
Daten, um den Bibliotheken gegenüber den politischen
Entscheidungsträgern eine Stimme zu verleihen.

Hier können Fachhochschulen und Universitäten einen entscheidenden
Beitrag leisten. Sie verfügen über exakt dieses Wissen und diese
Kompetenzen und sollten die öffentlichen Bibliotheken damit
unterstützen. Dies kann zum Beispiel durch die Vermittlung entsprechend
geschulter Fachkräfte, Schulungsangebote für Mitarbeitende oder schlicht
und einfach die verstärkte Durchführung von Forschungsarbeiten mit
anschliessend zur Verfügung gestellten, verständlich aufbereiteten
Management Summaries geschehen. An die Stelle der bisherigen eher
einseitigen Dienstleistungsbeziehung muss ein fruchtbarer gegenseitiger
Austausch treten. Bibliotheken nachhaltig zu fördern und entwickeln
sollte nebst dem Eigeninteresse der Bibliothek schliesslich auch im
ureigensten Interesse der Wissenschaft liegen, zieht sie doch nicht
zuletzt aus gut ausgestatteten und professionellen Bibliotheken das
Wissen und damit ihre Existenzgrundlage.

\hypertarget{basis-fuxfcr-erfolgreiches-gemeinsames-wachsen-in-der-zukunft}{%
\section{Basis für erfolgreiches gemeinsames Wachsen in der
Zukunft}\label{basis-fuxfcr-erfolgreiches-gemeinsames-wachsen-in-der-zukunft}}

Die geschilderten Erfahrungen führen mich zu einigen Schlussfolgerungen
zu dem, was es braucht, damit öffentliche Bibliotheken und Wissenschaft
gemeinsam wachsen können.

Eine professionellere, detailliertere und systematische Datenerhebung
(zum Beispiel in Form von elektronischen Erfassungen von Ausleihen auf
Abteilungsebene, aber auch von Nutzungsumfragen bei KundInnen) durch die
Bibliotheken, welche diese zu wissenschaftlichen Zwecken in
anonymisierter Form auch zu teilen befugt ist, führt zu einer
zuverlässigeren und umfassenderen Datenbasis für die wissenschaftlichen
Studien, die in diesem Gebiet durchgeführt werden. Diese Basis zu
schaffen ist klar eine Aufgabe für die einzelnen Bibliotheken, wobei
treibende Kräfte aus der Forschung beim Festlegen relevanter Daten und
beim Austausch zwischen Bibliotheken natürlich eine entscheidende Rolle
einnehmen können und sollen. Die Bibliotheken können auch selbstständig
gewisse Analysen der von ihnen erhobenen Daten vornehmen, um ihr Angebot
laufend zu optimieren. Solche Analysen sind in sich keine Wissenschaft
und erfüllen auch nicht deren Zwecke, geben aber wichtige Hinweise auf
interessante Forschungsgebiete und relevante Themenfelder.

Die Rolle der Wissenschaft sehe ich daher insbesondere in
bibliotheksübergreifenden, generalisierenden Analysen. Natürlich ist
hier eine enge Abstimmung mit der Bibliothekspraxis erforderlich, um
praxisrelevante Fragestellungen zu identifizieren. Andererseits muss die
Praxis ihrerseits natürlich auch proaktiv Hilfe suchen, damit populäre
Trends wie Sprachlernabteilungen nicht jahrelang ohne jegliche
Evaluation oder Überprüfung mit öffentlichen Mitteln gefördert werden.

Natürlich sind nicht alle Vorschläge und erst recht nicht von allen
Bibliotheken umsetzbar, beispielsweise aus finanziellen oder auch
anderen organisationsspezifischen Gründen. Hier ist sowohl von Seiten
der Wissenschaft als auch von Bibliotheken eine gewisse Kreativität
gefragt, um massgeschneiderte und realitätsnahe Lösungen zu gestalten
und anzubieten. Kooperationen mit Organisationen verschiedenster Art
können Bibliotheken helfen, trotz beschränktem Budget wissenschaftliche
Erkenntnisse umzusetzen.

Auch die Schaffung einer Austauschplattform sowohl zwischen Forschung
und Bibliotheken, als auch zwischen den verschiedenen Bibliotheken
scheint mir zentral. Eine solche kann gewährleisten, dass zwischen den
Bibliotheken eine wohlkoordinierte Datenbasis aufgebaut wird, welche den
Wissenschaftlern Generalisierbarkeit und den Bibliotheken Vergleiche und
eine individuelle Einordnung erlaubt. Gleichzeitig können aktuelle
wissenschaftliche Erkenntnisse auf diesem Weg an die Bibliotheken
übermittelt werden. Damit gewinnen die öffentlichen Bibliotheken
idealerweise eine mächtigere Stimme gegenüber der Politik und können die
künftige, sie betreffende Forschungs- und Finanzierungsagenda sowie
übergreifende Projekte stärker selbst bestimmen. Die Wissenschaft
ihrerseits kann dadurch nur gewinnen, ist sie doch selbst in höchstem
Masse von guten und starken Bibliotheken abhängig.

\hypertarget{quellen}{%
\section{Quellen}\label{quellen}}

Büchereiverband Österreichs (o.\,J.): \emph{Definition ÖB}. Abgerufen
von \url{https://www.bvoe.at/inhalt/definition_oeb} {[}29.02.2020{]}.

Skibar, Elisabeth (2017): \emph{Qualitative und quantitative
Beobachtung}. Abgerufen von
\url{http://www2.lernplattform.schule.at/ahs-vwa/pluginfile.php/2982/mod_page/content/141/Qualitative\%20und\%20quantitative\%20Beobachtung_AKT.pdf}
{[}15.06.2020{]}.

%autor
\begin{center}\rule{0.5\linewidth}{0.5pt}\end{center}

\textbf{Mahmoud Hemila} studierte Informations- und
Bibliothekswissenschaft an der Al Azhar Universität in Kairo, Ägypten,
sowie Informationswissenschaft an der Fachhochschule Graubünden,
Schweiz. Heute ist er für die Bibliothek der Eidgenössischen Technischen
Hochschule (ETH) in Zürich tätig und beschäftigt sich als Mitglied des
Teams Forschungsdatenmanagement und Datenerhalt mit digitalem
Datenerhalt.

\end{document}
