\documentclass[a4paper,
fontsize=11pt,
%headings=small,
oneside,
numbers=noperiodatend,
parskip=half-,
bibliography=totoc,
final
]{scrartcl}

\usepackage[babel]{csquotes}
\usepackage{synttree}
\usepackage{graphicx}
\setkeys{Gin}{width=.4\textwidth} %default pics size

\graphicspath{{./plots/}}
\usepackage[ngerman]{babel}
\usepackage[T1]{fontenc}
%\usepackage{amsmath}
\usepackage[utf8x]{inputenc}
\usepackage [hyphens]{url}
\usepackage{booktabs} 
\usepackage[left=2.4cm,right=2.4cm,top=2.3cm,bottom=2cm,includeheadfoot]{geometry}
\usepackage{eurosym}
\usepackage{multirow}
\usepackage[ngerman]{varioref}
\setcapindent{1em}
\renewcommand{\labelitemi}{--}
\usepackage{paralist}
\usepackage{pdfpages}
\usepackage{lscape}
\usepackage{float}
\usepackage{acronym}
\usepackage{eurosym}
\usepackage{longtable,lscape}
\usepackage{mathpazo}
\usepackage[normalem]{ulem} %emphasize weiterhin kursiv
\usepackage[flushmargin,ragged]{footmisc} % left align footnote
\usepackage{ccicons} 
\setcapindent{0pt} % no indentation in captions

%%%% fancy LIBREAS URL color 
\usepackage{xcolor}
\definecolor{libreas}{RGB}{112,0,0}

\usepackage{listings}

\urlstyle{same}  % don't use monospace font for urls

\usepackage[fleqn]{amsmath}

%adjust fontsize for part

\usepackage{sectsty}
\partfont{\large}

%Das BibTeX-Zeichen mit \BibTeX setzen:
\def\symbol#1{\char #1\relax}
\def\bsl{{\tt\symbol{'134}}}
\def\BibTeX{{\rm B\kern-.05em{\sc i\kern-.025em b}\kern-.08em
    T\kern-.1667em\lower.7ex\hbox{E}\kern-.125emX}}

\usepackage{fancyhdr}
\fancyhf{}
\pagestyle{fancyplain}
\fancyhead[R]{\thepage}

% make sure bookmarks are created eventough sections are not numbered!
% uncommend if sections are numbered (bookmarks created by default)
\makeatletter
\renewcommand\@seccntformat[1]{}
\makeatother

% typo setup
\clubpenalty = 10000
\widowpenalty = 10000
\displaywidowpenalty = 10000

\usepackage{hyperxmp}
\usepackage[colorlinks, linkcolor=black,citecolor=black, urlcolor=libreas,
breaklinks= true,bookmarks=true,bookmarksopen=true]{hyperref}
\usepackage{breakurl}

%meta
%meta

\fancyhead[L]{K. Schuldt\\ %author
LIBREAS. Library Ideas, 37 (2020). % journal, issue, volume.
\href{https://doi.org/10.18452/21539}{\color{black}https://doi.org/10.18452/21539}
{}} % doi 
\fancyhead[R]{\thepage} %page number
\fancyfoot[L] {\ccLogo \ccAttribution\ \href{https://creativecommons.org/licenses/by/4.0/}{\color{black}Creative Commons BY 4.0}}  %licence
\fancyfoot[R] {ISSN: 1860-7950}

\title{\LARGE{Zur Beziehung von Forschung und Öffentlichen Bibliotheken}}% title
\author{Karsten Schuldt} % author

\setcounter{page}{1}

\hypersetup{%
      pdftitle={},
      pdfauthor={Karsten Schuldt},
      pdfcopyright={CC BY 4.0 International},
      pdfsubject={LIBREAS. Library Ideas, 37 (2020).},
      pdfkeywords={Öffentliche Bibliotheken, Angewandte Forschung, Bibliothekswissenschaft},
      pdflicenseurl={https://creativecommons.org/licenses/by/4.0/},
      pdfcontacturl={http://libreas.eu},
      baseurl={https://doi.org/10.18452/21539},
      pdflang={de},
      pdfmetalang={de}
     }



\date{}
\begin{document}

\maketitle
\thispagestyle{fancyplain} 

%abstracts

%body
\hypertarget{vorbemerkung}{%
\section{0. Vorbemerkung}\label{vorbemerkung}}

Ich soll Forschung für Öffentliche Bibliotheken betreiben. Angestellt
als Bibliothekswissenschaftler an einer Fachhochschule ist dies, neben
der Lehre, die explizite Hauptaufgabe meines Arbeitsalltags. Kantonales
Hochschulgesetz, aber auch das Rektorat und der Hochschulrat meiner
Hochschule schreiben dabei explizit \enquote{Angewandte Forschung} vor.
Praxisorientiert. Innovativ. All das. Ich arbeite auch schon eine ganze
Reihe Jahre so, in kleinen oder grossen Projekten, mit Texten, durch die
Betreuung von Praktika und studentischen Abschlussarbeiten, mit
Weiterbildungsangeboten, Vorträgen und theoretischen Reflexionen.
Manchmal auch mit kritischen Anmerkungen.

Das alles ist Teil meines Lebens. Aber es ist ein irritierender,
demotivierender Teil, der mich oft ans Hinschmeissen denken lässt. Es
ist nämlich der Teil meines Lebens, der mir oft am Sinnlosesten
erscheint. Wieso, das ist Thema dieses Textes.

Auf den ersten Blick scheint es nämlich so, als würden Bibliotheken alle
Forschung, die sie nutzen könnten -- nicht nur meine, darum geht es mir
nicht --, die in den letzten Jahren auch mehr und mehr als Open Access
vorliegt, willfährig ignorieren und statt dessen den oft unfundierten,
teilweise grotesk falschen Behauptungen meinungsstarker Einzelner
folgen. Wozu dann aber überhaupt Forschen, wenn das bessere, fundiertere
Argument oft ignoriert wird? Ich weiss nicht, wie die Kolleg*innen an
anderen Fachhochschulen diese zum Teil persönlich frustrierende
Situation meistern. Ich habe mir Hobbys zugelegt, flüchte zum Teil in
Themen und Fragestellungen, die tatsächlich explizit fern der
Bibliothekspraxis sind. Vor allem aber versuche ich, die Situation
selber als Untersuchungsgegenstand zu nehmen und dabei zu abstrahieren.
Weg vom Persönlichen, hin zum Strukturellen. Wie in diesem Text.

Weil, um hier schon eine Antwort vorwegzugreifen, die Situation (das
Problem?) eine strukturelle ist. Eine, die angegangen werden kann, aber
vor allem von Seiten der Bibliotheken. Dazu muss die Struktur wohl
benannt werden. Die Forschung macht schon viel richtig. Sie muss
vielleicht von der Praxisorientierung loskommen, aber das ist eine
politische Frage, die nicht zwischen Forschung und Bibliothekswesen
alleine geregelt werden kann.

\hypertarget{das-problem}{%
\section{1. Das Problem}\label{das-problem}}

Nicht oft wird sich heutzutage darüber Gedanken gemacht, wie Forschung
für Öffentliche Bibliotheken sein soll, welche Themen, Probleme und
Fragen sie bearbeiten und was sie, knapp gesagt, Bibliotheken vermitteln
soll. Wenn sich aber Bibliotheken oder Bibliothekar*innen dazu äussern--
und in meiner Position höre ich manchmal, wohl öfter als andere, davon
--, geht es eigentlich immer in eine Richtung: Die Forschung, so die
Forderung, solle den Bibliotheken Richtlinien, Vorschläge, im Mindesten
Hinweise dazu liefern, wie sie Probleme, die die Bibliotheken sehen,
lösen und wie sie, die Bibliotheken, in Zukunft werden sollen. Es geht
immer wieder darum, dass die Forschung Lösungen liefern soll.

Die geforderten Lösungen sollen oft bestimmte Bedingungen erfüllen:

\begin{enumerate}
\def\labelenumi{(\arabic{enumi})}
\item
  Lösungen sollen umsetzbar sein. Möglichst genau und direkt.
  \enquote{Pfannenfertig}, wie man in der Schweiz sagt.
\item
  Gleichzeitig nicht zu konkret. Bibliotheken wollen sich auch nicht
  vorschreiben lassen, was sie tun sollen, sondern selber entscheiden.
\item
  Die Lösungen sollen positiv sein. Der Nachweis, dass bestimmte
  Vorstellungen, die Bibliotheken entwickeln, oder Hoffnungen, die sie
  an bestimmte Veränderungen binden, nicht zutreffen, ist nicht
  gewünscht.\footnote{Etwas, was mich immer wieder tief zugrunde
    irritiert, ist das Beharren von Bibliotheken darauf, Ressourcen in
    Projekte, neue Angebote und so weiter zu stecken, wenn eigentlich
    klar zu sehen ist, dass diese dadurch verschwendet werden. Ich weiss
    nie, woher die einzelnen Bibliotheken das Selbstbewusstsein nehmen,
    dass sie es besser könnten als all die Bibliotheken (und anderen
    Einrichtungen) vor ihnen, die oft mit den gleichen oder ähnlichen
    Ideen schon gescheitert sind.}
\end{enumerate}

Selbstverständlich ist das widersprüchlich. Wenn man davon ausgeht, dass
Forschung immer die systematische Gewinnung von neuem Wissen ist, kann
der dritte Punkt oft gar nicht erfüllt werden. Der erste und zweite
Punkt lässt sich eigentlich auch nicht gleichzeitig erfüllen.

Und dennoch, schaut man sich an, was an Fachhochschulen im DACH-Raum an
Projekten für Bibliotheken durchgeführt wird -- nicht nur, aber
selbstverständlich auch durch mich -- und vor allem, was für Forschungen
in Abschlussarbeiten oder Seminaren unternommen werden, sieht man immer
und immer wieder, dass versucht wird, diese Ansprüche irgendwie zu
erfüllen. Immer und immer wieder wird die Praxisorientierung
hochgehalten. Immer und immer wieder werden Lösungen für wahrgenommene
Probleme und Herausforderungen erarbeitet.\footnote{Darunter leidet alle
  andere notwendige Arbeit, um die Bibliothekswissenschaft als
  Wissenschaft weiter zu bringen: Theoriearbeit, Methodenentwicklung,
  kohärente Forschungsprogramme sind so nicht möglich. Aber das soll
  hier nicht Thema sein.} Das Studium in jeder Hochschule wird immer und
immer wieder an den wahrgenommenen Interessen der Bibliotheken
ausgerichtet, mit jeder Reform des Studiums neu. Viele Projekte
beinhalten ein Arbeitspaket, in welchem die jeweiligen Ergebnisse zu
Bibliotheken gebracht werden, um dort noch einmal besprochen zu werden,
damit sie auch wirklich für die Praxis genutzt werden können. So viele
Handreichungen, To-Do-Listen, Hand- und Praxisbücher werden publiziert
-- teilweise alleine, teilweise als Teil von Arbeiten, beispielsweise am
Ende zahlloser Bachelor- und Masterthesen --, dass es teilweise so
scheint, als gäbe es keine anderen Publikationsformen mehr. Und immer
mehr davon erscheint, wie schon gesagt, als Open Access.

Die Erfahrung ist aber, dass diese Ergebnisse von der Praxis nicht
genutzt werden, zumindest nicht so, wie es zu erwarten wäre. (Vielmehr
tauchen solche Themen manchmal nach einigen Jahren ganz unvermittelt
auf, oft in unerwarteten Anfragen.) Die Erfahrung ist eigentlich immer
ähnlich:

\begin{enumerate}
\def\labelenumi{(\arabic{enumi})}
\item
  In den meisten Fällen wird die Forschung von Öffentlichen Bibliotheken
  schlichtweg ignoriert. So oft tauchen Fragen auf, die schon lange mit
  einer Handreichung, Bachelorarbeit und so weiter beantwortet worden
  sind, dass es schwer fällt, an das zufällige Übergehen dieser
  Publikationen zu glauben. Es scheint eher ein systematisches
  Nicht-Wahrnehmen zu sein.
\item
  Wenn sie doch wahrgenommen wird, wird oft der Eindruck vermittelt, die
  Forschung sei zu komplex, zu lang, zu schwierig dargestellt. Die
  genauen Formulierungen sind verschieden und werden unterschiedlich
  höflich vorgetragen. Aber immer wieder wird mehr oder minder
  konstatiert, dass niemand die Zeit hätte, \enquote{all das zu lesen},
  zu verstehen und so weiter. Das trifft ganz unterschiedliche Formen
  von Publikationen: Kurze Artikel, Handreichungen, Blogposts,
  Monographien. Das scheint selten mit dem Inhalt oder der Form der
  Präsentation zu tun zu haben.
\item
  Hinzu kommt eine teilweise offensiv vorgetragene Haltung, der
  Forschung vorzuwerfen, nicht in der Praxis anwendbar zu sein. Sie sei
  zu weit von der Realität entfernt, wäre nicht direkt genug umzusetzen,
  würde über Dinge reden, die mit der tatsächlichen Bibliotheksarbeit
  wenig oder nichts zu tun hätten. Auch hier ist oft nicht ersichtlich,
  ob der tatsächliche Inhalt der jeweiligen Forschung einen Einfluss auf
  diese Haltung hat oder ob sie nicht einfach grundsätzlich ist.
\end{enumerate}

Schaut man sich diese Situation aus einem gedachten Aussenpunkt an,
scheint das Problem, wenn es überhaupt eines gibt, nicht unbedingt bei
der Forschung oder dem Anspruch der Praxisorientierung zu
liegen.\footnote{Auch wenn man einzelne Forschungsprojekte, -fragen,
  -ansätze sehr wohl kritisieren kann; aber dann vor allem an Kriterien
  von Forschung selber, kaum dafür, praxisfern zu sein.} Es scheint eher
so, als wären Praxis und Forschung zwei getrennte Welten. Dabei ist
klar: Forschung kann Wissen generieren, aber nicht umsetzen. Das müssen
immer andere machen, im hier diskutierten Fall die Bibliotheken selber.
Dazu muss dieses Wissen wahrgenommen werden. (Und das heisst meist zu
lesen, weil Forschung trotz allen Ansätzen, andere Publikationsformen
wie Software oder Daten wertzuschätzen, sich weiterhin hauptsächlich
schriftlich ausdrückt.)

Dieser Text soll, ist, die oben genannte Betrachtungsweise des
Verhältnisses von Forschung und Öffentlichen Bibliotheken einmal
umzudrehen. Anstatt zu fordern, dass Forschung dieses oder jenes tun
solle, wird hier gefragt, welche Voraussetzungen es auf der Seite
Öffentlicher Bibliotheken geben müsste, um die Forschung, die es schon
gibt, zu nutzen.\footnote{Diese Überlegungen gehen von Forschung direkt
  über und für Bibliotheken aus. Selbstverständlich liegen auch in
  anderen Disziplinen Ergebnisse vor, die für Bibliotheken relevant
  sind, beispielsweise in der Pädagogik oder der Soziologie.}

\hypertarget{vignette-eins}{%
\subsection{Vignette eins}\label{vignette-eins}}

\begin{quote}
\emph{Praktikumsbesuch in einer Öffentlichen Bibliothek in der Schweiz.
Aufgabe des Studierenden, den ich von Seiten meiner Fachhochschule aus
betreue, ist es, den Umbau des Bestandes in einem Teil der Bibliothek
vorzubereiten. Es gilt, einen Plan zu machen: Welche Medien sollen
wohin? Welche sollen bleiben, wo sie sind? Welche vielleicht ganz
deakquisiert werden? Der Plan wird von der Gruppe Kolleg*innen
erarbeitet, die für die betreffende Etage zuständig ist. Der Umbau ist
Teil eines Strategieprozesses für die ganze Bibliothek. Der Studierende
soll den Blick von Aussen liefern. Die Entscheidungen aber werden im
Team getroffen.}

\emph{Auf meinen Hinweis im Gespräch vor dem Praktikum hin hat der
Studierende Bibliotheken mit ähnlichen Projekten herausgesucht und diese
gefragt, wie sie vorgegangen sind. Das gab eine gewisse Wissensbasis aus
Erfahrungen und Überlegungen aus anderen Einrichtungen.}

\emph{Ich stelle jetzt aus Interesse die Frage auf welcher Basis in der
Bibliothek sonst Entscheidungen über den Bestandesaufbau getroffen
werden. Dies bringt den Studenten und die ihn betreuende Bibliothekarin
etwas ins Schlingern. Offenbar sind die Quellen sonst vor allem interne
Diskussionen und Vorstellungen der beteiligten Kolleg*innen. Sonst
wenig. Keine Empirie, keine Rezeption der existierenden Forschung, keine
Theorie, keine Erfahrungen aus anderen Bibliotheken. Kein Wunder, dass
die -- sicherlich mit Engagement erarbeiteten -- Ergebnisse sich am Ende
sehr im erwartbaren Rahmen halten.}

\emph{Ich stelle solche Fragen heute vorsichtiger. Meine Erfahrung ist,
dass Bibliotheken in Situationen, in denen ich es sinnvoll fände, den
vorhandenen Forschungs- und Erfahrungsstand zu recherchieren, oft nicht
fragen, ob es solch eine Forschung und solche Erfahrungen überhaupt
gibt.}
\end{quote}

\hypertarget{wie-treffen-bibliotheken-entscheidungen}{%
\section{2. Wie treffen Bibliotheken
Entscheidungen?}\label{wie-treffen-bibliotheken-entscheidungen}}

Wie und wann könnten Bibliotheken Ergebnisse aus der Forschung sinnvoll
nutzen? Wohl vor allem dann, wenn Entscheidungen getroffen werden:
Darüber, welche Angebote abgeschafft, beibehalten oder neu eingerichtet
werden. Dann, wenn Abläufe überprüft oder verändert werden. Und dann,
wenn strategische Ausrichtungen festgeschrieben werden, beispielsweise
in Strategieprozessen oder bei Neu- und Umbauten. In solchen nicht
alltäglich aber doch regelmässig auftauchenden Situationen könnten
Bibliotheken sinnvoll das Wissen nutzen, welches von der Forschung (und
sei es in Abschlussarbeiten) produziert wurde, um bessere Entscheidungen
zu treffen, um auf schon vorhandenem Wissen aufzubauen und um nicht
nochmal die Fehler zu wiederholen, die andere schon gemacht haben.

Wie aber treffen Bibliotheken solche Entscheidungen? Das ist erstaunlich
schwierig festzustellen. Es gibt praktisch keine direkten Einblicke in
konkrete Entscheidungsprozesse. Sie sind praktisch nie Thema von Studien
oder Fachpublikationen. Sicherlich gibt es eine ganze Reihe von
Publikationen, die unter dem Schlagwort \enquote{Bibliotheksmanagement}
diskutieren, wie solche Entscheidungsprozesse sein sollten. Aber auch
diese basieren in der Regel nicht auf Analysen der tatsächlichen
Entscheidungsprozesse, sondern machen vor allem Aussagen darüber, wie
sie sein könnten.\footnote{Manchmal fliesst in diese Texte das Wissen
  der Autor*innen um die Entscheidungsprozesse in den jeweils eigenen
  Bibliotheken ein, aber solche sind schwerlich verallgemeinerbar.}
Festzustellen ist zudem, dass derzeit eine ganze Reihe von
Entscheidungsmethoden, die eine breite Beteiligung von
Bibliothekar*innen unterschiedlicher Ebenen einer Bibliothek
ermöglichen, relativ grosse Begeisterung hervorzurufen scheinen, sowohl
in der Literatur als auch in der Praxis. Es scheint dabei oft so, als
sei diese Beteiligung etwas Neues, und als wären bislang Entscheidungen
nicht unter Beteiligung des Personals getroffen worden. Das ist
vielleicht ein Hinweis darauf, wer sonst Entscheidungen trifft: Die
Bibliotheksleitungen, nicht das Bibliothekspersonal.

Ansonsten kommen die meisten Hinweise dazu, wie die
Entscheidungsprozesse in Bibliotheken aufgestellt sind, vor allem von
Kolleg*innen, die von diesen enttäuscht sind. Diese Bemerkungen -- fast
nie dokumentierbar gemacht, also nicht in Artikeln, Blopgposts, bei
öffentlichen Auftritten; sondern unter der Hand, bei persönlichen
Treffen, auf den Fluren von Konferenzen, während individueller
Kommunikation, ganz selten einmal auf privaten Social Media-Accounts,
dafür aber recht oft -- sind erwartungsgemäss zynisch und überspitzt.
Deshalb sollte man sie aber nicht als unbrauchbar verwerfen, sondern sie
vorsichtig interpretieren. Nicht nur gibt es sonst ja nur wenig andere,
indirekte Einblicke. Solche Anmerkungen würden nicht gemacht werden,
wenn das Thema für die Kolleg*innen irrelevant wäre. Gleichzeitig ist
klar, dass die Situation nicht so schlecht und schon gar nicht überall
so ist, wie es aus diesen Anmerkungen spricht.\footnote{Die Frage der
  Reproduzierbarkeit ist damit selbstverständlich nicht beantwortet. Ich
  gehe hier ein wenig davon aus, dass meiner Darstellung einfach gefolgt
  wird -- wovon ich eigentlich immer abrate. Aber alle mit Kontakten im
  Bibliothekswesen können die Reproduktion selber angehen, indem sie im
  eigenen Rahmen Kolleg*innen ausserhalb der Leitungsebene aus
  verschiedenen Bibliotheken befragen. Es würde mich wundern, wenn deren
  Kommentare erheblich anders wären.}

Aber nehmen wir einmal zusammen, was wir aus solchen Anmerkungen und
indirekten Beobachtungen schliessen können, dann scheinen sich
Entscheidungen in Bibliotheken durch folgende Kriterien auszuzeichnen:

\begin{enumerate}
\def\labelenumi{(\arabic{enumi})}
\item
  Viele Entscheidungen werden intern in kleinen Gruppen, oft bei der
  Bibliotheksleitung, getroffen. In kleinen Bibliotheken scheint das oft
  zu heissen, dass tatsächlich alle Kolleg*innen einbezogen werden, aber
  ab einer bestimmten Grösse differenziert sich das. Viele Klagen
  beziehen sich darauf, dass sich Kolleg*innen aus Entscheidungen
  ausgeschlossen fühlen; verstärkt dann, wenn externe Berater*innen in
  die Entscheidungsprozesse einbezogen werden.
\item
  Bibliotheken beziehen noch recht oft andere Bibliotheken und deren
  Lösungen für bestimmte Probleme in die eigenen Entscheidungen mit ein,
  aber in einer recht spezifischen Form: Selten scheint nach den
  tatsächlichen Erfahrungen oder Überlegungen von anderen Bibliotheken
  gefragt zu werden. Vielmehr überwiegt der Modus der
  \enquote{Anregung}. Diese und ähnliche Worte werden oft benutzt, wenn
  Bibliothekar*innen beschreiben, was sie von anderen Bibliotheken
  erwarten. Es geht um \enquote{Ideen}, \enquote{Vorschläge},
  \enquote{Best Practice}, nicht um nachvollziehbare und überprüfbare
  Fakten.\footnote{So lässt sich vielleicht erklären, wieso in
    Bibliotheken bestimmte Konzepte weiter als zukunftsträchtig gelten,
    beispielsweise in den veröffentlichten Bibliotheksstrategien und
    Jahresberichten, die sich in anderen Bibliotheken schon als wenig
    wirksam gezeigt haben oder gar schon wieder eingestellt wurden.}
  (Zudem: Von Bibliotheken werden umgekehrt überwiegend datenarme und
  positive Darstellungen publiziert, die dann für Entscheidungsprozesse
  anderer Bibliotheken herangezogen werden, obwohl sie offensichtlich
  nicht das ganze Bild zeigen.)
\item
  Viele Entscheidungen, gerade in grösseren Bibliotheken, scheinen von
  einer kleinen Anzahl meinungsstarker Einzelpersonen geprägt zu sein,
  die immer wieder als Berater*innen herangezogen werden. Ihr Einfluss
  wird wohl als positiv wahrgenommenen, sonst würden sie nicht immer
  wieder für solche Beratungsaufträge ausgewählt werden. Erstaunlich ist
  jedoch, dass Bibliotheken kaum die Aussagen dieser Berater*innen
  hinterfragen oder mehr Fakten oder Daten einzufordern scheinen -- und
  dass sie zum Teil selbst dann auf sie zurückgreifen, wenn sie sie
  kritisch sehen. Das scheinen Hinweise darauf zu sein, dass die Aufgabe
  der Berater*innen nicht ist, Fakten oder geprüftes Wissen in die
  Bibliothek zu bringen, sondern dass sie vielleicht andere Funktionen
  übernehmen.
\item
  Hingegen scheint systematisch erhobenes Wissen - das eine bessere
  Basis für Entscheidungen wäre - bei den meisten Entscheidungsprozessen
  wenig Beachtung zu finden. Bibliotheken selber sammeln kaum
  systematisch Erfahrungen aus ähnlichen Bibliotheken. Gerade dieses
  Wissen ist es aber, welches in der Forschung erhoben wird.\footnote{Damit
    ist selbstverständlich nicht gesagt, dass solches Wissen perfekt
    wäre. Aber das soll in diesem Text auch nicht Thema sein.}
\item
  Auffällig ist in den letzten Jahren auch die Begeisterung für
  spezifische Methoden für die Entscheidungsfindung, die regelmässig als
  neu oder innovativ beschrieben werden -- obwohl sie ja eigentlich
  irgendwann nicht mehr neu sind --, welche recht losgelöst vom
  eigentlichen Thema -- und der Kritik an ihnen, die es meist in den
  Wissensfeldern, aus denen die Methoden stammen, gibt -- als Weg zu
  Entscheidungen dargestellt werden. (Aktuell wieder einmal solche, die
  als partizipativ beschrieben werden.) Dies ist ein No-Go in der
  Wissenschaft. Die Methode ist ein Werkzeug, dass zur Frage passen
  muss; nicht andersherum. In Bibliotheken scheint das nicht zu gelten.
\end{enumerate}

Diese Punkte kennzeichnen eine Struktur, die schwer zu
\enquote{beweisen} ist, die aber viele tatsächlich getroffene
Entscheidungen in Bibliotheken besser erklärt, als die einfach Annahme,
dass Bibliotheken immer nach der bestmöglichen, faktisch untermauerten
Lösung suchen würden.

Schaut man sich an, welche Entscheidungen in Bibliotheken tatsächlich
getroffen werden, beispielsweise wie und welche neue Bibliotheksgebäude
gebaut oder alte umgebaut werden, welche neuen Angebote geschaffen
werden und was in Bibliotheksstrategien und Jahresberichten geschrieben
wird, fallen -- unter dem Blickwinkel, ob und wie Forschung hier eine
Rollen spielen könnte -- drei Dinge auf, die sich mit dieser Struktur
erklären lassen:

\begin{enumerate}
\def\labelenumi{(\arabic{enumi})}
\item
  Viele Entscheidung scheinen -- auch wenn man direkt danach fragt --
  aus Überzeugungen, Hoffnungen, Wünschen, Interessen, internen
  Traditionen \enquote{im Haus} heraus getroffen werden. , Fast nie wird
  auf eine Datenbasis verwiesen, die einer kritischen Überprüfung
  standhalten würde. Viele Entscheidungen könnten auch, oft besser oder
  anders, auf Basis vorhandener Forschung getroffen werden, werden es
  aber offenbar nicht. Zumindest ist das fast nie sichtbar.
  Bibliotheksstrategien verweisen in der Regel auch nicht darauf, dass
  sie auf vorhandenes Wissen aufbauten. Dies scheint im Bibliothekswesen
  -- im Gegensatz zur Forschung -- akzeptabel zu sein.
\item
  Auffällig ist dagegen , dass viele Entscheidungen entgegen vorhandenem
  Wissen getroffen werden. Das mag nicht auffallen, wenn dieses Wissen
  im Entscheidungsprozess nicht aktiv gesucht wird, aber es ist
  tatsächlich nicht selten, dass Bibliotheken sich beispielsweise in
  neuen Angeboten versuchen oder Umbauten des Raumes mit bestimmten
  Überzeugungen angehen, bei denen -- nicht unbedingt immer in der
  Bibliothekswissenschaft, aber oft in anderen Disziplinen --
  ausreichend Wissen vorliegt, welches Grenzen dieser neuen Angebote und
  Hoffnungen schnell erkennen lässt. Bibliotheken investieren trotzdem
  die jeweils notwendigen Ressourcen.
\item
  Mit einer längeren Perspektive fällt auf, dass sich über die
  Jahrzehnte Diskurse, Themen, die als Probleme oder Gefahren für
  Bibliotheken beschrieben werden und auch die jeweils vorgeschlagenen
  Lösungen wiederholen. Es scheint oft, als würden Bibliotheken nicht
  aus dem Verlauf von Diskussionen und Entwicklungen lernen, sondern
  immer wieder zu ähnlichen Punkten zurückkehren, ähnliche Vermutungen
  äussern, ähnliche Situationen als Krise wahrnehmen, ähnliche
  Hoffnungen entwickeln und dann wieder ähnliche Lösungen versuchen --
  nur teilweise unter neuem Namen.
\end{enumerate}

\hypertarget{vignette-zwei}{%
\subsection{Vignette zwei}\label{vignette-zwei}}

\begin{quote}
\emph{Ein grösseres Netz von Öffentlichen Bibliotheken in Deutschland
hat eine Projektförderung erhalten. Der Leiter dieses Netzes, so erzählt
er mir, hat jetzt die Aufgabe, \enquote{herauszufinden}, wie
Programmierworkshops für eine spezifische Zielgruppe in der Bibliothek
gestaltet werden können, auch damit diese Gruppe früh mit dem
Programmieren in Kontakt kommt und dann in Zukunft mehr Personen dieser
Zielgruppe selber programmieren und darauf eine Karriere aufbauen.
Dieser Leiter ist weithin dafür bekannt, grosses Engagement zu zeigen.
Es ist also nicht so, als würde das zynisch klingen. Ich nehme ihm
sofort ab, dass er es ernst meint.}

\emph{Und trotzdem -- oder genau deshalb -- bin ich irritiert.
Programmierworkshops, auch für spezifische Zielgruppen, sind keine neue
Sache. Wie sie funktionieren können, ist immer und immer wieder
untersucht worden. Nicht seit Jahren, sondern seit Jahrzehnten ist dies
Thema erziehungswissenschaftlicher Forschung und Praxis in verschiedenen
Einrichtungen: In Schulen, in Jugendclubs und anderswo. Es gibt
ungezählte Konzepte, Modelle und auch Forschung zu dieser Frage.
Programmierworkshops in Bibliotheken sind immer wieder Thema
studentischer Abschlussarbeiten. Auch da ist viel Vorarbeit geleistet
worden. Aber so, wie mir dieser Leiter es erzählt, steht er ganz am
Anfang. Wie kann das sein? Wie hat er einen Projektantrag schreiben
können, ohne die vorhandene Forschung einzubeziehen? Wie konnte so ein
Antrag erfolgreich sein? Hat niemand recherchiert, ob dieser überhaupt
irgendwie untermauert ist? Bin ich zu sehr vom Schreiben
wissenschaftlicher Anträge geprägt, dass mir das absonderlich vorkommt?}

\emph{Schlimmer noch scheint mir seine Anmerkung, mit diesem Projekt
solle die spezifische Zielgruppe gefördert werden. Auch das ist wirklich
keine neue Idee. Praktisch in allen technisch-wissenschaftlichen
Bereichen laufen seit Jahrzehnten Programme, um unterschiedliche
unterrepräsentierte Gruppen an den jeweiligen Bereiche
\enquote{heranzuführen} -- und seit Jahrzehnten funktioniert das nicht
(sonst wären sie ja nicht mehr notwendig). Das Problem ist, wie auch
schon oft in der betreffenden Forschung diskutiert, nicht, das Menschen
aus solchen Gruppen zu wenig Interesse am Programmieren oder Physik oder
Ähnlichem hätten. Das Problem sind die sexistischen, rassistischen und
ähnlichen ausschliessenden Strukturen und Kulturen in diesen Bereichen,
die sexistisches, rassistisches und anderes ausschliessendes Denken und
Verhalten befördern und dazu führen, dass über die Jahre und
Karrierestufen immer mehr Angehörige unterrepräsentierter Gruppen diese
Bereiche wieder verlassen. Das ist vorliegendes und einfach zu findendes
Wissen. Wie sollte das bei diesem Projekt anders sein? Warum scheint der
Leiter dieses Bibliothekssystems das nicht wahrzunehmen? Wie gesagt:
Wäre er zynisch, würde ich vermuten, er wollte einfach nur das
Projektgeld für seine Bibliotheken. Aber ich denke, es ist etwas anders:
Er scheint dieses Wissen nicht zu nutzen. Das scheint ihm -- und
anderen, die solche Projektanträge akzeptieren -- nicht in den Sinn zu
kommen.}
\end{quote}

\hypertarget{kuxf6nnen-bibliotheken-wissen-aus-der-forschung-integrieren}{%
\section{3. Können Bibliotheken Wissen aus der Forschung
integrieren?}\label{kuxf6nnen-bibliotheken-wissen-aus-der-forschung-integrieren}}

In diesem Text denke ich nicht das erste Mal über den Zusammenhang von
Forschung und Öffentlichen Bibliotheken nach. Alles, was hier eher
abstrakt dargestellt wurde, könnte ich mit vielen, vielen Beispielen
untermauern. Es soll aber nicht der Eindruck entstehen, es ginge mir um
diese einzelnen, konkreten Beispiele oder darum, die jeweils neusten
Trends oder gar bestimmte Einzelpersonen zu kritisieren.\footnote{Und
  falls sich Einzelpersonen oder -institutionen in den Vignetten
  wiederzuentdecken glauben, können sie beruhigt sein. So oder so
  ähnlich hat es sich öfter abgespielt, bevor es hier zur Vignette
  wurde. Die Vignetten sind anonymisiert und inhaltlich leicht
  verschoben. Wer meint, erkannt worden zu sein, kann sich sicher sein:
  Die Situation ist mir mehr als einmal begegnet. Niemand ist
  \enquote{selber} gemeint, sondern immer die Struktur. (Dies liesse
  sich auch umdrehen: Ich bin mir sicher, dass andere Personen, wenn
  zufällig sie auf meine Position gelangt wären, ähnlich denken und
  argumentieren würden wie ich.)} Auf einem meiner Schreibtische liegt
die allererste Version eines Buches zum Thema (das vielleicht, wie
andere Bücher auch, nie über diesen Status hinauskommen wird). Eine
Frage, auf die ich dabei immer wieder gestossen bin -- und wegen der ich
das vorhergehende Kapitel wichtig finde -- ist diese: Gibt es überhaupt
einen Ort, einen Moment, einen Punkt, an dem Bibliotheken die
vorliegende Forschung integrieren können? Ich würde diese Frage hiermit
gerne dem Bibliothekswesen vorlegen. Wie und wann wären Bibliotheken
aufgestellt, um innerhalb ihrer Entscheidungsprozesse von dem Wissen,
welches in der Forschung produziert wird, zu profitieren?

Grundsätzlich scheinen nämlich im Alltag und auch in Projekten in
Öffentlichen Bibliotheken dafür gar keine Zeit und keine anderen
Ressourcen vorgesehen zu sein. Das schliesst nicht aus, dass einzelne
Kolleg*innen Anstrengungen unternehmen, Forschungen zu sichten,
wahrzunehmen, aus ihr zu lernen. Aber solange Entscheidungsprozesse
strukturell so organisiert sind, wie oben geschildert, würde sich
erklären, warum Forschung und Bibliothekspraxis nicht zusammenkommen,
egal wie praxisnah die Forschung zu sein versucht.

Dass die Bibliotheken im Allgemeinen Entscheidungen nicht so treffen,
dass sie jeweils strukturiert erarbeitetes und damit abgesichertes
Wissen nutzen, sondern vor allem \enquote{Anregungen} für eigene
Entscheidungsprozesse suchen, die sie intern verarbeiten können, führt
dann auch dazu, dass Bibliotheken von Forschung teilweise Dinge
erwarten, die diese gar nicht liefern kann.

Forschung liefert systematisch erarbeitetes Wissen in Form von Empirie,
Modellen, Tests von Annahmen, Theorie und teilweise auch, indem Fragen
\enquote{durchdacht} werden. Nicht immer gut, aber grundsätzlich
öffentlich und transparent (also dargestellt in einer Weise, die das
Entstehen des Wissen nachvollziehbar und wiederholbar macht) und
prinzipiell nicht im luftleeren Raum, sondern auf der Basis schon
vorhandenen Wissens. Diese Wissen ermöglicht Aussagen über die Realität.

Gefordert wird hingegen oft, dass Forschung Aussagen über die Zukunft
macht, Lösungen für Probleme (die die Bibliotheken als Probleme
wahrnehmen) liefert und Trends benennt. Oder anders: Forschung soll das
liefern, was man wohl auch von Berater*innen erwartet oder was man
sucht, wenn man bei anderen Bibliotheken nach \enquote{Anregungen}
schaut. Forschung wird so in die vorhandene Struktur, wie Bibliotheken
Wissen wahrnehmen, eingefügt. Aber das ist nicht das, was Forschung
kann.

\hypertarget{vignette-drei}{%
\subsection{Vignette drei}\label{vignette-drei}}

\begin{quote}
\emph{Eine Öffentliche Bibliothek, mit der meine Fachhochschule ein
überaus gutes Verhältnis hat, die Kolleg*innen zum Studieren zu uns
schicken, Praktika für unsere Studierenden anbieten und auch sonst
regelmässigen Kontakt hält und Interesse zeigt, liess uns vor einigen
Jahren Interviews mit Nutzenden durchführen. Es ging, im Rahmen eines
grösseren Strategieprozesses, darum, zu erfahren, wie diese die konkrete
Bibliothek und ihre Angebote wahrnehmen. Eine Standardfrage, dann auch
mit recht erwartbaren Ergebnissen.}

\emph{Was auffiel -- auch in ähnlichen Projekten in anderen
schweizerischen Bibliotheken -- war, dass die befragten Jugendlichen vor
allem ein Interesse daran hatten, die Bibliothek als Rückzugsort zu
benutzen -- nicht einmal so sehr als \enquote{Jugendliche}, die von
Erwachsenen in Ruhe gelassen werden wollen, sondern individuell, als
Personen, die von anderen Personen in Ruhe gelassen sein wollten, -- um
dort vor allem gedruckte Bücher auszuborgen und zu lesen. Solche, die
sie sich selber aussuchen, nicht Lehrpersonen oder ähnliche Autoritäten.
Alles andere -- DVDs, Musik, E-Books, ein möglicher Makerspace -- war
ihnen egal. Die Ergebnisse waren sehr eindeutig.}

\emph{Ein paar Jahre später ein Besuch in dieser Bibliothek. Mir wird
unter anderem der Plan für die neue Jugendabteilung vorgelegt: Wenig
Bücher, mehr elektronische Medien, ein Makerspace, mehr offene Fläche.
Oder, anders ausgedrückt: Das, was andere Bibliotheken auch machen. Aber
das Gegenteil davon, was sich die befragten jugendlichen Nutzer*innen
vorstellten. Das ist kein Einzelfall. Bibliotheken begeistern sich oft
für bestimmte Entwicklungen, auch wenn die Empirie oder Theorie noch
mehr dagegen sprichen, als in diesem Fall, wo immerhin die Hoffnung
besteht, so andere Jugendliche zu erreichen als die damals Befragten.
Was erstaunt, ist, dass diese Bibliothek die Ergebnisse von Projekten,
die sie selber finanziert hat, offensichtlich ignoriert.}
\end{quote}

\hypertarget{kuxf6nnte-es-anders-sein}{%
\section{4. Könnte es anders
sein?}\label{kuxf6nnte-es-anders-sein}}

Nehmen wir an, meine Beschreibung stimmt, dann wäre es eine Erklärung
für die Situation, in der sich Forschung in Bezug auf die
Bibliothekspraxis befindet. Das zu benennen kann schon helfen,
Erwartungen und Hoffnungen realistischer zu gestalten.

Aber die interessantere Frage ist ja, ob es anders sein könnte und
anders sein sollte. Selbstverständlich ist das erst einmal eine müssige
Frage: Alles könnte immer anders sein. Alle Institutionen könnten anders
funktionieren, alle Strukturen könnten andere Strukturen sein. Die Frage
ist, warum sie so sind, wie sie sind.

Aber wie wahrscheinlich wäre es überhaupt, dass Bibliotheken eine andere
Struktur für ihre Entscheidungen entwickeln können?

Die Antwort darauf scheint mir nicht eindeutig zu sein. Das Öffentliche
Bibliothekswesen in der DDR beispielsweise deutet eher darauf hin, dass
diese Struktur zu Öffentlichen Bibliotheken gehört. Dies mag
überraschen, war es doch ein Bibliothekswesen in einem anderen
politischen und gesellschaftlichen System als die Bibliothekswesen im
DACH-Raum heute. Es gab Direktiven und andere verbindliche Vorgaben. Die
Bibliotheken wurden als zusammenhängendes Netz verstanden, auf das
zentral eingewirkt werden konnte. Es gab mit dem Zentralinstitut für
Bibliothekswesen eine Institution, welche explizit praxisorientierte
Forschung -- selbstverständlich im politischen Rahmen der DDR, also
beispielsweise mit regelmässiger \enquote{Auswertung} jeweils aktueller
politischer Dokumente -- betrieb. Zahlreiche Studien zur Nutzung von
Bibliotheken, zur Rationalisierung von Bibliotheksarbeit, zur Planung
verschiedener Bibliothekstypen wurden von diesem durchgeführt und
publiziert.\footnote{Vergleiche die Reihe \enquote{Theorie und Praxis
  der Bibliotheksarbeit}, 1969-1989.} Viele Formulare, die für die
Planung von Bibliotheksarbeit genutzt werden sollten, wurden erstellt.
Und trotzdem es diese Einrichtung gab und obgleich die Bibliotheken in
der DDR einer zentralen Führung unterstanden, sind die Publikationen des
Instituts durchzogen von einem gewissen, immer wieder durchscheinenden,
Defätismus. Immer wieder wird darauf verwiesen, dass die vorhandenen
Formulare auch benutzt werden müssten, dass bestimmte Probleme schon in
früheren Publikationen geklärt worden seinen und das die Bibliotheken
dieses Wissen im Alltag nutzen müssten. Oder anders, auf das Thema
dieses Textes bezogen: Selbst unter diesen spezifischen Umständen, die
eigentlich die Nutzung des -- trotz aller ideologischen Verrenkungen --
systematisch erstellten Wissens befördern müssten, nutzten die
Bibliotheken es offenbar nicht. Zumindest nicht so wie erwartet.
Vielleicht ist es also Teil der Identität Öffentliche Bibliotheken
solches Wissen nicht nutzen, zumindest nicht direkt?

Für eine ähnliche Konstellation, die von Schulen (und Kindergärten) und
Erziehungswissenschaft, gibt es einige Untersuchungen. Wie nutzen
Schulen die Ergebnisse der erziehungswissenschaftlichen Forschung? Man
würde auch hier erwarten, dass sie dieses in ihr pädagogisches Handeln
integrieren. Schliesslich werden (im Vergleich zur
Bibliothekswissenschaft) für die Erziehungswissenschaft relativ viele
Ressourcen aufgewendet, um Modelle zu erstellen und zu Testen,
pädagogische und didaktische Interventionen zu gestalten und deren
Wirkung zu bestimmen und so weiter. Schulen könnten sich immer weiter
verändern, bei der Masse und Vielgestaltigkeit von
erziehungswissenschaftlicher Literatur auch in verschiedene,
selbstbestimmte Richtungen. Aber was die Forschung zum Transfer von
Wissen aus der Erziehungswissenschaft in die Schulpraxis immer wieder
zeigt, ist, dass nur eine kleine Zahl von Schulen diese wahrnehmen und
zur eigenen Entwicklung zu nutzen. Ein Grossteil entwickelt sich auf
anderer Basis, also eigenen Hoffnungen, Annahmen, Vermutungen. Das
Wissen aus der Erziehungswissenschaft kommt wenn, dann eher über den
Umweg der Kultusministerien beziehungsweise Erziehungsdirektionen in die
Schulen. Diese haben Strukturen entwickelt, das in den
Erziehungswissenschaften produzierte Wissen wahrzunehmen und umzusetzen.
Und gleichzeitig haben sie die Macht, Richtlinien zu erlassen (oder
anzuregen, dass die Politik sie erlässt).\footnote{Und auch das ist
  nicht einfach, wie diejenigen, welche die Einführung des Lehrplan 21
  in der Schweiz und Liechtenstein in den letzten Jahren verfolgt haben,
  wissen.} Dabei ist auch die Erziehungswissenschaft nicht
\enquote{praxisfern}, sondern unternimmt aktiv Schritte, ihre Ergebnisse
zu popularisieren, beispielsweise mit den regelmässig publizierten
\enquote{Nationalen Bildungsplänen}.

Insoweit befinden sich Bibliotheken vielleicht bei der Struktur ihrer
Entscheidungsprozesse im Einklang mit anderen Einrichtungen.\footnote{Wie
  ist zum Beispiel das Verhältnis von Museen und Museumsforschung, von
  Archiv und Archivwissenschaft? Das wäre eine interessante
  Anschlussfrage.} Das hiesse noch nicht, dass es so sein muss, wie es
ist. Aber es wäre ein Hinweis darauf, dass es so wie es ist nicht
ungewöhnlich ist und auch nicht spezifisch für Bibliotheken.

Gleichzeitig gibt es aber Felder, in denen es normal ist -- oder
zumindest zur professionellen Identität gehört, egal wie es von
einzelnen Vertreter*innen umgesetzt wird --, bei Entscheidungen das
vorhandene Wissen wahrzunehmen und einzubeziehen. In Verwaltungen und
Ministerien wird dies angestrebt, und wie gerade erwähnt beispielsweise
in Kultusministerien auch umgesetzt. In vielen Berufen, die mit Planung
und Umsetzung von Umweltschutz zu tun haben, ist es Normalität, auf
wissenschaftliches Wissen zurückzugreifen. Für die Medizin geht man
davon aus, dass sich das Personal ebenso kontinuierlich à jour hält. In
diesen und anderen Berufsfeldern ist es normal, dass Arbeitszeit für die
Recherchen, das Lesen und Verarbeiten von wissenschaftlicher Literatur
verwendet, Tagungen besucht und selber zum Diskurs beigetragen wird.
Warum sollte das im Bibliothekswesen nicht möglich sein?

Die Antwort auf die Frage, ob die Situation auch anders sein könnte,
lautet also wohl: Ja, aber dann würden sich Bibliotheken recht massiv
verändern.\footnote{Aber auch das hätte ein Vorbild im Bibliothekswesen.
  Die Transformation von der literaturorientierten zur
  nutzer*innenorientierten Bibliothek -- von der Theke zur Freihand --
  in den 1960er verschob auch massiv Strukturen, Denkweisen und die
  gesamte bibliothekarische Arbeit. Es ist also möglich, dass sich
  Bibliotheken grundlegend ändern.}

\hypertarget{vignette-vier}{%
\subsection{Vignette vier}\label{vignette-vier}}

\begin{quote}
\emph{Es gibt eine kleine, in den letzten Jahren aber sogar steigende,
Anzahl von Institutionen, bei denen Bibliotheken Geld für
Entwicklungsprojekte beantragen können. Die Summen, die ausgeschüttet
werden, die Zeiträume, die rechtlichen Strukturen der Institutionen sind
unterschiedlich. Die Ziele und Ausschreibungstexte hingegen sind recht
gleichförmig. Es geht immer wieder darum, dass Bibliotheken lokale
Projekte durchführen sollen, die gleichzeitig innovativ sein und dafür
sorgen sollen, dass sie sich mehr in ihre Gemeinde integrieren sowie
neue Rollen übernehmen. Immer muss dafür ein Antrag für das jeweilige
Projekt eingereiht werden.}

\emph{Ich bin ganz gut platziert, um entweder direkten Kontakt zu diesen
Einrichtungen zu haben oder unter der Hand aus ihnen zu hören. Alle
Institutionen haben das gleiche Problem, nämlich ihr Geld loszuwerden.
Es gibt viel weniger Eingaben, als sie sich erhoffen. Fast nie können
aus den Anträgen die Besten ausgesucht werden, weil es gar nicht
genügend für eine solche Wahl gibt. Form und Inhalt der eingereichten
Anträge entsprechen auch selten den Vorstellungen der Ausschreibenden.
Offenbar fällt es Bibliotheken schwer, Projekte zu entwerfen und diese
dann auch zu beschreiben -- was für Forschende wie mich, die das ständig
tun, erstaunlich ist; was aber auch erstaunlich ist, weil in vielen
Bibliotheken Kolleg*innen arbeiten, die an Fachhochschulen und
Universitäten studiert und dort zumindest in den letzten Jahren auch mit
Projektmanagement vertraut gemacht wurden. Es fällt Bibliotheken zudem
schwer, explizit neue Projekte zu entwerfen, also erst wahrzunehmen, was
es schon gibt und dann davon ausgehend etwas Eigenes, Neues zu
entwerfen. Viele Eingaben heben stattdessen offenbar darauf ab, etwas,
was es in anderen Bibliotheken schon gibt, noch einmal, nur halt lokal,
umzusetzen.}

\emph{Die mittelgebenden Institutionen behelfen sich: Mit Beratung der
Bibliotheken, mit Neu- und Umdefinition von Vorgaben, mit Auflagen bei
Projekten und anderem. Für mich -- der nicht das Problem hat, Geld
sinnvoll verteilen zu müssen -- scheint aber, dass das Problem
strukturell ist. Es gäbe so viel Wissen, auf das Bibliotheken
zurückgreifen könnten, um wirklich neue Projekte zu entwerfen oder aber
zumindest zu zeigen, warum ihre Projekte funktionieren würden. Es gibt
so viele Kompetenzen in Bibliotheken, wie man ein Projekt entwirft,
aufgleist und durchführt. Das ist seit Jahren Unterrichtsstoff im
Studium. Bibliotheken nutzen diese Kompetenzen nicht -- quer durch die
jeweils um Geld angegangenen Fördereinrichtungen, quer durch den
DACH-Raum (immer mit einzelnen Ausnahmen). Das ist eine Struktur.}
\end{quote}

\hypertarget{was-kuxf6nnte-anders-sein}{%
\section{5. Was könnte anders
sein?}\label{was-kuxf6nnte-anders-sein}}

Dieser Text beschreibt eine Situation, die zumindest, wenn man davon
ausgeht, dass die Aufgaben zwischen Forschung zum Bibliothekswesen auf
der einen Seite und dem Bibliothekswesen auf der anderen Seite, sinnvoll
verteilt sein sollten, negativ erscheint. Das ist sie aber nicht
unbedingt. Das wissenschaftliches Wissen nicht von den Einrichtungen
genutzt wird, die davon profitieren könnten, mag für das Funktionieren
dieser Einrichtungen, aber auch für die Wissenschaft als eigenständiges
Feld notwendig sein. Die Situation scheint wohl aus zwei Gründen eher
negativ: Zum einen, weil die Forschung gerade an Fachhochschulen immer
wieder dem Modus \enquote{anwendungsbezogener},
\enquote{praxisorientierter} Forschung unterworfen werden soll, was an
sich immer schwierig ist, aber kontraproduktiv wird, wenn die Praxis
keine Strukturen hat, um das in diesem Modus produzierte Wissen
überhaupt wahrzunehmen. Zum anderen, weil in den Fällen in denen die
Bibliothekspraxis sich einmal äussert, vor allem Forderungen in die
Richtung der Forschung gestellt werden, die diese oft schon erfüllt.

Es hat aber auch Vorteile, wenn diese beiden Ebenen -- Forschung und
Praxis -- getrennt voneinander existieren und funktionieren. (Nur so,
zum Beispiel, kann in der Forschung überhaupt abstrakt gedacht werden.)

In diesem Text habe ich anders über diese Situation nachzudenken
versucht. Es scheint klar zu sein, dass Bibliotheken sich eine ganze
Reihe von Möglichkeiten vergeben, wenn sie ihre Entscheidungsstrukturen
so lassen, wie sie sind. Sie verbrauchen Ressourcen für Projekte und
Entscheidungsprozesse, bei denen mit ein wenig mehr Recherche klar wäre,
dass sie nicht die angestrebten Ziele erreichen werden. Sie verbrauchen
unnötig Ressourcen, weil sie immer wieder zu ähnlichen Problemen und
Lösungen zurückkehren und nicht auf den schon einmal gemachten
Erfahrungen aufbauen. Die Beschreibung einer Situation macht es möglich,
über diese nachzudenken und zu entscheiden, ob sie verändert werden
soll. Aber diese Veränderung, wenn sie gewünscht wird, müssten
Bibliotheken selber durchführen.

Was wäre zu verändern?

\begin{enumerate}
\def\labelenumi{(\arabic{enumi})}
\item
  Bibliotheken müssen aktiv das Wahrnehmen von wissenschaftlichen Wissen
  in ihre Entscheidungsprozesse integrieren, ansonsten wird sich diese
  Struktur nicht verändern. Mindestens an den Punkten von
  Entscheidungen, in denen systematisch erhobenes Wissen eine sinnvolle
  Basis bieten würde, muss es normal werden, nach diesem zu suchen, es
  zu rezipieren und auch zu verwenden. Das bedeutet ganz konkret, Zeit
  aufzuwenden, um zu recherchieren, zu lesen, zu verstehen und
  auszuwerten. Und es bedeutet manchmal auch, eigene Wahrnehmungen
  gegenzuchecken, Hoffnungen aufzugeben oder -- viel eher -- zu
  erkennen, wenn man das machen will, was andere auch schon gemacht
  haben, und dann zu versuchen, auf deren Erfahrungen aufbauend
  weiterzugehen.
\item
  Dafür müsste es im professionellen Verständnis als Bibliothekar*in
  integriert werden, dieses Wissen wahrzunehmen und dafür Ressourcen
  einzusetzen, vor allem Zeit zum Lesen und Nachdenken. Wie Ressourcen,
  auch Arbeitszeit, verplant und was als Arbeit wertgeschätzt oder als
  sinnlos verworfen wird, ist immer eine politische Frage. Der Hinweis,
  man hätte \enquote{dafür} keine Zeit, ist immer ein Hinweis, dass
  diese Tätigkeit oder dieses Thema nicht als relevant angesehen wird.
  Das sagt immer etwas über das jeweilige Verständnis davon aus, was als
  professionell gilt, und über die Institution, an der man tätig ist.
  Aber gleichzeitig, beziehungsweise gerade deshalb, kann es immer auch
  geändert werden. Es ist kein unhintergehbares Argument. Professionen
  und Institutionen ändern sich. (Zumal in unserem Fall der Einsatz von
  Ressourcen für die Nutzung von Wissen aus der Forschung sich in
  besseren, andere Ressourcen einsparenden oder effektiver einsetzenden
  Entscheidungen niederschlagen würde.)
\item
  Bibliotheken müssten wohl ein realistischeres Verständnis davon
  entwickeln, was Forschung eigentlich macht und was für Wissen in ihr
  erarbeitet wird. (Am einfachsten geht das, indem eigene Forschung
  durchgeführt wird, wozu bibliothekarisches Personal mit
  Hochschulausbildung sehr wohl in der Lage ist.) Das würde auch
  ermöglichen, dass die unterschiedlichen Wissensformen angemessen
  bewertet würden: Forschung als systematisches Wissen. Anregungen
  meinungsstarker Einzelpersonen als informierte Einzelpositionen.
  Eigene Beobachtungen und Hoffnung als lokales Wissen. Methoden als
  Methoden, nicht als Lösungen. Positionen, die hinter auffällig vielen
  rhetorischen Mitteln versteckt werden, als argumentative schwache
  Positionen. Und so weiter.
\item
  Dann liesse sich auch ein neues Verständnis von Forschung und
  Bibliotheken entwickeln. Ansonsten bleibt es wohl bei der
  beschriebenen Situation, solange sich nicht anderes massiv ändert.
  (Beispielsweise die Struktur der Gesellschaft, aber wie gesagt scheint
  die Struktur der zu heute sehr anderen Gesellschaft in der DDR nicht
  zu einer anderen Beziehung von Forschung und Bibliothekspraxis geführt
  zu haben.)
\end{enumerate}

\hypertarget{vignette-fuxfcnf}{%
\subsection{Vignette fünf}\label{vignette-fuxfcnf}}

\begin{quote}
\emph{Ich wurde \enquote{Experte für Makerspaces}, weil ein Kollege es
vor Jahren einfach behauptete. (Vielen Dank an den Kollegen nochmal. Er
weiss, was er getan hat.) Seitdem habe ich einige Jahre über Makerspaces
geforscht. Habe alles, was mir in Deutsch, Englisch und Französisch zum
Thema unterkam, gelesen (und das zum Beispiel in zwei Sammelrezensionen
dargestellt). Ich habe in einem Projekt geholfen, dass die Stiftung
Bibliomedia Schweiz jetzt in der Deutschschweiz für Gemeinde- und
Schulbibliotheken mobile Makerspace-Toolkits anbietet. Ein Vorprojekt
dazu, das ich vor einigen Jahren leitete, wurde von anderen Bibliotheken
als Vorbild für ihre Makerspace-Projekte genutzt. Ich habe
Bachelorarbeiten zum Thema betreut und habe Anfragen abgelehnt, in Jurys
über Makerspace-Projekte zu sitzen. Ich würde also sagen, dass ich, auch
im Vergleich zu anderen, die über das Thema reden, recht viel zu
Makerspaces in Bibliotheken weiss.}

\emph{Eine Sache, auf die ich auf dieser Basis (und der wenigen Daten,
die wir über die Nutzung von Makerspaces in Bibliotheken haben), immer
wieder hinweise, wenn man mich fragt, ist die, dass sich Jugendliche für
diese Makerspaces nicht interessieren. Das bestätigt sich immer wieder.
Von Ausnahmen abgesehen, bricht das Interesse an all den Angeboten, die
Bibliotheken in diesem Bereich machen, zwischen 12 und 15 Jahren -- also
der Zeit, wenn Kinder zu Jugendlichen werden -- ab. Massiv. In
Bibliotheken finden sich wohl immer wieder Jugendliche, die gerne
helfen, Makerspaces mit zu betreuen. Aber als Angebot, um Jugendliche
anzusprechen, eignen sie sich nicht. (Das ist auch nicht erstaunlich,
sondern hat sich bei anderen Angeboten, die in den letzten Jahrzehnten
für Jugendliche gemacht wurden, auch immer wieder gezeigt.) Ich weiss
nicht, wie oft ich das schon dargestellt habe. Ich weiss auch nicht, wie
man das bei den Makerspaces und ähnlichen Angeboten, die es ja in
ausreichender Zahl gibt und die man besuchen kann, um selber
nachzuschauen, übersehen kann.}

\emph{Ich erhalte weiterhin -- wenn auch abnehmend -- Anfragen zu
Makerspaces. Soll Hinweise dazu geben, wie man sie aufbauen kann. Soll
Konzepte bewerten. Und so weiter. Praktisch jedesmal ist die Idee der
fragenden Bibliotheken, dass man mit den jeweiligen Makerspaces
Jugendliche ansprechen würde. Es werden dazu teilweise elaborierte
Vorstellungen entwickelt, die sich alle nicht bewahrheiten werden. Die
interessante Frage ist: Warum fragt man mich das überhaupt? Auf die
Idee, dass ich etwas zum Thema wüsste, kommt man doch nur, weil man
Beiträge von mir zum Thema wahrgenommen hat. Aber nicht den Inhalt? Das
passiert auch bei anderen Themen immer wieder, Makerspaces sind nur ein
sehr eindeutiges Beispiel. Ich, als Forschender, käme nicht auf die
Idee, jemand etwas zu fragen und nicht vorher deren / dessen Beiträge zu
lesen. Bibliotheken schon? Wenn ja, dann ist das strukturell.}
\end{quote}

%autor
\begin{center}\rule{0.5\linewidth}{0.5pt}\end{center}

\textbf{Karsten Schuldt}, Wissenschaftlicher Mitarbeiter am
Schweizerisches Institut für Informationswissenschaft, FH Graubünden.
Lebt in Berlin, Chur und Lausanne. Hobbys unter anderem Tee trinken und
zeitgenössische Lyrik lesen, Streetart finden und Musemsbesuche,
Rätoromanisch lernen und eine unvernünftig intensive Begeisterungen für
Keramik und das Fürstentum Liechtenstein.

\end{document}
