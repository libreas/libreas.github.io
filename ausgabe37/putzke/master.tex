\documentclass[a4paper,
fontsize=11pt,
%headings=small,
oneside,
numbers=noperiodatend,
parskip=half-,
bibliography=totoc,
final
]{scrartcl}

\usepackage[babel]{csquotes}
\usepackage{synttree}
\usepackage{graphicx}
\setkeys{Gin}{width=.4\textwidth} %default pics size

\graphicspath{{./plots/}}
\usepackage[ngerman]{babel}
\usepackage[T1]{fontenc}
%\usepackage{amsmath}
\usepackage[utf8x]{inputenc}
\usepackage [hyphens]{url}
\usepackage{booktabs} 
\usepackage[left=2.4cm,right=2.4cm,top=2.3cm,bottom=2cm,includeheadfoot]{geometry}
\usepackage{eurosym}
\usepackage{multirow}
\usepackage[ngerman]{varioref}
\setcapindent{1em}
\renewcommand{\labelitemi}{--}
\usepackage{paralist}
\usepackage{pdfpages}
\usepackage{lscape}
\usepackage{float}
\usepackage{acronym}
\usepackage{eurosym}
\usepackage{longtable,lscape}
\usepackage{mathpazo}
\usepackage[normalem]{ulem} %emphasize weiterhin kursiv
\usepackage[flushmargin,ragged]{footmisc} % left align footnote
\usepackage{ccicons} 
\setcapindent{0pt} % no indentation in captions

%%%% fancy LIBREAS URL color 
\usepackage{xcolor}
\definecolor{libreas}{RGB}{112,0,0}

\usepackage{listings}

\urlstyle{same}  % don't use monospace font for urls

\usepackage[fleqn]{amsmath}

%adjust fontsize for part

\usepackage{sectsty}
\partfont{\large}

%Das BibTeX-Zeichen mit \BibTeX setzen:
\def\symbol#1{\char #1\relax}
\def\bsl{{\tt\symbol{'134}}}
\def\BibTeX{{\rm B\kern-.05em{\sc i\kern-.025em b}\kern-.08em
    T\kern-.1667em\lower.7ex\hbox{E}\kern-.125emX}}

\usepackage{fancyhdr}
\fancyhf{}
\pagestyle{fancyplain}
\fancyhead[R]{\thepage}

% make sure bookmarks are created eventough sections are not numbered!
% uncommend if sections are numbered (bookmarks created by default)
\makeatletter
\renewcommand\@seccntformat[1]{}
\makeatother

% typo setup
\clubpenalty = 10000
\widowpenalty = 10000
\displaywidowpenalty = 10000

\renewcommand{\linethickness}{0.05em}

\usepackage{hyperxmp}
\usepackage[colorlinks, linkcolor=black,citecolor=black, urlcolor=libreas,
breaklinks= true,bookmarks=true,bookmarksopen=true]{hyperref}
\usepackage{breakurl}

%meta

%meta

\fancyhead[L]{R. Putzke \\ %author
LIBREAS. Library Ideas, 37 (2020). % journal, issue, volume.
\href{https://doi.org/10.18452/21546.2}{\color{black}https://doi.org/10.18452/21546.2}
{}} 
\fancyhead[R]{\thepage} %page number
\fancyfoot[L] {\ccLogo \ccAttribution\ \href{https://creativecommons.org/licenses/by/4.0/}{\color{black}Creative Commons BY 4.0}}  %licence
\fancyfoot[R] {ISSN: 1860-7950}

\title{\LARGE{Wünsche an die Forschung -- Gedanken aus der Praxis einer Öffentlichen Bibliothek}} % title
\author{Rebekka Putzke} % author

\setcounter{page}{1}

\hypersetup{%
      pdftitle={Wünsche an die Forschung -- Gedanken aus der Praxis einer Öffentlichen Bibliothek},
      pdfauthor={Rebekka Putzke},
      pdfcopyright={CC BY 4.0 International},
      pdfsubject={LIBREAS. Library Ideas, 37 (2020).},
      pdfkeywords={Öffentliche Bibliothek, Forschung, Praxisbezug, Forschungsrelevanz, Fachdiskurs},
      pdflicenseurl={https://creativecommons.org/licenses/by/4.0/},
      pdfcontacturl={http://libreas.eu},
      baseurl={https://doi.org/10.18452/21546.2},
      pdflang={de},
      pdfmetalang={de}
     }



\date{}
\begin{document}

\maketitle
\thispagestyle{fancyplain} 

%abstracts
\begin{abstract}
\noindent
\textbf{Kurzfassung}: Der Beitrag formuliert Anregungen für die
Forschung aus der Perspektive der bibliothekarischen Praxis in
Öffentlichen Bibliotheken. Er geht auf Möglichkeiten von Forschung aus
der Praxis heraus sowie die grundsätzliche Bedeutung wissenschaftlicher
Forschung für Öffentliche Bibliotheken und den dafür notwendigen
Fachdiskurs ein.
\end{abstract}

%body
Meiner Erfahrung nach lässt die praktische Arbeit in Öffentlichen
Bibliotheken eigene Forschung selten zu. Der Arbeitsalltag in Verbindung
mit gefühlt zu vielen Aufgaben und Baustellen für zu wenig Personal
verhindert dies und je kleiner die Bibliothek, desto eklatanter
wahrscheinlich die Problematik.\footnote{Die Situation an Hochschulen
  ist letztlich nicht anders, im Gegensatz zur Bibliothekspraxis gehört
  die Forschung dort jedoch zu den originären Aufgaben.} Dass
Nicht-Hochschulangehörige häufig nur begrenzt Zugang zu Fachliteratur
und Rechercheinstrumenten haben, kommt erschwerend hinzu. Es wäre
allerdings falsch, daraus zu schließen, es gäbe keinen Forschungsbedarf.

Aus diesem Grund ist die Praxis gerade auf die wissenschaftliche
Forschung angewiesen. Und obwohl gute und relevante Forschung
(natürlich!) nicht generell aus der Praxis heraus angestoßen werden
muss, kann es doch für beide Seiten gleichermaßen hilfreich sein, wenn
sie sich konkret darauf bezieht, Problemstellungen daraus ableitet oder
Etabliertes hinterfragt. Es gibt viele Fragestellungen, auf die die
Praxis keine befriedigenden Antworten liefern kann, obwohl sie diese für
fundierte strategische Entscheidungen und eine tragfähige
Zukunftsausrichtung benötigte.

Im Praxisbetrieb von Bibliotheken sind es häufig Projektbeschreibungen
und Förderanträge, die eine mangelnde Selbstreflexion im Arbeitsalltag
durchbrechen, weil die Antragsstellung ein Zurücktreten auf die
Metaebene und außerdem ein immer neues Nachdenken über konkrete
Evaluationsmöglichkeiten erzwingt. Generell gehen Forschung und
Theoriebildung -- mindestens in Ansätzen -- dann direkt von der Praxis
aus, wenn die wissenschaftliche Forschung die erforderlichen Instrumente
nicht bereitstellen kann. Sie erfolgt also eher aus der Not heraus, weil
die Wissenschaft selbst die im konkreten Fall benötigte Forschungsarbeit
nicht geleistet hat.\footnote{Z. B. Entwicklung der Leistungsindikatoren
  \enquote{Schülererfassungsgrad} und \enquote{Klassenerfassungsgrad}
  zur systematischen Analyse der Zusammenarbeit mit Schulen und
  Kindertagesstätten in den Städtischen Bibliotheken Dresden (vgl.
  Menzel \& Rabe, 2005)}

Und tatsächlich sehe ich, ohne jeden repräsentativen Anspruch, rein aus
den Fragestellungen, die mein Berufsalltag mir vor die Füße spült,
Forschungsbedarf zu vielen Themen.

Ein wesentliches Arbeitsfeld für viele Öffentliche Bibliotheken in
Deutschland ist nach wie vor die \textbf{\emph{Zusammenarbeit mit
Schulen}}, zu der die Recherche für meine Masterarbeit 2018 recht wenig
wissenschaftliche Literatur ergab. Hierzu wären aufwendige empirische
Studien und Befragungen an Schulen, Referendariats-Seminaren,
Hochschulen, beteiligten Institutionen (Schulämtern, Ministerien,
\ldots) sowie Bibliotheken sehr wünschenswert als Basis neuer
theoretischer Erkenntnisse. Obgleich das Thema der systematischen
Kooperation von Schulen und Bibliotheken in der Praxis von hoher
Relevanz und auch starker Präsenz im Fachaustausch ist (viele
Praxisberichte, Fortbildungen et cetera zu diesem Themenfeld), ist mein
Eindruck dazu eine Unterrepräsentiertheit in der Forschung.

Auch die konkrete \textbf{\emph{Bedeutung von Öffentlichen Bibliotheken
für ihre Kommune}} oder ihren Stadtteil beziehungsweise ihr direktes
Umfeld könnte verstärkt in den Blick genommen werden -- unter Aspekten
wie Dezentralisierung, Stärkung von Innenstädten/Ortskernen,
Bereitstellung von Services wie Informationszugang und Kulturprogrammen
vor Ort, Teil der Nahversorgung, Stärkung der Verbundenheit
Bürger*innen-Kommune.

Ein weiteres Forschungsfeld könnte die tatsächliche
\textbf{\emph{wirtschaftliche Bedeutung von Öffentlichen Bibliotheken
für Verlage}} sein. Diese ist den Akteuren vermutlich häufig nicht
bewusst. Kein Wunder, Öffentliche Bibliotheken kaufen in der Regel nicht
bei den Verlagen direkt, sondern über den (oft lokalen) Buchhandel, so
dass ihre Kaufkraft für Verlage kaum sichtbar wird. Mit einem gut
durchdachten Forschungsdesign könnte hier zum Nutzen von Öffentlichen
Bibliotheken wie Buchhandel und Verlagswirtschaft gleichermaßen Licht
ins Dunkel gebracht werden.

Auch einen differenzierten Blick auf den \textbf{\emph{Wandel von
Bibliotheken}} hielte ich für lohnenswert. Es steht außer Frage, dass es
in Öffentlichen Bibliotheken eine große Nachfrage nach elektronischen
Beständen sowie Arbeitsplätzen und Aufenthaltsqualität gibt. Doch muss
dies nicht zwingend zulasten von physischen Beständen gehen. Warum
konkret geht die Ausleihe von physischen Medien zurück? Welche Faktoren
spielen dabei eine Rolle? Welche Bestandsgruppen genau betrifft dies
(also deutlich feingranularere Feststellungen als ein allgemeiner
Rückgang von Buchausleihen oder ein Einbruch der
DVD-Ausleihen in Zeiten von Streaming-Angeboten)? Gibt es Unterschiede
zwischen kleinen und großen Bibliotheken oder vielleicht zwischen
Regionen mit unterschiedlich finanzkräftiger Bevölkerung? Welche Rolle
spielt dabei der Bestandsaufbau?

Überhaupt wären die klassischen Themen \textbf{\emph{Bestandsaufbau und
Bestandsvermittlung}} ein sehr ergiebiges, zugleich für die Praxis
äußerst nützliches Forschungsfeld. Welche Rolle spielen Standing Order
und Approval Plan genau? Welche unterschiedlichen Modelle gibt es? Wie
zuverlässig/qualitativ überzeugend sind diese in der Umsetzung? Und hat
ein Umstieg dahin von einem individuellen Bestandsaufbau irgendwelche
messbare Konsequenzen für Angebote und Akzeptanz der betreffenden
Bibliotheken (kommt es zur Stärkung anderer qualitativer Angebote durch
frei gewordene Personalkapazitäten? zum Rückgang von Ausleihen? zur
Vereinheitlichung von Beständen?)? Inwieweit unterscheiden sich Bestände
vergleichbarer Bibliotheken eigentlich generell (beispielsweise bei
Zweigstellen eines Bibliotheksnetzes, Bibliotheken vergleichbarer
Größe/vergleichbarer Etats oder einer Region)? Welche Bedeutung kommt
dem Lektorat/Fachreferendariat zu und wie wandelt sich dessen Rolle im
Detail? Öffentliche Bibliotheken werben gerne mit ihrer
Beratungsqualität und einem nicht von kommerziellen Interessen
geleiteten, sachlich orientierten, niederschwelligen Zugang zu
Informationen (im Sinne einfacher Zugänglichkeit, auch guter
Auffindbarkeit durch entsprechende Erschließung). Zugleich hat die
Deutsche Nationalbibliothek die relevanten Erschließungsleistungen für
Öffentliche Bibliotheken vorerst eingestellt\footnote{Die \emph{AG
  Erschliessung OEB-DNB} sucht inzwischen nach einer vertretbaren
  automatisierten Lösung.} und in Fachgesprächen erscheint es mir
manchmal, als sei meine Bibliothek eine der letzten kommunalen
Bibliotheken mit einem klassisch arbeitenden Lektorat. Trügt dieser
Eindruck und inwieweit hat das Einfluss auf den Bestandsaufbau, die
Nutzung und Akzeptanz der Bestände? Welche Faktoren, die wir in der
Praxis vielleicht gar nicht sehen, spielen hier eine Rolle?

Ich wünsche mir, dass diese und weitere aktuelle Themen aus der
Berufspraxis systematisch, transparent und möglichst jenseits von
ideologischen Grabenkämpfen umfassend betrachtet werden -- eine Aufgabe,
die die Praxis so nicht leisten kann. Das hat nicht nur mit mangelnden
Kapazitäten zu tun, sondern damit, dass die einzelnen Öffentlichen
Bibliotheken selbst natürlich immer Teil des Systems, mit individuellen
Problemstellungen und Richtungsentscheidungen, und damit auch nicht
neutral sind.\footnote{Womöglich hätte eine zentrale, übergeordnete
  Einrichtung wie das DBI (Deutsches Bibliotheksinstitut, 1978-1999)
  sich mancher dieser Fragestellungen annehmen können.}

Für die Praxis sehe ich generell einen klaren Gewinn aus relevanter,
qualitativ überzeugender Forschung: Sie fördert die Hinterfragung von
\enquote{Glaubenssätzen} und einer \enquote{das haben wir immer so
gemacht}-Haltung, wo es angebracht ist und es gute Gründe dafür gibt.
Sie unterstützt die Reflexion der eigenen Arbeitsweise und kann damit
gegebenenfalls auch die Position von Öffentlichen Bibliotheken, zum
Beispiel gegenüber Mittelgebern, untermauern. Darüber hinaus kann sie
geeignete Evaluationsinstrumente für verschiedenste Maßnahmen zur
Verfügung stellen und die Rechtfertigung oder Abschaffung bestimmter
Services, Vorgehensweisen oder Strukturen stützen, Grundlagen liefern
zur Verbesserung von Strategien, im Einzelfall sogar für weitergehende
Neuorientierungen. Wenn es der Forschung gelingt, hierfür Argumente
durch wissenschaftlich fundierte Ergebnisse zu liefern, so leistet sie
einen wertvollen Beitrag zur Objektivierung, erleichtert damit die
Überzeugungsarbeit nach innen und außen und ermöglicht es, solche teils
tiefgreifende Entscheidungen auf Basis valider Erkenntnisse zu treffen.
Damit kann sie ein Gegengewicht zur Verlockung schnelllebiger
Trends\footnote{Siehe dazu aktuell:
  \url{https://bildungundgutesleben.wordpress.com/2020/03/09/der-3-ort-ist-tot-was-lernen-wir-daraus/}
  (zuletzt abgerufen am 30.03.2020)} ohne ausreichende empirische
Grundlage darstellen und evidenzbasiertes Handeln maßgeblich
begünstigen. Forschung für Öffentliche Bibliotheken ist unverzichtbar!
Der von praktischen Erwägungen (auch strukturellen, finanziellen oder
ähnlichen Zwängen) freie Blick wissenschaftlicher Forschung, der dem
Innovationsdruck weniger oder zumindest in anderer Weise unterworfen
ist, kann eine unverstelltere Perspektive auf die Arbeit der Praxis
ermöglichen.\footnote{vgl. Wimmer, 2019, S. 307 f.}

Natürlich können auch andere Forschungsbereiche und Disziplinen als die
Bibliotheks- und Informationswissenschaft solche für die Praxis
wichtigen und wertvollen Forschungsbeiträge liefern (und haben dies
bereits getan) -- wie die Lesesozialisationsforschung, um nur ein
Beispiel aus meinem eigenen Arbeitsbereich zu nennen.

Wissenschaftliche Forschung egal welcher Disziplin muss frei sein und
darf ohne jede (zumindest sofort erkennbare) praktische Relevanz sein.
Sie kann sich keinesfalls nur auf aus der Praxis generierte Themen und
Fragestellungen beziehen. Vielmehr sollen von ihr eigene Impulse
ausgehen, die auf der Ebene der praktischen Arbeit vielleicht nie zum
Tragen gekommen wären. Wissenschaftliche Forschung und Berufspraxis
dürfen und müssen sich aber dennoch aufeinander beziehen und das kann
interessant und ergiebig für beide Seiten sein.\footnote{Vgl. Wimmer,
  2015} Wichtig für die Wahrnehmung wissenschaftlicher
Forschungsergebnisse in der Praxis ist aus meiner Sicht neben einer
guten Zugänglichkeit und Recherchierbarkeit auch eine Klarheit der
Argumente und der Sprache. Wesentliche Grundlage für einen gelingenden
Wissenstransfer in beide Richtungen ist ein enger und lebendiger
Austausch zwischen Theorie und Praxis, einschließlich eines
Gehörtwerdens von Forschungsbedarfen aus der Praxis, Nutzung von
Möglichkeiten der Verifizierung/Falsifizierung von Forschungsergebnissen
durch praktische Anwendung von Theorien, Umsetzung überzeugender
Vorschläge aus der Forschung in der praktischen Arbeit. Denn viele
Ansätze aus der Wissenschaft liefern einerseits innovative Ansätze für
die Praxis, benötigen diese jedoch zugleich als Korrektiv. Diese
Rückkopplung erhöht die Akzeptanz von Forschungsergebnissen in der
Praxis und kann fruchtbar für beide Felder sein.

\hypertarget{literatur}{%
\section{Literatur}\label{literatur}}

Menzel, S. \& Rabe, R. (2005). Das Projekt \enquote{Bibliothek und
Schule} in den Städtischen Bibliotheken Dresden. \emph{BIBLIOTHEK
Forschung Und Praxis}, 29(1). \url{https://doi.org/10.1515/BFUP.2005.74}

Wimmer, U. (2015). Es gibt nichts Praktischeres als eine gute Theorie!
Ein Plädoyer für mehr Theorie in der Bibliotheksarbeit. \emph{O-Bib. Das
Offene Bibliotheksjournal / Herausgeber VDB}, 2(3), \hbox{S. 81--88}.
\url{https://doi.org/10.5282/o-bib/2015H3S81-88}

Wimmer, U. (2019). Wo sind die Öffentlichen Bibliotheken in Forschung
und Lehre? Eine unbequeme Antwort. In Hauke, P. (Hg.), \emph{Öffentliche
Bibliothek 2030. Herausforderungen -- Konzepte -- Visionen}, Bad Honnef:
Bock+Herchen, 2019, S. 303--310. \url{http://dx.doi.org/10.18452/20169}

%autor
\begin{center}\rule{0.5\linewidth}{\linethickness}\end{center}

\textbf{Rebekka Putzke} hat einen Magisterabschluss in
Germanistik/Literaturwissenschaft, Philosophie und Physik und einen
Masterabschluss in Bibliotheks- und Informationswissenschaft. Sie
arbeitet nach Jahren als Bibliothekarin mit Schwerpunkt
Kinderbibliotheksarbeit sowie Leiterin einer Zweigbibliothek als
Lektorin für Kindermedien bei den Städtischen Bibliotheken Dresden.

\end{document}
