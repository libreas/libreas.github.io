\textbf{Kurzfassung}: Eine Teaching Library fördert durch die
Bereitstellung von Lernraum und die aktive Lehre die
Informationskompetenz ihrer Nutzenden. Der Begriff kommt ursprünglich
aus dem Bereich der Hochschulbibliotheken und ist im Zusammenhang mit
Öffentlichen Bibliotheken scheinbar selten gebräuchlich. Der folgende
Artikel geht den Fragen nach, ob dennoch Teaching Librarians in
Öffentlichen Bibliotheken in Deutschland gesucht und beschäftigt werden
und welche Berufskompetenzen gegebenenfalls dabei gefordert werden. Die
präsentierte Erhebung und Auswertung ist im Rahmen einer Vorstudie für
die Masterarbeit der Autorin entstanden (Menzel 2019).

\begin{center}\rule{0.5\linewidth}{0.5pt}\end{center}

\textbf{Abstract}: A Teaching Library supports users in improving their
information literacy. This includes the provision of working space and
access to information as well as teaching activities provided by so
called Teaching Librarians. The term was established by college and
research libraries. In the context of public libraries, it is rarely
used. This article examines whether Teaching Librarians are being hired
by German public libraries nonetheless, and if so, which competencies
are required for their work. The presented study was conducted as a
preliminary study to the author's master's thesis (Menzel 2019).
