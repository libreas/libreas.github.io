\documentclass[a4paper,
fontsize=11pt,
%headings=small,
oneside,
numbers=noperiodatend,
parskip=half-,
bibliography=totoc,
final
]{scrartcl}

\usepackage[babel]{csquotes}
\usepackage{synttree}
\usepackage{graphicx}
\setkeys{Gin}{width=.4\textwidth} %default pics size

\graphicspath{{./plots/}}
\usepackage[ngerman]{babel}
\usepackage[T1]{fontenc}
%\usepackage{amsmath}
\usepackage[utf8x]{inputenc}
\usepackage [hyphens]{url}
\usepackage{booktabs} 
\usepackage[left=2.4cm,right=2.4cm,top=2.3cm,bottom=2cm,includeheadfoot]{geometry}
\usepackage{eurosym}
\usepackage{multirow}
\usepackage[ngerman]{varioref}
\setcapindent{1em}
\renewcommand{\labelitemi}{--}
\usepackage{paralist}
\usepackage{pdfpages}
\usepackage{lscape}
\usepackage{float}
\usepackage{acronym}
\usepackage{eurosym}
\usepackage{longtable,lscape}
\usepackage{mathpazo}
\usepackage[normalem]{ulem} %emphasize weiterhin kursiv
\usepackage[flushmargin,ragged]{footmisc} % left align footnote
\usepackage{ccicons} 
\setcapindent{0pt} % no indentation in captions

%%%% fancy LIBREAS URL color 
\usepackage{xcolor}
\definecolor{libreas}{RGB}{112,0,0}

\usepackage{listings}

\urlstyle{same}  % don't use monospace font for urls

\usepackage[fleqn]{amsmath}

%adjust fontsize for part

\usepackage{sectsty}
\partfont{\large}

%Das BibTeX-Zeichen mit \BibTeX setzen:
\def\symbol#1{\char #1\relax}
\def\bsl{{\tt\symbol{'134}}}
\def\BibTeX{{\rm B\kern-.05em{\sc i\kern-.025em b}\kern-.08em
    T\kern-.1667em\lower.7ex\hbox{E}\kern-.125emX}}

\usepackage{fancyhdr}
\fancyhf{}
\pagestyle{fancyplain}
\fancyhead[R]{\thepage}

% make sure bookmarks are created eventough sections are not numbered!
% uncommend if sections are numbered (bookmarks created by default)
\makeatletter
\renewcommand\@seccntformat[1]{}
\makeatother

% typo setup
\clubpenalty = 10000
\widowpenalty = 10000
\displaywidowpenalty = 10000

\usepackage{hyperxmp}
\usepackage[colorlinks, linkcolor=black,citecolor=black, urlcolor=libreas,
breaklinks= true,bookmarks=true,bookmarksopen=true]{hyperref}
\usepackage{breakurl}

%meta
%meta

\fancyhead[L]{N. Bezerra Cardoso, A. Calil Elias Junior, E. Campos Machado\\ %author
LIBREAS. Library Ideas, 37 (2020). % journal, issue, volume.
\href{http://nbn-resolving.de/}
{}} % urn 
% recommended use
%\href{http://nbn-resolving.de/}{\color{black}{urn:nbn:de...}}
\fancyhead[R]{\thepage} %page number
\fancyfoot[L] {\ccLogo \ccAttribution\ \href{https://creativecommons.org/licenses/by/4.0/}{\color{black}Creative Commons BY 4.0}}  %licence
\fancyfoot[R] {ISSN: 1860-7950}

\title{\LARGE{Der Beitrag der „Forschungsgruppe öffentliche Bibliotheken in Brasilien“ zu Lehre, Forschung und bibliothekswissenschaftlicher Praxis}}% title
\author{Nathalice Bezerra Cardoso, Alberto Calil Elias Junior, Elisa Campos Machado} % author

\setcounter{page}{1}

\hypersetup{%
      pdftitle={Der Beitrag der „Forschungsgruppe öffentliche Bibliotheken in Brasilien“ zu Lehre, Forschung und bibliothekswissenschaftlicher Praxis},
      pdfauthor={Nathalice Bezerra Cardoso, Alberto Calil Elias Junior, Elisa Campos Machado},
      pdfcopyright={CC BY 4.0 International},
      pdfsubject={LIBREAS. Library Ideas, 37 (2020).},
      pdfkeywords={Brasilien, Öffentliche Bibliotheken, Bibliothekswissenschaft, Lehre, Forschung},
      pdflicenseurl={https://creativecommons.org/licenses/by/4.0/},
      pdfcontacturl={http://libreas.eu},
      baseurl={http://libreas.eu},
      pdflang={de},
      pdfmetalang={de}
     }



\date{}
\begin{document}

\maketitle
\thispagestyle{fancyplain} 

%abstracts
\begin{abstract}
\noindent
\textbf{Zusammenfassung:} Dieser Artikel stellt vor, wie die
Forschungsgruppe ``Öffentliche Bibliotheken in Brasilien: Reflexion und
Praxis'' (Grupo de Pesquisa ``Bibliotecas Públicas no Brasil: reflexão e
prática'', GPBP), die an der Bundesuniversität des Staates Rio de
Janeiro (Universidade Federal do Estado do Rio de Janeiro, UNIRIO)
angesiedelt ist, daran arbeitet, den Dialog zwischen der Universität und
den öffentlichen Bibliotheken Brasiliens durch Lehre, Forschung und
Öffentlichkeitsarbeit zu gewährleisten. Sieben Jahre Forschung haben
gezeigt, dass sowohl die Universität als auch die öffentlichen
Bibliotheken von einem Wissens- und Erfahrungsaustausch profitieren. Die
Universität profitiert, indem sie den Studierenden ermöglicht, die
Bibliothekspraxis zu erleben. Die öffentlichen Bibliotheken nutzen die
Ergebnisse der Forschung und der Projekte der Gruppe, um konkrete,
alltägliche Probleme zu lösen. Es ist zu hoffen, dass dieser Artikel
mehr Universitäten dazu motivieren wird, in die Forschung zu und mit
öffentlichen Bibliotheken zu investieren.

\textbf{Schlüsselwörter:} Bibliothekswissenschaft -- Brasilien.
Bibliothekswissenschaft -- Lehre und Forschung. Öffentliche Bibliotheken
-- Brasilien. Öffentliche Bibliotheken -- Lehre und Forschung
\end{abstract}

%body
\hypertarget{einleitung}{%
\section{1 Einleitung}\label{einleitung}}

Ausgehend von den Annahmen, dass die Welt in verschiedene Bereiche
segmentiert werden kann, dass Wissen in verschiedene Kategorien wie
wissenschaftlich, empirisch, theologisch et cetera eingeordnet werden
kann und dass Individuen, Gruppen, Institutionen, Prozesse und Praktiken
unterschiedliches Wissen produzieren, wurde im Jahr 2013 die
Forschungsgruppe Öffentliche Bibliotheken in Brasilien {[}Grupo de
Pesquisa Bibliotecas Públicas no Brasil (GPBP)\footnote{Online verfügbar
  unter: \url{http://culturadigital.br/gpbp} (07.01.2020).}, Übersetzung
der Verfasser{]} an der Bundesuniversität des Bundesstaates Rio de
Janeiro (UNIRIO)\footnote{Die Bundesuniversität des Bundesstaates Rio de
  Janeiro (UNIRIO) ist eine der 24 öffentlichen Hochschuleinrichtungen,
  die ein Studium für Bibliothekswissenschaft in Brasilien anbieten. Es
  gibt \enquote{57 Bachelorstudiengänge in Bibliothekswissenschaft. An
  24 öffentlichen Bundesuniversitäten werden 27 Studiengänge angeboten.
  Die Anzahl der Studiengänge und die Anzahl der Universitäten stimmen
  nicht überein, da einige Universitäten mehr als ein Studium anbieten.
  Zum Beispiel zählen Studiengänge, die morgens angeboten werden und
  solche, die abends angeboten werden, als zwei verschiedene
  Studiengänge}. (CALIL JUNIOR; SILVEIRA; SILVA; ROSA, 2015, S. 5,
  Übersetzung der Verfasser).} gegründet. Eines der Ziele der GPBP ist
es, Räume für die kollektive Kommunikation zwischen den verschiedenen
Individuen, Gruppen, Institutionen, Prozessen und Praktiken der
Bibliothekswissenschaft zu schaffen.

Es sei daran erinnert, dass die Vorstellung von unterschiedlichen
Sphären, die Wissen und seine Individuen trennen, auf die Moderne
zurückgeht, in der die Grenzen zwischen diesen Wissensbereichen
abgesteckt wurden. Religiöses Wissen war fast ausschließlich in Tempeln,
volkstümliches Wissen in Herkunftsgemeinschaften und wissenschaftliche
Erkenntnisse in Universitäten und Forschungseinrichtungen zu finden.

Dieser Entwurf der Moderne einer segmentierten Welt korrespondiert
allerdings nicht (mehr) mit der Realität und die Grenzen zwischen den
Wissenssphären, die einst strikt getrennt wurden, sind durchlässig
geworden.\footnote{Die in den vergangenen Jahrhunderten erlebten
  Transformationen haben die Relativierung dieser Grenzen ermöglicht.}
Wissenschaftliche Erkenntnisse, etwa, überschreiten die Mauern der
Universitäten und dringen in den Alltag verschiedener sozialer Gruppen
vor. Betrachten wir beispielsweise die öffentlichen Universitäten in
Brasilien heute, fällt auf, dass -- obwohl sie immer noch der Hauptort
für die Produktion und Zirkulation von wissenschaftlichem Wissen sind --
sich hier Wissensarten bemerkbar machen, die diese Definition
transzendieren.

Angesichts dieser Umstände offenbart sich die dringende Notwendigkeit
der Öffnung von Kanälen des Dialogs zwischen den Wissensbereichen.
Darüber hinaus zeigt sich, wie wichtig es ist, auf die Stimmen zu
achten, die in wissenschaftlichen Räumen widerhallen (und auf die, die
noch nicht widerhallen).

Auf der Suche nach dieser Übung des Zuhörens wird die Idee verteidigt,
dass sich Forschungsgruppen potenziell als gut geeignete Räume für die
Annäherung der Wissensbereiche präsentieren. In diesem Kontext ist die
Tätigkeit der Forschungsgruppe Öffentliche Bibliotheken in Brasilien:
Theorie und Praxis zu sehen, welche Forscher und Fachleute aus dem
ganzen Land zusammenbringt, die sich mit Studien und Forschungen zur
öffentlichen Bibliothekswissenschaft beschäftigen.

Zudem ist es wichtig, das aktuelle Informations-Ökosystem zu erfassen,
in dem \emph{Information Disorder} unterstützt von Postfaktizität
alltäglich ist. Dieses Ökosystem bietet die idealen Rahmenbedingungen
für die Entstehung und Ausbreitung von zum Beispiel
anti-wissenschaftlichen Bewegungen. Wir leben in einer Zeit, in der
wissenschaftliche Erkenntnisse nicht nur in Frage gestellt werden -- was
wünschenswert und von Nutzen ist, denn der Zweifel bewegt die
Wissenschaft --, sondern fast täglich mit Kampagnen, die sich über die
großen Medien und die digitalen sozialen Netzwerke verbreiten, heftig
angegriffen werden. Ein Beispiel dafür ist die aktuelle
Anti-Impf-Bewegung, die täglich Anhänger gewinnt oder die Situation im
Jahr 2019, in der es notwendig war zu erklären, dass die Erde nicht
flach ist. Dies ist die beängstigende Realität in Brasilien.

Bevor wir uns dem zentralen Gegenstand dieses Artikels zuwenden, soll
die aktuelle Situation in Brasilien kurz vorgestellt werden.

\hypertarget{brasilien-heute}{%
\section{2 Brasilien heute}\label{brasilien-heute}}

Gemessen an seiner Fläche (8.511.000 km²) und seiner Bevölkerung ist
Brasilien das fünftgrößte Land der Welt. Im Januar 2020 wurde die
brasilianische Bevölkerung nach Angaben des brasilianischen Instituts
für Geographie und Statistik (Instituto Brasileiro de Geografia e
Estatística -- IBGE) auf 211.330.900 Millionen Einwohner geschätzt.
Brasilien besteht aus 5.570 Gemeinden, die sich auf 5 Regionen
verteilen: Norden, Süden, Südosten, Mitte-Westen und Nordosten. Das Land
ist in einem föderativen politischen System organisiert, das 26
Bundesstaaten sowie den Bundesdistrikt, in dem sich die Hauptstadt
Brasilia befindet, umfasst.

Die meisten Gemeinden liegen in den Regionen Südosten, Süden und
Nordosten, in denen sich auch der größte Teil der brasilianischen
Bevölkerung und somit die wirtschaftlichen, sozialen und kulturellen
Ressourcen konzentrieren. In den Regionen Norden und Mittelwesten
befinden sich große Ausdehnungen von Wald und eine Vielzahl komplexer
Biome und hydrographischer Becken, das Amazonasgebiet und das Pantanal,
was zu einer im Vergleich zum Rest des Territoriums geringeren
Bevölkerungsdichte führt.

Nach Angaben des Instituts für angewandte Wirtschaftsforschung
(Instituto de Pesquisa Econômica Aplicada -- IPEA) erlebte das Land von
2003 bis 2011 eine große wirtschaftliche Expansion (LEVY, 2019) und
erreichte den sechsten Platz in der Rangliste der größten
Volkswirtschaften der Welt. Ab 2014 begann jedoch ein Prozess der
Rezession und Veränderungen in der politischen Verfassung des Staates,
welcher zum Rückfall auf Platz 9 führte. Gegenwärtig \enquote{haben in
Brasilien sechs Familien mehr Vermögen angesammelt, als die 105
Millionen am Fuß der Pyramide. Die Ungleichheit hat ein ethisch,
politisch und wirtschaftlich unhaltbares Niveau erreicht} (DOWBOR, 2019,
Übersetzung der Verfasser). Nach jüngsten Erhebungen der Vereinten
Nationen (UNO) steht Brasilien bezogen auf seine Einkommenskonzentration
an zweiter Stelle weltweit und liegt dabei nur hinter Katar (UNO, 2019).

Insbesondere bezüglich der Lese- und Schreibfähigkeit und
Alphabetisierung der brasilianischen Bevölkerung gilt nach den Daten des
Indikators für Funktionale Alphabetisierung (Indicador de Alfabetismo
Funcional -- Inaf)\footnote{Inaf ist eine Studie, die von zwei
  brasilianischen Institutionen (Instituto Paulo Montenegro und die NGO
  Ação Educativa) in Zusammenarbeit durchgeführt wird und das Ziel
  verfolgt, den Alphabetisierungsgrad der brasilianischen Bevölkerung
  zwischen 15 und 64 Jahren zu messen und ihre Lese-, Schreib- und
  Mathematikkenntnisse und -praktiken im Alltag zu bewerten.
  Weiterführende Informationen verfügbar unter:
  \url{https://ipm.org.br/inaf} (17.01.2020).} von 2018, dass drei von
zehn Jugendlichen und Erwachsenen im Alter von 15 bis 64 Jahren im Land
als funktionale Analphabeten gelten. Sie befinden sich also auf dem
niedrigsten Niveau der Fertigkeiten und des Schreibens.\footnote{Der
  Inaf stuft den Alphabetisierungsgrad in fünf Gruppen ein: Analphabeten
  (8~\%, die Wörter und Sätze nicht lesen können); rudimentäre (21~\%,
  die zum Beispiel keine Informationen in einem Kalender finden können);
  elementare (34~\%), mittlere/ intermediär (25~\%) und kompetent
  (12~\%, die im Alphabetisierungsranking stehen).} Dabei handelt es
sich um 29~\% der Gesamtbevölkerung, was etwa 38 Millionen Menschen
entspricht.

Brasilien ist ein Land, das sich durch Vielfalt und seine kontinentale
Dimension auszeichnet, reich an Kultur und natürlichen Ressourcen ist,
aber auch durch Ungleichheit in seinen verschiedenen sozialen,
kulturellen, wirtschaftlichen und bildungspolitischen Kontexten geprägt
ist. In diesem komplexen Universum präsentieren öffentliche Bibliotheken
Strategien, dem Mangel an kulturellen Räumen, an qualitativ hochwertiger
Bildung und an Zugang zu Information und Lektüre für einen großen Teil
der Bevölkerung, zu begegnen.

In Brasilien qualifizieren die Zugangsbedingungen, die Ausrichtung der
Sammlung, über die eine Bibliothek verfügt sowie der Kundenservice, ob
eine Bibliothek öffentlich ist oder nicht. In den meisten Fällen werden
sie von lokalen Regierungen gegründet und unterhalten und zu kommunalen
und staatlichen öffentlichen Bibliotheken bestimmt. Es gibt aber auch
Bibliotheken, die von gemeinnützigen Organisationen und
Jugendkollektiven geschaffen und unterhalten werden. In diesem Fall
spricht man von Gemeindebibliotheken.

\begin{quote}
\enquote{Die von der Regierung unterhaltene öffentliche Bibliothek wird
als öffentliches Gut verstanden, und ihre Räume und Sammlungen werden
als kulturelles Erbe einer bestimmten Gemeinschaft betrachtet. Sie ist
ein Recht des Volkes und eine Pflicht des Staates {[}...{]}} (MACHADO;
CALIL JUNIOR, 2019, S. 212, Übersetzung der Verfasser).
\end{quote}

In diesem Zusammenhang verzeichnete das nationale System öffentlicher
Bibliotheken (Sistema Nacional de Bibliotecas Públicas --
SNBP)\footnote{Online verfügbar unter: \url{http://snbp.cultura.gov.br}
  (07.01.2020).} im Jahr 2015 die Existenz von 6.057 öffentlichen
Bibliotheken. Diese sind über das gesamte Staatsgebiet verteilt, wobei
sich 1.957 der Bibliotheken in der Region Südosten konzentrieren, das
heißt in den Staaten São Paulo, Rio de Janeiro, Minas Gerais und
Espírito Santo. Zugleich zeigte die vom Brasilianischen Institut für
Geographie und Statistik (Instituto Brasileiro de Geografia e
Estatística -- IBGE) durchgeführte Umfrage zur kommunalen
Basisinformation im Jahr 2018, dass die Gesamtzahl der brasilianischen
Gemeinden mit öffentlichen Bibliotheken in vier Jahren um fast 10~\%
zurückging. Die Zahl fiel von 97,7~\% im Jahr 2014 auf 87,7~\% im Jahr
2018, so dass aktuell auf 34.000 Einwohner etwa eine Bibliothek kommt.

Vergleicht man die Daten der \emph{Library map of the world}\footnote{Online
  verfügbar unter: \url{https://librarymap.ifla.org/map} (17.01.2020).},
die von der \emph{International Federation of Library Associations and
Institutions} (IFLA) zur Verfügung gestellt wird, so gibt es in
Brasilien mit mehr als 211 Millionen Einwohnern weniger öffentliche
Bibliotheken als in Italien mit 60 Millionen Einwohnern. Das Land
verfügt also nicht über eine ausreichende Anzahl an Bibliotheken, um die
Informations-, Lese- und Kulturbedürfnisse der Bevölkerung zu
befriedigen.

Für die Ausbildung von Fachpersonal für die Beschäftigung in
öffentlichen Bibliotheken in Brasilien, sowie in Bibliotheken anderer
Art, wie Schul-, Universitäts- und Fachbibliotheken, gibt es 57
grundständige Studiengänge in Bibliothekswissenschaft mit einer
durchschnittlichen Dauer von 4 Jahren, die von öffentlichen und privaten
Hochschuleinrichtungen angeboten werden.

Es soll an dieser Stelle hervorgehoben werden, dass es im Gegensatz zu
anderen Ländern für die Ausübung des Berufs des Bibliothekars/der
Bibliothekarin in Brasilien \enquote{notwendig ist, einen
Bachelorabschluss in Bibliothekswissenschaft gemäß den Vorschriften des
Bildungsministeriums (MEC) zu haben und seine Registrierung beim
Regionalrat für Bibliothekswissenschaft seiner Region aktuell zu halten}
(CALIL JUNIOR; SILVEIRA; SILVA; ROSA, 2015, S. 4, Übersetzung der
Verfasser). Die große Mehrheit der brasilianischen Staats- und
Stadtbibliotheken verfügt über öffentliche Angestellte (Beamte), also
Fachpersonal, das in einem Auswahlverfahren -- in der Regel mit
Prüfungen -- für feste Stellen in der öffentlichen Verwaltung zugelassen
wurde.

Aus diesem Grund ist es unerlässlich, die Studien und Debatten über eine
Ausbildung kritischer Bibliothekare für die Arbeit in öffentlichen
Bibliotheken auszuweiten. Die grundständigen Studiengänge des Landes im
Bereich Bibliotheks- und Informationswissenschaft können als
generalistisch bezeichnet werden, da sie in ihren Lehrplänen keinen
Schwerpunkt auf die Besonderheiten legen, die die verschiedenen Arten
von Bibliotheken mit sich bringen. Der Mangel an Diskussionen und
spezifischen Inhalten im Grundstudium führt dazu, dass die Fachkräfte
nach ihrem Abschluss nach Spezialisierungs- und weiterführenden
Studiengängen suchen, die an öffentlichen und privaten Universitäten
angeboten werden, um ihr Wissen und ihre praktischen Kenntnisse zu
vertiefen sowie sich weiterführend zu qualifizieren.

Die Zahl der Fachkräfte, die jährlich einen Abschluss in
Bibliothekswissenschaft erwerben, reicht nicht aus, um alle vorhandenen
Stellen in den Bibliotheken ganz Brasiliens zu besetzen.

Hinzu kommen die Rückschritte im Bereich der \emph{Public Policy} im
Land, die seit dem Putsch gegen Präsidentin Dilma Rousseff im Jahr 2016
stattfinden. Neben der Auflösung des Kulturministeriums im Januar 2019
sind öffentliche Einrichtungen und Leitprogramme in den Bereichen
Kultur, Bildung, Gesundheit und Umwelt nicht nur Ressourcenkürzungen und
dem Abbau der öffentlichen Verwaltung zum Opfer gefallen, sondern auch
Delegitimierungskampagnen der Regierungsbehörden ausgesetzt.

Ausgehend von dieser Perspektive ist es von grundlegender Bedeutung, die
theoretischen und praktischen Aspekte zu diskutieren, die die Ausbildung
der Bibliothekswissenschaft mit sich bringen, insbesondere jene, die das
Verständnis der öffentlichen Bibliothek als kulturelle Einrichtung
betreffen und dabei vor allem jenes durch Macht- und Wissensstrategien
beeinflusste Verständnis, das sich auf die Gesellschaft auswirkt, in die
es eingebettet ist (FOUCAULT, 2006).

Es ist darauf hinzuweisen, dass die öffentlichen Bibliotheken, wenn auch
in unzureichender Anzahl, immer noch die kulturellen Einrichtungen mit
der größten Präsenz in den brasilianischen Gemeinden sind. In der von
Ungleichheiten geprägten brasilianischen Gesellschaft, in der es auf der
einen Seite Orte mit breitem Zugang zu kulturellen, bildungstechnischen,
politischen und Konsumgütern gibt, die aber auf der anderen Seite von
großen ausgegrenzten Regionen und Bevölkerungsgruppen umgeben sind,
können kulturelle Einrichtungen wie öffentliche Bibliotheken zu
pluralistischen Räumen der Inklusion und der sozialen Transformation
werden.

Nachdem die Situation in Brasilien im Überblick präsentiert wurde, ist
es nun möglich, das Ausbildungssystem der Bibliothekswissenschaft an der
UNIRIO und die Forschungsgruppe Öffentliche Bibliotheken in Brasilien
{[}port.: Grupo de Pesquisa Bibliotecas Públicas no Brasil (GPBP){]} in
ihrer Struktur, ihren institutionellen Verbindungen und den entwickelten
Aktivitäten vorzustellen.

\hypertarget{die-forschungsgruppe-uxf6ffentiche-bibliotheken-in-brasilien-gpbp}{%
\section{3 Die Forschungsgruppe Öffentiche Bibliotheken in
Brasilien
(GPBP)}\label{die-forschungsgruppe-uxf6ffentiche-bibliotheken-in-brasilien-gpbp}}

Zunächst soll das Studium der Bibliothekswissenschaft an der
\enquote{Hochschule für Bibliothekswissenschaften} der UNIRIO
vorgestellt werden, das eines der ältesten und traditionsreichsten des
Landes ist\footnote{Der Studiengang für Bibliothekswissenschaft an der
  \enquote{Hochschule für Bibliothekswissenschaften} der UNIRIO ist der
  älteste des Landes und Lateinamerikas und gilt als der drittälteste
  der Welt. Durch den Artikel 34 des Dekrets Nr. 8.835 vom 11. Juli 1911
  wurde der erste Studiengang strukturiert, dessen Aktivitäten offiziell
  am 10. April 1915 in der Nationalbibliothek begannen. 1969 wurde er in
  den Bund der selbstständigen Bundesschulen des Bundesstaates Guanabara
  (Fefieg) integriert, aus der später die UNIRIO hervorging.}.
Gegenwärtig bietet die Universität zwei inhaltsgleiche Studiengänge für
das Bachelorstudium an, einen vormittags und einen abends sowie einen
weiteren für das Studium \enquote{auf Lehramt} abends.\footnote{Das
  brasilianische Studium \enquote{auf Lehramt} {[}port.: curso de
  licenciatura{]} befähigt den Absolventen dazu, in dem Wissensgebiet,
  in dem er seinen Abschluss gemacht hat, sowohl in der frühkindlichen
  Bildung und an der Grundschule, als auch am Gymnasium als Lehrer tätig
  zu sein.} Um eine Vorstellung zu bekommen: Im ersten Semester des
Jahres 2020 verzeichnete die UNIRIO 816 Studenten, die in den drei
grundständigen Studiengängen eingeschrieben waren. Darüber hinaus wurde
im Jahr 2012 das weiterführende Studienprogramm für
Bibliothekswissenschaft mit einem anwendungsorientierten
Masterstudiengang ins Leben gerufen. Dabei handelt es sich ebenfalls um
den Ersten des Landes in diesem Wissenschaftsbereich.

Laut der Koordinierungsstelle für Weiterbildung für Menschen mit
Hochschulabschluss {[}port.: Coordenação de Aperfeiçoamento de Pessoas
de Nível Superior -- CAPES{]} ist das anwendungsorientierte
Masterstudium anders gestaltet als das forschungsorientierte
Masterstudium,

\begin{quote}
\enquote{es ist eine Art von weiterführendem Studiengang \emph{stricto
sensu}, der auf die Ausbildung von Fachpersonal in den verschiedenen
Wissensbereichen durch das Studium von Techniken, Prozessen oder Themen
ausgerichtet ist, die eine bestimmte Nachfrage des Arbeitsmarktes
befriedigen. Ziel ist es, einen Beitrag zum nationalen Produktionssektor
zu leisten, um ein höheres Niveau an Wettbewerbsfähigkeit und
Produktivität für Unternehmen und Organisationen, ob öffentlich oder
privat, zu erreichen. Folglich müssen die Entwürfe für neue Studiengänge
des Typus \enquote{anwendungsorientierter Master} eine curriculare Struktur
aufweisen, die die Verbindung aus aktuellem Wissen, der Beherrschung der
entsprechenden Methodik und der auf das spezifische Berufsfeld
ausgerichteten Anwendung betont. Dazu muss ein Teil des Lehrkörpers aus
Fachpersonal bestehen, das auf seinen Wissensgebieten für seine
Qualifikation und seine herausragenden Leistungen auf dem für den
Studiengangentwurf relevanten Gebiet anerkannt ist. Die Abschlussarbeit
des Studiums soll immer mit realen Problemen aus dem Tätigkeitsbereich
der Studierenden in Verbindung stehen und kann -- je nach Art des
Bereichs und Zweck des Studiums -- in verschiedenen Formaten präsentiert
werden} (CAPES, 2019, Übersetzung der Verfasser).
\end{quote}

Der Masterstudiengang Bibliothekswissenschaft an der UNIRIO, der 2012
seinen ersten Jahrgang hatte, entstand \enquote{aus der Wahrnehmung
eines Mangels an vertiefenden Studien auf dem Gebiet der
Bibliothekswissenschaft, um diese Fachleute in die Lage zu versetzen,
Probleme zu untersuchen, die sich aus dem Alltag der
Bibliothekswissenschaft in verschiedenen Bibliotheken ergeben}. (UNIRIO,
2013)

Nachdem das Studium der Bibliothekswissenschaft an der UNIRIO
vorgestellt wurde, soll nun die Arbeit der Forschungsgruppe präsentiert
werden. In der brasilianischen Universitätsstruktur bilden
Forschungsgruppen einen potenziellen Rahmen, der zur Erfüllung des
Auftrags öffentlicher Universitäten beitragen kann welcher darin liegt,
die Untrennbarkeit von Lehre, Forschung und \emph{Extensão}\footnote{Bei
  der \emph{Extensão} handelt es sich um einen Bestandteil der
  universitären Ausbildung in Brasilien. Im Rahmen der \emph{Extensão}
  erhalten Studierende während des Studiums ein Stipendium der
  Universität und arbeiten an universitätsexternen Projekten, die die
  Anwendung des theoretischen Wissens in der Gesellschaft zum Gegenstand
  haben. Die Extension-Projekte sind die praktische Verbindung der
  wissenschaftlichen Erkenntnisse aus Lehre und Forschung mit den
  Bedürfnissen der Gemeinschaft, in die die Universität eingebettet ist,
  wobei sie mit der sozialen Realität interagieren und diese verändern.
  \enquote{Die Idee der \emph{Extensão} ist mit der Überzeugung
  verbunden, dass das von den Forschungsinstitutionen generierte Wissen
  {[}...{]} die soziale Wirklichkeit transformieren soll, indem es ihre
  Defizite adressiert und sich nicht nur auf die Ausbildung der
  regulären Studierenden dieser Institution beschränkt}, Übersetzung und
  Hervorhebung der Verfasser, online verfügbar unter:
  \url{https://pt.wikipedia.org/wiki/Extens\%C3\%A3o_universit\%C3\%A1ria}
  (10.02.2020).} -- also die Verknüpfung von Theorie und Praxis -- zu
gewährleisten, die wir bibliotheks- und informationswissenschaftliche
Praxis nennen\footnote{Der Begriff der bibliothekarisch-informatorischen
  Praxis, der hier verwendet wird, ist inspiriert von dem Konzept der
  Praxis, das von Autoren wie Marx, Gramsci und Paulo Freire vertreten
  wird. Diese Praxis wird als eine dialektische Beziehung zwischen
  Theorie und Praxis verstanden, die soziale Veränderungen hervorruft.},
eine Formulierung, die von Paulo Freire\footnote{Paulo Freie
  (1921--1997) war ein brasilianischer Pädagoge und Philosoph, der in
  den 1960er Jahren eine Methode der Erwachsenenalphabetisierung
  einführte, mit der die Landarbeiter in 40 Tagen alphabetisiert wurden.
  Auf Grundlage dieser Erfahrung wurde ein Pilotprojekt für das
  Nationale Alphabetisierungsprogramm für Erwachsene gestartet, das
  durch den Staatsstreich von 1964 unterbrochen wurde. Freire erhielt 43
  Ehrendoktorwürden und gilt als einer der größten brasilianischen
  Intellektuellen. Sein Werk \enquote{Pädagogik der Unterdrückten}
  gehört zu den meistzitierten Werken weltweit (FREIRE, 2019).}
inspiriert wurde.

Die Forschungsgruppe Öffentliche Bibliotheken in Brasilien hat ihre
Aktivitäten im Jahr 2013 aufgenommen. Ihre Gründung erfolgte aus der
Sorge über die Bedingungen der brasilianischen öffentlichen
Bibliotheken, insbesondere den wachsenden Bedarf an qualifiziertem
Personal für die Arbeit in den Bibliotheken, die Notwendigkeit, das
Nachdenken über die Bibliothekspraxis im öffentlichen Raum zu fördern
und zu stärken und, wie bereits erwähnt, die Suche nach dem Dialog
zwischen verschiedenen Wissensträgern.

Die GPBP setzt sich zusammen aus Professoren der UNIRIO und Dozenten,
die an anderen brasilianischen Universitäten und Institutionen in
verschiedenen Regionen des Landes, wie beispielsweise an der
Bundesuniversität von Minas Gerais (UFMG) und am Kulturzentrum Luiz
Freire (CCLF) in Pernambuco, tätig sind. Studierende im Grund- und
Aufbaustudium sowie Fachpersonal von öffentlichen und
Gemeindebibliotheken sind ebenfalls Mitglied. Jede/r nimmt direkt oder
indirekt an Forschungs-, Lehr- und \emph{Extensão}-Projekten teil, die
sich um das Thema öffentliche Bibliotheken drehen.

Die Projekte der GPBP betreffen konstitutive Fragen der öffentlichen
Bibliothekswissenschaft in Brasilien. Die öffentliche
Bibliothekswissenschaft bezieht sich auf

\begin{quote}
\enquote{einen Ausschnitt oder einen Zweig der Bibliothekswissenschaft,
der sich mit den Besonderheiten von öffentlichen Bibliotheken und der
Ausbildung von Fachpersonal für die Tätigkeit in dieser Art von
Einrichtung beschäftigt. Als Disziplin befasst sie sich mit den
Ursprüngen, Funktionen, Zielen, Merkmalen und Konzepten von öffentlichen
Bibliotheken und den öffentlichen Bibliothekssystemen (national,
bundesstaatlich und kommunal). Außerdem befasst sie sich mit
Gemeindebibliotheken, die von Kollektiven unterhalten und verwaltet
werden, dem Lesen und der Lektürevermittlung als Praktiken in diesen
Räumen sowie mit den Fragen der öffentlichen Verwaltung und der
öffentlichen Kulturpolitik, die sich auf diese Art der öffentlichen
Einrichtung ausrichten} (MACHADO; CALIL JUNIOR, 2019, S. 211, freie
Übersetzung der Verfasser).
\end{quote}

In diesem Zusammenhang nimmt die GPBP \enquote{die Herausforderung an,
die Debatte über die öffentliche Bibliothekswissenschaft an der UNIRIO
zu fördern, um Wege für die Entwicklung eines kritischen Nachdenkens
über die bibliothekarische Praxis und das Handeln zugunsten des
Gemeinwohls innerhalb und außerhalb des universitären Raums aufzuzeigen}
(MACHADO; CALIL JUNIOR, 2019, S. 216, Übersetzung der Verfasser).

In Bezug auf die Lehre hat die GPBP an der UNIRIO das Fach
\enquote{öffentliche Bibliotheken} eingeführt. Studierende im Bachelor
und Masterstudium finden in der GPBP einen Ort für die Annäherung und
Vertiefung rund um die Themen der öffentlichen Bibliothekswissenschaft.

Zu den Aktivitäten dieses Fachs gehören Exkursionen zu öffentlichen
Bibliotheken und Gemeindebibliotheken, bei denen die Studentinnen und
Studenten die Möglichkeit haben, sich im Stadtraum zu bewegen, mit den
Lokalitäten in Kontakt zu kommen und einen Bericht über diese Erfahrung
zu schreiben. Aus einem dieser Berichte, verfasst von einer der
damaligen \emph{Extensão}-Stipendiatinnen aus einem der Projekte der
GPBP, entstand ein wissenschaftlicher Artikel, der dieser Erfahrung
Ausdruck verleiht. Diese Besuche ermöglichen es, Kontakte mit
Fachpersonal aus öffentlichen Bibliotheken zu knüpfen und die jeweiligen
Praktiken der Bibliotheken kennenzulernen. Auf Grundlage dieser
geknüpften Kontakte wurden Universitätsexterne eingeladen, an
Aktivitäten der Universität teilzunehmen. So hielt etwa ein
Literaturvermittler einen Workshop zur Literaturvermittlung für
Studierende der UNIRIO. Diese Veranstaltung war nicht nur für die
Studentinnen, Studenten und Professorinnen und Professoren hilf- und
lehrreich, sondern auch für den Referenten, der die Möglichkeit hatte,
von den Studierenden zu lernen und andere Perspektiven kennen zu lernen
-- so kam es zu einem Wissensaustausch.

Die Forschung und die Projekte der GPBP umfassen unter anderem folgende
Themen: Bürgerinformationsdienste in öffentlichen Bibliotheken,
Erstellung von Indikatoren für die Lektüre und das Leseverhalten,
Vermittlung und Leseförderung, grüne und nachhaltige Bibliotheken,
Nachhaltigkeit, Kulturpolitik, Begriffe und Konzepte öffentlicher
Bibliotheken, Entwicklung von Sammlungen, Lehre über öffentliche
Bibliotheken, Dienstleistungen, kulturelle Tätigkeiten,
Informationszugänglichkeit sowie Aktionen mit öffentlichen Bibliotheken
und Gemeindebibliotheken.

Die Aktivitäten der GPBP öffnen viele Türen, hauptsächlich aber bauen
sie Verbindungen auf und schaffen Beziehungen zwischen Akteuren, die
unterschiedliche Kenntnisse und unterschiedliche Weltverständnisse
\enquote{repräsentieren}. Derartige Beziehungen ermöglichten zum
Beispiel die Organisation verschiedener Veranstaltungen der GPBP. So hat
die Gruppe dank dieser entstandenen Partnerschaften etwa zwei Ausgaben
des Forums der Öffentlichen Bibliotheken sowie des Treffens
\enquote{Gemeindebibliotheken: zwischen Wissen und Handeln} {[}port.:
\enquote{Bibliotecas comunitárias: entre saberes e fazeres}, Übersetzung
der Verfasser{]} organisieren können, an denen Forschende,
Literaturvermittler, lokale Führungskräfte und öffentliche Vertreter
teilnahmen. Bei beiden Veranstaltungen hatte jede/r die Möglichkeit,
sich zu beteiligen und gehört zu werden. Diese Erfahrungen waren von
großer Bedeutung für die Annäherung von Universität und
Gemeindebibliotheken und beide Seiten profitierten von den neu gewonnen
Informationen und Kontakten. Als weiteres Beispiel für die Aktivitäten
der GPBP sei die Forschung zum Einfluss von Gemeindebibliotheken auf die
Bildung von Lesern in Brasilien genannt, an der die Mitglieder der GPBP
sowohl in der Planung als auch in der Durchführung beteiligt waren.
Diese Forschung vertiefte das Verständnis dafür, wie
Gemeindebibliotheken zu der Bildung von Leserinnen und Lesern beitragen.
Diese Erfahrungen führten zur Veröffentlichung des Buches \enquote{Das
Brasilien, das liest: Gemeindebibliotheken und kultureller Widerstand in
der Bildung von Lesern}. Es beinhaltet eine umfassende Kartierung und
Bestandsaufnahme von Gemeindebibliotheken im in Brasilien (es wurden 143
Bibliotheken in 16 Bundesstaaten erforscht) und ist frei verfügbar.

Es ist hervorzuheben, dass die Ergebnisse dieser Forschung auch aus
einem Zusammenspiel zwischen wissenschaftlichen Erkenntnissen und dem
Wissen der in Gemeindebibliotheken tätigen Personen hervorgegangen sind.
Silvia Castrillon (2018) weist darauf hin, dass die Forschung über
Gemeindebibliotheken die Möglichkeit eröffnet, die Bibliothek als Ort
aus anderen Perspektiven zu betrachten.

In diesem Zusammenhang erscheint es sinnvoll, eine Stimme zu erwähnen,
die die Beziehung zwischen der Bibliothek und ihrem räumlichen Umfeld
adressiert:

\begin{quote}
\enquote{{[}\ldots{]} sie sagen, dass es hier nur Gewalt gibt, dass die
Leute hier gewalttätig sind, aber es gibt viele gute Leute hier.
{[}...{]} Wir sind diejenigen, die weiterhin gewaltsam behandelt werden.
Wir lassen uns nicht von Hindernissen unterkriegen, sondern kämpfen
weiter, um die Realität des Ortes zu verändern und dem Stigma
des \enquote{Gesellschaftsmülls} entgegenzuwirken. Die Eroberung und Erhaltung
der Bibliothek seit etwa vierzig Jahren ist ein Beispiel für den
Reichtum des menschlichen Erbes, was das Viertel zu einem kulturellen
Bezugspunkt macht (GF03)} (FERNANDEZ; MACHADO; ROSA, 2018, S. 45,
Übersetzung des Verfassers).
\end{quote}

In Bezug auf \emph{Extensão} arbeitet GPBP seit 2014 an dem Projekt
\enquote{Öffentliche und Gemeindebibliotheken in Brasilien}, in dessen
Rahmen Studierende und Forschende in öffentlichen Bibliotheken tätig
sind. Ihre Aktivitäten umfassen dabei kulturelle Aktionen,
Literaturvermittlung, die Organisation und technische Bearbeitung von
Sammlungen und ein Kursangebot für Personen, die in dieser Art von
Bibliothek arbeiten. Ein Teil der Ergebnisse dieser Aktivitäten sind in
dem Artikel \enquote{Gemeindebibliotheken: zwischen Wissen und Handeln}
{[}port.: \enquote{Bibliotecas comunitárias: entre saberes e fazeres},
Übersetzung der Verfasser{]} festgehalten, der in der Zeitschrift Raízes
e Rumos (dt.: Wurzeln und Pfade, Übersetzung der Verfasser) der UNIRIO
veröffentlicht wurde (CALIL JUNIOR; MACHADO; KLEIN; SANTOS, 2018).
Insgesamt zeigt sich, dass die Forschung und die umfangreichen Projekte,
die in der und durch die GPBP entwickelt wurden, Kommunikations- und
Diskussionskanäle öffnen, in denen verschiedene Individuen und
Kompetenzen aufeinandertreffen, was eine Annäherung und Interaktion
zwischen der Universität und der Gesellschaft ermöglicht.

\hypertarget{schlussbemerkungen}{%
\section{4 Schlussbemerkungen}\label{schlussbemerkungen}}

In einer Zeit der politischen, wirtschaftlichen und sozialen Krise, in
der Desinformation alltäglich ist und Versuche, Minderheiten zum
Schweigen zu bringen, in der brasilianischen Gesellschaft Widerhall
finden, ist es dringend notwendig, solche pluralistischen Räume
aufrechtzuerhalten. Die Bedeutung der öffentlichen Bibliothek für die
Bevölkerung ist deshalb unbestreitbar ebenso wichtig wie die Bedeutung
der Schaffung besserer Bedingungen für die Fachkräfte, die in diesen
Orten arbeiten, um ihre täglichen Aufgaben ausüben zu können.

Seit 2016 können in Brasilien gezielte -- oft erfolgreiche -- Versuche
beobachtet werden, das Recht auf Zugang zu öffentlichen Institutionen
neu zu definieren. Von Leitartikeln in Zeitungen großer Mediengruppen
bis hin zu Ressourcenkürzungen seitens der aktuellen Regierung finden
sich vielfältige Strategien, um den öffentlichen Sektor abzubauen und
seine Institutionen und Dienstleistungen abzuwerten. Bei den weniger
informierten Teilen der Bevölkerung finden diese Maßnahmen Anklang.
Angesichts dieses Szenarios ist es dringend notwendig, Räume der
Kommunikation zwischen den verschiedenen Wissensbereichen zu schaffen,
die eine Annäherung der wissenschaftlichen Erkenntnisse an die
verschiedenen Räume des täglichen Lebens ermöglichen. Hier treten
Forschungsgruppen als potentielle Akteure für die Förderung von Dialogen
zwischen den verschiedenen Wissensbereichen auf, die konstitutiv für das
tägliche Leben sind.

Die öffentlichen brasilianischen Universitäten spielen in diesem
Zusammenhang -- durch Lehre, Forschung und \emph{Extensão} -- eine
fundamentale Rolle. Die GPBP versucht, Räume der Begegnung und des
Dialogs, der Diskussion und der bibliothekarischen Ausbildung zu
schaffen, indem sie Treffen sozialer Gruppen unterschiedlicher Herkunft
organisiert.

Ihrer Zielsetzung entsprechend setzt sich die Gruppe für die Ausweitung
der Forschung über öffentliche Bibliotheken und die Zunahme dieser Art
kulturellen Grundversorgung in Brasilien ein und versucht, zur
Ausbildung des Personals beizutragen, das nach Lösungen für die realen
Probleme des Zugangs zu Information, Lektüre und Kultur im Land sucht.
Die Forschungsergebnisse des GPBP werden auch im Alltag der öffentlichen
und kommunalen Bibliotheken genutzt.

Es ist zu hoffen, dass die in diesem Artikel präsentierten
brasilianischen Erfahrungen mit Lehre, Forschung und \emph{Extensão} im
Bereich der öffentlichen Bibliotheken einen Einblick, vor allem des
Umfangs der Herausforderungen, denen sich Brasilien gegenübersieht,
gegeben haben und andere Forschende motivieren, in die Forschung zu
diesem Thema zu investieren.

\hypertarget{literaturverzeichnis}{%
\section{Literaturverzeichnis}\label{literaturverzeichnis}}

CALIL JUNIOR, Alberto; SILVEIRA, Naira Christofoletti, SILVA, Laiza Lima
da; ROSA, Victor Soares. A biblioteca pública no currículo dos cursos
brasileiros de Biblioteconomia: relação entre o ensino e o manifesto da
IFLA. In: EDICIC, 7, 2015. \textbf{Desafíos y oportunidades de las
Ciencias de la Información y la Documentación en la era digital}: actas
del VII Encuentro Ibérico EDICIC 2015. Madrid, Universidad Complutense
de Madrid, 2015. Online verfügbar unter:
\url{https://eprints.ucm.es/34687/}. (03 fev. 2020).

CALIL JUNIOR, Alberto; MACHADO, Elisa Campos; KLEIN, Gabriela Falcão;
SANTOS, Luiza Goelzer Machado dos. Bibliotecas comunitárias: entre
saberes e fazeres. \textbf{Raízes e rumos}, Rio de Janeiro, v. 6, n.1,
S. 43-55, jan./jun. 2018.

CAPES. Mestrado Profissional: o que é? Brasília: CAPES, 2019. Online
verfügbar unter:
\url{https://www.capes.gov.br/avaliacao/sobre-a-avaliacao/mestrado-profissional-o-que-e}.
(20 jan. 2020).

CASTRILLON, Silvia. A biblioteca comunitária: uma oportunidade. In:
FERNANDEZ, Cida; MACHADO, Elisa; ROSA, Ester. \textbf{O Brasil que lê}:
bibliotecas comunitárias e resistência cultural na formação de leitores.
Olinda: CCLF; Brasil: RNBC, 2018. S. 9-11.

DOWBOR, Ladislau. \textbf{A economia desgovernada}: novos paradigmas.
São Leopoldo: IHU, 2019. Online verfügbar unter:
\href{http://www.ihu.unisinos.br/78-noticias/593739-a-economia-desgovernada-novos-paradigmas-artigo-de-ladislau-dowbor}{http://www.ihu.unisinos.br/78-noticias/593739-a-economia-desgover-nada-novos-paradigmas-artigo-de-ladislau-dowbor}.
(12 jan. 2020).

FERNANDEZ, Cida; MACHADO, Elisa; ROSA, Ester. \textbf{O Brasil que lê}:
bibliotecas comunitárias e resistência cultural na formação de leitores.
Olinda: CCLF; Brasil: RNBC, 2018.

FOUCAULT, Michel. \textbf{Ditos \& Escritos}: Estratégias de saber e
poder. 3. ed.~Rio de Janeiro: Forense Universitária, 2006. Coleção Ditos
\& Escritos, v. IV.

FREIRE, Paulo. \textbf{Educação como prática da liberdade}. 45. ed.~São
Paulo : Paz e Terra, 2019.

GRUPO DE PESQUISA BIBLIOTECAS PÚBLICAS NO BRASIL: reflexão e prática.
\textbf{Sobre o grupo}. Rio de Janeiro: UNIRIO, 2013. Online verfügbar
unter: \url{http://culturadigital.br/gpbp/}. (20 jan. 2020).

IFLA. \textbf{IFLA library map of the world}. Contries. Seattle: IFLA,
2017. Online verfügbar unter: \url{https://librarymap.ifla.org/map}. (03
fev. 2020).

INSTITUTO BRASILEIRO DE GEOGRAFIA E ESTATÍSTICA. \textbf{Pesquisa de
Informações Básicas Municipais}: Perfil dos Municípios Brasileiros 2018.
Rio de Janeiro: IBGE, 2019. Online verfügbar unter:
\url{https://biblioteca.ibge.gov.br/visualizacao/livros/liv101668.pdf}.
(20 jan. 2020).

LEVY, Paulo Mansur. Sumário. \textbf{Carta de Conjuntura IPEA}: seção II
economia mundial, n.~45, 4º. Trimestre, 2019.

MACHADO, Elisa; CALIL JUNIOR, Alberto. Articulação entre ensino,
pesquisa e extensão a partir das experiências do Grupo de Pesquisa
Bibliotecas Públicas no Brasil. In: BORGES, Jussara; NOVO, Hildenise
Ferreira (Org.) \textbf{Da organização do conhecimento à apropriação dos
saberes}: ensino e pesquisa em informação. Salvador: EDUFBA, 2019. S.
211-226.

MIRANDA, Marcos Luiz Cavalcanti de. Na Universidade Federal do Estado do
Rio de Janeiro (URINIO). In: \textbf{Chronos}: publicacao cultural da
UNIRIO. Universidade Federal do Estado do Rio de Janeiro. -- v. 1, n.~10
(2009) - Rio de Janeiro: UNIRIO, 2015, S. 54-80. Online verfügbar unter:
\url{http://www.unirio.br/proreitoriadeextensaoecultura/publicacoes/revista-chronos/ano-08-2013-numero-10-2014-100-anos-de-instalacao-da-escola-de-biblioteconomia}.
(20 jan. 2020).

UNO. \textbf{Relatório de desenvolvimento humano do PNUD destaca altos
índices de desigualdade no Brasil.} Brasil: UNO, 2019. Online verfügbar
unter:
\url{https://nacoesunidas.org/relatorio-de-desenvolvimento-humano-do-pnud-destaca-altos-indices-de-desigualdade-no-brasil/}.
(03 fev. 2020).

SISTEMA NACIONAL DE BIBLIOTECAS PÚBLICAS. \textbf{Informações das
bibliotecas públicas}. Brasília: SNBP, 2015. Online verfügbar unter:
\url{http://snbp.cultura.gov.br/bibliotecaspublicas/}. (20 jan. 2020).

UNIRIO. \textbf{Programa de Pós-graduação em Biblioteconomia}. História
e linhas de pesquisa. Rio de Janeiro: UNIRIO, 2013. Online verfügbar
unter: \url{http://www.unirio.br/ppgb/programa}. (20 jan. 2020).

%autor
\begin{center}\rule{0.5\linewidth}{0.5pt}\end{center}

\textbf{Nathalice Bezerra Cardoso} Stipendiatin -- Alexander von
Humboldt Stiftung. Bundeskanzler-Stipendium (BUKA) 2019--2020. Master in
Bibliothekswissenschaft. E-Mail:
\href{mailto:nathalice@gmail.com}{\nolinkurl{nathalice@gmail.com}}

\textbf{Alberto Calil Elias Junior} Bundesuniversität des Bundesstaates
Rio de Janeiro (UNIRIO) -- Rio de Janeiro, Brasilien.
Universitätsprofessor -- Departement für Bibliothekswissenschaft
(UNIRIO). Doktor in Sozialwissenschaften (UERJ). E-Mail:
\href{mailto:caliljr@unirio.br}{\nolinkurl{caliljr@unirio.br}}

\textbf{Elisa Campos Machado} Bundesuniversität des Bundesstaates Rio de
Janeiro (UNIRIO) -- Rio de Janeiro, Brasilien. Universitätsprofessorin
-- Departement für Bibliothekswissenschaft (UNIRIO). Doktor in
Informationswissenschaft (USP). E-Mail:
\href{mailto:emachado2005@gmail.com}{\nolinkurl{emachado2005@gmail.com}}

\end{document}
