\documentclass[a4paper,
fontsize=11pt,
%headings=small,
oneside,
numbers=noperiodatend,
parskip=half-,
bibliography=totoc,
final
]{scrartcl}

\usepackage[babel]{csquotes}
\usepackage{synttree}
\usepackage{graphicx}
\setkeys{Gin}{width=.4\textwidth} %default pics size

\graphicspath{{./plots/}}
\usepackage[ngerman]{babel}
\usepackage[T1]{fontenc}
%\usepackage{amsmath}
\usepackage[utf8x]{inputenc}
\usepackage [hyphens]{url}
\usepackage{booktabs} 
\usepackage[left=2.4cm,right=2.4cm,top=2.3cm,bottom=2cm,includeheadfoot]{geometry}
\usepackage{eurosym}
\usepackage{multirow}
\usepackage[ngerman]{varioref}
\setcapindent{1em}
\renewcommand{\labelitemi}{--}
\usepackage{paralist}
\usepackage{pdfpages}
\usepackage{lscape}
\usepackage{float}
\usepackage{acronym}
\usepackage{eurosym}
\usepackage{longtable,lscape}
\usepackage{mathpazo}
\usepackage[normalem]{ulem} %emphasize weiterhin kursiv
\usepackage[flushmargin,ragged]{footmisc} % left align footnote
\usepackage{ccicons} 
\setcapindent{0pt} % no indentation in captions

%%%% fancy LIBREAS URL color 
\usepackage{xcolor}
\definecolor{libreas}{RGB}{112,0,0}

\usepackage{listings}

\urlstyle{same}  % don't use monospace font for urls

\usepackage[fleqn]{amsmath}

%adjust fontsize for part

\usepackage{sectsty}
\partfont{\large}

%Das BibTeX-Zeichen mit \BibTeX setzen:
\def\symbol#1{\char #1\relax}
\def\bsl{{\tt\symbol{'134}}}
\def\BibTeX{{\rm B\kern-.05em{\sc i\kern-.025em b}\kern-.08em
    T\kern-.1667em\lower.7ex\hbox{E}\kern-.125emX}}

\usepackage{fancyhdr}
\fancyhf{}
\pagestyle{fancyplain}
\fancyhead[R]{\thepage}

% make sure bookmarks are created eventough sections are not numbered!
% uncommend if sections are numbered (bookmarks created by default)
\makeatletter
\renewcommand\@seccntformat[1]{}
\makeatother

% typo setup
\clubpenalty = 10000
\widowpenalty = 10000
\displaywidowpenalty = 10000

\usepackage{hyperxmp}
\usepackage[colorlinks, linkcolor=black,citecolor=black, urlcolor=libreas,
breaklinks= true,bookmarks=true,bookmarksopen=true]{hyperref}
\usepackage{breakurl}

%meta
%meta

\fancyhead[L]{S. Juen\\ %author
LIBREAS. Library Ideas, 37 (2020). % journal, issue, volume.
\href{http://nbn-resolving.de/}
{}} % urn 
% recommended use
%\href{http://nbn-resolving.de/}{\color{black}{urn:nbn:de...}}
\fancyhead[R]{\thepage} %page number
\fancyfoot[L] {\ccLogo \ccAttribution\ \href{https://creativecommons.org/licenses/by/4.0/}{\color{black}Creative Commons BY 4.0}}  %licence
\fancyfoot[R] {ISSN: 1860-7950}

\title{\LARGE{Die Bibliothek als physisches Angriffsziel während sozialer Unruhen}}% title
\author{Sara Juen} % author

\setcounter{page}{1}

\hypersetup{%
      pdftitle={Die Bibliothek als physisches Angriffsziel während sozialer Unruhen},
      pdfauthor={Sara Juen},
      pdfcopyright={CC BY 4.0 International},
      pdfsubject={LIBREAS. Library Ideas, 37 (2020).},
      pdfkeywords={Öffentliche Bibliothek, Forschung, soziale Unruhen, Gewalt, Polizei, Frankreich, Griechenland, Schweden, England},
      pdflicenseurl={https://creativecommons.org/licenses/by/4.0/},
      pdfcontacturl={http://libreas.eu},
      baseurl={http://libreas.eu},
      pdflang={de},
      pdfmetalang={de}
     }



\date{}
\begin{document}

\maketitle
\thispagestyle{fancyplain} 

%abstracts

%body
\hypertarget{vorwort}{%
\section{1 Vorwort}\label{vorwort}}

Unter dem Titel dieses Artikels reichte ich im September 2019 meine
Bachelorarbeit am Institut für Bibliotheks- und Informationswissenschaft
an der Humboldt-Universität zu Berlin ein. Das Thema beschäftigte mich
schon mein ganzes Studium, seit ich im ersten Semester einen Artikel des
französischen Soziologen Denis Merklen über Bibliotheksbrände in
Frankreich gelesen hatte. Die Reaktion, die ich von meinen Mitmenschen,
ob privat, auf der Arbeit in der Bibliothek, oder im wissenschaftlichen
Umfeld der Universität, bekommen habe, wenn ich dieses Thema ansprach,
war immer wieder dieselbe. Erstaunen, Unglaube und schnell auch Unmut
waren die Reaktionen, wenn ich erzählte, dass Bibliotheken während
sozialer Unruhen in Europa angegriffen werden. Und zwar nicht in ferner
Vergangenheit, sondern vor fünfzehn, zehn oder sieben Jahren. Niemand
schien über diese Ereignisse im Bilde zu sein. Auf eine unbestimmte Art
und Weise fühlen sich Menschen persönlich angegriffen, wenn sie
erfahren, dass Bibliotheken vorsätzlich zerstört werden. Diese
Reaktionen, genauso wie mein eigenes Erstaunen und Ungewahr sein dieser
Vorfälle, motivierte mich, mehr darüber heraus zu finden und zu
dokumentieren. Die Ergebnisse meiner Forschung, die Schwierigkeiten, mit
denen ich zu kämpfen hatte während des Prozesses und die Erkenntnisse,
die ich daraus gewonnen habe, werden in diesem Artikel vorgestellt.

\hypertarget{einleitung}{%
\section{2 Einleitung}\label{einleitung}}

Seit Beginn des 21. Jahrhunderts hat Europa eine Welle sozialer Unruhen
massiven Ausmaßes erlebt (vergleiche Dikeç 2017, S. 2; Moran; Waddington
2016, S. 1; Mayer et al.~2016, S. 3f). Über die wochenlangen
Ausschreitungen in den Vorstädten von Frankreich von 2005 wurde
ausführlich in der internationalen Presse berichtet. Es gab unzählige
Bilder von brennenden Autos. Bilder von Jugendlichen, die sich
Straßenschlachten mit der Polizei lieferten. 2008 folgte Griechenland,
2011 England und 2013 passierte es in Schweden. Dies sind die Unruhen,
die für diese Arbeit wichtig sind; sie stellen aber keine vollständige
Liste dar. Andere kollektive Massenbewegungen fanden statt, allerdings
nicht mit der Dimension an Zerstörung. Es ist eine Reaktion meist junger
Menschen auf soziale Ungerechtigkeit, Stigmatisierung und Segregation
(vergleiche Thörn et al.~2016, S. 8f).

Dass während dieser Unruhen Bibliotheken Ziele von Angriffen wurden, ist
kaum bekannt. Die öffentliche Bibliothek als Einrichtung für die
Bevölkerung, gleichzeitig aber auch als eine Institution des Staates,
hat es mit ihrer Stellung in diesen Konflikten nicht leicht (vergleiche
Merklen 2015, S. 539). Trotzdem scheint es abwegig, eine Institution des
Wissens, welche allen zugänglich ist, in Brand zu stecken. Es hat den
Anschein, als ob sich der Mensch damit nur selbst Schaden zufügt, sich
etwas beraubt. Warum passiert es also trotzdem?

Diese und weitere Fragen ergaben die Forschungsfrage: \emph{\enquote{Was
führt dazu, dass Bibliotheken zu Angriffszielen während sozialer Unruhen
werden?}}

Um diese komplexe, disziplinübergreifende Frage so gut wie möglich
beantworten zu können, mussten die beiden Hauptthemen, die Bibliothek
und die sozialen Unruhen, zusammengeführt werden. In Abschnitt 3 wird
der aktuelle Forschungsstand erläutert. Im Anschluss gibt Abschnitt 4
eine kurze Übersicht über die sozialen Unruhen in Europa seit 2000. In
Abschnitt 5 wird die Methodik vorgestellt, mit der die Vorfälle
erarbeitet wurden, um nachfolgend auf die einzelnen Ereignisse
einzugehen. Danach wird auf Vorfälle in Südafrika eingegangen, die
Ähnlichkeiten mit den europäischen aufweisen (Abschnitt 6). Anschließend
folgt die Diskussion (Abschnitt 7), in der die vorgestellten Ereignisse
auf Parallelen und Abweichungen untersucht werden. Im abschließenden
Fazit (Abschnitt 8) wird zusammenfassend auf die Fragestellung und die
Schwierigkeiten eingegangen, die bei dem Versuch der Beantwortung dieser
Frage aufgetaucht sind. Darauf folgt ein Ausblick auf zukünftige
Forschungsfragen und Problemstellungen.

\hypertarget{aktueller-forschungsstand}{%
\section{3 Aktueller
Forschungsstand}\label{aktueller-forschungsstand}}

Wie in der Einleitung bereits erwähnt wurde, weist die bearbeitete
Themenstellung eine Lücke in der Bibliotheks- und
Informationswissenschaft auf. Der einzige Artikel, der in deutscher
Sprache ermittelt werden konnte, trotz intensiver Recherche, ist ein
Beitrag von Denis Merklen in der Fachzeitschrift BuB (Forum Bibliothek
und Information). In diesem Beitrag fasst er die Ergebnisse aus seiner
20-jährigen, sozialwissenschaftlichen Forschung zu Bibliotheksbränden in
Frankreich zusammen (vergleiche Merklen 2015, S. 536--539). Das Thema
wurde also in einem Bibliotheks- und Informationswissenschaftlich
relevantem Medium vorgestellt, wurde aber nicht von dieser untersucht.
In seinem Artikel spricht Merklen das Problem der nicht vorhandenen
Debatte über die Bibliotheksangriffe an:

\begin{quote}
\enquote{Es gibt Gewalttaten, die eine Flut von Stellungnahmen und große
Emotionen hervorrufen. Es gibt andere, über die wird kaum geredet. So
weiß ein großer Teil der Öffentlichkeit nichts von den Attacken, die
seit einiger Zeit auf Bibliotheken in Frankreich verübt werden.} (ebenda
S. 536).
\end{quote}

Eine Feststellung, die durch meine Nachforschungen bestätigt werden
kann. Das allgemeine Schweigen zu dieser Thematik deutet auf eine
Überforderung und Reaktionsunfähigkeit hin. Der vorliegende Artikel ist
ein Versuch, dies zu überwinden und das Problem sichtbar zu machen,
damit ein Diskurs darüber stattfinden kann.

Denis Merklen hat das Phänomen der Bibliotheksbrände über Jahre
untersucht und 2013 in einem Buch veröffentlicht. In \emph{Pourquoi
brûle-t-on des bibliothèques?} (zu Deutsch: \emph{Warum werden
Bibliotheken angezündet?}) beleuchtet er einzelne Attacken und die
Umstände, in denen sie vorgefallen sind. In Frankreich wurden von 1994
von 2004 74 Brandanschläge auf Bibliotheken gezählt. Die Liste erhebt
keinen Anspruch auf Vollständigkeit (vergleiche Merklen 2013, S. 40ff).
Er konnte die meisten Brände politischen Ereignissen, wie Wahlen,
zuordnen. Außerdem stellte er fest, dass in mehreren Fällen die
Bibliothek das alleinige Ziel der Attacken war. Keine anderen Gebäude
oder Fahrzeuge wurden beschädigt. Bei den Angriffen ging es um reine
Sachbeschädigung, sie fanden nie tagsüber oder während der
Öffnungszeiten der betroffenen Bibliotheken statt (vergleiche Merklen
2015, S. 536f).

Die Sozialwissenschaft hat sich ausführlich mit dem Gegenstand der
sozialen Unruhen in den Großstädten Europas im 21. Jahrhundert
auseinandergesetzt. Der allgemeine Tenor lautet, dass die Ursache für
diese Art von Massenbewegungen auf das Versagen der Politik und der
neoliberalen Demokratie zurückzuführen ist (vergleiche Dikeç 2017, S. 4;
Thörn et al.~2016, S. 8f). Die Autor\_innen des Buches \emph{Urban
Uprisings -- Challenging Neoliberal Urbanism in Europe} (Mayer et
al.~2016) sehen ein Problem darin, dass zum einen selten Zusammenhänge
zwischen Aufständen und städtebaulichen Ansätze untersucht werden und
zum anderen dass die Unterscheidung zwischen sozialen Unruhen und
sozialen Bewegungen für die Analyse kollektiven Handelns hinderlich ist
(vergleiche ebenda 2016, S. 8). Matthew Moran und David Waddington
untersuchen die Vorfälle unter dem Aspekt des Verhaltens von Gruppen.
Sie stellen fest, dass in Untersuchungen von Massenverhalten oft mit
veralteten und widerlegten Ansätzen gearbeitet wird (vergleiche Moran;
Waddington 2016, S. 3). Mustafa Dikeç stellt, in seinem 2017 erschienen
Buch \emph{Urban Rage. The revolt of the excluded} (Dikeç 2017) eine
\enquote{wave of urban anger} (ebenda, S. 2) fest, die seit der
Jahrhundertwende zunimmt. Er sagt weiter, dass die Zunahme von urbaner
Aufruhr ein Indiz dafür ist, dass die liberale Demokratie daran
gescheitert ist, das Problem der Segregation in Angriff zu nehmen
(vergleiche ebenda, S. 4). Die sozialen Unruhen, die in Europa in den
letzten Jahrzehnten stattgefunden haben, werden in der Wissenschaft als
politische Handlungen wahrgenommen. Eine Vorgehensweise um auf
Missstände, Ungerechtigkeit und Ungleichheit aufmerksam zu machen.

Es gibt eine deutliche Diskrepanz zwischen der Wissenschaft und der
Presse. Um die Ausschreitungen in England als Beispiel zu nehmen: Der
Guardian hat eine Fotostrecke der Titelblätter diverser Zeitungen
zusammengestellt, die über die Ereignisse berichtet haben. Die
Schlagzeilen reichen von \enquote{Flaming Morons. Thugs and thieves
terrorise Britain's streets} bis zu \enquote{mob rules} (the Guardian
2011, o.S.).

Ein weiteres relevantes Wissenschaftsgebiet ist die Stadtsoziologie,
welche ein Spezialgebiet innerhalb der Sozialwissenschaft ist. Die
Stadtsoziologie untersucht Strukturen innerhalb der Städte und ihrer
Bevölkerung. Die Problematik der sozialen Unruhen wird oft in
Zusammenhang gebracht mit städtebaulichen und deren strukturellen
Konzepten. Das stadtpolitische Programm, \emph{la politique de la
ville,} welches entwickelt wurde, um den Problemen in den französischen
Vorstädten zu begegnen, ist Gegenstand verschiedener Untersuchungen.
Mustafa Dikeç beschäftigt sich in seinem 2007 erschienen Buch
\emph{Badlands of the Republic: Space, Politics and Urban Policy}
spezifisch mit der Problematik des stadtpolitischen Programms
(vergleiche Dikeç 2007) und auch deutsche Wissenschaftler\_innen haben
sich damit auseinandergesetzt. In dem im Jahr 2016 erschienenen Buch
\emph{Fraktale Metropolen. Stadtentwicklung zwischen Devianz,
Polarisierung und Hybridisierung} gehen sie eingehend auf diese Thematik
ein und erläutern deren Schwachpunkte und Auswirkungen (vergleiche
Weber; Kühne (Hrsg.) 2016).

\hypertarget{soziale-unruhen-in-europuxe4ischen-grouxdfstuxe4dten-des-21.-jahrhunderts}{%
\section{4 Soziale Unruhen in europäischen Großstädten des 21.
Jahrhunderts}\label{soziale-unruhen-in-europuxe4ischen-grouxdfstuxe4dten-des-21.-jahrhunderts}}

Die hier besprochenen Unruhen (Frankreich, Griechenland, England und
Schweden) fanden, mit der Ausnahme von Griechenland, in Vorstädten mit
"armer" Bevölkerung statt. Die Jugendarbeitslosigkeit ist hoch, ein
großer Anteil der Bevölkerung sind Menschen mit so genanntem
Migrationshintergrund oder Ausländer\_innen. Dennoch ist die Diversität
in den betroffenen Stadtteilen hoch, da sie aus Gruppen verschiedenster
Herkunftsländer bestehen (vergleiche Musterd 2005, S. 332). Die
Infrastruktur ist meistens schlecht und die Wohnkomplexe weisen Mängel
und dringenden Renovationsbedarf auf. Die räumliche Trennung der
Bürger\_innen von der restlichen Stadt wird oft durch die
städtebaulichen Konzepte der Viertel verstärkt (vergleiche Weber 2016,
S. 29). Allerdings dürfen auch Vorgänge wie die Gentrifizierung von
Stadtteilen nicht vergessen werden, da sie zur städtischen Segregation
beitragen (vergleiche Thörn et al.~2016, S. 7).

Hinzu kommt, dass in den betroffenen Stadtteilen die Polizeipräsenz hoch
ist. Grundlose Personenkontrollen und rassistisches Verhalten seitens
der Polizei sind keine Seltenheit, zum Beispiel die Kontrolle von
Menschen auf Grund ihres \enquote{ausländischen} Aussehens (vergleiche
Thörn 2013, S. 54f). So waren die Auslöser, nicht zu verwechseln mit der
Ursache, für die Unruhen allesamt Polizeieinsätze mit Todesfolge.
Unmittelbar vor jedem der vier vorgestellten Ereignisse kam mindestens
ein Mensch durch Eingriff der Polizei ums Leben. In Paris sind zwei
Jugendliche gestorben, die vor der Polizei geflüchtet sind. In Athen
wurde ein Junge von der Polizei erschossen, weil er zur falschen Zeit am
falschen Ort war. In London wurde ein Mann erschossen, weil der Verdacht
auf Waffenbesitz bestand. In Stockholm wurde ein Mann während seiner
Verhaftung erschossen. Die Aufstände brachen oft nicht unmittelbar nach
dem Auslöser aus und wenn, dann flachten sie nach kurzer Zeit wieder ab,
erst ein zweites Ereignis, auch im Zusammenhang mit der Polizei, führte
zu langanhaltenden (manchmal wochenlangen) Ausschreitungen (vergleiche
Mucchielli 2016, S. 136; Sernhede 2014, S. 82; Slater 2016, S. 124).

\hypertarget{die-bibliothek-als-angriffsziel}{%
\section{5 Die Bibliothek als
Angriffsziel}\label{die-bibliothek-als-angriffsziel}}

\hypertarget{methodik}{%
\subsection{5.1 Methodik}\label{methodik}}

Für die Erarbeitung dieses Themas wurden Quellen in schriftlicher Form
verwendet. Diese bestehen aus Fachbüchern, Zeitschriftenartikeln,
fachrelevanten Webseiten, Zeitungsberichten und dem persönlichen
E-Mail-Verkehr mit Expert\_innen. Da der Erforschung von Angriffen auf
Bibliotheken während sozialer Unruhen des 21. Jahrhunderts bisher kaum
Beachtung geschenkt wurde, ging der schriftlichen Arbeit eine intensive
Recherche voraus. Um mehr Details zu den einzelnen Bibliotheksangriffen
und deren Umstände zu erfahren, wurden Expert\_innen in verschiedenen
Ländern per E-Mail angeschrieben. Es bestand Kontakt zu
Wissenschaftler\_innen folgender Länder: England, Griechenland und
Schweden, sowie zu Bibliotheksmitarbeiter\_innen in England und
Schweden. Ferner konnte ein Fotograf in England wichtige, erste Hinweise
zu dem Bibliotheksbrand in Salford (Manchester) liefern (persönliche
Korrespondenz mit Joel Goodman vom 23.07.19).

Die im nächsten Kapitel besprochenen Vorfälle wurden unter dem Aspekt
der Umstände, des Zeitpunkts, der geographischen Lage und der
Sichtbarkeit ausgewählt. Die Umstände beziehen sich klar darauf, dass
der Angriff auf die Bibliothek während sozialer Unruhen passiert sein
muss, um in diese Arbeit Einzug zu erhalten. Für den Zeitpunkt wurde
festgelegt, dass nur die Ereignisse aus der jüngeren Vergangenheit
berücksichtigt werden. Dies ergab, dass sich die Untersuchung auf
Vorfälle ab dem Jahr 2000 konzentriert. Die geographische Lage wurde auf
Europa eingeschränkt. Durch die Beispiele ist ersichtlich, dass Angriffe
auf Bibliotheken kein Phänomen eines spezifischen Landes sind, sondern
gesamteuropäisch auftreten. Während der Recherche wurden auch Fälle von
Anschlägen auf Bibliotheken außerhalb Europas entdeckt. Die Sichtbarkeit
bezieht sich auf den praktischen Grund der Auffindbarkeit. Es konnte
nicht ermittelt werden, ob lokal und ereignisbezogen noch mehr
Bibliotheken betroffen waren, wenn darüber nicht in (online) verfügbaren
Medien berichtet wurde.

\hypertarget{die-ermittelten-ereignisse}{%
\subsection{5.2 Die ermittelten
Ereignisse}\label{die-ermittelten-ereignisse}}

\hypertarget{paris-frankreich-2005}{%
\subsubsection{5.2.1 Paris -- Frankreich --
2005}\label{paris-frankreich-2005}}

Die am besten dokumentierten Anschläge auf Bibliotheken finden sich in
Frankreich. Der Soziologe Denis Merklen hat das Phänomen der
Bibliotheksbrände über Jahre analysiert. Allein im Jahr 2005 wurden in
den Wochen während der Unruhen 34 Bibliotheken angezündet. Allerdings
ist dies nur die \enquote{sichtbarste, die symbolischste und die
spektakulärste Manifestation einer konfliktreichen und komplexen
Beziehung zwischen den Bibliotheken und ihren Stadtvierteln} (Merklen
2015, S. 537).

Viel öfter kommt es vor, dass in Bibliotheken eingebrochen wird, um
diese zu verwüsten, etwas zu stehlen oder dass die Fenster mit Steinen
beworfen werden (vergleiche ebenda). Viele Bibliotheken wurden in
Frankreich in Mediatheken (\emph{Médiathèque}) umgerüstet und umbenannt,
um einen \enquote{elitären Graben} zu vermeiden. In den Mediatheken
wurden -- abgesehen von Büchern -- Filme, Musik und das Internet
integriert, da diese Medien als \enquote{populärer} gelten (vergleiche
Merklen 2013, S. 26). Für die vorliegende Arbeit wurde eine Bibliothek
exemplarisch herausgegriffen, die \emph{Médiathèque Gulliver} in
Saint-Denis. Sie steht für viele weitere Bibliotheken in Frankreich.

Ausgelöst wurden die Unruhen vom November und Dezember 2005 durch einen
Vorfall in der Siedlung \emph{cité du Chêne pointu}, im Ort
Clichy-sous-Bois im Nordosten von Paris. Auf der Flucht vor einer der
häufigen, willkürlichen Polizeikontrollen versteckten sich drei
Jugendliche in einem Umspannwerk. Zwei erlitten dabei tödliche
Stromschläge, der Dritte wurde schwer verletzt (vergleiche Dikeç 2006,
S. 159). Nachdem die Polizei eine Tränengasgranate in den
Eingangsbereich einer Moschee warf (vergleiche Mucchielli 2016, S. 136),
gab es über mehrere Wochen Ausschreitungen zwischen jungen Erwachsenen
und der Polizei.

Die eingangs erwähnte \emph{Médiathèque Gulliver} befindet sich in dem
Ort Saint-Denis, nördlich von Paris, im Quartier Floréal. Die Bibliothek
wurde im Juni 2004 eröffnet. Sie wurde im Rahmen einer
Quartierentwicklungsmaßnahme gebaut. Teile der vorhandenen
Großwohnsiedlung, die \emph{cité de La Saussaie}, wurden abgerissen, um
Raum zu schaffen für neue, kleinere Häuser, mehr Straßen, Grünflächen
und die bereits genannte Bibliothek \emph{Médiathèque Gulliver}
(vergleiche Merklen 2013, S. 154). Nach Abschluss des Projekts konnten
viele Mieter\_innen nicht zurückkehren, da der Wohnraum weniger und die
Mieten teurer geworden waren. Die Bibliothek liegt in der Mitte des
Stadtteils. Durch die bunte Fassade hebt sich die Bibliothek deutlich
von der sonst grauen Umgebung ab und ergibt einen auffallenden Kontrast
zu den restlichen Gebäuden (vergleiche ebenda S. 43).

Im Viertel Floréal, genauso wie in den anderen Nachbarschaften, in denen
Bibliotheken angegriffen wurden, leben Menschen und Familien, für die es
immer beschwerlicher wird ihre Lebensunterhaltskosten durch Arbeit zu
decken und die somit gezwungen sind die Hilfe des Staates in Anspruch zu
nehmen. Die Bibliotheken agieren an Orten, die politisch isoliert,
sozial vernachlässigt sind und gewalttätige Strukturen aufweisen
(vergleiche Merklen 2015, S. 539).

Die Médiathèque Gulliver wurde mehrmals gezielt angegriffen. 2004 wurden
die Fenster des Lesesaals zertrümmert, Computerarbeitsplätze zerstört
und Geräte entwendet. Knapp ein Jahr später, im Oktober 2005, wurde die
mittlerweile ersetzte Ausstattung erneut gestohlen. Nach diesem
Zwischenfall wurde in der Bibliothek ein Wachschutz eingesetzt. Das
Wachpersonal löschte dann auch den Brand, ausgelöst durch einen
Molotow-Cocktail, in der Nacht vom 5. auf den 6. November 2005
(vergleiche Zit. n.~Merklen 2013, S. 45).

\hypertarget{athen-griechenland-2008}{%
\subsubsection{5.2.2 Athen -- Griechenland --
2008}\label{athen-griechenland-2008}}

Griechenland erlebte im Dezember 2008 eine Revolte in einem seit der
Absetzung des Militärregimes und dem daraus erfolgten Übertritt in die
Demokratie 1974 nicht dagewesenen Ausmaß (vergleiche Dikeç 2017, S.
156).

Am Abend des 6. Dezembers 2008 gerieten zwei Polizisten in einen Streit
mit Jugendlichen. Einer der Polizeibeamten eröffnete anschließend das
Feuer auf eine Gruppe unbeteiligter Jugendlicher und ein 15-jähriger
Schüler starb auf der Stelle (vergleiche Kanellopoulos 2012, S. 173).
Daraufhin entbrannten landesweite Ausschreitungen, die drei Wochen
anhielten.

Die spontanen Ausschreitungen wurden am Tag nach der Tat mit einer
Demonstration durch die Stadt fortgesetzt, die in Straßenschlachten mit
der Polizei, der Zerstörung von Bankfilialen und öffentlichen Gebäuden
endete. Die Ausschreitungen griffen auf Städte in ganz Griechenland
über. Viele Schüler\_innen beteiligten sich an den Protesten, da sie
sich mit dem 15-jährigen Opfer identifizierten. Außerdem wurden
Universitäten besetzt, hauptsächlich von Studierenden, linken
politischen Gruppen und Anarchist\_innen. Es gab auch beteiligte
Gruppen, die man als militant bezeichnen kann. In Athen betraf dies die
\emph{Polytechnic School, the School of Economics} und \emph{the Law
School} (vergleiche Kanellopoulos 2012, S. 174f).

Eine Besonderheit dieser sozialen Unruhen war, dass sie klassen- und
herkunftsübergreifend Menschen vereinte. Gemeinsame Sorgen wie
Arbeitslosigkeit und daraus resultierende Existenzängste brachte
Menschen verschiedener Gesellschaftsschichten zusammen und die geteilte
Erfahrung von Polizeiwillkür vereinte junge Einheimische mit
Immigrant\_innen der zweitenGeneration (vergleiche Dikeç 2017, S. 169).

Über den Brand, der die Bibliothek für Europäisches Recht zerstörte, ist
in offiziellen Quellen kaum etwas zu finden. Der erste Hinweis fand sich
auf Wikipedia in dem Artikel über die Aufstände in Griechenland
(vergleiche Wikipedia). Aufgrund dieses Hinweises versuchte ich mehr
über den Vorfall heraus zu finden, was erfolglos blieb. Da in der
Recherche zu den Unruhen in Griechenland gewisse Autor\_innen häufiger
auftauchten, wurde versucht mit diesen in Kontakt zu treten. Durch die
Korrespondenz mit einem Wissenschaftler, der sich in seiner Forschung
intensiv mit den Unruhen in Griechenland auseinandersetzt, konnte
Näheres zu der Einrichtung in Erfahrung gebracht werden.

Die niedergebrannte Bibliothek für Europäisches Recht befindet sich in
der Nähe zur Juristischen Fakultät (Law School), welche zum Zeitpunkt
der Unruhen besetzt war, im Zentrum von Athen. Sie liegt an einer der
Hauptstraßen, durch die die meisten Demonstrationen gingen. Hier kommt
es öfter zu gewalttätiger Auseinandersetzung zwischen Protestierenden
und der Polizei. Die Juristische Fakultät nutzt einige Stockwerke des
betroffenen Gebäudes, um die Bibliothek für Europäisches Recht
unterzubringen (persönliche Korrespondenz mit Anonymisiert vom
10.09.19). Es konnte nicht erschlossen werden, ob es sich um einen
gezielten Angriff auf die Bibliothek handelte oder ob das Gebäude im
allgemeinen als Ziel galt. Das ausgerechnet die Bibliothek für
Europäisches Recht zum Opfer der Flammen wurde, hat einen hohen
Symbolgehalt im Zusammenhang mit den Ursachen der Unruhen, allerdings
wären weitere Ausführungen in diese Richtung reine Spekulation. Um
Genaueres über die Umstände und Absichten erfahren zu können, müsste man
direkt mit den involvierten Personen sprechen, was außerhalb der
Möglichkeiten liegt.

\hypertarget{manchester-england-2011}{%
\subsubsection{5.2.3 Manchester -- England --
2011}\label{manchester-england-2011}}

Anfang August 2011 wurde ein Mann während eines Polizeieinsatzes
erschossen. Der 29- Jährige war auf dem Weg nach Hause, in die
\emph{Broadwater Farm Housing Estate} in Tottenham, London. Diese
Großwohnsiedung, die hauptsächlich aus Sozialwohnungen besteht, ist seit
Ausschreitungen in Mitte der 1980er Jahre stark stigmatisiert
(vergleiche Slater 2016, S. 124). Es wurde von Seiten der Polizei
vermutet, dass das spätere Opfer eine Waffe trug. Die Umstände, wie es
zum Tod des Mannes kam, waren unklar. Die Polizei gab falsche
Informationen an die Presse und versäumte es, ein aufklärendes Gespräch
mit seiner Familie zu führen.

Dass ein junger, schwarzer Mann von der Polizei getötet wurde, war der
Auslöser für die Aufstände, dass sie aber ein solches Ausmaß annahmen,
hat noch andere Gründe. Es ist wahrscheinlicher, dass man von der
Polizei angehalten und durchsucht wird, wenn man nicht-weiß ist. Die
Polizeipraktiken weisen aber noch auf ein anderes Problem hin: Die
willkürliche und drastische Machtausübung durch die Polizei betrifft
alle Menschen, hauptsächlich Jugendliche, egal welcher Hautfarbe
(vergleiche Dikeç 2017, S. 59). Es scheint so, als ob durch diese
geteilte, kollektive Erfahrung die Ausweitung der Ausschreitungen
möglich war. Eine der betroffenen Städte, nach London, war Manchester.

In Salford, einem Vorort von Manchester, fielen die Proteste
vergleichsweise kurz aus. Am Nachmittag und Abend des 9. August 2011 gab
es im Zentrum der Ortschaft, rund um ein Shoppingcenter, Ausschreitungen
und Plünderungen. Obwohl sich die wirtschaftliche Lage in Salford über
die letzten 20 Jahre verbessert hat, ist das betroffene Gebiet von hoher
Arbeitslosigkeit, Armut (die Kinderarmut liegt bei fast 75\,\%) und
Ungleichheit gezeichnet. Die Investitionen, die in dieser Gegend
getätigt wurden, kamen nicht der Bevölkerung zugute (vergleiche Morrell
et al.~2011, S. 19).

Die ersten Hinweise, dass auch bei diesen Ausschreitungen eine
Bibliothek betroffen war, fanden sich in einem Zeitungsartikel des
Guardian (vergleiche Lewis 2011, o.S.). Weitere Nachforschungen führten
mich auf die Webseite des Fotografen Joel Goodman, der die
Ausschreitungen dokumentierte (vergleiche Goodman 2013, o.S.). Auf
Anfrage konnte er mir den Namen und die Lage der Bibliothek mitteilen
(persönliche Korrespondenz mit Joel Goodman vom 23.07.19). Betroffen war
die Broadwalk Bibliothek, die zu diesem Zeitpunkt keine aktive
Bibliothek mehr war. Durch den Kontakt zu einem Bibliothekar vor Ort
konnten einige Details in Erfahrung gebracht werden. Die Bibliothek
wurde 2010 geschlossen. Das Gebäude wurde aber weiterhin von den Salford
Community Libraries genutzt. Es beherbergte den Kundenservice, die
Fernleihe und hier wurden alle neuen Bücher angeliefert, bevor sie in
die entsprechenden Bibliotheken gesendet wurden. Außerdem nutzte die
Stadt die ehemalige Bibliothek als Archiv für ihre Bücher (persönliche
Korrespondenz mit Stephen Keefe vom 09.09.19). Es wird von einem Verlust
von tausenden Büchern gesprochen (vergleiche Dadson 2012, S. 5).

Die Bibliothek befand sich in unmittelbarer Nähe zum Shoppingcenter,
welches das Hauptziel der Ausschreitungen und Plünderungen war. Es
konnte nicht festgestellt werden, ob die Bibliothek ein zufälliges oder
bewusstes Ziel war. Von Seiten der Mitarbeiter\_innen vor Ort wird die
Vermutung geäußert, dass die Bibliothek als Ziel ausgesucht wurde, weil
es sich dabei um ein öffentliches (Staats-)Gebäude handelt (persönliche
Korrespondenz mit Stephen Keefe vom 09.09.19). Die Broadwalk Library
befindet sich nun in einem neuen Gebäude, unweit des alten Standortes
und trägt den Namen Pendleton Library. Sie wurde in das sogenannte
Gateways Konzept eingegliedert. Dieses Konzept beinhaltet, dass
verschiedene Dienstleistungen des Staates an einem Ort zu finden sind
(persönliche Korrespondenz mit Stephen Keefe vom 10.09.19).

\hypertarget{stockholm-schweden-2013}{%
\subsubsection{5.2.4 Stockholm -- Schweden --
2013}\label{stockholm-schweden-2013}}

Die Unruhen in Stockholm wurden durch die Tötung eines Mannes während
eines Polizeieinsatzes ausgelöst. Am 13. Mai 2013 wollte die Polizei im
Stadtteil Husby einen 69- jährigen Mann in seiner Wohnung festnehmen, da
sein Verhalten bedrohlich erschien. Während dieses Einsatzes wurde der
Mann erschossen. Anwohner\_innen und die ortsansässige soziale
Organisation \emph{Megafonen} versammelten sich und verlangten eine
Erklärung sowie eine unabhängige Untersuchung der Ereignisse. Das erste
Statement, der Polizei, dass der Mann im Krankenhaus verstarb, stellte
sich als Falschinformation heraus. Nachbar\_innen konnten mit einem Foto
beweisen, dass der Mann tot aus seiner Wohnung gebracht wurde. Die
Anwohner\_innen sahen das Verhalten der Polizei als von offizieller
Seite geduldete Diskriminierung an, welche sich mit ihrem Alltagserleben
deckt (vergleiche Sernhede 2014, S. 81f).

Über die nächsten Tage wuchsen die Spannungen, bis, sechs Tage später,
Jugendliche anfingen Fahrzeuge anzuzünden. Die Auseinandersetzungen
breiteten sich auf andere Viertel in Stockholm und im Laufe der Woche
auf weitere Großstädte in Schweden aus (vergleiche ebenda S. 82).

\begin{quote}
\enquote{Entscheidend ist dabei, dass die Urbanisierung in ihrer
derzeitigen Form eine tiefe rassistische Komponente hat -- die
Stockholmer Innenstadt ist zu einer durchgehend gentrifizierten Enklave
für die weiße Mittel- und Oberklasse geworden, während die ärmsten
Vororte zunehmend nicht-weiß sind.} (Thörn 2013 S. 51).
\end{quote}

Husby ist einer dieser Vororte. Rassismus und Polizeigewalt sind
Dauerthemen. So war dann auch die Polizei eines der Hauptziele während
der Unruhen. Andere Ziele waren größtenteils Autos und einige Gebäude,
die angezündet wurden. Während dieser Zeit wurde auch die
Stadtteilbibliothek von Husby attackiert. Sie wurde nicht in Brand
gesetzt, allerdings wurden alle Scheiben eingeschlagen. Eine intensive
Recherche ergab nur eine Randnotiz in zwei Artikeln, erstaunlicherweise
aber mit Fotos der Bibliothek, von Reuters (vergleiche Filkz; Shanley
2013, o.S.) und dasselbe bei BBC (vergleiche Evans 2013, o.S.).

Durch eine Anfrage an Professor Ove Sernhede, der mehrere Publikationen
zum Thema der sozialen Unruhen in Schweden veröffentlicht hat, konnte
ein Kontakt zu einem anderen Wissenschaftler, Dr.~René León Rosales,
hergestellt werden. Dieser ist in Stockholm ansässig, kennt den
betroffenen Stadtteil Husby gut und hat sich auch mit den Ereignissen
und im speziellen der Rolle von sozialen Bewegungen und
Jugendorganisationen auseinandergesetzt (vergleiche Ålund; Rosales 2017,
S. 123). Mit seiner Hilfe konnte wenigstens in Erfahrung gebracht
werden, um welche Bibliothek es sich handelt und wo sie sich genau
befindet (persönliche Korrespondenz mit René León Rosales vom 01.08.19).
Die Bibliothek ist ein Teil des Bürgerhauses in Husby und wurde laut
offizieller Webseite im März 2013, nach der Renovierung der
Räumlichkeiten, neu eröffnet (vergleiche Husby bibliotek 2019, o.S.). Ob
die Neueröffnung in Zusammenhang mit der Attacke zwei Monate später
stand, konnte nicht in Erfahrung gebracht werden. Es wurde auch nicht
klar, ob die Attacke dem Bürgerhaus im Allgemeinen galt, oder ob die
Bibliothek ein bewusstes Ziel war.

\hypertarget{suxfcdafrika}{%
\section{6 Südafrika}\label{suxfcdafrika}}

Erwähnenswert an dieser Stelle sind die (Brand-)Anschläge während
sozialer Unruhen auf Bibliotheken in Südafrika. Parallelen zu den
Angriffen in Europa sind nicht von der Hand zu weisen. Von 2009 bis 2013
wurde von einem unabhängigen Politik-Blog 15 Brandanschläge auf
Bibliotheken während sozialer Unruhen gezählt. Es wird darauf
hingewiesen, dass dies nur die Vorfälle sind, die in den Medien
auffindbar waren und es handelt sich nur um die Brandanschläge. Andere
Arten von Angriffen, wie Einbruch und Vandalismus, wurden nicht
mitgezählt (vergleiche van Onselen 2013, o.S.). Auch hier konstatiert
der Autor, wie Denis Merklen in seiner Untersuchung, das auffällige
Schweigen von Seiten der Presse und der Regierung (vergleiche ebenda).

Der Autor führt weiter aus, dass die vielen Brandanschläge auf
Bibliotheken auf die (fehlgeleitete) Politik zurückzuführen seien. Die
politischen Anführer würden Unwissenheit fördern und haben kein
Interesse daran, Orte der Ideen und des Wissens zu unterstützen und zu
schützen. Systematisch wird die Relevanz von Bibliotheken und
Wissenseinrichtungen untergraben (vergleiche ebenda). Die Aussagen von
Protestierenden unterscheiden sich auch nicht von denen in Frankreich.
Es wird davon gesprochen, dass man von der Regierung gesehen werden
möchte und deswegen öffentliche Gebäude anzündet (vergleiche ebenda).

In dem Artikel \emph{Burning Libraries for the People: Questions and
Challenges for the Library Profession in South Africa} spricht der Autor
Peter Lor von \enquote{Post-Apartheid Library Burnings} (Lor 2013, S.
360). Viele der Proteste werden als \enquote{service delivery protests}
(ebenda S. 361) bezeichnet. Lor stellt eine Zusammenfassung der
Untersuchungen vor, die zu den Unruhen gemacht wurden. Aus dieser geht
hervor, dass die beteiligten Wissenschaftler\_innen die gleichen
Hintergründe für die Proteste identifiziert haben wie ihre Kolleg\_innen
in Europa: grundlegende Ursachen wie Arbeitslosigkeit und Armut,
schlechte örtliche Regierung, folglich unzureichender Zugang zu
Leistungen des Staates und die daraus resultierenden politischen
Spannungen (vergleiche ebenda S. 364).

Es wird darauf hingewiesen, dass die Bibliothek als Symbol der
Unterdrückung wahrgenommen wird und deshalb die Wut der Protestierenden
auf sich zieht (vergleiche ebenda S. 360). Der Artikel untersucht die
Auswirkungen der Brände auf das Bibliothekswesen anhand einer
Inhaltsanalyse von Aussagen und Kommentaren, die auf der Webseite von
LIASA (Library and Information Association of South Africa) gepostet
wurden. Es wird hervorgehoben, dass von Verbandseite weder eine
zeitnahe, professionelle Stellungnahme zu den Vorfällen abgegeben wurde,
noch weitere Untersuchungen angestrebt wurden (vergleiche ebenda S.
365f).

Auch nach 2013 sind Anschläge auf Bibliotheken in Südafrika ein Mittel
zur Äußerung von Ungerechtigkeit und Ungleichheit. Während der
Studierendenproteste von 2016, gegen die angekündigte Erhöhung von
Studiengebühren, wurden weitere Bibliotheken Opfer von Brandstiftung.
Die Brände sollen auch den Unmut ausdrücken über die immer noch
bestehende Ungleichheit zwischen schwarzen und weißen Studierenden in
Südafrika (vergleiche Powell 2016, o.S.).

\hypertarget{diskussion}{%
\section{7 Diskussion}\label{diskussion}}

Werden die Attacken auf die Bibliotheken der vier vorgestellten
Ereignisse betrachtet, können nur zwei sichere Parallelen gezogen
werden. Zum einen der Umstand, dass sie während sozialer Aufstände
angegriffen wurden. Das war der Hauptgrund, warum sie in diese Arbeit
Einzug erhalten haben. Die zweite Parallele ist das weitgehende
Schweigen über die Bibliotheksangriffe in der Öffentlichkeit und der
Fachwelt.

Denis Merklen hat dieses Phänomen in Frankreich untersucht. Er hat das
Schweigen in verschiedenen Gruppen ausgemacht. Die erste Gruppe, die
nach einem Angriff auf eine Bibliothek schweigt, sind die
Verursacher\_innen selbst. Es gibt keine Stellungnahme oder anderen
Kommentar, die den Sinn des Geschehenen erklären würde. Darauf folgt das
Schweigen der Politik und der Presse. Ein öffentlicher Diskurs über
Angriffe auf Bibliotheken findet nicht statt. Merklen merkt auch an,
dass dieser Thematik in den Sozialwissenschaften zu wenig Aufmerksamkeit
geschenkt wird (vergleiche Merklen 2015, S. 537). Wie schon in dieser
Arbeit erwähnt wurde, findet dieses Thema auch innerhalb der
internationalen Bibliothekswelt kaum Resonanz. Es scheint, als ob das
Unvermögen, auf eine Tat wie diese zu reagieren, auch die Diskussion
darüber im Keim erstickt. Es sollte aber ein Diskurs darüber
stattfinden, um Erklärungen dafür zu finden, um diesen Vorfällen
begegnen und im besten Fall vorbeugend agieren zu können.

Die Untersuchung hat gezeigt, dass Unterschiede zahlreicher sind als
Parallelen. Über Merkmale, wie zum Beispiel die Fassade oder das Gebäude
an sich, konnten keine Gemeinsamkeiten festgestellt werden. Dies kann im
Einzelfall ein Grund sein, warum eine Bibliothek als Ziel ausgesucht
wird, es scheint aber keine allgemeingültige Motivation zu sein.
Abgesehen von den untersuchten Brandanschlägen in Frankreich konnten
auch keine weiteren Aussagen oder Hinweise darauf gefunden werden, dass
die Attacken auf Bibliotheken in ihrer Funktion als Wissensinstitution
galten. Dies kann allerdings auch den Kapazitäten und dem Umfang dieser
Arbeit geschuldet sein. Auf diese Problematik wird im Fazit noch näher
eingegangen. Auch dass Bibliotheken gezielt als einziges Gebäude
ausgesucht wurden, wie es Merklen in Frankreich feststellte, kann bei
den vorliegenden Beispielen, außer bei der Bibliothek in Frankreich,
nicht bestätigt werden, da die entsprechenden Aussagen dafür fehlen.

Öffentliche Gebäude, zu denen Bibliotheken zählen, sind oft Ziele der
Zerstörung während Ausschreitungen. Die sozialen Unruhen des 21.
Jahrhunderts richten sich gegen einen Staat, der darin versagt hat,
Teile der Gesellschaft zu unterstützen und gleichberechtigt zu
behandeln. Es scheint der Fall zu sein, dass die Bibliothek in einer
Konfliktsituation wie dieser zwischen Bürger\_innen und dem Staat steht
und somit als staatliche und \enquote{feindliche} Einrichtung angesehen
wird. Die Bibliothek als eine öffentliche Einrichtung für die
Nachbarschaft, gleichzeitig aber auch Teil des Staates, hat eine
herausfordernde Stellung in diesen Konflikten (vergleiche Merklen 2015,
S. 537). Demnach würde allerdings der repräsentative Charakter der
Bibliothek, als Gebäude und Institution, wieder in den Mittelpunkt
rücken.

\begin{quote}
\enquote{Die Macht der Nationalstaaten architektonisch zu
materialisieren, erscheint nun heute und besonders hierzulande anmaßend.
Ein Bibliotheksneubau lässt genau an dieser Stelle einen Spielraum zum
Ausdruck der immateriellen Stärke -- in Form von Wissen und Kultur.}
(Goldberg 2015, S. 11).
\end{quote}

Dies könnte dazu beitragen, dass die Bibliothek mit ihren Angeboten,
trotz ihres öffentlichen Charakters, in den Augen der Protestierenden
nicht zugänglich für sie ist. Olaf Eigenbrodt beschäftigt sich in einem
Artikel mit der Frage, wie eine Stadt in der Wissensgesellschaft
aussieht und welche Bibliotheken sie braucht (vergleiche Eigenbrodt
2007, o.S.). Er äußert sich zu öffentlichen Räumen, zu denen er klar
auch die Bibliothek zählt, folgendermaßen:

\begin{quote}
\enquote{Solche niedrigschwelligen Angebote wirken zwar einerseits der
Segregation entgegen und verschaffen Menschen Zugang zu Informationen
und Kommunikationsmöglichkeiten, den sie sonst nicht hätten, sind aber
auch immer die Orte, an denen gesellschaftliche Friktionen und
Spannungen offensichtlich werden. Zudem muss immer gefragt werden, ob
solche Räume nicht im Wesentlichen dem Wunschdenken politisch
sozialliberal ausgerichteter Angehöriger der Mittelschicht entspringen.
Die Schaffung so genannter soziokultureller Zentren z. B. ist oft genug
ein wesentlicher Antrieb der Gentrification, die dann in den
benachbarten Wohngebieten stattfindet. Die Angebote dieser Zentren sind
häufig nur für diejenigen interessant, die sie gestalten.} (vergleiche
ebenda).
\end{quote}

Die Spannungen, die sich in öffentlichen Räumen, speziell auch
Bibliotheken, offenbaren, spricht auch Denis Merklen an. Die
Umbenennungen von Bibliotheken zu Mediatheken in Frankreich, um einen
\enquote{elitären Graben} zu vermeiden (vergleiche Merklen 2013, S. 26),
deuten darauf hin, dass Bibliotheken von gewissen Gesellschaftsgruppen
als nicht zugänglich oder nicht für sie bestimmte Einrichtungen
angesehen werden.

\begin{quote}
\enquote{Durch die Brandstiftung Schrecken hervorrufen, versuchen,
Aufmerksamkeit darauf zu lenken, dass der Lebensraum dieser Viertel auch
ein Teil des Lebensraums aller ist, versuchen, etwas von dem
\enquote{Heiligen} zu treffen, dass die Bibliothek repräsentiert, um
verständlich zu machen, dass die Lage ernst ist, darin liegt vielleicht
ein Teil des Sinns dieser Akte.} (Merklen 2015, S. 538).
\end{quote}

Ein primärer Grund für die Attacken scheint die Sichtbarkeit zu sein.
Menschen, die sich übergangen, ignoriert und ungerecht behandelt fühlen,
versuchen auf ihre Lage aufmerksam zu machen. Oft sind sozialen Unruhen
kollektive Aktionen und Anstrengungen einzelner Gruppen vorausgegangen.
Diese wurden von offizieller Seite nicht beachtet oder es wurde nicht
angemessen darauf reagiert (vergleiche Sernhede 2014, S. 84f).

Ob (fehlende) Angebote der Bibliothek, nicht oder nur wenig vorhandene
Diversität des Bibliothekspersonals und die daraus resultierende
Schwierigkeit, sich mit der Institution zu identifizieren, Gründe für
einen Bibliotheksangriff sind, konnte in der nötigen Tiefe in dieser
Arbeit nicht untersucht werden. Deshalb werden hier auch keine weiteren
Vermutungen angestellt. Dennoch sind es die Themen wert, auch im
Zusammenhang mit der vorliegenden Thematik weiter untersucht zu werden.

Ebenfalls relevant wäre die Untersuchung der Rolle und Wirkung des
Wachpersonals, das häufig in Bibliotheken eingesetzt wird. Eventuell
würde die Untersuchung dieser Frage aufschlussreiche Ansätze bezüglich
der Offenheit der Bibliothek aus der Perspektive der potentiellen
Nutzer\_innen ergeben und gegebenenfalls eine weitere Lücke in den
Diversitätsbemühungen der Bibliotheken schließen.

Zusammenfassend kann gesagt werden, dass die Analyse der vier hier
untersuchten Angriffe auf Bibliotheken während sozialer Unruhen ergeben
hat, dass es doch wenige nachweisbare Parallelen zwischen den einzelnen
Fällen gibt. Es ist deutlich geworden, dass die Bibliothek als Symbol
eine Angriffsfläche bietet und dass die Bibliothek in Ausnahmezuständen
nicht als Verbündete, sondern eher als eine Institution der Gegenseite
wahrgenommen wird. Der Umstand, dass Ausschreitungen und soziale
Bewegungen zugenommen haben, sollte zum Anlass genommen werden, die
Rolle der Bibliothek in diesem Szenario eingehend zu untersuchen.

\hypertarget{fazit-und-ausblick}{%
\section{8 Fazit und Ausblick}\label{fazit-und-ausblick}}

Der Umstand, dass die Erforschung des vorliegenden Themas quasi noch
nicht erfolgt ist, ermöglichte Ereignisse in einen Zusammenhang zu
stellen, der so noch nicht vorhanden war. Es bedeutete aber auch, dass
kaum eine Möglichkeit darauf bestand, auf geeignete, bereits
existierende Forschungsliteratur zurückzugreifen.

Im Laufe des Arbeitsprozesses zeigte sich, dass nicht nur das Fehlen von
passender Fachliteratur eine Schwierigkeit darstellt, sondern auch die
limitierten Ressourcen. Um mehr und genauere Informationen über die
einzelnen Vorfälle zu erfahren, müsste die Möglichkeit bestehen, mit den
Akteuren und den betroffenen Bibliotheksmitarbeiter\_innen unmittelbar
in Kontakt zu treten sowie die Geschehnisse vor Ort im städtischen und
gesellschaftlichen Zusammenhang erschließen zu können und dies am besten
so zeitnah wie möglich nach einem solchen Ereignis. Beweggründe aus der
Entfernung und ohne persönlichen Kontakt zu involvierten Personen zu
ermitteln, ist ein schwieriges, wenn nicht sogar unmögliches Unterfangen
und liefert keine befriedigenden Erkenntnisse.

Der gesellschaftliche Kontext, in dem sich die Forschungsfrage bewegt,
konnte weitestgehend erläutert werden. Die Ursprünge der sozialen
Unruhen können, grob gesagt, auf politisches Versagen in Fragen der
Gleichbehandlung und ihren Begleiterscheinungen wie Segregation,
Rassismus und Stigmatisierung zurückgeführt werden. Die Tatsache, dass
Bibliotheken Opfer von Anschlägen während sozialer Unruhen in Europa
werden scheint an deren Symbolcharakter sowie ihrem Status als
staatliche Institution zu liegen. Hier wäre es in einer weiterführenden
Untersuchung interessant zu erforschen, warum dies zum Beispiel in den
USA, so weit bekannt, nicht passiert.

Es scheint weitestgehend bewiesen zu sein, dass die Bibliothek selten,
aber nicht nie, in ihrer Funktion als Wissensinstitution angegriffen
wird. Mit Gewissheit kann dies allerdings erst gesagt werden, wenn die
Vorfälle tiefgreifend und umfassend untersucht werden, wie es Denis
Merklen in Frankreich vorgemacht hat.

Aus bibliotheks- und informationswissenschaftlicher Perspektive wäre es
wünschenswert, wenn diese Thematik Einzug in die aktuelle Debatte sowie
Forschung erhält. Könnte es doch hilfreich sein in der Beantwortung von
Fragen nach der Dringlichkeit von Diversität in Bibliotheken sowie um
die Erarbeitung von Inklusionskonzepten vorantreiben. Die Zusammenarbeit
mit Bürger\_innen könnte neue Erkenntnisse darüber liefern, was
Bibliotheken unternehmen können, um Gesellschaftsgruppen anzusprechen,
welche sich bis jetzt von der Welt der Bibliotheken ausgeschlossen
fühlen. Es liegt auf der Hand, dass die Bibliotheken nicht die Probleme
der sozialen Ungerechtigkeit und deren Ausdruck in Form von Protesten
lösen können. Trotzdem könnte die Beschäftigung mit dem Thema helfen,
die Bibliothek als Institution aus einer anderen Perspektive zu
betrachten und damit potenzielle Hürden zu identifizieren und zu
bearbeiten.

Ein weiterer Ansatz von zukünftiger Forschung in dieser Thematik wäre
der der Internationalität. Würden Wissenschaftler\_innen aus den
betroffenen Ländern/ Städten gemeinsam die Vorfälle untersuchen, wären
Vergleiche und die Erarbeitung von Studien erfolgsversprechender, als
wenn dies dezentralisiert Ortsfremde vornehmen würden. Hinzu kommt die
Interdisziplinarität. Gäbe es die Möglichkeit, dieses Thema zusammen mit
Sozialwissenschaftler\_innen, Stadtsoziolog\_innen, Politik- und
Kulturwissenschaftler\_innen zu erforschen, ergäbe dies ein umfassendes
Gesamtbild, welches die aktuellen gesellschaftlichen Herausforderungen
widerspiegeln würde, die wiederum die Bibliotheksarbeit beeinflussen
könnten.

\hypertarget{literaturverzeichnis}{%
\section{Literaturverzeichnis}\label{literaturverzeichnis}}

Ålund, Aleksandra; Rosales, René Leon: Becoming an Activist Citizen:
Individual Experiences and Learning Processes within the Swedish
Suburban Movement. In: Journal of Education and Culture Studies,
Vol.1(2) 2017, S. 123--140. Online verfügbar unter:
\url{https://doi.org/10.22158/jecs.v1n2p123}.

Dadson, Emma: Emergency Planning and Response for Libraries, Archives
and Museums. London 2012.

Dikeç, Mustafa: Badlands of the Republic. Space Politics and Urban
Policy. Oxford 2007.

Dikeç, Mustafa: Urban Rage. The revolt of the excluded. Yale 2017.

Eigenbrodt, Olaf: Gibt es die Stadt noch -- und welche Bibliothek
braucht sie? In: LIBREAS. Library Ideas, Vol. 8/9 2007. Online verfügbar
unter: \url{https://libreas.eu/ausgabe8/001eig.htm}.

Evans, Stephen: Stockholm riots throw spotlight on Swedish inequality.
In: BBC Online. 2013. Online verfügbar unter:
\url{https://www.bbc.com/news/world-europe-22650267}.

Filks, Ilze; Shanley, Mia: Stretched by riots, Swedish police call
reinforcements. In: Reuters Online. 2013. Online verfügbar unter:
\url{https://www.reuters.com/article/us-swedenriots/stretched-by-riots-swedish-police-call-reinforcements-idUSBRE94N0LN20130524}.

Goodman, Joel: 2 years ago today: Camden, Salford and Manchester Riots.
2013. Online verfügbar unter:
\href{https://joelgoodman.net/2013/08/09/camden_salford_manchester_riots_london_august_2011_photography/}{https://joelgoodman.net/2013/08/09/camden\_salford\_manchester\_riots\_london\_ august\_2011\_photography/}.

Goldberg, Polina: Von Bibliotheken und Leselandschaften. In:
BaunetzWoche, Vol. 401 2015, S. 8--22. Online verfügbar unter:
\url{https://www.baunetz.de/baunetzwoche/baunetzwoche_ausgabe_4265793.html}.

Guardian: UK riots front pages -- in pictures. London 2011. Online
verfügbar unter: \newline
\url{https://www.theguardian.com/media/gallery/2011/aug/09/uk-riots-front-pages-in-pictures}

Husby bibliotek. 2019. Online verfügbar unter:
\url{https://biblioteket.stockholm.se/bibliotek/husby-bibliotek}.

Kanellopoulos, Kostas: The Accidental Eruption of an Anarchist Protest.
In: Seferiades, Seraphim; Johnston, Hank (Hrsg.): Violent Protest,
Contentious Politics and the Neoliberal State. London 2012, S. 171--181.

Lewis, Paul: A fire lit in Tottenham that burned Manchester: the
rioters' story. In: The Guardian. London 2011. Online verfügbar unter:
\url{https://www.theguardian.com/uk/2011/dec/05/tottenham-manchester-rioters-story-england}.

Lor, Peter: Burning Libraries for the People: Questions and Challenges
for the Library Profession in South Africa. In: Libri. International
Journal of Libraries and Information Services, Vol.63(4) 2013, S.
359--372. Online verfügbar unter:
\url{https://doi.org/10.1515/libri-2013-0028}.

Mayer, Margit et al.~(Hrsg.): Urban Uprising. Challenging Neoliberal
Urbanism in Europe. London 2016.

Merklen, Denis: Pourquoi brûle-t-on des bibliothèques?. Villeurbanne
2013.

Merklen, Denis: Feuer und Schweigen -- Wenn Bibliotheken brennen -- Ein
Essay über die Bibliotheksbrände in Frankreich. In: BuB - Forum
Bibliothek und Information, Vol. 67 (8/9) 2015, S. 536-539. Online
verfügbar unter: \url{https://b-u-b.de/wp-content/uploads/2015-08.pdf}.

Moran, Matthew; Waddington, David: Riots. An International Comparison.
London 2016.

Morrell, Gareth et al.: The August riots in England. Understanding the
involvement of young people. London 2011. Online verfügbar unter:
\href{https://assets.publishing.service.gov.uk/government/uploads/system/uploads/attachment_data/file/60531/The_20August_20Riots_20in_20England_20_pdf__201mb_.pdf}{https://assets.publishing.service.gov.uk/govern-ment/uploads/system/uploads/attachment\_data/file/60531/The\_20August\_20Riots\_20in\_20 England\_20\_pdf\_\_201mb\_.pdf}.

Mucchielli, Laurent: Unruhen in der aktuellen französischen
Gesellschaft. Eine elementare Form politischer Aushandlungsprozesse. In:
Weber, Florian; Kühne, Olaf (Hrsg.): Fraktale Metropolen.
Stadtentwicklung zwischen Devianz, Polarisierung und Hybridisierung.
Wiesbaden 2016, S. 131--143.

Musterd, Sako: Social and ethnic segregation in Europe. Level, Causes,
and Effects. In: Journal of Urban Affairs. Vol. 27, Nr. 3 2005, S.
331--348. Online verfügbar unter:
\url{https://doi.org/10.1111/j.0735-2166.2005.00239.x}.

Powell, Anita: South Africa Protesters Use Fire to Illuminate
Grievances. 2016. Online verfügbar unter:
\url{https://www.voanews.com/africa/south-africa-protesters-use-fire-illuminate-grievances}.

Sernhede, Ove: Youth rebellion and social mobilisation in Sweden. In:
Soundings. A journal of politics and culture. Vol. 56 2014, S. 81--91.
Online verfügbar unter: \url{https://www.muse.jhu.edu/article/544263}.

Slater, Tom: The Neoliberal State and the 2011 English Riots: A Class
Analysis. In: Mayer, Margit et al.~(Hrsg.): Urban Uprising. Challenging
Neoliberal Urbanism in Europe. London 2016, S. 121--148.

Thörn, Catharina: Der Aufstand in Stockholm und der Mythos der
schwedischen Sozialdemokratie. In: Sozial.Geschichte Online. Vol. 11
2013, S. 48--58. Online verfügbar unter:
\url{https://nbn-resolving.org/urn:nbn:de:hbz:464-20131010-075949-9}.

Thörn, Håkan et al.: Re-Thinking Urban Social Movements, ‚Riots' and
Uprisings: An Introduction. In: Mayer, Margit et al.~(Hrsg.): Urban
Uprising. Challenging Neoliberal Urbanism in Europe. London 2016, S.
3--55.

Van Onselen, Gareth: Burning Books. The African way. 2013. Online
verfügbar unter: \newline
\url{https://inside-politics.org/2013/02/14/burning-books-the-african-way/}.

Weber, Florian; Kühne, Olaf (Hrsg.): Fraktale Metropolen.
Stadtentwicklung zwischen Devianz, Polarisierung und Hybridisierung.
Wiesbaden 2016.

Weber, Florian: Urbane Mosaike, Fragmentierungen, stadtpolitische
Interventionen. Die banlieues und die Stadtpolitik politique de la
ville. In: Weber, Florian; Kühne, Olaf (Hrsg.): Fraktale Metropolen.
Stadtentwicklung zwischen Devianz, Polarisierung und Hybridisierung.
Wiesbaden 2016, S. 21--55.

Wikipedia: 2008 Greek riots. Online verfügbar unter:
\url{https://en.wikipedia.org/wiki/2008_Greek_riots\#cite_ref-62}.

%autor
\begin{center}\rule{0.5\linewidth}{0.5pt}\end{center}

\textbf{Sara Juen}, *1983, Sortimentsbuchhändlerin, von Okt. 2016 --
Sept.~2019 Bachelorstudium in Bibliotheks- und Informationswissenschaft
an der Humboldt Universität zu Berlin, seit Oktober 2019 im Master für
Informationswissenschaft (HU). ORCID:
\url{https://orcid.org/0000-0003-0725-8592}

\end{document}
