\documentclass[a4paper,
fontsize=11pt,
%headings=small,
oneside,
numbers=noperiodatend,
parskip=half-,
bibliography=totoc,
final
]{scrartcl}


\usepackage{synttree}
\usepackage{graphicx}
\setkeys{Gin}{width=.4\textwidth} %default pics size

\graphicspath{{./plots/}}


\usepackage[T1]{fontenc}
\usepackage[utf8]{inputenc}
%\usepackage{amsmath}
\usepackage[ukrainian,ngerman]{babel}
\usepackage[german]{csquotes}
\usepackage [hyphens]{url}
\usepackage{booktabs} 
\usepackage[left=2.4cm,right=2.4cm,top=2.3cm,bottom=2cm,includeheadfoot]{geometry}
\usepackage{eurosym}
\usepackage{multirow}
\usepackage[ngerman]{varioref}
\setcapindent{1em}
\renewcommand{\labelitemi}{--}
\usepackage{paralist}
\usepackage{pdfpages}
\usepackage{lscape}
\usepackage{float}
\usepackage{acronym}
\usepackage{eurosym}
\usepackage{longtable,lscape}
\usepackage{mathpazo}
\usepackage[normalem]{ulem} %emphasize weiterhin kursiv
\usepackage[flushmargin,ragged]{footmisc} % left align footnote
\usepackage{ccicons} 
\setcapindent{0pt} % no indentation in captions

%%%% fancy LIBREAS URL color 
\usepackage{xcolor}
\definecolor{libreas}{RGB}{112,0,0}

\usepackage{listings}

\urlstyle{same}  % don't use monospace font for urls

\usepackage[fleqn]{amsmath}

%adjust fontsize for part

\usepackage{sectsty}
\partfont{\large}

%Das BibTeX-Zeichen mit \BibTeX setzen:
\def\symbol#1{\char #1\relax}
\def\bsl{{\tt\symbol{'134}}}
\def\BibTeX{{\rm B\kern-.05em{\sc i\kern-.025em b}\kern-.08em
    T\kern-.1667em\lower.7ex\hbox{E}\kern-.125emX}}

\usepackage{fancyhdr}
\fancyhf{}
\pagestyle{fancyplain}
\fancyhead[R]{\thepage}

% make sure bookmarks are created eventough sections are not numbered!
% uncommend if sections are numbered (bookmarks created by default)
\makeatletter
\renewcommand\@seccntformat[1]{}
\makeatother

% typo setup
\clubpenalty = 10000
\widowpenalty = 10000
\displaywidowpenalty = 10000

\usepackage{hyperxmp}
\usepackage[colorlinks, linkcolor=black,citecolor=black, urlcolor=libreas,
breaklinks= true,bookmarks=true,bookmarksopen=true]{hyperref}
\usepackage{breakurl}

%meta
\expandafter\def\expandafter\UrlBreaks\expandafter{\UrlBreaks%  save the current one
  \do\a\do\b\do\c\do\d\do\e\do\f\do\g\do\h\do\i\do\j%
  \do\k\do\l\do\m\do\n\do\o\do\p\do\q\do\r\do\s\do\t%
  \do\u\do\v\do\w\do\x\do\y\do\z\do\A\do\B\do\C\do\D%
  \do\E\do\F\do\G\do\H\do\I\do\J\do\K\do\L\do\M\do\N%
  \do\O\do\P\do\Q\do\R\do\S\do\T\do\U\do\V\do\W\do\X%
  \do\Y\do\Z}
%meta

\fancyhead[L]{Redaktion LIBREAS\\ %author
LIBREAS. Library Ideas, 41 (2022). % journal, issue, volume.
%\href{https://doi.org/10.18452/xxx}{\color{black}https://doi.org/10.18452/xxx}
{}} % doi 
\fancyhead[R]{\thepage} %page number
\fancyfoot[L] {\ccLogo \ccAttribution\ \href{https://creativecommons.org/licenses/by/4.0/}{\color{black}Creative Commons BY 4.0}}  %licence
\fancyfoot[R] {ISSN: 1860-7950}

\title{\LARGE{Das liest die LIBREAS, Nummer \#10 (Frühling / Sommer 2022)}}% title
\author{Redaktion LIBREAS} % author

\setcounter{page}{1}

\hypersetup{%
      pdftitle={Das liest die LIBREAS, Nummer \#10 (Frühling / Sommer 2022)},
      pdfauthor={Redaktion LIBREAS},
      pdfcopyright={CC BY 4.0 International},
      pdfsubject={LIBREAS. Library Ideas, 41 (2022)},
      pdfkeywords={Literaturübersicht, Bibliothekswissenschaft, Informationswissenschaft, Bibliothekswesen, Rezension, literature overview, library science, information science, library sector, review},
      pdflicenseurl={https://creativecommons.org/licenses/by/4.0/},
      pdfcontacturl={http://libreas.eu},
      baseurl={},
      pdflang={de},
      pdfmetalang={de}
     }



\date{}
\begin{document}

\maketitle
\thispagestyle{fancyplain} 

%abstracts

%body
Beiträge von Eva Bunge (eb), Sara Juen (sj), Ben Kaden (bk), Yannick
Paulsen (yp), Karsten Schuldt (ks)

\hypertarget{zur-kolumne}{%
\section{1. Zur Kolumne}\label{zur-kolumne}}

Ziel dieser Kolumne ist es, eine Übersicht über die in der letzten Zeit
erschienene bibliothekarische, informations- und
bibliothekswissenschaftliche sowie für diesen Bereich interessante
Literatur zu geben. Enthalten sind Beiträge, die der LIBREAS-Redaktion
oder anderen Beitragenden als relevant erschienen.

Themenvielfalt sowie ein Nebeneinander von wissenschaftlichen und
nicht-wissenschaftlichen Ansätzen wird angestrebt und auch in der Form
sollen traditionelle Publikationen ebenso erwähnt werden wie
Blogbeiträge oder Videos beziehungsweise TV-Beiträge.

Gerne gesehen sind Hinweise auf erschienene Literatur oder Beiträge in
anderen Formaten. Diese bitte an die Redaktion richten. (Siehe
\href{http://libreas.eu/about/}{Impressum}, Mailkontakt für diese
Kolumne ist
\href{mailto:zeitschriftenschau@libreas.eu}{\nolinkurl{zeitschriftenschau@libreas.eu}}.)
Die Koordination der Kolumne liegt bei Karsten Schuldt, verantwortlich
für die Inhalte sind die jeweiligen Beitragenden. Die Kolumne
unterstützt den Vereinszweck des LIBREAS-Vereins zur Förderung der
bibliotheks- und informationswissenschaftlichen Kommunikation.

LIBREAS liest gern und viel Open-Access-Veröffentlichungen. Wenn sich
Beiträge dennoch hinter eine Bezahlschranke verbergen, werden diese
durch \enquote{{[}Paywall{]}} gekennzeichnet. Zwar macht das Plugin
\href{http://unpaywall.org/}{Unpaywall} das Finden von legalen
Open-Access-Versionen sehr viel einfacher. Als Service an der
Leserschaft verlinken wir OA-Versionen, die wir vorab finden konnten,
jedoch auch direkt. Für alle Beiträge, die dann immer noch nicht frei
zugänglich sind, empfiehlt die Redaktion Werkzeuge wie den
\href{https://openaccessbutton.org/}{Open Access Button} oder
\href{https://core.ac.uk/services/discovery/}{CORE} zu nutzen oder auf
Twitter mit
\href{https://twitter.com/hashtag/icanhazpdf?src=hash}{\#icanhazpdf} um
Hilfe bei der legalen Dokumentenbeschaffung zu bitten.

Die bibliographischen Daten der besprochenen Beiträge aller Ausgaben
dieser Kolumne finden sich in der öffentlich zugänglichen Zotero-Gruppe:
\url{https://www.zotero.org/groups/4620604/libreas_dldl/library}.

\hypertarget{artikel-und-zeitschriftenausgaben}{%
\section{2. Artikel und
Zeitschriftenausgaben}\label{artikel-und-zeitschriftenausgaben}}

\hypertarget{vermischte-themen}{%
\subsection{2.1 Vermischte Themen}\label{vermischte-themen}}

Vosberg, Dana ; Lütjen, Andreas (2021). \emph{Bestandscontrolling bei
elektronischen Ressourcen: Entscheidungshilfen für die Lizenzierung}.
In: o-bib 8 (2021) 1, \url{https://doi.org/10.5282/o-bib/5672}

Es werden die Ergebnisse einer Umfrage unter deutschen
Hochschulbibliotheken dahingehend ausgewertet, wie diese Entscheidungen
darüber treffen, welche elektronischen Ressourcen (wieder-)lizenziert
werden. Dabei zeigt sich ein uneinheitliches Vorgehen mit einigen
Tendenzen. Grössere Einrichtungen benutzen zum Beispiel tendenziell mehr
Daten für diese Entscheidungen als kleinere. Es werden vor allem
\enquote{Kosten pro Download} als wesentlicher Wert für eine
Entscheidung genannt, aber die Grenzwerte dafür werden -- wie auch alle
anderen Grenzwerte -- pro Einrichtung jeweils selber festgelegt. Es
zeigt sich auch, dass Bibliotheken offenbar vor der Kündigung von
Lizenzen zurückschrecken und zuvor auf anderen Wegen versuchen, Kosten
zu drücken. Es ist ein interessanter Einblick in die Praxis von
Bibliotheken.

An einigen Stellen scheint die Interpretation der Daten der Umfrage
übertrieben. Beispielsweise schliessen die Autor*innen, dass es bei den
Bibliotheken einen Wunsch zur Kommunikation über das Thema
Bestandscontrolling gäbe, aber es ist nicht klar, ob sich das wirklich
aus den Daten ergibt oder aus der Hoffnung der Autor*innen selber. (ks)

\begin{center}\rule{0.5\linewidth}{0.5pt}\end{center}

Jethro, Duane (2021). \emph{ASH: Memorializing the 2021 University of
Cape Town Library Fire}. In: Material Religion: The Journal of Objects,
Art und Belief 17 (2021) 5: 671--677,
\url{https://doi.org/10.1080/17432200.2021.1991117}

Dieser Essay interpretiert die Nachwirkungen des Brandes, bei dem am 18.
April 2021 grosse Teile der Bibliothek der University of Cape Town Feuer
fingen, als gemeinsame Arbeit am Erinnern -- sowohl Erinnern der
Bibliothek und der Möglichkeiten, die diese mit ihren Sammlungen bot,
die jetzt nicht mehr vorhanden sind, als auch der spezifischen Zeit. Der
Brand fiel in eine Zeit heftiger Debatten um das südafrikanische
Universitätssystem -- dem Zugang zu ihm, dessen rassistischen Wurzeln,
der Verantwortung -- und vermittelte den Eindruck, real zu machen, was
bislang vor allem als Rhetorik verbreitet wurde: Der Universität «on
fire». (ks)

\begin{center}\rule{0.5\linewidth}{0.5pt}\end{center}

Gofman, Ari ; Leif, Sam A. ; Gundermann, Hannah ; Exner, Nina (2021).
\emph{Do I Have To Be An \enquote{Other} To Be Myself? Exploring Gender
Diversity In Taxonomy, Data Collection, And Through The Research Data
Lifecycle}. In: Journal of eScience Librarianship 10 (2021) 4: e1219,
\url{https://doi.org/10.7191/jeslib.2021.1219}

Der Text versucht, vor allem Bibliothekar*innen, welche Forschende beim
Forschungsdatenmanagement beraten, Argumente und Alternativen an die
Hand zu geben, um Geschlecht in der Forschung und dann in den Daten
abzubilden. Die Begrenzung von Fragen nach dem Geschlecht in Umfragen
und Ähnlichem auf zwei binäre Möglichkeiten wird von den Autor*innen als
vollkommen ungenügend angesehen. Aber auch Versuche, die Möglichkeiten
der Abbildung geschlechtlicher Identitäten zu erweitern, indem als
dritte Möglichkeit ein «Other» eingeführt wird, werden kritisiert. Hier
würden Menschen, die sich nicht als männlich oder weiblich verorten
(wollen), gezwungen, sich als «andere» zu definieren. Die Autor*innen
führen den Begriff der «data violence» (geprägt von Anna Hoffmann) ein,
um die Wirkung dieses Vorgehens zu beschreiben.

Grundsätzlich werden im Text verschiedene Vorgehensweisen für die
Datenerhebung vorgestellt, welche solche data violence verringern
sollen: Vom Verzichten auf die Abfrage nach dem Geschlecht, wenn sie für
die Forschung nicht notwendig ist, bis zu verschiedenen Formen, Menschen
zu ermöglichen, ihre geschlechtliche Identität frei anzugeben.

Was hier an diesem Text von Interesse ist, ist dass es von den
Autor*innen als eine Aufgabe von Bibliothekar*innen im Bereich
Forschungsdatenmanagement angesehen wird, auf inklusivere Formen der
Forschung hinzuwirken. (ks)

\begin{center}\rule{0.5\linewidth}{0.5pt}\end{center}

Asiedu, Nasir Koranteng ; Appiah, Deborah Kore ; Alhassan, Ishawu
(2021). \emph{Examination Pressure: Assessment of an Academic Library's
Late-Night Service to Patrons}. In: International Information \& Library
Review {[}Latest Articles{]},
\url{https://doi.org/10.1080/10572317.2021.1993720} {[}Paywall{]}

Diese kurze Umfrage zur Nutzung der Bibliothek der C.K. Tedam University
of Technology and Applied Sciences in Navrongo, Ghana während ihrer
langen Öffnungszeit (von 10:00 bis 05:00 morgens) zeigt etwas, das schon
aus vielen anderen Umfragen bekannt ist: Es ist egal, wie lange eine
Bibliothek geöffnet hat, die Nutzer*innen wollen immer mehr.
Bemerkenswert ist, dass bei der Frage, wann die Bibliothek zumeist
benutzt wird, 0 Studierende den Zeitraum von 03:00 bis 05:00 angeben,
sondern vor allem (84 von 157 Antworten auf diese Frage) die Zeit von
10:00 bis 12:00. Aber dann, bei der Frage, ob die Bibliothek 24 Stunden
am Tag geöffnet sein sollte, 89 \enquote{stark zustimmen} und 17
\enquote{zustimmen} (von 151 Antworten). Die Bibliothek schloss daraus
ein Interesse und hat jetzt zumindest in der Prüfungszeit tatsächlich
rund um die Uhr geöffnet. Aber eigentlich ist es eher ein weiterer
Hinweis darauf, dass die konkrete Frage, ob die Öffnungszeiten einer
Bibliothek erhöht werden sollten, in solchen Umfragen keine sinnvolle
ist: Die Nutzer*innen werden bis zum extremen 24/7 immer eine Ausweitung
wünschen. Die Frage ist, ob sie dies auch nutzen, wenn es eingeführt ist
-- und da zeigt die gleiche Umfrage jetzt schon, dass
\enquote{Randzeiten} eher wenig Besuche verzeichnen.

Ein anderes Ergebnis dieser Umfrage, welches andere Umfragen
untermauert, ist, dass Nutzer*innen -- nach ihren eigenen Angaben -- die
Bibliothek zwischen 2 (59 von 168 Antworten) und 3--4 (47) Stunden
nutzen. Auch die Nutzung bis zu einer Stunde (30) ist relativ hoch.
Längere Aufenthalte -- auf die hin Bibliotheken ja verstärkt durch
Angebote wie Cafés oder Orte zum Entspannen ausgerichtet werden -- sind
weiterhin die Ausnahme. (ks)

\begin{center}\rule{0.5\linewidth}{0.5pt}\end{center}

Marler, Will (2021). \emph{\enquote*{You can't talk at the library}: the
leisure divide and public internet access for people experiencing
homelessness}. In: Information, Communication \& Society {[}Latest
Articles{]}, \url{https://doi.org/10.1080/1369118X.2021.2006742}
{[}Paywall{]}

In dieser ethnographischen Studie, durchgeführt 2017 bis 2019 in einer
Öffentlichen Bibliothek und einer sozialen Einrichtung in Chicago, geht
es inhaltlich um die Nutzung des Internets durch Menschen ohne festen
Wohnsitz, alle zwischen 50--65 Jahren alt. Beide Einrichtungen bieten
die kostenlose Nutzung des Internets an, aber in verschiedenen
institutionellen Settings. Die soziale Einrichtung kann frei, auch ohne
Anmeldung, genutzt werden. Die Rechner dort stehen ebenso zur Verwendung
bereit, ohne klare Regeln in Bezug auf Nutzungszeit oder Lautstärke. In
der Bibliothek ist der Gebrauch der Computer an eine Bibliothekskarte
gebunden, für die eine amtliche Anmeldung im Bundesstaat notwendig ist
(die Räume der Bibliothek selber können auch ohne eine solche besucht
werden). Zudem gibt es Regeln, die durch Software und Security
durchgesetzt werden. In beiden Einrichtungen ist der Grund, Computer und
Internetzugang anzubieten, Zugang zu digitalen Angeboten zu ermöglichen.

Der Autor fokussiert jetzt darauf, wie Menschen ohne festen Wohnsitz
diesen Zugang nutzen. Dabei stellt er heraus, dass sie einen starken
Fokus auf \enquote{Freizeitnutzung} legen, die ihnen ermöglicht, sowohl
mit Stress, der mit ihrer Lebenssituation zu tun hat, umzugehen als auch
soziale Kontakte aufzubauen. Oder anders: Die Nutzung als
\enquote{leisure time} ermöglicht ihnen, ihren Alltag besser zu
gestalten. Der Unterschied ist dabei, dass dies in der sozialen
Einrichtung besser funktioniert, als in der Bibliothek. Der Fokus in
letzterer ist auf eine \enquote{produktive Nutzung} der Computer
ausgerichtet. Der Autor formuliert hier als Anforderung, auch die
Möglichkeit einzuräumen, Computer für andere Dinge nutzen zu können,
beispielsweise für das Abspielen von Musik.

Interessant ist, dass die Öffentlichen Bibliotheken in Chicago
selbstverständlich, wie so viele andere auch, sich ebenso als soziale
Orte verstehen, an denen sich alle Mitglieder der Community treffen
können. Es ist bemerkenswert, dass sich dies in der konkreten Umsetzung
zumindest für Menschen in der Situation, keinen stabilen Wohnsitz zu
haben, weniger umsetzt als in der sozialen Einrichtung, die eher auf
Menschen in ihrer Situation ausgerichtet ist. (Die Bibliothek wird vom
Autor eher als Ort geschildert, in welchem sich Menschen in
unterschiedlichen sozialen Lagen versammeln.) (ks)

\begin{center}\rule{0.5\linewidth}{0.5pt}\end{center}

Ziegler, Sophia (2022). \emph{Toward Empathetic Digital Repositories: An
Interview with Diego Pino Navarro}. In: Journal of Critical Digital
Librarianship 2 (2022) 1: 1--10,
\url{https://digitalcommons.lsu.edu/jcdl/vol2/iss1/1/}

Diego Pino Navarro ist einer der Entwickler von Archipelago Commons, der
Software, welche von verschiedenen Bibliotheken in New York City genutzt
wird, um ihre Sammlungen digital zu präsentieren. In diesem Interview
reflektiert er, auch auf seiner Erfahrung als \enquote{Brown man} in der
IT-Branche, der aus Südamerika in die USA eingewandert ist und einen
Hintergrund als First Nation People mitbringt, was es heisst, ein
Repository für Bibliotheken zu bauen, das versucht, möglichst wenig zu
standardisieren und möglichst viel auf die Erfahrungen der Bibliotheken
selber zu hören. Dabei kritisiert er nicht nur Firmen, die aus
ökonomischen Gründen Software vereinheitlichen, sondern auch Haltungen
von Software Engineers, die Software und nicht die tatsächliche Nutzung
und Effekte dieser Nutzung in den Mittelpunkt der Entwicklungsarbeit zu
stellen. (ks)

\begin{center}\rule{0.5\linewidth}{0.5pt}\end{center}

Seale, Maura ; Hicks, Alison ; Nicholson, Karen P. (2022). \emph{Toward
a Critical Turn in Library UX}. In: College \& Research Libraries 83
(2022) 1: 6--24, \url{https://doi.org/10.5860/crl.83.1.6}

Library UX ist als Begriff im DACH-Raum eher seltener bekannt, aber als
Praxis schon etabliert: Es geht dabei um das Design von Bibliotheken und
ihrer Angebote, wobei die \enquote{Nutzer*innen} und ihre Interessen in
den Mittelpunkt gestellt werden. Mittels methodischer Rahmen sollen
diese Interessen eruiert und daraus dann weitere Entwicklungen einer
Bibliothek gestaltet werden. Im DACH-Raum bekannt ist \enquote{Design
Thinking} als Methode, die dem Library UX zugeordnet werden kann. Aber
auch kleinere Projekte, die sich an ethnologischen Methoden orientieren,
oder viele Umfragen lassen sich grundsätzlich dem zuordnen, was im
englischsprachigen Bibliothekswesen als Library UX bezeichnet wird.
Insoweit lässt sich die in diesem Text geäusserte Kritik von Seale et
al.~auch auf diese Praxen in Bibliotheken im DACH-Raum übertragen.

Grundsätzlich zeigen sie, dass Library UX als reines Methodenset
neutraler Methoden verstanden wird und die Grundannahmen, die mit ihm
einhergehen -- beispielsweise wer oder was Nutzer*innen sind und dass
ihre Erfahrungen der Ausgangspunkt aller Entscheidungen über
Bibliotheken sein sollen -- nicht reflektiert werden. Dies sei in einer
Zeit, in der es im Bibliothekswesen immer klarer wird, dass dieses von
politischen, ökonomischen und sozialen Entwicklungen ebenso betroffen
sei wie alle anderen Einrichtungen auch, nicht mehr nachvollziehbar.

Auch Methoden des Library UX (oder im DACH-Raum eher Design Thinking)
oder die Annahmen hinter diesen, seien nicht neutral, sondern --
insbesondere wenn nicht durchdacht -- eher dazu geeignet,
gesellschaftliche Ausschlussprozesse im Bibliotheksbereich zu
reproduzieren oder gar zu verstärken. Sie gehen dies anhand einer Anzahl
von Methoden, die in diesem Bereich verwendet werden, durch. Der Text
ist eine Anregung dazu, sich kritisch mit ihnen auseinanderzusetzen und
sie nicht einfach nur einzusetzen. (ks)

\begin{center}\rule{0.5\linewidth}{0.5pt}\end{center}

Rosen, Hannah ; Feather, Celeste ; Grogg, Jill ; Lair, Sharla (2022).
\emph{LYRASIS Research and an Inclusive Approach to Open Access in the
United States}. In: LIBER Quarterly 32 (2022): 1--18,
\url{https://doi.org/10.53377/lq.11078}

In diesem Artikel werden die Open-Access-Aktivitäten eines der grossen
Konsortien für Bibliotheken, Archive und Museen in den USA geschildert.
Das ist deshalb interessant, weil hier eine recht andere Landschaft
gezeigt wird, als sie aus Europa bekannt ist. Beispielsweise finden sich
hier viele Bibliotheken, die keine APCs zahlen können oder wollen. Es
gibt auch keine, aus Europa bekannten, nationalen Programme oder
Verträge. Das Konsortium unterstützt Open-Access-Initiativen, aber eher
solche, die ohne APCs funktionieren.

In gewisser Weise zeigt dies auf, dass die Entwicklungen im
Open-Access-Bereich wirklich nicht so sein müssen, wie sie es gerade
sind. Die Entwicklung in Europa ist nicht alternativlos. (ks)

\begin{center}\rule{0.5\linewidth}{0.5pt}\end{center}

Johnson, Charlotte M. (2022). \emph{Revisiting the Library Storage
Literature Review}. In: Collection Management {[}Latest Articles{]},
\url{https://doi.org/10.1080/01462679.2022.2043978} {[}Paywall{]}

Die Autorin stellt die Themen zusammen, welche zwischen 1995 -- dem
Jahr, in welchem schon eine ähnliche Übersicht erschien -- und 2021 in
Artikeln und anderen Publikationsformen zum Thema Magazinierung und
Magazinbau diskutiert oder untersucht wurden. Dies waren ihrer Recherche
nach 139, allerdings inkludierte sie nur englisch-sprachige. Der Text
ist ein guter Einstieg in das Thema und zeigt gleichzeitig relevante
Entwicklungen der letzten Jahrzehnte auf. Darunter sind beispielsweise
die Verbreitung von spezifischer Magazinsoftware, der Trend zum
gemeinsamen Betrieb von Magazinen durch mehrere Bibliotheken (in recht
unterschiedlichen Formen) und die Entwicklungen in Sachen Technik,
insbesondere der Verbreitung von elektronischen Dokumentlieferungen.
(ks)

\begin{center}\rule{0.5\linewidth}{0.5pt}\end{center}

Watstein, Sarah Barbara ; Johns, Elizabeth M. ; Puente, Mark A. ; Hahn,
Jim (edit.) (2022). \emph{Reference Services Review: Special Issue:
Anti-Racist Action in Libraries}. In: Reference Service Review 50 (2022)
1,
\url{https://www.emerald.com/insight/publication/issn/0090-7324/vol/50/iss/1}

In dieser Schwerpunktnummer reflektieren Bibliothekar*innen aus den USA
sowohl darüber, warum und wie Bibliotheken in rassistische Strukturen
eingebunden sind als auch darüber, wie dies zu ändern wäre. Das alles im
Kontext der US-amerikanischen Gesellschaft. Der Kontext ist bedeutsam:
Es wird -- ohne, dass es reflektiert würde -- immer im Rahmen und Denken
des US-amerikanischen Bibliothekswesens diskutiert. Beispielsweise wird
sich in den Beiträgen kontinuierlich auf die US-amerikanische
Bibliotheksgeschichte oder auch US-amerikanische Einrichtungen und
Diskurse bezogen.

Thematisch geht es in vielen Beiträgen darum, die Erfahrungen von
Bibliothekar*innen, die BIPOC sind (nicht nur African Americans, sondern
auch Hispanic Americans und anderen Gruppen) zu schildern. In anderen
wird diskutiert, wie sich das Bibliothekswesen in den USA entwickelt und
dabei rassistische Strukturen reproduziert hat, auch entgegen dem
Verständnis von Bibliotheken als offener Einrichtung. Eine Anzahl von
Texten schildert auch, wie Bibliotheken vorgehen oder vorgehen sollten,
um diese Situation zu ändern. Diese sind dann recht unterschiedlich.
Einige schildern konkrete Initiativen, andere theoretische Zugänge
(inklusive, was vielleicht überrascht, klaren Bezügen zu sozialistischen
und marxistischen Theorien). Allen gemein ist die Überzeugung, dass sich
eine Veränderung nur durch tief in Strukturen und Identitäten
eingreifende Entwicklungen erreichen lässt und dass dafür eine
theoretische, auf der Geschichte von Gesellschaft und Bibliotheken
aufbauende Beschreibung des Status Quo notwendig ist. (ks)

\begin{center}\rule{0.5\linewidth}{0.5pt}\end{center}

Glagla-Dietz, Stephanie ; Gabermann, Stephanie (2020).
\emph{Standardnummern für Personen : Qualitätsverbesserung durch das
Zusammenspiel intellektueller und maschineller Formalerschließung.} In:
Dialog mit Bibliotheken 32 (2020) 2, S. 20--25,
\url{https://nbn-resolving.org/urn:nbn:de:101-2020062250}

ORCID-iDs setzen sich zunehmend als Identifikatoren und Wegweiser zu
persönlichen Profilen mit Publikationsnachweisen durch. Die
Non-Profit-Organisation ORCID ermöglicht Wissenschaftler*innen das
Führen eines eigenen wissenschaftlichen Profils, inklusive
unkomplizierter Verwaltung der eigenen Publikationsdaten. Bereits seit
einiger Zeit sind sie deshalb auch in Personennormdatensätzen der GND
vertreten. Seit Frühjahr 2020 werden diese Standardnummern auch bei den
maschinellen Erschließungsverfahren der DNB verwendet, die automatisiert
Titeldaten untereinander und mit Normdaten verknüpfen. Die angegebenen
ORCID-iDs in den digital verfügbaren Publikationen können ohne
menschliches Zutun direkt den korrekten beteiligten Personen zugeordnet
werden, sofern die ORCID-iD im GND-Profil hinterlegt ist. Dies ist eine
bedeutende Neuerung, da Netzpublikationen aufgrund ihrer schieren Zahl
nicht intellektuell verknüpft werden können. Die Grundlage für diese
Vorgehensweise ist ein ausreichend hoher Anteil von hinterlegten
ORCID-iDs in GND-Datensätzen. Deswegen wird diese Zahl über verschiedene
Verfahren automatisiert oder händisch erhöht. Zum Beispiel ist es
inzwischen auch möglich, seine Publikationen im Katalog der DNB zu
claimen und somit sowohl die Publikationen seinem ORCID-Profil als auch
seine ORCID iD zu seinem GND-Datensatz hinzuzufügen. Die GND ist ein
großes Kooperationsprojekt und bedarf seit jeher der Zusammenarbeit
vieler Bibliotheken. Nun kann man auch als individuelle
ORICD-iD-Inhaber*in unkompliziert dazu beitragen, dass Verknüpfungen
besser funktionieren, Dubletten vermieden werden und auch die
maschinelle Erschließung von Netzpublikationen auf einem höheren Niveau
stattfinden kann. (yp)

\begin{center}\rule{0.5\linewidth}{0.5pt}\end{center}

Wastl, Jürgen (2017). \emph{Forschungsinformationssysteme: Not oder
Tugend? : Reaktive und proaktive Strategien zur Implementierung von
Forschungsinformationssystemen und innovative Ansätze für die Zukunft.}
In: BIT online 20 (2017) 2, S. 99--112,
\url{https://doi.org/10.17863/CAM.10357},
\url{http://www.b-i-t-online.de/heft/2017-02/fachbeitrag-wastl.pdf}

Forschungsinformationssysteme (FIS) etablieren sich seit einigen Jahren
als Ablösung von institutionellen Forschungsbibliographien. Sie umfassen
dabei nicht nur die vollständige Dokumentation von Publikationen der
Angehörigen der Einrichtung, sondern grundsätzlich aller
Forschungsaktivitäten, etwa Projekten und Auszeichnungen. Sie verknüpfen
die Informationen von verschiedenen Entitäten und unterstützen einen
Austausch mit anderen Systemen. Charakteristisch ist eine Zusammenarbeit
verschiedener Bereiche, zum Beispiel von Bibliotheken und
Forschungsabteilungen.

Jürgen Wastl geht in seinem Artikel am Beispiel des FIS der Universität
Cambridge auf Vorteile, Notwendigkeiten, Akzeptanzprobleme und lokale
Besonderheiten ein. Der ursprüngliche Grund für die Einführung war der
große Aufwand bei neuen Zusammentragungen und Evaluationen für
finanzielle Förderungen. Inzwischen kann man zahlreiche weitere Gründe
für die Implementierung eines FIS anführen und weiß um typische
Schwierigkeiten, die damit einhergehen. Herausforderungen sind dabei zum
einen, dass man nicht der Ursprung der Daten ist und auch mit
ermöglichter Einspeisung in das FIS viel für eine hohe
Qualitätssicherung der Daten leisten muss. Zum anderen ist die Akzeptanz
bei den eigenen Forschenden entscheidend, damit Publikationen und
Projekte eingetragen werden und auf Vollständigkeit basierende Services
funktionieren.

Aus dem anfänglichen retrospektiven Berichterstattungssystem entwickelte
sich schließlich ein modernes Netzwerk aus Forschungsinformationen, das
durch eine gesicherte Interoperabilität auf vielfache Weise einen
nachhaltigen Mehrwert garantiert und Wissenschaftler*innen ein
nützliches Werkzeug anbietet. Wastls Erfolgsrezept für ein gutes FIS ist
die Einbindung des Großteils der Angehörigen sowie ein maßgeschneiderter
Aufbau auf die lokalen Bedürfnisse. Dabei soll das
Forschungsinformationssystem ohne Zwang genutzt werden und stattdessen
durch Attraktivität überzeugen. (yp)

\begin{center}\rule{0.5\linewidth}{0.5pt}\end{center}

Ausschuss für Wissenschaftliche Bibliotheken und Informationssysteme
(2021). \emph{Datentracking in der Wissenschaft: Aggregation und
Verwendung bzw. Verkauf von Nutzungsdaten durch Wissenschaftsverlage :
ein Informationspapier des Ausschusses für Wissenschaftliche
Bibliotheken und Informationssysteme der Deutschen
Forschungsgemeinschaft},
\url{https://www.dfg.de/download/pdf/foerderung/programme/lis/datentracking_papier_de.pdf}

Das Informationspapier des AWBI (Ausschuss für Wissenschaftliche
Bibliotheken und Informationssysteme) beschreibt schwerpunktmäßig die
neue Praxis bezüglich Datentracking auf Seiten der Wissenschaftsverlage
und in welchem Kontext diese Entwicklung stattfindet. Dabei positioniert
er sich sehr kritisch gegenüber einer systematischen Sammlung von
Nutzerdaten und der kommerziellen Nutzung dieser. Große
wissenschaftliche Verlage scheinen sich auch im Kontext der
Open-Access-Transformation und wechselnder Finanzierungsstrukturen auf
das neue Geschäftsmodell als \enquote{Data Analytics Businesses}
auszurichten. Die Erfassung von Nutzungsdaten wird dabei für den
gesamten Wissenschaftskreislauf beworben. Um dies zu erreichen, koppelt
man die Datensammlung an verschiedene Services, unter anderem zu
Forschungsdatenmanagement und Forschungssoftware. Beispiele wie das
System \enquote{Pure} von Elsevier ermöglichen dabei die Erfassung von
Tätigkeiten Einzelner sowie ganzer Institutionen und umfassen mehrere
Schritte im Forschungsprozess. Es tritt die aus anderen Bereichen
bekannte Abwägung zwischen komfortablen Lösungen und der Kontrolle über
die eigenen Daten auf. Der AWBI sieht die Gefahr einer völligen
Privatisierung der Informationen über deutsche Wissenschaftsprozesse,
-entwicklungen und -inhalte.

Die Aggregation und Auswertung von wissenschaftlichen Aktivitäten und
Profilen wird allerdings nicht grundsätzlich verurteilt, schließlich
kann die Wissenschaft selbst von deren Auswertung profitieren. Statt
einer unkontrollierten Vermengung mit kommerziellen Interessen soll
dabei Transparenz mithilfe deutlicher rechtlichen Regulierungen gelten,
die eine aufgeklärte Einwilligung Einzelner vorsieht. Außerdem soll die
Wissenschaft starken Einfluss auf die Praxis haben. Bis dahin wird zu
vermehrter Achtsamkeit bei Verträgen mit Verlagen bezüglich Data Privacy
geraten. (yp)

\begin{center}\rule{0.5\linewidth}{0.5pt}\end{center}

Berichterstattung zur Entlassung von Gerald Leitner als
IFLA-Generalsekretär. In: Biblioteksbladet 2022,
\url{https://www.biblioteksbladet.se/?sida=1\&s=gerald+leitner\&sort=date}

Das schwedische Biblioteksbladet ist eine der wenigen (vielleicht sogar
die einzige der) bibliothekarischen Publikationen, die systematisch
versucht, den Gründen der Entlassung Gerald Leitners als
IFLA-Generalsekretär auf den Grund zu gehen. Nachdem die IFLA in ihrem
Bericht aus dem Meeting des Governing Board vom 8. April
(\url{https://www.ifla.org/news/ifla-news-from-the-governing-board-meeting/})
die Gründe für die Entlassung offen ließ, veröffentlichte
Biblioteksbladet verschiedene Interviews, Kommentare, Timelines und
andere Recherchen. Die IFLA bestreitet einige der in Biblioteksbladet
getätigten Aussagen in zwei später veröffentlichten Statements
(\url{https://www.ifla.org/news/ifla-gb-response-to-biblioteksbladet-article/},
\url{https://www.ifla.org/news/termination-of-the-contract-with-sg/}).
(eb)

\hypertarget{covid-19-und-die-bibliotheken-vierte-welle}{%
\subsection{2.2 COVID-19 und die Bibliotheken, Vierte
Welle}\label{covid-19-und-die-bibliotheken-vierte-welle}}

Yatcilla, Jane Kinkus ; Young, Sarah (2021). \emph{Library Responses
During the Early Days of the Pandemic: A Bibliometric Study of the 2020
LIS Literature}. In: Journal of Library Administration (Latest
Articles), \url{https://doi.org/10.1080/01930826.2021.1984139}
{[}Paywall{]}

Ein Zeichen dafür, dass die Pandemie schon eine ganze Zeit Auswirkungen
hat, ist, dass ausreichend viele Texte zum Themenbereich COVID-19 und
die Bibliotheken erschienen sind, um diese strukturiert auszuwerten. Im
Titel dieser Studie wird diese Auswertung zwar bibliometrisch genannt,
aber eigentlich wurde hier vor allem eine Codierung von Titeln und
Abstracts vorgenommen.

Insgesamt werteten die Autorinnen 273 dieser Texte (die sie aus LISA und
LISTRA gewonnen hatten) aus und konnten zeigen, dass die Pandemie
Bibliotheken in der ganzen Welt betroffen hat, das alle Bibliothekstypen
betroffen waren, aber auch, dass die Themen der Texte relativ ähnlich
waren. Vor allem ging es um \enquote{remote services} und um
Bibliotheksbestände.

Beide Autorinnen postulieren, dass die relativ schnellen
Veröffentlichungen auch zeigen würden, dass während der Pandemie
Qualitätssicherungsmassnahmen für wissenschaftliche Zeitschriften
gelockert worden wären. Grundsätzlich sei dies durch die Situation
verständlich, müsse aber nach der Pandemie wieder geändert werden.
Wirklich nachzuvollziehen ist das aber nicht: Es wird nur gezeigt, dass
relativ schnelle Veröffentlichungen möglich waren, aber ob dies wirklich
Einfluss auf die Qualität der Publikationen hatte, kann mit der in
dieser Studie verwendeten Methode nicht gezeigt werden. (ks)

\begin{center}\rule{0.5\linewidth}{0.5pt}\end{center}

Trembach, Stan ; Deng, Liya (2021). \emph{A window of opportunity:
Sustained excellence in academic library response to the challenges of
COVID-19}. In: College \& Undergraduate Library {[}Latest Articles{]},
\url{https://doi.org/10.1080/10691316.2021.1995921} {[}Paywall{]}

Diese Studie endet, obwohl sie die Situation von Bibliotheken in der
COVID-19 Pandemie als herausfordernd darstellt, erstaunlich positiv.
Bibliotheken würden zeigen, dass sie pro-aktiv auf die Herausforderungen
in dieser Situation reagieren. Damit gäbe es auch Hoffnung, dass sie
sich langfristig gut entwickeln können.

Untersucht wurden 50 kleinere Hochschulbibliotheken in den USA (je an
Einrichtungen mit unter 10.000 Studierenden), um nicht unbedingt nur die
grössten Bibliotheken zu studieren und gleichzeitig, um eine gewisse
Diversität der untersuchten Einrichtungen zu gewährleisten. In einem
ersten Schritt wurden im Herbst 2020 die Websites der Bibliotheken
systematisch auf Angebote und Hinweise zur Pandemie durchsucht. Im
zweiten Schritt wurden die Direktor*innen dieser Bibliotheken Anfang
2021 in einer Umfrage angegangen.

Es zeigt sich, dass die Bibliotheken in vielen Punkten ähnlich
reagierten: Reference Services wurden überall Online angeboten, die
E-Ressourcen beworben, Informationen zur Pandemie verbreitet. Gespalten
waren sie dabei, ob sie (Ende 2020) physischen Zugang zur Bibliothek
gewährten oder die Ausleihe mit Pick-Up Diensten ermöglichten. (Je rund
die Hälfte tat es oder tat es nicht.) Anfang 2021 war die Situation dann
diverser. Ob Bibliotheken Zugang zu ihren Räumen ermöglichten und für
wen, war ganz unterschiedlich geregelt. In der Umfrage zeigte sich, dass
fast alle antwortenden Direktor*innen über den Einfluss der Pandemie auf
die eigene Bibliothek und Community besorgt waren. Viele beschrieben die
Situation als ständiges Krisenmanagement. Für die Zukunft regten sie zum
Beispiel an, Katastrophenpläne auszuarbeiten, oder zu überprüfen, welche
bibliothekarischen Arbeiten auch langfristig online organisiert werden
können. (ks)

\begin{center}\rule{0.5\linewidth}{0.5pt}\end{center}

Akullo, Winny Nekesa ; Okojie, Victoria ; Diouf, Antonin Benoit ;
Kotsokoane, Nthabiseng (2021). \emph{Role of Library Associations in
Supporting the Library Sector during the COVID-19 Pandemic in Africa}.
In: International Information \& Library Review {[}Latest Articles{]},
\url{https://doi.org/10.1080/10572317.2021.1990566} {[}Paywall{]}

Hier werden die Ergebnisse einer Umfrage unter nationalen
Bibliotheksverbänden in Afrika über ihre Erfahrungen während der
COVID-19 Pandemie publiziert. Leider wird nicht angeben, zu welchem
Zeitpunkt diese durchgeführt wurde, aber die Ergebnisse deuten auf Ende
2020, Anfang 2021 hin.

Es antworteten 18 der 30 angeschriebenen Verbände. Fast alle sind auch
Mitglied der IFLA und anderer Vereinigungen und sind somit also
untereinander vernetzt. Diese Netzwerke sorgten dafür, dass
Informationen ausgetauscht werden konnten. Alle unternahmen es, ihre
Mitgliedsbibliotheken über die Pandemie und deren Auswirkungen zu
informieren. Viele, aber nicht alle, organisierten dafür auch
Onlineveranstaltungen. Für die meisten stellte sich als wichtigste
Herausforderung die schlechten Internetverbindungen dar. Genutzt wurden
dafür auch -- oder gerade -- die Facebook-Auftritte der
Bibliotheksverbände. Als mittelfristiges Ziel wurde, auf der Basis der
Erfahrungen aus der Pandemie, genannt, Katastrophenpläne zu erarbeiten.
(ks)

\begin{center}\rule{0.5\linewidth}{0.5pt}\end{center}

Altman, Adam (2022). \emph{Library Technology and Its Perceptions at
Small Institutions of Higher Education: The COVID-19 Factor}. In Journal
of Library Administration 62 (2022) 1: 67--84,
\url{https://doi.org/10.1080/01930826.2021.2006986} {[}Paywall{]}

Dieser kurze Texte bestätigt, was an anderer Stelle schon vermutet
wurde: Die Wahrnehmung der digitalen Angebote, welche Bibliotheken
machen, durch Bibliothekar*innen als auch Nutzer*innen, hat sich während
der COVID-19 Pandemie verändert. Der Autor fragte -- im Rahmen seiner
Dissertation -- in sechs Bibliotheken kleinerer Hochschulen (unter 5000
Studierende) in den USA Vertreter*innen dieser beiden Gruppen -- einmal
per Interviews, einmal per Umfrage unter den Bibliotheken -- ab. Dabei
zeigte sich, wenig überraschend, dass die digitalen Angebote während der
Pandemie ausgebaut wurden.

Interessanter ist die Wahrnehmung durch die Nutzer*innen: Auf der einen
Seite nahmen sie diesen Ausbau auch wahr und nutzten die Angebote auch
mehr, auf der anderen Seite bezog sich dies aber auf Datenbanken und
elektronische Medien. Andere Angebote wurden nur selten bemerkt.
Insbesondere die im Bereich Social Media scheinen kaum bemerkt worden zu
sein. (ks)

\begin{center}\rule{0.5\linewidth}{0.5pt}\end{center}

Wahler, Elizabeth A. ; Spuller, Rbecca ; Ressler, Jacob ; Bolan,
Kimberly ; Burnard, Nathaniel (2022). \emph{Changing Public Library
Staff and Patron Needs Due to the COVID-19 Pandemic}. In: Journal of
Library Administration 62 (2022) 1: 47--66,
\url{https://doi.org/10.1080/01930826.2021.2006985} {[}Paywall{]}

Der Anfang dieses Textes liest sich wie ein Fall von \enquote{vocational
awe} -- wenn Bibliotheken ihre Position aufzuwerten scheinen, indem sie
sich als Einrichtungen mit einer viel grösseren Bedeutung beschreiben,
als sie sinnvoll sein können. Hier wird zum Beispiel behauptet, dass sie
nicht nur \enquote{Dritte Orte} seien, sondern die soziale Basis für den
Kampf gegen \enquote{food deserts} und für die psychosoziale Gesundheit
ihrer Community. Aber wenn man sich dort durchkämpft, gelangt man zum
eigentlichen Thema, nämlich einer Umfrage unter Nutzer*innen und
Personal eines mittelgrossen Öffentlichen Bibliothekssystems (14
Zweigstellen) in Allen County, Indiana, zu den Auswirkungen von
COVID-19. Die hat ihren eigenen Wert, auch wenn man dann den
übertriebenen Conclusions, welche die Bibliothek wieder als rettende
Einrichtung für die Community darstellen, nicht folgen muss.

Die Umfrage wurde eigentlich als normale Umfrage für die
Weiterentwicklung der Bibliotheken durchgeführt, fiel aber auf den
Zeitraum von Ende 2020 und Anfang 2021. Deshalb wurde sie um offene
Fragen zu COVID-19 erweitert. Die Auswertung dieser Fragen wird im
Artikel dargestellt.

Grundsätzlich muss gesagt werden, dass je ein Drittel der antwortenden
Nutzer*innen und des Personals etwas zu dieser Frage sagten, die je
anderen zwei Drittel nicht. Zudem zeigte sich, dass zumindest die
antwortenden Nutzer*innen, eher weisser, eher reicher (mehr
Homeownership) und älter waren als der Durchschnitt der Community. Sie
sind wohl tendenziell privilegierter.

Die möglichen Probleme und Veränderungen durch COVID-19, die in der
bibliothekarischen Literatur 2020 besprochen wurden, traten zwar auf,
aber nur bei einer kleinen Zahl der Befragten. Von den betreffenden
Antworten (900) der Nutzer*innen erwähnten 37.1\,\% grundsätzliche
Veränderungen bei Aktivitäten (beispielsweise weniger Freund*innen
treffen, mehr daheim sein), 24.9\,\% mehr Isolation, 15.7\,\% entweder
Arbeitsplatzwechsel oder mehr Stress im Job (beispielsweise durch
Umstellung auf Onlineangebote), 16.7\,\% Probleme, die Bibliothek zu
nutzen, vor allem durch reduzierte Öffnungszeiten und Angebote. 12.0\,\%
berichteten von zusätzlichen Herausforderungen als Eltern, insbesondere
im Finden von interessanten Aktivitäten und Unterstützung ihrer Kinder
beim Online-Unterricht. Probleme mit dem Zugang zu Angeboten an sich
(8,8\,\%) und der Zunahme von mentalen Problemen (8.0\,\%) wurden
seltener berichtet. Dass die Bibliothek anders als zuvor genutzt wurde,
zum Beispiel andere Medien entliehen oder an sich mehr gelesen wurde,
vermerkten dann nur noch 5.4\,\% und die verstärkte Nutzung von
digitalen Angeboten 4.3\,\%.

Das Personal beschrieb als Veränderung vor allem reduzierte Kontakte mit
Nutzer*innen (28.3\,\%), die Notwendigkeit, Bestimmungen im Bezug auf
COVID-19 durchzusetzen (22.1\,\%) oder Arbeit nach Online zu verlagern
(12.4\,\%). Aber angesichts der Zahlen scheinen die Auswirkungen von
COVID-19 auch nicht so gross gewesen zu sein. Zumindest nicht so, dass
sie viel erwähnt wurden. (ks)

\begin{center}\rule{0.5\linewidth}{0.5pt}\end{center}

Eva, Nicole (2021). \emph{Information Literacy Instruction during
COVID-19}. In: Partnership: The Canadian Journal of Library and
information Practice and Research, 16 (2021) 1,
\url{https://doi.org/10.21083/partnership.v16i1.6448}

Shin, Nancy ; Pine, Sally ; Martin, Carolyn ; Bardyn, Tania (2021).
\emph{Academic Library Instruction in the Time of a COVID-19 Pandemic --
Lessons Learned}. In: Journal of Web Librarianship {[}Latest
Articles{]}, 2021, \url{https://doi.org/10.1080/19322909.2021.2015046}
{[}Paywall{]}

Eine kurze Umfrage unter Wissenschaftlichen Bibliotheken in Kanada (Eva
2021), verschickt im November 2020, zu Veränderungen in Anzahl und
Methodik von Information Literacy Sessions, zeigte, dass -- im Vergleich
zum Herbstsemester 2019 -- die Zahl dieser Veranstaltungen massiv
zurückging. Bibliotheken aller Grössen boten diese weniger an und
erhielten auch weniger Anfragen für diese. Eine kleine Zahl von
Bibliotheken berichtete von mehr Kontakten mit Studierenden oder mehr
Aktivitäten auf ihren Learning Management Systemen. Der Grossteil der
Veranstaltungen wurde in einer Form -- also entweder Online oder in
Präsenz -- durchgeführt, nur eine sehr kleine Anzahl in hybrider Form.

Eine ähnliche, im Frühling 2021 durchgeführte Umfrage in den USA (Shin
et al.~2021) kam zu dem Ergebnis, dass die grösste Herausforderung von
Bibliothekar*innen war, sich auf die Online-Kommunikation mit
Studierenden und das Durchführen von Veranstaltungen im Online-Modus
umzustellen. Ein grosser Teil hat dies jetzt gelernt und wird es wohl in
der Zeit nach der Pandemie weiter betreiben. Der Artikel zur Umfrage
ruft Bibliotheken aufgrund dieser Daten auf, in die Infrastruktur und
den Kompetenzaufbau beim Personal für solche Online-Kommunikation zu
investieren. (ks)

\begin{center}\rule{0.5\linewidth}{0.5pt}\end{center}

Ayeni, Philips O. ; Agbaje, Blessed O. ; Tippler, Maria (2021). \emph{A
Systematic Review of Library Services Provision in Response to COVID-19
Pandemic}. In: Evidence Based Library and Information Practice 16 (2021)
3, \url{https://doi.org/10.18438/eblip29902}

Diese \enquote{Systematic Review} versucht, einen Überblick über die
Reaktionen von Bibliotheken während der COVID-19 Pandemie zu geben. Die
Ergebnisse sind wenig überraschend: Bibliotheken übertrugen ihre Arbeit
und Aufgaben weitenteils in die digitale Sphäre und nutzen dafür
praktisch alle sinnvoll denkbaren technischen Möglichkeiten: Vorhandene
Lernmanagementsysteme, Videokonferenztools und Social Media. Zudem
übernahmen sie es, allgemeine Informationen über die Pandemie zu
verbreiten.

Was an dem Text hervorsticht, ist, dass er in gewisser Weise zeigt, dass
die Methodik der \enquote{Systematic Reviews}, die aus der Medizin
übernommen ist und dort vor allem genutzt wird, um wissenschaftliches
Wissen zu medizinischen Fragen zu systematisieren, für Fragen zu
Bibliotheken nur eingeschränkt geeignet ist. Die Autor*innen führen die
Methode vollständig durch, aber dadurch beschränken sie zum Beispiel
ihre Quellenbasis ohne jede Not: Sie nutzen nur englischsprachige, peer
reviewte Literatur, die in fachlichen Datenbanken verzeichnet ist, was
sinnvoll ist, um in der Medizin nur Forschungsliteratur zu nutzen, aber
nicht, wenn man Erfahrungen aus Bibliotheken, die andere
Publikationskulturen haben, systematisieren will. Zusätzlich analysieren
sie die 23 Artikel, die sie finden, so auch tiefgehender, als es
notwendig für die eigentlichen Ergebnisse ist. (ks)

\begin{center}\rule{0.5\linewidth}{0.5pt}\end{center}

Charbonneau, Deborah H. ; Vardell, Emily (2022). \emph{The impact of
COVID-19 on reference services: a national survey of academic health
sciences librarians.} In: Journal of the Medical Library Association 110
(2022) 1: 56--62, \url{https://doi.org/10.5195/jmla.2022.1322}

Für diesen Artikel wurde eine weitere Umfrage dazu durchgeführt, wie
Bibliotheken während der COVID-19 Pandemie gehandelt haben -- hier
fokussiert auf Bibliothekar*innen in Medizinbibliotheken in den USA und
deren Reference Service. Die Ergebnisse sind wenig überraschend. Sie
zeigen, dass Bibliotheken ihre Arbeit zumindest in diesem Bereich auch
Online organisieren konnten. Zudem zeigte sich, dass vor allem die
Nutzer*innen die Angebote der Bibliotheken nutzen, die dies auch zu
anderen Zeiten taten.

Erwähnt wird der Artikel aber, weil hier auch noch \enquote{challenging
questions} gefragt wurde. Die gab es. All die Verschwörungstheorien,
welche während der Pandemie verbreitet wurden, fanden sich auch in
solchen Fragen wieder. Aber: Dabei handelte es sich nur um einen kleinen
Teil. Der Grossteil war ernsthaften Informationsbedürfnissen geschuldet.
(ks)

Singh, Kanupriya ; Bossaller, Jenny S. (2022). \emph{It's Just Not the
Same: Virtual Teamwork in Public Libraries.} In: Journal of Library
Administration {[}Latest Articles{]},
\url{https://doi.org/10.1080/01930826.2022.2057130} {[}Paywall{]}

Mithilfe von acht Interviews mit Direktor*innen von Öffentlichen
Bibliotheken im Südwesten der USA wird in dieser Studie versucht zu
klären, ob deren virtuelle Arbeit während der Pandemie Auswirkungen in
der Zukunft haben wird. Die These ist dabei unterschwellig, dass das,
was gut funktionierte, auch weiter betrieben wird.

Sicherlich wissen Direktor*innen nicht genau, wie sich das gesamte
Personal während der Pandemie fühlte oder auch jetzt fühlt. Aber sie
zeichnen ein recht positives Bild: Alle Bibliotheken schafften es,
relativ schnell umzustellen und sowohl die eigenen Angebote virtuell als
auch die eigene Arbeit in Teams zu organisieren, insbesondere die
Arbeitsmeetings. Auch der Umgang mit neuer Software für Online-Arbeit
gelang im Ganzen gut. Und grundsätzlich wird von positiven Erfahrungen
berichtet, beispielsweise, dass in Online-Meetings sich auch Personen
äusserten, die sonst oft schweigen. Andere Vorteile, insbesondere die
Ortsunabhängigkeit, werden dazu führen, dass zumindest zum Teil weiter
virtuell gearbeitet wird.

Im Titel des Textes ist die Meinung dazu angesprochen, dass es einen
Unterschied zwischen Meetings sowie Angeboten vor Ort und denen virtuell
gibt. Aber es fällt offenbar schwer, festzumachen, was genau dieses
\enquote{Etwas} ist. Es wird umschrieben als direkter Kontakt, der vor
Ort anders möglich sei. Aber erstaunlicherweise verbleibt das alles sehr
im Ungefähren. (ks)

\hypertarget{die-arbeit-in-bibliotheken}{%
\subsection{2.3 Die Arbeit in
Bibliotheken}\label{die-arbeit-in-bibliotheken}}

Desmeules, Robin Elizabeth (2020). \emph{The Bookbinding of Hortense P.
Cantlie for McGill Library. Surfacing a Legacy of Invisible Labor in the
Stacks}. In: Libraries: Culture, History, and Society 4 (2020) 2:
139--161, \url{https://doi.org/10.5325/libraries.4.2.0139} {[}Paywall{]}

Desmeules versteht ihre Studie als Beitrag zur Sichtbarmachung der in
Bibliotheken selber geleisteten Arbeit an den Sammlungen. Sie
thematisiert, dass diese -- zumeist von Frauen übernommene -- Arbeit
dequalifiziert und unsichtbar gemacht wird, während sie nicht nur für
das Funktionieren der Bibliotheken, sondern auch aller Arbeit, die auf
dem Funktionieren der Bibliothek aufbaut, notwendig ist. Beispielsweise
erscheine es oft so, als würde Forschung Arbeit sein, die auch
Reputation erlaubt -- also zum Beispiel durch Autor*innenschaft von
Artikeln sichtbar wird --, während Infrastrukturen, wie die Sammlungen
von Bibliotheken, einfach als immer irgendwie bestehender Hintergrund
gelten.

Anhand der freischaffenden Buchbinderin Hortense P. Cantlie, welche vor
allem in den 1950er und 1960er Jahren für die Rare Book and Special
Collections der Bibliothek der McGill University, Montreal, tätig war,
versucht sie dann den Nachweis zu führen, dass diese \enquote{Arbeit im
Hintergrund} ebenso notwendig und wichtig ist. Sie -- selber
Bibliothekarin an dieser Einrichtung -- kann dazu auf einige erhaltene
Spuren zurückgreifen: Die Zettel, auf denen Cantlie ihre Arbeit am
jeweiligen Buch vermerkte, sind in vielen Fällen immer noch in den
Büchern eingelegt (was, wie Desmeules bemerkt, auch einer Entscheidung
des Bibliothekspersonals bedurfte). Zudem wurde ihr Nachlass für die
Bibliothek selber erworben. In den Unterlagen der Bibliothek selber
finden sich auch Hinweise. Das unterscheidet Cantlie von vielen anderen
Kolleg*innen, welche am Funktionieren von Bibliotheken beteiligt sind.
Aber es ermöglicht Desmeules auch, zu zeigen, dass es sich unfragbar um
konkrete, benennbare und sichtbar zu machende Arbeit handelte, die
Cantlie leistete und nicht um reine Servicedienste. Ohne diese Arbeit
wären viele Forschungsarbeiten mit dem konkreten Bestand heute nicht
mehr möglich. (ks)

\begin{center}\rule{0.5\linewidth}{0.5pt}\end{center}

Merga, Margaret K. (2021). \emph{What is the literacy supportive role of
the school librarian in the United Kingdom?} In: Journal of
Librarianship and Information Science 53 (2021) 4: 601--614,
\url{https://doi.org/10.1177/0961000620964569}

Das eigentliche Thema dieser Studie, nämlich eine Analyse von
Arbeitsplatzbeschreibungen für Schulbibliothekar*innen in
Grossbritannien daraufhin, welche Rolle sie in der Förderung von
Kompetenzen spielen sollen, ist sehr spezifisch. Im Ergebnis zeigt sich
auch vor allem, dass es zwar die Erwartung gibt, dass sie eine Rolle
spielen, aber dass diese recht offen und unkonkret ist.

Interessanter für Leser*innen im DACH-Raum wird die Darstellung zum
Status Quo von Schulbibliotheken in verschiedenen englisch-sprachigen
Ländern recht weit am Beginn des Artikels sein. Hier wird sichtbar, dass
es in praktisch allen diesen Ländern eine Rückentwicklung gibt. Die
Professionalisierung der unterschiedlichen Schulbibliothekswesen nimmt
ab, insbesondere wird immer mehr nicht spezifisch ausgebildetes Personal
eingestellt. Auch die Zahl der Schulbibliotheken selber ist im
Schwinden. Es ist eine Erinnerung daran, dass
Professionalisierungsstrategien von Verbänden auch nicht funktionieren
können. (ks)

Young, Alyssa (2021). \emph{A Librarian and Biochemist's Experience
Building a Collaborative Partnership in the Classroom and Beyond}. In:
Issues in Science and Technology Librarianship 31 (2021) 98,
\url{https://doi.org/10.29173/istl2596}

In diesem kurzen Hands-on Text wird berichtet, wie aus einer einfachen
Anfrage eines Forschenden um Unterstützung für seine Lehre durch
aufmerksames Zuhören und kontinuierliches Zusammenarbeiten eine starke
Verbindung zwischen einer Bibliothekarin und diesem Forschenden
entstand. Inhaltlich ist das vielleicht nicht sehr aufregend, aber es
ist eine Erinnerung daran, dass die Etablierung solcher
\enquote{Partnerschaften} kontinuierliche Arbeit von beiden Seiten
bedeutet. (ks)

\begin{center}\rule{0.5\linewidth}{0.5pt}\end{center}

Saleem, Qurat Ul Ain ; Ali, Amna Farzand ; Ashiq, Murtaza ; Rehman,
Shafiq Ur (2021). \emph{Workplace harassment in university libraries: A
qualitative study of female Library and Information Science (LIS)
professionals in Pakistan}. In: The Journal of Academic Librarianship 47
(2021) 6: 102464, \url{https://doi.org/10.1016/j.acalib.2021.102464}

Diese Studie, die auf Interviews basiert, zeigt, dass auch in Pakistan
Frauen im Bibliothekswesen einem hohen Mass an Belästigung -- sowohl
verbal als auch physisch -- ausgesetzt sind. Die Autor*innen betonen
dabei, dass über das Thema weithin geschwiegen wird und es zum Beispiel
schwer war, überhaupt Frauen zu finden, die sich befragen liessen. Zudem
zeigen sie, dass Gesetze gegen Belästigung alleine das Problem nicht
lösen, sondern solche -- so ihre Interpretation -- erst einmal bekannt
sein müssen und dann ihre Einhaltung auch eingefordert werden muss.

Der Artikel gibt, neben dem Einblick in die Situation in Pakistan --
oder, genauer, Lahore -- auch eine Übersicht zum Kenntnisstand an sich.
Auch dieser ist ernüchternd: Das Problem Belästigung am Arbeitsplatz
wird zwar international immer wieder benannt, aber ist deshalb noch
nirgendwo verschwunden, auch nicht im Bibliothekswesen. (ks)

\begin{center}\rule{0.5\linewidth}{0.5pt}\end{center}

Hanell, Fredrik ; Ahlryd, Sara (2021). \emph{Information work of
hospital librarians: Making the invisible visible.} In: Journal of
Librarianship and Information Science (Online First),
\url{https://doi.org/10.1177/09610006211063202}

Die Studie untersucht anhand von drei Krankenhausbibliotheken in
Schweden, welche Arbeit der Bibliotheken für das medizinische Personal
\enquote{unsichtbar} ist und wie Bibliotheken strukturell sowie im
Alltag versuchen, diese sichtbar zu machen. Dazu wurden sowohl
Interviews mit Bibliotheksleitungen und Bibliothekar*innen durchgeführt
als auch der Alltag beobachtet. Interessant ist dies, weil in Schweden
grundsätzlich auf Evidence Based Medicine gesetzt wird -- also alle
medizinischen Entscheidungen auf dem jeweils bestmöglichen
wissenschaftlichen Wissen und Daten aufbauen sollen. Deshalb könnte man
erwarten, dass die Bibliotheken -- deren Aufgabe es ist, dies zu
unterstützen -- regelmässig vom Krankenhauspersonal benutzt werden. Das
ist nicht der Fall.

Hanell und Ahlryd können zeigen, dass auch in diesen Bibliotheken -- wie
sich schon in früheren Studien zeigte -- der Grossteil der Arbeit der
Bibliothekar*innen für das medizinische Personal nicht sichtbar ist.
Bibliotheken versuchen, dem entgegenzuwirken, indem sie ständig Kontakte
pflegen, versuchen, zu beraten und Angebote zu machen. Erfolgreich sind
sie damit in bestimmten Zusammenhängen, vor allem, wenn sie die eigenen
Kompetenzen bei der Recherche und Projektmanagement herausstellen. (ks)

\hypertarget{forschungsdatenmanagement-und-data-librarianship}{%
\subsection{2.4 Forschungsdatenmanagement und Data
Librarianship}\label{forschungsdatenmanagement-und-data-librarianship}}

Ziegler, Sophia ; Powell Duncan, Leah ; Costello, Gina (2021).
\emph{Editors' Introduction}. In: Journal of Critical Digital
Librarianship 1 (2021) 1: 1--4,
\url{https://digitalcommons.lsu.edu/jcdl/vol1/iss1/1}

Berry, Dorothy (2021). \emph{Centering the Margins in Digital Project
Planning}. In: Journal of Critical Digital Librarianship 1 (2021) 1:
15--22, \url{https://digitalcommons.lsu.edu/jcdl/vol1/iss1/3}

Wernimont, Jacqueline (2021). \emph{Listening, Care, and Collections as
Data}. In: Journal of Critical Digital Librarianship 1 (2021) 1: 23--42,
\url{https://digitalcommons.lsu.edu/jcdl/vol1/iss1/4}

Torres, Alejandra (2021). \emph{Using Digital Libraries to Engage the
Whole Student: Culturally Sustaining Pedagogies, Trauma-Informed
Classrooms, and Project-Based Learning}. In: Journal of Critical Digital
Librarianship 1 (2021) 1: 43--55,
\url{https://digitalcommons.lsu.edu/jcdl/vol1/iss1/5}

Das \emph{Journal of Critical Digital Librarianship} ist eine neue
OA-Zeitschrift, deren Programm im Titel angegeben ist.
\enquote{Critical} heisst im US-amerikanischen Kontext jede Theorie,
welche über gesellschaftliche Strukturen, deren Reproduktion und
Auswirkung nachdenkt. Dies kann sich auf rassistische, sexistische,
ökonomische aber auch weitere Strukturen beziehen, welche ein Auswirkung
darauf haben, wie in der Gesellschaft Ressourcen, Chancen und
Handlungsmöglichkeiten verteilt sind. Im Journal sollen nun Beiträge
erscheinen, welche dieses Nachdenken über Strukturen auf Bibliotheken
und deren digitale Angebote übertragen. Die drei Herausgeberinnen
betonen in ihrem Editorial, dass für die kritische Bearbeitung dieser
Themen auch schon andere bibliothekarische Zeitschriften zur Verfügung
stehen, sie aber einen Ort bieten möchten, an dem auf das
\enquote{digitale Bibliothekswesen} fokussiert werden kann.

Die Beiträge in der ersten Ausgabe sind allesamt direkt eingeladen
worden (beziehungsweise ein Interview) und berichten direkt aus der
bibliothekarischen Praxis. Insoweit lässt sich aus ihnen noch nicht
schliessen, was an Themen sich in Zukunft etablieren wird. Gemeinsam ist
ihnen aber bisher allen, dass sie betonen, dass auch die digitalen
Angebote von Bibliotheken nicht in einem Raum frei von Strukturen
entstehen würden, sondern das Ergebnis von zahlreichen Entscheidungen
sind, welche gerade durch die genannten Strukturen geprägt waren und
sind. Die Auswirkungen dieser Strukturen -- beispielsweise dazu, was
gesammelt, digitalisiert oder wie bearbeitet wird -- haben
längerfristige Wirkung und können zum Beispiel wieder dazu beitragen,
Strukturen zu reproduzieren. Es gäbe keine einfache Lösung, sondern es
gälte, Veränderung langfristig zu denken und umzusetzen sowie emphatisch
am Verstehen und dann Auflösen der kritisierten Strukturen zu arbeiten.
(ks)

\begin{center}\rule{0.5\linewidth}{0.5pt}\end{center}

Currie, Amy ; Kilbride, William (2021). \emph{FAIR Forever?
Accountabilities and Responsibilities in the Preservation of Research
Data}. In: International Journal of Digital Curation 16 (2021) 1: 1--16,
\url{https://doi.org/10.2218/ijdc.v16i1.768}

Auch wenn dieses Paper wie eine Studie daherkommt, sollte es eher als
Policy Paper verstanden werden. Eine Arbeitsgruppe der European Open
Science Cloud formuliert hier, basierend auf Interviews, Fokusgruppen
und der Lektüre relevanter Dokumente, Forderungen im Bezug auf den
Umgang mit Forschungsdaten. Grundsätzlich geht es darum, die Einhaltung
der FAIR-Prinzipien auch für die Langzeitarchivierung einzufordern und
nicht nur für den Projektzeitraum. Hierfür müssten Infrastrukturen und
Personal zur Verfügung gestellt werden.

Das ist bestimmt richtig, aber selbstverständlich ist die European Open
Science Cloud auch selber die Einrichtung, welche für die beteiligten
Länder genau dies ermöglichen soll. Insoweit scheint das Papier vor
allem der Produktion von Argumenten und eines Arbeitsplanes für die
Cloud darzustellen. (ks)

\begin{center}\rule{0.5\linewidth}{0.5pt}\end{center}

Rod, Alisa B. ; Isuster, Marcela Y. ; Chandler, Martin (2021).
\emph{Love Data Week in the time of COVID-19: A content analysis of Love
Data Week 2021 events}. In: The Journal of Academic Librarianship 47
(2021): 102449, \url{https://doi.org/10.1016/j.acalib.2021.102449}

Angetrieben von der Frage, wie Wissenschaftliche Bibliotheken
(potentiell) weltweit die je im Februar um den Valentinstag herum
durchgeführte «Love Data Week» -- in welcher vor allem
Hochschulbibliotheken versuchen, auf Angebote und Fragen rund um das
Forschungsdatenmanagement aufmerksam zu machen -- während der COVID-19
Pandemie angingen, sammelten die Autor*innen von möglichst vielen
Homepages Angaben dazu und werteten sie mittels mehrfacher Durchgänge
der inhaltlichen Codierung aus. Selbstverständlich sind die Ergebnisse
darauf beschränkt, welche Beiträge sie überhaupt fanden und vielleicht
auch, welche sie sprachlich verstehen konnten. (Aus dem DACH-Raum finden
sich in den Daten die HU Berlin und die EPFL Lausanne. Hochschulen aus
anderen Sprachräumen als Englisch, Französisch, Deutsch fanden sich
nicht.)

Die Ergebnisse zeigen zum Ersten, eine Liste von Themen, welche
angesprochen wurden und zum Zweiten, eine Liste der Formen von
Veranstaltungen. Es gab eine Diversität von Themen und Angeboten, aber
auch klare Tendenzen. Wenig überraschend ist wohl, dass es möglich war,
Veranstaltungen Online anzubieten. Die meisten Veranstaltungen
beschäftigten sich mit «Product/Service Awareness» (22.73\,\%) und
«Research Data Management» (19.83\,\%) im Allgemeinen. Weiterhin ging es
auch oft um spezifische Tools. Auch wurden vor allem Workshops
durchgeführt (57.9\,\%). Selbst wenn die Autor*innen es etwas anders
darstellen, drängt sich doch der Eindruck auf, dass die «Love Data Week»
vor allem eine Marketing-Veranstaltung ist, in der Bibliotheken
vorstellen, was sie an Angeboten aufgebaut haben. (Was grundsätzlich
nicht zu kritisieren ist.) Anderes kommt zwar vor -- beispielsweise die
Themen «Data for Black Lives», «Indigenous Data» oder «Data Feminism»
--, aber immer nur am Rand. (ks)

\begin{center}\rule{0.5\linewidth}{0.5pt}\end{center}

Bishop, Bradley Wade (2022). \emph{Data Services Librarians'
Responsibilities and Perspectives on Research Data Management}. In:
Journal of eScience Librarianship 11 (2022) 1: e1226,
\url{https://doi.org/10.7191/jeslib.2022.1226}

Forschungsdatenmanagement und dazugehörige Services liegen seit einigen
Jahren im Aufgabengebiet einer wachsenden Anzahl von Bibliotheken. Die
Interviewstudie, von welcher in diesem Text berichtet wird, versuchte zu
erfassen, was dies in der tatsächlichen Arbeit von Bibliotheken
bedeutet. Befragt wurden zehn Bibliothekar*innen an grossen
Hochschulbibliotheken in den USA, welche in diesem Bereich arbeiten.
Dargestellt werden die Ergebnisse fast direkt verbatim, also auch ohne
grosse Interpretation.

Die Autorin zeigt, dass in der Literatur eine ganze Reihe von Aufgaben
angedacht werden, welche für Bibliotheken möglich wären. In der Realität
finden sich aber vor allem Training, Beratung und Outreach, die Review
von Research-Data-Management-Plänen und die Lokalisierung von schon
vorhandenen Daten für Forschende. Data Curation, der Betrieb von
Repositories oder gar die Arbeit (Analyse, Visualisierung) von Daten
passiert kaum. Keine der befragten Bibliothekar*innen hat eine
Ausbildung für ihre Arbeit genossen, sondern sich das notwendige Wissen
während der Arbeit angeeignet. Weiterhin betont die Autorin, dass es
sich bei den Befragten um Angehörige grosser und damit ressourcenstarker
Einrichtungen handelt. Für kleinere Einrichtungen wären die Spielräume
und damit wohl auch die tatsächlich möglichen Services geringer. (ks)

\hypertarget{monographien-und-buchkapitel}{%
\section{3. Monographien und
Buchkapitel}\label{monographien-und-buchkapitel}}

\hypertarget{vermischte-themen-1}{%
\subsection{3.1 Vermischte Themen}\label{vermischte-themen-1}}

Magerski, Christine ; Karpenstein-Eßbach, Christa (2019).
\emph{Literatursoziologie: Grundlagen, Problemstellungen und Theorien.}
{[}Lehrbuch{]} Wiesbaden: Springer VS {[}gedruckt{]}

Literatursoziologie wird in diesem Überblickswerk vor allem als
Soziologie der Literatur verstanden, also der konkreten, publizierten
Literatur und ihrer Beziehungen zu gesellschaftlichen Entwicklungen.
Fragen der Institutionalisierung von Einrichtungen im Literaturbetrieb
oder ‒ was Bibliothekar*innen wohl eher interessieren würde ‒ der
Verbindung von ästhetischen Urteilen und sozialen Schichten ‒ ergo: Wer
liest was wofür? Wer bewertet Literatur wie und was hat dies mit der
sozialen Position dieser Person zu tun? ‒ sind dabei nur Unterthemen,
welche jeweils in einem Kapitel verhandelt werden.

Das Buch liefert eine Übersicht dazu, was an Literatur soziologisch
angegangen werden kann und auch schon angegangen wurde. Es liefert keine
konkreten Daten, die man sich vielleicht erhoffen würde, sondern vor
allem Forschungsfragen. Gezeigt wird, dass die Literatursoziologie in
Wellen betrieben wurde, vor allem Anfang des 20. Jahrhunderts und in den
1970er Jahren. Aktuell befindet sie sich nicht in einer Hochphase.

Irritierend an dem Buch ist, dass es explizit als Lehrbuch bezeichnet
wird, aber eher einen Fliesstext darstellt. Man würde von Lehrbüchern
eher klare Einführungen, Modelle und so weiter erwarten, die auch
schrittweise durchgearbeitet werden könnten. Das ist hier nicht der
Fall. Man merkt auch das Herkommen der Autorinnen aus der
Literaturwissenschaft: Wie in dieser werden immer wieder Romane oder
andere literarische Texte als Beispiele herangezogen, aber mit dem
Verständnis, dass diese den Leser*innen schon bekannt wären oder halt
nachgelesen werden. Und letztlich ist die Literatursoziologie, die hier
präsentiert wird, vor allem die aus dem deutschsprachigen Raum. Ausser
dann, wenn es sich nicht umgehen lässt, werden nur Autor*innen aus
diesem Sprachraum angeführt. (ks)

\begin{center}\rule{0.5\linewidth}{0.5pt}\end{center}

Cello, Serena (2019). \emph{La littérature des banlieues. Un engagement
littéraire contemporain.} Canterano: Aracne editrice {[}gedruckt{]}

Die Banlieues ‒ also die von Hochhäusern geprägten, verkehrstechnisch
und sozial schlecht angebundenen Wohngebiete am Rand französischer
Grossstädte ‒ sind ein Ort sozialer Imagination der französischen
Gesellschaft, die diese als nationale Spielart von US-amerikanischer
Ghettos wahrnimmt, inklusive der Ausprägung eigener Kulturen und
Lebensweisen der dort Lebenden und gleichzeitig aber auch ein Ort, an
dem Menschen leben. Oft sozial ausgegrenzte. Und sie sind
selbstverständlich ein Ort, an dem Literatur oder andere Formen von
Kultur ‒ gerne wird, unter anderem in dieser Studie, der französische
Rap angeführt ‒ produziert wird.

Die kurze, unterhaltsam zu lesende Studie von Cello nimmt zwölf Romane,
die von Schriftsteller*innen aus Banlieues stammen, zudem in diesen
spielen und alle im Jahr 2006 erschienen, zur Basis, um zu fragen, was
diese gemeinsam auszeichnet, aber auch wie sie in der französischen
Literatur und Gesellschaft zu verorten sind. Cello zeigt, dass die
eigenen linguistischen Formen, die sich im Leben im Banlieu entwickelt
haben, genauso wie der konkrete Lebensort Banlieu, sich in den Romanen
widerspiegeln. Allerdings immer in direktem Bezug zur französischen
Gesellschaft. Die linguistischen Formen sind von den Sprachen geprägt,
die bei der Migration der Elterngeneration nach Frankreich mitgebracht
wurden, aber auch von den Soziolekten der französischen Unterschichten
der letzten Jahrhunderte (\enquote{argot}), der französischen
Jugendsprache (vor allem dem \enquote{verlan}) und der digitalen
Kommunikation. Der Lebensort Banlieu ist davon geprägt, dass die dort
lebende Bevölkerung anderswo ausgegrenzt ist, aber im Banlieu
\enquote{ihren eigenen Ort} findet, den sie eigenständig gestaltet.

Das Gleiche zeigt Cello für die Romane und der Wahrnehmung derselben in
der französischen Kultur: Sie sind zum Beispiel selber geprägt von
Romanen von Schriftsteller*innen, die in den 1990er Jahren das Leben von
Migrant*innen in Frankreich zeigten, obgleich die Schreibenden der
\enquote{neuen Welle}, die Cello untersucht, fast keine eigenen
Migrationserfahrungen haben, sondern seit ihrer Geburt in Frankreich
leben. Die Interpretation der Romane geschieht zudem immer unter dem
Blickwinkel der engagierten französischen Literatur. Kurz gesagt: In
gewisser Weise werden sie immer anhand des Rahmens interpretiert, den
zum Beispiel Zola oder Sartre etabliert haben ‒ aber die
Schriftsteller*innen verorten sich auch selber in diesem Rahmen, weil
sie halt trotz aller Ausgrenzungen und intellektuellen Grenzziehungen,
Teil der französischen Gesellschaft sind. (ks)

\begin{center}\rule{0.5\linewidth}{0.5pt}\end{center}

Stachokas, George (2020). \emph{The Role of the Electronic Resources
Librarian}. (Chandos Information Professional Series) Cambridge ;
Kidlington: Chandos {[}gedruckt{]}

Charakteristisch für Bücher, welche in dieser Reihe erscheinen, ist,
dass die Buchtitel und Abstracts oft etwas anderes versprechen, als dann
wirklich Hauptthema der Publikation ist. So auch bei diesem: Vermuten
könnten man, dass es hauptsächlich darum geht, zu bestimmen, was
Electronic Resources Librarians tun und welche Funktion sie in ihren
Einrichtungen haben. Das wird aber nur im letzten Kapitel mittels einer
Auswertung von öffentlich zugänglichen Job- und Aufgabenbeschreibungen
angegangen. Hauptsächlich ist das Buch eine Darstellung der Entwicklung
wie Wissenschaftliche Bibliotheken in den USA seit den 1990er Jahren
elektronische Medien im Bestandsmanagement behandelt haben. Das Buch
beginnt mit Datenbanken, sowohl auf CD-Rom als auch elektronisch
vermittelt, und endet bei Online-Datenbanken, E-Books und elektronischen
Zeitschriften.

Der Aufbau ist dabei chronologisch. Es geht nicht um die
Technikentwicklung per se, auch wenn diese mit dargestellt wird, sondern
darum, wie Bibliotheken die Aufgaben, die sich für sie mit diesen Medien
stellten, umgingen. Der Autor hat dabei einen recht fatalistischen
Blick: Er beschreibt die Entwicklung, wie sie sich bis heute ergeben
hat, als folgerichtig und alle Diskussionen, Abweichungen, anderen
Versuche von Bibliotheken, mit diesen Medien umzugehen ‒ die eigentlich
in ihrer jeweiligen Zeit ihre Berechtigung hatten ‒ als fehlgeleitet.
Ausserdem ist er, wie er auch selber erwähnt, auf die USA fokussiert.
Der Text besteht auch zu grossen Teilen darin, dass der Autor die seiner
Meinung nach wichtigen Entwicklungen in der jeweiligen Zeit
zusammenfasst und dann lange Daten aus jeweils zeitgenössischen Artikeln
aus der bibliothekarischen Fachliteratur anführt. Das ist nicht immer
hilfreich, weil nicht klar ist, was die prozent-genauen Angaben aus
alten Umfragen oder die Preisangaben aus anderen Jahrzehnten zum
Verständnis des Themas beitragen. Es ist nie klar ersichtlich, warum der
Autor bestimmte ältere Texte ausgewählt hat, um sie darzustellen. Er
verortet sie auch nicht in der jeweiligen Diskussion.

Und dennoch hat das Buch seine interessanten Seiten. Es zeigt, dass die
Technikentwicklung immer mit einer Entwicklung der Aufgaben von
Bibliotheken, aber auch der internen Struktur von Bibliotheken verbunden
ist. Oft wurden erst besondere Abteilungen oder Einzelstellen für die
Handhabung bestimmter Medien geschaffen, um diese dann später in
grössere Abteilungen zu integrieren. (ks)

\begin{center}\rule{0.5\linewidth}{0.5pt}\end{center}

Bardola, Nicola ; Hauck, Stefan ; Jandrlic, Mladen ; Rak, ALexandra ;
Schäfer, Christoph ; Schweikart, Ralf (2020). \emph{Wie Kinder Bücher
lesen. Mehr als ein Wegweiser}. Hamburg: Carlsen Verlag {[}gedruckt{]}

Zugegeben, als der Rezensent die Fernleihbestellung für dieses Buch
aufgab, hoffte er darauf, eine einfach geschriebene Übersicht zum Stand
der Forschung über das Lesen von Kindern zu erhalten. Beim ersten Blick
auf das gelieferte Buch war dann klar, dass es das nicht ist. Schon der
Aufbau des Textes, das Layout (viele kurze Abschnitte, viele
Überschriften, viele kurze Einschübe, alles sehr bunt) und die
durchgängig im ganzen Buch vorhandenen Illustrationen (von Regina Kehn)
vermitteln eher den Eindruck eines Jugendsachbuchs.

Aber auch das ist es nicht. Es ist ein schlechtes Buch und wenn es
tatsächlich den Stand der Leseforschung und anderer Schriften zum Lesen
von Kindern im DACH-Raum darstellt, dann wäre in diesem Bereich einiges
zu verbessern. So ist zum Beispiel nie klar, an wen sich das Buch
richtet: Mal geht es in einer Übersicht darum, wie ein Kinderbuch
gestaltet sein soll oder wie Verlage Reihen planen ‒ aber es richtet
sich nicht an Verlage. Mal geht es darum, welche Kinder- und
Jugendbuchreihen existieren ‒ aber es richtet sich auch nicht wirklich
an Buchhändler*innen oder Bibliothekar*innen. Mal wird darüber
gesprochen, was Eltern tun sollten ‒ aber es richtet sich auch nicht an
Eltern. Und oft werden irgendwelche Bonmots über das Lesen oder kleine
Geschichten zusammengetragen. Wozu, wird nie klar. Und das alles, wie
gesagt, in einem Stil wie ein Jugendsachbuch ‒ aber an Jugendliche
richtet es sich auch nicht.

Zudem ist das Buch inhaltlich unausgewogen und oft auch widersprüchlich.
Beispielsweise wird das Lesen an elektronischen Geräten teilweise
regelrecht verteufelt. Ausser, wenn sie als innovativ bezeichnet werden
können. Es wird an sich viel gegen Handys gesagt, aber die Daten über
die Mediennutzung, die am Anfang des Buches ‒ illustriert ‒ zitiert
werden, zeigen, dass das Fernsehen das Medium Nummer eins für Kinder und
Jugendliche ist und das auch noch mehr Medien genutzt werden. Das
interessiert das ganze Buch über nie. Zudem haben die Autor*innen auch
fast das gesamte Buch über eine Fixierung darauf, zu behaupten, dass es
einen Trend dazu gäbe, geschlechtlich diverse Medienangebote zu machen,
aber dass dies falsch wäre, weil angeblich Jungen Jungen und Mädchen
Mädchen sein wollen und das halt so sei. An sich besteht das Buch aus
vielen Behauptungen, die selten überhaupt begründet werden. Negativ
fällt auch auf, wie oft behauptet wird, \enquote{die Leseforschung}
würde dieses oder jenes sagen, aber dann Buchhändler*innen oder
Autor*innen namentlich als Expert*innen für das Lesen von Kindern (was
ja gar nicht ihr direkter Fokus ist) interviewt werden, als wäre das
strukturiert erarbeitete Wissen der Forschung weniger relevant als das
der Personen, die ja Bücher auch verkaufen müssen.

Das ganze Buch vermittelt den Eindruck, als wären die Autor*innen
irgendwie gedrängt, alles mögliche zum Thema loswerden zu wollen ‒ aber
immer eher gefühlt, als irgendwie inhaltlich untermauert. Alle Zitate
oder Fakten, die angeführt werden, erscheinen so auch nur danach
ausgesucht, dass sie die Aussagen der Autor*innen unterstützen. Wirklich
etwas über das Lesen von Kindern ‒ ein Thema, dass für Bibliotheken
relevant wäre ‒ erfährt man hier nicht. (ks)

\begin{center}\rule{0.5\linewidth}{0.5pt}\end{center}

Wilson, Robert ; Mitchell, James (2021). \emph{Open Source Library
Systems: A Guide}. Lanham ; Boulder ; New York ; London: Rowman \&
Littlefield. {[}gedruckt{]}

Das Buch ist eine kurze Darstellung existierender Open Source Software,
welche (grösstenteils) speziell für den Einsatz in Bibliotheken gedacht
sind. Die Software wird jeweils dargestellt anhand ihrer Geschichte, der
Einsatzmöglichkeiten und zum Teil auch der Community um diese Software.
Wenn vorhanden, werden auch Anbieter vorgestellt, welche die jeweilige
Software für Bibliotheken hosten. Einige kurze Interviews mit Aktiven
aus den Communities ergänzen diese Darstellungen. Mit Kapiteln zu ILS;
Digital Repositories, Discovery Systems, Resource Sharing (Fernleihe)
und Electronic Resource Management deckt das Buch auch die gängigsten
Bereiche des Bibliotheks- und Bestandsmanagements ab. Eingerahmt ist
dies in einer kurzen Darstellung der Entwicklung von Open Source
Software und einem Aufruf der Autoren, in Bibliotheken mehr davon
einzusetzen. Sie gehen auch auf Argumente ein, die in Bibliotheken gegen
diesen Einsatz vorgebracht werden und liefern Gegenargumente.

Während das alles sympathisch ist, darf das Buch nicht als vollständige
Sammlung verstanden werden. Es ist immer ein US-spezifischer Blick, der
im europäischen Raum wohl ergänzt werden müsste. Das französische Open
Source ILS PMB kommt zum Beispiel genauso wenig vor, wie das
schweizerische Rero ILS. Das heisst nicht, dass die Autoren nicht auch
Beispiele von ausserhalb der USA anführen oder Engagierte aus anderen
Ländern interviewen würden. Aber eine Software muss in den USA
eingesetzt werden, um für sie relevant zu sein. Zudem könnte man
selbstverständlich weitere Kategorien an Software ergänzen, die in
einigen Bibliotheken genutzt wird, beispielsweise solche für
statistische und bibliometrische Analysen. (ks)

\begin{center}\rule{0.5\linewidth}{0.5pt}\end{center}

Sekretariat der Deutschen Bischofskonferenz (Hrsg.) (2021).
\emph{Katholische Büchereiarbeit. Selbstverständnis und Engagement}.
(Arbeitshilfen, 324) Bonn: Sekretariat der Deutschen Bischofskonferenz,
\url{https://www.dbk-shop.de/media/files_public/d9d2895a238a6c5d03453dc849c201d6/DBK_5324.pdf}

Die Arbeit der katholischen Büchereien in Deutschland geschieht oft
relativ unbeachtet neben dem Öffentlichen Bibliothekswesen. Die
Büchereien nehmen an der Bibliotheksstatistik teil und übernehmen auch
im Auftrag von Gemeinden die Aufgaben Öffentlicher Bibliotheken, aber
daneben haben sie mit dem Borromäusverein und dem Sankt Michaelsbund,
deren Publikationen, Aus- und Weiterbildungen, Besprechungsdienst und
Fachstellen bei den Erzbistümern, ihre eigenen Strukturen. Historisch
entstanden in der Zeit des \enquote{Kirchenkampfes}, stellen sie heute
eine Seite der kirchlichen Arbeit dar und sind personell getragen von
Ehrenamtlichen.

Das Papier gibt als aktuellste Stellungnahme der Bischofskonferenz zu
diesem Bereich einen kurzen Überblick der Arbeit dieser Büchereien. Es
ist weniger Handreichung, als vielmehr eine Wertschätzung der Büchereien
durch die Bischofskonferenz. Sie werden zuerst in den Rahmen einer sich
als offen und modern verstehenden Kirche im Rahmen der Gedanken des
Zweiten Vatikanischen Konzils gestellt und anschliessend in ihrer Arbeit
beschrieben. Sie werden dabei als Teil der Bildungsarbeit verstanden,
welche aus dem Verkündigungsauftrag der Bibel und dem Streben nach dem
Erhalt der Schöpfung, wie er im Zweiten Vatikanischen Konzil als Aufgabe
der Kirche bestimmt wurde. Für Nichtkatholik*innen ist das Papier auch
ein kurzer Einblick in eine gewisse Subkultur. Die Darstellung der
Aufgaben der Büchereien ist eingebunden in eine Deutung der Gesellschaft
als ständig in rasanter, potentiell gefährlicher Veränderung begriffen.
In dieser Gesellschaft würde an festen Strukturen fehlen,
Individualisierung wäre überall vorherrschend und in gewisser Weise auch
bedrohlich. Die Kirche und die Büchereien würden demgegenüber helfen,
Halt zu finden. Es ist eine leicht befremdliche Sicht auf die
Gesellschaft und auf die Kirche selber, obgleich die geleistete Arbeit
der Büchereien, die im Dokument ebenfalls beschrieben wird, beachtlich
ist. (ks)

\begin{center}\rule{0.5\linewidth}{0.5pt}\end{center}

Soulas, Christine (dir.) (2017). \emph{(Ré)aménager une bibliothèque}.
(La boîte à outils, 42) Villeurbanne Cedex: Presse de l'Enssib.
{[}gedruckt{]}

Die Reihe, in der dieses Buch erschienen ist, will Bibliotheken in
Frankreich praktische Handreichungen für die konkrete Bibliotheksarbeit
zur Verfügung stellen. Hier, in diesem Band, geht es vor allem um das
Einrichten und den Umbau von Innenräumen, oft anhand konkreter
Beispiele. Teilweise wird besprochen, wie man Möbel ausgewählt,
Planungsprozesse mit Architekt*innen und der Bürokratie (inklusive der
gesetzlichen Grundlagen, was das Buch spezifisch französisch macht, da
diese selbstverständlich anderswo so nicht gelten) werden dargestellt,
aber auch Überlegungen dazu, wie Nutzer*innen einbezogen werden können.
Nicht immer scheinen die einzelnen Themen zu den kurzen Darstellungen
(keine mehr als zehn A5-Seiten) zu passen. Ein wenig irritierend ist
auch, dass das Buch ‒ wohl, weil dies dem Layout der Reihe folgt ‒ bei
diesem Thema fast gänzlich ohne Illustrationen oder Skizzen auskommt.

Interessant für den DACH-Raum ist das Buch wegen der Unterschiede.
Vieles, was man von so einem Buch erwarten würde, findet sich auch hier.
Der Dritte Ort wird besprochen, Formen der Einbeziehung von
Nutzer*innen, die Notwendigkeit mit Architekt*innen zusammenzuarbeiten.
Aber nicht unbedingt so prominent, wie das im DACH-Raum passieren würde.
Was als Thema hervorsticht, ist einerseits ein Fokus auf das Lesen, die
Leseförderung und den Kampf gegen den Analphabetismus (\enquote{lutte
contre l'illettrisme}). Und andererseits, aber damit zusammenhängend,
ein Fokus auf die Frage, wie unterschiedliche Bevölkerungsgruppen,
gerade Ausgegrenzte, erreicht werden können. (ks)

\begin{center}\rule{0.5\linewidth}{0.5pt}\end{center}

Pyne, Lydia (2021). \emph{Postcards. The rise and fall of the world's
first social networ}k. London: Reaktion Books {[}gedruckt{]}

In ihrer sehr empfehlenswerten Auseinandersetzung mit der Geschichte der
Post- und Ansichtskartenkultur widmet sich Lydia Pyne auch der
Ansichtskartensammlung der New York Public Library (S. 174--186). Sie
ist Teil der Bildersammlung des Hauses, die wiederum 1915, also in der
Noch-Hochphase des Mediums (\enquote{golden age of postcards}) begründet
wurde. Interessanterweise kann man sich einzelne Karten wie auch andere
Medieneinheiten entleihen. Gebrauch machen davon offenbar besonders gern
Studierende aus dem Bereich Mode und Design. Außerdem schauen zufällige
Besucher*innen der Bibliothek gern in die Schubladen, in denen sich die
Karten grob geographisch sortiert finden und suchen,
\enquote{ausnahmslos}, wie die die leitende Bibliothekarin Jessica Cline
zitiert wird, sofort nach Karten aus ihren Herkunftsorten. Der
Bestandsaufbau erfolgt allerdings nicht sonderlich systematisch, sondern
beruht vorwiegend auf Spenden von Sammelnden und von Mitarbeitenden des
Hauses, die von Reisen und Ausflügen Karten für die Sammlung mitbringen.
Lydia Pyne integriert diese Bibliotheksepisode in die Auseinandersetzung
mit der Rolle von Ansichtskarten als Zeugnis geopolitischer
Verschiebungen. Das macht sich auch bei klassifikatorischer Erschließung
und Präsentation der Ansichtskarten in der Bibliothek bemerkbar, wie die
Autorin am Beispiel der Herausforderung einer eindeutigen Einordnung von
Karten mit Motiven der Krim beschreibt. Ähnliches gilt für sich ändernde
Benennungen, wie Lydia Pyne am Beispiel \enquote{Sri Lanka} ausführt.
Aus Sicht der Philokartie und der Kommunikationsphilosophie ist ein
weiterer von ihr ausgeführter Gedanke ebenfalls interessant: die einst
postalisch zirkulierenden Karten werden über die Verfügbarmachung im
Bibliotheksbestand in potentiell unendliche weitere Zirkulationsprozesse
eingespeist. (bk)

\begin{center}\rule{0.5\linewidth}{0.5pt}\end{center}

\pagebreak

Robertson, Guy (2021). \emph{Disaster planning for special libraries.}
Oxford: Chandos Publishing {[}gedruckt{]}

Während es schon viele gute Publikationen zum Thema Notfallplanung in
Bibliotheken gibt, so gehen doch die wenigsten davon auf die speziellen
Bedürfnisse und Umstände von Spezialbibliotheken ein. Die Publikation
von Guy Robertson schließt diese Lücke sehr gut. Er verbindet die
Belange und Situation der Spezialbibliothek immer mit der oft
vorhandenen übergeordneten Einrichtung (sei es Firma, Museum, Verband
oder ähnliches) und thematisiert auch Kommunikationsstrategien,
Synergien und mögliche Problemstellen dieser Konstellation.

Insgesamt fokussiert sich das Werk klar auf die im Titel angesprochene
Planung. Es werden also nicht die besten Behandlungsmöglichkeiten bei
Wasserschäden erläutert oder Methoden der Entschimmelung verglichen.
Vielmehr liegt die Intention ganz klar auf dem organisatorischen
Planungsprozess. Hier wird Wert auf ein strukturiertes Vorgehen und
umfassende vorbeugende Maßnahmen gelegt, die auch Situationen umfassen,
die man im ersten Moment vielleicht nicht mit speziell
bibliothekarischer Notfallplanung verbindet -- wie etwa Diebstähle,
Datenverlust, Stromausfall oder der längerfristige Ausfall von
Kernpersonal.

Positiv fällt auch auf, dass die einzelnen Themen oft mit Zitaten aus
der Praxis ergänzt werden, womit trockene Theorie anschaulicher wird.
Außerdem werden ausgiebig emotionale Reaktionen der Beteiligten
diskutiert, die die Handlungsfähigkeit im Ernstfall oder auch im
Nachgang eines Notfalls beeinträchtigen können. Bibliothekar*innen sind
nunmal keine Roboter, die immer ohne zu zögern und strickt nach
Notfallplan handeln.

An so mancher Stelle würde man sich etwas mehr kapitelübergreifende
Struktur oder verbindende Elemente im Text wünschen. Auch die
Detailtiefe variiert manchmal an etwas unerwarteten Stellen. Die
gewählten Beispiele haben in der Gänze einen durchaus
angloamerikanischen Charakter und sind nicht immer auf
DACH-Gegebenheiten anwendbar -- dies liegt aber auch in der Natur von
Spezialbibliotheken, die ja einfach sehr heterogen gestaltet sind.
Insgesamt ist es also ein durchaus nützliches Buch, das viele Fragen
aufwirft, die man im eigenen Planungsprozess zu beantworten suchen muss.
(eb)

\begin{center}\rule{0.5\linewidth}{0.5pt}\end{center}

Zakaria, Rafia (2022). \emph{Against white Feminism. Wie weisser
Feminismus Gleichberechtigung verhindert.} München: Hanserblau
{[}gedruckt{]}

Das Buch \emph{Against white Feminism} kann als Streitschrift für den
Feminismus und seine umwälzende Kraft gelesen werden. Rafia Zakaria geht
es darum, mit dem Mythos des \emph{weissen} Feminismus als Ursprung,
Treiber und alleiniger Kampf von Feminist*innen aus dem globalen Norden
aufzuräumen. Sie macht klar, dass der \emph{weisse} Feminismus nicht
unbedingt (aber meistens) etwas mit der Hautfarbe der Protagonist*innen
zu tun hat, sondern damit wie Feminismus definiert wird, welche Personen
durch ihn sichtbar oder unsichtbar gemacht werden und wen er
voranbringt. \emph{Weisser} Feminismus ist, laut Zakaria, ein
Feminismus, der nur wenigen Individuen zugute kommt (meistens
hochgebildete, \emph{weisse} Frauen aus der oberen Mittelschicht) und
diese auf der beruflichen Karriereleiter nach oben steigen lässt. Dabei
wird nicht darauf geachtet, dass sich für die Mehrheit der Frauen und
anderen marginalisierten Gruppen nichts ändert und dass das patriarchale
System auf diese Art weiter gestützt wird. Rafia Zakaria zeigt auf, in
wie vielen Teilen der Welt gleichzeitig zum \emph{weissen} Feminismus
(oder schon davor), Frauen und Verbündete für ihre Anliegen gekämpft und
feministische Konzepte entwickelt haben. Diese wurden aber von dem
\emph{weissen} Feminismus entweder ignoriert, absorbiert oder sogar
bekämpft. Die Autorin zeigt in ihrem Buch die vielschichtigen und
unterschiedlichen Gruppen, die sich der feministischen Bewegung
verschrieben haben und fordert dazu auf, die Unterschiede sowie die
Gemeinsamkeiten anzuerkennen und so die Grundlage zu schaffen
\emph{weisse} Machtstrukturen zu demontieren. Es sollen Räume geschaffen
werden, in denen alle zusammenkommen können, um zu diskutieren, sich
auszutauschen und um sich auf das gemeinsame Ziel zu besinnen, der Kampf
für die Gerechtigkeit für Alle. (sj)

\hypertarget{bibliotheksgeschichte}{%
\subsection{3.2 Bibliotheksgeschichte}\label{bibliotheksgeschichte}}

Manley, Keith A. (2018). \emph{Irish Reading Societies and Circulating
Libraries founded before 1825. Useful knowledge and agreeable
entertainment}. Dublin: Four Courts Press {[}gedruckt{]}

Wie im Titel angegeben, geht es in diesem Buch um Bibliotheken, die bis
1825 in Irland gegründet und betrieben wurden, also bevor sich die Idee
etablierte, dass Öffentliche Bibliotheken steuerfinanziert sein und
professionell betrieben werden sollten. Es beginnt mit der Wende vom 17.
zum 18. Jahrhundert. Für Personen, die sich mit der irischen Geschichte
auskennen, ist ersichtlich, dass es sich dabei um formative Jahrzehnte
handelte. Das Entstehen der katholisch-nationalistisch irischen
Befreiungsbewegung fällt in diese Zeit, ebenso massive Umgestaltungen
der Gesellschaft durch Industrialisierung, aber auch
Auswanderungsbewegungen. Ein Problem des Buches ist, dass dieses Wissen
grundsätzlich vorausgesetzt wird. Beispielsweise wird auf die Reading
Societies eingegangen, aber ihre Bedeutung als Institutionen, an denen
sich die Befreiungsbewegung etablierte, wird nur kurz angedeutet. So ist
für Personen, die das Buch aus rein bibliotheksgeschichtlichem Interesse
lesen, wenig klar, warum diese Societies zeitweise verfolgt wurden.

Neben diesem Grundwissen über irische Geschichte, das vorausgesetzt
wird, stellt das Buch das Ergebnis einer offenbar jahrelangen,
intensiven Recherche ‒ aus der auch weitere Arbeiten des Autors
entstanden sind ‒ dar. Unzählige Hinweise auf die betreffenden
Bibliotheken wurden versammelt: Hinweise und Werbeanzeigen in Zeitungen,
Archivquellen, Erwähnungen in Erzählungen, Reiseberichten, privaten
Briefen und so weiter. Diese sind im Anhang auch nachgewiesen.

Im Haupttext werden sie, getrennt nach verschiedenen Bibliothekstypen,
dargestellt. Der Autor versucht jeweils eine übergreifende Darstellung,
aber oft ist dies doch eine Nacherzählung der Quellen selber, oft mit
wenig Kontext und auch abhängig davon, wie detailliert diese Quellen
sind. Es werden beispielsweise häufig Geldwerte zitiert ‒ für
Leihgebühren, für Buchanschaffungen und so weiter ‒, ohne dass dargelegt
wird, wie viel Geld das jeweils im Vergleich zu anderen Preisen oder dem
Einkommen von Personen darstellt. Teilweise werden lange Passagen aus
Berichten und Briefen angeführt, teilweise sind es reine Listen von
Fundstellen.

Was aber aus dem Buch zu lernen ist, ist wie vielfältig die Landschaft
der Bibliotheken vor den Public Libraries war. Sowohl in ihren Zielen
(beispielsweise religiöse Erziehung, Selbsterziehung, Selbsthilfe von
Berufsvereinigungen, kommerzielle Ziele) als auch in ihren
Organisationsformen (beispielsweise als Angebot von Buchhandlungen oder
als eigenständige kommerzielle Unternehmen, als Selbsthilfevereine, als
religiöse oder andere Stiftungen für die Öffentlichkeit, als
philanthropische Unternehmen oder auch als Gegenstand von
Beziehungsnetzwerken zwischen Einzelpersonen). Was das Buch zeigen will,
ist, wie wichtig das Buch als Objekt auch vor dem 19. Jahrhundert in der
irischen Gesellschaft war ‒ und das gelingt. (ks)

\begin{center}\rule{0.5\linewidth}{0.5pt}\end{center}

Zalar, Jeffrey T. (2019). \emph{Reading and Rebellion in Catholic
Germany, 1770--1914.} (Publications of the German Historical Institute)
Cambridge ; New York ; Port Melbourne ; New Delhi ; Singapore: Cambridge
University Press {[}gedruckt{]}

Bibliotheksgeschichtlich relevant ist dieses Buch, da in ihm die
Geschichte des Borromäus-Vereins und der Volksbibliotheken, welche
dieser Verein in Deutschland als explizit katholische gründete, erzählt
und in den Zeitkontext bis zum Beginn des ersten Weltkrieges eingeordnet
wird. Fokus ist aber eigentlich etwas anderes, nämlich die Versuche der
katholischen Kirche und Elite, in einer Zeit des rasanten Wandels und
Bedeutungsverlustes kirchlicher Institutionen eine Art Lesedisziplin
aufrecht zu erhalten und dem ständigen Scheitern dieser Versuche. Sie
scheitern, so der immer wieder vom Autor gemachte Punkt, weil sich die
\enquote{einfachen} Katholik*innen nicht vorschreiben lassen, wie und
was sie zu lesen hätten.

In die untersuchte Zeit fallen die moderne \enquote{Leserevolution} ‒
also das massive Ansteigen der Alphabetisierungsrate im 19. Jahrhundert,
das Entstehen von Publikationsindustrien und neuer Medienformen wie
Zeitschriften oder Broschüren sowie die Veralltäglichung des Lesens als
normale Form der Freizeitbeschäftigung breiter Bevölkerungsschichten ‒
und damit die Abnahme der Möglichkeit der Kontrolle des Lesens
allgemein. Die Industrialisierung, welche auch zum Anstieg der Mobilität
der Bevölkerung führte und damit vormalige Strukturen untergrub sowie
der Bedeutungsverlust der katholischen Strukturen durch die
Säkularisierungen erst durch die napoleonische Regierung, dann den
preussischen Staat und später das Deutschen Reich, fallen ebenfalls in
diesen Zeitraum. Auf all dies reagierte die katholische Kirche in
Deutschland unter anderem mit Versuchen, eine katholische Kultur zu
behaupten. So wurden anfänglich eigene Regeln für das richtige Lesen
propagiert. Später wurde auf die Behauptung protestantischer Eliten im
preussischen Staat, später im Deutschen Reich, dass der Katholizismus an
sich zur Rückständigkeit der Bevölkerung führen würde sowie Vorwürfe im
Kulturkampf, die katholische Bevölkerung sei nicht national gesinnt
(sondern auf Rom hin orientiert) versucht, mit dem Aufbau einer eigenen
Kultur zu reagieren. Der Autor berichtet über diese Versuche und wie die
katholische Bevölkerung, aber auch viele Pfarrer selber, darauf immer
nur zum Teil so reagierten, wie von der Kirche gewünscht. Vielmehr zog
diese schnell mit der restlichen Bevölkerung und deren Lesepräferenzen
gleich, wenn die Voraussetzungen dafür geschaffen waren (beispielsweise
mussten erst Schulen etabliert und, nachdem unter Napoleon und Preussen
die meisten katholischen Universitäten geschlossen wurden, wieder neue
Universitäten eröffnet werden). Sie unterschieden sich dann meist von
vergleichbaren Personen darin, dass sie auch explizit katholische
Literatur ‒ Heiligenviten, Gebetsbücher und so weiter ‒ lasen, aber halt
nicht nur.

Die Bibliotheken, welche der Borromäus-Verein zuerst an den Kirchen
selber etablieren und von Pfarrern betrieben wissen wollte, waren eine
der wichtigsten Waffen der Kirche in diesem Kampf. So sollte
\enquote{das gute Buch} verbreitet werden. Der Autor zeigt, dass aber
auch dieser Kampf sich den wechselnden Interessen der katholischen
Bevölkerung beugen musste. Anstatt nur katholische Werke anzubieten,
mussten sie bald eine möglichst breite Literatur anbieten, wobei die
Belletristik im Mittelpunkt des Interesses stand.

Der Autor behauptet auch mehrfach, dass andere Historiker*innen diesen
Widerstand der katholischen Bevölkerung gegen die Versuche, ihr Lesen zu
kontrollieren, ignoriert hätten. Allerdings zeigt er das nicht
überzeugend. (ks)

\begin{center}\rule{0.5\linewidth}{0.5pt}\end{center}

Pettegree, Andrew ; Weduwen, Arthur der (2021). \emph{The Library: A
Fragile History}. London: Profil Books {[}gedruckt{]}

Dieses Buch tritt in gewisser Weise mit dem Versprechen an, eine
umfassende Geschichte der Bibliotheken, von der Antike bis heute, zu
erzählen. Darauf deuten Titel, Ausstattung, der Klappentext und auch das
Vorwort hin. Gemessen daran ist es aber leider enttäuschend.

Zuerst: Die Geschichte, die erzählt wird, ist eine eurozentristische,
mit Fokus auf Grossbritannien und die USA sowie in Teilen der
Niederlanden, Deutschland und der Schweiz. Auch scheinen,
erstaunlicherweise für ein Buch aus dem 21. Jahrhundert, Geschichten und
Beispiele aus protestantischen Gesellschaften denen aus katholischen
gegenüber bevorzugt worden zu sein. Zum Beispiel wird teilweise
detailliert auf Bibliotheken in England und Schottland eingegangen, aber
nur einmal auf irische ‒ obwohl die Autoren unter anderem eine
Fellowship am Trinity College, Dublin, als Teil ihrer Recherchen
erwähnen und in der US-amerikanischen Version des Buches ein Bild aus
dieser Bibliothek als Cover verwendet wurde. Bei Geschichten aus anderen
Ländern scheint es immer wieder so, als würden nur diejenigen erwähnt,
welche unumgänglich sind, wenn so eine Geschichte erzählt wird.
Erschreckender für ein Buch, das eine gesamte Geschichte der
Bibliotheken erzählen will, ist aber, dass Bibliotheken aus anderen als
europäisch-US-amerikanischen Kulturen (mit Ausnahme einiger britischer
Kolonien) praktisch nicht vorkommen. Das Buch vermittelt den Eindruck
einer Geschichtsschreibung aus dem 19. Jahrhundert, bei der das eigene
Land im Mittelpunkt steht, andere \enquote{Kulturnationen} erwähnt
werden, aber der Rest der Welt praktisch als \enquote{geschichtslos}
angesehen wird. Zudem sind es auch eher kleine Episoden, die erzählt
werden. Ein richtiger roter Faden ist dabei ebenso wenig zu erkennen wie
eine Methodik für die Auswahl dieser Episoden.

Was die Autoren als Geschichte erzählen wollen ist folgendes:
Bibliotheken wurden immer wieder neu aufgebaut, aber gleichzeitig auch
zerstört oder gingen einfach ein, weil es kein Interesse an ihnen gab.
Es hätte immer Triebkräfte einzelner Personen benötigt, um Bibliotheken
zu etablieren. Fehlten diese, lösten sich über die Zeit die Bibliotheken
auch wieder auf. Allerdings: So überzeugend dies auf den ersten Blick
scheint, ist es nicht. Die Autoren berichten eigentlich keine neuen
Erkenntnisse über die Bibliotheksgeschichte, sondern tragen viel
Bekanntes (und zum Teil auch schon in dieser Kolumne besprochenes)
zusammen. Dabei landen sie zuletzt bei Bibliotheken, die direkt von
Staaten und staatlichen Institutionen getragen werden und damit nicht
mehr abhängig sind von Triebkräften einzelner Personen. Auch vorher
berichten sie immer und immer wieder über Klosterbibliotheken, die
geplündert oder von Staaten übernommen wurden, was aber darauf
hindeutet, dass diese Bibliotheken auch nicht von einzelnen Mönchen und
Nonnen aus Eigeninteresse zusammengetragen wurden, sondern -- wie die
Autoren selber betonen -- von Institutionen, die persönliche
Eigeninteressen und Leben (im Sinne von Zeiträumen, in den eine Nonne
oder ein Mönch lebte) überspannen. Die Geschichte, die sie erzählen,
widerlegt ihre These also selber. Was kein schlechtes Ergebnis für eine
Studie wäre, wenn dies am Ende auch so gesagt würde. Stattdessen
postulieren die Autoren am Ende, dass Bibliotheken auch in Zukunft als
Sammlung gedruckter Bücher, trotz aller Veränderungen,
Medienentwicklungen und Aufgaben, die Bibliotheken übernehmen, immer
weiter bestehen oder neu gegründet werden würden.

Leider zeigen die Autoren auch nicht, wieso sie überhaupt bestimmte
Beispiele oder Traditionen, über die sie berichten, ausgewählt haben und
andere unerwähnt lassen. Man kann oft auch nur vermuten, warum sie bei
einzelnen Themen, lokalen Geschichten oder Beispielen in die Tiefe gehen
und bei anderen nicht; warum sie mal die Bibliotheksgeschichte aus einem
Land erzählen und mal die aus einem anderen; warum so viele Beispiele
aus protestantisch geprägten Regionen stammen und so weiter. Deshalb ist
das Buch auch nicht als Überblickswerk zu empfehlen. (ks)

\hypertarget{social-media}{%
\section{4. Social Media}\label{social-media}}

{[}diesmal keine Beiträge{]}

\hypertarget{konferenzen-konferenzberichte}{%
\section{5. Konferenzen,
Konferenzberichte}\label{konferenzen-konferenzberichte}}

Curdt, Constanze ; Dierkes, Jens ; Helbig, Kerstin ; Lindstädt, Birte ;
Ludwig, Jens ; Neumann, Janna ; Parmaksiz, Uta (2021). \emph{Data
Stewardship Im Forschungsdatenmanagement} ‒ \emph{Rollen,
Aufgabenprofile, Einsatzgebiete: Überblick: 11. DINI/nestor Workshop,
16. und 17.11.2020.} In: Bausteine Forschungsdatenmanagement 4 (2021)
3:70--81, \url{https://doi.org/10.17192/bfdm.2021.3.8347}

Dieser Tagungsbericht zu der Frage, was genau Data Stewardship in der
Praxis ist, fasst die Ergebnisse der Beiträge so zusammen, dass klar
wird, dass es darauf keine richtige Antwort gibt. Auf der Veranstaltung
wurden Umfragen, Praxisberichte und theoretische Reflexionen vorgelegt,
die in der Gesamtsicht zeigen, dass Data Stewardship zwar von
verschiedenen Einrichtungen im Forschungsprozess als sinnvolle Aufgabe
angesehen wird, aber die Grenzen in der konkreten Praxis verschwimmen.
Es geht vor allem darum, dass Verantwortung (personell aber auch
infrastrukturell) für Daten und deren nachhaltige Verfügbarkeit
übernommen wird.

Der Bericht zeigt auch, dass Data Stewardship immer auch Komponenten
über die konkreten Daten hinaus beinhaltet und das Bibliotheken nicht
die einzigen Einrichtungen auf diesem Feld sind. (ks)

\hypertarget{populuxe4re-medien-zeitungen-radio-tv-etc.}{%
\section{6. Populäre Medien (Zeitungen, Radio, TV
etc.)}\label{populuxe4re-medien-zeitungen-radio-tv-etc.}}

Goldstein, Richard (2022). \emph{Autherine Lucy Foster, First Black
Student at U. of Alabama, Dies at 92.} In: New York Times, March 03,
2022, Section A, Page 23.
\url{https://www.nytimes.com/2022/03/02/us/autherine-lucy-foster-dead.html}

Am 02. März 2022 verstarb die Schwarze Lehrerin und Aktivistin Autherine
Lucy Foster im Alter von 92 Jahren. Sie war 1956 die erste Schwarze
Studierende in Alabama, nachdem sie sich auf einen Studienplatz
eingeklagt hatte. Die Gerichtsentscheidung ließ sie zum Studium zu,
untersagte ihr aber die Benutzung von Mensa und Wohnheimen. Sie hatte
sich für Library Science eingeschrieben, konnte aber nur drei Tage
studieren, da es zu Tumulten an der Universität kam und die
Universitätsleitung Autherine Lucy daraufhin unter dem Vorwand, es sei
zu ihrer eigenen Sicherheit, wieder suspendiert. Kurze Zeit später wurde
sie exmatrikuliert. Die Exmatrikulation wurde erst 1988 wieder
aufgehoben. Sie schrieb sich umgehend wieder ein, diesmal für
Erziehungswissenschaften und schloss das Studium 1992 erfolgreich ab.
Drei Wochen vor ihrem Tod, im Februar 2022, erhielt das College of
Education der University of Alabama den Namen Autherine Lucy Hall. (bk)

\begin{center}\rule{0.5\linewidth}{0.5pt}\end{center}

Lee, Stephanie M. (2021). \emph{A Data Sleuth Challenged A Powerful
COVID Scientist. Then He Came After Her}. In: BuzzFeed.News, 18.10.2021,
\url{https://www.buzzfeednews.com/article/stephaniemlee/elisabeth-bik-didier-raoult-hydroxychloroquine-study}

Beschrieben wird in diesem Artikel vor allem die Arbeit von Elisabeth
Bik, die es sich zur Aufgabe gemacht hat, wissenschaftliche Artikel auf
Fehler und Betrug hin zu untersuchen. Das tut sie nicht, um daraus
Profit zu schlagen oder Verschwörungstheorien zu unterstützen, sondern
als Teil der wissenschaftlichen Kommunikation. Sie ist erfolgreich darin
und trägt damit dazu bei, dass Wissenschaft an sich besser wird, da
veröffentlichte Artikel, die Fehler oder plagiierte Passagen enthalten,
zurückgezogen oder korrigiert werden müssen.

Das wird ihr von vielen hoch angerechnet, aber sie macht sich damit auch
Feinde in der Wissenschaftscommunity. Die Reaktion eines betroffenen
Forschers ist der Aufhänger des Artikels. Es wird sehr klar, dass dieser
falsch liegt und, anstatt seine Fehler einzugestehen, selber eine
Verschwörungstheorie kreiert. In dieser steht er selber im Mittelpunkt,
als jemand, der zu Unrecht verfolgt werden würde.

Aber neben diesem Aufhänger ist der Artikel eine gute und leicht zu
lesende Einführung darin, was Bik und einige andere Forschende tun, aber
auch, wie schwierig ihre Arbeit sein kann. Denn, wie im Artikel betont
wird, selbst wenn Wissenschaft oft als ein System beschrieben wird, dass
sich selber korrigieren würde, funktioniert dies offensichtlich nicht so
einfach, wenn sich die Beteiligten nicht an die Spielregeln halten. (ks)

\begin{center}\rule{0.5\linewidth}{0.5pt}\end{center}

Ross-Hellauer, Tony; Goldenberg, Anna (2022). \emph{Sind
wissenschaftliche Zeitschriften überflüssig geworden, Herr
Ross-Hellauer?} In: Falter. 23.03.2022, Nr. 12, Seite 19

Auf der Wissenschaftsseite der Österreichischen Wochenzeitung
\emph{Falter} wurde Tony Ross-Hellauer, Leiter der Open and Reproducible
Research Group am Institute for Interactive Systems and Data Science der
Technischen Universität Graz als Wissenschaftler der Woche gebeten,
seine Einschätzung zur titelgebenden Frage darzulegen. Er spannt dabei
den Bogen über eine Beschreibung von Open Access, ausgehend von der
Diagnose (fünf Wissenschaftsverlage dominieren, die Publikation in einer
bestimmten Zeitschrift ist für Forschende karrierewirksam, sowohl
Subskriptionen als auch Publikationsgebühren sind unangemessen teuer)
über mediale Aspekte (heute ist wissenschaftliches Publizieren online
statt gedruckt) und den Ursprung der OA-Bewegung (vor 20 Jahren,
Budapest Initiative) und den beiden bekannten Strategielinien zur Lösung
(Bibliotheken schaffen Publikationsinfrastrukturen und / oder Forschende
publizieren selbst). Ergebnis und Antwort: Wissenschaftliche
Zeitschriften sind für Ross-Hellauer überflüssig. Die wissenschaftliche
Kommunikation sollte viel interaktiver werden. Erst publizieren und im
Anschluss filtern wäre eine Variante. Journals und Wissenschaftler*innen
könnten \enquote{Playlists} von noch nicht reviewten Beiträgen
erstellen, die wiederum andere reviewen würden. Das alles beschreibt er
in knapperer Form als die dieser Zusammenfassung. Naturgemäß wird daraus
ein Open-Access-Hottake, der all die Mühen der Ebenen ausblendet, die
genau dafür verantwortlich sind, dass sich seine Publikations-Utopie
oder auch auch die der Open-Access-Erklärungen bislang nicht
befriedigend eingelöst hat. (bk)

\begin{center}\rule{0.5\linewidth}{0.5pt}\end{center}

Söffner, Jan (2022). \emph{Das kritische Denken der Intellektuellen
verschwindet aus der Öffentlichkeit}. In: Neue Zürcher Zeitung / nzz.ch
(29.04.2022)
\url{https://www.nzz.ch/feuilleton/open-access-wissenschaftliche-texte-haben-keine-leser-mehr-ld.1681122}

In einem, vorsichtig formuliert, eher wenig Fachkenntnis aufweisenden
Artikel beschäftigt sich der Kulturtheoretiker Jan Söffner mit dem
Verschwinden des Intellektuellen und den Defiziten von Open Access. Den
Ausgangspunkt bildet für ihn das kritische Denken, damit der kritisch
denkende Mensch und, als Stimmgabel seiner Elegie, Aldus Manutius
(1449-1515), venezianischer Verleger, Erfinder des Taschenbuchs und für
Jan Söffner der Elon Musk seiner Zeit. Dank der Entwicklung eines im
Gegensatz zu Gutenbergs Bibeln erheblich billigeren Druckmediums und dem
dabei entstehenden Buchmarkt, wurden, so seine These, Privatgelehrte und
\enquote{public intellectuals} zu den Vertretern eines \enquote{frischen
Denkens}. Die Taschenbuchkultur des 20. Jahrhunderts gilt dem Autor als
Höhepunkt dieser Entwicklung. Heute sei diese Entwicklungslinie am Ende,
denn einerseits dominiert \enquote{algorithmisch-mathematisches Wissen}
im Duett mit dem Bedeutungsverlust \enquote{sprachlicher
Intellektualität}. Und andererseits: Open Access. Und dabei genau
genommen der auf APC beruhende Weg des Open Access: \enquote{Der Markt
hat begonnen, das Geld nicht mehr mit der Leserschaft, sondern mit den
Autoren und den Universitäten zu erwirtschaften.}

Jan Söffner ignoriert dabei die wissenschaftsmediengeschichtliche
Tatsache, dass findige und einem Manutius ebenbürtige Verleger*innen
bereits früh im 20. Jahrhundert begannen, Publikationen allein für den
Erwerb durch Universitätsbibliotheken und unter Abforderungen hoher
Druckkostenzuschüsse von den Autor*innen zu produzieren. Heute nun
verlangen die Universitäten selbst wiederum, so der Autor, den
Autor*innen \enquote{digitale Gratispublikationen} ab und vergraben
diese anschließend im \enquote{Orkus irgendwelcher Massenspeicher}, also
vermutlich Repositorien. Das führe dazu, dass \enquote{kritisches
Denken} unsichtbar wird. Danach erklingt im Text Kritik an den
wissenschaftlichen Großverlagen Elsevier und Springer, der man kaum
widersprechen mag, die aber nicht immer so schlicht formuliert daher
kommt wie hier. Offenbar geht Jan Söffner davon aus, dass in einer
idealen Welt jeder wissenschaftliche Text automatisch Teil der
Suhrkamp-Kultur werden kann und per Taschenbuch im Buchhandel existieren
sollte. Die Folgen einer so entfesselten Präsenz im Sortimentsbuchhandel
und das Ringen um die öffentliche Wahrnehmung per Regal und Schaufenster
wären eine erstklassige Fabel für einen zweitklassigen dystopischen
Roman, bleiben bei Jan Söffner aber außen vor.

Generell ist das Leitmotiv seines Textes ein wenig stimmiges, von der
Realität weitgehend abgelöstes Bild des Funktionssystems des
wissenschaftlichen Publizierens. Im letzten Absatz wird dann deutlich,
dass all die Wissenschafts- und Open-Access-Kritik eigentlich nur als
Basslinie eines Loblieds auf das gedruckte Buch darstellt, dass das
Digitale medial vielfach übertrifft: als \enquote{technisch
überlegene{[}s{]} Medium der Intellektualität}, als Abbildungsmediums
des \enquote{Sinns} im Gegensatz zu \enquote{Daten}, als Optimalmedium
für das Nachvollziehen von Gedankengängen und als Reklusionsmedium, bei
dem die Nicht-Vernetzung dafür sorgt, dass die Lektüre nicht durch
Algorithmen mitverfolgt wird. Jan Söffner bewegt sich damit durchaus im
Sound der melancholischen digitalisierungskritischen Feuilletontradition
eines Roland Reuss oder Uwe Jochum. Er demonstriert zugleich, dass
solche Schwanengesänge erstaunlich zeitlos sind. Bis hinein in die
Rhetorik und Dramaturgie der Argumentation ist sein Artikel nämlich
einer, den man ohne Datumseindruck auch auf den April 2008 datieren
könnte. (bk)

\begin{center}\rule{0.5\linewidth}{0.5pt}\end{center}

Rud (2022) \emph{Uni-Bibliothek für Offenheit ausgezeichnet}. In:
Schwälmer Allgemeine. Ausgabe: 24.02. 2022, Seite 6

Die Schwälmer Allgemeine, Zeitung für Lokalnachrichten aus Schwalmstadt
im Schwalm-Eder-Kreis, meldet kurz, dass die knapp 60 Kilometer vom
Redaktionsbüro entfernte Universitätsbibliothek Kassel das \enquote{Open
Library Badge} erhielt. Sie bezieht sich damit vermutlich auf die eine
Woche vor der Meldung herausgegebene Pressemitteilung der Universität
Kassel
(\url{https://www.uni-kassel.de/uni/aktuelles/meldung/2022/02/17/universitaetsbibliothek-kassel-erhaelt-open-library-badge-2020}).
Die Leser*innen der Zeitung erfahren bei der Gelegenheit, was den
Ausschlag für die Auszeichnung gibt: \enquote{Kostentransparenz beim
Erwerb, das Sichtbarmachen und Fördern von frei zugänglichen
Open-Access-Publikationen, die freie Nutzung von Fotos der Bibliothek
und die Bereitstellung von Lehr- und Lernmaterialien unter offener
Lizenz.} (bk)

\begin{center}\rule{0.5\linewidth}{0.5pt}\end{center}

Kraemer, Bärbel (2021). \emph{Bad Belzig: \enquote{Bibliotheksplakate
aus der DDR}.} In: Fläming 365,
\url{https://flaeming365.de/web-stories/bibliotheksplakate-aus-der-ddr-tobias-bank/}

Ein kurzer, stark bebilderter Bericht zur Ausstellung
\enquote{Bibliotheksplakate aus der DDR}, welche im Sommer 2021 in Bad
Belzig (und später an anderen Orten) gezeigt wurde. Tobias Bank
(Historiker, unter anderem auch in der Politik aktiv) hat sie -- neben
anderen Alltagsobjekten aus der DDR -- zusammengetragen. Im Bericht
finden sich auch einige Abbildungen dieser Druckwerke, die zwischen
Aufklärung, Werbung und reiner Propaganda anzusiedeln sind. (ks)

\begin{center}\rule{0.5\linewidth}{0.5pt}\end{center}

Müller, Georg (2022). \emph{Bibliotheken wollen rasch an alte Zeiten
anknüpfen}. In: Freie Presse, 27.04.2022.,
\url{https://www.freiepresse.de/erzgebirge/zschopau/bibliotheken-wollen-rasch-an-alte-zeiten-anknuepfen-artikel12141065}

Es wird in zeitungstypisch knapper Form über die Effekte der
Corona-Pandemie auf die öffentlichen Bibliotheken in den
Erzgebirgsstädten Wolkenstein, Zschopau und Olbernhau berichtet.
Offenbar führten die 3G-Maßnahmen zu erheblichen Rückgängen in der
Besuchsfrequenz und damit auch bei den Ausleihen. In Zschopau wurde auch
ein Trend zu stärkeren Nutzung digitaler Bibliotheksangebote
festgestellt. In allen drei Bibliotheken liegt der Schwerpunkt darauf,
aktiv, unter anderem über Veranstaltungsprogramme, auf das Niveau der
Nutzungszahlen in den vorpandemischen Jahren zurückzukehren. Leicht wird
dies vermutlich nicht, wie Sabine Fritzsch von der Stadtbibliothek
Olbernhau betont: \enquote{Vor den Bibliotheken liege ein Kraftakt.}
(bk)

\begin{center}\rule{0.5\linewidth}{0.5pt}\end{center}

Levitin, Mia (2022). \emph{The love of books}. In: FT Weekend, 16/17
April 2022, S. 8

In einer Sammelrezension zu drei Neuerscheinungen (Rebecca Lee: How
Words Get Good: The Story of Making a Book.; Emma Smith: Portable Magic:
A History of Books and Their Reasers; Jeff Deutsch: In Praise of Good
Bookstores.) trägt die Literaturjournalistin Mia Levitin einige aktuelle
Beobachtungen zur Buchkultur und Buchwirtschaft zusammen. So kann
beispielsweise eine deutliche Zunahme des Absatzes von gedruckten
Büchern in den Kernmärkten für die Jahre 2020 und 2021 registriert
werden. Dieser Trend wird sich 2022, so die Vorhersage, fortsetzen.
Zugleich werden sich auch hier Lieferketten- und Produktionsprobleme als
Bremsklötze auswirken. Zwei Genres treiben den Boom: \enquote{adult
fiction} und \enquote{young adult fiction}, letzteres offenbar durch die
Verschränkung mit Social-Media-Kommunikation (Stichwort:
\emph{Booktok}).

Für die nachhaltige Beliebtheit des Mediums Buch bieten wiederum die
drei besprochenen Titel einige Argumente. Mit Rebecca Lee wird die Linie
zur religiösen Ur-Buchkultur (bible=biblio) gezogen. Emma Smith
argumentiert mit der haptischen Qualität des Mediums (\enquote{physical
bookhood}), die den potentiellen Leser*innen schon per Form und Paratext
entscheidende Informationen zum jeweiligen Titel liefert. E-Reader,
lange als Totenglöckchen des Gedruckten verhandelt, können die
sensorische und, wenn man so will, topologische Qualität eines
greifbaren Objektes nicht ersetzen. Das Buch ist körperlich responsiv
und sein Gebrauch erzeugt sogar Geräusche, die sich als Klangbild der so
genannten \enquote{Library Sounds} durchaus als ASMR-tauglich erweisen.
Außerdem erklärt Smith den Erfolg der Paperbacks als
Demokratisierungsform der Buchkultur: Die frühen Titel von Penguin
kosteten exakt soviel wie ein Krug Bier oder eine Schachtel Zigaretten,
hatten bei den Rezipienten aber ganz anderen Wirkungen. Jeff Deutsch
huldigt nicht nur klassischen Buchhandlungen, sondern setzt sich auch
mit dem Phänomen des \enquote{Zu viele Bücher, zu wenig Zeit}
auseinander, die Susan Sontag von ihrer Bibliothek als einem
\enquote{archive of longings} sprechen ließ. \emph{Tsundoku} ist das
Wort, dass die japanische Sprache der globalen Bibliophilie zur
Benennung dieses Phänomens geschenkt hat.

Und sofern man nicht auf Bücher steht, fährt man vielleicht auf ihnen,
wie Levitin noch einmal mit Lee als Trivia einsprengselt: Teile des
Midland Expressway (M6) wurden nämlich mit dem Stampfmasse von 2,5
Millionen makulierter Liebesromanen unterlegt. Und zwar als
Geräuschdämpfer mit einer Quote von 45.000 Exemplaren pro Autobahnmeile.
(Mehr dazu:
\url{http://news.bbc.co.uk/2/hi/uk_news/england/west_midlands/3330245.stm}
) (bk)

\begin{center}\rule{0.5\linewidth}{0.5pt}\end{center}
\selectlanguage{ukrainian}
Пастернак, C. 
\selectlanguage{ngerman}(1923). 
\selectlanguage{ukrainian}\emph{Всенародня бібліотека в Київі}.
\selectlanguage{ngerman}In:
\selectlanguage{ukrainian}Селянська правда
\selectlanguage{ngerman}, 02.03.1923, S. 4.

Der Autor berichtet aus der Perspektive des Erscheinungszeitpunkts (März
1923) über den Auftrag und die Geschichte der Ende 1918 gegründeten
Nationalbibliothek der Ukraine in Kyiv, die bis dato auch in der Ukraine
wenig bekannt war. Das Anliegen der Bibliothek war der Aufbau einer
Sammlung nach dem Vorbild der Nationalbibliothek in Paris, der
Bibliothek des British Museum und der damaligen wissenschaftlichen
Bibliothek in Petrograd. Es wird nicht ohne Stolz berichtet, dass die
Kyiver Bibliothek in nur vier Jahren eine Sammlung aufzubauen vermochte,
deren Größe nahezu alle Bibliotheken in den USA übertrifft. Dies gelang
vor allem durch die Aufnahme zahlreicher ukrainischer
Bibliotheksbestände und Nachlassbibliotheken ukrainischer Gelehrter
sowie zahllose Spenden. Außerdem erhält sie Pflicht- und Tauschexemplare
aus den Republiken der UdSSR und über internationale Verbindungen. Im
Lesesaal wurde ein Handbestand von 100.000 Büchern und 89.000 Zeitungen
für das Publikum bereitgestellt. (bk)

\begin{center}\rule{0.5\linewidth}{0.5pt}\end{center}

\selectlanguage{ukrainian}Петро Зленко 
\selectlanguage{ngerman}(1933) 
\selectlanguage{ukrainian}\emph{Бібліотеки української еміграції у ЧСР}. 
\selectlanguage{ngerman}In:
\selectlanguage{ukrainian}Діло, 
\selectlanguage{ngerman}Nr.1, 01.01.1933, S. 6

Der in Prag im Exil lebende ukrainische Bibliograf, Verleger und
Buchwissenschaftler Petro Zlenko berichtet ausführlich über ukrainische
Exilbliotheken in Prag. Diese dienen, laut dem Autor, nicht allein der
Literaturversorgung sondern auch als wichtige spirituelle Stütze für die
im Exil befindlichen Autoren, Journalisten und Studenten. Ein Großteil
der aus der Ukraine nach Prag geschafften Bestände, die ganze
Bibliotheken umfassen, befindet sich im Museum des Befreiungskampfes der
Ukraine (\selectlanguage{ukrainian}Музей визвольної боротьби України\selectlanguage{ngerman}). Daneben gibt es eine Reihe
weitere ukrainischer Bibliotheken, die sich überwiegend in Kultur- und
Bildungseinrichtungen der ukrainischen Diaspora befinden, wobei der
Autor vom \enquote{slawischen Prag} (\enquote{\selectlanguage{ukrainian}славянська Прага\selectlanguage{ngerman}}) schreibt. Er betont die Rolle dieser Bestände als Kulturerbe und
verleiht seinem Optimismus Ausdruck, dass die so wohl noch auf lange
Sicht im slawischen Prag befindliche ukrainischen Emigranten einen
wichtigen Beitrag für die Entwicklung der slawischen Kultur leisten
werden. (bk)

\hypertarget{abschlussarbeiten}{%
\section{7. Abschlussarbeiten}\label{abschlussarbeiten}}

{[}diesmal keine Beiträge{]}

\hypertarget{weitere-medien}{%
\section{8. Weitere Medien}\label{weitere-medien}}

\emph{Special Feature: South African Library Week} (2021). In: Imbiza:
Journal for African Writing, 1 (2021) 1: 22--27 {[}gedruckt{]}

Imbiza ist ein 2021 neu erschienenes Magazin (mit Basis an der
University of Petroria, Südafrika), welches sich als Forum für das
Schreiben in Afrika versteht, obgleich der Fokus -- zumindest in dieser
Ausgabe -- alleine auf Südafrika liegt. Es ist kein rein literarisches
Magazin, auch wenn eine Anzahl an Kurzgeschichten und Gedichten
publiziert wurde. Vielmehr umfasst diese Ausgabe auch eine Spannbreite
an Essays und Interviews, beispielsweise mit Koleka Putuma und Fred
Khumalo, über das illegale «Radio Freedom» des ANC während der Zeit der
Apartheid, über die Sklaverei, welche die Basis für den südafrikanischen
Weinbau legte, zur Übersetzung von Gesundheitsinformationen oder auch
das Erzählen in der südafrikanischen Literatur. Konsequenterweise
erscheint das Heft nicht rein in Englisch, sondern zum Teil auch in
Kiswahili und IsiXhosa.

Erwähnt werden soll es hier, weil -- offenbar ohne dass es eine grossen
Erklärung durch die Redaktion bedurfte --, in diesem neuen Magazin
gleich der erste Schwerpunkt zu Bibliotheken ist: Einige Autor*innen
berichten hier über die Bedeutung von Bibliotheken in ihrem Leben, auch
während der Apartheid. Es wird beklagt, dass es heute noch lange nicht
ausreichend viele Öffentliche und Schulbibliotheken in Südafrika gäbe.
Zudem wird eine NGO vorgestellt, die gerade solche Schulbibliotheken
(und Computerräume) im ländlichen Raum einrichtet.

Dem Rezensent fiel dieses Heft zu der Zeit in die Hand (in der für
solche Entdeckungen zu empfehlenden Buchhandlung InterKontinental in
Berlin-Friedrichshain, 2021 auch ausgezeichnet mit dem Deutschen
Buchhandlungspreis), als in Deutschland Autor*innen sich in eine
Kampagne der Verlage einspannen liessen, um die rechtliche
Gleichstellung von elektronischen und physischen Medien zu verhindern.
Wenn auch etwas kurz und in einem sehr rosigen Ton gehalten, war der
Schwerpunkt eine Erinnerung daran, dass im Allgemeinen Autor*innen und
Forschende im Bereich Literatur und Bibliotheken auf einer Seite stehen
-- auf der, die vor allem ein Interesse hat, die möglichst weite
Verbreitung von Literatur zu ermöglichen. (ks)

\begin{center}\rule{0.5\linewidth}{0.5pt}\end{center}

Hall, Edith (2021). \emph{Ancient Greek and Roman Libraries}. (Vortrag,
10.10.2021) London: Gresham College,
\url{https://www.gresham.ac.uk/lectures-and-events/ancient-libraries}

Im Rahmen der öffentlichen Vorträge, welche vom Gresham College in
London seit Jahrhunderten organisiert werden, bot Prof.~Edith Hall,
Historikerin, einen Überblick dazu, was über antike Bibliotheken bekannt
ist. Der Vortrag ist recht kurzweilig, Hall geht auf die Repräsentation
dieser Bibliotheken in späterer Literatur und im Film ein.

Sie zeigt, dass es recht wenige konkrete Zeugnisse dieser Bibliotheken
gibt, die Geschichtsschreibung also vor allem ein Puzzlespiel betreibt.
Dieses ist zudem von späteren Vorstellungen überlagert. Geht man aber
auf die antiken Quellen zurück, so zeigen sich bestimmte Traditionen --
beispielsweise die Vorstellung, dass es Menschen gäbe, die Bücher als
Statussymbol sammeln, aber nicht lesen, aber auch, dass das Gründen von
Bibliotheken zum Status einer Person beitrug --, aber vor allem auch
Unterschiede zu heute.

Insbesondere bei den Antworten auf Fragen, die im Anschluss gestellt
wurden, merkt man auch, dass die Vortragende nicht selber Bibliothekarin
ist. Fachtermini und Grundfakten über heutige Bibliotheken sind ihr
nicht bekannt. Es ist also eine sehr historische Sicht, die vermittelt
wird. Der Vortrag war Teil einer Serie unter dem Titel \enquote{Books,
Libraries and Civilization}
(\url{https://www.gresham.ac.uk/watch-now/series/books-libraries}). (ks)

\begin{center}\rule{0.5\linewidth}{0.5pt}\end{center}

Gillespie, Alexandra (2021). \emph{The 2021 Alexander C. Pathy Lecture
on the Book Arts - "On the Silk Roads Project."}, Toronto: Fisher Rare
Book Library, \url{https://youtu.be/nOqSzpN9uz0}

Der Vortrag über eine Ausstellung im Aga Khan Museum, Toronto, sowie dem
Forschungsprojekt an der University of Toronto, auf dem diese
Ausstellung basierte, ist in gewisser Weise eine Reflektion über die
kolonialen oder zumindest eurozentristischen Traditionen, auf denen
Buchgeschichte und Sammlungen in Bibliotheken wie zum Beispiel der
Fisher Rare Book Library an der University of Toronto, basiert.
Grundsätzlich geht es in der Ausstellung darum, die Geschichte des
Buches und seiner Nutzung von einem anderen Fokus als Europa aus zu
erzählen. Das Forschungsprojekt umfasst rund zwanzig Personen in Toronto
und zahlreiche Kollaborateur*innen in der ganzen Welt, welche Bücher aus
anderen Kulturen mit unterschiedlichen, nicht nur humanistischen,
Methoden untersuchten und zeigt das Vorhandensein von Buchkulturen auf,
bei denen zum Beispiel die Erfindung des Buchdrucks im 15. Jahrhundert
in Europa keine relevante Veränderung mit sich brachte oder die auch das
Buch nicht vorrangig als abgeschlossene Monographie, sondern als
lebendes Dokument dachten.

Während diese Erweiterung des Blickwinkels eine willkommene Erweiterung
des Wissenshorizonts über Buchgeschichte darstellt, beschäftigen sich
grosse Teile der anschliessenden Fragerunden mit Themen wie Provenienz
-- wobei es hier nicht, wie oft im DACH-Raum aus guten Gründen, um den
Nationalsozialismus geht, sondern vor allem um Kolonialismus -- und auch
Fragen danach, wie ein respektvoller Umgang mit den heute in
Bibliotheken vorhandenen Texten gefunden werden kann. Gillespie stellt
in ihren Antworten dar, dass es darauf keine einfachen Antworten gibt,
sondern es sich eher zeigt, dass sich aktuell eine neue Landschaft an
Fragen eröffnet, welche die Forschung, die Institutionen, die heute die
Sammlungen halten -- also auch Bibliotheken -- und die betroffenen
Communities in den nächsten Jahrzehnten beschäftigen wird. (ks)

\begin{center}\rule{0.5\linewidth}{0.5pt}\end{center}

o.A. (2022). \emph{KADEWE. Haus der Geschichten. Zum Jubiläum}. In:
KaDeWe 115 Jahre für immer Dein. {[}Kund*innenmagazin{]}. Berlin:
KaDeWe, 2022. S. 72

Zur Eröffnungsausstattung des Kaufhaus des Westens (KaDeWe) am 27, März
1907 gehörte auch eine Kaufhausbibliothek, bei der nach dem Prinzip
einer kommerziellen Leihbücherei Kund*innen \enquote{für 15 Mark Gebühr
monatlich bis zu acht Bücher leihen} konnten. (bk)

%autor

\end{document}
