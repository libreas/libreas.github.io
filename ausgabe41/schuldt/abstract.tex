\textbf{Zusammenfassung:} Anhand der vor allem in den 1980er Jahren
verbreiteten Technologie der interaktiven Videotextsysteme und den
daraus resultierenden Reaktionen von Bibliotheken, wird versucht, ein
Modell für den Zusammenhang von Bibliotheken und neuen Technologien zu
erarbeiten. Konzentriert wird sich dabei auf das französische Minitel-
und das bundesdeutsche Btx-System. Im Ergebnis zeigt sich, dass
Bibliotheken relativ schnell reagieren, neue Technologien daraufhin zu
hinterfragen, ob sie für Aufgaben in Bibliotheken genutzt werden können.
Es zeigt sich aber auch, dass die tatsächliche Verbreitung in
Bibliotheken massiv von Entwicklungen ausserhalb von Bibliotheken
beeinflusst wird. Der Artikel endet mit ersten Eckpunkten für ein
solches Modell und offenen Forschungsfragen.

\begin{center}\rule{0.5\linewidth}{0.5pt}\end{center}

\textbf{Abstract:} Using the technology of interactive videotex systems,
which became widespread especially in the 1980s, and the reaction of
libraries towards these technologies as an example, an attempt was made
to develop a model for the relationship between libraries and new
technologies. The focus is on the French Minitel- and the German
Btx-system. Results show that libraries react relatively quickly when
they question new technologies as to whether they can be used for tasks
in libraries themselves. But also that the actual dissemination is
massively influenced by developments outside of libraries themselves.
The article ends with first sketches for such a model and open research
questions.
