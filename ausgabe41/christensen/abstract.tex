\textbf{Zusammenfassung:} In den letzten Jahren ist das Angebot an
Academic Search Engines für die Recherche nach Fachliteratur zu allen
Wissenschaftsgebieten stark angewachsen und ergänzt die beliebten
kommerziellen Angebote wie Web of Science oder Scopus. Der Artikel zeigt
die wesentlichen Unterschiede zwischen bibliothekarischen
Discovery-Systemen und Academic Search Engines wie Base, Dimensions oder
Open Alex auf und diskutiert Möglichkeiten, wie beide von einander
profitieren können. Diese Entwicklungsperspektiven betreffen Aspekte wie
die Kontextualisierung von Wissen, die Datenmodellierung, die
automatischen Datenanreicherung sowie den Zuschnitt von Suchräumen.
