\documentclass[a4paper,
fontsize=11pt,
%headings=small,
oneside,
numbers=noperiodatend,
parskip=half-,
bibliography=totoc,
final
]{scrartcl}

\usepackage[babel]{csquotes}
\usepackage{synttree}
\usepackage{graphicx}
\setkeys{Gin}{width=.4\textwidth} %default pics size

\graphicspath{{./plots/}}
\usepackage[ngerman]{babel}
\usepackage[T1]{fontenc}
%\usepackage{amsmath}
\usepackage[utf8x]{inputenc}
\usepackage [hyphens]{url}
\usepackage{booktabs} 
\usepackage[left=2.4cm,right=2.4cm,top=2.3cm,bottom=2cm,includeheadfoot]{geometry}
\usepackage{eurosym}
\usepackage{multirow}
\usepackage[ngerman]{varioref}
\setcapindent{1em}
\renewcommand{\labelitemi}{--}
\usepackage{paralist}
\usepackage{pdfpages}
\usepackage{lscape}
\usepackage{float}
\usepackage{acronym}
\usepackage{eurosym}
\usepackage{longtable,lscape}
\usepackage{mathpazo}
\usepackage[normalem]{ulem} %emphasize weiterhin kursiv
\usepackage[flushmargin,ragged]{footmisc} % left align footnote
\usepackage{ccicons} 
\setcapindent{0pt} % no indentation in captions

%%%% fancy LIBREAS URL color 
\usepackage{xcolor}
\definecolor{libreas}{RGB}{112,0,0}

\usepackage{listings}

\urlstyle{same}  % don't use monospace font for urls

\usepackage[fleqn]{amsmath}

%adjust fontsize for part

\usepackage{sectsty}
\partfont{\large}

%Das BibTeX-Zeichen mit \BibTeX setzen:
\def\symbol#1{\char #1\relax}
\def\bsl{{\tt\symbol{'134}}}
\def\BibTeX{{\rm B\kern-.05em{\sc i\kern-.025em b}\kern-.08em
    T\kern-.1667em\lower.7ex\hbox{E}\kern-.125emX}}

\usepackage{fancyhdr}
\fancyhf{}
\pagestyle{fancyplain}
\fancyhead[R]{\thepage}

% make sure bookmarks are created eventough sections are not numbered!
% uncommend if sections are numbered (bookmarks created by default)
\makeatletter
\renewcommand\@seccntformat[1]{}
\makeatother

% typo setup
\clubpenalty = 10000
\widowpenalty = 10000
\displaywidowpenalty = 10000

\usepackage{hyperxmp}
\usepackage[colorlinks, linkcolor=black,citecolor=black, urlcolor=libreas,
breaklinks= true,bookmarks=true,bookmarksopen=true]{hyperref}
\usepackage{breakurl}

%meta
%meta

\fancyhead[L]{A. Christensen\\ %author
LIBREAS. Library Ideas, 41 (2022). % journal, issue, volume.
%\href{https://doi.org/10.18452/xxx}{\color{black}https://doi.org/10.18452/xxx}
{}} % doi 
\fancyhead[R]{\thepage} %page number
\fancyfoot[L] {\ccLogo \ccAttribution\ \href{https://creativecommons.org/licenses/by/4.0/}{\color{black}Creative Commons BY 4.0}}  %licence
\fancyfoot[R] {ISSN: 1860-7950}

\title{\LARGE{Wissenschaftliche Literatur entdecken: Was bibliothekarische Discovery-Systeme von der Konkurrenz lernen und was sie ihr zeigen können}}% title
\author{Anne Christensen} % author

\setcounter{page}{1}

\hypersetup{%
      pdftitle={Wissenschaftliche Literatur entdecken: Was bibliothekarische Discovery-Systeme von der Konkurrenz lernen und was sie ihr zeigen können},
      pdfauthor={Anne Christensen},
      pdfcopyright={CC BY 4.0 International},
      pdfsubject={LIBREAS. Library Ideas, 41 (2022)},
      pdfkeywords={Discovery Systeme, Academic Search Engines, Bibliothekskataloge, OpenAlex},
      pdflicenseurl={https://creativecommons.org/licenses/by/4.0/},
      pdfcontacturl={http://libreas.eu},
      baseurl={},
      pdflang={de},
      pdfmetalang={de}
     }
	 



\date{}
\begin{document}

\maketitle
\thispagestyle{fancyplain} 

%abstracts
\begin{abstract}
\noindent
\textbf{Zusammenfassung:} In den letzten Jahren ist das Angebot an
Academic Search Engines für die Recherche nach Fachliteratur zu allen
Wissenschaftsgebieten stark angewachsen und ergänzt die beliebten
kommerziellen Angebote wie Web of Science oder Scopus. Der Artikel zeigt
die wesentlichen Unterschiede zwischen bibliothekarischen
Discovery-Systemen und Academic Search Engines wie Base, Dimensions oder
Open Alex auf und diskutiert Möglichkeiten, wie beide von einander
profitieren können. Diese Entwicklungsperspektiven betreffen Aspekte wie
die Kontextualisierung von Wissen, die Datenmodellierung, die
automatischen Datenanreicherung sowie den Zuschnitt von Suchräumen.
\end{abstract}

%body
\begin{center}
\textbf{Dieser Text ist Dr.~Dieter Müller (1948--2022) in Liebe und
Dankbarkeit gewidmet.} \\
\vspace{.5em}***
\end{center}

\emph{Vorbemerkung: Dieser Text sollte eigentlich eine Rezension für
OpenAlex werden. Diese Academic Search Engine wurde Ende 2021
angekündigt und erscheint hinsichtlich ihres Datenmodells und der
grundsätzlichen Offenheit von Code und Daten als besonders
vielversprechende Ergänzung des bestehenden Marktes. Da zum Zeitpunkt
der Veröffentlichung jedoch nur die Schnittstelle, nicht aber eine
eigene Benutzungsoberfläche zur Verfügung stand und ein Test ohne
Heranziehung zusätzlicher Tools nicht erfolgen konnte, werden nun die
Academic Search Engines insgesamt aus bibliothekarischer Perspektive
betrachtet. In diese Betrachtung sind neben den zitierten Quellen vor
allem auch meine persönlichen Erfahrungen und Bewertungen aus über
15-jähriger theoretischer und praktischer Beschäftigung mit
Discovery-Systemen eingeflossen.}

Über Jahrzehnte galt in Bibliotheken: Kataloge für Bestandsnachweise,
Bibliografien für darüber hinausgehende, vor allem thematische Suchen.
Mit Discovery-Systemen versuchen Bibliotheken, diese Dualität
aufzuheben, um einen zentralen Einstiegspunkt für die Recherche
anzubieten, der sowohl lokale Bestände als auch zentrale bibliografische
Ressourcen recherchierbar macht. Die Vielzahl von Herausforderungen, die
es dafür in den Bereichen Metadatenmanagement, Technik und Lizenzen zu
bewältigen gibt, sind noch nicht überwunden, und die Systeme sind weit
hinter den Erwartungen zurückgeblieben.\footnote{Christensen, Anne, und
  Matthias Finck. 2021. \enquote{Discovery-Systeme: Eine Analyse ihrer
  Geschichte und Gegenwart mit dem Hype-Zyklus}. \emph{Bibliothek
  Forschung und Praxis} 45 (3): 497--508.
  \url{https://doi.org/10.1515/bfp-2021-0039}}

Gleichzeitig sind neben dem klassischen, fachübergreifenden Web of
Science in den vergangenen 20 Jahren weitere Plattformen für die
Recherche nach wissenschaftlicher Literatur entstanden -- die Academic
Search Engines. Unter diesen war unter anderem Google mit dem Ableger
Google Scholar ein Pionier, aber auch die Bielefeld Academic Search
Engine (BASE). Diese geht genauso wie Google Scholar auf das Jahr 2004
zurück und ist inzwischen ein etabliertes und unabhängiges Instrument
mit internationalem Renommee. Besonders in den letzten Jahren ist das
Angebot solcher Suchmaschinen noch einmal weiter gewachsen: Es gibt
sowohl kommerzielle Anbieter, bei denen wesentliche Teile des Angebotes
kostenlos nutzbar sind, als auch Dienste, die ausschließlich auf
öffentliche Finanzierung und freie Zugänglichkeit und Nachnutzung
setzen.\footnote{Tay, Aaron. 2020. \enquote{The next generation
  discovery citation indexes - a review of the landscape in 2020}.
  \url{http://musingsaboutlibrarianship.blogspot.com/2020/11/the-next-generation-discovery-citation.html}}

Die Rezeption der Academic Search Engines in Bibliothekskreisen ist
erstaunlich zurückhaltend, obwohl diese eine Rolle als potenzielle
Empfänger von Metadaten aus Repositorien spielen, und sie
Beobachtungsgegenstand der bibliometrischen Forschung sind.\footnote{Zum
  Beispiel durch das Kompetenzzentrum Bibliometrie,
  \url{https://bibliometrie.info/}} In Erwerbungsabteilungen werden
mitunter Vergleiche zwischen den kommerziellen Diensten Scopus und Web
of Science gezogen, um Entscheidungen über den (Weiter-)Bezug dieser
Dienste zu treffen. Was jedoch fehlt, ist eine Auseinandersetzung mit
den Funktionen, Technologien und Strategien von Academic Search Engines
im Zusammenhang mit den bibliothekarischen Discovery-Systemen.

Kaum eine Universitätsbibliothek verzichtet inzwischen darauf, neben
oder sogar statt des klassischen Bibliothekskataloges, ein
Discovery-System anzubieten, das neben den OPAC-Funktio\-nalitäten einen
erweiterten Suchraum sowie eine zeitgemäße Benutzungsoberfläche umfasst.
Eine Orientierung an Suchmaschinen, egal ob allgemeiner Natur oder eben
Academic Search Engines, findet inzwischen kaum mehr statt -- obgleich
der Impuls zur Entwicklung dieser Systeme eben aus jener Orientierung
stammte. Trotzdem haben sich nur wenige Prinzipien der
Suchmaschinentechnologie in den Discovery-Systemen durchsetzen können.
Vielmehr wurde das Augenmerk darauf gelegt, die Funktionalitäten des
OPACs möglichst gut nachzubauen beziehungsweise zu integrieren. Dies hat
zum Phänomen der \enquote{Reduktion auf einfachste
Suchmöglichkeiten}\footnote{Mattmann, Beat, und Noah Regenass. 2021.
  \enquote{Eine neue Form der Recherche in Bibliotheken}.
  \emph{Bibliothek Forschung und Praxis} 45 (2): 304--16.
  \url{https://doi.org/10.1515/bfp-2021-0010}} geführt, das die
ursprüngliche Idee der Discovery-Systeme ad absurdum führt.
Möglicherweise kann ein Blick auf nicht-bibliothekarische Suchsysteme
helfen, der in einer gewissen Stagnation befindlichen Entwicklung der
Discovery-Systeme wieder mehr Dynamik zu verleihen.

\section{Academic Search Engines -- Der Stand der Dinge}

Suchmaschinen für wissenschaftliche Literatur und anderer
Fachinformationen gibt es wie erwähnt bereits seit fast 20 Jahren, und
sie unterziehen sich auch schon so lange den Vergleichen mit
bibliothekarischen Angeboten.\footnote{Kortekaas, Simone, und Bianca
  Kramer. 2014. \enquote{Thinking the unthinkable -- doing away with the
  library catalogue}. \emph{Insights: the UKSG journal} 27 (3): 244--48.
  \url{https://doi.org/10.1629/2048-7754.174}} Ebenso werden sie
regelmäßig miteinander in Bezug auf ihren Abdeckungsgrad und der Eignung
für einzelne Fragestellungen verglichen.\footnote{Visser, Martijn, Nees
  Jan van Eck, und Ludo Waltman. 2021. \enquote{Large-scale comparison
  of bibliographic data sources: Scopus, Web of Science, Dimensions,
  Crossref, and Microsoft Academic}. \emph{Quantitative Science Studies}
  2 (1): 20--41. \url{https://doi.org/10.1162/qss_a_00112}} Damit, dass
Crossref als Serviceplattform für Wissenschaftsverlage im Jahr 2013 eine
Schnittstelle zum kostenlosen Abruf von Metadaten bereitgestellt hat,
konnten auch verlagsunabhängige Anbieter eine kritische Masse an
Metadaten regelhaft abrufen und über eigene Dienste recherchierbar
machen. Diesen Dienst nutzen auch einzelne bibliothekarische
Discovery-Systeme.\footnote{Kooperativer Bibliotheksverbund
  Berlin-Brandenburg (KOBV). 2020. \enquote{KOBV-Portal mit
  Crossref-Index und neue ALBERT-Instanzen online}.
  \url{https://www.kobv.de/kobv-portal-mit-crossref-index-und-neue-albert-instanzen-online/}}

Academic Search Engines gibt es von kommerziellen wie auch
gemeinnützigen Anbietern. Die Benutzung ist in der Regel kostenlos,
zumindest in Bezug auf die Funktionalitäten der Literaturrecherche. Mit
dem Microsoft Academic Graph stand zwischen 2015 und 2021 das Angebot
eines kommerziellen Dienstleisters zur Verfügung, der mit über 120
Millionen Referenzen über die größte Datenbasis verfügte und diese auch
zur Benutzung über eine Schnittstelle bereitstellte.\footnote{Microsoft
  Academic Graph:
  \url{https://www.microsoft.com/en-us/research/project/microsoft-academic-graph/}}

Diese Datenbasis wurde von OpenAlex übernommen, einem Ende 2021
angekündigten Dienst, der bereits jetzt über eine Schnittstelle benutzt
werden kann und in naher Zukunft auch mit einer eigenen
Web-Schnittstelle ausgestattet werden soll. Allein wegen der Datenbasis
ist OpenAlex ein besonders vielversprechendes Instrument. Hinzu kommt,
dass OpenAlex in jeder Hinsicht den Prinzipien der Openness folgt, also
nicht nur auf Nutzung freier Metadaten und freie Bereitstellung
derselben setzt, sondern auch die Software, die für die Aggregation und
Nachbearbeitung der Metadaten benutzt wird, quelloffen zur Nachnutzung
zur Verfügung stellt.\footnote{GitHub-Repositorium
  \enquote{openalex-guts}:
  \url{https://github.com/ourresearch/openalex-guts}} OpenAlex wird von
dem Non-Profit-Unternehmen OurResearch betrieben, das unter anderem auch
den Dienst Unpaywall betreibt. OurResearch hat für den Aufbau von
OpenAlex nach eigenen Angaben Fördergelder in Höhe von 4,5 Millionen
US-Dollar einer gemeinnützigen Stiftung bekommen.\footnote{OpenAlex
  About-Webseite: \url{https://openalex.org/about}}

Inwieweit diese Gelder ausreichen, um den geplanten Dienst dauerhaft
anbieten zu können, wird sich zeigen. Mitanbieter wie die ebenfalls
gemeinnützige Plattform TheLens.org finanzieren sich über
Stiftungsgelder, zum Teil aber auch unterstützt über Crowdfunding. Die
kommerziellen Mitbewerber verfügen vermutlich über deutlich mehr Mittel.
Die zur Verlagsgruppe Holtzbrinck gehörende Firma Digital Science
erzielt mit ihren Diensten, zu denen die Academic Search Engine
Dimensions gehört, beispielsweise einen jährlichen Umsatz von 76,1
Millionen US-Dollar.\footnote{Growjo. Digital Science Revenue and
  Competitors. \url{https://growjo.com/company/Digital_Science}}

Neben der Bereithaltung der technischen Infrastrukturen für die
verarbeiteten Datenmengen investieren die Academic Search Engines auch
stark in die konzeptionelle Entwicklung ihrer Dienste. Ein weiteres
Auszeichnungsmerkmal von Academic Search Engines ist die Orientierung an
Use Cases, also erwarteten Nutzungsszenarien. Für die prototypische
geisteswissenschaftliche Suchmaschine SoNAR wurden diese Use Cases
gemeinsam mit künftigen Nutzenden des Systems entwickelt.\footnote{Balck,
  Sandra, Menzel, Sina, Petras, Vivien, Schnaitter, Hannes, and Zinck,
  Josefine. 2022. \enquote{Fluch und Segen der Visualisierung -
  Unterschiedliche Zielfunktionen im Forschungsprozess der historischen
  Netzwerkanalyse}. Zenodo. \url{https://doi.org/10.5281/zenodo.6328123}}
Das Team hinter OpenAlex publiziert regelmäßig \enquote{Tips of the
Day}, in denen die Funktionen der Schnittstelle demonstriert
werden.\footnote{Siehe zum Beispiel:
  \url{https://www.twitter.com/OpenAlex_org/status/1501583631873044488}}

Diese beiden Dienste, SoNAR und OpenAlex, haben außerdem gemeinsam, dass
sie bibliografische Daten nicht nur aggregieren und auf ein
einheitliches Format bringen, sondern auch eine andere Form der
Modellierung für die Daten wählen und anstelle von
Suchmaschinentechnologie Graphdatenbanken verwenden, um die in der Regel
vielfältigen Beziehungen zwischen einzelnen Elementen der
bibliografischen Beschreibungen korrekt darzustellen und sinnvoll
durchsuchbar und visualisierbar zu machen (vergleiche Abschnitt 1). Mit
diesen Technologien dürften zumindest einige der Academic Search Engines
den bibliothekarischen Erschließungsprinzipien näher kommen als die
derzeitigen Discovery-Systeme, die bei der Vernetzung von Daten nach
formalen oder inhaltlichen Kriterien hinter den Bibliothekskatalogen
zurückbleiben.

Wegen ihrer vergleichsweise innovativen Ansätze bei der
Datenmodellierung und -haltung, aber auch wegen des Zuschnitts ihrer
Suchräume und der völlig anderen Herangehensweisen an nachträgliche
Erschließung und Erschließungsqualität, verdienen die Academic Search
Engines mehr Aufmerksamkeit durch Bibliothekar*innen.

Was können Discovery-Systeme heute von Academic Search Engines lernen?

\hypertarget{wissenszusammenhuxe4nge-besser-darstellen}{%
\section{Wissenszusammenhänge besser
darstellen}\label{wissenszusammenhuxe4nge-besser-darstellen}}

Ein wesentliches Unterscheidungsmerkmal von Academic Search Engines und
bibliothekarischen Recherchetools betrifft die Kontextualisierung von
Wissen. Während sich bibliothekarische Angebote darauf beschränken,
ähnliche Treffer auf Grundlage von ähnlicher Erschließung oder
Klickzahlen zu ermitteln, ist die Heranziehung von Zitationen bei allen
Academic Search Engines Standard. Auch wenn sich die Academic Search
Engines meistens nicht explizit darauf beziehen, ist der Science
Citation Index als erste interdisziplinäre \enquote{Suchmaschine}
diesbezüglich prägend für die gesamte neue Generation der hier
betrachteten Instrumente.\footnote{Small, Henry. 2018. \enquote{Citation
  Indexing Revisited: Garfield's Early Vision and Its Implications for
  the Future}. \emph{Frontiers in Research Metrics and Analytics} 3.
  \url{https://doi.org/10.3389/frma.2018.00008}}

Da in bibliothekarischen Katalogen und Discovery-Systemen in der Regel
sowohl der klassische Bibliotheksbestand als auch Artikeldaten enthalten
sind, wäre eine Heranziehung der beispielsweise inzwischen über Crossref
verfügbaren Zitationen nur für eine Teilmenge der Daten in diesen
Systemen relevant. Es scheint seitens der Anbieter von
bibliothekarischen Discovery-Systemen, zumindest derzeit, als zu große
Herausforderung angesehen zu werden, eine solche Inkonsistenz
verständlich zu vermitteln, vor allem an die bibliothekarische
Zielgruppe. Allerdings bleibt damit ein großes Potenzial ungenutzt, um
neben einer formalen und inhaltlichen Ordnung auch eine
\enquote{Bedeutungseinordnung} für Literatur zu erlauben. Grundlage
dafür könnten neben den Zitationen einerseits auch die Heranziehung von
Exemplarzahlen bei monografischer Literatur sein, wenn man annimmt, dass
diese Zahl ein Indikator für hohe Nachfrage und breite Nutzung ist.
Insgesamt sind die bibliothekarischen Suchsysteme, insbesondere die
Discovery-Systeme, bei der Erprobung und Evaluierung solcher Ansätze
sehr zurückhaltend.

Zu einer Bedeutungseinordnung von Treffermengen kann auch eine
Visualisierung von Treffermengen und den darin enthaltenen Bezügen
beitragen. Academic Search Engines benutzen Visualisierungen bereits
jetzt in weitaus stärkerem Maße, als dies bibliothekarische Tools tun --
allerdings häufig nur für eher eindimensionale Aspekte wie
Erscheinungsjahre, aber eben auch für Zitationsnetzwerke. Es ist
anzunehmen, dass neben Zitationen auch inhaltliche und ausgewählte
formale Aspekte wie eine institutionelle Zugehörigkeit geeignet sind, um
optisch aufbereitet dargestellt zu werden und auf diese Weise die
Erforschung von Wissenszusammenhängen zu ermöglichen.

Um die visuelle Darstellung komplexer Wissenszusammenhänge zu
ermöglichen, nutzen viele Academic Search Engines andere Datenmodelle
und Datenbank-Technologien als klassische Suchmaschinen. So setzen
beispielsweise OpenAlex, der Prototyp der fachspezifischen (und derzeit
nur prototypisch verfügbaren) Suchmaschine SoNAR oder Open Knowledge
Graph auf Graphdatenbanken und entsprechende Modellierungen der
Metadaten, um sowohl textuelle als auch grafische Zusammenhänge
aufzubereiten. Graphdatenbanken ermöglichen es, die einzelnen
Bestandteile der bibliografischen Beschreibung als Entitäten zu
definieren und die Zusammenhänge der Entitäten untereinander zu
beschreiben. Mit der Fokussierung auf die Zusammenhänge zwischen
Entitäten unterscheiden sie sich stark von den relationalen Datenbanken,
die hinter Bibliothekskatalogen stehen, oder den Suchmaschinenindices
wie Solr.

Zusammenhänge gibt es innerhalb bibliografischer Datensets in
vielfältiger Form: Es handelt sich dabei um formale Zusammenhänge, zum
Beispiel bei Artikeln, die Teile einer Zeitschrift sind, bei
Zugehörigkeiten von Personen zu Institutionen oder bei einzelnen Teilen
einer Schriftenreihe oder eines mehrbändigen Werkes. Vor allem sind
Bibliotheken aber stolz auf ihre inhaltliche Erschließung, mit der über
Schlagwörter oder Klassifikationen thematische Zusammenhänge erzeugt
werden. Eine verbesserte Ausnutzung dieser Erschließungsdaten innerhalb
von Discovery-Systemen wird immer wieder gefordert.\footnote{Heidrun
  Wiesenmüller. 2021. \enquote{Verbale Erschließung in Katalogen und
  Discovery-Systemen -- Überlegungen zur Qualität}. In \emph{Qualität in
  der Inhaltserschließung}, herausgegeben von Michael Franke-Maier, Anna
  Kasprzik, Andreas Ledl, und Hans Schürmann, 279--302. De Gruyter Saur.
  \url{https://doi.org/10.1515/9783110691597-014}} Mit den gängigen
Datenmodellierungen -- in der Regel einer Konvertierung von Daten aus
proprietären Bibliotheksmanagement-Systemen nach MARC -- werden sich die
Herausforderungen jedoch vermutlich nicht lösen lassen, da das
Datenmodell keine hierarchischen Beziehungen zulässt und Verknüpfungen,
zum Beispiel zu übergeordneten Begriffen, in der Regel über eigene
Identifikatoren der Herkunftssysteme erstellt werden. Die konsequente
Verwendung von Linked Open Data könnte die Verwendung der
Erschließungsdaten in Discovery-Systemen verbessern.\footnote{Pohl,
  Adrian, und Patrick Danowski. 2013. \enquote{Linked Open Data in der
  Bibliothekswelt: Grundlagen und Überblick}. In \emph{(Open) Linked
  Data in Bibliotheken}, herausgegeben von Patrick Danowski und Adrian
  Pohl, 1--44. De Gruyter Saur.
  \url{https://doi.org/10.1515/9783110278736.1}}

\hypertarget{zuschnitt-der-suchruxe4ume-uxfcberdenken}{%
\section{Zuschnitt der Suchräume
überdenken}\label{zuschnitt-der-suchruxe4ume-uxfcberdenken}}

Einer der wesentlichen Unterschiede zwischen Academic Search Engines und
klassischen Rechercheinstrumenten wie Bibliothekskataloge, Discovery
Systeme und Fachdatenbanken besteht darin, dass die Academic Search
Engines ihre Inhalte nicht selbst kuratieren, sondern automatisiert
beziehen. Allerdings erfolgt die Beschaffung von Literatur in
Bibliotheken auch nicht mehr ausschließlich auf Grundlage intellektuell
getroffener Auswahlentscheidungen, sondern im E-Medien-Bereich gleichsam
nach Marktlage (E-Book-Pakete, National- oder Allianzlizenzen).

Die meisten Academic Search Engines beziehen in irgendeiner Weise Daten
aus Crossref ein, das sich als Non-Profit-Organisation der Auszeichnung
von wissenschaftlicher Literatur mit eindeutigen Identifikatoren (DOIs)
verschrieben hat. Die Mitgliedschaft bei Crossref ist Content-Anbietern
im wissenschaftlichen Bereich vorbehalten, so dass der Datenbestand in
Crossref auf diese Weise auch als kuratiert gelten kann, wenngleich
natürlich auf einem anderen Niveau als es bei Erwerbungsentscheidungen
in Bibliotheken der Fall ist, da diesen in der Regel weitaus
differenziertere Qualitätskriterien zu Grunde liegen.

Die Datengrundlage von bibliothekarischen Discovery-Systemen bilden in
der Regel die Katalogs- und Bestandsdaten der eigenen Bibliothek, die
mit Suchmaschinentechnologie aufbereitet werden. Da die wenigsten
Bibliotheken diese Aufbereitung selbst erledigen können, nutzen sie
entweder gemeinschaftlich betriebene Indices wie den K10Plus Zentral
oder kommerzielle Angebote wie EDS.\footnote{Wikipedia. \enquote{EBSCO
  Discovery Service}.
  https://de.wikipedia.org/wiki/EBSCO\_Discovery\_Service} Die
Möglichkeiten der Einflussnahme auf die Inhalte in diesen Indices sind
für die einzelnen Bibliotheken ausgesprochen begrenzt. Die Aufnahme von
individuell gewünschten zusätzlichen Metadatenkollektionen ist aus
technischen und lizenzrechtlichen Gründen oftmals unmöglich oder mit
großen Hürden verbunden. Das bedeutet in der Regel, dass lokal durchaus
sehr bedeutsame Datenkollektionen nicht ohne Weiteres über die
Discovery-Systeme zugänglich gemacht werden können. Für Bibliotheken mit
nicht katalogisierten Altbeständen oder selbst erstellten Bibliografien
stellt das einen erheblichen Nachteil dar.

Die Bielefeld Academic Search Engine hingegen erlaubt es, dass
Content-Anbieter eigene Quellen liefern können, vorausgesetzt es handelt
sich dabei um zumindest teilweise frei zugängliche Volltexte. Eine
niedrigschwellige Öffnung der Discovery-Indices für das Einspielen von
Metadaten und/oder Volltexten wäre auch mit Blick auf
Forschungsinformationssysteme wünschenswert, die sich immer weiter in
der universitären Informationslandschaft etablieren und die
Discovery-Systeme um zwar nicht bibliografische, aber dennoch lokal
bedeutsame Daten bereichern können.

\hypertarget{offenheit-fuxfcr-automatisierte-erschlieuxdfung-erhuxf6hen}{%
\section{Offenheit für automatisierte Erschließung
erhöhen}\label{offenheit-fuxfcr-automatisierte-erschlieuxdfung-erhuxf6hen}}

Voraussetzung für die niedrigschwellige Öffnung der Discovery-Indices
für weitere Inhalte ist eine weitgehend maschinelle Vorverarbeitung und
Anreicherung der Daten. Allerdings ist bereits jetzt eine häufige Klage
von Bibliotheksmitarbeitenden, dass die aktuell im Einsatz befindlichen
Verarbeitungsmechanismen schlecht nachvollziehbar seien -- obwohl zum
Beispiel für den K10Plus einschlägige Dokumentationen offen
bereitstehen. Auch erste Versuche der Deutschen Nationalbibliothek zur
automatischen Erschließung wurden in der Bibliotheks-Community eher
kritisch beurteilt.\footnote{Wiesenmüller, Heidrun. 2018.
  \enquote{Maschinelle Indexierung am Beispiel der DNB : Analyse und
  Entwicklungsmöglichkeiten}. In \emph{o-bib. Das offene
  Bibliotheksjournal.} VDB, Bd. 5 Nr. 4 (2018), 141--153.
  \url{https://doi.org/10.5282/o-bib/2018H4S141-153}}

Die Kriterien für die Bemessung der Datenqualität sind bei Academic
Search Engines und Disco\-very-Systemen höchst unterschiedlich.
Bibliothekarische Systeme gehorchen den einschlägigen Richtlinien für
die formale Erfassung und setzen bei der inhaltlichen Erschließung
nahezu ausschließlich auf intellektuelle Verfahren. Im Gegensatz dazu
gilt beispielsweise bei Crossref, dass sich gute Datensätze vor allem
durch die Verfügbarkeit von Zitationen, offenen IDs wie ORCID, Abstracts
und Volltextlinks auszeichnen.\footnote{Meddings, Kirsty, und Anna
  Tolwinska. 2018. \enquote{How good is your metadata?}
  \url{https://www.crossref.org/blog/how-good-is-your-metadata/}}

Angesichts der Tatsache, dass es sich bei Academic Search Engines in der
Regel um Volltext-Suchmaschinen handelt, spielt die inhaltliche
Erschließung eine eher untergeordnete Rolle und wird hauptsächlich für
die Zuordnung von Werken zu Wissenschaftsgebieten genutzt. Diese
Zuordnung erfolgt über automatische Verfahren.\footnote{Vergleiche zum
  Beispiel Hook, Daniel W., Simon J. Porter, und Christian Herzog. 2018.
  \enquote{Dimensions: Building Context for Search and Evaluation}. In
  \emph{Frontiers in Research Metrics and Analytics} 3: 6.
  \url{https://doi.org/10.3389/frma.2018.00023}} Bei OpenAlex erfolgt
eine thematische Auszeichnung nach eigenen Konzepten und
Wikidata.\footnote{OpenAlex documentation. \enquote{Concept}.
  \url{https://docs.openalex.org/about-the-data/concept}} Die Bielefeld
Academic Search Engine setzt bereits seit etwa 2011 auf die
automatisierte Anreicherung der nachgewiesenen Daten durch Systemstellen
der Dewey Decimal Classification.\footnote{Waltinger, Ulli, Alexander
  Mehler, Mathias Lösch, und Wolfram Horstmann. 2011.
  \enquote{Hierarchical Classification of OAI Metadata Using the DDC
  Taxonomy}. In \emph{Advanced Language Technologies for Digital
  Libraries}, herausgegeben von Raffaella Bernardi, Sally Chambers,
  Björn Gottfried, Frédérique Segond, und Ilya Zaihrayeu, 6699:29--40.
  Lecture Notes in Computer Science. Berlin, Heidelberg: Springer Berlin
  Heidelberg. \url{https://doi.org/10.1007/978-3-642-23160-5_3}}

Eine Offenheit dafür, automatisierte inhaltliche Erschließung zu
betreiben, ist auch in einschlägigen Projekten großer wissenschaftlicher
Bibliotheken erkennbar.\footnote{Arndt, Susanne, Berrit Genat, und Mila
  Runnwerth. 2020. \enquote{Evaluierung von annif an der TIB - ein
  Werkstattbericht}. Zenodo.
  \url{https://doi.org/10.5281/zenodo.4304303}} Für eine breite Adaption
von maschinellen Erschließungsverfahren in Bibliotheken scheint jedoch
noch ein erheblicher Vertrauens- und Kompetenzaufbau notwendig zu sein,
der möglicherweise aber auch durch den Einsatz von Verfahrensweisen des
maschinellen Lernens angetrieben werden könnte. Diese Verfahrensweisen
erlauben eine kontinuierliche Weiterentwicklung der automatischen
Erschließung und haben das Potenzial, eine hohe Erschließungskonsistenz
zu erreichen -- was wiederum ein essentieller Bestandteil der
bibliothekarischen Qualitätskriterien darstellt. Gleichzeitig zeichnet
sich jedoch auch ab, dass automatisierter Verfahren nicht ausreichen
werden und es nach wie vor Bedarf an intellektueller Erschließung gibt.
\footnote{Auer, Sören, Allard Oelen, Muhammad Haris, Markus Stocker,
  Jennifer D'Souza, Kheir Eddine Farfar, Lars Vogt, Manuel Prinz,
  Vitalis Wiens, und Mohamad Yaser Jaradeh. 2020. \enquote{Improving
  Access to Scientific Literature with Knowledge Graphs}. In
  \emph{Bibliothek Forschung und Praxis} 44 (3): 516--29.
  \url{https://doi.org/10.1515/bfp-2020-2042}}

\hypertarget{fazit}{%
\section{Fazit}\label{fazit}}

Die Zielsetzungen und strategischen Ausrichtungen von Academic Search
Engines und bibliothekarischen Suchsystemen unterscheiden sich in vielen
Punkten. Es gibt jedoch einige augenscheinliche Möglichkeiten, wie beide
voneinander profitieren können.

Grundsätzlich ist die Innovationskraft auf Seiten der Academic Search
Engines höher als im Bereich der bibliothekarischen Discovery-Systeme.
Der Einsatz von automatisierten Verfahren für die Erschließung und
Anreicherung von Daten ist angesichts der großen Datenmengen bei den
meisten Angeboten selbstverständlich. Auch die Modellierung von Daten
als Wissensgraphen ist für den Bibliotheksbereich vielversprechend,
insbesondere dann, wenn auch Visualisierungen von Beziehungen gewünscht
sind. Entsprechende explorative Projekte gibt es mit SoNAR oder dem
Datendienst der SLUB Dresden bereits.\footnote{SLUB Dresden.
  \enquote{SLUB LOD API documentation}.
  \url{https://data.slub-dresden.de/}} Diese haben das Potenzial, die
derzeit eher evolutionär verlaufende Entwicklung der Discovery-Systemen
zu beschleunigen. Voraussetzung hierfür ist allerdings die Bereitschaft
von Bibliotheken, Technologien mit einem relativ geringen Reifegrad als
Chance zu verstehen und sich auf iterative Prozesse bei der
Weiterentwicklung einzulassen, die zudem zuvor erarbeitete
Nutzungsszenarien nicht außer Acht lassen sollte.

Unabhängig davon haben einige der frei verfügbaren Academic Search
Engines durchaus das Potenzial, an die Stelle von kommerziellen Lösungen
wie Web of Science oder Scopus treten zu können und damit ein
bibliotheksseitiges Bekenntnis zu Openness zu stärken. Die
interdisziplinäre Ausrichtung der meisten Academic Search Engines macht
sie für viele Fragestellungen in den zunehmend ebenfalls
interdisziplinären wissenschaftlichen Diskursen relevant. Ihre offenen
Schnittstellen ermöglichen es außerdem, Datensets herunterzuladen und
eigene Analysen durchzuführen, beispielsweise um Übersichtsartikel zu
identifizieren oder Zitationsnetzwerke zu erforschen.\footnote{Williams,
  Brett. 2020. \enquote{Dimensions \& VOSViewer bibliometrics in the
  reference interview}. In \emph{Code4Lib Journal}, Nr. 47.
  \url{https://journal.code4lib.org/articles/14964}}

Allerdings können Academic Search Engines ihrerseits auch von
Bibliotheken und ihren Erschließungswerkzeugen profitieren, ganz konkret
von den jahrzehntelang gewachsenen und als Linked Open Data vorliegenden
Normdateien. Eine kleine, aber hoffentlich wachsende Anzahl von
Klassifikationen wird als Linked Open Data veröffentlicht und könnte als
zusätzliches System der Wissensorganisation herangezogen
werden.\footnote{SkoHub: \url{https://skohub.io/}} Es ist verwunderlich,
dass die Academic Search Engines bislang kaum auf einschlägige Normdaten
aus dem bibliothekarischen Bereich zurückgreifen, insbesondere da
beispielsweise mit dem Virtual International Authority File\footnote{Virtual
  International Authority File: \url{https://viaf.org}} ein erstklassig
nutzbares Instrument für die Disambiguierung von Personennamen verfügbar
ist.

Die verschiedenen Ökosysteme im Internet -- seien es allgemeine soziale
Netzwerke, journalistische Angebote oder Plattformen für die Recherche
und Publikation von wissenschaftlicher Literatur und wissenschaftlichen
Daten -- werden in Zukunft noch stärker als bisher auf Interoperabilität
setzen müssen.\footnote{FAIR-Prinzipien:
  \url{https://www.go-fair.org/fair-principles/}} Auch kommerzielle
Systeme wollen mehr untereinander agieren, wie die Vision des
Metaversums zeigt. Bibliotheken und andere Informationsanbieter,
insbesondere die gemeinnützigen Academic Search Engines haben die
Möglichkeit, gemeinsam an einer offenen, den FAIR-Prinzipien genügenden
Vision von wissenschaftlichen Rechercheplattformen zu arbeiten und damit
einen Beitrag zu einer offenen digitalen Infrastruktur für die
Wissenschaft zu leisten.\footnote{Zuckerman, Ethan. 2020. \enquote{The
  Case for Digital Public Infrastructure}. Columbia University.
  \url{https://doi.org/10.7916/d8-chxd-jw34}} Beide sollten in einen
Austausch kommen, um diese Möglichkeit zu nutzen. In die
bibliothekarischen Discovery-Systeme ist in den vergangenen Jahren viel
Zeit und Geld geflossen, vor allem, um den Status quo des klassischen
OPACs zu erhalten -- was aber nicht der ursprünglichen Zielsetzung
entspricht und innerhalb der Discovery-Community kritisch reflektiert
wird.\footnote{Keßler, Kristof, Anne Christensen, Schrader Jarmo, und
  Jan F. Maas. 2019. \enquote{Discovery-Systeme, die ihrem Namen Ehre
  machen: Verbesserungspotenziale für bibliothekarische Suchmaschinen}.
  \url{https://nbn-resolving.org/urn:nbn:de:0290-opus4-162505}} Die
Academic Search Engines zeigen auf, wie Bibliotheken mit mehr
Innovationsfreude, Pragmatismus und Orientierung an Use Cases bessere
Discovery-Systeme schaffen können.

%autor
\begin{center}\rule{0.5\linewidth}{0.5pt}\end{center}

\textbf{Anne Christensen} ist Bibliothekarin und hat über 20 Jahre in
wissenschaftlichen Bibliotheken gearbeitet, zuletzt als stellvertretende
Direktorin der Bibliothek an der Leuphana Universität Lüneburg. Seit
2020 ist sie Beraterin und Projektmanagerin bei der Hamburger Firma
effective WEBWORK, die Open-Source-Systeme für Bildungseinrichtungen und
Bibliotheken implementiert. Anne Christensen ist außerdem als
Lehrbeauftragte in verschiedenen bibliothekarischen Ausbildungsgängen
tätig.

\end{document}
