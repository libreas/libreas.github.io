\begin{center}\rule{0.5\linewidth}{0.5pt}\end{center}

\textbf{Franziska Stanzel} (\url{https://orcid.org/0000-0003-0053-8604})
ist seit Oktober 2021 Teamleiterin des Open Access Monitors in der
Zentralbibliothek des Forschungszentrums Jülich. Sie absolvierte das
praktische Jahr des Bibliotheksreferendariates an der Bibliothek des kiz
der Universität Ulm und die Theorie an der BSB München. Zuvor studierte
sie Biochemie (Bachelor) an der Universität Bayreuth sowie Chemische
Biologie (Master of Science) an der Friedrich-Schiller-Universität Jena.

\textbf{Irene Barbers} (\url{https://orcid.org/0000-0003-2011-7444}) ist
Projektverantwortliche für den Open Access Monitor. Sie hat
wissenschaftliches Bibliothekswesen (Dipl.-Bibl.) und Bibliotheks- und
Informationswissenschaft (MA LIS) an der FH Köln studiert und ist
Leiterin des Fachbereichs Literaturerwerbung in der Zentralbibliothek
des Forschungszentrums Jülich. Sie ist Mitglied im Executive Committee
und im Board of Directors des Projekts COUNTER sowie Mitglied des
EZB-Beirats (2022--2025).

\textbf{Philipp Pollack} (\url{https://orcid.org/0000-0002-3660-5752})
ist für die technische Umsetzung des Open Access Monitors
verantwortlich. Er ist seit 2011 am Forschungszentrum Jülich angestellt.
Dort hat er zunächst ein duales Studium zum Mathematisch-technischen
Softwareentwickler absolviert und anschließend den Master of Science im
Fach Informatik an der Heinrich-Heine-Universität Düsseldorf erlangt.

\textbf{Barbara Lindstrot} (\url{https://orcid.org/0000-0003-2487-9040})
ist an der Pflege der Daten des Open Access Monitors beteiligt. Nach dem
Abschluss als Dipl.-Dokumentarin an der Fachhochschule Potsdam war sie
zunächst bei „d\&a Dokumentationsservice GmbH'' in Potsdam angestellt.
Seit 2009 arbeitet sie im Team Lizenzmanagement in der Zentralbibliothek
des Forschungszentrums Jülich.
