\textbf{Zusammenfassung:} Im Zusammenhang mit der
Open-Access-Transformation rückt die Auswertung großer Datenmengen immer
mehr in den Fokus von Bibliotheken, da die Anzahl der wissenschaftlichen
Publikationen beständig ansteigt. Diese stetig anwachsende Datenmenge
muss zuerst nutzbar gemacht werden, bevor fundierte Aussagen
beispielsweise zu einrichtungsbezogenen Publikationsoutputs möglich
sind. Hier setzt der Open Access Monitor (OAM) an, welcher als
Schnittstelle zur Zusammenführung diverser Quellsysteme wie Unpaywall,
Dimensions, Web of Science und Scopus fungiert. Dazu ist der OAM
strukturell dreigeteilt: Die Daten befinden sich in der Datenbank
(Backend), welche über die REST-Schnittstelle (API) abgefragt oder über
die Weboberfläche (Frontend) präsentiert und visualisiert werden können.

Durch die Nachnutzung einer Vielzahl an Quellsystemen müssen die Daten
homogenisiert werden, um vollständige Datenbestände ohne Dubletten zu
realisieren. Dafür müssen Zeitschriftentitel oder
Einrichtungsbezeichnungen vereinheitlicht werden, um die ursprünglichen
Einträge aus den Quellsysteme den entsprechenden Datensätzen im OAM
zuordnen zu können. Im Falle der Einrichtungsnamen werden diese mit
persistenten Identifikatoren (PID) angereichert. Für die Daten von
manchen Datenbanken können die dort hinterlegten
Einrichtungsnormierungen nicht direkt auf Organisations-Identifier
(ROR-IDs) gemappt werden, weshalb der Umweg über die Rohformen der
Affiliationsangaben der Autor*innen gewählt wird.

Dieses Mapping der Affiliationsangaben ist eine umfangreiche und
komplexe Aufgabe, da zum einen die gelieferten Angaben häufig nicht
eindeutig sind und zum anderen eine klare Trennung der Einrichtungen,
insbesondere bei Universitätskliniken, eine intellektuelle Bearbeitung
erfordert. Der hochkomplexe Vorgang, aus einer Vielzahl an Datenquellen
einen einheitlichen Datensatz zu generieren, wird im Beitrag aufgezeigt,
wobei ein besonderer Schwerpunkt auf die Normierungsprozesse sowie die
Vergabe der Open-Access-Kategorien gelegt wird.

Die Metadatenqualität bleibt eine beständige Herausforderung, gleiches
gilt für das Thema der Verfügbarkeit und Nachhaltigkeit der angebundenen
Quellsysteme. Die Anbindung offener Datenquelle wäre wünschenswert -- es
entspräche den Zielen der uneingeschränkten (Nach-)Nutzbarkeit der
OAM-Daten. Ob beispielsweise OpenAlex als nicht-kommerzielle Datenbank
als weiteres Quellsystem für den OAM in Frage kommt, wird abschließend
diskutiert.

\begin{center}\rule{0.5\linewidth}{0.5pt}\end{center}

\textbf{Abstract:} In the light of the Open Access transformation, the
analysis of large amounts of data is increasingly important for
libraries, whereas the number of scholarly publications is constantly
growing. Large amounts of data must first be made usable before any
substantiated analysis can be made, e.g.~regarding institution-related
publication outputs. This is where the Open Access Monitor (OAM) comes
in, which acts as an interface for merging data from various source
systems such as Unpaywall, Dimensions, Web of Science and Scopus. For
this purpose, the OAM is structurally divided into three parts: the
backend hosts the data, which can be queried via the API, and is
presented and visualized in the frontend.

All data, coming from various source systems, must be homogenized in
order to realize complete data sets without creating duplicates. Journal
titles or institution names have to be standardized to allow assigning
the original entries from the source systems to the corresponding data
records in the OAM. In the case of institution names, these are enriched
with persistent identifiers. Given the way the data is organized in some
of the source databases, the institution names cannot be mapped directly
to organization identifiers (ROR-IDs) in some cases. Therefore, the raw
forms of the author's affiliation information are used in the mapping
process.

Affiliation mapping is an extensive and complex task, since the data
provided are often ambiguous and at the same time a clear distinction of
institutions, especially in the case of university hospitals, requires
intellectual processing. The highly complex process of generating a
uniform data set from a multitude of data sources will be demonstrated,
with a special focus on the normalization processes as well as the
assignment of Open Access categories.

Metadata quality remains a constant challenge, as does the issue of
availability and sustainability of the connected source systems. The use
and integration of open data sources is generally desirable -- it would
be in line with the OAM's goal of unrestricted (re-) usability of the
OAM data. The pros and cons of using non-commercial databases are
discussed using OpenAlex as an example.
