\textbf{Kurzfassung}: «GOAL -- Unlocking the Green Open Access
Potential» (\url{https://opengoal.ch/}) ist die kollektive Bestrebung
einer Gruppe von wissenschaftlichen Bibliotheken zur Förderung des
bisher wenig beachteten Potentials von Grün Open Access in der Schweizer
Publikationslandschaft. Nicht oder nur marginal im Fokus der Nationalen
Open-Access-Strategie der Schweiz sind die Fachhochschulen und
Pädagogischen Hochschulen sowie kleine landessprachliche Verlage.
Zunehmend seltener anerkannt wird auch der Weg der
Open-Access-Zweitveröffentlichung oder «Grün Open Access». Es scheint,
als ob sich der Ball bei der Umsetzung von Open Access zunehmend im
Besitz der «Read \& Publish» Verträge und in der Förderung des Goldenen
Weges befindet. Die «grüne Alternative» führt ein wenig wahrgenommenes
Schattendasein am Rande des Spielfeldes und gilt als wenig attraktiv.
Offenbar können auch die Argumente der Nachhaltigkeit und der erhöhten
Sichtbarkeit nicht genügend überzeugen. Aus Sicht des Projektes GOAL
eröffnet der Grüne Weg jedoch einen in der Schweiz noch wenig genutzten
Handlungsspielraum, der sich jenseits der grossen Verlage oder bekannten
Universitäten befindet und ein nicht zu unterschätzendes Potential hat.
Das ist der Moment, in dem wir das Auge auf praxisorientierte
Zeitschriften richten, welche von kleinen Verlagen, Fachgesellschaften
oder öffentlichen Institutionen in den verschiedenen Landessprachen
publiziert werden. Diese Zeitschriften haben bisher noch wenig Beachtung
erhalten, obwohl sie eine entscheidende Rolle für Forschende an Fach-
sowie Pädagogischen Hochschulen spielen. Hier ist das Spielfeld noch
wenig besetzt und Open Access kann, unter Berücksichtigung der
Ressourcen der Redaktionen und Hochschulen, noch aufgebaut werden. In
unserem Beitrag stellen wir das Projekt vor, präsentieren die Ergebnisse
von anderthalb Jahren Arbeit und stellen diese zur Diskussion.

\textbf{Abstract}: ``GOAL -- Unlocking the Green Open Access Potential''
(\url{https://opengoal.ch/}) is the collective effort of a group of
academic libraries to promote the hitherto little-noticed potential of
Green Open Access in the Swiss publishing landscape. Universities of
applied sciences and of teacher education as well as small publishers in
vernacular languages have until now been only marginally in the focus of
the Swiss National Open Access Strategy. Similarly, the option of
self-archiving or ``Green Open Access'' is hardly recognized as a
solution. ``Read \& publish'' agreements and the ``Gold Open Access''
option seem to be the preferred and most promoted forms of implementing
Open Access. The ``green alternative'' lives thus a little-perceived
shadowy existence on the fringes of the field and is seen as having
little appeal. Apparently, the arguments of sustainability and increased
visibility are not convincing enough either. From the point of view of
the GOAL project, however, the ``green path'' opens up a scope for
action that is still little used in Switzerland, which lies beyond the
big publishing houses and well-known universities and has a potential
that should not be underestimated. It is time to turn our attention to
practice-oriented journals published by small publishers, professional
societies, or public institutions in the various national languages.
These journals have so far received little attention, even though they
play a crucial role for researchers at universities of applied sciences
as well as universities of teacher education. Here, the playing field is
still open and Open Access can be expanded, taking always into account
the resources of the editorial offices and universities. In our
contribution, we introduce the project, present the results of one and a
half years of work, and put them up for discussion.
