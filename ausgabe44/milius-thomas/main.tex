\documentclass[a4paper,
fontsize=11pt,
%headings=small,
oneside,
numbers=noperiodatend,
parskip=half-,
bibliography=totoc,
final
]{scrartcl}

\usepackage[babel]{csquotes}
\usepackage{synttree}
\usepackage{graphicx}
\setkeys{Gin}{width=.4\textwidth} %default pics size

\graphicspath{{./plots/}}
\usepackage[ngerman]{babel}
\usepackage[T1]{fontenc}
%\usepackage{amsmath}
\usepackage[utf8x]{inputenc}
\usepackage [hyphens]{url}
\usepackage{booktabs} 
\usepackage[left=2.4cm,right=2.4cm,top=2.3cm,bottom=2cm,includeheadfoot]{geometry}
\usepackage[labelformat=empty]{caption} % option 'labelformat=empty]' to surpress adding "Abbildung 1:" or "Figure 1" before each caption / use parameter '\captionsetup{labelformat=empty}' instead to change this for just one caption
\usepackage{eurosym}
\usepackage{multirow}
\usepackage[ngerman]{varioref}
\setcapindent{1em}
\renewcommand{\labelitemi}{--}
\usepackage{paralist}
\usepackage{pdfpages}
\usepackage{lscape}
\usepackage{float}
\usepackage{acronym}
\usepackage{eurosym}
\usepackage{longtable,lscape}
\usepackage{mathpazo}
\usepackage[normalem]{ulem} %emphasize weiterhin kursiv
\usepackage[flushmargin,ragged]{footmisc} % left align footnote
\usepackage{ccicons} 
\setcapindent{0pt} % no indentation in captions
\usepackage{xurl} % Breaks URLs

%%%% fancy LIBREAS URL color 
\usepackage{xcolor}
\definecolor{libreas}{RGB}{112,0,0}

\usepackage{listings}

\urlstyle{same}  % don't use monospace font for urls

\usepackage[fleqn]{amsmath}

%adjust fontsize for part

\usepackage{sectsty}
\partfont{\large}

%Das BibTeX-Zeichen mit \BibTeX setzen:
\def\symbol#1{\char #1\relax}
\def\bsl{{\tt\symbol{'134}}}
\def\BibTeX{{\rm B\kern-.05em{\sc i\kern-.025em b}\kern-.08em
    T\kern-.1667em\lower.7ex\hbox{E}\kern-.125emX}}

\usepackage{fancyhdr}
\fancyhf{}
\pagestyle{fancyplain}
\fancyhead[R]{\thepage}

% make sure bookmarks are created eventough sections are not numbered!
% uncommend if sections are numbered (bookmarks created by default)
\makeatletter
\renewcommand\@seccntformat[1]{}
\makeatother

% typo setup
\clubpenalty = 10000
\widowpenalty = 10000
\displaywidowpenalty = 10000

\usepackage{hyperxmp}
\usepackage[colorlinks, linkcolor=black,citecolor=black, urlcolor=libreas,
breaklinks= true,bookmarks=true,bookmarksopen=true]{hyperref}
\usepackage{breakurl}

%meta
%meta

\fancyhead[L]{S. Milius \& W. Thomas\\ %author
LIBREAS. Library Ideas, 44 (2023). % journal, issue, volume.
\href{https://doi.org/10.18452/28261}{\color{black}https://doi.org/10.18452/28261}
{}} % doi 
\fancyhead[R]{\thepage} %page number
\fancyfoot[L] {\ccLogo \ccAttribution\ \href{https://creativecommons.org/licenses/by/4.0/}{\color{black}Creative Commons BY 4.0}}  %licence
\fancyfoot[R] {ISSN: 1860-7950}

\title{\LARGE{Logical Methods in Computer Science -- Erfahrungsbericht über die Gründung einer internationalen Open-Access-Zeitschrift}}% title
\author{Stefan Milius \and Wolfgang Thomas} % author

\setcounter{page}{1}

\hypersetup{%
      pdftitle={Logical Methods in Computer Science – Erfahrungsbericht über die Gründung einer internationalen Open-Access-Zeitschrift},
     pdfauthor={Stefan Milius, Wolfgang Thomas},
      pdfcopyright={CC BY 4.0 International},
      pdfsubject={LIBREAS. Library Ideas, 44 (2023).},
      pdfkeywords={Diamond Open Access, scholar-led, Informatik, wissenschaftliche Zeitschrift, Proceedings, wissenschaftsgeleitetes Publizieren},
      pdflicenseurl={https://creativecommons.org/licenses/by/4.0/},
      pdfurl={https://doi.org/10.18452/28261},
      pdfdoi={10.18452/28261},
      pdflang={de},
      pdfmetalang={de}
     }



\date{}
\begin{document}

\maketitle
\thispagestyle{fancyplain} 

%abstracts
\begin{abstract}
\noindent
\textbf{Kurzfassung}: \emph{Logical Methods in Computer Science} ist
eine wissenschaftsgesteuerte Diamond-Open-Access-Zeitschrift im Gebiet
der Theoretischen Informatik. Wir berichten über die Entwicklung der
Zeitschrift seit ihrer Gründung in 2004 und stellen Herausforderungen
und Erfolgsfaktoren dar.

\begin{center}\rule{0.5\linewidth}{0.5pt}\end{center}

\textbf{Abstract}: \emph{Logical Method in Computer Science} is a
scholar-led diamond open access journal in theoretical computer science.
We report on the development of the journal since its foundation in 2004
and discuss challenges and success factors.
\end{abstract}

%body
\hypertarget{vorbemerkung}{%
\section{Vorbemerkung}\label{vorbemerkung}}

Um das Jahr 2000 herum wurde in der internationalen Community der
theoretischen Informatik mehr und mehr Unzufriedenheit über den enormen
Preisanstieg für wissenschaftliche Zeitschriften laut. Mehr und mehr
Bibliotheken waren durch die Preispolitik kommerzieller internationaler
Verlage gezwungen, Abonnements zu kündigen oder eigentlich unerwünschte
Pakete von Zeitschriften zu ordern. Damit verlor das Publizieren in
Zeitschriften, mit seinem Zweck einer freien weltweiten Verbreitung
wissenschaftlicher Erkenntnisse, zunehmend an Wirkung; die hohen
\enquote{Paywalls} behinderten den wissenschaftlichen Austausch. Es
formte sich langsam der Gedanke, Zeitschriften nach einem neuen Modell
zu etablieren, das man \enquote{wissenschaftsgesteuert} nennen konnte
und deren normalerweise digitaler Bezug nur geringe oder gar keine
(jedenfalls angemessene) Kosten verursachen sollte. Zum Beispiel verließ
im Jahre 2004 das Editorial Board der international führenden
Zeitschrift \emph{Journal of Algorithms} nahezu geschlossen den Verlag
Elsevier und gründete in der amerikanischen \emph{Association of
Computing Machinery (ACM)} die neue Zeitschrift \emph{Transactions of
Algorithms}. In einem komplementären Zweig der theoretischen Informatik,
in welchem Methoden der Logik, der Programmierung und der Verifikation
von Programmen im Vordergrund stehen, wurde zur gleichen Zeit eine
völlig neue Zeitschrift aus der Taufe gehoben: \emph{Logical Methods of
Computer Science} (LMCS), \url{https://lmcs.episciences.org/}. Sie ist
eine Diamond Open Access Zeitschrift, die durch einen gemeinnützigen
Verein, ansässig in Deutschland, getragen wird und die seit ihrer
Gründung im Jahre 2004 inzwischen eine international führende Position
erreicht hat.

Im vorliegenden Bericht zeichnen wir nach, wie diese Erfolgsgeschichte
von LMCS möglich wurde, welche Schwierigkeiten zu überwinden waren und
welche Perspektiven (auch für ähnliche Vorhaben in anderen
Wissenschaftsfeldern) wir sehen.

\hypertarget{vor-der-gruxfcndung-der-zeitschrift}{%
\section{Vor der Gründung der
Zeitschrift}\label{vor-der-gruxfcndung-der-zeitschrift}}

Die Diskussionen im Vorfeld der Gründung von LMCS erstreckten sich über
einige Jahre. Man kann drei Aspekte herausheben, die entscheidend für
den späteren Erfolg waren:

\begin{itemize}
\tightlist
\item
  Der allgemeine Wunsch der wissenschaftlichen Community, dass die
  damals aktuelle Situation hinsichtlich Publikationen in Zeitschriften
  so nicht weiterbestehen sollte,
\item
  eine Person mit Initiative, großer Entschlusskraft, Begeisterung und
  Hartnäckigkeit, die gemeinsam mit einem kleinen Team als tragende
  Säule fungierte, mit bescheidener aber eben hinreichender technischer
  Infrastruktur,
\item
  die Gewinnung von international renommierten und maßstabsetzenden
  Personen als Grün\-dungs-Board und die Etablierung eines exzellenten
  ersten Editorial Board.
\end{itemize}

Der erstgenannte Punkt war leicht zu erkennen. Man musste nur in den
Kaffeepausen damals stattfindender wissenschaftlicher Konferenzen
herumhören, um die Unzufriedenheit mit den Effekten wahrzunehmen, welche
die Kommerzialisierung und Konzentration der global agierenden
Wissenschaftsverlage mit sich brachten.

Entscheidender war, dass es einen energischen Initiator gab: Professor
Jiří Adámek, Inhaber des Lehrstuhls für Theoretische Informatik an der
Technischen Universität Braunschweig. Mit seinem Team, das auf den
einschlägigen Konferenzen für LMCS Werbung machte, schulterte er die
enormen Aufgaben, die nicht zuletzt auf technischer Ebene in den ersten
Jahren anstanden.

Schließlich galt es, führende, international hochangesehene
Persönlichkeiten im Feld \enquote{Logic in Computer Science} für das
Projekt zu gewinnen. Es gelang, das Gründungs-Board mit Dana S. Scott
(USA), Moshe Vardi (USA), Gordon Plotkin (Großbritannien) neben Jiří
Adámek (Deutschland) zu besetzen. Das war ein wichtiger Schritt, denn
diese konnten durch ihre Empfehlungen und Kontakte ein erstes Editorial
Board von LMCS zusammenstellen, das die Community sofort als
\enquote{exzellente Adresse} akzeptierte. Diese Anfangsreputation (auch
dokumentiert durch konsequente Ablehnung mittelmäßiger Arbeiten) war die
Grundlage für den hohen Anspruch an wissenschaftliche Qualität, den LMCS
von Beginn an vertrat und der dann auch die internationale Attraktivität
von LMCS begründete.

\hypertarget{die-anfangsphase}{%
\section{Die Anfangsphase}\label{die-anfangsphase}}

Beginnend mit dem 2. Quartal 2005 erschien LMCS zunächst mit 16 Artikeln
und eines Special Issue.

Die Prominenz des ersten Editorial Boards und der ersten Autor/inn/en
war aber nur eine Ebene von LMCS. Ein anderer Aspekt betraf die
Herausbildung einer funktionsfähigen technischen Infrastruktur. Von
Beginn an wurde LMCS als Overlay von arXiv.org konzipiert, also auf
Basis des Publikationsservers ArXiv der Cornell University und wie eine
klassische Zeitschrift mit \enquote{Volumes} und \enquote{Issues}
organisiert, um auch durch die Zitationen die Akzeptanz bei der
Community zu erreichen. Vor der eigentlichen Publikation stand aber als
kritische Hürde die Organisation des Prozesses der Begutachtung
(Reviewing). Dieser Prozess verlangt die Koordination des
Zusammenwirkens von Autor/inn/en, jeweils mindestens eines Mitglieds des
Editorial Board (in Konfliktfällen inklusive Editor-in-Chief), außerdem
zweier oder mehr mit der Begutachtung befasster Personen, die alle eine
gemeinsame technische Plattform brauchen. Dabei geht es nicht nur um den
Austausch von Manuskripten und Gutachten, sondern auch um Mahnungen,
Begleitkorrespondenz und dergleichen. Anfangs wurde eine Onlineplattform
basierend auf dem Open Journal System (OJS) von 2004 beim Lehrstuhl von
Jiří Adámek (unter anderem mit eigenem Peer-Review-System, Unterstützung
von Special Issues) um- und weiterentwickelt. In den Anfangsjahren war
die Arbeit an dieser Plattform eine große Herausforderung. Das Team von
Jiří Adámek verbrachte zahllose Stunden (auch schlaflose Nächte) mit
dieser Aktivität. Nicht zuletzt musste trotz strenger Auflagen, welche
die Autor/inn/en bei der Formatierung von Artikeln in LaTeX zu beachten
hatten, das finale Layouting durch Mitarbeitende am Lehrstuhl Adámek
erfolgen.

Dennoch: Das stetig wachsende Renommé von LMCS gab allen Beteiligten die
Zuversicht, dass hier etwas mit einem über den Tag hinaus einen
bleibenden Wert entstand.

Eine wesentliche Komponente der Akzeptanz von LMCS waren die Special
Issues. Ein Special Issue besteht aus einer Auswahl von Arbeiten, die
bei einer internationalen Konferenz mit sehr guter Bewertung akzeptiert
wurden, dann von den Autor/inn/en erweitert und vervollständigt einem
gründlichen zweiten Reviewing unterzogen werden sowie nach erneuter
Annahme als Zeitschriftenartikel publiziert werden. Die Herausgabe eines
Special Issue wird meistens durch die Programmkomiteevorsitzenden der
jeweiligen Konferenz übernommen -- gegebenenfalls in Zusammenarbeit mit
einem regulären Mitglied des Editorial Boards der Zeitschrift. Dieses
Modell der Veröffentlichung von erweiterten Versionen ausgewählter
Konferenzartikel in einer Zeitschrift ist der Standard innerhalb der
Informatik. Eine Einladung zu einem Special Issue gilt bei Autor/inn/en
als sehr prestigeträchtig.

In 2005 wurde für das erste Special Issue von LMCS die Konferenz
\emph{Logic in Computer Science} (LICS) gewonnen, die führende Konferenz
in diesem Forschungsfeld. Nachdem die ersten Special Issues erfolgreich
veröffentlicht worden waren, wurde klar, dass sie für die Zeitschrift
ein großer Gewinn sein würden -- aber auch zusätzlichen Aufwand
bedeuteten. Daher wurde ab 2007 die Funktion
\enquote{Special-Issue-Editor} geschaffen, die von Benjamin Pierce sehr
engagiert ausgefüllt wurde. Für die Zeitschrift bringen die Special
Issues die besten Artikel ein, was insbesondere auch durch die
Vorauswahl durch das Programmkomitee einer kompetitiven Konferenz
bedingt ist. Allerdings besteht hier auch die Gefahr der Verwässerung
der hohen Qualität der Zeitschrift, denn es gibt neben den großen
führenden Konferenzen noch viele weitere und zudem noch kleine
beziehungsweise spezialisierte Workshops, die auch gern erweiterte
Versionen von Artikeln in einer Zeitschrift unterbringen. Hier war es
für die Erhaltung der hohen Reputation der Zeitschrift wichtig, Grenzen
zu setzen. Bei LMCS führte das ziemlich schnell zu der Regel, Special
Issues nur von größeren kompetitiven Konferenzen anzunehmen und
Workshops prinzipiell auszuschließen. Die Rolle des
Special-Issue-Editors und des Boards bei der Auswahl von Konferenzen und
Special Issues ist daher ein nicht zu unterschätzender Erfolgsfaktor für
die Zeitschrift. Nachdem Benjamin Pierce 2014 von dieser Rolle
zurückgetreten war, wurde sie bis 2019 von Stefan Milius und seitdem
durch zwei Personen, Brigitte Pientka und Fabio Zanasi, ausgefüllt.

Technisch wurden die Special Issues als Overlay über die regulären
Issues realisiert. Das bedeutet, dass jeder in einem Special Issue
eingereichte Artikel sofort nach Akzeptanz zunächst in dem dann
laufenden regulären Volume und Issue erscheint und somit nicht auf die
anderen Artikel des Special Issues warten muss. Nach Abschluss des
Peer-Review-Prozesses für alle Einreichungen eines Special Issues
erscheint dieses dann auf einer speziellen Webseite, die alle
enthaltenen Artikel verlinkt und an dieser Stelle zusammenbindet.

\hypertarget{konsolidierung-und-schritte-der-institutionalisierung}{%
\section{Konsolidierung und Schritte der
Institutionalisierung}\label{konsolidierung-und-schritte-der-institutionalisierung}}

Nach der Phase der Etablierung von LMCS und der Akzeptanz als
hochqualitative Zeitschrift durch die Community stellten sich im Laufe
der Jahre eine Reihe von Herausforderungen.

Ein wichtiger Punkt war die Indexierung der Zeitschrift. Dass die
Zeitschrift in die Indizierungsdienste DBLP\footnote{Das Digital
  Bibliography \& Library Project (DBLP, \url{https://dblp.org/}) ist
  eine Onlinebibliografie wissenschaftlicher Publikationen
  (hauptsächlich Konferenz- und Journalbeiträge) aus der Informatik.
  DBLP wird betrieben vom Schloss Dagstuhl -- Leibniz-Zentrum für
  Informatik.}, DOAJ\footnote{Das Directory of Open Access Journals
  (DOAJ, \url{https://www.doaj.org/}) ist das international wichtigste
  Verzeichnis für wissenschaftliche, qualitätsgesicherte
  Open-Access-Zeitschriften. Das DOAJ wird betrieben von der
  Non-Profit-Organisation IS4OA.}, Mathematical Reviews und dem
Zentralblatt Math\footnote{Mathematical Reviews und das Zentralblatt für
  Mathematik (heute nur kurz zbMATH) sind Referatorgane auf dem Gebiet
  der Mathematik. Die Mathematical Reviews werden von der American
  Mathematical Society herausgegeben (Online-Zugang kostenpflichtig
  \url{https://mathscinet.ams.org/}), zbMATH wird von der European
  Mathematical Society, dem Fachinformationszentrum Karlsruhe und der
  Heidelberger Akademie der Wissenschaften herausgegeben (kostenfreier
  Zugang \url{https://zbmath.org/}).} aufgenommen wurde, war
verhältnismäßig einfach zu erreichen. Schwieriger gestaltete sich die
Aufnahme in das Web of Science oder Scopus. Hier ist zunächst
anzumerken, dass für die Aufnahme einer Zeitschrift in diese Datenbanken
mindestens fünf Jahre an Veröffentlichungshistorie vorliegen müssen.
Eine Aufnahme kam für LMCS daher frühestens ab 2009 in Frage. Das
Problem hierbei ist nun, dass es für einen Großteil der Autor/inn/en,
zum Beispiel in den südeuropäischen Staaten, bei der Beantragung von
Forschungsgeldern wichtig ist, ihre vorherigen Veröffentlichungen
möglichst in Zeitschriften und Konferenzen mit hohem Impact-Faktor im
Web of Science untergebracht zu haben. Diese Autor/inn/en stehen daher
gezwungenermaßen der Einreichung ihrer Arbeiten bei noch jungen
Zeitschriften kritisch gegenüber oder sehen ganz davon ab. LMCS schaffte
es im Jahr 2010, sowohl im Web of Science als auch Scopus aufgenommen zu
werden. Allerdings wurde der damalige Auswahlprozess von den
Verantwortlichen der Zeitschrift als intransparent empfunden. Weder
waren die Kriterien, nach denen eine Auswahl erfolgte, klar, noch wer an
der Entscheidung beteiligt war. In der Tat wurde die Aufnahme von LMCS
ohne Angabe von Gründen zunächst abgelehnt. Die Aufnahme erfolgte
schließlich bei einer erneuten Bewerbung ein Jahr später.

Aus unserer Sicht ist nicht hinnehmbar, dass für wissenschaftsgetriebene
Diamant Open Access Zeitschriften eine Aufnahme in Indexierungsdienste
kommerzieller Verlage von intransparenten Prozessen abhängt. Es ist
dringend anzuraten, dass die Politik in den entsprechenden Ländern hier
umsteuert und bei der Bewertung von Wissenschaftler/inne/n (etwa für die
Vergabe von Forschungsgeldern) nur Indexierungsdienste berücksichtigt,
die transparente, wissenschaftsgesteuerte Verfahren anwenden.

Eine weitere wichtige Herausforderung war, dass mit dem Ruhestand des
Zeitschriftgründers Professor Adámek der Weiterbetrieb der Zeitschrift
in der zweiten Hälfte der 2010er Jahre konkret gefährdet war. Die
Webseite der Zeitschrift und damit sie selbst -- als eine reine
Online-Zeitschrift -- lag auf den Servern an seinem Lehrstuhl, wo auch
das Peer-Review-System lief. Dieses beruhte zwar auf der
Standardsoftware OJS, war aber im Laufe der Jahre durch einen Entwickler
am Lehrstuhl immer weiter entwickelt worden und am Ende nur noch durch
diese eine Person wartbar. Wie bereits erwähnt, erfolgte auch das
Layouting der akzeptierten Artikel durch einen Mitarbeiter des
Lehrstuhls. Zuletzt existierten noch einige bescheidene finanzielle
Mittel, die der Zeitschrift im Laufe der Jahre dankenswerterweise durch
eine größere Forschungseinrichtung und zwei Organisationen gespendet
worden waren: dem Centrum Wiskunde \& Informatica (CWI) Amsterdam, dem
Verein, der die European Joint Conferences on Theory and Practice of
Software (ETAPS) trägt, und der European Association for Theoretical
Computer Science (EATCS). Diese Mittel wurden am Lehrstuhl verwaltet.

Um der Gefahr des Wegbrechens der gesamten technischen Infrastruktur der
Zeitschrift sowie aller finanziellen Mittel zu entgehen, wurde 2014 als
erster Schritt ein gemeinnütziger Verein gegründet, dessen Mitglieder
die Mitglieder des damaligen Governing Boards der Zeitschrift waren.
Dieses war schon einige Jahre zuvor um vier weitere Personen erweitert
worden: Luca Aceto (Island), Rajeev Alur (USA), Prakash Panangaden
(Kanada) und Wolfgang Thomas (Deutschland). Der Verein konnte zunächst
die finanziellen Mittel der Zeitschrift aufnehmen und als juristische
Person auch als Herausgeber der Zeitschrift fungieren. Zudem trat mit
Professor Lars Birkedal in 2014 ein neuer Editor-in-Chief (EiC) auf den
Plan, wobei gleichzeitig die Rolle des EiC reguliert wurde. Das Amt des
EiCs ist seitdem auf höchstens sechs Jahre begrenzt, und er/sie wird für
jeweils drei Jahre von den Mitgliedern des Governing Boards gewählt.
Diese Begrenzung hat mehrere Vorteile. Zum einen ist es praktisch
unmöglich, jemanden für diese ehrenamtliche und recht arbeitsintensive
Aufgabe zu gewinnen, wenn die zeitliche Perspektive potentiell das
gesamte weitere Berufsleben umfasst. Zum anderen ist es auch für die
Zeitschrift wichtig, dass das verantwortliche Personal rotiert. (Dies
steht im Kontrast zu Zeitschriften kommerzieller Verlage, bei denen die
EiCs mit guten Gehältern bezahlt werden und -- wie viele Beispiele
zeigen -- jahrzehntelang im Amt verbleiben.)

Das Ziel muss sein, dass die Zeitschrift als \enquote{Community Effort}
von eben dieser wissenschaftlichen Gemeinschaft verstanden wird und
nicht als Projekt weniger Enthusiasten. Gleichzeitig muss aber auch eine
strenge Qualitätskontrolle bei Nachbesetzungen des EiCs und im Editorial
Board aufrecht erhalten werden. In diesem Sinne ist auch die
Umstrukturierung des Governing Boards in 2022 zu verstehen, die auf
Initiative des aktuellen EiC Stefan Milius, der seit 2020 im Amt ist,
zurückgeht. Heutzutage besteht das Governing Board der Zeitschrift aus
21 Mitgliedern. Gut die Hälfte repräsentieren derzeit Steuerungskomitees
führender Konferenzen, die den Themenkreis der Zeitschrift berühren und
mehr oder weniger oft ihre Special Issues in ihr publizieren. Daneben
sind auch die vorherigen und aktuellen Managing Editoren (EiC und
Special Issue-Editoren) sowie die ursprünglichen Governing
Board-Mitglieder mit von der Partie, die aber im Laufe der nächsten
Jahre nach einem vereinbarten Schema schrittweise zurücktreten werden.
Auf diese Weise wird ein gewisser Ausgleich zwischen institutionellem
Langzeitgedächtnis und gleichzeitiger Erneuerung geschaffen. Wichtig ist
dabei insbesondere, dass die Mitgliedschaften und ihre Länge im
Governing Board nunmehr reguliert und die Regeln auch schriftlich
niedergelegt sind. Dies war aus unserer Sicht ein weiterer wichtiger
Schritt zur Professionalisierung der Zeitschrift. Diese Schritte der
Erweiterung und Erneuerung waren für die festere Verankerung in der
Community sehr wichtig.

Das zuvor erwähnte drohende Wegbrechen der technischen Infrastruktur war
Mitte der 2010er Jahre allerdings das drängendste Problem, und
entsprechend musste es während Lars Birkedals erster Amtszeit als EiC
auch prioritär angegangen werden. Für eine Weile bestand der Plan, dass
die Zeitschrift zukünftig vom Leibniz-Zentrum für Informatik in Dagstuhl
gehostet wird. Dort werden seit 2008 sehr erfolgreich die \emph{Leibniz
International Proceedings in Informatics} (LIPIcs) herausgegeben -- eine
Reihe, die Proceedings ausgewählter führender Konferenzen in der
Informatik publiziert.\footnote{Vergleiche M. Herbstritt, W. Thomas,
  LIPIcs -- an Open Access Series for International Conference
  Proceedings, in: ERCIM News No. 107, Sept 2016,
  \url{https://ercim-news.ercim.eu/en107/r-s/lipics-an-open-access-series-for-international-conference-proceedings}.}
Dagstuhl stellt hier die gesamte Publikationsinfrastruktur basierend auf
OJS zur Verfügung inklusive eines hervorragenden technischen Supports.
Allerdings ist die Reihe nicht Diamond Open Access -- es gibt bei den
LIPIcs Autorengebühren, die mit weniger als Hundert Euro pro Artikel
zugegebenermaßen relativ gering sind. Für Konferenzen stellt dies daher
kein allzu großes Problem dar, da diese Autorengebühren relativ leicht
in den Teilnahmegebühren der Konferenz verschwinden. Für eine
Zeitschrift sieht dies aber anders aus, da hier die Autorengebühren auch
wirklich von den Autor/inn/en übernommen werden müssten. Im LMCS
Governing Board gab es daher große Bedenken, ob eine Migration nach
Dagstuhl und die damit einhergehende Einführung von Autorengebühren --
wie vernünftig und niedrig sie im Vergleich zu denen von bekannten
kommerziellen Verlagen auch sein mochten -- nicht Autor/inn/en in großer
Zahl der Zeitschrift entfremden würde. Es war und ist innerhalb der
Theoretischen Informatik nach wie vor unüblich und bei Autor/inn/en
wenig akzeptiert, Autorengebühren zu entrichten. Hinzu kommt, dass es
für etliche Autor/inn/en keine Unterstützung für Autorengebühren durch
ihre Institutionen gibt.

Glücklicherweise ergab sich, vermittelt durch Professor Claude Kirchner,
dem Direktor für Wissenschaft und Technik des \emph{Institut National de
Recherche en Informatique et en Automatique} (INRIA) in Frankreich, eine
Alternative -- die Episciences-Platform. Jene Plattform für Diamond Open
Access Zeitschriften wurde in Frankreich seit den frühen 2010er Jahren
entwickelt. Dies geschah analog wie bei LMCS auf Initiative einzelner
Persönlichkeiten. Aber anders als LMCS wurde Episciences gleich zu
Beginn an großen zentralistisch organisierten Institutionen angelegt,
nämlich dem \emph{Centre pour la Communication Scientifique Directe}
(CCSD) und dem INRIA. Für LMCS ergab sich damit ein Angebot, das sowohl
das Hosting der Zeitschrift als auch personelle Ressourcen für
Systementwicklung und technischen Support umfasst. Dies ermöglichte es
LMCS, das Diamond Open Access Modell ohne die Erhebung von
Autorengebühren aufrecht zu erhalten, als die Zeitschrift dann in 2016
zu Episciences umzog. Natürlich war die Migration vom alten OJS-System
am Braunschweiger Lehrstuhl nach Episciences aufwändig. In Episciences
musste beispielsweise erst die Möglichkeit von Special Issues geschaffen
werden. Die Mitarbeitenden bei Episciences, die der Zeitschrift die
Registrierung der Digital Object Identifier (DOI) der Artikel abnahmen,
mussten sich erst entsprechende technische Unterstützung schaffen. Das
Peer-Review-System hatte etliche Schwachstellen, die erst im Laufe der
Jahre verbessert wurden und teilweise immer noch werden. Wir müssen hier
aber betonen, dass es trotz aller (anfänglichen) Schwierigkeiten mit
Episciences dort immer ein offenes Ohr für unsere Vorschläge gibt und im
Rahmen der auch dort begrenzten Ressourcen immer versucht wird, zeitnah
eine Lösung zu finden und das System zu verbessern und
weiterzuentwickeln.

Mit der Migration zu Episciences ergab sich auch eine kleine, aber
wichtige Änderung beim Overlay-Konzept beziehungsweise dem
Zusammenwirken von LMCS und arXiv. Vor der Migration war es für die
Autor/inn/en erst nach der Akzeptanz eines eingereichten Artikels
verpflichtend, diesen auch bei arXiv zu veröffentlichen. Bei Episciences
ist es aber Teil des Konzeptes, dass dies vor Einreichung passiert und
somit alle Versionen eines Artikels, die während der Begutachtung
entstehen, auch öffentlich sichtbar werden und somit der Prozess mehr
Transparenz gewinnt. Zudem ist damit die Hoffnung verbunden, dass
bereits die eingereichten Artikel eine noch höhere Qualität aufweisen.
Bei LMCS hatten die Gründer allerdings bewusst entschieden, nur die
finale akzeptierte Version auf arXiv und dann in der Zeitschrift zu
veröffentlichen, ganz so wie es auch bei den klassischen gedruckten
Zeitschriften der Fall ist. In der Tat gab es in Folge der Migration zu
Episciences einige wenige Male den Wunsch von Autor/inn/en, dass nur die
finale akzeptierte Version auf arXiv veröffentlicht würde. Im Laufe der
Jahre wurde dieser Wunsch allerdings immer seltener; er wurde nun schon
seit einiger Zeit nicht mehr geäußert. Das hat sicher auch damit zu tun,
dass es sich in der Community, gerade bei jüngeren Autor/inn/en,
weitgehend durchgesetzt hat, Preprints zügig nach ihrer Fertigstellung
sowieso bei einem öffentlichen Repository wie arXiv einzustellen.

Ein weiterer, für die Zeitschrift sehr wichtiger, Punkt ist das
Layouting. In der theoretischen Informatik liefern Autor/inn/en ihre
Artikel am Ende des Peer-Review-Prozesses im Grunde druckreif in LaTeX
im Journal-Stil gesetzt ab. Das Layouting besteht dann lediglich aus
einer Kontrolle des Schriftsatzes und der Einhaltung des
Zeitschriftenstils sowie darin, Metadaten wie Volume/Issue Nummer und
DOI einzupflegen und dann die Endversion eines Artikels nach Genehmigung
durch die Autor/inn/en in der Zeitschrift zu veröffentlichen. Während
diese Aufgabe bis 2016 durch einen Mitarbeiter im Team von Jiří Adámek
mit Unterstützung durch das damalige OJS-basierte System geschah,
mussten ab dem Zeitpunkt der Migration Freiwillige gefunden werden, die
diese Aufgabe ehrenamtlich übernahmen. Zudem bestand zunächst gar kein
und auch später kein hundertprozentig passender Support für das
Layouting durch die Episciences-Plattform. Die Aufgabe wurde dann
letztlich von einigen Doktoranden und jungen PostDocs übernommen, die
mit viel Enthusiasmus für die Sache des Diamond Open Access der
Zeitschrift zu Hilfe kamen, und das, obwohl ein Posten als Layout-Editor
einen bei weitem nicht so prestigeträchtigen Punkt für einen
akademischen Lebenslauf darstellt wie beispielsweise die Mitgliedschaft
im Editorial Board einer Zeitschrift. Die Layout-Editoren haben im Laufe
der Jahre ihre Aufgaben durch eine Reihe von selbst entwickelten
Softwarewerkzeugen partiell automatisiert. Zuletzt haben fünf
freiwillige Layout-Editoren bei LMCS geholfen, bis diese vor kurzem ein
wenig durch die Einstellung einer studentischen Hilfskraft entlastet
werden konnten. Die Hilfskraft wird momentan durch Mittel des
Lehrstuhls, an dem der jetzige EiC beschäftigt ist, finanziert. In
unseren Augen ist es eine für unser Wissenschaftssystem beschämende
Tatsache, dass derartige kleine (man mag auch sagen: lächerliche) Summen
für die Infrastruktur wertvoller wissenschaftlicher Aktivitäten bisher
nicht angemessen und nachhaltig finanziert werden können.

\hypertarget{lessons-learned-die-aktuelle-situation-und-herausforderungen}{%
\section{Lessons learned, die aktuelle Situation und
Herausforderungen}\label{lessons-learned-die-aktuelle-situation-und-herausforderungen}}

Insgesamt ist die Entwicklung von LMCS bis hierher eine
Erfolgsgeschichte. Die Zeitschrift veröffentlicht aktuell zwischen 100
und 120 Artikel im Jahr, ist damit bei weitem die größte Zeitschrift bei
Episciences und eine der international führenden im Bereich ihres
Themenspektrums. Dies wurde durch mehrere Faktoren möglich: Zunächst
durch das Engagement einer kleinen Gruppe von anfänglichen Gründern
sowie die Tatsache, dass in einer Zeit, als es praktisch noch keine
sonstigen Finanzierungsmöglichkeiten für Diamond Open Access gab, ein
deutscher Lehrstuhlinhaber seine Ressourcen dafür geopfert hat. Ganz
wesentlich war aber auch das konstante Hochhalten der wissenschaftlichen
Qualität auf allen Ebenen, besonders bei der Auswahl der Mitglieder des
Editorial Boards und bei der Begutachtung der eingereichten Arbeiten. In
diesem Sinne erwies sich für LMCS die Veröffentlichung der Special
Issues von führenden Konferenzen des Fachgebietes als wichtiges, wenn
nicht das wichtigste Zugpferd. Dies ist aber ein sehr fachspezifisches
Phänomen, das es so in anderen Wissenschaftsgebieten und selbst in
benachbarten Gebieten wie der Mathematik unseres Erachtens nicht gibt.

Technisch ist es heutzutage viel einfacher als im Jahr 2004, eine
wissenschaftsgetriebene Zeitschrift aus der Taufe zu heben. In der
theoretischen Informatik ist dies kürzlich wieder durch die Gründung der
Zeitschrift \emph{TheoretiCS},
\url{https://theoretics.episciences.org/}, geschehen. Deren
Themenspektrum ist breiter als das von LMCS und umfasst die gesamte
theoretische Informatik, also auch Algorithmik, Komplexitätstheorie und
so weiter. Hier findet nun auch eine gegenseitige Befruchtung statt.
Einerseits hat TheoretiCS von den Erfahrungen, die wir mit LMCS gemacht
haben (zum Beispiel mit Episciences, der Vereinsgründung, mit der
Indexierung) bei der Gründung sehr profitiert. Andererseits gibt es
Aspekte bei TheoretiCS, die auf LMCS zurückwirken, wie zum Beispiel die
neue Zusammensetzung des Governing Boards. TheoretiCS probiert auch
einige ganz neue Ideen aus, beispielsweise einen neuartigen zweistufigen
Peer-Review-Prozess, und weicht somit von den klassischen der Community
bekannten Prozessen der traditionellen Zeitschriften ab. Diese
Möglichkeiten sind sicher auch auf die Pionierarbeit von LMCS
zurückzuführen, nach der in der Community die Existenz
wissenschaftsgetriebener reiner Onlinezeitschriften nun ein Stück
Normalität geworden ist.

Die Entwicklung von LMCS zeigt, dass zentrale
Publikationsinfrastrukturen wie Episciences oder die Infrastruktur in
Dagstuhl (wenn es auch nicht Diamond Open Access ist) einen
Schlüsselfaktor für wissenschaftsgetriebene Zeitschriften darstellen.
Eine Open Source-Software wie OJS ist ein guter Startpunkt -- aber
keineswegs hinreichend. Darüberhinaus werden Server für das Hosting und
das Personal für die Weiterentwicklung, Pflege und Wartung der laufenden
Zeitschriften-Plattform und der Support für die Autor/inn/en, Mitglieder
des Editorial Boards und Begutachtende benötigt. Eine einzelne
wissenschaftsgetriebene Diamond Open Access Zeitschrift kann eine solche
Infrastruktur nicht aus eigener Kraft bereitstellen. Solche
Infrastrukturen müssen daher den wissenschaftlichen Communities zur
Verfügung gestellt werden. Frankreich versucht genau dies mit
Episciences und neueren Initiativen (weiter) zu entwickeln; dieses
Beispiel sollte Schule machen. Die Verantwortlichen sollten sich dabei
nicht aufgrund von Lobbyarbeit der kommerziellen Verlage dazu bringen
lassen, Autorengebühren zu erheben. Wir sehen es als Aufgabe der
Verantwortlichen in Politik und Wissenschaft an, Hemmnisse, die an die
Privilegien mittelalterlicher Zünfte erinnern, entschlossen zu
überwinden (zügiger als nur in Jahrhunderten!) und
wissenschaftsgesteuerte Publikationsplattformen, die nebenbei auch
finanziell hocheffizient sind, ohne Einschränkung zu unterstützen.

Ein Thema, das gerade auch aktuell mehr in den Fokus rückt, ist die
konsortiale Finanzierung von wissenschaftsgetriebenen Diamond Open
Access Zeitschriften, ein hochwillkommener Schritt in Richtung einer
Unterstützung aus öffentlichen Mitteln für LMCS, die bis vor kurzem
praktisch unmöglich war. In den 2010er Jahren gab es zwar ein
DFG-Programm zur Förderung von wissenschaftlichen Zeitschriften, aber
eine Finanzierung daraus war nur als Anschubfinanzierung und lediglich
temporär möglich. Für eine bereits etablierte Zeitschrift wie LMCS war
dies kaum von Interesse, ganz abgesehen vom zusätzlichen Aufwand für die
Verantwortlichen der Zeitschrift für einen entsprechenden Projektantrag.
Die Open Access Unterstützung durch die Bibliotheken hat sich (nach
unserer Wahrnehmung) ebenfalls bis vor Kurzem fast ausschließlich darauf
konzentriert, Autorengebühren zu finanzieren. Eine Diamond Open Access
Zeitschrift wie LMCS wurde selbst von den Universitäten der für die
Zeitschrift verantwortlichen Wissenschaftler/innen nicht unterstützt.

Eine spürbare Abhilfe für dieses Problem versprechen aktuelle
Initiativen wie das KOALA-Projekt der TIB Hannover
(\url{https://projects.tib.eu/koala/}), das Zeitschriften auf der einen
Seite und Geldgeber wie Universitätsbibliotheken auf der anderen Seite
im Rahmen einer konsortialen Finanzierung zusammenführt. LMCS nimmt
zusammen mit sechs weiteren Zeitschriften im Bereich Mathematik und
Informatik an der aktuellen KOALA-Finanzierungsrunde teil. Die nun
mögliche Finanzierung von einigen studentischen Hilfskräften für das
Layouting ist eine große Erleichterung und wird die aktuell ehrenamtlich
tätigen Layout-Editoren erheblich entlasten. Neben den vom französischen
Staat für Entwicklung und Betrieb von Episciences getragenen Kosten sind
dann im Wesentlichen alle mit der Herausgabe der Zeitschrift verbundenen
Kosten gedeckt, da Editor/inn/en und Gutachter/innen in der
theoretischen Informatik traditionell ehrenamtlich arbeiten. Freilich
ist KOALA ein für drei Jahre konzipiertes Projekt und noch keine
Dauerlösung. Eine nachhaltige Finanzierung von LMCS ist nach wie vor
offen.

Es bleibt sicher spannend, wohin die Reise von LMCS in der Zukunft geht.
Momentan scheint vieles in Bewegung zu geraten. Neue Publikationsformen
entstehen, wie beispielsweise die Peer Communities In
(\url{https://peercommunityin.org/}), oder es werden neue Ideen wie Open
Peer Review ausprobiert. Wichtig sind politische Entscheidungen, die
Diamond Open Access Publizieren unterstützen und sicherstellen, dass
wissenschaftliches Publizieren auch wissenschaftsgesteuert erfolgt.

%autor
\begin{center}\rule{0.5\linewidth}{0.5pt}\end{center}

\textbf{Stefan Milius} (Orcid: 0000-0002-2021-1644) ist Professor für
Theoretische Informatik an der Friedrich-Alexander-Universität
Erlangen-Nürnberg. Seine Forschungsinteressen liegen im Bereich Semantik
von rekursiven Prozessen und Systemen, Logik in der Informatik und
Automatentheorie. Bei der Diamond Open Access Zeitschrift \emph{Logical
Methods in Computer Science} hat er seit ihrer Gründung im Jahre 2004 in
verschiedenen Rollen mitgewirkt. So war er 2014--2019 der Special Issue
Editor der Zeitschrift und ist 2020 zum Editor-in-Chief gewählt worden.

\textbf{Wolfgang Thomas} (Orcid: 0000-0002-4453-3525) ist (emeritierter)
Professor am Lehrstuhl für Logik und Theorie diskreter Systeme in der
Fachgruppe Informatik der RWTH Aachen. Sein Hauptarbeitsgebiet ist die
Verbindung von Logik und Automatentheorie. Er hat bei der Gründung
mehrerer nicht-kommerzieller Publikationsorgane mitgewirkt; so war er
Mitglied im ersten Editorial Board der \emph{ACM Transactions on
Computational Logic,} der \emph{Leibniz International Proceedings in
Informatics (LIPIcs)} und der Zeitschrift \emph{Logical Methods in
Computer Science}. In der letztgenannten Zeitschrift ist er seit 2010
Mitglied des Governing Board.

\end{document}