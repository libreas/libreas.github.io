\documentclass[a4paper,
fontsize=11pt,
%headings=small,
oneside,
numbers=noperiodatend,
parskip=half-,
bibliography=totoc,
final
]{scrartcl}

\usepackage[babel]{csquotes}
\usepackage{synttree}
\usepackage{graphicx}
\setkeys{Gin}{width=.4\textwidth} %default pics size

\graphicspath{{./plots/}}
\usepackage[ngerman]{babel}
\usepackage[T1]{fontenc}
%\usepackage{amsmath}
\usepackage[utf8x]{inputenc}
\usepackage [hyphens]{url}
\usepackage{booktabs} 
\usepackage[left=2.4cm,right=2.4cm,top=2.3cm,bottom=2cm,includeheadfoot]{geometry}
\usepackage[labelformat=empty]{caption} % option 'labelformat=empty]' to surpress adding "Abbildung 1:" or "Figure 1" before each caption / use parameter '\captionsetup{labelformat=empty}' instead to change this for just one caption
\usepackage{eurosym}
\usepackage{multirow}
\usepackage[ngerman]{varioref}
\setcapindent{1em}
\renewcommand{\labelitemi}{--}
\usepackage{paralist}
\usepackage{pdfpages}
\usepackage{lscape}
\usepackage{float}
\usepackage{acronym}
\usepackage{eurosym}
\usepackage{longtable,lscape}
\usepackage{mathpazo}
\usepackage[normalem]{ulem} %emphasize weiterhin kursiv
\usepackage[flushmargin,ragged]{footmisc} % left align footnote
\usepackage{ccicons} 
\setcapindent{0pt} % no indentation in captions
\usepackage{xurl} % Breaks URLs

%%%% fancy LIBREAS URL color 
\usepackage{xcolor}
\definecolor{libreas}{RGB}{112,0,0}

\usepackage{listings}

\urlstyle{same}  % don't use monospace font for urls

\usepackage[fleqn]{amsmath}

%adjust fontsize for part

\usepackage{sectsty}
\partfont{\large}

%Das BibTeX-Zeichen mit \BibTeX setzen:
\def\symbol#1{\char #1\relax}
\def\bsl{{\tt\symbol{'134}}}
\def\BibTeX{{\rm B\kern-.05em{\sc i\kern-.025em b}\kern-.08em
    T\kern-.1667em\lower.7ex\hbox{E}\kern-.125emX}}

\usepackage{fancyhdr}
\fancyhf{}
\pagestyle{fancyplain}
\fancyhead[R]{\thepage}

% make sure bookmarks are created eventough sections are not numbered!
% uncommend if sections are numbered (bookmarks created by default)
\makeatletter
\renewcommand\@seccntformat[1]{}
\makeatother

% typo setup
\clubpenalty = 10000
\widowpenalty = 10000
\displaywidowpenalty = 10000

\usepackage{hyperxmp}
\usepackage[colorlinks, linkcolor=black,citecolor=black, urlcolor=libreas,
breaklinks= true,bookmarks=true,bookmarksopen=true]{hyperref}
\usepackage{breakurl}

%meta

%meta

\fancyhead[L]{Vorstand LIBREAS-Verein\\ %author
LIBREAS. Library Ideas, 44 (2023). % journal, issue, volume.
\href{https://doi.org/10.18452/28270}{\color{black}https://doi.org/10.18452/28270}
{}} % doi 
\fancyhead[R]{\thepage} %page number
\fancyfoot[L] {\ccLogo \ccAttribution\ \href{https://creativecommons.org/licenses/by/4.0/}{\color{black}Creative Commons BY 4.0}}  %licence
\fancyfoot[R] {ISSN: 1860-7950}

\title{\LARGE{In eigener Sache: Bericht über die Aktivitäten des LIBREAS-Vereins 2022/2023}}% title
\author{Vorstand LIBREAS-Verein} % author

\setcounter{page}{1}

\hypersetup{%
      pdftitle={In eigener Sache: Bericht über die Aktivitäten des LIBREAS-Vereins 2022/2023},
      pdfauthor={Vorstand LIBREAS-Verein},
      pdfcopyright={CC BY 4.0 International},
      pdfsubject={LIBREAS. Library Ideas, 44 (2023)},
      pdfkeywords={Vereinsbericht, LIBREAS},
      pdflicenseurl={https://creativecommons.org/licenses/by/4.0/},
      pdfcontacturl={http://libreas.eu},
      baseurl={},
      pdflang={de},
      pdfmetalang={de},
      pdfdoi={10.18452/28270},
      pdfurl={https://doi.org/10.18452/28270}
     }



\date{}
\begin{document}

\maketitle
\thispagestyle{fancyplain} 

%abstracts

%body
\emph{Vorbemerkung: Die Mitgliederversammlung des LIBREAS-Vereins hat in
ihrer Sitzung vom 29.11.2022 beschlossen, dass der Vereinsvorstand den
Bericht über die Vereinsaktivitäten im Sinne der Transparenz künftig (in
gekürzter Form) jeweils in der auf die Mitgliederversammlung folgenden
Ausgabe der LIBREAS.Library Ideas veröffentlicht. Personenbeziehbare
Daten werden dabei ausgelassen, sofern nicht die ausdrückliche
Zustimmung der betreffenden Person(en) vorliegt. Ebenso werden Details
ausgelassen, die das Vereinsvermögen betreffen. Sie können durch
Mitglieder des Vereins beim Vereinsvorstand jederzeit erfragt werden
beziehungsweise werden in den Protokollen der Versammlungen
aufgeschlüsselt und mit den Mitgliedern geteilt.}

\hypertarget{berichtszeitraum}{%
\section{Berichtszeitraum}\label{berichtszeitraum}}

Der Bericht bezieht sich auf den Zeitraum von der Mitgliederversammlung
2022 (29.11.2022) bis zur Mitgliederversammlung 2023 (15.11.2023).

\hypertarget{vorstand}{%
\section{Vorstand}\label{vorstand}}

Dem Vereinsvorstand gehörten im Berichtszeitraum Matti Stöhr
(Vorsitzender), Dr.~Karsten Schuldt (stellvertretender Vorsitzender),
Jana Rumler (Schriftleiterin), Dr.~Maxi Kindling (Finanzerin) und Ben
Kaden (Ressort LIBREAS.Library Ideas) an. Der Vorstand hat sich
regelmäßig getroffen und bei Bedarf virtuell ausgetauscht.

\hypertarget{mitglieder}{%
\section{Mitglieder}\label{mitglieder}}

Der LIBREAS-Verein hat 52 Mitglieder (Stand 15.11.2023). Davon waren 49
persönliche Mitglieder sowie drei Fördermitglieder.

\hypertarget{vereinsfinanzen}{%
\section{Vereinsfinanzen}\label{vereinsfinanzen}}

Die Einnahmen des LIBREAS-Vereins setzten sich im Haushaltsjahr
2022/2023 aus den Mitgliedsbeiträgen und Spenden zusammen. Ausgaben
wurden getätigt für das Hosting der Webauftritte, Kontoführungsgebühren
und die Servicepauschale für die L4F-Website. Darüber hinaus waren die
größten Ausgabenposten ein Stipendium sowie ein Freiticket für die
Open-Access-Tage 2023 in Berlin. Die Kasse wird jährlich geprüft und das
Ergebnis im Rahmen der Mitgliederversammlung berichtet. Es gab keine
Beanstandungen.

\hypertarget{redaktion}{%
\section{Redaktion}\label{redaktion}}

Der Schwerpunkt der Vorstands- und der Vereinstätigkeit liegt in der
Redaktion der LIBREAS. Im Berichtszeitraum lagen die Ausgaben
\href{https://libreas.eu/ausgabe43/inhalt/}{\#43 Soziologie der
Bibliothek} sowie
\href{https://libreas.wordpress.com/2023/05/04/cfp-44-grassroots-open-access/}{\#44
Grassroots Open Access}.

Im Mai 2023 gab es eine gemeinsame Veranstaltung von LIBREAS und der
Vernetzungs- und Kompetenzstelle Open Access Brandenburg (vergleiche
\href{https://open-access-brandenburg.de/einladung-zum-33-open-access-smalltalk-grass-roots-open-access-gemeinsam-mit-libreas-e-v/}{Einladung
zum 33. Open-Access-Smalltalk \enquote{Grassroots Open Access}}).

\hypertarget{kommunikation}{%
\section{Kommunikation}\label{kommunikation}}

Der Social-Media-Kanal
\href{https://twitter.com/libreas}{LIBREAS.Twitter} wurde bis circa
Oktober 2023 mehr oder weniger regelmäßig bespielt. Aufgrund der
aktuellen Entwicklungen auf der Plattform wird eine Weiterführung
vorerst ausgesetzt. Parallel wurde der
\href{https://openbiblio.social/@libreas}{LIBREAS.Mastodon}-Kanal
aufgebaut. Zusätzlich wurde in geringem Maß der
\href{https://www.instagram.com/libreas.libraryideas/}{LIBREAS.Instagram}-Kanal
gepflegt.

\hypertarget{stipendien}{%
\section{Stipendien}\label{stipendien}}

Für die Teilnahme an den Open-Access-Tagen 2023 wurde ein Stipendium in
Höhe von 300,00 Euro vergeben. Der Stipendiat wird einen Bericht für die
LIBREAS verfassen. Der Verein hat sich zudem an der in diesem Jahr
erstmals durchgeführten Freiticket-Aktion bei den Open-Access-Tagen 2023
beteiligt und ein Freiticket gesponsert.

\hypertarget{libraries4future}{%
\section{Libraries4Future}\label{libraries4future}}

Der Vereinsvorstand hat sich bei der Verbreitung von Informationen via
\href{https://climatejustice.global/@libraries4future}{Mastodon}
eingebracht. Der
\href{https://twitter.com/Libraries4F}{Twitter/X}-Account wird nicht
mehr bespielt. Der Verein unterstützte weiterhin die Betreuung der
\href{https://libraries4future.org/}{F4F-Website} finanziell. Die
Initiative sucht weiterhin nach Menschen, die sich aktiv einbringen
können. L4F-Material kann jederzeit bestellt werden.

\hypertarget{website-des-vereins}{%
\section{Website des Vereins}\label{website-des-vereins}}

Aufgrund eines technischen Problems war die Website des Vereins längere
Zeit nicht erreichbar. Das Problem ist inzwischen behoben.

\hypertarget{teilnahme-an-veranstaltungen-fuxfcr-libreaslibreas-vereinsvorstand}{%
\section{Teilnahme an Veranstaltungen für
LIBREAS/LIBREAS-Vereinsvorstand}\label{teilnahme-an-veranstaltungen-fuxfcr-libreaslibreas-vereinsvorstand}}

Am 06.10.2023 nahmen zwei Vorstandsmitglieder am vom Institut für
Bibliotheks- und Informationswissenschaft der Humboldt-Universität zu
Berlin und dem Einstein Center Digital Futures ausgerichtete
Veranstaltung
\href{https://www.digital-future.berlin/aktuelles/aktuelles-im-detail/news/recap-edit-a-thon-fuer-mehr-diversitaet-in-der-wikipedia1/?tx_news_pi1\%5Bcontroller\%5D=News\&tx_news_pi1\%5Baction\%5D=detail\&cHash=0b82716e054f4337bb13e5e04b7ce765}{\enquote{Edit-a-thon:
Diversität in Wikipedia-Beträgen zur Bibliotheks- und
Informationswissenschaft}} teil.

Am 03.11.2023 beteiligte sich LIBREAS an der ersten digitalen
Sprechstunde des
\href{https://www.bib-info.de/berufspraxis/medien-an-den-raendern}{Netzwerkes
\enquote{Medien an den Rändern}}.

Am 21.11.2023 nahm LIBREAS am ersten Treffen der
\enquote{\href{https://chaos.social/@fuzzyleapfrog/111362578487652002}{Queerbrarians}}
teil.

%autor

\end{document}
