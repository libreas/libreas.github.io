Das Öffentliche Bibliothekswesen -- vor den 1970er Jahren eher
Büchereiwesen genannt -- ist immer eingelassen in die zeitgenössischen
Denkweisen und politischen Entwicklungen der Gesellschaft, im den es
existiert. Gleichzeitig findet die Weiterentwicklung der Büchereiarbeit
sowie der Infrastrukturen des Büchereiwesens statt, egal welche
Ideologie jeweils aktuell vorherrschend sind. Der Artikel zeigt dies am
Beispiel des Grenzbücherdienst e.V. auf, welcher in den 1920er bis
1940er Jahren in Deutschland eine bedeutende Rolle im damaligen
Bibliothekswesen spielte, obwohl die Grundüberzeugungen, die der Verein
vertrat, heute als rechtsextrem gelten können. Der Verein ging von einem
völkischen Verständnis aus, nach dem die Welt in Völker eingeteilt wäre,
die sich untereinander bekämpfen würden und sah es als seine Aufgabe
darin, die «Erhaltung» des deutschen Volkes in den sogenannten
«Grenzregionen» sicherzustellen. Für seine Arbeit konnte der Verein
offenbar enorme Ressourcen mobilisieren. Er war mit den entstehenden
staatlichen Infrastrukturen des Büchereiwesens, insbesondere den damals
gegründeten Fachstellen, verbunden und war zudem eingelassen in die
zeitgenössischen Diskurse des Büchereiwesens, insbesondere der
Büchereipädagogik. Er nutzte nicht nur die gleichen Instrumente, wie sie
auch von deutschen Büchereien genutzt wurden, die der Verein nicht
erreichte. Vielmehr trieben einige Mitglieder des Vereins in ihren
beruflichen Funktionen, beispielsweise als Fachstellenleitungen, die
Entwicklung des Bücherwesens selber mit voran. Der Text schildert, nach
einer notwendigen historischen Kontextualisierung, die vom Verein
verbreiteten Diskurse und seine konkrete Arbeit, zuerst während der
Weimarer Republik (1918-1933), dann während des Nationalsozialismus
(1933-1945). Dabei wird sich zeigen, dass die Ideologie und konkrete
Tätigkeit des Vereins ohne grössere Probleme auch in das
nationalsozialistische Büchereiwesen integriert werden konnte. Am Ende
stellt sich die Frage, was aus dieser Geschichte über das heutige
Öffentliche Bibliothekswesen gelernt werden kann.

\begin{center}\rule{0.5\linewidth}{0.5pt}\end{center}

Public libraries -- before the 1970s called «Büchereien» in german -- is
always connected to the contemporary discourses and political
developments of the society there in. Regardless of the dominant
ideologies, the advancements of the libraries work as well as the
infrastructures of librarianship continues. This article shows this by
example of the Grenzbüchereidienst e.V., which played an important role
in german public librarianship during the 1920s to 1940s, although the
basic beliefs of this association are seen as extreme right-wing
ideologies today. The association followed a ``völkische''
understanding, by which the world is divided into ``Völker'' (ethnic
groups) which are in a constant struggle between each other. He saw his
goal in the upkeep of the german ``Volk'' in the so called
``Grenzregionen'' (border regions). For their work the association was
apparently able to mobilise a great deal of ressources. He was conceted
to the infrastructures of libarrianship, which where developing at that
time, especially to the new founded ``Fachstellen'' (central departemts
for the support of public libraries) and accepted the contemporary
discourses of german public librarianship, expressly the
``Büchereipädagogik'' (a form of planed education through public
libraries, unique for libraries in the german speaking countries). The
association used the same instruments and methods like those public
libraries which she did not reach. In fact, members of the association
pushed the development of public librarianship at that time
themselves,in their professional roles, e.g.~as directors of the
aforementioned ``Fachstellen''. After a necessary historical
contextualisation, this article depicts the discourses as well as the
concret activities of the association during the Weimar republic
(1918-1933) and the nationalsocialist regime (1933-1945). Thereby it
will become clear apparent, that the basic belives as well as the work
of the association could easily be integrated into the public
librarianship during the nazi area. At the end, the article raises
questions, what could be learned from this history for todays public
librarianship.
