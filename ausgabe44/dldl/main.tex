\documentclass[a4paper,
fontsize=11pt,
%headings=small,
oneside,
numbers=noperiodatend,
parskip=half-,
bibliography=totoc,
final
]{scrartcl}

\usepackage[babel]{csquotes}
\usepackage{synttree}
\usepackage{graphicx}
\setkeys{Gin}{width=.4\textwidth} %default pics size

\graphicspath{{./plots/}}
\usepackage[ngerman]{babel}
\usepackage[T1]{fontenc}
%\usepackage{amsmath}
\usepackage[utf8x]{inputenc}
\usepackage [hyphens]{url}
\usepackage{booktabs} 
\usepackage[left=2.4cm,right=2.4cm,top=2.3cm,bottom=2cm,includeheadfoot]{geometry}
\usepackage[labelformat=empty]{caption} % option 'labelformat=empty]' to surpress adding "Abbildung 1:" or "Figure 1" before each caption / use parameter '\captionsetup{labelformat=empty}' instead to change this for just one caption
\usepackage{eurosym}
\usepackage{multirow}
\usepackage[ngerman]{varioref}
\setcapindent{1em}
\renewcommand{\labelitemi}{--}
\usepackage{paralist}
\usepackage{pdfpages}
\usepackage{lscape}
\usepackage{float}
\usepackage{acronym}
\usepackage{eurosym}
\usepackage{longtable,lscape}
\usepackage{mathpazo}
\usepackage[normalem]{ulem} %emphasize weiterhin kursiv
\usepackage[flushmargin,ragged]{footmisc} % left align footnote
\usepackage{ccicons} 
\setcapindent{0pt} % no indentation in captions
\usepackage{xurl} % Breaks URLs

%%%% fancy LIBREAS URL color 
\usepackage{xcolor}
\definecolor{libreas}{RGB}{112,0,0}

\usepackage{listings}

\urlstyle{same}  % don't use monospace font for urls

\usepackage[fleqn]{amsmath}

%adjust fontsize for part

\usepackage{sectsty}
\partfont{\large}

%Das BibTeX-Zeichen mit \BibTeX setzen:
\def\symbol#1{\char #1\relax}
\def\bsl{{\tt\symbol{'134}}}
\def\BibTeX{{\rm B\kern-.05em{\sc i\kern-.025em b}\kern-.08em
    T\kern-.1667em\lower.7ex\hbox{E}\kern-.125emX}}

\usepackage{fancyhdr}
\fancyhf{}
\pagestyle{fancyplain}
\fancyhead[R]{\thepage}

% make sure bookmarks are created eventough sections are not numbered!
% uncommend if sections are numbered (bookmarks created by default)
\makeatletter
\renewcommand\@seccntformat[1]{}
\makeatother

% typo setup
\clubpenalty = 10000
\widowpenalty = 10000
\displaywidowpenalty = 10000

\usepackage{hyperxmp}
\usepackage[colorlinks, linkcolor=black,citecolor=black, urlcolor=libreas,
breaklinks= true,bookmarks=true,bookmarksopen=true]{hyperref}
\usepackage{breakurl}

%meta

%meta

\fancyhead[L]{Redaktion LIBREAS\\ %author
LIBREAS. Library Ideas, 44 (2023). % journal, issue, volume.
%\href{https://doi.org/10.18452/...}{\color{black}https://doi.org/10.18452/...}
{}} % doi 
\fancyhead[R]{\thepage} %page number
\fancyfoot[L] {\ccLogo \ccAttribution\ \href{https://creativecommons.org/licenses/by/4.0/}{\color{black}Creative Commons BY 4.0}}  %licence
\fancyfoot[R] {ISSN: 1860-7950}

\title{\LARGE{Das liest die LIBREAS, Nummer \#13 (Herbst–Winter 2023)}}% title
\author{Redaktion LIBREAS} % author

\setcounter{page}{1}

\hypersetup{%
      pdftitle={Das liest die LIBREAS, Nummer \#13 (Herbst–Winter 2023)},
      pdfauthor={Redaktion LIBREAS},
      pdfcopyright={CC BY 4.0 International},
      pdfsubject={LIBREAS. Library Ideas, 44 (2023)},
      pdfkeywords={Literaturübersicht, Bibliothekswissenschaft, Informationswissenschaft, Bibliothekswesen, Rezension, literature overview, library science, information science, library sector, review},
      pdflicenseurl={https://creativecommons.org/licenses/by/4.0/},
      pdfcontacturl={http://libreas.eu},
      baseurl={},
      pdflang={de},
      pdfmetalang={de},
      pdfdoi={10.18452/...},
      pdfurl={https://doi.org/10.18452/...}
     }



\date{}
\begin{document}

\maketitle
\thispagestyle{fancyplain} 

%abstracts

%body
Beiträge von Ben Kaden (bk), Karsten Schuldt (ks), Viola Voß (vv)

\hypertarget{zur-kolumne}{%
\section{1. Zur Kolumne}\label{zur-kolumne}}

Ziel dieser Kolumne ist es, eine Übersicht über die in der letzten Zeit
erschienene bibliothekarische, informations- und
bibliothekswissenschaftliche sowie für diesen Bereich interessante
Literatur zu geben. Enthalten sind Beiträge, die der LIBREAS-Redaktion
oder anderen Beitragenden als relevant erschienen.

Themenvielfalt sowie ein Nebeneinander von wissenschaftlichen und
nicht-wissenschaftlichen Ansätzen wird angestrebt und auch in der Form
sollen traditionelle Publikationen ebenso erwähnt werden wie
Blogbeiträge oder Videos beziehungsweise TV-Beiträge.

Gerne gesehen sind Hinweise auf erschienene Literatur oder Beiträge in
anderen Formaten. Diese bitte an die Redaktion richten. (Siehe
\href{http://libreas.eu/about/}{Impressum}, Mailkontakt für diese
Kolumne ist
\href{mailto:zeitschriftenschau@libreas.eu}{\nolinkurl{zeitschriftenschau@libreas.eu}}.)
Die Koordination der Kolumne liegt bei Karsten Schuldt, verantwortlich
für die Inhalte sind die jeweiligen Beitragenden. Die Kolumne
unterstützt den Vereinszweck des LIBREAS-Vereins zur Förderung der
bibliotheks- und informationswissenschaftlichen Kommunikation.

LIBREAS liest gern und viel Open-Access-Veröffentlichungen. Wenn sich
Beiträge dennoch hinter eine Bezahlschranke verbergen, werden diese
durch \enquote{{[}Paywall{]}} gekennzeichnet. Zwar macht das Plugin
\href{http://unpaywall.org/}{Unpaywall} das Finden von legalen
Open-Access-Versionen sehr viel einfacher. Als Service an der
Leserschaft verlinken wir OA-Versionen, die wir vorab finden konnten,
jedoch auch direkt. Für alle Beiträge, die dann immer noch nicht frei
zugänglich sind, empfiehlt die Redaktion Werkzeuge wie den
\href{https://openaccessbutton.org/}{Open Access Button} oder
\href{https://core.ac.uk/services/discovery/}{CORE} zu nutzen oder auf
Twitter mit
\href{https://twitter.com/hashtag/icanhazpdf?src=hash}{\#icanhazpdf} um
Hilfe bei der legalen Dokumentenbeschaffung zu bitten.

Die bibliographischen Daten der besprochenen Beiträge aller Ausgaben
dieser Kolumne finden sich in der öffentlich zugänglichen Zotero-Gruppe:
\url{https://www.zotero.org/groups/4620604/libreas_dldl/library}.

\hypertarget{artikel-und-zeitschriftenausgaben}{%
\section{2. Artikel und
Zeitschriftenausgaben}\label{artikel-und-zeitschriftenausgaben}}

\hypertarget{vermischte-themen}{%
\subsection{2.1 Vermischte Themen}\label{vermischte-themen}}

Omar, Abbas Mohamed ; Mambo, Henry ; Samzugi, Athumani ; Ali, Zuhura
Haroub (2023). \emph{Responding to the Lifelong Learning Targets:
Collaborative Efforts of Public and School Libraries}. In: International
Information \& Library Review {[}Latest Articles{]},
\url{https://doi.org/10.1080/10572317.2023.2215672} {[}Paywall{]}

Der Titel des Artikels hat mit dem eigentlich interessanten Inhalt
nichts zu tun. Er wird zum Framing einer Studie zu den Öffentlichen und
Schulbibliotheken in Sansibar (Tansania) benutzt, die für sich genommen
einen interessanten Überblick zum Stand der beiden Bibliotheksformen auf
dieser Insel beziehungsweise Provinz liefert. Das Framing zitiert das
Sustainable Development Goal 4 (\enquote{Ensure inclusive and equitable
quality education and promote lifelong learning opportunities for all})
und übersetzt dies als (a) alle Bibliotheken sollten sich auf Angebote
konzentrieren, welche Menschen nach dem Schulbesuch die ständige
Weiterbildung ermöglichen und (b) Schul- und Öffentliche Bibliotheken
sollten dabei ständig und geplant zusammenarbeiten. Um dies zu
überprüfen, wurde eine Umfrage unter 100 Schulbibliothekar*innen und 40
Öffentlichen Bibliothekar*innen durchgeführt sowie Interviews und
Fokusgruppen.

Angesichts des viel zu übertriebenen Anspruchs erscheint die Situation
in Sansibar im Artikel nun negativ: Die Bibliotheken würden nur selten
zusammenarbeiten, die Angebote hätten eher lokalen Charakter und würden
auch nur manchmal erkennbare Effekte haben. Lässt man aber diesen
überhöhten Anspruch beiseite, scheint durch die Ergebnisse ein anderes
Bild durch: Offenbar hat Sansibar eine lebendige Bibliotheksszene,
inklusive weithin verbreiteter Schulbibliotheken. Die Bibliotheken haben
verschiedene Angebote, die lokal ausgerichtet sind. Die
Bibliothekar*innen haben Vorstellungen davon, was sie mit ihrer Arbeit
erreichen wollen und können, sie reflektieren dies und werden wohl
dementsprechend planen. Ein wenig liest sich der Artikel so, als würde
erst ein von aussen, im Sinne der Sustainable Development Goals, an das
Bibliothekswesen herangetragener, unkonkreter Anspruch die tatsächliche
Arbeit in den Bibliotheken Sansibars abwerten. (ks)

Grondin, Karen A. (2023). \emph{How does my library support
accessibility?: Let me count the ways.} In: Public Services Quarterly 19
(2023) 2: 139--146, \url{https://doi.org/10.1080/15228959.2023.2189209}
{[}Paywall{]}

Der Beitrag referiert sechs kürzlich erschienene Artikel zu Studien im
Bereich \enquote{Accessibility} (also Enthinderung) und Bibliotheken aus
dem US-amerikanisch/kanadischen Raum. Es wird (vergleichbar mit dieser
Kolumne hier) immer kurz dargestellt, was die Fragestellungen, Methoden,
Ergebnisse und sich daraus ergebenden Praxishinweise waren. Sicherlich,
alles mit einem Fokus auf die Länder, rechtlichen Regelungen und
weiteren Kontexten in diesen beiden Ländern. Aber als eine schnelle
Übersicht zum aktuellen Stand zu diesem relevanten Thema ist es dennoch
sehr hilfreich. (ks)

Ma, Lai (2023). \emph{The Platformisation of Scholarly Information and
how to Fight it.} In: LiberQuarterly 33 (2023) 1,
\url{https://doi.org/10.53377/lq.13561}

Ma fasst in diesem Artikel gut nachvollziehbar zusammen, dass sich die
grossen Wissenschaftsverlage zu Datenfirmen entwickelt haben, welche vor
allem daraufhin ausgerichtet sind, Daten über die Nutzung von Artikeln,
Forschungsdaten und so weiter zu sammeln, auszuwerten und
profitorientiert zu nutzen. Sie geht auf die Auswirkungen dieser
Entwicklung auf die Wissenschaftskommunikation ein: Beispielsweise
verweist sie darauf, dass die Firmen mit diesem Geschäftsmodell ein
strukturelles Interesse daran haben, möglichst viele Artikel zu
publizieren, was zumindest einen Druck auf die Qualitätssicherung dieser
Artikel bedeutet. Grundsätzlich ist klar, dass sie all diese
Entwicklungen ablehnt und argumentiert, dass Wissenschaftliche
Bibliotheken dies in ihrer Gesamtheit tun sollten. Abschliessend
diskutiert sie vier Ansätze, wie Bibliotheken gegen diese Entwicklung
vorgehen könnten: (1) Forschende über diese Entwicklungen informieren.
(2) Bibliotheksetats nutzen, um scholar-led publishing zu unterstützen.
(3) Helfen, öffentliche Publikationsinfrastrukturen aufzubauen und das
Copyright zu verändern. Und (4) sich dafür einsetzen, dass die
Assessment-Strukturen der Wissenschaft dahingehend reformiert werden,
dass sie nicht mehr auf den Daten der grossen Verlage aufbauen.

Der ganze Artikel ist ein zusammenfassender Einstieg in das Thema, aber
gleichzeitig hat man auch schnell den Eindruck, das alles schon mehrfach
gelesen (und die vier Ansätze auch schon in der Praxis gesehen) zu
haben. Es ist wohl vor allem ein guter Text für Studierende und
Auszubildende, um in den gesamten Themenkomplex einzuführen, aber
keiner, der inhaltlich Neues präsentiert. (ks)

Edford, Rachel Lynn (2022). \emph{Figuring Embedded Librarianship: An
Analysis of the Embedded Journalist Metaphor in the Professional
Discourse}. In: New Review of Academic Librarianship.
\url{https://doi.org/10.1080/13614533.2022.2122854} {[}OA-Version:
\url{https://stars.library.ucf.edu/ucfscholar/1141}{]}

Was macht ein:e \enquote{embedded librarian} genau, und wo kommt
eigentlich diese Metapher her? Während die erste Frage nicht so leicht
zu beantworten ist -- \enquote{The literature regarding embedded
librarianship is both plentiful and diverse, with little agreement on
how to define the concept} (S. 2) -- kann man der zweiten nachgehen:
Anhand von Veröffentlichungen zur Verwendung von Metaphern im
Bibliothekswesen allgemein (\enquote{library as museum},
\enquote{library as school}, \enquote{library as trench},
\enquote{library as an ecosystem}) und zur Embedded-Metapher im
Besonderen, die sich im Laufe der Zeit von der
Embedded-Journalist-Herkunft der Metapher (aus dem sie eigentlich
stammt) lösen und stattdessen den Ursprung zum Beispiel im Umfeld der
Geologie verorten.

Es könnte interessant sein, deutsche, französische und weitere
nicht-englische Veröffentlichungen zum Thema daraufhin durchzusehen, von
welcher Basis sie für die Verwendung des Begriffes ausgehen.

Und für die Berufsbilddebatten hierzulande und anderswo könnte das Fazit
von Edford bedenkenswert sein: \enquote{More work needs to be done to
interrogate the professional discourse of LIS related to embedded
librarianship considering the historical developments of the profession.
If the profession continues to move away from stereotypes of the
collection centred mouser towards a user-centred service model, one
might question whether embedded librarianship is really a subset of
librarianship or if it's just part of being a librarian in the
twenty-first century} (S. 16). (vv)

Asher, Andrew ; Briney, Kristin ; Goben, Abigail (2023). \emph{Valid
questions: the development and evaluation of a new library learning
analytics survey.} In: Performance Measurement and Metrics 24 (2023) 2:
101--119, \url{https://doi.org/10.1108/PMM-04-2023-0009} {[}Paywall{]}

Die Autor*innen üben Kritik daran, wie in Bibliotheken und der
Bibliotheksforschung Umfragen eingesetzt werden. Diese würden zumeist
relativ einfach entworfen, nicht evaluiert oder getestet und dann auch
meist nur einmal eingesetzt. Zudem würde sie selten vollständig
dokumentiert sein. Dies würde guter wissenschaftlicher Praxis
widersprechen und zum Beispiel dazu führen, dass Daten, die mit diesen
Umfragen erhoben werden, nicht reproduziert oder einem Peer-Review
unterzogen werden können. Besser konstruierte Umfragen würden auch
helfen, tiefergehende Fragen zu beantworten, als es in der
Bibliothekspraxis oft der Fall ist sowie Ergebnisse vergleichbarer zu
machen. Es wäre möglich, Umfragen qualitativ besser zu konstruieren und
dann beispielsweise auch mehrfach zu nutzen. Die Autor*innen führen die
Konstruktion einer solchen Umfrage, inklusive der von ihnen geforderten
Tests und Evaluation, vor.

Während die Kritik an sich berechtigt ist, scheint in dem gesamten Text
eine wichtige Frage übergangen worden zu sein: Warum setzen Bibliotheken
zumeist einfach konstruierte Umfragen ein? Reichen ihnen die Ergebnisse
für ihre Praxis vielleicht aus, auch wenn sie Ansprüchen an die
wissenschaftliche Qualität nicht standhalten? Ist beispielsweise eine
Reproduktion überhaupt erwünscht? Was der Text nämlich auch zeigt, ist,
dass die von den Autor*innen gewünschte hohe Qualität ihren
\enquote{Preis} hat: Sie bedeutet viel Arbeit, inklusive des Einbezugs
zahlreicher externer Personen. Eventuell ist dieser Preis für den
Erkenntnisgewinn, den sich Bibliotheken erhoffen, einfach zu hoch. (ks)

Lenstra, Noah ; Roberts, Joanna (2023). \emph{Public Libraries and
Health Promotion Partnerships: Needs and Opportunities}. In: Evidence
Based Library and Information Practice, 18 (2023) 1, 76--99,
\url{https://doi.org/10.18438/eblip30250}

Der Text berichtet über eine Umfrage unter Personal aus Public Libraries
in South Carolina zu der Frage, welche Formen von \enquote{Health
Promotion} -- das meint eine Anzahl von Veranstaltungen und
Informationsangeboten im Bereich von Gesundheit, Essen und Sport -- sie
als sinnvoll ansehen, welche sie anbieten und auch, was die einzelnen
Bibliotheken davon abhält, sie anzubieten. Zudem wurde gefragt, von wem
und in welcher Form sie sich Unterstützung zu diesem Thema wünschen
würden.

Die Ergebnisse sind sehr spezifisch für die USA, sowohl was die
Ausrichtung der Bibliotheken auf Veranstaltungen und Outreach angeht als
auch was die konkreten Probleme sind. Beispielsweise ist die
Opioid-Krise ein relevantes Thema.

Insoweit lässt sich viel nicht in den DACH-Raum übertragen. Aber vier
Punkte scheinen auch ausserhalb den USA zu gelten. (a) Die befragten
Bibliotheken machen alle viele Angebote, aber fast keines dieser
Angebote wird evaluiert. Es gibt Vorträge zu gesundem Essen oder auch
gemeinsame Spaziergänge, aber die Bibliotheken wissen nicht, was der
Effekt für die Nutzer*innen ist. Es wird viel vermutet, aber wenig
überprüft. (b) Auch, wenn die Bibliotheken immer wieder neue Angebote
ausprobieren, geht es bei ihnen meistens darum, Informationen oder
Medien zu vermitteln. (c) Grundsätzlich wünschen sich Bibliotheken, mehr
im Bereich \enquote{Health Promotion} tun zu können, aber gleichzeitig
wollen sie auch nicht weitere Kooperationen eingehen. Dem Wunsch nach
mehr Kontakten steht die Einschätzung gegenüber, keine Ressourcen für
noch mehr Arbeit und Aufgaben zur Verfügung zu haben. (d) Die
Autor*innen betonen, dass es wichtig ist, direkt Bibliothekar*innen zu
befragen, um die tatsächliche Situation in den Bibliotheken zu erheben,
da diese einen besonderen Blick auf die Möglichkeiten und Grenzen der
Arbeit vor Ort haben. (ks)

McGrath, Kelley (2023). \emph{Musings on Faceted Search, Metadata, and
Library Discovery Interfaces}. In: Cataloging \& Classification
Quarterly 61 (2023) 5-6: 439--490,
\url{https://doi.org/10.1080/01639374.2023.2222120}

Facetten als Suchoptionen sind heute Standard in Bibliothekskatalogen,
aber das war selbstverständlich nicht immer so. Vor gut zehn bis zwanzig
Jahren gab es dazu in der bibliothekarischen Literatur tiefergehende
Diskussionen und auch verschiedene Softwareentwicklungen. Die jetzige
Situation ist nicht perfekt, aber etabliert.

Der Text ist -- ohne dies klar als Ziel zu benennen -- eine Einführung
in das gesamte Thema, inklusive einer Darstellung der Geschichte der
Facettensuche, der Probleme und Lösungsansätze sowie zahlreichen
Beispielen, anhand derer die Probleme aufgezeigt werden, beispielsweise
bei der Bestimmung von Zeitabständen, die automatisiert als Facetten
eines Suchergebnisses dargestellt werden. Obwohl am Ende des Textes auch
mögliche Fragen für weitere Forschung benannt werden, trägt er selbst
wenig zum Erkenntnisfortschritt bei. Eher liest er sich wie ein
Lehrskript oder auch ein Text, der eigentlich zu einem kurzen
Einführungswerk werden sollte, aber nicht ganz die notwendige Länge
erreichte. So sollte er auch gelesen werden: Als Einstieg in das Thema,
eventuell auch als Ergänzung des Katalogisierungsunterrichts. (ks)

Allard, Danielle ; Oliphant, Tami ; Lieu, Angela (2023).
\emph{\enquote{Finding a Way To Say \enquote*{No}}: Library Employees'
Responses to Sexual Harassment as Emotional Labour.} In: Proceedings of
the Association for Information Science and Technology 60 (2023) 1:
31--40, \url{https://doi.org/10.1002/pra2.766} {[}Paywall{]}

Dieser Text ist aus zwei Gründen schwierig zu lesen: Erstens wegen des
Themas selbst. In ihm werden Erfahrungen von Bibliothekar*innen, die
diese als sexuelle Belästigung erlebt haben, wiedergegeben. Obgleich es
sich dabei nicht um körperliche, sondern verbale Belästigungen handelt,
ist es in dieser Masse doch verstörend. Zweitens wählten die Autor*innen
eine objektivierende Darstellungsweise, die praktisch von den ganzen
Einzelfällen abstrahiert. Das ist zwar ein wissenschaftliches Vorgehen,
aber es fühlt sich trotzdem falsch an.

Grundsätzlich führten die Autor*innen eine Umfrage unter
Bibliothekar*innen in Kanada durch, über Belästigung durch Nutzer*innen,
bei denen sie 505 Antworten erhielten. Sicherlich kann man davon nicht
direkt auf die Gesamtheit dieser Vorfälle schliessen, aber es zeigt
doch, dass eine erschreckend grosse Zahl von Kolleg*innen zu diesem
Thema Erfahrungen beitragen kann. Es scheint in gewisser Weise, dass
solche Vorfälle zum Berufsalltag \enquote{dazugehören}. Die Autor*innen
interpretieren die Antworten so, dass Bibliothekar*innen
\enquote{emotional labor} leisten, um auf diese Vorfälle zu reagieren.
Durch die professionelle Haltung, öffentlichen Service für alle
anzubieten, wären sie mehr oder minder gezwungen, zu reagieren, ohne
diese Haltung zu verletzen -- und müssten anschliessend oft allein ihre
Gefühle und Reaktionen verarbeiten. Die Ergebnisse zeigen, dass nur eine
kleine Zahl Kolleg*innen auf diese Belästigungen direkt reagieren.
Vielmehr lenken sie diese um und stellen \enquote{the comfort of the
patron} (ebenda, S. 37) in den Fokus. Für die Autor*innen hat dies nicht
nur mit dem Servicecharakter der Bibliotheken zu tun, sondern auch mit
ihrer internen Struktur. Viele Kolleg*innen hätten Angst, dass es sich
negativ auf ihr Arbeitsverhältnis auswirken würde, wenn sie direkt
ablehnend reagieren würden. (Bibliothekar*innen, die schon länger in
Bibliotheken arbeiten und höhere Hierarchiestufen innehaben, haben in
den Antworten eher davon berichtet, direkt gegen solche Belästigungen
vorgegangen zu sein.)

Die Ergebnisse sind, selbstverständlich, auf die Situation in Kanada
bezogen. Aber es ist schwer vorstellbar, dass die Situation im DACH-Raum
so gross anders ist. Leider. (ks)

\hypertarget{kuxfcnstliche-intelligenz}{%
\subsection{2.2 Künstliche Intelligenz}\label{kuxfcnstliche-intelligenz}}

Roland Meyer: \emph{The New Value of the Archive. AI Image Generation
and the Visual Economy of \enquote*{Style}.} In: IMAGE. Zeitschrift für
interdisziplinäre Bildwissenschaft. Band 37, 19. Jg., (1)2023, S.
100--111. \url{https://doi.org/10.1453/1614-0885-1-2023-15458} (Zum
Zeitpunkt der Veröffentlichung dieser LIBREAS-Ausgabe ist diese DOI
defekt. Der Artikel kann alternativ unter dieser URL gefunden werden:
\url{https://image-journal.de/the-new-value-of-the-archive/}) Der Text
diskutiert Text-zu-Bild-Generatoren wie DALL·E 2, Midjourney oder Stable
Diffusion. Diese sind in der Lage, aufgrund der Eingabe sogenannter
Prompts beliebige Bilder zu erzeugen. Im Gegensatz zu Bilddatenbanken
handelt es sich bei generativen Werkzeugen nicht um ein Retrieval, bei
dem aufgrund einer Datenabfrage entsprechend erschlossene Inhalte
identifiziert und ausgeben werden, sondern um auf Wahrscheinlichkeiten
beruhende, unikale Bildproduktionen. Wie diese Bilder zustande kommen,
hängt vom jeweiligen KI-Modell und dem Material, auf dem das Modell
trainiert wurde, ab. Der Text eröffnet zwei spannende
Analyseperspektiven: Einerseits beschreibt er eine Ähnlichkeit zur
Stockfotografie, bei denen ebenfalls über beschreibende Keywords dazu
passende Bilder ausgegeben werden. Die generative Bildproduktion greift
jedoch nicht auf einen Pool von vorhandenen Daten zurück, sondern
produziert Bildangebote je nach Eingabeaufforderung neu. Dies
transformiert den Stellenwert und die Rolle von Einzelbildern ebenso
grundlegend wie die damit zusammenhängenden Geschäfts- und
Verwertungsmodelle. Im Unterschied zur Stockfotografie rücken nicht mehr
einzelne Bilder, sondern Muster ins Zentrum der Verwertung.\\
Entsprechend erheblich sind die diesen Mustern zugrundeliegenden
Trainingsdaten, deren Kuration und Bereitstellung wiederum selbst zu
einem Geschäftsmodell werden könnte. Roland Meyer konzentriert sich aber
als zweiten Aspekt auf den \enquote{Stil}, der sich zwangsläufig aus den
für die Modelle verfügbaren Trainingsdaten, also entsprechend crawlbaren
Bildern ergibt. Die Bildgenerierung verarbeitet und stilisiert diese
buchstäblich, weshalb die generierten Bilder, so das Argument des
Autors, bestimmte Stilmerkmale älterer technischer Medienformen
aufweisen und reproduzieren. Es handelt sich bei den KI-generierten
graphischen Darstellungen also um Bilder, in denen Bilder aufgehen oder
besser noch, mögliche Bilder beziehungsweise Bilder über Bilder
(\enquote{Images about Images}); das heißt nicht um Zitate, sondern um
Stilisierungen, wobei die Prompts jeweils semantische Rahmungen setzen.
(bk)

\emph{LHTN special issue on ChatGPT}. (2023) In: Library Hi Tech News 40
(2023) 3,
\url{https://www.emerald.com/insight/publication/issn/0741-9058/vol/40/iss/3}
{[}Paywall{]}

Die Zeitschrift Library Hi Tech News (nicht zu verwechseln mit der
Library Hi Tech) ist dafür bekannt, kurze, überblickshafte Artikel zu
publizieren, die vor allem aus einem auf Technik und Innovation
fixierten Blickwinkel heraus geschrieben werden. In diesen werden
regelmässig gerade aktuelle Technologien als Entwicklungen beschrieben,
welche (a) einen Einfluss auf Bibliotheken haben werden, (b) innovativ
beziehungsweise disruptiv seien und (c) deshalb von Bibliotheken
beherrscht werden müssten. Kritische, systematisch vorgehende Analysen;
Verortungen in der Geschichte der Technikentwicklung in Bibliotheken
oder auch nur tiefergehende Texte finden sich in dieser Zeitschrift
eigentlich nie. Sie ist, wenn man es polemisch ausdrücken will,
praktisch das Fachblatt für prototypische \enquote{Tech-Bros} im
Bibliothekswesen.

Dass diese Zeitschrift im Frühling 2023 eine Schwerpunktnummer zu
ChatGPT und Bibliotheken veröffentlicht hat, ist nur folgerichtig. Die
Ausgabe zeigt aber auch, wie wenig sinnvoll diese Art von
Publikationskultur ist: Alle Artikel (sieben plus ein Editorial)
beschränken sich praktisch darauf, entweder darüber zu rätseln, wie
ChatGPT vielleicht in Zukunft in Bibliotheken oder \enquote{angrenzenden
Feldern} (Pädagogik, Forschung und so weiter) eingesetzt werden wird --
aber ohne, dass klar ist, auf welcher Basis diese Vorhersagen getroffen
werden und teilweise (insbesondere im Artikel \emph{Artificial
intelligence chatbots in academic libraries: the rise of ChatGPT} von
Adebowale Jeremy Adetayo (ebenda, S. 18--21,
\url{https://doi.org/10.1108/LHTN-01-2023-0007} {[}Paywall{]})) in einer
Form, die daran zweifeln lässt, ob klar ist, dass es sich bei ChatGPT um
ein Large Language Model (LLM) handelt und nicht um eine Art
\enquote{Datenzusammenführungsmaschine}, die alle Probleme der Arbeit
mit Daten lösen wird. Oder aber die Artikel erklären grundsätzlich, wie
ChatGPT funktioniert (das tun drei der sieben Artikel und wiederholen
dabei eigentlich immer die gleichen Punkte). Bei zwei Artikeln und dem
Editorial fanden es die jeweiligen Autor*innen sinnvoll, Teile von
ChatGPT erstellen zu lassen. Gerade das Editorial zeigt damit aber eher
etwas über die Zeitschrift selbst als über ChatGPT -- nämlich, wie
angedeutet, dass ein Grossteil der in ihr publizierten Texte aus
Sprechblasen sowie aus nicht auf Fakten basierenden Vorhersagen besteht
und wirklich ohne Probleme auch von LLMs erstellt werden können. Dieses
Editorial liest sich nicht anders, als die Editorials anderer Ausgaben
und ist genauso wenig konkret in seinen Aussagen.

Über die Zukunft von LLMs in Bibliotheken erfährt man hier wenig. Einzig
in der Kolumne von Donna Ellen Frederick (\emph{ChatGPT: a viral
data-driven disruption in the information environment}, ebenda, S.
4--10, \url{https://doi.org/10.1108/LHTN-04-2023-0063} {[}Paywall{]})
wird ChatGTP einigermassen als neue Technologie verortet: Nämlich als
eine, die in einer Reihe von Technologien steht, die immer in Konkurrenz
mit älteren und bald auch neueren stehen wird, ohne sich für alle
potentiellen Anwendungsfälle zu etablieren. (ks)

\hypertarget{covid-19-und-die-bibliotheken}{%
\subsection{2.3 Covid-19 und die
Bibliotheken}\label{covid-19-und-die-bibliotheken}}

Clarke, Rachel Ivy ; Grimm, Alexandra ; Zhang, Bo ; Stanton, Katerina
Lynn (2023). \emph{Time, Tasks, and Toll: Changes in Library Work During
the COVID-19 Pandemic}. In: Journal of Library Administration 63 (2023)
4: 421--445, \url{https://doi.org/10.1080/01930826.2023.2201717}
{[}Paywall{]}

Berichtet wird hier über eine über verschiedene Kanäle weit verteilte
Umfrage, die im August und September 2020 Personen, die in
US-amerikanischen Bibliotheken arbeiteten, danach befragte, welche
Arbeit sie explizit tun und welche Arbeit sie ein Jahr vorher, also
2019, taten. (Die Autor*innen betonen explizit, dass sie nicht von
\enquote{Bibliothekar*innen} sprechen, sondern alle Personen, die in
Bibliotheken tätig sind, erfassen wollten, egal, wie deren
Jobbezeichnung war.) Grundsätzlich war die Umfrage als Teil eines
Forschungsprojektes zu \enquote{unsichtbarer Arbeit} in Bibliotheken
(also solcher, die geleistet wird, ohne von ausserhalb der Institutionen
wahrgenommen zu werden) geplant gewesen, wurde dann aber wegen der
Pandemie um Fragen nach den Veränderungen zum letzten Jahr ergänzt. Der
Text wertet nur die Veränderungen aus. Antworten lagen am Ende von 949
Personen aus allen Bundesstaaten und Territorien der USA, ausser South
Dakota, vor.

Grundsätzliches Ergebnis war, dass die geleistete Arbeitszeit zwischen
2019 und 2020 leicht zurückging (im Durchschnitt um 1,5 Stunden pro
Woche und Person), aber in den Öffentlichen Bibliotheken weit mehr (1,91
Stunden pro Woche) als in den Wissenschaftlichen (0,04 Stunden pro
Woche). (Andere Bibliothekstypen waren zu wenig unter den Antworten
vertreten, so dass auf eine Auswertung verzichtet wurde.) Gleichzeitig
stieg die eigentliche Arbeit, die zu leisten war. Insbesondere beklagten
die Kolleg*innen das \enquote{Verwischen} von Grenzen zwischen Arbeit
und Alltag, zudem die viele Zeit, welche in virtuellen Treffen verbracht
werden musste und vor allem den schwer objektiv zu messenden
\enquote{emotionalen Tribut}, der ihnen abverlangt wurde. Insbesondere,
dass sie gegenüber Nutzer*innen und anderen Kolleg*innen ständig einen
positiv gestimmten Eindruck vermitteln mussten, setzte ihnen offenbar
zu.

Die Autor*innen diskutieren einige mögliche Gründe für die Unterschiede
zwischen den beiden Bibliothekstypen, die aber teilweise sehr
US-spezifisch sind, zum Beispiel die prekären Anstellungsverhältnisse in
Öffentlichen Bibliotheken. Zuzustimmen ist aber unbedingt ihrer
Einschätzung, dass ein erhöhtes Arbeitsvolumen bei gleichzeitig weniger
Arbeitszeit auf längere Zeit weder für Personal noch Bibliotheken selbst
nachhaltig war. (ks)

Petersen, David (2023). \emph{Remote and Hybrid Work Options for Health
Science Librarians: A Survey of Job Postings Before and After the
COVID-19 Pandemic}. In: Medical Reference Services Quarterly 42 (2023)
2: 153--162, \url{https://doi.org/10.1080/02763869.2023.2194144}
{[}Paywall{]}

Der Autor untersucht anhand von Stellenausschreibungen für
Medizinbibliothekar*innen in den USA, ob sich zwischen 2018 und 2022 die
Zahl der Stellen vermehrt hat, in denen die Arbeit von ausserhalb der
eigentlichen Bibliothek (also \enquote{remote work}) explizit ermöglicht
wird. Dabei geht er davon aus, dass während der Pandemie klar geworden
ist, wie viel dieser Arbeit tatsächlich \enquote{von daheim} geleistet
werden kann. Er postuliert, dass interessierte Stellensuchende deshalb
auch darauf achten würden, ob diese Möglichkeiten erwähnt werden. Wenn
sich Bibliotheken dahingehend ändern würden, dass sie dies verstärkt
ermöglichen, sollte dies -- so seine Annahme -- in den Stellenanzeigen
auch sichtbar sein. Die Medizinbibliotheken, die er als
Untersuchungsgegenstand wählte, weil er selbst in solch einer arbeitet,
eignen sich für diese Frage besonders, weil diese an sich immer
schneller Veränderungen umsetzen, als andere Bibliotheken.

Alas, seine Auswertung zeigt keine Veränderung in Richtung \enquote{mehr
Homeoffice} an. Die Zahl der Stellenanzeigen an sich ist mit und nach
der Pandemie merklich gestiegen, aber die Zahl der Stellenanzeigen,
welche remote work als Möglichkeit erwähnen, ist dabei nur in geringerem
Masse angestiegen. Oder anders gesagt: So richtig scheinen die
Bibliotheken sich an diesem Punkt durch die Erfahrungen in der Pandemie
nicht verändert zu haben. (ks)

Sobol, Barbara ; Goncalves, Aline ; Vis-Dunbar, Mathew ; Lacey, Sajni ;
Moist, Shannon ; Jantzi, Leanna ; Gupta, Aditi ; Mussell, Jessica ;
Foster, Patricia L. ; James, Kathleen (2023). \emph{Chat Transcripts in
the Context of the COVID-19 Pandemic: Analysis of Chats from the AskAway
Consortia}. In: Evidence Based Library and Information Practice 18
(2023) 2, S. 73--92, \url{https://doi.org/10.18438/eblip30291}

Die Autor*innen untersuchen mit einer Anzahl von (Open Source) Tools die
Chats von Auskunftsgesprächen, welche 2019--2021 (also während der
aktiven Phasen der Covid-19 Pandemie) in British Columbia, Kanada,
zwischen Bibliothekar*innen und Nutzer*innen Wissenschaftlicher
Bibliotheken geführt wurden. Eine grosse Zahl der im dortigen
Bundesstaat ansässigen Hochschulbibliotheken (aber auch nicht alle) hat
sich zu einem Konsortium zusammengeschlossen, welches gemeinsam einen
Chatservice für Fragen an die Bibliotheken betreibt. Diese Gespräche
laufen alle über den gleichen Dienstleister und sind deshalb auch alle
in den gleichen Dateiformaten gespeichert, mit den gleichen Tags
versehen et cetera. Die Autor*innen konnten über diesen Anbieter auf
mehr als 70.000 dieser Gespräche zurückgreifen, die auch für
Bibliotheken aller Hochschulformen, die im Bundesstaat vertreten sind
(also vor allem private und öffentlich finanzierte sowie solche
unterschiedlicher Grösse) geführt wurden.

Grundsätzliche Erwartung war, dass die Pandemie in diesen Chats sichtbar
sein müsste. Gefragt wurde, welche Veränderungen es über die Zeit gab.
Allerdings zeigte sich, dass zwar die Zahl der Fragen, die an die
Bibliotheken gestellt wurde, stieg, was mit den Schliessungen der
physischen Zugänge selbst zu erklären war, aber ansonsten fast
durchgängig gleich blieb. In der Hochphase der Lockdowns wurde die
Sprache der Nutzer*innen etwas informeller, zugleich gingen Fragen
bezüglich der Fernleihe in der Zeit zurück, als im Bundesstaat auch
keine Fernleihe angeboten wurde. Aber ansonsten scheint die Pandemie
keine wirklichen Auswirkungen auf die gestellten Fragen und damit --
abgeleitet -- die Interessen der Nutzer*innen gehabt zu haben.

Interessant an der Studie ist zudem, dass die Autor*innen zu Beginn
einen Überblick zu vergleichbaren Studien, welche eine grosse Anzahl an
Chats aus Bibliotheken analysieren, zusammenfassen und dabei
feststellen, dass sich bislang kein Ansatz etabliert hat, wie eine
solche Analyse im Idealfall durchgeführt werden kann. Fast jede Studie
scheint jeweils anders vorzugehen und -- das wird eher angedeutet als
gesagt -- es wäre sinnvoll, die verschiedenen Methoden einmal mittels
einer Metastudie zu erkunden. (ks)

\hypertarget{forschungsdatenmanagement}{%
\subsection{2.4
Forschungsdatenmanagement}\label{forschungsdatenmanagement}}

Farrell, Shannon L. ; Kelly, Julia A. ; Hendrickson, Lois G. ; Mastel,
Kristen L. (2023). \emph{A Pilot Study to Locate Historic Scientific
Data in a University Archive}. In: Issues in Science and Technology
Librarianship (2023) 103, \url{https://doi.org/10.29173/istl2728}

Was hier Pilotstudie genannt wird, ist in der Realität ein Test, welche
Forschungsdaten in analoger Form eigentlich in einem Universitätsarchiv
liegen könnten und mit welchem Aufwand sie zu finden sind. Das
betreffende Archiv ist das der University of Minnesota, an der die
Autor*innen des Textes alle (in der Bibliothek) arbeiten.

Was mit diesem Text gezeigt wird, ist, dass eine ganze Reihe von Daten
in Form von Forschungstagebüchern, Listen, Sammlungen und so weiter in
das Archiv gelangt sind, die allesamt wohl auch in der heutigen
Forschung (nicht der Geschichtswissenschaft, bei der dies immer stimmt,
sondern beispielsweise auch den Umweltwissenschaften) genutzt werden
könnten. Es bedarf der Suche nach ihnen, inklusive der Archivarbeit wie
dem Suchen in Findbüchern, dem Beachten der Provenienz und der Arbeit
mit analogen Materialien. Dies dauert seine Zeit. Zudem sind die Daten
alle unterschiedlich gut dokumentiert, was weitere Herausforderungen mit
sich bringt.

Die Autor*innen betonen, dass ihrer Erfahrung nach in den Archiven eine
grosse Zahl von Daten liegen, die noch erschlossen werden müssten. Sie
postulieren, dass dies ein zukünftiges Arbeitsfeld für
Bibliothekar*innen sein kann. Das mag stimmen, aber ein wenig liest sich
der Text auch, als würden sich die Kolleg*innen eine weitere Aufgabe
suchen. Ob es ein Interesse von Seiten der Forschung gibt, diese Daten
zu nutzen, ist selbstverständlich noch nicht klar. (ks)

Holmes, Martin ; Jenstadt, Janelle ; Huculak, J. Matthew (edit.) (2023):
\emph{Special Issue: Project Resiliency in the Digital Humanities}. In:
DHQ: Digital Humanities Quarterly 17 (2023) 1,
\url{https://digitalhumanities.org/dhq/vol/17/1/index.html}

Entstanden im Zusammenhang eines Symposiums, welches wiederum als Teils
eines Projekts über das \enquote{erfolgreiche Beenden} von
Digital-Humanities-Projekten und deren mögliche nachhaltige
Archivierung, war, thematisieren die Beiträge dieser Schwerpunktnummer
vor allem letzteres: Was sind die Erfahrungen und Möglichkeiten, um die
Ergebnisse von Digital Humanities Projekten, die ja meist aus Homepages
bestehen, langfristig zu archivieren.

Das ist für Bibliothekar*innen kein neues oder inhaltlich überraschendes
Thema. Im Projekt wurden Prinzipien aufgestellt, beispielsweise dass
Daten am Ende von Projekten immer in offenen Standards vorliegen müssen
oder dass die Projekthomepages \enquote{geflattet} (also zu
eigenständigen, nicht-interaktiven Homepages) werden müssen, um
langfristig aufbewahrt werden zu können. Zudem muss dieses
\enquote{Ende} schon von Beginn eines Projekts an geplant sein. All
diese Prinzipien sollten eigentlich Teil guter
Forschungsdatenmanagementpraxis sein. Aber was aus den Texten zu lernen
ist, ist, dass dies für aktive Forschende in den Digital Humanities
offenbar immer wieder neue Themen und Vorgaben sind. Offenbar wird bei
diesen Projekten tatsächlich nicht an deren Langlebigkeit gedacht,
sondern teilweise daran, möglichst Neues auszuprobieren. Teilweise
scheinen Forschende aber auch Probleme zu haben, ein Projekt
\enquote{wirklich} zu beenden, also in gewisser Weise loszulassen und
nicht ständig über die Projektlaufzeit hinaus doch noch an ihm zu
arbeiten, also zum Beispiel Daten zu ergänzen.

Das muss nicht als Vorwurf an Forschende formuliert werden. Offenbar
sind es verschiedene Denkweisen, die sich hier zeigen: Die von
Forschenden auf der einen Seite und die von den Institutionen, welche
nachher manchmal damit betraut werden, die Projekte irgendwie zu
erhalten, also vor allem Bibliotheken oder Archive. Interessant ist, zu
sehen, dass diese unterschiedlichen Denkweisen immer wieder
aufeinandertreffen. Sie werden wohl nie einfach geklärt werden, sondern
müssen auch in Zukunft immer wieder aktiv zusammengebracht werden. Oder
anders gesagt: \enquote{Forschungsdatenmanagement} wird auch in Zukunft
bedeuten, immer wieder neu und immer wieder neuen Forschenden zu
erläutern, warum ein solches notwendig ist. Dieses Wissen scheint sich
nicht mit der Zeit zu etablieren. (ks)

Rod, Alisa B. (2023). \emph{It Takes a Researcher to Know a Researcher:
Academic Librarian Perspectives Regarding Skills and Training for
Research Data Support in Canada}. In: Evidence Based Library and
Information Practice, 18 (2023) 2, 44--58,
\url{https://doi.org/10.18438/eblip30297}

Um zu klären, was Forschungsdatenmanagement für Bibliotheken konkret
bedeutet, führte die Autor*in semi-strukturierte Interviews mit zwölf
Bibliothekar*innen kanadischer Hochschulen durch, welche teilweise seit
Jahren in diesem Bereich arbeiten. Sicherlich gibt es dabei nationale
Besonderheiten. Beispielsweise sind die kanadischen Universitäten von
der Regierung aktuell aufgefordert, institutionelle Strategien für das
Forschungsdatenmanagement zu erarbeiten. Für diese Strategiearbeit
werden auch Bibliothekar*innen herangezogen. Aber darüber hinaus ergaben
die Interviews auch Ergebnisse, die sich wohl verallgemeinern lassen.

An den meisten Hochschulen ist die
\enquote{Forschungsdatenmanagementarbeit} an den Bibliotheken verortet,
aber was genau dies heisst, ist fast nie klar geregelt. Vielmehr sind
die Arbeitsinhalte und Abgrenzungen zu anderen Aufgaben oft sehr
schwammig. Bei einem Drittel der Befragten ist diese Arbeit sogar eine,
die sie neben anderen bibliothekarischen Aufgaben erledigen. Sie setzt
sich zumeist zusammen aus (1) der Beantwortung von Anfragen zu
spezifischen Datensätzen, (2) dem Unterrichten von Modulen zur
Datenanalyse, (3) die Übernahme des Hochladens von Daten für Forschende,
inklusive der notwendigen Vorbereitungen und (4) der Beratungen zum
Schreiben von Datenmanagementplänen. Punkt (5), die Arbeit an
Strategien, ist vielleicht nur temporär. Ganz selten, (6) führen die
Bibliothekar*innen selbst Studien mit Forschungsdaten durch oder (7)
sind in Forschungsprojekte eingebunden.

Die Autor*in will auch wissen, welche Kompetenzen für diese Arbeit
notwendig sind und ob diese Kompetenzen im Studium (in Kanada)
vermittelt werden. Da die Aufgaben aber offen sind, ist es auch schwer,
die konkreten Kompetenzen zu vermitteln. Die Befragten fühlten sich vom
Studium gut vorbereitet, wobei sie vor allem vom Überblickswissen (zum
Beispiel, was Metadaten sind) und der Fähigkeit, sich schnell in neue
Themen einzuarbeiten, profitieren. Ein Punkt, der auch im Titel des
Artikels erwähnt wird, ist, dass die Bibliothekar*innen denken, dass es
notwendig ist, selbst mit Daten geforscht zu haben, um erfolgreich die
Arbeit im Forschungsdatenmanagement leisten zu können. (ks)

Pares, Nicolas ; Organisciak, Peter (2023). \emph{The Effects of
Research Data Management Services: Associating the Data Curation
Lifecycle with Open Research Output}. In: College \& Research Libraries
84 (2023) 5: 751--766, \url{https://doi.org/10.5860/crl.84.5.751}

Die Ergebnisse der Umfrage, welche in diesem Artikel präsentiert werden,
sind auf einer konzeptionellen Ebene relevanter, als auf der
eigentlichen faktischen Ebene. Faktisch wurden hier die Meinungen von 46
Personen, rund die Hälfte Forschende und Personen, welche in ihrer
Arbeit das Forschungsdatenmanagement unterstützen, gesammelt und
ausgewertet. Das sind relativ wenige Personen, zudem alle in den USA
tätig und auf Seiten der Forschenden nur solche, die auch schon
Erfahrungen mit dem Forschungsdatenmanagement haben. Insoweit sind auch
die eigentlichen Ergebnisse sehr speziell.

Konzeptionell wird mit der Umfrage aber eine relevante Problematik bei
der Planung von Services im Bereich Forschungsdatenmanagement sichtbar.
Es geht darum, ob die Möglichkeiten von Forschenden (in den USA),
Forschungsdaten offen zu publizieren, von bestimmten Gegebenheiten
abhängen, zum Beispiel davon, ob Services in diesem Bereich aufgebaut
sind, wo sie in der Hochschule angesiedelt sind oder auch, welche
Position die jeweiligen Forschenden haben. Was sich in den Ergebnissen
zeigte, war, dass das Vorhandensein und die Ansiedlung solcher Services
an Bibliotheken offenbar einen fördernden Charakter hat. Oder anders
gesagt: Dass Services angeboten und dann auch finanziert werden,
vermittelt den Eindruck, dass das Forschungsdatenmanagement mehr
Relevanz hat, als andere wissenschaftspolitische Vorgaben. Es wird ihm
also institutionell von den Hochschulen Wertigkeit zugeschrieben, was
von den Forschenden wahrgenommen wird. Was sich hingegen nicht zeigte,
war eine Verbindung zum bekannten Forschungsdatenkreislauf: Welche
Services konkret existieren und wo die in diesem Kreislauf angesiedelt
sind, scheint keinen richtigen Einfluss darauf zu haben, ob Forschende
ihre Daten offen veröffentlichen oder nicht. Das ist relevant, weil sich
dieser Kreislauf als Planungsinstrument durchgesetzt hat. Bibliotheken
(und andere Einrichtungen an Hochschulen sowie Forschungsförderer)
nutzen ihn, um zu planen, welche Services eingerichtet werden, mit
Personal und Mittel ausgestattet werden und so weiter. Aber -- und das
ist die interessante konzeptionelle Überlegung, welche der Artikel
anstösst -- es scheint nicht so (was weiter überprüft werden müsste),
also ob dieser Kreislauf für Forschende eine Bedeutung hätte. Er ist ein
Modell und wie jedes Modell bildet er nur ein mögliches Bild der
Realität ab. Es gilt zu überprüfen, ob dieses Modell vielleicht zu weit
von der realen Arbeitspraxis von Forschenden entfernt ist -- was zum
Beispiel erklären würde, warum viele dieser Services nur von einem
kleinen Teil von Forschenden genutzt werden (wie viele andere Studien
zum Thema, die auch schon in dieser Kolumne beleuchtet wurden). (ks)

Kouper, Inna (2023). \emph{Data Curation in Interdisciplinary and Highly
Collaborative Research}. In: International Journal of Digital Curation
17 (2023) 1, \url{https://doi.org/10.2218/ijdc.v17i1.835} (Zum Zeitpunkt
der Veröffentlichung dieser LIBREAS-Ausgabe ist diese DOI defekt. Der
Artikel kann alternativ unter dieser URL gefunden werden:
\url{http://www.ijdc.net/article/view/835})

In diesem Paper werden 169 Artikel (erschienen seit 2011) ausgewertet,
die alle über das Datenmanagement in interdisziplinären
Forschungsprojekten berichteten. Der Fokus dieser Recherche war,
Erfahrungen über dieses Datenmanagement zu versammeln sowie nach
allgemeinen Trends und Empfehlungen für die weitere Praxis zu schauen.
Nach den Papers wurde systematisch recherchiert und sie wurden
anschliessend auch systematisch ausgewertet. Auffällig ist schon zu
Beginn, dass es eine recht hohe Zahl von diesen doch spezifischen
Berichten ist, die gefunden wurden. Insoweit, so kann man schliessen,
gibt es schon eine gelebte Praxis des Forschungsdatenmanagements in
grossen, interdisziplinären Projekten.

Des Weiteren zeigen diese Berichte immer wieder ähnliche
Herausforderungen auf: Interdisziplinäre Arbeit heisst, damit umzugehen,
dass unterschiedliche Disziplinen sehr verschiedene Herangehensweisen
haben, auch an Daten und das Datenmanagement. Es galt in allen diesen
Projekten immer, ein gemeinsames Verständnis der Teilnehmer*innen aktiv
herzustellen. Dies bezog sich nicht nur auf die technische, sondern auch
auf die interpersonelle Ebene. Gleichzeitig zeigt diese Übersicht, dass
immer wieder neue Wege des Datenmanagements gesucht (und gefunden)
werden. Zwar erwähnen viele Artikel, dass es wichtig ist, dies möglichst
früh im Forschungsprozess zu planen und zum Beispiel dann auch Kontakt
zu Bibliotheken oder Archiven zu haben. Aber gleichzeitig wurde in den
meisten Fällen, die in den Papers beschrieben werden, diese Kontakte in
der Realität erst sehr spät im Forschungsprozess (oft am Ende)
aufgenommen.

Fast alle der ausgewerteten Paper sind Berichte oder Case Studies. Der
Artikel zeigt auch, dass es bislang fast keine systematische Forschung
zur konkreten Datenmanagementpraxis in interdisziplinären Projekten gibt
(und benennt dies, berechtigt, als Leerstelle). (ks)

Khan, Nushrat ; Thelwall, Mike ; Kousha, Kayvan (2023). \emph{Data
sharing and reuse practices: disciplinary differences and improvements
needed.} In: Online Information Review 47 (2023) 6: 1036--1064,
\url{https://doi.org/10.1108/OIR-08-2021-0423} {[}Paywall{]},
Open-Access-Version:
\url{https://wlv.openrepository.com/handle/2436/625075}

In einer weiteren Umfrage, über die in diesem Text berichtet wird, wurde
versucht, die konkreten Praktiken des Forschungsdatenmanagements von
Wissenschaftler*innen zu erfahren. Was diese Studie von ähnlichen
Unterfangen abhebt, ist, dass versucht wurde, eine möglichst grosse
Anzahl von Forschenden aus möglichst unterschiedlichen Disziplinen zu
erreichen. Anschliessend wurden die Praktiken zwischen den Disziplinen
verglichen. Um das zu erreichen, wurden Einladungen zur Umfrage direkt
an rund 70.000 Forschende geschickt, die als Erstautor*innen von
Artikeln verzeichnet waren, welche 2018 oder 2019 erschienen und in der
Scopus-Datenbank nachgewiesen wurden. Beantwortet wurde die Umfragen
dann von 3.257 Personen, die tatsächlich aus sehr verschiedenen
Disziplinen stammten.

Die Ergebnisse zeigen -- wenig überraschend, aber hier mit Daten
abgesichert --, dass sich diese Praktiken (Daten teilen, Daten suchen
und Daten nutzen) je nach den Disziplinen und auch Feldern innerhalb der
Disziplinen unterscheiden. Es zeigt sich auch, dass es einen Lerneffekt
gibt: Je länger Forschende aktiv sind, umso sicherer sind sie im Teilen,
Finden und Nutzen von Daten. Die umfangreiche Darstellung der Daten im
Artikel zeigt auch, dass die Teilchenphysik und die Astronomie mit ihrem
aktiven Datenmanagement wirklich eine Besonderheit darstellen. Sie
zeigen auch, dass Forschende zur Suche von Datensätzen vor allem auf
schon publizierte Artikel (die dann auf Datensätze verweisen), ihnen
bekannte Repositorien und Hinweise von Kolleg*innen zurückgreifen.
Re3data.org als Instrument und systematische Suchstrategien, wie sie von
Bibliotheken als sinnvoll angesehen werden, werden in der Praxis von
Forschenden nur sehr selten genutzt. (ks)

Paulo Cezar Vieira Guanaes ; Sarita Albagli: Direito Autoral sobre dados
de pesquisa no ecossistema da Comunicação Científica. In:
Transinformação 35 • 2023
\url{https://doi.org/10.1590/2318-0889202335e226918}

In ihrem Aufsatz beschäftigen sich Paulo Cezar Vieira Guanaes (Fundação
Oswaldo Cruz, Escola Politécnica de Saúde Joaquim Venâncio. Rio de
Janeiro) und Sarita Albagli (Instituto Brasileiro de Informação em
Ciência e Tecnologia -- Universidade Federal do Rio de Janeiro, Programa
de Pós-Graduação em Ciência da Informação) mit dem Stand der Diskussion
um die Anwendung von Copyright auf Forschungsdaten. Sie stellen in ihrer
auf Literaturanalysen sowie der Auswertung juristischer Materialien und
Fragebogen basierten Umfragen beruhenden Studie fest, dass das Prinzip
offener und geteilter Forschungsdaten bei den Stakeholdern (Forschende,
Journals, Förderer, Wissenschaftsinfrastruktur) durchaus verstanden und
gewünscht wird. Probleme gibt es jedoch bei der praktischen
Implementierung. Es fehlen beispielsweise sowohl ein Set an
praxistauglichen verbindlichen Anforderungen, ein Gratifikationssystem
für die entsprechenden datenaufbereitenden Arbeiten und
Veröffentlichungsschritte als auch eine verbindliche rechtliche
Regelung, die bestimmte Formen von Daten grundsätzlich von einem
Copyright oder einer Nutzungsbeschränkung ausschließt. Für die
Mitarbeitenden der Forschungsdateninfrastrukturen beziehungsweise
Repository Manager und ebenfalls für die Herausgeber*innen von
Zeitschriften bedeutet die Studie, dass die Fragen der offenen
Forschungsdatenpublikation sowie der damit zusammenhängenden rechtlichen
Aspekte bisher keine Priorität haben. Daran ändern auch entsprechend
ausgerichtete Projekte wie Fiocruz und SciELO wenig. Die Autor*innen
empfehlen zudem für die weitere Forschung zum Thema eine Untersuchung
der Geschäftsmodelle für das Publizieren von Forschungsdaten, da sie
befürchten, dass sich kommerzielle Ansätze analog zu Article Processing
Charges auch in diesem Bereich durchsetzen könnten. Parallel sehen sie
die Notwendigkeit zur rechtlichen und wissenschaftspolitischen
Absicherung des Ziels der Openness für Forschungsdaten (\enquote{um
direito especial a favor das práticas de dados abertos}) (bk)

\hypertarget{open-access}{%
\subsection{2.5 Open Access}\label{open-access}}

Linna, Anna-Kaarina ; Ylönen, Irene ; Salmi, Anna (2023).
\emph{Monitoring Organizational Article Processing Charges (APCs) using
Bibliographic Information Sources: Turku University Library Case}.
{[}Alternativer Titel: Monitoring organizational Article Processing
Charges (APCs) using external sources. Turku University Library case{]}
In: Liber Quarterly 33 (2023) 1--23,
\url{https://doi.org/10.53377/lq.13361}

Im Artikel wird über ein Projekt an der genannten Turku University in
Finnland berichtet, bei dem untersucht wurde, ob die tatsächlich von der
Universität gezahlten APCs in den Daten abgebildet sind, welche die
Universität sammelt. Oder aber, ob mittels Datenbankrecherchen andere
Artikel nachgewiesen werden können, für die wohl auch -- aber nicht
direkt in den \enquote{normalen} Geldflüssen sichtbar -- solche Gebühren
bezahlt wurden. Zudem sollte, falls letzteres der Fall ist, geschaut
werden, ob es dabei auffällige Faktoren gibt, beispielsweise ob die
Gebühren an bestimmte Verlage öfter \enquote{unterreportet} werden.
Relevant ist dies auch, weil die Universität grundsätzlich Daten an
OpenAPC liefert und deshalb ein Interesse daran hat, dass diese
vollständig sind.

Im Projekt zeigte sich, dass es -- wenn auch mit hohem Zeitaufwand --
möglich ist, solche \enquote{undokumentierten} APCs nachzuweisen und
zwar in einer hohen Zahl. Für Artikel, die 2017 und 2018 erschienen
sind, waren dies wohl 48 \% (242) versus 52 \% (265), bei denen die
Geldflüsse für die APCs bekannt waren. Dabei zeigte sich auch, dass
diese Verteilung in bestimmten Verlagen -- namentlich MDPI und American
Chemical Society -- weit eher in Richtung \enquote{unterreportet}
verschoben ist. Die Universität hat jetzt die Dokumentation ihrer
Geldflüsse angepasst und hofft, mehr der APCs ordentlich zu
dokumentieren. Es ist aber zu vermuten, dass solche Überprüfungen in
anderen Hochschulen ähnliche Ergebnisse erbringen würden, was zum
Beispiel Auswirkungen auf die Qualität der Daten in OpenAPC hat. (ks)

Nazim, Mohammad ; Bhardwaj, Raj Kumar (2023). \emph{Open access
initiatives in European countries: analysis of trends and policies}. In:
Digital Library Perspectives 39 (2023) 3: 371--392,
\url{https://doi.org/10.1108/DLP-06-2022-0051} {[}Paywall{]}

Dieser Artikel -- wieder einmal einer, der sich mit dem Thema Open
Access beschäftigt, aber selbst nicht als solcher erscheint -- liefert
einen interessanten Aussenblick auf die im Titel genannten
Open-Access-Initiativen, die aktuell in Europa vorangetrieben werden.
Die Autor*innen schauen gewissermassen aus Indien auf diese
Entwicklungen und versuchen sich einen Überblick zu schaffen. Einerseits
basiert der Text deshalb auf einer Ansammlung von Daten, die jeweils
möglichst vollständig sind (bis hin zur Angabe, dass es in Liechtenstein
null Open Access Repositories gibt), andererseits zeigen die
versammelten Daten und deren Bewertung auch, dass der Fokus auf
Offenheit, welcher in den Debatten um Open Access im DACH-Raum
dominiert, nicht unbedingt der einzig mögliche ist. Die Autor*innen
interessieren sich eher für Angaben wie den Einfluss der Initiativen auf
den durchschnittlichen Impact Factor der Wissenschaften europäischer
Länder -- also Fragen, die im DACH-Raum immer weniger Bedeutung haben.
Er ist eine gute Erinnerung daran, dass auch Themen wie Open Access oder
Forschungsevaluationen immer in spezifischen kulturellen oder nationalen
Rahmen verortet sind. (ks)

\hypertarget{bestandsmanagement}{%
\subsection{2.6 Bestandsmanagement}\label{bestandsmanagement}}

Getsay, Heather ; Chen-Gaffey, Aiping (2023). \emph{COUNTER in context:
a case study on journal package usage}. In: Journal of Electronic
Resources Librarianship 35 (2023) 3: 195--206,
\url{https://doi.org/10.1080/1941126X.2023.2224670} {[}Paywall{]}

Es wird in diesem Text eine Analyse von COUNTER-Daten für eine
spezifische Bibliothek (Bailey Library, Slippery Rock University of
Pennsylvania) und ein spezifisches Zeitschriftenpaket (Health \& Life
Sciences and Physical Sciences von Elsevier) durchgeführt. Die
Ergebnisse bestätigen vor allem die Vermutungen der Autor*innen (beide
aus der genannten Bibliothek), beispielsweise dass viele Zeitschriften
kaum, dafür einige Zeitschriften sehr viel benutzt werden. Interessant
ist der Text vor allem, weil hier einmal ausführlich, gut
nachvollziehbar und auch mit vielen Darstellungen unterstützt, eine
solche Auswertung vorgeführt wird, inklusive der Ableitung von
Entscheidungen für das zukünftige Bestandsmanagement. Sicherlich wird
dies auch so in zahlreichen Bibliotheken durchgeführt. Hier aber wird,
praktisch als Anschauungsmaterial, einmal öffentlich sichtbar gemacht,
was mit diesen Daten möglich ist. Der Beitrag lässt sich auch gut für
die Lehre oder das Selbststudium nutzen. (ks)

Robson, Diane ; Bryant, Sarah ; Sassen, Catherine (2023). \emph{Video
Game Equipment Loss and Durability in a Circulating Academic
Collection}. In: Evidence Based Library and Information Practice 18
(2023) 3: 53--68, \url{https://doi.org/10.18438/eblip30294}

Im Text werden die Erfahrungen, welche die Bibliothek der University of
North Texas in Denton mit ihrer Sammlung von Videospielen und den
dazugehörigen Geräten seit 2009 gemacht hat, dargestellt. Es geht vor
allem darum, wie gross die Verluste bei den Geräten war. Gleichzeitig
wird begründet, warum diese Bibliothek überhaupt eine solche Sammlung
hat (nicht nur für die Freizeitgestaltung der Studierenden, sondern auch
für Unterricht und Forschung) und wie das konkrete Bestandsmanagement
der Geräte organisiert ist (inklusive Pläne für die Überprüfung von
Vollständigkeit und Funktionalität bei der Rückgabe, der Pflege der
Geräte sowie des Managements von Verlust und Verbrauch). Relevant ist,
dass die Zahlen zwar einen kontinuierlichen \enquote{Verbrauch} an
Geräten (im Sinne von Verlust und von Defekten) von rund 19\,\% über den
gesamten Zeitraum zeigen, die Autor*innen aber schliessen, dass sich
dies im erwartbaren Rahmen bewegt. Die Defekte hätten sich alle durch
normalen Gebrauch der Geräte ergeben, nicht durch mutwillige Zerstörung.
Nach über zehn Jahren mit dieser Sammlung stellen sie fest: \enquote{One
of the biggest hurdles related to beginning this type of collection is
overcoming undue fears about equipment loss in relation to other library
collections.} (ebenda: 64). (ks)

\hypertarget{monographien-und-buchkapitel}{%
\section{3. Monographien und
Buchkapitel}\label{monographien-und-buchkapitel}}

\hypertarget{vermischte-themen-1}{%
\subsection{3.1 Vermischte Themen}\label{vermischte-themen-1}}

Munn, Luke (2022). \emph{Automation is a Myth}. Stanford: Stanford
University Press, 2022, \url{https://doi.org/10.1515/9781503631434}
{[}gedruckt beziehungsweise Paywall{]}

Der Titel dieses eher kurzen Buches sagt schon, dass der Autor hier die
Vorstellung einer \enquote{Automation}, welche sich in naher Zukunft auf
der ganzen Welt verbreiten und das Leben aller Menschen, insbesondere
ihrer konkreten Arbeit, verändern würde, als einen modernen Mythos
analysiert. Er ist Forscher im Bereich zwischen Technik und Kultur, also
grundsätzlich Technologie und ihrer Entwicklung nicht abgeneigt. Aber in
diesem Buch insistiert er darauf, dass die Rede von einer
grundsätzlichen Automatisierung falsche Behauptungen aufstellt, welche
sich anhand einer Analyse der realen Entwicklungen widerlegen lassen.
Dies funktioniert auch, weil diese Behauptungen keineswegs neu sind,
sondern seit langer Zeit existieren. Seine Kritik ist dabei nicht auf
eine Gruppe von Personen gerichtet. Vielmehr betont er explizit, dass
sie sowohl für diejenigen gilt, welche Automatisierung als problematisch
begreifen (im Sinne von \enquote{Vernichtung von Arbeitsplätzen}), als
auch für diejenigen, welche Automatisierung als grundsätzlich positiv
ansehen (im Sinne der Möglichkeit, dass Menschen weniger Arbeit leisten
müssen).

In den Kapiteln geht er mehrere Unterpunkte des Mythos durch und betont,
dass das, was \enquote{Automatisierung} konkret heisst, wie sie
umgesetzt wird und wofür, immer vom lokalen Kontext abhängig ist und für
Menschen immer, stratifiziert anhand von gesellschaftlichen Strukturen
(zum Beispiel hoch- und niedrigqualifizierte Arbeiter*innen, nationale
Minderheiten und Mehrheiten, Geschlechter), Unterschiedliches bedeutet.
Insbesondere bestreitet er, dass Automatisierung Arbeit ersetzt und
zeigt, dass sie stattdessen selbst immer neue Arbeit hervorbringt. Das
Ganze ist in einem leicht zugänglichen, essayistischen Stil geschrieben.
Gerade da Automatisierung heute vor allem Datenanalyse bedeutet, ist das
Buch auch eine gute Erinnerung daran, dass jede technische Entwicklung
und ihr Einsatz nicht einfach kontextlos \enquote{passiert}, sondern
immer gesteuert werden kann. (ks)

Vnuk, Rebecca (2022). \emph{The Weeding Handbook: A Shelf-by-Shelf
Guide.} (Second Edition) Chicago: ALA Editions, 2022.

Es fehlt an deutschsprachiger Literatur, die sich mit den praktischen
Fragen des Bestandsmanagements beschäftigt. Unter den zahlreichen
Handbüchern und Praxishilfen für Bibliotheken, die kontinuierlich
erscheinen, finden sich erstaunlich wenige, die zum Thema machen, wie
Medien erworben oder lizenziert, wie sie verwaltet, ausgeliehen,
magaziniert oder auch ausgesondert werden. Dabei ist dies weiterhin der
Hauptteil der praktischen Arbeit in Bibliotheken, in Öffentlichen
vielleicht noch mehr als in anderen Bibliothekstypen. Im
englischsprachigen Raum gibt es mehr solcher Publikationen. Eventuell
hat dies einfach mit dem zahlenmässig grösseren Markt für
Bibliotheksliteratur zu tun oder auch damit, dass mit den Verlagen des
US-amerikanischen (und, im Fall anderer solcher Literatur, britischen)
Bibliotheksverbandes explizit auf die Zielgruppe Bibliothekar*innen
ausgerichtete Publikationsorte existieren. So oder so: Die
englischsprachigen Handbücher zum Bestandsmanagement sind grundsätzlich
auch für Bibliotheken im deutschsprachigen Raum von Interesse, weil sie
grundsätzlich die gleichen Fragen behandeln -- egal, ob sich die
Bibliotheken in Kansas oder Bochum befinden. Gleichzeitig aber sind die
jeweiligen Beispiele, Formeln, Gesetze, Infrastrukturen und so weiter in
diesen Büchern immer auf die USA (oder, in anderen Fällen, auf
Grossbritannien) bezogen.

Dies gilt auch für dieses Buch zur planmässigen Aussonderung in Public
Libraries und Schulbibliotheken. Es gibt einen recht schnell und einfach
zu lesenden Überblick zum Thema, also vor allem zu der Frage, warum in
Bibliotheken ausgesondert werden muss, wie dabei vorzugehen ist, auch
wie das Personal und die Öffentlichkeit dabei einbezogen werden können.
Gleichwohl ist es an den US-amerikanischen Kontext angepasst. In weiten
Teilen des Buches geht die Autorin die einzelnen Signaturgruppen der
Dewey Decimal Classification durch und beschreibt jeweils, auf was bei
den Medien in diesen Gruppen zu achten ist -- aber die Bibliotheken im
DACH-Raum nutzen alle andere Aufstellungsklassifikationen.

Und dennoch ist das Buch zu empfehlen. Einerseits sind die
Grundprinzipien auch ausserhalb der USA gültig: (1) Eine Bibliothek
benötigt die ständige Aussonderung, sie ist weder Archiv noch Museum.
(2) Aussonderung sollte sich an klaren Kriterien orientieren, aber dann
doch jeweils pro Medium abwägen, wie hart sie angewandt werden müssen.
(3) Aussonderung geschieht im Idealfall ständig und geplant, zum
Beispiel anhand einer Jahresplanung. Andererseits sind die Interviews
mit Bibliothekar*innen, welche über das ganze Buch gestreut sind,
hilfreich. Sie berichten aus sehr unterschiedlichen Bibliotheken, aber
es wird sichtbar, dass sie alle mit den gleichen Problemen umgehen
müssen. Es ist also in gewisser Weise \enquote{universell}. Zuletzt
hilfreich ist der mehrfach gemachte Vorschlag, einen \enquote{cart of
shame} zu betreiben, auf dem wirklich schlimme Medien behalten werden,
die ausgesondert wurden (also sowohl inhaltlich als auch vom Zustand her
schlimm). Sie überzeugen sowohl Kolleg*innen als auch die Träger, die
lokale Politik oder auch die breite Öffentlichkeit schnell von der
Notwendigkeit eine kontinuierlichen Aussonderung. (ks)

\hypertarget{bibliotheks--und-mediengeschichte}{%
\subsection{3.2 Bibliotheks- und
Mediengeschichte}\label{bibliotheks--und-mediengeschichte}}

Kienhorst, Hans ; Poirters, Ad (2023). \emph{Book Collections as
Archaeological Sites: A Study of Interconnectedness and Meaning in the
Historical Library of the Canonesses Regular of Soeterbeeck ; with a
Catalogue of the Soeterbeeck Collection, Compiled with Eefje Roodenburg,
and Pictures by Anton Houtappels}. (Nijmegen Art Historical Studies
XXIX) Turnhout: Brepols Publishers, 2023. {[}gedruckt{]}

Dieses Buch ist eines, das sich in gewisser Weise aufdrängt -- in seiner
physischen Form. Es ist in einem grossen Format (28 × 21 cm) gedruckt,
fast 700 Seiten dick und zudem auf extra schwerem, gestrichenen Papier
gedruckt, mit zahllosen, gross gesetzten (aber auch qualitativ auffällig
guten) Bildern. Kurz: Es liegt nicht in der Hand, sondern muss
geschleppt und kann praktisch nur am Schreibtisch gelesen werden. Es
verlangt Aufmerksamkeit.

Aber ist sie berechtigt? Thema des Buches ist die Bibliothek, welche im
Laufe der Jahrhunderte in einem Konvent des katholischen Frauenordens
der Sepulchrinerinnen in Soeterbeek (in Dommel, Niederlande) angesammelt
wurde, wobei die Autoren zu dieser Bibliothek alle Handschriften und
Bücher zählen, die einst in diesem Konvent vorhanden waren, egal wo sie
heute stehen. Der Konvent, mit den ersten Anfängen im Jahr 1448, wurde
1997 aufgelöst, als die letzte Schwester in ein Altersheim zog. Die
Medien stehen verstreut in verschiedenen Bibliotheken in den
Niederlanden (wenn auch mit Schwerpunkten). Was das Buch jetzt liefert,
ist, jedes einzelne dieser Bücher eingehend zu beschreiben, inklusive
seiner Materialität, seines Inhalts, seiner Nutzungsspuren und es dann,
jeweils als Teil der Sammlung, in einen grösseren Kontext einzuordnen.
Beispielsweise wird thematisiert, welche Rolle solche Genres wie
Stundenbücher oder Katechismen zu verschiedenen Zeiten spielten. Die
These, welche die Autoren immer wieder hervorheben, ist, dass man die
einzelnen Bücher als Objekte einer \enquote{Archäologie} nutzen kann und
somit von ihnen ausgehend auch mehr über das Leben im Konvent selbst
lernt. Dies führen sie exemplarisch (immer gedacht, dass dies auch in
anderen \enquote{alten Bibliotheken} möglich ist) an dieser Bibliothek
vor, unterstützt durch Bilder aller Bücher und -- was den zweiten Teil
des Buches darstellt -- einen möglichst vollständigen Katalog.

Grundsätzlich ist das überzeugend. Wer sich durch das Buch
hindurcharbeitet, hat nachher einen Einblick in die gesamte Geschichte
des Konvents und auch das Leben und Handeln der Schwestern, inklusive
der Veränderungen, die sich in der Geschichte des Konvents ergaben. Aber
was nicht überzeugt, ist die Behauptung, dass dies etwas Neues wäre,
welche die Autoren kontinuierlich wiederholen. Bücher und Handschriften
auch als Objekte zu untersuchen und dabei nicht nur auf den Inhalt
einzugehen, sondern auch auf die Ikonographie, die Schrift, das
Material, die Druck-, Schreib- und Maltechniken und die verschiedenen
Nutzungsweisen zu thematisieren, ist heute ebenso etabliert, wie die
Untersuchung von Mediensammlungen als Gesamtheit (also mit Fragen
danach, was gesammelt wurde, wie es geordnet wurden und so weiter). Es
wird nur selten so eingehend und tiefgreifend anhand einer spezifischen
Sammlung gemacht, wie in diesem Buch (auch wenn es das, insbesondere für
verstreute Sammlungen aus Konventen schon mehrfach gab). Es ist ein
Buch, bei dem man sich nachher (wenn man es in die Bibliothek
zurückgeschleppt hat), trotz allem Respekt für die Arbeit, die hier
geleistet wurde, fragt, ob sie wirklich notwendig war. (ks)

Sharpe, Richard (2023). \emph{Libraries and Books in Medieval England.
The Role of Libraries in a Changing Book Economy ; The Lyell Lectures
for 2018--2019}. Oxford: Bodleian Library Publishing, 2023.
{[}gedruckt{]}

Richard Sharpe war Professor in Oxford mit einem Fokus auf
mittelalterliche Bücher und Diplomatie. Er war auch eine treibende Kraft
hinter der dritten (dann digitalen) Version der Bibliographie
\enquote{Medieval Libraries of Great Britain} (MLGB), die alle Bücher,
bei denen sich das Vorhandensein in britischen Bibliotheken des
Mittelalters nachweisen lässt, systematisch erfasst. Die Bibliographie
ist eines der wichtigen Hilfsmittel für die mediävistische Forschung in
Grossbritannien und in gewisser Weise eines dieser Werke, die von der
übersichtlichen Community britischer Mediävist*innen gemeinsam
erarbeitet, kritisiert und ergänzt wird. Das hier angezeigte Buch stellt
nun eine leicht überarbeitete Reihe von Vorlesungen vor, die Sharpe am
Ende der Arbeit an der neuen Ausgabe der MLGB gehalten hat. In gewisser
Weise reflektiert er hier über die Grenzen dieser Bibliographie, aber
auch über weitere Fragen zum Thema mittelalterliche Bibliotheken. Die
Zielgruppe der Vorlesungen ist nicht benannt, aber man hat die gesamte
Zeit den Eindruck, dass er sich vor allem an seine Fachkolleg*innen
richtet. Nicht nur geht er teilweise tief in die eigentlichen Dokumente
(Ex-Libris und handschriftliche Vermerke in Büchern selbst,
Bibliothekskatalogen und andere Listen) hinein, um seine Thesen zu
untermauern, wie dies in der Mediävistik normal ist. Er scheint auch
implizit immer ein Wissen über die heutigen Bestände an
mittelalterlichen Büchern in Oxford und Cambridge sowie grundsätzliches
Wissen über die Geschichte Englands vorauszusetzen. Die ganze Zeit über
erzeugt das Buch ein wenig den Eindruck, als sei man als Leser*in selbst
in Oxford, fest angestellt und mit der Zeit ausgestattet, sich intensiv
und tief in die dort vorhandenen Quellen versenken und an ihnen sehr
spezifische Fragen bearbeiten zu können, während draussen der Herbst
aufzieht und Tee getrunken wird. Insoweit ist es für Personen, die nicht
aus der Mittelalterforschung kommen, teilweise schwer zu lesen -- ein
wenig wie Nachrichten aus einer leicht entrückten Welt. Gleichwohl: Es
ist in gewisser Weise auch das Vermächtnis eines Forschers, der hier
seinen Wissensschatz und grundsätzlichen Überlegungen zum Thema so
ausbreitet, dass sie eine Basis für die weitere Entwicklung seiner
Profession darstellen können. Denn unerwartet starb Sharpe Anfang 2020,
also kurz nach dem Ende dieser Vorlesungsreihe, an einem Herzinfarkt.
(Vergleiche den Nachruf auf ihn: Ramsay, Nigel (2020). Richard Sharpe
obituary. In: The Guardian, 05.05.2023,
\url{https://www.theguardian.com/education/2020/may/05/richard-sharpe-obituary})

Was Sharpe in diesem Buch betont, ist, dass es weiterhin, auch in der
Forschung selbst, ein zu statisches Bild von mittelalterlichen
Bibliotheken gäbe. Zu diesem trüge auch die Bibliographie, die er ja mit
betreute, selbst bei, weil sie zu sehr den Eindruck vermittelt, dass
Bibliotheken praktisch immer nur wachsende Sammlungen gewesen wären. Er
verkompliziert eine ganze Anzahl dieser Vorstellungen über
mittelalterliche Bibliotheken. Bibliotheken wären, wie heute, keine
Sammlungen gewesen, sondern es wäre in ihnen -- wenn wir die heutige
bibliothekarische Terminologie verwenden wollen -- Bestandsmanagement
betrieben worden. Bestände seien, je nach den Interessen und
Entwicklungen von Klöstern, auf- und abgebaut sowie ausgetauscht worden.
Die MLGB würde immer nur zeigen, dass ein Buch zu einem bestimmten
Zeitpunkt in einer Bibliothek vorhanden gewesen wäre, aber nicht, wie
lange. Zudem zeigt er, dass das Bild von Mönchen, die in Skriptorien
Bücher kopieren, einerseits zu einfach ist und offenbar Schreiber
ignoriert, die \enquote{von aussen} kamen und durch die Klöster für
spezifische Kopier- und Schreibarbeiten angestellt wurden sowie
andererseits auch solche Fragen ausblendet wie die, wo die zu
kopierenden Bücher und diese Schreiber, die man anstellen konnte,
überhaupt hergekommen wären. Grundsätzlich postuliert er, (1) dass es
auch schon vor dem Buchdruck in Grossbritannien einen existierenden
Markt für die Produktion von Büchern gab, (2) dass an diesem nicht nur
Klöster und Universitäten beteiligt waren sowie (3) dass Bibliotheken
auch im Mittelalter eher \enquote{lebende Organismen} waren und nicht
immer nur wachsende Sammlungen. Dies alles gälte es in Zukunft zu
beachten, wenn man über diese Bibliotheken nachdenken möchte. (ks)

Forrest Kelly, Thomas (2019). \emph{The Role of the Scroll: An
Illustrated Introduction to Scrolls in the Middle Ages}. New York ;
London: W.W. Norton \& Company, 2019. {[}gedruckt{]}

Kössinger, Norbert (2020). \emph{Schriftrollen: Untersuchungen zu
deutschsprachigen und mittelniederländischen Rotuli}. (Münchener Texte
und Untersuchungen zur deutschen Literatur des Mittelalters, 148)
Wiesbaden: Reichert Verlag, 2020. {[}gedruckt{]}

Die beiden hier gemeinsam angezeigten Bücher haben das gleiche Thema und
kommen auch in vielem zu den gleichen Aussagen. Aber sie beziehen sich
überhaupt nicht aufeinander -- was eventuell damit zu erklären ist, dass
das zweite Buch (Kössinger 2020) eine Habilitation darstellt, die schon
2014 verteidigt, aber erst 2020 publiziert wurde, während das erste Buch
(Forrest Kelly 2019) in der Zeit dazwischen erschien.

Zuerst zum Buch von Forrest Kelley (2019). Von der Aufmachung und auch
den Texten her scheint dieses Buch für die Auslage in Museumshops
konzipiert worden zu sein: Geschrieben von einem etablierten
Musikwissenschaftler, der inhaltlich auch im Thema steht, hat es
trotzdem den Anschein eines Coffee Table Books. Gedruckt auf gutem,
schwerem Papier, in vollen Farben, mit Abbildungen auf praktisch allen
Druckseiten, die oft auch den ganzen verfügbaren Platz einer Seite
einnehmen. Zudem ist der Text eher kurzweilig und ohne grossen
wissenschaftlichen Apparat (also ohne viele Fussnoten und so weiter)
geschrieben. Es ist ein schönes Buch, mit zahlreichen, ebenso schönen
Abbildungen von Schriftrollen oder Details dieser Rollen. Durchgängig
sind die Bilder zudem in umfangreichen Bildunterschriften kommentiert
und werden so jeweils ohne den Haupttext kontextualisiert. Es ist in
gewisser Weise ein \enquote{Bilderbuch} über die im Titel genannten
Schriftrollen aus dem (europäischen) Mittelalter. Oder anders gesagt:
Das Buch kann man, positiv gemeint, als \enquote{public science}
beschreiben.

Gleichwohl hat der Autor eine These, die er auch gut nachvollziehbar
erhärtet, und die für die Mediengeschichte relevant ist: Die Etablierung
des Codex -- also des Buches -- im vierten Jahrhundert unserer
Zeitrechnung hat nicht dazu geführt, dass Menschen aufgehört hätten,
Schriftrollen zu erstellen und zu nutzen. Es gab einen Medienwandel aber
kein vollständiges Verschwinden der Schriftrollen. Vielmehr etablierten
sich für Schriftrollen im Mittelalter spezifische Nutzungsweisen. Das
gesamte Mittelalter über wurden Schriftrollen erstellt, gelesen und
benutzt -- und dabei hat sich offenbar auch das Wissen darüber, was
Schriftrollen sind und wie sie produziert wurden, erhalten. Der Autor
zeigt das, indem er fünf verschiedene Nutzungsformen von Schriftrollen
(als Listen, als Karten und Zeitleisten, als Texte für
Schauspieler*innen, als private religiöse Objekte sowie als Teil von
Ritualen) durchgeht und zu diesen jeweils einige Beispiele detailliert
darstellt. (Zuvor betont er, dass er sich auf Europa fokussiert, liefert
aber dennoch einen Überblick zur Nutzung von Schriftrollen in anderen
Kulturen.) Was er damit erreicht, und immer wieder selbst thematisiert,
ist, die Vorstellung von einem Medienwandel, bei dem eine Medienform
(Schriftrollen) vollständig durch eine andere (den Codex) ersetzt wird,
als zu einfach zu kennzeichnen. Dies ist grundsätzlich keine neue
Erkenntnis, aber hier an einem Beispiel exemplifiziert, dass sonst als
prototypisch für solch einen \enquote{Medienbruch} dargestellt wird --
nämlich dem angeblichen \enquote{Ende} der Schriftrollen durch die
Etablierung des Buches. Aber selbst hier stimmt das einfache Bild nicht.
Vielmehr war die Realität komplexer: Kodex und Schriftrollen existierten
über Jahrhundert \enquote{nebeneinander}.

Im zweiten hier besprochenen Buch (Kössinger 2020) geht es ebenfalls um
Schriftrollen des europäischen Mittelalters (der Autor macht
diesbezüglich ähnliche Einschränkungen seiner Untersuchungsobjekte und
gibt ähnliche Hinweise dazu, dass Schriftrollen in anderen Kulturen
anders verwendet wurden, wie Forrest Kelly). Der Fokus liegt aber auf
deutsch- und niederländischsprachigen Rollen. Wie angedeutet, handelt es
sich bei dem Buch um eine Dissertation. Dies ist ihm anzumerken, nicht
nur, aber auch im textlastigen Layout. Abbildungen der untersuchten
Rollen sind vorhanden, aber in einem gesonderten Bildteil am Ende des
Buches sowie in einer Anzahl von Beilagen. In der Arbeit wird versucht,
diese Schriftrollen -- oder, wie im Titel zu lesen ist, \enquote{Rotuli}
-- möglichst genau zu analysieren, vor allem mit den Methoden der
Philologie. Aber der Autor liefert auch eine Begriffsgeschichte für
\enquote{Rotuli} sowie mehrere Typologien (zum Beispiel nach Inhalt oder
Verwendungszweck). Zudem bemüht er grundsätzlich die Medientheorie, um
am Ende eine eigene Medientheorie für mittelalterliche Schriftrollen zu
skizzieren. Er versammelt und systematisiert weiterhin die heute
existierenden Schriftrollen, die im Fokus seiner Arbeit stehen.
Hauptteil der Arbeit ist die philologische Analyse ausgewählter Rollen,
was heisst, dass diese formal und inhaltlich beschrieben werden und
anschließend eine Edition (also eine Abschrift des Textes, inklusive
Anmerkungen zu Varianten) sowie eine kritische Einordnung dieses Textes
vorgenommen werden. Das ist alles sichtbar eine umfangreiche Arbeit
gewesen, die in einer Habilitation notwendig ist -- aber als Buch ist
das alles recht schwer zu lesen, teilweise auch sehr disparat.
Interessant ist das letzte, dann recht kurze Kapitel mit der Skizze
einer Medientheorie von Rotuli: Hier betont auch Kössinger, dass der
Medienwandel hin zum Codex nicht bedeutete, dass die Schriftrollen
verschwanden, sondern vielmehr, dass sie neue, eigenständige Funktionen
übernahmen. Er insistiert darauf, dass dies erreicht wurde, indem Rollen
-- im Gegensatz zum Codex, aber auch zu Schiefertafeln oder Steinen, auf
denen Text angebracht wurde -- praktisch unendlich verlängert werden
konnten und damit ein anderes Schriftbild (mise en page) hervorbrachten,
als es der Codex mit seinen fest definierten Seiten tat. Dies wurde dann
zum Beispiel für Listen, liturgische Texte oder Theatertexte genutzt.
Grundsätzlich liefert Kössinger tiefergehende Analysen einzelner
Schriftrollen als Forrest Kelly und auch mehr Typologien. Aber in der
Grundaussage stimmen beide überein. (ks)

Camarade, Hélène; Galmiche, Xavier ; Jurgenson, Luba (dir.) (2023).
\emph{Samizdat: Publications clandestines et autoédition en Europe
centrale et orientale (années 1950-1990)}. Paris: Nouveau Monde
éditions, 2023. {[}gedruckt{]}

Samizdat ist der Sammelbegriff für Medien, die während des \enquote{real
existierenden Sozialismus} in den betreffenden Ländern, aber ausserhalb
der offiziellen Kanäle, erstellt und verteilt wurden. Wie genau dies
vonstatten ging, wer diese Medien erstellte und aus welchen Gründen, war
sowohl zu verschiedenen Zeitpunkten als auch in den verschiedenen
Ländern jeweils unterschiedlich. Beispielsweise wurden in der
Sowjetunion nach dem Tod Stalins 1953 mehr Samizdat-Publikationen
produziert als zuvor und dann zur Zeit der Perestroika noch mehr. In der
DDR wurden Spielräume der Kirchen ausgenutzt, deren Medien für
\enquote{den internen Gebrauch} von der Zensur ausgenommen waren --
Spielräume, die es in anderen Ländern so nicht gab. In Polen dagegen war
die Solidarność ein Katalysator für die Produktion und Verbreitung von
Samizdat.

Das von Camarade, Galmiche und Jurgenson herausgegebene Buch liefert nun
eine Übersicht über die verschiedenen Formen und Themen des Samizdat.
Während dies bislang zumeist für einzelne Länder gemacht wurde, ist der
Anspruch hier, eine internationale Sicht zu liefern. Gelöst wurde dieser
Anspruch mit einer Sammlung von Beiträgen, die dann aber selbst meist
doch auf einzelne Länder fokussieren. Die meisten Autor*innen des Buches
sind Forschende an französischen Hochschulen, aber es sind auch
Autor*innen aus den betreffenden Ländern vertreten. Leider sind die
einzelnen Artikel recht kurz und oft nur erste Skizzen. Oft sind es
wenig mehr als Aufzählungen von Samizdat-Publikationen aus jeweils einem
Land. Erst in späteren Teilen des Buches geht es um inhaltliche Fragen
(beispielsweise jüdischer Samizdat, christlicher Samizdat, Samizdat zu
LGBTQ-Themen) sowie in einem Text auch um technische Fragen der
Samizdat-Produktion (Galmiche, Xavier: \emph{Fabrication et économie du
samizdat}. In: ebenda, S. 195--202.). Eine kleine Zahl von Texten geht
auch auf \enquote{Tamizdat} (im Ausland erschienene Literatur aus diesen
Ländern, die oft wieder in das betreffende Land eingeschmuggelt oder
dort reproduziert wurde) und \enquote{Magnitizdat} (Publikationen auf
Tonband und Kassetten) ein. Eine wirkliche Übersicht, welche über ein
Land hinausgeht und dabei auch Besonderheiten des Samizdat thematisiert,
beispielsweise dass diese nur zum Teil überliefert ist, da sie ja oft
gerade \enquote{unter der Hand} verteilt wurde, liefert nur die
Einleitung (Camarade, Hélène ; Galmiche, Xavier ; Jurgenson, Luba:
\emph{Introduction: Penser le samizdat. Définitions, histoires,
perspectives}. In: ebenda, S. 9--27). Inhaltlich ist diese Einleitung
der dichteste Text des Werkes.

Verstreut finden sich im Buch Vorstellungen von Sammlungen von Samizdat,
beispielsweise der Sammlung der Robert-Havemann-Gesellschaft (Berlin)
oder dem Archiv Bürgerbewegung Leipzig. Das wird für weitergehende
Forschungen hilfreich sein. Zwei Artikel stellen zudem kurz
Digitalisierungsprojekte von Samizdat vor. Alles in allem ist das Buch
eine Einführung in das Thema, das auch versucht, die verschiedenen
internationalen Stände der Forschung zu referenzieren. Eine
übergreifende Studie zum Samizdat ist es allerdings nicht. (ks)

Adams, Robyn ; Glomski, Jacqueline (edit.) (2023).
\emph{Seventeenth-Century Libraries: Problems and Perspectives}.
(Library of the Written Word, 114 ; The Handpress World, 92). Leiden ;
London: Brill, 2023. {[}gedruckt{]}

In ihrem Vorwort behaupten die Herausgeber*innen dieses Buches, dass die
(europäischen) Bibliotheken des 17. Jahrhunderts -- also im Jahrhundert
nach der Erfindung des Buchdrucks mit beweglichen Lettern -- die
Grundlagen für die heutigen Bibliotheken gelegt hätten. Sie würden aber
von der Forschung trotz dieser Bedeutung nicht gebührend beachtet. Das
Buch soll einen ersten Schritt dahin machen. Aber dieser Anspruch wird
nicht eingelöst: Auch dieses Werk besteht aus einzelnen Beiträgen, die
einigermassen unvermittelt nebeneinander stehen und nur dadurch
zusammengehalten werden, dass sie sich mit Bibliotheken oder
Buchsammlungen des genannten Jahrhunderts befassen. Dabei sind die
einzelnen Artikel nicht zu kritisieren: Sie nähern sich alle historisch
einzelnen Fragen oder Objekten und untersuchen diese zumeist auf der
Basis der Buchgeschichte, teilweise bis hinein in kleinste Details. Aber
diese Fragen sind oft sehr spezifisch: In Richard Fosters Beitrag geht
es zum Beispiel um die Klassifikationen und Aufstellungssystematiken von
Universitätsbibliotheken in England, die sich aus vorhandenen Katalogen
erschliessen lassen; Robyn Adams beschäftigt sich mit dem Netzwerk,
welches Thomas Bodley aufbaute, um die von ihm gestiftet Bibliotheken in
Oxford langfristig abzusichern und Francesca Galligan untersucht
Reisebibliotheken, die für Adlige hergestellt wurden. Eine Anzahl von
Autor*innen macht sich auch Gedanken dazu, welche Forschungen durch die
Digitalisierung möglich wurden (wobei diese Überlegungen nicht wirklich
darüber hinausgehen, dass der Zugang zu digitalisierten Quellen möglich
geworden ist). Das Buch liest sich eher wie die Schwerpunktnummer einer
Zeitschrift, nicht wie ein in sich abgeschlossenes Werk, das die
Grundthese der Herausgeber*innen untermauern würde. (ks)

\hypertarget{social-media}{%
\section{4. Social Media}\label{social-media}}

{[}Diesmal keine Beiträge.{]}

\hypertarget{konferenzen-konferenzberichte}{%
\section{5. Konferenzen,
Konferenzberichte}\label{konferenzen-konferenzberichte}}

{[}Diesmal keine Beiträge.{]}

\hypertarget{populuxe4re-medien-zeitungen-radio-tv-etc.}{%
\section{6. Populäre Medien (Zeitungen, Radio, TV
etc.)}\label{populuxe4re-medien-zeitungen-radio-tv-etc.}}

ADN: \emph{Bücher sichergestellt}. In: Berliner Zeitung, 10/11.07.1993,
S. 17\\
Im Juli 1993 wurden in einem Berliner \enquote{Trödelladen} anhand von
Kennzeichnungen in den Exemplaren, aus der Amerika-Gedenkbibliothek
gestohlene Bücher durch einen Nutzer der Bibliothek und offenbar auch
des Geschäfts entdeckt. (bk)

dpa: \emph{Bibliotheken für starke Förderung der Demokratie}. In:
Kölnische Rundschau, 01.11.2023, S.\,8 / Kultur

Der Deutsche Bibliotheksverbund (dbv) positioniert sich eindeutig zur
von der Bundesregierung geplanten Kürzung der Mittel für die
Bundeszentrale für politische Bildung. Damit würde ein zentrales
Instrument der Demokratieförderung in Deutschland erheblich geschwächt.
Geplant ist ein Einkürzen der Mittel von 96 Millionen Euro im Jahr auf
76 Millionen. Die Entscheidung im Bundestag ist für den Dezember 2023
geplant. (bk)

Bain, Mark (2023). \emph{Central Library at 135: \enquote*{It's more
than books... it's about the people of Belfast}}. In: Belfast Telegraph,
13. Oktober 2023,
\url{https://m.belfasttelegraph.co.uk/life/books/central-library-at-135-its-more-than-books-its-about-the-people-of-belfast/a924956978.html}

Die Eröffnung der Zentralbibliothek in Belfast, Nordirland vor genau 135
Jahren nimmt dieser Artikel zum Anlass, kurz die Geschichte und heutige
Bedeutung der Bibliothek darzustellen. Erstaunlich ist tatsächlich, dass
sich die Bibliothek weiterhin in dem Gebäude befindet, das damals für
sie gebaut wurde. Grundsätzlich stellt der Text die Bibliothek sehr
positiv, als wichtige Einrichtung für die Bevölkerung der Stadt dar (die
allgemeinen Probleme britischer Public Libraries werden nicht
angesprochen, vielleicht weil es ein Jubiläum ist, das positiv begangen
werden soll). Der Text skizziert auch die Geschichte der Bibliothek in
den letzten 135 Jahren, wobei der Fokus auf der Gründung und den beiden
Weltkriegen liegt, während die Troubles nur in einem Nebensatz erwähnt
werden. Angereichert ist der Text mit zahlreichen Bildern aus der
Bibliothek selber, vor allem aus den 1960er Jahren. (ks)

AP: \emph{Alabama library mistakenly adds children's book to
\enquote{explicit} list because of author's name}. In: AP News. 10.
Oktober 2023
\url{https://apnews.com/article/alabama-childrens-book-banned-f5179e7fb9c523b84e901453e07cfb5d}

In der Huntsville-Madison County Public Library wurde das Buch
\enquote{Read Me a Story, Stella} der Autorin Marie-Louise Gay als
anstößig beanstandet und auf eine Liste nicht erwünschter Kinderbücher
gesetzt. Grund war, wie die Bibliotheksleiterin Cindy Hewitt erklärte,
der Nachname der Autorin, der ihr Kinderbuch in einem
\enquote{proaktiven} Bestandsscreening zu einem von 223 Titeln machte,
die als potentiell für Kinder ungeeignet eingeschätzt wurden. (bk)

Deutschlandfunk Kultur: \emph{Russische Bücher aus
Universitätsbibliothek in Warschau gestohlen}. In: Deutschlandfunk
Kultur. 02.11.2023
\url{https://www.deutschlandfunkkultur.de/russische-buecher-aus-universitaetsbibliothek-in-warschau-gestohlen-104.html}\\
mg/gp: \emph{Z Biblioteki Uniwersytetu Warszawskiego zginęły cenne
woluminy. Rektor poinformował o zwolnieniu dyrektorki}. In: TVN Waszawa.
31.10.2023
\url{https://tvn24.pl/tvnwarszawa/srodmiescie/warszawa-kradziez-w-bibliotece-uniwersytetu-warszawskiego-zwolniona-dyrektorka-placowki-decyzje-rektora-oswiadczenie-zwolnionej-kobiety-7417559}\\
Odessa Journal: \emph{Around 80 Russian books from the 19th century were
stolen from Warsaw University's library}. In: Odessa Journal. 02.11.2023
\url{https://odessa-journal.com/around-80-russian-books-from-the-19th-century-were-stolen-from-warsaw-universitys-library}

Aus der Universitätsbibliothek in Warschau wird ein großer
Bibliotheksdiebstahl gemeldet. Betroffen sind bis zu 80 russische Bücher
aus dem 19. Jahrhundert, die durch Dummies ersetzt und entwendet wurden.
Ein ähnliches Vorgehen wurde anscheinend zuvor bereits in der Lettischen
Nationalbibliothek festgestellt. Die Diebstähle erfolgten offenbar auf
Bestellung. Die betroffenen Bücher wurden gezielt aus dem Magazin
angefordert. Als Folge des Vorgangs verlor die Direktorin des Hauses,
Anna Wołodko, ihre Stelle. Auch eine geplante öffentliche Aufstellung
älterer Titel im allgemeinen Lesesaal wurde zurückgenommen. (bk)

Frank Bachner: \emph{22 Fälle in Tempelhof-Schöneberg: Berliner
Publizist wegen Buchzerstörungen und rechter Schmierereien angeklagt.}
In: TAGESSPIEGEL / tagesspiegel.de, 01.11.2023.
\url{https://www.tagesspiegel.de/berlin/22-falle-in-tempelhof-schoneberg-berliner-publizist-wegen-buchzerstorungen-und-rechter-schmierereien-angeklagt-10713955.html?bezuggrd=CHP\&utm_source=cp-vollversion}

In Berlin erhebt die Staatsanwalt Anklage gegen einen 32-jährigen Mann,
der in der Zentralbibliothek Tempelhof-Schöneberg 22 Bücher zerstört und
unter anderem mit Schriftzügen, die den russisches Angriffskrieg gegen
die Ukraine gutheißen, beschmiert haben soll. Weiterhin soll er in und
an der Bibliothek Schriftzüge mit rechtskonservativen Provokationen
hinterlassen haben. Wenn er keine Bücher zerstöre, publiziere er
Schriften \enquote{zum Thema Kaisertum und Monarchie}, so der
TAGESSPIEGEL. (bk)

Marcel Hilbert: \emph{\enquote{Unendlich traurig und wütend}.
Büchertresor unter der Platane schon wieder Ziel blinder
Zerstörungswut}. In: Ostthüringer Zeitung, 01.11.2023, Seite 13

Im thüringischen Gera wurde zum wiederholten Mal ein vor der Stadt- und
Regionalbibliothek aufgestellter offener Bücherschrank zerstört, dieses
Mal durch Brandstiftung. (bk)

mio: \emph{Unbekannte zertrümmern Scheiben von Bücher-Box}. In:
Nordkurier, 04.10.2023, S. 11

Auch in Neubrandenburg wurden zwei öffentliche Bücherschränke
beschädigt. In einem Fall wurden Scheiben einer entsprechend
umgewidmeten Telefonzelle eingeschlagen. In einem zweiten Fall wurden
Bücher angezündet. (bk)

Sead Fadilpašić: \emph{Even public libraries aren't safe from
ransomware, as Canada's biggest is hit}. In: Techradar, 02.11.2023.
\url{https://www.msn.com/en-us/money/other/even-public-libraries-arent-safe-from-ransomware-as-canadas-biggest-is-hit/ar-AA1jhWu5}

Die Toronto Public Library wurde im Oktober 2023 Opfer einer
Ransomware-Attacke. In der Folge konnten die Website,
Nutzer*innenaccounts und die digitalen Sammlungen nicht mehr aufgerufen
werden. Andere, externe, digitale Dienste wie Overdrive, Kanopy oder
BiblioBoard blieben nutzbar und wurden entsprechend auf einer temporären
Webseite verlinkt. (bk)

Ella Creamer: \emph{Most libraries to provide \enquote*{warm banks}
again this winter.} In: The Guardian. 17.10.2023
\url{https://www.theguardian.com/books/2023/oct/17/most-libraries-to-provide-warm-banks-again-this-winter}

Auch in diesem Jahr bieten sehr viele (93\,\%) der öffentlichen
Bibliotheken in England, Wales und Nordirland beheizte Räume für einen
Aufenthalt, also de facto Wärmestuben, an. Es ist eine Fortsetzung der
2022 initiierten \enquote{Warm Hubs}-Initiative, die in Folge von
Energiepreissteigerungen und generell steigenden Lebenshaltungskosten
eingeführt wurde. Ein Großteil (79\,\%) erwartet einen steigenden
Bedarf. Viele Bibliotheken planen, das Aufwärmangebot mit weiteren
Aktivitäten wie Spielesessions, Yoga, und Handwerksclubs sowie
Schulungen zu Themen wie Haushaltsplanung zu verbinden. (bk)

\hypertarget{abschlussarbeiten}{%
\section{7. Abschlussarbeiten}\label{abschlussarbeiten}}

{[}Diesmal keine Beiträge.{]}

\hypertarget{weitere-medien}{%
\section{8. Weitere Medien}\label{weitere-medien}}

Canadian School Libraries (2023). \emph{CSL Statement: Book Challenges
and Censorship in Canada's School Libraries}.
\url{https://www.canadianschoollibraries.ca/wp-content/uploads/2023/05/CSLstatement_BookChallenges_Censorship_May2023.pdf}

Diese offizielle Stellungnahme der professionellen Organisation der
kanadischen Schulbibliotheken gegen Versuche, Bücher aus den
Schulbibliotheken entfernen zu lassen, ist deshalb interessant, weil sie
sich nicht auf allgemeine Aussagen beschränkt. Stattdessen zeigt das
Statement, dass sich bei diesen Versuchen (a) um organisierte Kampagnen
handelt (es werden von Gruppen Listen von Büchern erstellt, die
\enquote{challenged} werden sollen sollen und dann en masse Beschwerden
geschrieben), und (b) dass es sich bei den Gründen, die für diese
Beschwerden angegebenen werden praktisch immer um vorgeschobene Gründe
handelt. Grundsätzlich gehe es den aktiven Gruppen darum, Bücher mit
LGTBQI+-Inhalten zu entfernen, egal, wie sie dies begründen.

Demgegenüber betont das Statement, dass Schulbibliotheken für alle
Schüler*innen zugänglich sind und für die Interessen aller Schüler*innen
Medien anbieten müssen. Es verweist auf spezifisch kanadische Gesetze
und richterliche Entscheidungen, die dies bekräftigen. Aber, und das ist
ebenso wichtig, dann erheben sie die Forderung, dass in professionelles
Schulbibliothekspersonal investiert werden muss, das die Ausbildung und
Durchhaltekraft hat, bei allen diesen Beschwerden auf die
professionellen, an grundsätzlichen Kriterien orientierten und
transparenten Entscheidungswege zu insistieren -- und so die
Schulbibliotheken gegen die politischen Kampagnen zu schützen.
{[}Vergleiche dazu auch die ergänzende Stellungnahme des kanadischen
Verlegers James Saunders (Saunders, James (2023). \emph{Making the Case
for Professionalism to Take on Book Challenges}. In: Canadian School
Libraries Journal 7 (2023) 2,
\url{https://journal.canadianschoollibraries.ca/making-the-case-for-professionalism-to-take-on-book-challenges/})
sowie weitere Artikel in der gleichen Ausgabe der genannten Zeitschrift
(\url{https://journal.canadianschoollibraries.ca/category/vol-7-no-2-spring-2023/}).{]}
(ks)

%autor

\end{document}
