\documentclass[a4paper,
fontsize=11pt,
%headings=small,
oneside,
numbers=noperiodatend,
parskip=half-,
bibliography=totoc,
final
]{scrartcl}

\usepackage[babel]{csquotes}
\usepackage{synttree}
\usepackage{graphicx}
\setkeys{Gin}{width=.8\textwidth} %default pics size

\graphicspath{{./plots/}}
\usepackage[ngerman]{babel}
\usepackage[T1]{fontenc}
%\usepackage{amsmath}
\usepackage[utf8x]{inputenc}
\usepackage [hyphens]{url}
\usepackage{booktabs} 
\usepackage[left=2.4cm,right=2.4cm,top=2.3cm,bottom=2cm,includeheadfoot]{geometry}
\usepackage[labelformat=empty]{caption} % option 'labelformat=empty]' to surpress adding "Abbildung 1:" or "Figure 1" before each caption / use parameter '\captionsetup{labelformat=empty}' instead to change this for just one caption
\usepackage{eurosym}
\usepackage{multirow}
\usepackage[ngerman]{varioref}
\setcapindent{1em}
\renewcommand{\labelitemi}{--}
\usepackage{paralist}
\usepackage{pdfpages}
\usepackage{lscape}
\usepackage{float}
\usepackage{acronym}
\usepackage{eurosym}
\usepackage{longtable,lscape}
\usepackage{mathpazo}
\usepackage[normalem]{ulem} %emphasize weiterhin kursiv
\usepackage[flushmargin,ragged]{footmisc} % left align footnote
\usepackage{ccicons} 
\setcapindent{0pt} % no indentation in captions
\usepackage{xurl} % Breaks URLs

%%%% fancy LIBREAS URL color 
\usepackage{xcolor}
\definecolor{libreas}{RGB}{112,0,0}

\usepackage{listings}

\urlstyle{same}  % don't use monospace font for urls

\usepackage[fleqn]{amsmath}

%adjust fontsize for part

\usepackage{sectsty}
\partfont{\large}

%Das BibTeX-Zeichen mit \BibTeX setzen:
\def\symbol#1{\char #1\relax}
\def\bsl{{\tt\symbol{'134}}}
\def\BibTeX{{\rm B\kern-.05em{\sc i\kern-.025em b}\kern-.08em
    T\kern-.1667em\lower.7ex\hbox{E}\kern-.125emX}}

\usepackage{fancyhdr}
\fancyhf{}
\pagestyle{fancyplain}
\fancyhead[R]{\thepage}

% make sure bookmarks are created eventough sections are not numbered!
% uncommend if sections are numbered (bookmarks created by default)
\makeatletter
\renewcommand\@seccntformat[1]{}
\makeatother

% typo setup
\clubpenalty = 10000
\widowpenalty = 10000
\displaywidowpenalty = 10000

\usepackage{hyperxmp}
\usepackage[colorlinks, linkcolor=black,citecolor=black, urlcolor=libreas,
breaklinks= true,bookmarks=true,bookmarksopen=true]{hyperref}
\usepackage{breakurl}

%meta
%meta

\fancyhead[L]{Redaktion LIBREAS\\ %author
LIBREAS. Library Ideas, 44 (2023). % journal, issue, volume.
%\href{https://doi.org/10.18452/27066}{\color{black}https://doi.org/10.18452/27066}
{}} % doi 
\fancyhead[R]{\thepage} %page number
\fancyfoot[L] {\ccLogo \ccAttribution\ \href{https://creativecommons.org/licenses/by/4.0/}{\color{black}Creative Commons BY 4.0}}  %licence
\fancyfoot[R] {ISSN: 1860-7950}

\title{\LARGE{Editorial \#44: Grassroots Open Access}}% title
\author{Redaktion LIBREAS} % author

\setcounter{page}{1}

\hypersetup{%
      pdftitle={Editorial \#44: Grassroots Open Access},
      pdfauthor={Redaktion LIBREAS},
      pdfsubject={LIBREAS. Library Ideas, 44 (2023).},
      pdfkeywords={Soziologie, Bibliothekswissenschaft, Bibliothekswesen},
      pdflicenseurl={https://creativecommons.org/licenses/by/4.0/},
      pdfcopyright={CC BY 4.0 International},
      pdfcontacturl={http://libreas.eu},
      pdfurl={https://doi.org/10.18452/27066},
      pdfdoi={10.18452/27066},
      pdflang={de},
      pdfmetalang={de}
     }



\date{}
\begin{document}

\maketitle
\thispagestyle{fancyplain} 

%abstracts

%body
Im Jahr 2023 wurde das zwanzigste Jubiläum der Berliner Erklärung über
offenen Zugang zu wissenschaftlichem Wissen\footnote{Berliner Erklärung
  über den offenen Zugang zu wissenschaftlichem Wissen:
  \url{https://openaccess.mpg.de/Berliner-Erklaerung}, 22. Oktober 2003}
begangen. Und zwar mit gemischten Gefühlen. Die Berliner Erklärung
markierte zwar weder den Ursprung von Open Access, noch war sie die
erste Deklaration zum Thema. Aber gerade im deutschsprachigen Raum gilt
sie als maßgebliches Schlüsseldokument. Auch wenn wir sie heute lesen,
können wir uns leicht vom Idealismus und den großen Zielen anstecken
lassen, die digitalen Möglichkeiten des damals noch jüngeren Internets
für eine neue, inklusivere, ausdrücklich nicht-kommerziell dominierte
Gestaltung der wissenschaftlichen Kommunikation zu nutzen.

Hunderte Institutionen weltweit haben die Erklärung über die Jahre
unterschrieben und sich zu ihren Zielen bekannt. Die Ziele der Erklärung
haben sich allerdings nur bedingt eingelöst. Zur Kommerzialisierung des
wissenschaftlichen Publizierens durch große Wissenschaftsverlage kam die
Kommerzialisierung des Open Access durch große Wissenschaftsverlage und
Datendienstleister hinzu, die zu genau den verstärkten Abhängigkeiten
führten, die gemäß der Ziele der Berliner Erklärung eigentlich abgelöst
werden sollten. Die technischen Innovationen dehnten die Möglichkeiten
der Verwertung sogar noch aus: Mittlerweile geht es nicht nur um die
kommerzielle Steuerung des offenen Zugangs zu Publikationen, sondern
auch um die Verwertung von Publikationsdaten und digitalen Datenspuren
wissenschaftlichen Handelns aller Art (das sogenannte Datentracking).
Wie sich zeigt, war Open Access offen genug, um auf der Offenheit
aufbauend neue Geschäftsmodelle und Sekundärmärkte zu schaffen. Diese
Märkte werden insbesondere in Deutschland derzeit durch den Abschluss
der Projekt DEAL-Verträge\footnote{DEAL-Konsortium:
  \url{https://deal-konsortium.de/}} gefestigt. Mensch mag diese
Verträge einerseits als große Errungenschaft einer Kooperation der
großen Wissenschaftsorganisation sehen und den daraus resultierenden
deutlichen Anstieg an Open Access in wissenschaftlichen Zeitschriften
feiern. Auf der anderen Seite stimmt uns Open-Access-Aktivist*innen, die
mindestens seit der Berliner Erklärung idealistischen Motiven folgend an
der Bewegung teilnehmen und diese voranbringen wollen, diese Entwicklung
weniger positiv. Ein wenig fühlt es sich an, als hätten wir gerade mit
dem konsequenten Plädoyer von Offenheit gerade die Strukturen gestärkt,
die wir eigentlich überwinden wollten. Daher fragen wir uns: Wie viel
haben wir wirklich für ein Open Access im Sinne der Berliner Erklärung
erreicht? Und führt dieser Weg des kommerziellen Open Access' nicht
wieder in eine Sackgasse? Ist es zu spät, einen anderen Weg zu gehen?

Zurückblickend wurden von den heute üblichen Wegen des Open Access im
Jahr 2003 vor allem zwei diskutiert: Gold, also die
Open-Access-Erstpublikation in Zeitschriften und Grün, also die
Zweitveröffentlichung von Publikationen auf Repositorien. Die
kommerziellen Verlage entwickelten aus dem Gold etwas, was für sie
glänzt, nämlich das Geschäftsmodell der Open-Access-Publikationsgebühren
für Zeitschriftenartikel. Die üblichen Macht-, Wirk- und
Reputationsmechanismen in der Wissenschaft sorgten dafür, dass sich an
vielen Stellen des Publikationsverhaltens wenig grundsätzlich ändert. Am
Ende zahlen öffentliche Institutionen -- und jetzt eben für die
Publikation und nicht mehr für den Bezug der Zeitschriftentitel.
Beziehungsweise manchmal auch für beides.

Repositorien, meist auch von öffentlichen Einrichtungen betrieben,
entwickelten dagegen keine oder kaum monetäre Geschäftsmodelle. Deshalb
wird grünes Open Access heute mit nicht-kommerziellem Open Access
gleichgesetzt und bildet so einen noch deutlicheren Kontrast zum
Goldenen Weg.

\begin{figure}
\centering
\includegraphics{fig.jpg}
\caption{Redaktionsorte XXIII: Am Schreibtisch in Brandenburg an der
Havel.}
\end{figure}

Die Open-Access-Farbenlehre hat sich seit 2003 dynamisch
entwickelt\footnote{Siehe etwa die Präsentation von Stefan Schmeja aus
  dem Oktober 2020: \enquote{Der Open-Access-Regenbogen - welche Farben
  hat er und brauchen wir sie
  wirklich?}\url{https://doi.org/10.5281/zenodo.4133872}. Dass es auch
  für die vermeintliche Spezifizierung \enquote{Diamond Open Access}
  unterschiedliche Lesarten geben kann, zeigten Kolleg*innen der TIB
  Hannover anschaulich: Dellmann, S., van Edig, X., Rücknagel, J., \&
  Schmeja, S. (2022). Facetten eines Missverständnisses: Ein
  Debattenbeitrag zum Begriff „Diamond Open Access". O-Bib. Das Offene
  Bibliotheksjournal / Herausgeber VDB, 9(3), 1--12.
  \url{https://doi.org/10.5282/o-bib/5849}}. Hinzugekommen ist
insbesondere die Spielart \enquote{Transformationsvertrag}, die aktuell
in Deutschland besonders prominent über das DEAL-Konsortium
implementiert wird. Auf Ebene der Wissenschaftspolitik scheint dabei ein
blinder Fleck zu bestehen und größer zu werden, hat doch die Evaluation
von in anderen Ländern vorhandenen Transformationsverträgen strukturelle
Probleme offenbart: Analysen der europäischen coalitionS\footnote{Robert
  Kiley (2020): \enquote{Transformative Journals: analysis from the 2022
  reports}
  \url{https://www.coalition-s.org/blog/transformative-journals-analysis-from-the-2022-reports/}},
des Rates der Europäischen Union\footnote{Council of the European Union
  (2023), No 9616/23 (\enquote{High-quality, transparent, open,
  trustworthy and equitable scholarly publishing})
  \url{https://data.consilium.europa.eu/doc/document/ST-9616-2023-INIT/en/pdf}}
und des schwedischen Hochschulverbunds (SUHF)\footnote{Sveriges
  universitets- och högskoleförbund, SUHF (2023): \enquote{Open access:
  Need to move away from transformative agreements}, 20.10.2023,
  \url{https://www.su.se/english/news/open-access-need-to-move-away-from-transformative-agreements-1.683787}}
stellen fest, dass durch Transformationsverträge vielmehr eine
Konsolidierung (statt flächendeckender Transformation) und
Fortschreibung des Status Quo im wissenschaftlichen Publikationsmarkt
stattfindet. Sie empfehlen daher, Abstand von Transformationsverträgen
zu nehmen. Und doch beobachten wir in Deutschland einen gegenteiligen
Trend: Die Liste \enquote{Welche Einrichtungen sind schon dabei} des
DEAL-Konsortiums informiert nahezu tagesaktuell über die Anzahl der
Einrichtungen, die einen DEAL-Vertrag mit Elsevier mit einer
fünfjährigen Laufzeit (2024--2028) abgeschlossen haben.\footnote{\url{https://deal-konsortium.de/vertraege/elsevier}}
Big Open Access scheint sich auch entgegen den Erkenntnissen der
Wissenschaftstrukturforschung durchzusetzen.

Wie viel Platz bleibt da noch für Grassroots Open Access, also Ansätze,
die sich stärker den nicht-kommerziellen und idealistischen Gedanken der
Open Access-Erklärungen verpflichtet fühlen? Es mag nicht die Quantität
des Publikationsaufkommens darstellen, doch es gibt sie -- die
Alternativen zu Big Open Access. Und wir freuen uns sehr, dass im
Schwerpunkt dieser Ausgabe verschiedene Ansätze vorgestellt werden.

\hypertarget{zum-schwerpunkt}{%
\section{Zum Schwerpunkt}\label{zum-schwerpunkt}}

Stefan Milius und Wolfgang Thomas stellen die Zeitschrift \emph{Logical
Methods in Computer Science} vor, die bereits 2004 (also ein Jahr nach
der Berliner Erklärung) als Diamond-Open-Access-Zeitschrift von
Wissenschaftler*innen für Wissenschaftler*innen in der Theoretischen
Informatik gegründet wurde. In ihrem Erfahrungsbericht schildern sie
ausführlich, wie nach anfänglicher Improvisation die zunehmende
Professionalisierung gelang und welche Herausforderungen (und
Möglichkeiten) sich durch das verlagsunabhängige Agieren ergeben.

Enrique Corredera und Valérie Andres stellen das Schweizer Projekt GOAL
vor, in dessen Rahmen Mitarbeitende aus fünf Fachhochschulen und
Pädagogischen Hochschulen aus der Schweiz den grünen Weg -- also das
Open-Access-Zweitveröffentlichen auf Repositorien -- zu stärken, indem
einerseits Kooperationen mit kleinen und mittelständischen Verlagen auf-
und ausgebaut (insbesondere mit Fokus auf Open-Access-Verlagspolicies)
und andererseits vorbildhafte, nachnutzbare Workflows für Repositorien
entwickelt werden.

Tobias Steiner widmet sich dem Themenfeld wissenschaftsgeleitetes
(Open-Access-) Publizieren. Dabei liefert er einführend eine
umfangreiche Erläuterung des Terms \enquote*{scholar-led}, stellt im
Folgenden zahlreiche Initiativen vor und ordnet sie hinsichtlich ihrer
Bedeutung in der \enquote*{Open-Access-Bewegung} ein, die -- so
erläutert Steiner eingangs -- nicht \emph{eine} Bewegung ist, sondern
eigentlich eine Sammelbegriff für eine Vielzahl verschiedener Ansätze,
welche sich dem übergeordneten Ziel des freien Zugangs auf
unterschiedlichen Wegen und mit unterschiedlichen Motiven nähern.

Philipp Falkenburg erläutert in einem Praxisbericht das Engagement der
Vernetzungs- und Kompetenzstelle Open Access Brandenburg (VuK) im Rahmen
des Open Access Tracking Project: Das OATP ist ein Community-basiertes
Projekt zur Verschlagwortung von Webinhalten und ermöglicht also
kollaboratives Organisieren von online verfügbaren Informationen.
Philipp Falkenburg schildert, wie das \emph{social tagging} im
Arbeitsalltag der VuK verankert ist und ruft (nicht nur, aber auch die
Open-Access-Bewegung) zur Mitarbeit auf. Diese Anregung wurde auch in
die Redaktion der LIBREAS eingebracht und dort gerne aufgenommen.

Christian Erlinger und Jens Bemme befassen sich mit der
\enquote{Wikifizierung} publizistischer Arbeit am Beispiel eines
Citizen-Science-Projekts, in dem eine gedruckte, heimatkundliche
Buchpublikation ins Wikiversum übertragen wurde. Dieser
Open-Science-Ansatz umfasst unter anderem die offen lizenzierte
Bereitstellung von Bildmaterial, die Anreicherung mit strukturierter
Information und Verlinkungen und die Beschreibung und offene
Bereitstellung der Objekte in Wikidata. Der so entstandene Datensatz
kann nun weiterverwendet werden und jede Bearbeitung kann durch
\enquote{Edits} dokumentiert und auf den Diskussionsseiten besprochen
werden. Diese Form der offenen Wissensproduktion verknüpft die Bereiche
der Heimatforschung, des Denkmalschutzes und der Landeskunde mit
Open-Data-Werkzeugen des \emph{Wikiversums} und den dort angewendeten
Arbeitsweisen. Die Autoren schließen die Dokumentation mit einem Aufruf
an Regionalbibliotheken, in diesem Sinne \enquote{Open-GLAM-Labore} zu
unterstützen.

\hypertarget{weitere-beitruxe4ge}{%
\section{Weitere Beiträge}\label{weitere-beitruxe4ge}}

Außerhalb der Schwerpunktes thematisiert Karsten Schuldt in einem
bibliothekshistorischen Text die Arbeit des völkischen
\enquote{Grenzbüchereidienstes} in den Jahren der Weimarer Republik und
des Nationalsozialismus. Der Verein war eng mit der Entwicklung des
Volksbüchereiwesens in Deutschland verbunden, was angesichts seiner
politischen Ausrichtung erschreckend erscheint.

Ben Kaden und Linda Freyberg geben in ihrem Artikel einen Überblick über
Makerspaces und Library Labs in wissenschaftlichen Bibliotheken. Sie
ermöglichen die gemeinsame Nutzung digitaler Werkzeuge. Ansätze von
Makerspaces in öffentlichen Bibliotheken hatten dabei zunächst primär
das Ziel, die Kompetenzvermittlung über verschiedene Domänen hinweg zu
unterstützen. In Wissenschaftlichen Bibliotheken liegen die Schwerpunkte
von \enquote{Digital Makerspaces}, \enquote{Scholarly Makerspaces} oder
\enquote{Labs} in der digitalen forschenden Ausrichtung und in der
Auseinandersetzung mit digitalen und digitalisierten Beständen --
Ansätze, die eng mit den Digital Humanities verbunden werden. In ihrem
Artikel zeigen die beiden Autor*innen eine Reihe von Beispielen für
Digital Makerspaces in wissenschaftlichen Bibliotheken auf.

Seit letztem Jahr wird der Tätigkeitsbericht des LIBREAS-Vereins zur
Förderung der bibliotheks- und informationswissenschaftlichen
Kommunikation e.\,V., in der LIBREAS veröffentlicht. Für das Jahr
2022/23 erfolgt das mit dieser Ausgabe.

\hypertarget{zum-cover}{%
\section{Zum Cover}\label{zum-cover}}

Unsere Covertiere, alles Spiegelschafe, stehen normalerweise im kleinen
Zoo des Naherholungsgebietes im Parc de Sauvabelin, oberhalb von
Lausanne.\footnote{\url{https://www.prospecierara.ch/de/tiere/rassenportraets/schafportraets/spiegelschaf.html},
  \url{https://www.prospecierara.ch/de/erleben/karte-der-vielfalt/detailseite.html?tx_psrfeusers\%5BshowUid\%5D=1341}}
Der Zoo wird von der Stadt Lausanne in Zusammenarbeit mit der Stiftung
Pro Specia Rara (\url{https://www.prospecierara.ch/}) betrieben. Diese
Stiftung setzt sich für die Erhaltung der Biodiversität ein, indem sie
die Pflege alter Sorten von Obst, Gemüse und Kräutern sowie alter
Tierrassen fördert. Seit 2011 ist die Stiftung auch in Deutschland aktiv
(\url{https://www.prospecierara.de/}) und kann in beiden Ländern auch
direkt unterstützt werden.

\hypertarget{traurige-abschiede}{%
\section{Traurige Abschiede}\label{traurige-abschiede}}

Das Jahr neigt sich dem Ende, landläufig ist das ein Anlass für einen
Blick zurück und ein Resümee: Dieses Jahr lässt uns aufgrund von zwei
Todesfällen sehr bedrückt zurück.

Am Jahresanfang haben wir bestürzt vom unerwarteten Tod von Alessandro
Blasetti erfahren. Alessandro war seit 2016 Open-Access-Beauftragter des
Wissenschaftszentrums Berlin für Sozialforschung.\footnote{Siehe auch
  den Nachruf des WZB
  \url{https://wzb.eu/de/news/das-wzb-trauert-um-alessandro-blasetti}}
Er hat am WZB ausgeklügelte Workflows rund um Open Access aufgebaut und
präzise (im positiven Sinne) Listen geführt. Wissenschaftler*innen des
WZB wie auch Kolleg*innen aus der Open Access Community haben in
Alessandro einen kompetenten Ansprechpartner zu allen Aspekten des Open
Access gefunden, sein besonderes Steckenpferd war der \enquote{Grüne
Weg}. Alessandro war in verschiedenen Zusammenhängen mit Mitgliedern der
LIBRE\-AS-Redaktion kollegial, teils sogar freundschaftlich verbunden.
Nicht nur, aber auch bei den Open-Access-Tagen in Berlin war es
wortwörtlich unfassbar, dass Alessandro nicht dabei war. Wir vermissen
ihn schmerzlich -- seine Kompetenz und Expertise, seine Liebe zum
Detail, seine offene und herzliche Art.

Zum Jahresende waren wir ein weiteres Mal sehr betroffen, als wir vom
Tod von Jenny Delasalle erfahren haben. Jenny war zuletzt
Open-Access-Beauftragte der Charité und leitete das Open-Access-Team der
Medizinischen Bibliothek der Charité. Schon 2013 publizierte Jenny einen
Beitrag in der LIBREAS und berichtete über die Etablierung des
Forschungsdatenmanagements an der University of Warwick, wo sie
erfolgreich auch ein institutionelles Repositorium und einen der ersten
Open-Access-Publikationsfonds Großbritanniens verantwortet
hatte.\footnote{Jenny Delasalle, \enquote{Research Data Management at
  the University of Warwick: recent steps towards a joined-up approach
  at a UK university}. LIBREAS. Library Ideas, 23 (2013).
  \url{https://libreas.eu/ausgabe23/10delasalle/}} Besonders in der
Berliner Open Access Community hinterlässt Jenny eine große Lücke. Wir
werden sie als offenen, herzlichen und lustigen Menschen in Erinnerung
behalten, in der Zusammenarbeit zudem sehr engagiert für eine faire
offene Wissenschaft, professionell, lösungsorientiert und dabei stets
den „Finger in die Wunde legend".\footnote{Siehe auch den Beitrag mit
  Abschiedsworten von Jenny im Open Access Blog Berlin
  \url{https://blogs.fu-berlin.de/open-access-berlin/2023/12/12/abschied-von-jenny-delasalle-what-will-survive-of-us-is-love/}}

Vor diesem Hintergrund fällt es uns schwer, leichtherzig und wohlwollend
Abschied von 2023 zu nehmen. Und doch wollen wir es versuchen. Auch und
besonders im Andenken an Alessandro und Jenny. Wir sind dankbar, dass
wir mit zwei so wunderbaren Menschen zusammenarbeiten durften. Wir sind
dankbar, dass wir miteinander gelacht, voneinander gelernt und zusammen
Open Access ein Stückchen weiter vorangebracht haben. Wir sind dankbar,
dass uns gerade der Verlust daran erinnert, wie wichtig die Menschen in
unserem Netzwerk sind und dass wir zuallererst doch einfach Menschen
sind. Menschen mit Familien, Freund*innen, Interessen, sozialen und
politischen Aktivitäten und vielem mehr.

Passt auf Euch auf. Seid idealistisch. Bleibt offen.

Ihre / eure Redaktion LIBREAS. Library Ideas

(Berlin, Brandenburg an der Havel, Göttingen, Lausanne, München)

%autor

\end{document}
