\documentclass[a4paper,
fontsize=11pt,
%headings=small,
oneside,
numbers=noperiodatend,
parskip=half-,
bibliography=totoc,
final
]{scrartcl}

\usepackage[babel]{csquotes}
\usepackage{synttree}
\usepackage{graphicx}
\setkeys{Gin}{width=.4\textwidth} %default pics size

\graphicspath{{./plots/}}
\usepackage[ngerman]{babel}
\usepackage[T1]{fontenc}
%\usepackage{amsmath}
\usepackage[utf8x]{inputenc}
\usepackage [hyphens]{url}
\usepackage{booktabs} 
\usepackage[left=2.4cm,right=2.4cm,top=2.3cm,bottom=2cm,includeheadfoot]{geometry}
\usepackage{eurosym}
\usepackage{multirow}
\usepackage[ngerman]{varioref}
\setcapindent{1em}
\renewcommand{\labelitemi}{--}
\usepackage{paralist}
\usepackage{pdfpages}
\usepackage{lscape}
\usepackage{float}
\usepackage{acronym}
\usepackage{eurosym}
\usepackage{longtable,lscape}
\usepackage{mathpazo}
\usepackage[normalem]{ulem} %emphasize weiterhin kursiv
\usepackage[flushmargin,ragged]{footmisc} % left align footnote
\usepackage{ccicons} 
\setcapindent{0pt} % no indentation in captions

%%%% fancy LIBREAS URL color 
\usepackage{xcolor}
\definecolor{libreas}{RGB}{112,0,0}

\usepackage{listings}

\urlstyle{same}  % don't use monospace font for urls

\usepackage[fleqn]{amsmath}

%adjust fontsize for part

\usepackage{sectsty}
\partfont{\large}

%Das BibTeX-Zeichen mit \BibTeX setzen:
\def\symbol#1{\char #1\relax}
\def\bsl{{\tt\symbol{'134}}}
\def\BibTeX{{\rm B\kern-.05em{\sc i\kern-.025em b}\kern-.08em
    T\kern-.1667em\lower.7ex\hbox{E}\kern-.125emX}}

\usepackage{fancyhdr}
\fancyhf{}
\pagestyle{fancyplain}
\fancyhead[R]{\thepage}

% make sure bookmarks are created eventough sections are not numbered!
% uncommend if sections are numbered (bookmarks created by default)
\makeatletter
\renewcommand\@seccntformat[1]{}
\makeatother

% typo setup
\clubpenalty = 10000
\widowpenalty = 10000
\displaywidowpenalty = 10000

\usepackage{hyperxmp}
\usepackage[colorlinks, linkcolor=black,citecolor=black, urlcolor=libreas,
breaklinks= true,bookmarks=true,bookmarksopen=true]{hyperref}
\usepackage{breakurl}

%meta

\fancyhead[L]{N. Schmidt\\ %author
LIBREAS. Library Ideas, 40 (2021). % journal, issue, volume.
\href{https://doi.org/10.18452/23808}{\color{black}https://doi.org/10.18452/23808}
{}} % doi 
\fancyhead[R]{\thepage} %page number
\fancyfoot[L] {\ccLogo \ccAttribution\ \href{https://creativecommons.org/licenses/by/4.0/}{\color{black}Creative Commons BY 4.0}}  %licence
\fancyfoot[R] {ISSN: 1860-7950}

\title{\LARGE{Überlegungen für die Dekolonialisierung wissenschaftlicher Bibliotheken in Europa
}}% title
\author{Nora Schmidt} % author

\setcounter{page}{1}

\hypersetup{%
      pdftitle={Überlegungen für die Dekolonialisierung wissenschaftlicher Bibliotheken in Europa
},
      pdfauthor={Nora Schmidt},
      pdfcopyright={CC BY 4.0 International},
      pdfsubject={LIBREAS. Library Ideas, 40 (2021).},
      pdfkeywords={Wissenschaftliche Kommunikation, Wissenschaftliche Bibliotheken, Dekolonialisierung, Szientometrie, Critlib, Bestandsmanagement, Scholarly communication, Academic libraries, Decolonialism, Scientometrics, Critlib, Collection management, Decolonization},
      pdflicenseurl={https://creativecommons.org/licenses/by/4.0/},
      pdfcontacturl={http://libreas.eu},
      baseurl={http://libreas.eu},
      pdflang={de},
      pdfmetalang={de}
     }


\date{}
\begin{document}

\maketitle
\thispagestyle{fancyplain} 

Anmerkung der LIBREAS-Redaktion: Bei diesem Artikel handelt es sich um eine Zweitveröffentlichung. Die Erstveröffentlichung erfolgte in: Young Information Scientist Vol. 6(2021), \url{https://doi.org/10.25365/yis-2021-6-1}.

%abstracts
\begin{abstract}
\noindent
\textbf{Kurzfassung}: \textit{Zielsetzung} -- Der Artikel fasst die Argumentation
meiner Dissertation zusammen. Darin wird untersucht, wie die Arbeit mit
wissenschaftlichen Publikationen in Bibliotheken und bei
Informationsdienstleistern daran mitwirkt, Beiträge zum
Wissenschaftssystem aus dem \enquote{Globalen Norden} zuprivilegieren. Die
dadurch reproduzierte soziale Ungerechtigkeit kann als \enquote{Kolonialität}
bezeichnet werden. Schließlich stellt der Artikel Optionen vor, wie
Bibliotheken ihre eigene Dekolonialisierung in die Wege leiten können.

\textit{Forschungsmethoden} -- Die Zusammenfassung ist rein argumentativ und
bezieht sich nur implizitauf die vielfältigen empirischen Studien der
Dissertation.

\textit{Ergebnisse} -- Die Durchlässigkeit für wissenschaftliche
Kommunikationsmedien \linebreak aus dem \enquote{Globalen Süden} wird als zentrale Weiche
für Kolonialität erkannt, insbesondere die Abtrennung der
Regionalwissenschaften von den Kerndisziplinen, die Inklusionskriterien
von bibliographischen Datenbanken und Paketprodukten von
Informationsdienstleistern, Szientometrie und die damit verbundene
Quantifizierung von wissenschaftlicher Kommunikation, die
Kommerzialisierung von Open Access und die Orientierung von
Bestandsentwicklung am \enquote{Bedarf}.

\textit{Schlussfolgerungen} -- Neutralität als einer der Kerne bibliothekarischer
Berufsethik sollte in Europa als kulturell demütige Neutralität
rekonzeptualisiert werden, um Kolonialität entgegen zu treten, die
eigene privilegierte Position zu reflektieren und einen erhöhten Aufwand
zu akzeptieren. Dies führt zwar zu mehr Komplexität im
Wissenschaftssystem, bietet aber die Chance auf produktive Irritation
und Anreize für Kooperation.

\begin{center}\rule{0.5\linewidth}{0.5pt}\end{center}

\textbf{Abstract}: \textit{Objective} -- The article summarises the argumentation
of my PhD thesis that investigates how the work with scholarly
publications in libraries and by information service providers
privileges contributions to the research system from the \enquote{Global
North}. The thereby reproduced social injustice can be termed
\enquote{coloniality}. The article presents options for libraries to initiate
their own decolonization.

\textit{Methods} -- The summary is solely argumentative and refers to the diverse
empirical studies of the PhD thesis only implicitly.

\textit{Results} -- The permeability for scholarly communication media from the
\enquote{Global South} is recognised as an important passage for coloniality,
in particular the separation of area studies from the core disciplines,
the inclusion criteria of bibliographic databases and package products
from information service providers, scientometrics and the associated
quantification of scholarly communication, the commercialization of Open
Access and the focus of collection development on \enquote{demand}.

\textit{Conclusions} -- Neutrality as one of the core values of the librarian`s
professional ethics should be re-conceptualized in Europe as culturally
humble neutrality, in order to counter coloniality, reflect on one's own
privileged position and accept increased effort. Although this leads to
more complexity in the research system, it offers the opportunity for
productive irritation and incentives for cooperation.
\end{abstract}

%body
Diesem Beitrag liegt folgende Dissertation zugrunde: Schmidt, Nora
(2020). \enquote{The Privilege to Select. Global Research System,
European Academic Library Collections, and Decolonisation}. Phd Thesis.
Lund: Lund University, Faculties of Humanities and Theology, Lund
Studies in Arts and Cultural Sciences, 2. Sep. 2020. DOI:
\href{https://doi.org/10.5281/zenodo.4011295}{10.5281/zenodo.4011295}

\hypertarget{vorwort}{%
\section{1 Vorwort}\label{vorwort}}

Dieser Artikel fasst die Ergebnisse meines Dissertationsprojekts
zusammen, und ähnelt stark der Zusammenfassung in schwedischer Sprache,
die in der Dissertation enthalten ist (siehe Schmidt 2020). Das
empirische Fundament des Projekts besteht aus vielen einzelnen Studien,
die selbst auf Forschungsdaten aus verschiedenen Quellen basieren.
Anstatt diese Studien zusammenzufassen, werde ich im Folgenden die
Hauptargumentationslinie der Dissertation darstellen. Der Umfang einer
Zusammenfassung würde es lediglich erlauben, einige ausgewählte
Autorïnnen und Werke aus der sehr breiten und vielfältigen
Literaturbasis der Arbeit herauszustellen. Um diese Privilegierung zu
vermeiden, die von der tatsächlichen Erarbeitung der Argumentation eine
verzerrte Vorstellung vermitteln würde, werde ich hier nur implizit auf
diese Literaturbasis verweisen. Diese Entscheidung folgt der Erkenntnis,
dass alles Wissen sozial ist. Viele für meine Arbeit relevante Themen
wurden in verschiedenen Kontexten bereits diskutiert. Meine Kenntnis der
in diese Wissensproduktion eingeflossenen Beiträge ist ohnehin notwendig
unvollständig. Die Argumentation in der vorliegenden Form ist jedoch
meine eigene, und ihre Herleitung mit genauer Referenzierung findet sich
in gleicher Abfolge in meiner Dissertation. Durch das Verfassen dieser
Zusammenfassung habe ich die Möglichkeit, zumindest für diesen Moment
von meinem Privileg zurückzutreten, einzelne Beiträge anderer
herauszuheben.

\hypertarget{forschungsproblem-ziele-und-aufbau-der-dissertation-und-dieser-zusammenfassung}{%
\section{2 Forschungsproblem, Ziele und Aufbau der Dissertation und
dieser
Zusammenfassung}\label{forschungsproblem-ziele-und-aufbau-der-dissertation-und-dieser-zusammenfassung}}

Unter mehr oder weniger großem Aufwand ist es heute möglich, weltweit
für Kommunikation erreichbar zu sein. Dieser weltgesellschaftliche
Zustand ist jedoch enorm durch soziale Ungleichheit und Ungerechtigkeit
geprägt, was sich auch in sozialgeographischen Verhältnissen spiegelt,
und auf historischem Kolonialismus und seinen Folgen, der Kolonialität
(\emph{coloniality}), beruht. Insoweit unterprivilegierte Weltregionen
werden häufig als \enquote{Globaler Süden} bezeichnet. Auch im
Wissenschaftssystem ist diese Unterscheidung präsent (siehe Abschnitt
\protect\hyperlink{weltges}{3}). Ziel der Dissertation ist es, sichtbar
zu machen, wie die Arbeit mit wissenschaftlichen Informationen in
Bibliotheken und bei Informationsdienstleistern, deren Dienste
Bibliotheken mit zumeist öffentlichen Geldern zukaufen, dazu beiträgt,
Kolonialität fortzusetzen.

Wissenschaftliche Kommunikation wird oft als eine Art von Kommunikation
konzipiert, deren einzelne Beiträge über Referenzen global eng
miteinander verbunden sind. Unter anderem spielen sowohl die aktuelle
Relevanz für identifizierbare Zielgruppen, als auch der
Veröffentlichungsort eine wichtige Rolle dafür, welche Referenzen
angegeben werden, und zwar in beide Richtungen: Autorïnnen wählen
Literatur, die sie entdecken konnten und die sie für relevant halten;
aber sie nehmen auch Rücksicht darauf, was gewisse Zielgruppen sowie die
Reviewerïnnen und die Herausgeberïnnen ihrer Manuskripte für relevant
halten mögen. Wie die szientometrische Studie in der Dissertation für
eine Stichprobe von südostafrikanischer Literatur der Geistes- und
Sozialwissenschaften zeigt, ist im \enquote{Globalen Norden}
veröffentlichte Literatur leichter zu entdecken als im \enquote{Globalen
Süden} veröffentlichte Literatur; sie erscheint schon allein deswegen --
völlig unabhängig vom Inhalt -- als relevanter, und ist daher in diesem
Prozess privilegiert.

Die Dissertation analysiert sowohl konzeptionell als auch empirisch, wie
sich das Wissenschaftssystem als global konstruiert, und dabei
Unterscheidungen zwischen Zentrum und Peripherie einzieht (siehe
Abschnitt \protect\hyperlink{peripherie}{4}). Sie zeigt, inwiefern eine
Stichprobe südostafrikanischer Publikationen zur sozial- und
geisteswissenschaftlichen Grundlagenforschung in die globale
wissenschaftliche Kommunikation integriert ist (siehe Abschnitt
\protect\hyperlink{sziento}{5}). Publikationsorte geben Hinweise zur
Auffindbarkeit innerhalb der wissenschaftlichen Informationssysteme, die
vor allem im \enquote{Globalen Norden} den meisten Forschenden in mehr
oder weniger größerer Breite -- abhängig vom Budget der Institutionen,
die ihnen zugänglich sind -- zur Verfügung stehen. Darüber hinaus werden
die Auswirkungen einiger aktuellen globalen Entwicklungen auf die
Infrastrukturen des südostafrikanischen Publikationswesens erörtert. Da
für diese Zwecke keine ausreichend inklusive bibliografische Datenquelle
verfügbar ist, muss sie erstellt werden. Ein zusätzliches Ziel dieser
Arbeit ist es daher, einige blinde Flecken der Szientometrie zu
reflektieren.

Am Beispiel wissenschaftlicher Bibliotheken in Europa wird schließlich
untersucht, wie die angedeuteten problematischen sozialen Strukturen mit
der bibliothekarischen Berufsethik, insbesondere mit dem Prinzip der
Neutralität, und mit verbreiteten Vorgängen bei der Bestandsentwicklung
und -pflege in wissenschaftlichen Bibliotheken zusammenhängen (siehe
Abschnitt \protect\hyperlink{biblio}{6}). Abschließend stelle ich in
Abschnitt \protect\hyperlink{liste}{7} eine Liste von Vorschlägen
bereit, die als Grundlage für eine professionelle Debatte über die
Dekolonialisierung einer wissenschaftlichen Bibliothek dienen kann.

\hypertarget{weltges}{%
\section{3 Das Wissenschaftssystem in der
Weltgesellschaft}\label{weltges}}

Das Wissenschaftssystem erfüllt mit seiner hochspezialisierten
Kommunikation eine bestimmte Funktion in der Weltgesellschaft, nämlich
ihre Versorgung mit neuem und verlässlichem wissenschaftlichen Wissen.
Ich gehe davon aus, dass Forschende, insbesondere in den Sozial- und
Geisteswissenschaften, dabei die Interessen der lokalen Bevölkerung am
Forschungsstandort zumindest bis zu einem gewissen Grad vertreten, und
dass sich die Interessen der Bevölkerung im \enquote{Globalen Süden}
mehr oder weniger von jenen der Bevölkerung im \enquote{Globalen Norden}
unterscheiden, weil sich die kulturellen und sozialen Bedingungen
unterscheiden. Nachdem die große Mehrheit der Weltbevölkerung im
\enquote{Globalen Süden} lebt, leicht verfügbare Forschungsergebnisse
jedoch überwiegend an Forschungsstandorten im \enquote{Globalen Norden}
erzeugt werden, kann also angenommen werden, dass das Forschungssystem
seine Funktion im Hinblick auf die Mehrheit der Interessen nur äußerst
ungenügend erfüllt.

Durch die Kolonialisierung und die damit verbundene gewaltsame Ersetzung
von Bildungsinstitutionen verhinderte der \enquote{Globale Norden} das
Gedeihen lokaler Wissensproduktions- und Rezeptionssysteme im
\enquote{Globalen Süden}. Die Kolonialisierung verdrängte diese lokalen
Systeme zugunsten des seinerzeit entstehenden Wissenschaftssystems des
\enquote{Globalen Nordens}, das mittlerweile in praktisch allen Ländern
des \enquote{Globalen Südens} institutionalisiert wurde. Diese
Strukturüberlagerung, die bislang im \enquote{Globalen Norden} kaum
hinterfragt wurde, ist ein Beispiel für Kolonialität.

Die Ungleichheit der Teilnahme an der Kommunikation in den Sozial- und
Geisteswissenschaften folgt den Gräben sozialer Ungerechtigkeit, die in
der Weltgesellschaft insgesamt allgegenwärtig sind. Nach der Befreiung
vom politischen Kolonialismus änderte sich die Perspektive der
Forschenden aus dem \enquote{Globaler Norden} auf den \enquote{Süden}
kaum. Verfehlte Entwicklungspolitik (\emph{developmentalism}) und der
Kalte Krieg stabilisierten eine ausgrenzende Sichtweise, die durch die
Klassifizierung der Wissenschaft in \enquote{klassische} Disziplinen
einerseits, und Regionalwissenschaften andererseits, weiter gestärkt
wurde. Da diese beiden Wissenskorpora mittels physischer und
informativer Distanzierung, beispielsweise in Bibliotheken, voneinander
isoliert werden, bezeichne ich dieses Problem als \enquote{Abtrennung
der Regionalwissenschaften} (\emph{area studies incarceration}). So
werden beispielsweise Beiträge von auf dem afrikanischen Kontinent
ansässigen Philosophïnnen häufig den Afrikawissenschaften zugeordnet.

Auch die verbreiteten Methoden, mit denen sich das Wissenschaftssystem
selbst beobachtet -- evaluiert, tragen zur Fortsetzung der Kolonialität
bei. Bestimmte wissenschaftliche Publikationen werden eher nicht
entdeckt und weniger referenziert als andere, weil sie nicht in den im
\enquote{Globalen Norden} erstellten Zitationsindizes verzeichnet sind.
Der bedeutendste dieser Indizes, \emph{Web of Science} (\textsc{Wos}),
war zunächst in erster Linie als Werkzeug zum Auffinden von
wissenschaftlichen Informationen gedacht, um die Schneeballmethode mit
quantitativen Daten anzureichern: Forschende hangeln sich bei dieser
Methode von Referenz zu Referenz durch die Forschungsliteratur, um die
Entwicklung einer Debatte nachvollziehen zu können. Quantitative Daten
zur Häufigkeit einzelner Referenzen helfen, Schlüsselbeiträge und die
Zeitschriften, in denen sie gewöhnlich erscheinen, schnell zu
identifizieren. Forschungsfördereinrichtungen verwendeten
Zitationsindizes jedoch bald als ihr Hauptwerkzeug, um das
Wissenschaftssystem zu beobachten, und die Mainstream-Bibliometrie
etablierte sich auf dieser Grundlage.

Die Inklusionskapazität eines jeden Indizes ist allerdings technisch
begrenzt. Der Aufwand bei der Inklusion von im \enquote{Globalen Süden}
erscheinenden Zeitschriften ist allein schon dadurch besonders hoch,
weil sie häufig im \enquote{Globalen Norden} kaum zu beschaffen sind
oder, im Falle von elektronischen Publikationen, den im
\enquote{Globalen Norden} gesetzten technischen Standards mitunter nicht
folgen. Dadurch wird die Ausgrenzung dieser Publikationen weiter
verstärkt. Selbst wenn Datenbanken aus dem \enquote{Globalen Süden}
z. B. von \textsc{Wos} inkludiert werden, so geschieht das als
\emph{Regional Citation Indexes}, womit sie den Status des eingegrenzten
Ausgegrenzten erhalten. Zudem machen die kommerziellen Unternehmen, die
Zitationsindizes betreiben, ihre Inklusionsentscheidungen kaum
transparent. Die veröffentlichten Inklusionskriterien von \textsc{Wos}
legen den Ausschluss von Beiträgen aus dem \enquote{Globalen Süden} aus
formalen, nicht inhaltlichen Gründen nahe. Die Mainstream-Bibliometrie
und die Inklusionskriterien der Indizes, auf die sie sich stützt, tragen
zur Verfestigung der Zentrum-Peripherie-Struktur im Wissenschaftssystem
und seiner Kopplung an sozialgeographische Strukturen bei (siehe
Abschnitt \protect\hyperlink{peripherie}{4}).

Wissenschaftliche Bibliotheken befinden sich als Vermittlerinnen
zwischen kommerziellen Anbietern von wissenschaftlichen Informationen
und Forschenden in einer problematischen Position. Auswahlprozesse bei
der Bestandsentwicklung werden zunehmend ausgelagert (teilweise an
unbezahlte Benutzerïnnen). Statt den Benutzerïnnen unmittelbar
Informationsressourcen zugänglich zu machen, fungieren Bibliothekarïnnen
zunehmend als Helpdesk, Überwachungs- und Marketingdienst für die
Container-Produkte der kommerziellen Anbieter. Die Bibliothek als
Organisation des öffentlichen Dienstes unter neoliberalen Vorzeichen
kann sich privatwirtschaftlichen Interessen schwer entziehen. Die
Handlungsfähigkeit der Bibliotheken in der Wissensvermittlung und die
Durchlässigkeit für Publikationen aus dem \enquote{Globalen Süden} wird
dezimiert.

Die Umstellung des Publikationswesens auf Open Access und die damit
verbundene Verbesserung der Auffindbarkeit von so publizierten
Forschungsergebnissen klingt vielversprechend, aber die Finanzierung
dieser Open-Access-Publikationsservices ist entscheidend: Wenn von
Autorïnnen oder ihren Institutionen für die Veröffentlichung jene
Gebühren an die Verlage gezahlt werden, die sie fordern, verlieren
Bibliotheken das einzige kraftvolle Instrument ihrer Handlungsfähigkeit
gegenüber den kommerziellen Verlagen: ihre Macht als Großkunden in den
Preisverhandlungen.

Während diese Macht der Bibliotheken im \enquote{Globalen Süden} noch
nie stark ausgeprägt war, erlangen sie auch mit der Konstitution von
Open Access als neuem Standard keine vorteilhaftere Position: Die
Bereitstellung von oder Mitwirkung an global verfügbaren
nicht-kommerziellen Open-Access-Publikationsservices ist mit Kosten
verbunden, die häufig die Teilnahme des \enquote{Globalen Südens}
ausschließen, da hierfür eine aktuelle \textsc{It}-Infrastruktur und
erhebliche Personalressourcen erforderlich sind. Die Entwicklung dieser
global verfügbaren Infrastrukturen findet also weiterhin in
beträchtlichem Ausmaß im \enquote{Globalen Norden} statt, vor dem
Hintergrund der entsprechenden lokalen Interessen, während bestehende
lokale Infrastrukturen im \enquote{Globalen Süden} dagegen häufig aus
der Zeit fallen und eingehen.

In der Verlags- und Informationsbranche des \enquote{Globalen Nordens}
decken zunehmend die Produkte einzelner Unternehmen den gesamten
Forschungsworkflow ab. Forschende bekommen geringe Anreize, sich z. B.
durch nicht-kommerzielle Angebote ihrer Institutionen oder ihrer
Fachcommunities unterstützen zu lassen oder sich selbst für die
Weiterentwicklung von nicht-kommerziellen Infrastrukturen einzusetzen.
Der One-Stop-Shop als Software-Ökosystem verspricht den reibungslosen
Zugriff auf alle während der Forschung erzeugten Daten, von der Lektüre
bis zum Sammeln von Feedback zur Publikation. Jede zwischenzeitlich
verwendete Software soll wie aus einem Guss wirken, ohne spürbare
Einstiegshürden und Lernkurven. Das ist nur erreichbar, wenn den
Forschenden Entscheidungen abgenommen werden, denn die Vielfältigkeit an
Dokumenten, die während des Forschungsprozesses konsultiert und erstellt
werden, erfordert ebenso vielfältige Überlegungen zu Fragen der
Darstellung, der Verknüpfung mit anderen Dokumenten, der Zugänglichkeit
für unterschiedliche Öffentlichkeiten etc. Wenn Forschende sich im
One-Stop-Shop unhinterfragt einrichten, vermeiden sie die Konfrontation
mit einer Welt, die nicht durch ein kommerzielles Unternehmen gefiltert
und geordnet erscheint. Nicht nur die Unternehmen selbst, sondern auch
die Forschenden, die ihre Produkte routiniert einsetzen, schirmen sich
immer mehr von Irritationen ab, die von außen kommen und die Entwicklung
von individuellen oder in einer Bezugsgruppe diskutierten
Bewältigungsstrategien erfordern. Was nicht auf dem Radar der allesamt
im \enquote{Globalen Norden} ansässigen
Informationsdienstleistungsunternehmen erscheint, wie eben ein Großteil
der Publikationen aus dem \enquote{Globalen Süden}, entzieht sich dann
mit hoher Wahrscheinlichkeit der Aufmerksamkeit dieser Forschenden --
und in der Folge auch der Aufmerksamkeit der Studierenden, die von ihnen
unterrichtet werden.

Bibliotheken im \enquote{Globalen Norden} unterstützen die Unternehmen
bei der Durchsetzung der One-Stop-Shop-Strategie, in der jede neue
Funktionalität als Begründung einer Preissteigerung fungieren kann. Die
Angebote der einzelnen Unternehmen basieren nach wie vor auf den
exklusiven wissenschaftlichen Informationen in ihrem Portfolio und sind
daher untereinander selten austauschbar, auch wenn es Überschneidungen
gibt. Die Unternehmen operieren also in einer Umgebung, die sich äußerst
vorhersehbar verhält, und in der es im Wettbewerb keine Verlierer geben
kann. Die Unternehmen müssen lediglich sicherstellen, dass sich entweder
eine ausreichende Anzahl von nachgefragten Publikationen im Portfolio
befindet, über deren Verwertungsrechte das Unternehmen allein verfügt,
oder dass \emph{tier one journals} mit einem auf horrenden
Publikationsgebühren basierenden Open-Access-Geschäftsmodell
Einreichungen garantieren, weil die quantifizierte
Wissenschaftskommunikation die Publikation in solchen Zeitschriften als
Erwartung an Forschungskarrieren setzt. Die Nutzungsdaten, die diese
Unternehmen sammeln, sind äußerst wertvoll, da sie einer interessanten
Bevölkerungsgruppe zuzuordnen sind: der globalen Bildungselite. Durch
die Bereitstellung von aus öffentlichen Mitteln bezahlten Zugängen zu
den Plattformen, auf denen es von Trackern nur so wimmelt, ermöglichen
Bibliotheken, dass diese Unternehmen doppelt profitieren.

Auch wenn im Bibliothekswesen wenig Bewusstsein für diese Problematik
herrscht, so hat doch der Begriff \enquote{Diversität} im Bibliotheks-
und auch im Wissenschaftskontext zunehmende Präsenz erlangt, was auf
eine wachsende Sorge um soziale Gerechtigkeit hinweist. Sogenanntes
\enquote{Diversity Management} könnte jedoch Kolonialität sogar
stabilisieren, wenn es dabei in erster Linie um symbolische Aktivitäten
geht, die der benannten Sorge Ausdruck verleihen, und sich bemühen, dem
Ausgeschlossenen Raum zu geben, aber an den grundlegenden, oben
skizzierten Zusammenhängen nicht zu rütteln vermögen. Ich schlage daher
vor, sich aus einer privilegierten Position heraus stattdessen auf das
Konzept der kulturellen Demut (\emph{cultural humility}) zu
konzentrieren. Es verschiebt den Fokus vom \enquote{Anderen} auf die
Selbstbeobachtung und Selbstkritik der Privilegierten.

Schließlich spielt auch die gesellschaftsweite Präferenz für
quantifizierte Kommunikation eine große Rolle für die Kolonialität im
Publikationswesen. Forschende stehen unter zunehmendem Druck, in großen
Mengen zu publizieren, und müssen dabei notwendig die Qualität
vernachlässigen. Diese Entwicklung hat die Entstehung von
\enquote{räuberischem Publizieren} (\emph{predatory publishing})
ausgelöst. In der öffentlichen Debatte wurden Verlage, die mit
Falschaussagen zu ihren Qualitätssicherungsprozessen und kurzen
Publikationszeiten gegen vergleichsweise geringe Gebühren locken, häufig
mit dem \enquote{Globalen Süden} in Verbindung gebracht: einerseits,
weil als \enquote{räuberisch} identifizierte Verlage häufig dort ihren
Sitz haben, und andererseits, weil ihre Autorïnnen nicht selten mit
Forschungseinrichtungen im \enquote{Globalen Süden} affiliiert sind, was
eine verallgemeinernde Unterstellung mangelnder Kompetenz begünstigt.
Jedenfalls wurden in der Debatte auch Verlage als \enquote{räuberisch}
bezeichnet, die nachweislich mit den besten Absichten operieren. Verlage
im \enquote{Globalen Süden} trifft eine solche Beschuldigung aufgrund
ihrer unterprivilegierten Ausgangsposition besonders hart. Darüber
hinaus motivierte die Debatte Forschungseinrichtungen, Listen
referierter Zeitschriften einzuführen, in denen Forschende publizieren
müssen, wenn ihre Publikationen in Auswahlprozessen \enquote{zählen}
sollen. Vergleichsweise schwer zugängliche Zeitschriften, die im
\enquote{Globalen Süden} erscheinen, halten kaum Einzug in diese Listen
und verlieren somit Autorïnnen.

Selbstverständlich muss das Wissenschaftssystem Wege finden, aus dem
überwältigend hohen Volumen früherer wissenschaftlicher Kommunikation
irgendwie jenes zu exponieren, was potentiell für die aktuelle
Kommunikation relevant ist. Die Komplexität muss reduziert werden, und
technische Systeme, die bei dieser Aufgabe helfen, wie z. B.
Relevanzrankings von Suchmaschinen für Bibliotheken (\emph{discovery
systems}), sind willkommen, da sie einen geringen Eingabeaufwand
erfordern und unmittelbar brauchbare Ergebnisse liefern. Wenn sich
Forschende aber auf diese Systeme verlassen, ohne ihre Funktionsweise in
Frage zu stellen und ohne für ihre Literaturrecherchen reflexive
Methoden einzusetzen, wirken sich die internen Strukturen dieser
technischen Systeme natürlich auch unmittelbar auf die wissenschaftliche
Kommunikation aus. Nur wenn eine übermäßige Vereinfachung erkannt, mehr
Durchlässigkeit und Komplexität zugelassen wird, kann Hoffnung auf mehr
soziale Gerechtigkeit und kulturelle Demut im Wissenschaftssystem
bestehen. Die Geschichte der Etablierung des globalen
Wissenschaftssystems nimmt die wissenschaftliche Kommunikation selbst in
die Verantwortung, über Privilegien zu reflektieren und aktiv von ihnen
zurückzutreten. Diese Verantwortung erstreckt sich auch auf alle im
System involvierten Institutionen und Personen, einschließlich der
wissenschaftlichen Bibliotheken und der Forschenden.

\hypertarget{peripherie}{%
\section{4 Zentrum und Peripherie in der wissenschaftlichen
Kommunikation}\label{peripherie}}

Bei der Untersuchung von Ursprung und Entwicklung des dualen Begriffs
von Zentrum/Peripherie fällt auf, dass insbesondere die neueren
Konzeptualisierungen eine räumliche, häufig sogar geografische Bedeutung
tragen, die sich auf den \enquote{Globalen Norden/Süden} bezieht. Diese
Begriffsbildung hat eine Reihe von Nachteilen, allen voran, dass sie die
Akkumulation von referenziell produzierten Peripherien im
\enquote{Globalen Süden} und von Zentren im \enquote{Globalen Norden}
stabilisiert. Auf diese Weise werden Privilegien (re)produziert, und
damit auch soziale Ungerechtigkeit. Daher sollte der Raumbezug durch
einen Bezug zur inneren Differenzierung sozialer Systeme ersetzt werden.

Wissenschaftliche Debatten bilden notwendigerweise Zentren aus, auch
ohne das Zutun von in technischen Systemen verankerten
Relevanzkriterien. Die Kommunikation dockt vermehrt an bestimmte
Forschungsergebnisse an, was zufällig oder wissenschaftlich begründbar
geschieht. Diese Forschungsergebnisse verknüpfen sich dann über Thema,
Theorie oder Methode mit anderen, bis eine Verdichtung entsteht, auf die
außerordentlich häufig Bezug genommen wird. Was für eine solche
Kommunikationsverdichtung peripher ist, kann für eine andere jedoch
zentral sein. Die Konstellation zeichnet sich durch eine hohe
potentielle Dynamik aus, auch wenn bestimmte Adressen -- Forschende,
Forschungsstätten, theoretische oder methodologische Schulen --
Geschichte machen, also Zentralität akkumulieren. Jedoch kann es keine
Zentren ohne Peripherie geben: Sie sind auf die Peripherie angewiesen.
Das ist allerdings nicht ihr einziger Daseinszweck, denn wo sich
periphere Kommunikation akkumuliert, ist auch ein Bereich, in dem
erkenntnistheoretische Risiken eingegangen werden können. Das wiederum
erzeugt ein großes Innovationspotential.

Die räumliche Unterscheidung von Zentrum/Peripherie spiegelt sich
semantisch in der Unterscheidung von \enquote{internationalen/lokalen
Zeitschriften} wider. Die Definitionen dieses Begriffspaars sind
variantenreich und oft auch nur impliziert. \enquote{Internationale
Zeitschriften} werden im Allgemeinen mit hohem Prestige und globaler
Sichtbarkeit verknüpft, während \enquote{lokale Zeitschriften} eher nur
für das lokale Publikum relevant oder gar von fragwürdiger Qualität sein
sollen. Ein global agierendes Informationsdienstleistungsunternehmen
überträgt leicht den immer positiv konnotierten Status der
Internationalität auf die Zeitschriften, die es verlegt. Was diese
Definitionen nicht berücksichtigen, ist das Potential \enquote{lokaler
Zeitschriften}, unabhängig von großen Verlagen zu arbeiten und
beispielsweise Autorïnnenrichtlinien zu entwerfen, die von den im
\enquote{Globalen Norden} etablierten Standards abweichen. Hier wird ein
Spielraum für produktive Irritationen in der wissenschaftlichen
Kommunikation eröffnet, die über technische Innovationen hinausgehen.
Aufgrund der geltenden Paradigmen von Evaluationsprozessen in
Forschungsmanagement und Bibliotheken sind diese Kapazitäten im Abbau.
Zeitschriften, die solche Zugangsmöglichkeiten für Entwicklungsimpulse
bieten, verschwinden bereits in großer Zahl.

Die zuvor angedeutete problematische Konstellation bezüglich der
Entwicklung des Open- \linebreak Access-Publizierens taucht hier erneut auf: Die
meisten Institutionen im \enquote{Globalen Süden} können es sich kaum
leisten, die lokalen Infrastrukturen, zu denen auch \enquote{lokale
Zeitschriften} gehören, zuverlässig zu betreiben. Selbst wenn technische
Infrastrukturen bereitstehen, so braucht es freigestelltes Personal, um
sie mit Inhalten zu befüllen. Ein wahrscheinliches Szenario ist
vielmehr, dass Forschungsergebnisse weiterhin in die engen Korsetts von
\enquote{internationalen} Publikationsorten eingepasst werden, um die
Generalisierungsbarriere zu überwinden. Eine Generalisierungsbarriere
besteht immer dann, wenn Forschung ihren Kontext im \enquote{Globalen
Süden} explizit macht und in der Folge ihre globale Relevanz
angezweifelt wird. Für Forschung aus dem \enquote{Globalen Norden}
besteht kaum eine Generalisierungsbarriere, weil gar nicht erwartet
wird, dass ihr Kontext über ein Mindestmaß hinaus explizit gemacht wird.
Das \enquote{Andere}, das Periphere, wird nicht im \enquote{Norden},
sondern im \enquote{Süden} lokalisiert. Im Hinblick auf Prestige und
Karriereentwicklung ist es also von Nachteil, sich im \enquote{Globalen
Süden} mit Forschungsproblemen aus dem lokalen Kontext zu befassen, es
sei denn, damit lässt sich unmittelbar ein Einkommen sichern, wie bei
der weit verbreiteten Auftragsforschung. Diese ist jedoch im globalen
Wissenschaftssystem kaum sichtbar und leidet unter mangelnder
Qualitätssicherung ebenso wie darunter, dass dauerhafte Zugänglichkeit
und Archivierung häufig nicht gewährleistet sind.

Szientometrie und Bibliometrie unterstützen die Reproduktion einer
räumlichen Unterscheidung zwischen Zentrum und Peripherie, wenn sie
Länder oder Institutionen als Zentren der wissenschaftlichen
Kommunikation identifizieren, und zwar quantitativ, auf Grundlage
fragwürdiger Daten. Bibliografische Datenbanken und jene Studien, die
auf diesen Datenbanken basieren, beobachten wissenschaftliche
Kommunikation lediglich in dritter und vierter Ordnung, suggerieren
jedoch durch ihre wissenschaftlich akzeptierten Methoden eine weitaus
größere Unmittelbarkeit. Die verbreitete Überzeugung, dass
beispielsweise ein Gelistetsein in \textsc{Wos} hohe Qualität oder gar
Exzellenz in der Forschung anzeigt, beeinflusst eine Reihe von
Entscheidungen im Kontext von Forschungsförderung,
Karriereentwicklungen, Platzierung in der Wissenschaftskommunikation und
in Rechercheinstrumenten, die direkt darauf wirken, wer wann was
beforscht und wie die Ergebnisse wahrgenommen werden. Wie jedoch oben
bereits erläutert, ist das Gelistetsein nicht unwesentlich auf eher
technisch-pragmatische und generell intransparente Entscheidungen von
Informationsdienstleistungsunternehmen zurückzuführen, die gewöhnlich
keinerlei Anzeichen von kultureller Demut erkennen lassen.

Auch wenn das Wissenschaftssystem nicht ohne Komplexitätsreduktion
auskommt, so scheint es dennoch an der Zeit, es mit einem höheren Grad
an Komplexität zu belasten, um die Kolonialität zu überwinden, die das
System durch akkumulierte räumliche Peripherien im \enquote{Globalen
Süden} reproduziert. Kolonialität löst alles, was als in dieser Hinsicht
peripher erkannt wird, vom Kern der Relevanz und Exzellenz ab, und dient
so der Verringerung der Komplexität im Wissenschaftssystem. Um soziale
Gerechtigkeit zu erreichen, werden alternative Mittel zur
Komplexitätsreduktion benötigt. Anstelle eines Programms, welches das
System reparieren soll, müssen zunächst die gegenwärtigen Mittel zur
Komplexitätsreduktion einschließlich seiner fragwürdigen Kriterien in
ihrer Wirkmächtigkeit ausgesetzt und additive Ansätze wie
\enquote{Diversity Management} vermieden werden. Um die Entstehung von
neuen Strukturen anzuregen, die einem globalen System und sozialer
Gerechtigkeit angemessen sind, braucht das System zunächst einmal mehr
Komplexität.

\hypertarget{sziento}{%
\section{5 Dekoloniale Szientometrie}\label{sziento}}

Um Szientometrie mit dekolonialer Sensibilität und kultureller Demut zu
versehen, sind Kenntnisse über den spezifischen Kontext erforderlich, in
dem wissenschaftliche Kommunikation stattfindet. Darüber hinaus kann sie
sich nicht auf eine bestimmte Datenquelle beschränken, wenn ihre
Erkenntnisse über die Beschreibung dieser Datenquelle hinausgehen
sollen. Quellen müssen auf ihre Inklusionskriterien und -beschränkungen
explizit untersucht werden. Da das Kombinieren verschiedener Quellen
viele manuelle Schritte erfordert, sind Stichproben dann -- abhängig von
den Kapazitäten des szientometrischen Projekts -- notwendigerweise
klein. Eine Datenerfassung, die auf individuellen Publikationslisten und
institutionellen Forschungsinformationen basiert, kann dabei helfen,
Veröffentlichungen aufzuspüren, die nicht in bibliographischen
Datenbanken gelistet sind.

Die drei wichtigsten Ergebnisse der dekolonialen szientometrischen
Studien der Dissertation können wie folgt zusammengefasst werden:
Erstens sind \enquote{lokale} südostafrikanische Zeitschriften in Bezug
auf ihre Autorïnnenschaft oft sehr \enquote{international}. Zweitens
wird die südostafrikanische sozial- und geisteswissenschaftliche
Literatur, die im \enquote{Globalen Süden} publiziert wurde, durchaus im
\enquote{Globaler Norden} wahrgenommen -- zumindest lassen in
\emph{Google Scholar} verzeichnete Zitationen dies vermuten. Drittens
verliert die Beobachtung aus früheren Studien, dass Forschende aus dem
\enquote{Globaler Süden} eher Forschung aus dem \enquote{Globalen
Norden} zitieren, tendenziell ihre Überzeugungskraft, wenn die
Datenbasis auf \enquote{lokale Zeitschriften} erweitert wird --
zumindest gilt dies für die kleine Fallstudie, in der ich die
Institutionszugehörigkeit derjenigen untersuchte, die Arbeiten
mauritischer Autorïnnen zitieren.

Wenn sich südostafrikanische Autorïnnen zunehmend auf
Veröffentlichungsorte im \enquote{Globalen Norden} fokussieren,
verbessert sich natürlich die Auffindbarkeit und Sichtbarkeit ihrer
Werke. Dieser Effekt mag sowohl von den Autorïnnen als auch von ihren
Leserïnnen begrüßt werden, hat jedoch seinen Preis: die weitere Erosion
der lokalen oder regionalen Publikationsinfrastrukturen, und mit ihr die
Erosion der lokalen Eigenarten bei der Qualitätssicherung, Auswahl und
Editierung von Publikationen. Wenn südostafrikanische Forschende
weiterhin an dieser Arbeit, die für wissenschaftliche Kommunikation von
größter Bedeutung ist, teilhaben, werden sie dies zunehmend unter
Bedingungen tun, die vollständig im \enquote{Globalen Norden} definiert
werden.

Die szientometrischen Untersuchungen in der Dissertation zeigen
außerdem, wie unzureichend die von Bibliotheken zur Verfügung gestellten
Suchinstrumente sind, wenn der Anspruch lautet, globale Forschung
wenigstens repräsentativ auffindbar zu machen. Während Bibliotheken und
die Unternehmen, von denen sie gewöhnlich Informationsdienstleistungen
beziehen, auf standardisierte bibliografische Metadatenformate bestehen,
um ihre Indizes zu befüllen, ist der völlig andere Webcrawl-Ansatz von
\emph{Google} zumindest in Bezug auf die Abdeckung weit überlegen, auch
im Hinblick auf Literatur, die in \enquote{lokalen Zeitschriften} im
\enquote{Globalen Süden} erscheint. Dennoch ist es höchst problematisch,
Forschenden deshalb \emph{Google Scholar} (\textsc{Gs}) für die
Literaturrecherche zu empfehlen. Die Inklusionskriterien sind rein
formal und technisch, und es gibt keine Möglichkeit, die Inklusion
bestimmter Datensätze zu beantragen. Niemand ist für die Plattform
Ansprechpartnerïn, weshalb auch die Frage nach der Verantwortung
kritisch ist. \textsc{Gs} wird von einem Unternehmen betrieben, dessen
Kerngeschäft auf Werbung und der Erfassung von Benutzerïnnendaten
basiert. Es ist anzunehmen, dass die Hauptmotivation von \emph{Google},
das kostenlos verwendbare Produkt anzubieten, nicht darin besteht, den
Informationsbedarf von Forschenden zu decken. Aus diesen Gründen ist es
nicht nachhaltig, sich auf \textsc{Gs} zu verlassen, auch wenn der
Webcrawl-Ansatz gegenüber der kontrollierten Inklusion grundsätzlich
eine Erhöhung der Komplexität zur Folge hat, was ja im Rahmen dieser
Argumentation ein positiver Effekt wäre. Während es für das
Wissenschaftssystem immer ein Risiko darstellt, wesentliche
Infrastrukturen von wirtschaftlichen Interessen abhängig zu machen,
potenziert sich das Risiko im Falle von \textsc{Gs} dadurch, dass die
Bereitstellung von wissenschaftlicher Information nur ein winziges
Nischenprojekt für das Unternehmen darstellt, das, wie bereits so viele
Sparten von \emph{Google}, auch plötzlich eingestellt werden könnte.

\hypertarget{biblio}{%
\section{6 Dekolonialisierung von Beständen wissenschaftlicher
Bibliotheken in Europa}\label{biblio}}

Der Bestand einer Bibliothek kann als Dienst verstanden werden, der für
die Benutzerïnnen Materialien bereitstellt, welche die Ressourcen
ergänzen, die bereits mit Hilfe von Standardsuchwerkzeugen auffindbar
und zugänglich sind, wie z. B. mit Hilfe von \textsc{Gs}. Bestände, die
sich durch kulturelle Demut auszeichnen, schließen Ressourcen zu Themen
ein, von denen bekannt ist, dass sie für die Benutzerïnnen Relevanz
haben. Darüber hinaus sollten solche Bestände aber auch
Veröffentlichungen von kleinen Verlagen zu diesen Themen beinhalten,
selbst wenn dies eine Abweichung von Standardworkflows bei der Auswahl
und Beschaffung von Medien erfordert.

In der Literatur diskutierte Methoden zur Bestandsevaluation fokussieren
normalerweise darauf, wie gut die Literatur abgedeckt ist, die in den
Standard-Datenbanken der jeweiligen Fachbereiche verzeichnet ist,
während alles darüber hinaus Gehende in eine undifferenzierte Grauzone
verschoben wird. Kulturelle Demut erfordert thematische Offenheit, die
durch einen Evaluationssansatz erreicht werden kann, der auf
geografischer Verteilung von Verlagen und Institutionszugehörigkeiten
von Autorïnnen basiert.

Derzeit wird die intellektuelle Literaturauswahl durch Bibliothekarïnnen
immer seltener und ist nur eine ergänzende Methode zu Produkten von
Informationsdienstleistungsunternehmen, die eine Vorauswahl treffen und
statistisch-algorithmisch den vermeintlichen Bedarf der Benutzerïnnen
messen, an dem dann die Akquisition mehr oder weniger automatisiert
ausgerichtet wird (\emph{demand driven acquisition}). Insbesondere kann
als \enquote{Bedarf} nur erfasst werden, was bereits durch die
Suchwerkzeuge auffindbar ist. Auch wird die Verfügbarkeit von
Benutzerïnnendaten dazu verwendet, \enquote{Bedarfe} zu priorisieren --
je nach hierarchischer Position der Benutzerïnnen, womit eine
Machtkomponente in das System einfließt, die sich leicht
verselbstständigen kann, wenn die kritische bibliothekarische Kontrolle
des Systems z. B. aus Zeitgründen vernachlässigt wird. Es ist weiters
festzustellen, dass Bibliothekstechnologien und Bestände aufgrund der
seit Jahrzehnten voranschreitenden Konsolidierung von
Informationsdienstleistungsunternehmen zunehmend zu Homogenisierung
neigen, insbesondere durch die Lizenzierung von großen
Ressourcen-Paketen (\emph{big deals}) und durch Verträge, die schon vor
deren Erscheinen die Lieferung bestimmter Ressourcen anhand vereinbarter
inhaltlicher und formaler Kriterien \enquote{zur Ansicht} festlegen
(\emph{approval plans}). In den Produkten dieser Unternehmen spiegeln
sich allgegenwärtige soziale Vorurteile, die mit der Verwendung dieser
Produkte unkritisch reproduziert werden. Ein wichtiger Grundpfeiler der
bibliothekarischen Berufsethik, die Neutralität -- ganz gleich, ob sie
als passiv den \enquote{Bedarf} bedienend oder als aktive Bereitstellung
von diversen, aber nichtsdestotrotz \enquote{zentralen} Materialien
verstanden wird, steht nicht im Widerspruch zu den beschriebenen neueren
Methoden der Bestandsentwicklung. Eine so verstandene Neutralität wird
unter den Bedingungen von Kolonialität \emph{ad absurdum} geführt.

Um die Reproduktion sozialer Vorurteile in Bibliotheksbeständen
anzugehen, während Benutzerïnnen ein notwendig selektiver Zugang zur
Grundgesamtheit aller veröffentlichten Literatur ermöglicht wird, sind
langwierige und kostspielige Akquisitionsprozesse erforderlich. Es ist
ein zirkuläres Argument, wenn dieses Ziel wegen mangelnden
\enquote{Bedarfs} der Benutzerïnnen nicht verfolgt wird. Benutzerïnnen
können keinen \enquote{Bedarf} an etwas anmelden, das für sie unsichtbar
ist. Die von der Bibliothek angebotenen Suchwerkzeuge sollten auch auf
\enquote{periphere} Materialien verweisen, um produktive Irritationen zu
ermöglichen. Die allgemein positive Konnotation von Neutralität im
Kontext des Bibliothekswesens kann nur aufrechterhalten werden, wenn sie
als kulturell demütige Neutralität konzipiert ist.

\hypertarget{liste}{%
\section{7 Implikationen}\label{liste}}

Der Charakter der Dissertation ist hauptsächlich argumentativ, und
schließt viele Begriffsdiskussionen ein, die durch kleinere empirische
Studien gestützt werden. Als Desiderat stelle ich im Anhang eine Liste
von Vorschlägen bereit, die als Grundlage für eine professionelle
Debatte über die Dekolonialisierung einer wissenschaftlichen Bibliothek
dienen kann. Weil ich annehme, dass die Leserïnnen dieser
Zusammenfassung besonders an diesen Vorschlägen interessiert sind,
stelle ich sie hier unter dem Titel \enquote{Kulturelle Demut für
Bibliothekarïnnen wissenschaftlicher Bibliotheken} bereit:

\begin{enumerate}
\def\labelenumi{\arabic{enumi}.}
\item
  Erkennen und analysieren Sie Ihre eigenen Privilegien, vermeiden Sie
  deren Nutzung und die Reproduktion von Privilegien in der
  Gesellschaft.
\item
  Praktizieren Sie kulturelle Demut in Bezug auf Personalauswahl,
  Management und Öffentlichkeitsarbeit in Ihrer Bibliothek, und teilen
  Sie Ihre Erfahrungen mit dem bibliothekarischen Nachwuchs.
\item
  Informieren Sie sich über das wissenschaftliche Publizieren in
  benachteiligten Kontexten.
\item
  Hinterfragen Sie \enquote{Bedarf} als Hauptkriterium für die
  Bestandsentwicklung.
\item
  Unterstützen Sie die Indexierung von \enquote{peripherer} Literatur.
\item
  Fordern Sie auch von Informationsdienstleistungsunternehmen und ihren
  Produkten kulturelle Demut. (Konsortial-)Boykott ist eine Option.
\item
  Unterstützen Sie gemeinnützige, nicht-kommerzielle
  Publikationsinfrastrukturen lokal \linebreak und global, um die Macht, Standards
  setzen zu können, von kommerziellen Verlagen auf Forschende und ihre
  Institutionen zu übertragen.
\item
  Unterstützen Sie keine Open-Access-Geschäftsmodelle, bei denen
  autorïnnenseitig Gebühren anfallen, da diese sozial ungerecht sind und
  Bibliothekarïnnen ihre Verhandlungsmacht gegenüber kommerziellen
  Verlagen nehmen.
\item
  Öffnen Sie die \emph{black box} Ihrer Bestände, insbesondere der
  lizenzierten Inhalte: Evaluieren Sie kritisch nach
  Veröffentlichungsorten und Autorïnnenherkunft.
\item
  Öffnen Sie die \emph{black box} auch für die Benutzerïnnen: Machen Sie
  in Ihrem Suchinstrument Felder für Veröffentlichungsorte und
  Autorïnnenherkunft verfügbar.
\item
  Konzipieren Sie das Bestandsmanagement als kollektive Aufgabe neu:
  Bibliothekarïnnen, Forschende und Studierende sollten kooperieren, um
  an sozialen Vorurteilen zu rütteln und die Priorisierung von
  Akquisitionen abzuwägen.
\item
  Initiieren Sie Diskussionsgruppen zur Dekolonialisierung des
  Curriculums, in denen Bibliothekarïnnen, Forschende und Studierende
  kooperieren, um Kursinhalte und Literaturlisten mit mehr kultureller
  Demut zu erstellen.
\item
  Kooperieren Sie mit Bibliotheken in der Nähe, um gemeinsam umfassende
  Bestände mit gemeinsamen Suchwerkzeugen bereitzustellen. Lernen Sie
  von Bibliothekarïnnen, die auf Ressourcen aus dem \enquote{Globalen
  Süden} spezialisiert sind.
\item
  Beenden Sie die Abtrennung der Regionalwissenschaften, indem Sie die
  entsprechenden Materialien über beschreibende Metadaten in die sozial-
  und geisteswissenschaftlichen Kerndisziplinen einordnen.
\item
  Beschreiben und bewerben Sie Ihre Bestände gegenüber Benutzerïnnen,
  Geldgebern und der Öffentlichkeit als durch kulturelle Demut geprägte
  Bestände.
\item
  Beenden Sie Debatten innerhalb Ihrer Institution nicht, indem Sie
  Leitlinien fixieren -- Leitlinien sollten immer lebendige Dokumente
  bleiben.
\end{enumerate}

\section{Literatur}

Schmidt, Nora (2020). \textit{The privilege to select. global research system, european academic library collections, and decolonisation}. PhD Thesis. Lund: Lund University, Faculties of Humanities and Theology, Lund Studies in Arts and Cultural Sciences. DOI: \href{https://doi.org/10.5281/zenodo.4011295}{10.5281/zenodo.4011295}

%autor
\begin{center}\rule{0.5\linewidth}{0.5pt}\end{center}

\textbf{Nora Schmidt}, MA, MSc, PhD \textbar{} Universität für Musik
und darstellende Kunst Wien, Wien, Österreich \textbar ORCID:
\href{https://orcid.org/0000-0002-7105-9515}{0000-0002-7105-9515}

\end{document}
