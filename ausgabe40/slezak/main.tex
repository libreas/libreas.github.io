\documentclass[a4paper,
fontsize=11pt,
%headings=small,
oneside,
numbers=noperiodatend,
parskip=half-,
bibliography=totoc,
final
]{scrartcl}

\usepackage[babel]{csquotes}
\usepackage{synttree}
\usepackage{graphicx}
\setkeys{Gin}{width=.4\textwidth} %default pics size

\graphicspath{{./plots/}}
\usepackage[ngerman]{babel}
\usepackage[T1]{fontenc}
%\usepackage{amsmath}
\usepackage[utf8x]{inputenc}
\usepackage [hyphens]{url}
\usepackage{booktabs} 
\usepackage[left=2.4cm,right=2.4cm,top=2.3cm,bottom=2cm,includeheadfoot]{geometry}
\usepackage{eurosym}
\usepackage{multirow}
\usepackage[ngerman]{varioref}
\setcapindent{1em}
\renewcommand{\labelitemi}{--}
\usepackage{paralist}
\usepackage{pdfpages}
\usepackage{lscape}
\usepackage{float}
\usepackage{acronym}
\usepackage{eurosym}
\usepackage{longtable,lscape}
\usepackage{mathpazo}
\usepackage[normalem]{ulem} %emphasize weiterhin kursiv
\usepackage[flushmargin,ragged]{footmisc} % left align footnote
\usepackage{ccicons} 
\setcapindent{0pt} % no indentation in captions

%%%% fancy LIBREAS URL color 
\usepackage{xcolor}
\definecolor{libreas}{RGB}{112,0,0}

\usepackage{listings}

\urlstyle{same}  % don't use monospace font for urls

\usepackage[fleqn]{amsmath}

%adjust fontsize for part

\usepackage{sectsty}
\partfont{\large}

%Das BibTeX-Zeichen mit \BibTeX setzen:
\def\symbol#1{\char #1\relax}
\def\bsl{{\tt\symbol{'134}}}
\def\BibTeX{{\rm B\kern-.05em{\sc i\kern-.025em b}\kern-.08em
    T\kern-.1667em\lower.7ex\hbox{E}\kern-.125emX}}

\usepackage{fancyhdr}
\fancyhf{}
\pagestyle{fancyplain}
\fancyhead[R]{\thepage}

% make sure bookmarks are created eventough sections are not numbered!
% uncommend if sections are numbered (bookmarks created by default)
\makeatletter
\renewcommand\@seccntformat[1]{}
\makeatother

% typo setup
\clubpenalty = 10000
\widowpenalty = 10000
\displaywidowpenalty = 10000

\usepackage{hyperxmp}
\usepackage[colorlinks, linkcolor=black,citecolor=black, urlcolor=libreas,
breaklinks= true,bookmarks=true,bookmarksopen=true]{hyperref}
\usepackage{breakurl}

%meta
%meta

\fancyhead[L]{G. Slezak, S. Schmelzer, A. Ruscher, J. Paintner, E. Celebi, D. Baumgartner\\ %author
LIBREAS. Library Ideas, 40 (2021). % journal, issue, volume.
\href{https://doi.org/10.18452/23805}{\color{black}https://doi.org/10.18452/23805}
{}} % doi 
\fancyhead[R]{\thepage} %page number
\fancyfoot[L] {\ccLogo \ccAttribution\ \href{https://creativecommons.org/licenses/by/4.0/}{\color{black}Creative Commons BY 4.0}}  %licence
\fancyfoot[R] {ISSN: 1860-7950}

\title{\LARGE{Rassismen im Bücherregal? Ein Praxisbericht von der kolonialen Spurensuche in der Bibliothek bis hin zu einer rassismuskritischen Bildung}}% title
\author{Gabriele Slezak \and Sarah Schmelzer \and Andrea Ruscher \and Jonas Paintner \and Erem Celebi \and Dani Baumgartner} % author

\setcounter{page}{1}

\hypersetup{%
      pdftitle={Rassismen im Bücherregal? Ein Praxisbericht von der kolonialen Spurensuche in der Bibliothek bis hin zu einer rassismuskritischen Bildung},
      pdfauthor={Gabriele Slezak, Sarah Schmelzer, Andrea Ruscher, Jonas Paintner, Erem Celebi, Dani Baumgartner},
      pdfcopyright={CC BY 4.0 International},
      pdfsubject={LIBREAS. Library Ideas, 40 (2021)},
      pdfkeywords={Bibliothek, Dekolonisierung, library, decolonization},
      pdflicenseurl={https://creativecommons.org/licenses/by/4.0/},
      pdfcontacturl={http://libreas.eu},
      baseurl={https://doi.org/10.18452/23805},
      pdflang={de},
      pdfmetalang={de}
     }
	 



\date{}
\begin{document}

\maketitle
\thispagestyle{fancyplain} 

%abstracts
\begin{abstract}
\noindent
Bibliotheken verwalten Wissen. Bibliotheken sind Räume für
Menschen. Was passiert, wenn all das auf kolonialer Vergangenheit
aufbaut? Die C3-Bibliothek für Entwicklungspolitik in Wien hat 2018
begonnen, sich aktiv mit der eigenen Geschichte rassistischer Praktiken
auseinanderzusetzen. Wir verstehen Dekolonialisierung von Bibliotheken
als Prozess, der aktiv gestaltet werden muss. Dieser braucht einen
Kompetenzaufbau für die inhaltliche Auseinandersetzung mit Rassismen in
überwiegend weißen Institutionen durch Erwerben von theoretischem Wissen
und kritische, selbstreflexive Auseinandersetzung mit rassistischen
Denkweisen wie auch eine diskursive und partizipative Ausgestaltung mit
breiter Einbindung der Akteur*innen. Damit Letzteres gelingen kann gilt
es auf der Ebene der Nutzer*innen und Bürger*innen eine direkte und
offene Thematisierung zu initiieren, die aktive Teilhabe und Engagement
aller an der Debatte fördert. Für die C3-Bibiothek ist dafür die
Vermittlung von Informationskompetenz als transformatorisches Werkzeug
zentral. Sie ermöglicht ein Engagement, das sich manipulativen,
entmenschlichenden und kolonisierenden Prinzipien widersetzen kann. Der
Beitrag stellt erste Erfahrungen aus der Praxis und Reflexionsergebnisse
dem Austausch mit gleichgesinnten Bibliotheken im deutschsprachigen Raum
zur Verfügung.
\end{abstract}

%body
\hypertarget{unser-ausgangspunkt-die-c3-bibliothek-fuxfcr-entwicklungspolitik}{%
\section{Unser Ausgangspunkt: Die C3-Bibliothek für
Entwicklungspolitik}\label{unser-ausgangspunkt-die-c3-bibliothek-fuxfcr-entwicklungspolitik}}

Die Österreichische Forschungsstiftung für Internationale Entwicklung
(ÖFSE), die Organisation Frauen*solidarität und die Bildungsstelle
BAOBAB betreiben gemeinsam die C3-Bibliothek für
Entwicklungspolitik\footnote{C3-Bibliothek für Entwicklungspolitik: Über
  uns, vergleiche \url{https://www.centrum3.at/bibliothek/ueber-uns}}.
Mit einem Bestand von etwa 80.000 gedruckten und einem Vielfachen an
digitalen Medien ist sie Österreichs größte wissenschaftliche und
pädagogische Fachbibliothek zu Internationaler Entwicklung,
Frauen*Gender und Globalem Lernen. Sie ist öffentlich zugänglich, ihre
Zielgruppe ist divers und sie versteht sich als Ort des Wissens, der
Bildung, der Information und der Begegnung für alle. Ein vielfältiges
Veranstaltungsprogramm und zahlreiche Aktivitäten fördern gemeinsames
Lernen, Dialog und Austausch.

Ebenso verfügt die Bibliothek über einen umfassenden wissenschaftlichen
Bestand zur Entwicklungsforschung. Er ist das Ergebnis von über 50
Jahren Sammlungsaktivität, ergänzt durch einige Schenkungen. Gerade in
älteren Bestandssegmenten findet sich darin Literatur mit rassistischen
Inhalten. Für die Bibliothek ergibt sich dadurch ein
Spannungsverhältnis.

Einerseits können jene Teilbestände mit rassistischen Inhalten auch
wertvolles Quellenmaterial für antirassistische und postkoloniale
Forschung sein. Deshalb will die Bibliothek sie gemäß ihrem Sammelprofil
der Wissenschaft zugänglich machen. Andererseits werden jene
problematischen Inhalte schon allein durch ihre Aufstellung in einer
wissenschaftlichen Institution reproduziert und legitimiert. Bestände,
die aus der Kolonial- und Missionsvergangenheit stammen, tragen so zum
Fortbestehen kolonialer Ideologie und zur Konstruktion rassistischer
Perspektiven auf die Kolonisierten bei. Die Präsenz solcher Titel
beeinträchtigt die Lernumgebung erheblich und widerspricht maßgeblich
dem Anspruch eines Raumes, in dem sich alle willkommen und dazu
eingeladen fühlen, in einen offenen Austausch zu treten.

Trotzdem -- oder gerade deshalb -- war das Bibliotheksteam gefordert zu
handeln. 2018 wurde daher beschlossen, einen Reflexionsprozess
einzuleiten. Ziel war es, Strategien der Dekolonisierung und
rassismuskritischen Bildung für die Bibliothek zu entwickeln. Auf die
Ebenen dieser Auseinandersetzungen und die Erfahrungen der C3-Bibliothek
für Entwicklungspolitik in diesem Prozess wird in den folgenden
Abschnitten eingegangen.

\hypertarget{koloniale-spurensuche}{%
\section{Koloniale Spurensuche}\label{koloniale-spurensuche}}

Eine fundierte Auseinandersetzung mit Rassismen in der eigenen
Bibliothek kommt nicht ohne eine kritische Betrachtung des jeweiligen
Bestandes aus. Zu diesem Zwecke analysierten wir 2018 einen ausgewählten
Bestandsabschnitt exemplarisch. Ergebnisoffen und gestützt von
multiperspektivischer Forschungsliteratur wurden präsente Rassismen
sowie historische Entwicklungen untersucht und dokumentiert. (Die
Analyse wurde unter anderem im Zuge der vbib20 vorgestellt und kann
online nachgesehen werden: \url{https://av.tib.eu/media/47210}, DOI:
\url{https://doi.org/10.5446/47210})

Der dekoloniale Blick auf Sammlungen bringt erfahrungsgemäß einen
Mehrwert für die jeweilige Einrichtung: Er liefert Anregung für
Perspektivenwechsel, Inklusion und gesellschaftspolitische
Stellungnahme. Sind Mitarbeiter*innen folglich mit Formen von Rassismus
und kolonialen Machtverhältnissen, die in ihren Medien transportiert
werden, vertraut, kann konstruktiv über entsprechende Maßnahmen
diskutiert werden. Im Zuge dessen wurden auch erste Überlegungen
angestellt, wie eine lebendige Debatte über Praxen der Dekolonisierung
und Rassismuskritik in der Institution und im Dialog mit Nutzer*innen
geführt werden kann.

\hypertarget{kompetenzaufbau-als-wissenschaftliche-basis}{%
\section{Kompetenzaufbau als wissenschaftliche
Basis}\label{kompetenzaufbau-als-wissenschaftliche-basis}}

Ergebnis des ersten Reflexionsprozesses war außerdem, dass ein Raum für
fundierte Auseinandersetzung mit wissenschaftlichen Diskursen zur
Thematik geschaffen werden sollte. Kristallisationspunkte existierender
Debatten um Kolonialismus in Bibliotheken sind Postkoloniale Studien,
Rassismusforschung und rassismuskritische Bildung sowie Fragen rund um
Dekolonisierung von Wissen im Allgemeinen.

2018 gab es aber kaum Fachliteratur zur kritischen Auseinandersetzung
von Bibliotheken in Österreich beziehungsweise im deutschsprachigen
Raum. Demgegenüber erwies sich die professionelle Auseinandersetzung mit
Rassismen in und um Bibliotheken im angloamerikanischen Raum als
wesentlich weiter fortgeschritten.\footnote{Vergleiche: Ruscher, Andrea;
  Schmelzer, Sarah; Baumgartner, Dani; Slezak, Gabi: \enquote{Rassismen
  in Bibliotheksbeständen. Im Spannungsfeld zwischen Sammelauftrag und
  Bildungsarbeit}. In: Köstner-Pemsel, Christina et al.~(Hg.):
  Künstliche Intelligenz in Bibliotheken. 34. Österreichischer
  Bibliothekartag Graz 2019. S. 346--348.} So wurde schnell klar, dass
ein Kompetenzaufbau in diesem komplexen Feld nur über Kooperationen mit
externen Partner*innen funktionieren kann. Da Auslöser der Debatte in
der C3-Bibliothek ein Bestandssegment mit rassistischen Inhalten in
Bezug auf afrikanische Gesellschaften war, bot sich eine Zusammenarbeit
mit dem Institut für Afrikawissenschaften der Universität Wien an.

\hypertarget{lernraum-mit-dekolonialem-anspruch}{%
\subsection{\texorpdfstring{Lernraum mit \enquote*{dekolonialem
Anspruch}}{Lernraum mit `dekolonialem Anspruch'}}\label{lernraum-mit-dekolonialem-anspruch}}

So entstand ein interdisziplinär ausgerichtetes Seminar, das von Oktober
2019 an über drei Semester hinweg auch Interessierte außerhalb der
Afrikawissenschaften erreichte. Im Austausch mit Studierenden der
Politikwissenschaften, Internationale Entwicklung, Publizistik,
Psychologie und der Geschichtswissenschaft etablierte sich ein Lernraum
für rassismuskritische Bildung an der Universität, in dem hinterfragt
wurde, wie Wissen ausgewählt, aufbereitet und zur Verfügung gestellt
wird. Insbesondere der Prozess der Bewertung von Wissensinhalten und
darin implizite Hierarchien standen im Fokus.

Im Zuge der Seminare bearbeiteten die Studierenden unterschiedliche
Fallbeispiele mit qualitativen sozial- und kulturwissenschaftlichen
Forschungsmethoden. Diese bildeten den Rahmen für eine Annäherung an
unterschiedliche theoretische und methodische Ansätze anhand von
exemplarischen Untersuchungen von Medien wie Kinderbüchern,
Reiseführern, Bildbänden, Lehrbüchern und deren Aufbereitung in
Universitätsbibliotheken, Museumsbibliotheken sowie öffentlichen
Büchereien. Die Themen reichten von kolonialen Spuren in
Ordnungssystemen (siehe Beitrag von Sandra Sparber in dieser Ausgabe,
\url{https://doi.org/10.18452/23803}) bis hin zur Verantwortung
öffentlicher Bibliotheken im Lichte der Kolonialgeschichte und Critical
Whiteness Studies.\footnote{In den ersten Monaten waren auch Exkursionen
  in Bibliotheken und Museen sowie die Teilnahme an antirassistischen
  Projekten und wissenschaftlichen Vorträgen möglich, die aufgrund der
  Covid19-Maßnahmen ab März 2020 entfallen mussten.}

\hypertarget{standortbestimmung}{%
\subsection{Standortbestimmung}\label{standortbestimmung}}

Im ersten Semester galt es den \enquote*{dekolonialen Anspruch} der
Seminarreihe zu definieren (Andreotti 2011, Andreotti 2015). Zur
Orientierung bei der kolonialen Spurensuche dienten insbesondere
Denkräume und Debatten zu Dekolonisierung der Wissenschaft an
afrikanischen Universitäten wie die Ateliers de la Pensee de
Dakar\footnote{Siehe \url{https://www.picuki.com/tag/ADLP2019}} oder die
African Studies Association of Africa (ASAA)\footnote{Siehe
  \url{https://2019conference.as-aa.org/}}\emph{.} Damit sollte einem
wissenschaftshistorischen Ansatz der Afrikawissenschaften Rechnung
getragen werden, die eigene Disziplin kritisch zu reflektieren. Eine
ernsthafte Auseinandersetzung mit dem Erbe des Kolonialismus\footnote{Jürgen
  Osterhammel stellt fest, dass es sich bei Kolonialismus nicht um jedes
  Herrschaftsverhältnis handelt, sondern dass koloniale Herrschaft durch
  einen absoluten Anspruch des \enquote{Kolonialherren} gegenüber
  \enquote{Anderen} und auf deren Ressourcen gekennzeichnet ist. Vergl.:
  Osterhammel, Jürgen (1995): Kolonialismus. Geschichte, Formen, Folgen.
  München: Beck. S. 19ff.} setzt hierbei voraus, Theoretiker*innen auf
dem afrikanischen Kontinent und aus der Diaspora miteinzubeziehen.

In der Praxis untersuchten Seminarteilnehmer*innen beispielsweise
Psychologie-Lehrbücher, die nach wie vor in der universitären Ausbildung
eingesetzt werden, auf implizite Rassismen. In diesem Kontext wiesen sie
etwa nach, wie in psychologischer Fachliteratur nach wie vor
eurozentrische Überlegenheitsansprüche legitimierte werden. Eine andere
Untersuchung war kolonialen und sexistischen Spuren in einer
Dauerausstellung des Weltmuseums in Wien zum Themenkomplex Migration
gewidmet. Die Studierenden kamen zu dem Ergebnis, dass Repräsentationen
migrierender Menschen durch problematische Prozesse des
\enquote*{Othering} mit diskriminierenden Vermittlungspraktiken eng
verflochten waren.

\hypertarget{postkolonialer-fokus}{%
\subsection{Postkolonialer Fokus}\label{postkolonialer-fokus}}

Im zweiten Seminardurchgang lag der Schwerpunkt stärker auf
postkolonialen Theoreti-\linebreak ker*innen, da diese das theoretische Rückgrat für
Debatten im gesamten Themenfeld bilden. Denker*innen aus dem Globalen
Süden, wie Edward Said\footnote{Vergleiche: Said, Edward (1978):
  Orientalism. New York: Pantheon Books.}, Gayatri Chakravorty
Spivak\footnote{Vergleiche: Landry, Donna et al.~(Hg.) (1996): The
  Spivak reader. Selected works of Gayatri Chakravorty Spivak. London:
  Routledge.} oder Joseph-Achille Mbembe\footnote{Vergleiche: Mbembe,
  Achille (2011): On the postcolony. Berkely: Universtiy of California
  Press.}, lieferten die nötige Basis für Analysen der ideologischen
Funktion von Kolonialismus und für den Entwurf effektiver
antirassistischer Strategien im bibliothekarischen Raum.

Wissen über die Gesellschaften Afrikas legitimierte und stütze Praktiken
der Kolonialherrschaft. Es sollte dazu beitragen, die Kolonisierten
\enquote{besser} beherrschen und ausbeuten zu können. Das Argument des
\enquote{objektiven Sammelauftrags} einer Bibliothek steht dazu in
direktem Widerspruch. Denn die Bibliothek wirkt als Ort der
Wissensproduktion, ist damit Teil gesellschaftlicher Machtausübung und
bei Weitem kein vermeintlich \enquote{neutraler Ort}. Tatsächlich
entstanden viele wissenschaftliche Bibliotheken in Zeiten kolonialer
Expansionspolitik, sodass gerade auch ihr Gründungsauftrag kritischer
Analyse bedarf. Entsprechend untersuchte das Seminar die Wirkmächtigkeit
und Vielschichtigkeit des Fortbestands kolonialer Diskurse in
Bibliotheken bis in die Gegenwart.

Elisa Frei analysierte im Zuge dessen unterschiedliche Ausgaben des
bekannten österreichischen Kinderbuches \enquote{\emph{Hatschi Bratschis
Luftballon''}.\footnote{Vergleiche: Ginzkey, Franz Karl; Mor von
  Sunnegg; Morberg, Erich; Heydemann, Klaus (2019): Hatschi-Bratschi's
  Luftballon: eine Dichtung für Kinder. Faksimile der Erstausgabe aus
  dem Jahr 1904, 1. Auflage. Wien : European University Press
  Verlagsgesellschaft m.b.H., Ibera Verlag.} In ihrer Seminararbeit
zeigte sie anschließend, wie rassifizierte Beschreibungen
Minderwertigkeit konstruierten und wie sich diese im Bewusstsein der
Leser*innen festsetzt. Dazu interviewte sie Bibliothekar*innen einer
wissenschaftlichen Bibliothek und einer öffentlichen Bücherei, die das
Kinderbuch trotz rassistischer Inhalte im Bestand behielten. Deren
Positionierungen zeugten zum einen von einem Bewusstsein für die
Problematik rassistischer Inhalte im Bestand, zum anderen verdeutlichen
sie das Spannungsfeld, das aufgrund des freien Informationszugangs,
hohen Entlehnzahlen des} Klassikers österreichischer Kinderliteratur''
oder des Auftrags der Literaturversorgung von Universitätsangehörigen
für Studium, Forschung und Lehre zu Bilderbüchern entsteht.

\hypertarget{einblicke-in-dekoloniale-strategien-von-bibliotheken}{%
\subsection{Einblicke in dekoloniale Strategien von
Bibliotheken}\label{einblicke-in-dekoloniale-strategien-von-bibliotheken}}

Im dritten und vorerst letzten Semester der Seminarreihe kam eine
stärker reflexive Ebene hinzu, die die rassismuskritische Perspektive
von Studierenden und Lehrenden aus dem Globalen Norden auf
Wissensproduktion an Universitäten ins Zentrum rückte. Voraussetzung
dafür war, dass sich alle Beteiligten etwaige eigene Privilegien bewusst
machten. Den entscheidenden Beitrag hierzu leistete die verstärkte
Teilnahme von Schwarzen Studierenden und People of Color (PoC), die im
gemeinsamen Lernraum institutionelle Ungerechtigkeiten an
Universitätsinstituten, Schulen und Formen von epistemischer
Gewalt\footnote{Im Konzept der Epistemischen Gewalt wird eurozentrische
  Dominanz in der Wissenschaft mit materieller Ungleichheit
  zusammengedacht. Wissenschaft, die oft als gewaltlos eingestuft wird,
  kann dadurch als Instrument der Unterdrückung kritisiert werden.
  (Vergleiche unter anderem: Brunner, Claudia (2020). Epistemische
  Gewalt. Wissen und Herrschaft in der kolonialen Moderne. Bielefeld:
  transcript. \url{https://doi.org/10.14361/9783839451311}} explizit
benannten. Die so offengelegten Machtasymmetrien wurden im Kontext des
Seminars als Auswirkungen kolonialer Geschichte, Denkmuster~und
rassistischer Gesellschaftsstrukturen fassbar gemacht und intensiv
diskutiert.

Auch in diesem Semester brachten die Analysen verschiedene konkrete
Beispiele für koloniale Kontinuitäten in Bibliotheken und Büchereien
hervor. Die Bandbreite reichte von Schulbibliotheken bis hin zum Katalog
der Fachbereichsbibliothek für Publizistik der Universität Wien. Darüber
hinaus entstanden im Seminar Handlungsoptionen und Utopien, wie
bibliothekarische Ordnungsmechanismen durchbrochen und Formen
epistemischer Gewalt in öffentlichen Einrichtungen entgegengewirkt
werden können.

Da der Umgang mit rassistischen und kolonialen Diskursen im
angloamerikanischen Raum auf eine längere Geschichte zurückblickt,
erwiesen sich hierbei insbesondere \enquote*{best practice}-Beispiele
aus Großbritannien als inspirierend. Im Seminar brachte Jonas Paintner
diese Aspekte aufgrund seiner persönlichen Studienerfahrung in London
ein: Unter Hashtags wie \#DecoloniseEducation,
\#DecoloniseTheUniversity, \#LiberateTheCurriculum oder
\#DecoloniseTheLibrary wird in Großbritannien viel diskutiert und neu
gedacht. Einige Universitätsbibliotheken sind dabei auf dem Weg, ihre
zentrale Rolle im gesellschaftlichen Wissenssystem anzuerkennen. Sie
betonen die Macht, die sie ausüben, indem sie bestimmte Wissensinhalte
auswählen, ordnen, klassifizieren und zugänglich machen. Das trägt dazu
bei, dass tradierte, rassistische Literaturkanons weiterbestehen, oder
aber dass genau diese Regime aufgebrochen werden. Im Sinne des
Aufbrechens versuchen einige Bibliotheken, Raum für Narrative zu bieten,
die dem kolonial geprägten Mainstream entgegentreten. Einfache, aber
effektive Mittel sind beispielsweise regelmäßig Literatur von
marginalisierten Autor*innen zu empfehlen oder Leselisten um diverse
Perspektiven zu ergänzen.\footnote{Hierzu kann etwa bei der Bibliothek
  der Goldsmiths University of London nachgelesen werden:
  \url{https://www.gold.ac.uk/library/about/liberate-our-library/}}

Wodurch sich die Dekolonisierungsdebatte an britischen Universitäten
auch auszeichnet, ist das Bewusstsein, dass Dekolonisierung nur
funktionieren kann, wenn sie auf allen Ebenen stattfindet. Von der
institutionellen Struktur, über Lehrende, Forschende,
Verwaltungsangestellte, bis hin zu den Studierenden sind alle gefragt,
sich und ihr Wissen einzubringen. Partizipation und Austausch sind
entscheidend, um koloniale Muster zu entlarven und in der Folge
abzubauen.\footnote{Eine umfassende Anleitung zur Dekolonisierung von
  Bildungseinrichtungen gibt der Online-Kurs \enquote{Decolonising
  Education. From Theory to Practice} der University of Bristol. Dieser
  ist über FutureLearn erreichbar:
  \url{https://www.futurelearn.com/courses/decolonising-education-from-theory-to-practice}}
Umgelegt auf die Bibliothek bedeutet das, dass Nutzer*innen wie
Akteur*innen aktiv in den Prozess involviert werden müssen: Zuhören,
Intervenieren und gemeinsam Neu-Konzipieren gilt als zukunftsweisende
Strategie.

\hypertarget{fazit-aus-drei-semestern}{%
\subsection{Fazit aus drei Semestern}\label{fazit-aus-drei-semestern}}

Ausgehend vom Bestreben, Rassismen und koloniale Denkmuster in
unterschiedlichen institutionellen Wissenssystemen aufzuspüren, gelang
es über drei Semester, diese im Sinne einer Global Citizenship Education
mit dekolonialem Anspruch aufzugreifen und in die Analyse von
Bibliotheken einzubeziehen.\footnote{Vergleiche: Andreotti, Vanessa:
  \enquote{The Question of the \enquote*{Other} in Global Citizenship
  Education. A Postcolonial Analysis of Telling Case Studies in
  England.} In: Shultz, Lynette et al.~(Hg.) (2011): Global Citizenship
  Education in Post-Secondary Institutions. Theories, Practices,
  Policies. New York: Peter Lang. S. 140--157.

  Andreotti, Vanessa et al.: \enquote{Mapping interpretations of
  decolonization in the context of higher education.} In:
  Decolonization. Indigeneity, Education \& Society Vol. 4, No.~1 (2015)
  S. 21--40.} Zugleich war auch die Rolle von Bibliotheken als Orte der
Begegnung, Auseinandersetzung und Interaktion von zentralem Interesse.
Im Sinne transformativen Lernens und Lehrens wurde die Wirkmächtigkeit
und Vielschichtigkeit kolonialer Diskurse somit nicht nur aufgespürt, es
kam auch zu Begegnungen mit Erfahrungen von Differenz, Grenzen der
eigenen Wahrnehmung und Irritation seitens aller Beteiligten, die für
progressive Reflexionsprozesse mobilisiert wurden.\footnote{Vergleiche:
  Mecheril, Paul; Klingler, Birte: \enquote{Universität als
  transgressive Lebensform. Anmerkungen, die gesellschaftliche
  Differenz- und Ungleichheitsverhältnisse berücksichtigen.} In:
  Darowska, Lucyna et al.~(Hg.) (2010): Hochschule als transkultureller
  Raum? Kultur, Bildung und Differenz in der Universität. Bielefeld:
  transcript. S. 83--116 sowie Mecheril, Paul et al.~(2013): Differenz
  unter Bedingungen von Differenz. Zu Spannungsverhältnissen
  universitärer Lehre. Wiesbaden: Springer.} Bibliotheken mit ihren
unterschiedlichen Tätigkeitsbereichen und Akteur*innen blieben somit
nicht nur Untersuchungsfeld. Vielmehr dienen sie als Orte des kritischen
Lernens. Das Seminar begann, sich in die Diskurse einzumischen und den
Raum Bibliothek aktiv mitzugestalten.

\hypertarget{zuruxfcck-in-die-bibliothek-rassismuskritische-bildung-und-informationskompetenz-zusammendenken}{%
\section{Zurück in die Bibliothek: Rassismuskritische Bildung und
Informationskompetenz
zusammendenken}\label{zuruxfcck-in-die-bibliothek-rassismuskritische-bildung-und-informationskompetenz-zusammendenken}}

Nach diesem Abriss aus Theorie und Praxis kehren wir zurück in den
Kosmos der C3-Bibliothek für Entwicklungspolitik. Die konkreten
Arbeitsfelder für Bibliotheken, um aktiv an einer Dekolonialisierung der
eigenen Institution zu arbeiten und schrittweise zu einer
Dekolonialisierung von Wissens- und Bildungssystemen beizutragen, sind
vielfältig: Die kritische Analyse des eigenen Bibliotheksbestands ist
ein Ausgangspunkt. Darauf kann zum Beispiel das Hinterfragen der eigenen
Beschlagwortungs-, Beschreibungs- und Klassifizierungspraxen, die
Gestaltung des Erwerbungsprozesses, der marginalisierte Stimmen
systematisch integriert und damit einen multiperspektivischen
Bestandsaufbau ermöglicht, sowie antirassistische Kommunikation und
rassismuskritische Veranstaltungsformate folgen. Eine Spezialform von
Veranstaltungen, die spezifisch für Bibliotheken wichtig sind, stellen
Vermittlungsangebote für Informationskompetenz dar. Im Folgenden soll
ein exemplarischer Schwerpunkt auf dieses Tätigkeitsfeld gelegt werden.

Die Vermittlung von Informationskompetenz wird als eine zentrale Aufgabe
von Bibliotheken definiert.\footnote{Vergleiche: Franke, Fabian:
  \enquote{Die Förderung von Informationskompetenz ist Kernaufgabe von
  Bibliotheken -- und nicht nur der Senf zur Bratwurst!} In: o-bib. Das
  offene Bibliotheksjournal. Bd. 4 Nr. 1 (2017), S. IV-V, DOI:
  \url{https://doi.org/10.5282/o-bib/2017H1}} Im Kontext von
Digitalisierung, Fake News und nicht zuletzt Covid-19 rückt das Thema
Medien- und Informationskompetenz in den Fokus von Bildungsstrategien,
so unter anderem im \enquote{Digital Education Action Plan (2021-2027)}
der EU-Kommission\textbf{.} Das Verständnis von Informationskompetenz
für die praktische Arbeit von Bibliotheken in Deutschland und Österreich
bildet der Leitfaden \enquote{Referenzrahmen
Informationskompetenz}\footnote{Klingenberg, Andreas; Deutscher
  Bibliotheksverband e. V. (2016): Referenzrahmen Informationskompetenz.
  Online verfügbar:
  \url{https://web.archive.org/web/20201125044717/https://www.bibliotheksverband.de/fileadmin/user_upload/Kommissionen/Kom_Infokompetenz/2016_11_neu_Referenzrahmen-Informationskompetenz_endg__2__Kbg.pdf}
  (letzter Aufruf: 30.06.2021; Anmerkung Redaktion: Link aktualisiert am
  07.12.2021).} ab. Der Prozess, kompetent im Umgang mit Informationen
zu werden, wird hier umfassend in der Abfolge der Teilkompetenzen
\enquote{Suchen}, \enquote{Finden}, \enquote{Wissen},
\enquote{Darstellen} und \enquote{Weitergeben} beschrieben. Im Fokus
stehen Fertigkeiten wie \enquote{Suchbegriffe formulieren},
\enquote{gezielt nach Medien suchen} oder \enquote{den Suchprozess
dokumentieren}, nach Hapke eine \enquote{funktional-objektive Sicht des
\enquote*{Erwerbens} von Fähigkeiten zum Umgang mit
Information}\footnote{Hapke, Thomas: \enquote{Informationskompetenz
  anders denken - Zum epistemologischen Kern von \enquote*{information
  literacy}.} In: Sühl-Strohmenger, Wilfried (Hg.) (2016): Handbuch
  Informationskompetenz. 2. Auflage. Berlin {[}u.a.{]}: De Gruyter. S.
  13.}. Die Teilkompetenz \enquote{Wissen} und darunter Aspekte, wie
\enquote{neue Informationen und Bekanntes {[}\ldots{]} in einen größeren
Zusammenhang stellen} öffnen den Raum für kritisch-reflektierende
Aspekte von Informationskompetenz.

Die Formate zur Vermittlung von Informationskompetenz der C3-Bibliothek
für Entwicklungspolitik richten sich vor allem an Studierende und seit
einigen Jahren auch an Schüler*innen ab der 10. Schulstufe. Dabei sind
die Informationskompetenz-Workshops für die Zielgruppe Schüler*innen
Teil eines Projekts, im Zuge dessen auch jährlich ein Preis für
herausragende vorwissenschaftliche Arbeiten\footnote{\enquote{Vorwissenschaftliche
  Arbeit} bezeichnet die Abschlussarbeiten an Höheren Schulen in
  Österreich. Sie sind ein Teil der standardisierten
  kompetenzorientierten Reifeprüfung.} aus dem Themenfeld Internationale
Entwicklung, der \enquote{C3-Award}, vergeben wird. Durch dieses Projekt
haben wir die Möglichkeit, Schüler*innen von Beginn ihrer ersten
wissenschaftlichen Arbeit bis zum Abschluss dieser zu begleiten und
Einblick in ihren Forschungsprozess zu erhalten. Im Feld der
Informationskompetenzvermittlung erlaubt uns die unmittelbare Erkenntnis
zum Informationsbedarf Jugendlicher, die Wirkung unserer begleitenden
Angebote und möglichen Weiterentwicklungen auf ihre Bedürfnisse
abzustimmen.

Stellen wir uns nun vor diesem Hintergrund die Frage, wie unsere
Workshops und Vermittlungsangebote zu Informationskompetenz gestaltet
sein müssen, um Schüler*innen zu unterstützen, eine herausragende
vorwissenschaftliche Arbeit zu schreiben, kommen wir klar zu folgender
Antwort: Es sind nicht die funktionalen Fertigkeiten, wie
Suchworte-Formulieren oder In-der-Suchmaschine-Recherchieren. Vielmehr
braucht es kritisch-reflektierende Kompetenzen der jungen Forschenden.
Sie müssen Interessenlagen, Machtstrukturen und Wissensasymmetrien in
der Auswahl und Analyse der Quellen erkennen, bewusst damit umgehen und
nicht zuletzt auch sich selbst im Forschungsprozess verorten.\footnote{Vergleiche:
  Slezak, Gabriele: \enquote{C3-Award: Jugendliche forschen zu
  Entwicklung} In: Gmainer-Pranzl, Franz; Rötzer, Anita (Hg.) (2020):
  Shrinking Spaces. Mehr Raum für globale Zivilgesellschaft. Frankfurt
  a.M.: Peter Lang Ltd.~S. 293 f. DOI:
  \url{https://doi.org/10.3726/b17612}} Diese kritisch-reflektierenden
Aspekte der Informationskompetenz sind es, die die Grundlage bilden, um
komplexe, vernetzte Sachverhalte erfolgreich zu bearbeiten.

Eine solche kritische Informationskompetenz berücksichtigt soziale,
politische und wirtschaftliche Machtverhältnisse als Rahmen für
Wissensproduktion, Informationsverteilung und -zu\-gang.\footnote{Vergleiche:
  Gregory, Lua; Higgins, Shana (Hg.) (2013): Information literacy and
  social justice. Radical Professional Praxis. Sacramento: Library Juice
  Press. S. 4.} Dabei muss sich die \enquote{globale
Informationsungleichheit} \footnote{Kutner, Laurie: \enquote{Rethinking
  Information Literacy in a Globalized World} In: Communications in
  Information Literacy, Vol. 6, Iss. 1 (2012), S. 28. DOI:
  \url{https://doi.org/10.15760/comminfolit.2012.6.1.115}} bewusst
gemacht werden und eine Reflexion darüber erfolgen, wessen Perspektiven
und Stimmen gehört und welche nicht gehört werden, und wer für wen
spricht.\footnote{Vergleiche: Andreotti, Vanessa: \enquote{(Towards)
  decoloniality and diversality in global citizenship education.} In:
  Globalisation, Societies and Education. 9:3-4 (2011). S. 381--197.
  DOI: \url{https://doi.org/10.1080/14767724.2011.605323}} In einer
komplexen, dynamischen und global-vernetzten Welt ist eine derartige
kritische Informationskompetenz der Schlüssel zu einer engagierten
Bürger*innenschaft. Die Vermittlung von Informationskompetenz verfügt so
gesehen über eine transformatorische Komponente: Sie ist Werkzeug zur
Stärkung eines breiten globalen Engagements. Sie ermöglicht ein
Engagement, das sich manipulativen, entmenschlichenden und
kolonisierenden Prinzipien widersetzen kann. \footnote{Vergleiche: Kos,
  Denis, Špiranec, Sonja: \enquote{Debating Transformative Approaches to
  Information Literacy Education: A Critical Look at the Transformative
  Learning Theory.} In: Kurbanoğlu, Serap et al.~(Hg.) (2014):
  Information Literacy. Lifelong Learning and Digital Citizenship in the
  21st Century. ECIL 2014. Communications in Computer and Information
  Science, ECIL2014. Cham: Springer. S. 428. DOI:
  \url{https://doi.org/10.1007/978-3-319-14136-7_45}}

Es liegt dementsprechend nahe, die Vermittlung von kritischer
Informationskompetenz auch mit Rassismuskritik zu verbinden. Daher haben
wir unsere Aktivitäten im Bereich der dekolonialen Bibliotheksarbeit in
unser Informationskompetenzangebot integriert. Konkret wurde ein eigener
Online-Workshop erstellt, der genau diese Brücke schlagen soll. Der
Workshop \emph{\enquote{Kolonialismus. Auswirkungen im Heute verstehen
und Rassismus erkennen}} ist ein zweistündiges, modulares und
interaktives Online-Angebot für die Zielgruppe Schüler*innen der
10.--11. Schulstufe. Um die komplexe Materie innerhalb dieses Formates
zugänglich zu machen, wurde eine Aufteilung in vier Module gewählt.
Jedes Modul besteht aus einem Expert*innen-Input in Form eines vorab
produzierten Kurzvideos und einer darauffolgenden Interaktion, bei
welcher die Teilnehmer*innen in unterschiedlicher Art aktiv werden und
in Diskussionen einsteigen können.

Der Workshop startet mit einer Heranführung an das Themenfeld Rassismus
und Kolonialismus. Dafür erklärt die Journalistin und Aktivistin Vanessa
Spannbauer, warum die Auseinandersetzung mit Rassismus für alle
notwendig ist, bevor sie den Historiker Walter Sauer zu kolonialen
Spuren im öffentlichen Raum Österreichs interviewt. Nach diesem ersten
Modul, das Kolonialismus in der eigenen städtischen Umgebung verortet
und damit direkten Bezug zum Umfeld der Teilnehmer*innen herstellt,
taucht die Gruppe mit dem zweiten Modul tiefer in einen spezifischen
öffentlichen Raum ein, nämlich in die Medienwelt der Bibliothek. Darin
erklären Bibliothekar*innen, warum Quellenkritik notwendig ist und wie
sie rassismuskritisch funktionieren kann. Für dieses Modul wurden
exemplarische Werke aus dem Bestand der C3-Bibliothek für
Entwicklungspolitik ausgewählt. Mittels einer Whiteboard-Animation wird
ihre Untersuchung auf unterschiedliche Formen von Rassismus hin
vorgeführt -- das sind etwa Othering, Fremdbezeichnungen oder der
White-Savior-Komplex. Ziel dessen ist es, explizite wie implizite
Erscheinungsformen von Rassismus zu benennen und Schüler*innen damit
einen Anreiz zu geben, in Zukunft Medien, die sie selbst verwenden,
einer ähnlichen Prüfung zu unterziehen.\footnote{Der Videoinput zum
  Modul 2 \enquote{Quellenkritik: Koloniale Spurensuche in der
  Bibliothek} kann hier nachgesehen werden:
  \url{https://youtu.be/3iq6Av8DSAg}} Im dritten Modul geht es weiterhin
um Medien und deren kritische Betrachtung, jedoch mit einem anderen
Fokus. Ein Ausschnitt aus dem TED Talk \enquote{The Danger of the Single
Story} von Chimamanda Ngozi Adichie (2009)\footnote{Adichie, Chimamanda
  Ngozi (2009): The Danger of a Single Story. TEDGlobal 2009. Online
  verfügbar:
  \url{https://www.ted.com/talks/chimamanda_ngozi_adichie_the_danger_of_a_single_story?language=de}
  (letzter Aufruf: 23.06.2021).} wird verwendet, um zu veranschaulichen,
dass es dominante und marginalisierte Narrative gibt und wie diese
unsere Wahrnehmung der Welt prägen. Das abschließende Modul führt wieder
weg vom Schwerpunkt auf Informationskompetenz und hin zum Alltag von
jungen Menschen, die aktivistisch tätig sind und ihre Einsichten und
Erfahrungen zu Rassismus in Österreich teilen.

Für die Erstellung der Inputs war die Kooperation mit externen
Partner*innen, die sich selbst als Black and People of Colour (BPoC)
identifizieren beziehungsweise aus dem aktivistischen Bereich kommen,
absolut entscheidend. Ein mehrheitlich weißes Projektteam läuft
andernfalls Gefahr, Rassismuserfahrungen nicht gerecht zu werden und
Rassismen zu reproduzieren. Eine zentrale Rolle hat dabei Erem Celebi
gespielt, der 2019 mit dem C3-Award ausgezeichnet wurde und in weiterer
Folge als Praktikant Teil des Projektteams war. Er war vor allem im
vierten Modul involviert, wofür er als PoC drei andere BPoC interviewte.
Neben Studium und weiteren Jobs war es eine gewisse Herausforderung für
ihn, auch noch an diesem Projekt mitzuwirken. Er hat sich jedoch dafür
entschieden und brachte seine Perspektive mit viel Erfolg ein. Für das
Ergebnis des Workshops war es enorm bereichernd, dass er Einblick in
seine Erfahrungswelt und die seiner Freunde gewährte. Für ihn selbst war
es ebenfalls erkenntnisreich. Er nahm es als angenehm war, sich mit
nahen Freund*innen über Rassismus in Österreich zu unterhalten anstatt
in eine Diskussion mit Vertreter*innen der weißen Mehrheitsgesellschaft
einsteigen zu müssen. Sie alle sind sich einig, dass sie aktivistisch
gegen Rassismus auftreten, nicht aber als wandelndes Rassismuslexikon
zur Verfügung stehen möchten. Die Forderung, sich selbstständig zu
bilden und eigene Vorurteile aufzudecken, geht an alle Mitglieder der
Gesellschaft -- Ressourcen gibt es genug und mit ihrem eindrücklichen
Video nun noch eine weitere.\footnote{Der Videoinput zum Modul 4
  \enquote{Kritisch beleuchtet: Rassismus in Österreich heute} kann hier
  nachgesehen werden: \url{https://youtu.be/SRz7DvSFTP0}}

\hypertarget{ausblick}{%
\section{Ausblick}\label{ausblick}}

Die Erfahrungen der C3-Bibliothek seit 2018 zusammenfassend verstehen
wir die Dekolonialisierung von Bibliotheken als Prozess, der aktiv
gestaltet werden muss.

Es hat sich gezeigt, dass ein Kompetenzaufbau für eine inhaltliche
Auseinandersetzung mit Rassismen in überwiegend weißen Institutionen
notwendig ist. Dazu gehört neben dem Erwerben von theoretischem Wissen
auch die kritische, selbstreflexive Auseinandersetzung mit rassistischen
Denkweisen.

Über diesen internen Prozess hinaus -- oder vielmehr als zentraler
Bestandteil dessen -- braucht es aber auch eine diskursive und
partizipative Ausgestaltung des Prozesses mit breiter Einbindung der
Akteur*innen. Dabei gilt es auf der Ebene der Nutzer*innen und
Bürger*innen eine direkte und offene Thematisierung zu initiieren, die
aktive Teilhabe und Engagement aller an der Debatte fördert.

Eine Schlüsselrolle kommt dabei der kritischen Informationskompetenz zu,
die dazu befähigt, sich manipulativen, entmenschlichenden und
kolonisierenden Prinzipien zu widersetzen.

In diesem Sinne stellt die Entwicklung solcher transformatorischer
Bildungsangebote als Teil dekolonisierender Praktiken in Bibliotheken
ein zentrales Element dar.

\hypertarget{literatur}{%
\section{Literatur}\label{literatur}}

Adichie, Chimamanda Ngozi (2009): The Danger of a Single Story.
TEDGlobal 2009. Online verfügbar:
\url{https://www.ted.com/talks/chimamanda_ngozi_adichie_the_danger_of_a_single_story?language=de}
(letzter Aufruf: 06.12.2021).

Andreotti, Vanessa: \enquote{The Question of the \enquote*{Other} in
Global Citizenship Education. A Postcolonial Analysis of Telling Case
Studies in England.} In: Shultz, Lynette et al.~(Hg.) (2011): Global
Citizenship Education in Post-Secondary Institutions. Theories,
Practices, Policies. New York: Peter Lang. S. 140--157.

Andreotti, Vanessa: \enquote{(Towards) decoloniality and diversality in
global citizenship education.} In: Globalisation, Societies and
Education, 9:3-4 (2011) S. 381--197. DOI:
\url{https://doi.org/10.1080/14767724.2011.605323}

Andreotti, Vanessa et al.: \enquote{Mapping interpretations of
decolonization in the context of higher education.} In: Decolonization.
Indigeneity, Education \& Society Vol. 4, No.~1 (2015) S. 21--40.

Brunner, Claudia (2020). Epistemische Gewalt. Wissen und Herrschaft in
der kolonialen Moderne. Bielefeld: transcript.
\url{https://doi.org/10.14361/9783839451311}

Franke, Fabian: \enquote{Die Förderung von Informationskompetenz ist
Kernaufgabe von Bibliotheken -- und nicht nur der Senf zur Bratwurst!}
In: o-bib. Das offene Bibliotheksjournal. Bd. 4 Nr. 1 (2017), S. IV--V.
DOI: \url{https://doi.org/10.5282/o-bib/2017H1}

Gregory, Lua; Higgins, Shana (Hg.) (2013): Information literacy and
social justice. Radical Professional Praxis. Sacramento: Library Juice
Press.

Hapke, Thomas: \enquote{Informationskompetenz anders denken. Zum
epistemologischen ern von \enquote*{information literacy}.} In:
Sühl-Strohmenger, Wilfried (Hg.) (2016): Handbuch Informationskompetenz.
2. Auflage. Berlin {[}u.a.{]}: De Gruyter.

Klingenberg, Andreas; Deutscher Bibliotheksverband e. V. (2016):
Referenzrahmen Informationskompetenz. Online verfügbar unter:
\href{https://web.archive.org/web/20201125044717/https://www.bibliotheksverband.de/fileadmin/user_upload/Kommissionen/Kom_Infokompetenz/2016_11_neu_Referenzrahmen-Informationskompetenz_endg__2__Kbg.pdf}{https://web.archive.org/web/20201125044717/\linebreak\-https://www.bibliotheksverband.de/fileadmin/user\_upload/Kommissionen/Kom\_Info\linebreak\-kompetenz/2016\_11\_neu\_Referenzrahmen-Informationskompetenz\_endg\_\_2\_\_Kbg.pdf}
(letzter Aufruf: 30.06.2021; Anmerkung Redaktion: Link aktualisiert am
07.12.2021).

Kos, Denis, Špiranec, Sonja: \enquote{Debating Transformative Approaches
to Information Literacy Education: A Critical Look at the Transformative
Learning Theory.} In: Kurbanoğlu, Serap et al.~(Hg.) (2014): Information
Literacy. Lifelong Learning and Digital Citizenship in the 21st Century.
ECIL 2014. Communications in Computer and Information Science, ECIL2014.
Cham: Springer. S.427--435. DOI:
\url{https://doi.org/10.1007/978-3-319-14136-7_45}

Kutner, Laurie: \enquote{Rethinking Information Literacy in a Globalized
World} In: Communications in Information Literacy, Vol. 6, Iss. 1
(2012), S.24--33. DOI:
\url{https://doi.org/10.15760/comminfolit.2012.6.1.115}

Landry, Donna et al.~(Hg.) (1996): The Spivak reader. Selected works of
Gayatri Chakravorty Spivak. London: Routledge.

Mbembe, Achille (2011): On the postcolony. Berkely: Universtiy of
California Press.

Mecheril, Paul; Klingler, Birte: \enquote{Universität als transgressive
Lebensform. Anmerkungen, die gesellschaftliche Differenz- und
Ungleichheitsverhältnisse berücksichtigen.} In: Darowska, Lucyna et al.
(Hg.) (2010): Hochschule als transkultureller Raum? Kultur, Bildung und
Differenz in der Universität. Bielefeld: transcript. S. 83--116.

Mecheril, Paul et al.~(2013): Differenz unter Bedingungen von Differenz.
Zu Spannungsverhältnissen universitärer Lehre. Wiesbaden: Springer.

Osterhammel, Jürgen (1995): Kolonialismus. Geschichte, Formen, Folgen.
München: Beck.

Ruscher, Andrea; Schmelzer, Sarah; Baumgartner, Dani; Slezak, Gabi:
\enquote{Rassismen in Bibliotheksbeständen. Im Spannungsfeld zwischen
Sammelauftrag und Bildungsarbeit}. In: Köstner-Pemsel, Christina et al.
(Hg.) (2020): Künstliche Intelligenz in Bibliotheken. 34.
Österreichischer Bibliothekartag Graz 2019. Graz: Uni-Press Graz. S.
339--351.

Said, Edward (1978): Orientalism. New York: Pantheon Books.

Slezak, Gabriele: \enquote{C3-Award: Jugendliche forschen zu
Entwicklung} In: Gmainer-Pranzl, Franz; Rötzer, Anita (Hg.) (2020):
Shrinking Spaces. Mehr Raum für globale Zivilgesellschaft. Frankfurt
a.M.: Peter Lang Ltd. DOI: \url{https://doi.org/10.3726/b17612}

%autor
\begin{center}\rule{0.5\linewidth}{0.5pt}\end{center}

\textbf{Gabriele Slezak} studierte Afrikawissenschaften mit Schwerpunkt
Soziolinguistik (Dr.in) an der Universität Wien, der Université de
Ouagadougou und der Universität Bayreuth. Seit 1997 lehrt und forscht
sie an der Universität Wien mit einem Schwerpunkt im Bereich
Mehrsprachigkeit in institutionellen Kontexten, Bildungssysteme und
transdisziplinäre Forschung in Westafrika. Seit 1998 arbeitet sie in der
ÖFSE, aktuell als Leitung der Öffentlichkeitsarbeit und
Wissenschaftskommunikation.

\textbf{Sarah Schmelzer} studierte Slawistik und Literaturwissenschaft
(Mag.a) und absolvierte ein postgraduales Studium Bibliotheks- und
Informationsmanagement an der Donau Universität Krems (MSc). Nach
verschiedenen beruflichen Stationen, darunter die Leitung der Bibliothek
des Goe\-the-Instituts in St.~Petersburg, verantwortet sie seit 2012 den
Bereich Bibliothek in der ÖFSE.

\textbf{Andrea Ruscher} studierte Geschichte (BA) und Global Studies
(MA) an der Universität Wien und arbeitet seit April 2019 im Bereich
Bibliothek der ÖFSE in der C3-Bibliothek für Entwicklungspolitik. In
Rahmen dieser Tätigkeit bearbeitete sie ein historisches
Bestandssegments exemplarisch und untersuchte darin präsente Rassismen
im historischen Kontext.

\textbf{Erem Celebi} studiert zurzeit Molekulare Biotechnologie (BSc) am
FH-Campus Wien. Nachdem er den C3-Award im Jahr 2019 für seine VwA zum
Thema \enquote{Die Frau als Ware -- Frauenhandel im 21. Jahrhundert} gewann,
konzipierte er im Rahmen eines C3-Projekts ein Modul zum Thema
\enquote{Rassismen in Österreich}.

\textbf{Dani Baumgartner} studierte Soziologie (BA) und Gender Studies
und absolvierte 2017/2018 den ULG Library and Information Studies an der
ÖNB. Seit 2016 arbeitet Dani Baumgartner für die Frauen*solidarität in
der C3-Bibliothek für Entwicklungspolitik.

\textbf{Jonas Paintner} studiert(e) Publizistik (Bakk.phil.) sowie
Internationale Entwicklung in Wien und Global Political Economy (MA) in
London. Nach Tätigkeit in der Wissenschaftskommunikation der ÖFSE,
freiberuflicher rassismuskritischer Bildungsarbeit und einem Lehrauftrag
zu Migrationspädagogik an der Universität Krems forscht er derzeit im
Postgraduiertenprogramm des Deutschen Instituts für Entwicklungspolitik
(DIE) zu Modalitäten transnationaler Wissenskooperationen.

\end{document}
