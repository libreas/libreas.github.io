\textbf{Kurzfassung:} Dieser Artikel thematisiert den Umgang von Bibliotheken mit
Medien, deren Inhalte Rassismen enthalten, exemplarisch anhand des
rassistischen Kinderbuches \enquote{Hatschi Bratschis Luftballon} von Franz
Karl Ginzkey in den Büchereien Wien und der Fachbereichsbibliothek für
Germanistik, Nederlandistik und Skandinavistik. Die Grundlage der
Untersuchung bilden zwei Expert:inneninterviews, die mit Vertreter:innen
der beiden Institutionen geführt wurden. Ziel des Artikels ist es, die
unterschiedlichen Argumentationen der beiden Bibliotheken für die
Aufnahme des Kinderbuchs in den jeweiligen Bestand darzustellen. Die
Gegenüberstellung der Argumentationen beider Bibliotheksarten -- bei den
Büchereien Wien handelt es sich um eine Öffentliche und bei der
Fachbereichsbibliothek um eine Wissenschaftliche Bibliothek -- zeigen,
dass diese aufgrund von Bestandspolitik und Zielpublikum divergieren,
zum Teil aber mit ähnlichen Fragen und Herausforderungen konfrontiert
sind.''

\begin{center}\rule{0.5\linewidth}{0.5pt}\end{center}

\noindent \textbf{Abstract:}  In this article the handling of libraries with information
sources that contain racism is examined by the racist children's book
\enquote{Hatschi Bratschis Luftballon} by Franz Karl Ginzkey in the Büchereien
Wien and the Fachbereichsbibliothek für Germanistik, Nederlandistik und
Skandinavistik. Two expert interviews which were conducted with
representatives of both institutions provide the basis for this
analysis. The purpose of this article is to describe the different
argumentations of the libraries to include the children's book into
their library collections. The comparison of the arguments of both types
of libraries -- the Büchereien Wien are public and the
Fachbereichsbibliothek is a scientific library -- show that they diverge
due to inventory policy and their target audience, however, they are
confronted with similar questions and challenges.''
