\documentclass[a4paper,
fontsize=11pt,
%headings=small,
oneside,
numbers=noperiodatend,
parskip=half-,
bibliography=totoc,
final
]{scrartcl}

\usepackage[babel]{csquotes}
\usepackage{synttree}
\usepackage{graphicx}
\setkeys{Gin}{width=.4\textwidth} %default pics size

\graphicspath{{./plots/}}
\usepackage[ngerman]{babel}
\usepackage[T1]{fontenc}
%\usepackage{amsmath}
\usepackage[utf8x]{inputenc}
\usepackage [hyphens]{url}
\usepackage{booktabs} 
\usepackage[left=2.4cm,right=2.4cm,top=2.3cm,bottom=2cm,includeheadfoot]{geometry}
\usepackage{eurosym}
\usepackage{multirow}
\usepackage[ngerman]{varioref}
\setcapindent{1em}
\renewcommand{\labelitemi}{--}
\usepackage{paralist}
\usepackage{pdfpages}
\usepackage{lscape}
\usepackage{float}
\usepackage{acronym}
\usepackage{eurosym}
\usepackage{longtable,lscape}
\usepackage{mathpazo}
\usepackage[normalem]{ulem} %emphasize weiterhin kursiv
\usepackage[flushmargin,ragged]{footmisc} % left align footnote
\usepackage{ccicons} 
\setcapindent{0pt} % no indentation in captions

%%%% fancy LIBREAS URL color 
\usepackage{xcolor}
\definecolor{libreas}{RGB}{112,0,0}

\usepackage{listings}

\urlstyle{same}  % don't use monospace font for urls

\usepackage[fleqn]{amsmath}

%adjust fontsize for part

\usepackage{sectsty}
\partfont{\large}

%Das BibTeX-Zeichen mit \BibTeX setzen:
\def\symbol#1{\char #1\relax}
\def\bsl{{\tt\symbol{'134}}}
\def\BibTeX{{\rm B\kern-.05em{\sc i\kern-.025em b}\kern-.08em
    T\kern-.1667em\lower.7ex\hbox{E}\kern-.125emX}}

\usepackage{fancyhdr}
\fancyhf{}
\pagestyle{fancyplain}
\fancyhead[R]{\thepage}

% make sure bookmarks are created eventough sections are not numbered!
% uncommend if sections are numbered (bookmarks created by default)
\makeatletter
\renewcommand\@seccntformat[1]{}
\makeatother

% typo setup
\clubpenalty = 10000
\widowpenalty = 10000
\displaywidowpenalty = 10000

\usepackage{hyperxmp}
\usepackage[colorlinks, linkcolor=black,citecolor=black, urlcolor=libreas,
breaklinks= true,bookmarks=true,bookmarksopen=true]{hyperref}
\usepackage{breakurl}

%meta
%meta

\fancyhead[L]{M. Strickert\\ %author
LIBREAS. Library Ideas, 40 (2021). % journal, issue, volume.
\href{https://doi.org/10.18452/23807}{\color{black}https://doi.org/10.18452/23807}
{}} % doi 
\fancyhead[R]{\thepage} %page number
\fancyfoot[L] {\ccLogo \ccAttribution\ \href{https://creativecommons.org/licenses/by/4.0/}{\color{black}Creative Commons BY 4.0}}  %licence
\fancyfoot[R] {ISSN: 1860-7950}

\title{\LARGE{Zwischen Normierung und Offenheit -- Potenziale und offene Fragen bezüglich kontrollierter Vokabulare und Normdateien}}% title
\author{Moritz Strickert} % author

\setcounter{page}{1}

\hypersetup{%
      pdftitle={Zwischen Normierung und Offenheit – Potenziale und offene Fragen bezüglich kontrollierter Vokabulare und Normdateien},
      pdfauthor={Moritz Strickert},
      pdfcopyright={CC BY 4.0 International},
      pdfsubject={LIBREAS. Library Ideas, 40 (2021)},
      pdfkeywords={Bibliothek, Dekolonisierung, library, decolonization},
      pdflicenseurl={https://creativecommons.org/licenses/by/4.0/},
      pdfcontacturl={http://libreas.eu},
      baseurl={https://doi.org/10.18452/23807},
      pdflang={de},
      pdfmetalang={de}
     }
	 



\date{}
\begin{document}

\maketitle
\thispagestyle{fancyplain} 

%abstracts
\begin{abstract}
\noindent
Der Rückgriff auf kontrollierte Vokabulare und Normdaten
ist im Zuge der Zugänglichmachung von analogen und digitalen Ressourcen
zentral. Der vorliegende Text diskutiert (ethisch-theoretische)
Fragestellungen im Umgang mit kontrollierten Vokabularen als spezifisch
eurozentrische Wissensordnungen.
\end{abstract}

%body
Bei der umfassenden Zugänglichmachung von analogen und digitalen
Ressourcen ist die Erschließung des Materials mittels Metadaten von
zentraler Bedeutung. Ein Teilbereich dessen stellt die Sacherschließung
dar, bei der die Ressourcen auf Grundlage von inhaltlich-thema\-tischen
Gesichtspunkten erschlossen werden. Mittels Rückgriff auf kontrollierte
Vokabulare und Normdaten im Erschließungsprozess ist eine tiefergehende
Recherche möglich, da unter anderem auch Synonyme gefunden werden und
Mehrsprachigkeit mitunter berücksichtigt werden kann.

Der vorliegende Text~diskutiert ethisch-theoretische Fragestellungen im
Umgang mit diesen Wissensordnungen, die durch ihren Inhalt und ihre
Struktur/Konzeption eine eurozentrische Ausrichtung aufweisen. Dieser
Artikel kann in der gegebenen Kürze nur eine Bestandsaufnahme der
Situation, Problemfelder und ersten Lösungsansätzen bieten und versteht
sich daher als Aufschlag zu einer daran anknüpfenden Diskussion, die wir
im Fachinformationsdienst Sozial- und Kulturanthropologie (FID SKA) und
dem Netzwerk Koloniale Kontexte gerne weiterführen möchten. Der Thematik
wird sich aus Sicht des FID SKA genähert, welcher sich in der laufenden
Projektphase mit einer Anreicherung und Aktualisierung der
deutschsprachigen Gemeinsamen Normdatei (GND) um ethnologische Begriffe
und Personen befasst. Der Autor dieses Artikels leitet das Teilprojekt
zur GND, das darauf abzielt, den Status quo zu erfassen, Leerstellen,
fehlende Verknüpfungen sowie veraltete oder problematische Begriffe zu
identifizieren und im nächsten Schritt weitergehend zu
bearbeiten.\footnote{Einen Überblick über dahingehende Handlungsfelder
  liefert Nina Frank in ihrer Abschlussarbeit am Institut für
  Bibliotheks- und Informationswissenschaft der Humboldt-Universität
  (Frank 2007). Einige Änderungsvorschläge sind mittlerweile durch die
  GND-Kooperative umgesetzt worden.}

\hypertarget{i.}{%
\subsubsection{I.}\label{i.}}

Im Erschließungsprozess sollen sowohl analoge als auch digitale
Ressourcen bestenfalls mittels strukturierter Metadaten und
kontrollierter Vokabulare beschrieben werden. Dies kann sowohl auf
formaler Ebene durch Nennung der Verfasser*innen, des Titels, des
Verlags et cetera geschehen oder auf inhaltlicher Ebene, indem bestimmte
Sachverhalte mittels Schlagwörtern kenntlich gemacht werden. Wurde
hierbei lange Zeit auf intellektuelle Erschließung durch
Bibliothekar*innen und Dokumentar*innen zurückgegriffen, kommen
zusehends maschinelle Ansätze zum Einsatz beziehungsweise werden weiter
ausgebaut.\footnote{Die Deutsche Nationalbibliothek hat im Jahr 2010
  begonnen, den stetig wachsenden Posten der digitalen Medienwerke durch
  maschinelle Erschließungsverfahren zu bearbeiten. Seit 2017 werden
  auch zusehends physische Medien auf diese Weise erschlossen (Mödden et
  al.~2018, S. 30).} Material, das nicht in textueller Form vorliegt,
wie zum Beispiel Bilder, muss erst mit Metadaten beschrieben werden, um
recherchierbar zu werden.

Während der Rückgriff auf unkontrollierte Stichwörter bei der
(Volltext-) Recherche nur die Begrifflichkeiten findet, welche die
Verfasser*innen oder die Suchenden verwendet haben, ermöglichen
normierte Schlagwörter präzisere, sichere und nachvollziehbare
Ergebnisse.\footnote{Ein Problem bei der Verwendung freier Stichwörter
  durch Autor*innen ist, dass Suchende nur mit exakt denselben Begriffen
  Sucherfolge erzielen können. Synonyme, Homonyme (zum Beispiel Bank),
  Pluralformen und andere Abweichungen führen dann nicht zum Titel. Eine
  weitere Schwierigkeit entsteht bei der Suche in anderen Sprachen und
  bei unterschiedlichen Schreibweisen (Beall 2008, 439 ff.).} Dabei wird
auf kontrollierte Vokabulare zurückgegriffen, die eine konsistente
Vergabe von Schlagwörtern gewährleisten sollen. Darin enthaltene
Begrifflichkeiten durchlaufen eine terminologische Kontrolle, die dabei
auf die Mehrdeutigkeit von Begriffen reagiert, Schreibweisen festlegt
und definiert, wie Sachverhalte bevorzugt bezeichnet werden sollen.
Hinterlegte Synonyme können auf die festgelegte Vorzugsbenennung
verweisen und liefern bei der Recherche ebenfalls
Suchtreffer.\footnote{Grundsätzlich liefert der Rückgriff auf
  kontrollierte Schlagwörter präzisere und umfassendere Suchergebnisse
  durch eine erhöhte Beschreibungsqualität und zusätzliche
  Sucheinstiege. Gross et al. beschreiben in ihrer Metastudie über die
  Auswirkung des kontrollierten Vokabulars auf die Ergebnisse der
  Schlagwortsuche, dass durchschnittlich etwa ein Drittel der Treffer
  bei der Schlagwortsuche verloren gehen würden, wenn diese aus den
  Katalogeinträgen entfernt oder nicht mehr enthalten wären (Gross et
  al.~2015, S. 31).} Insbesondere bei der Erschließung großer
Datenmengen, beispielsweise durch automatisierte maschinelle Verfahren,
ist die Bezugnahme auf kontrolliertes Vokabular unersetzlich.\footnote{Die
  automatisierte Erschließung basiert darauf, dass nach der
  Identifikation einzelner relevanter Themen/Termini bei der Suche nach
  passenden Annotationsbegriffen auf kontrollierte Vokabulare
  beziehungsweise Normdateien zu Abgleichszwecken zurückgegriffen wird.
  Für eine detailliertere Verfahrensübersicht siehe Uhlmann (2013).}

In Normdateien gesammelte Normdaten sind bedeutsam, wenn Entitäten
eindeutig identifiziert und durch eine kontrollierte Schreibweise
benannt werden sollen.\footnote{Entität meint in diesem Kontext eine
  Informationseinheit, die zweifelsfrei zu identifizieren ist, wie die
  Bezeichnung für einen Sachverhalt. Entitäten können durch
  Begriffsdefinitionen einschließlich Quellenangaben und eindeutige
  Kennnummern, in Form von persistenten Identifikatoren (zum Beispiel
  Digital Object Identifier) ergänzt werden.} Durch ergänzende
Regelwerke gelingt es, ein hohes Maß an Standardisierung bei der
Datenerfassung und dem Datenaustausch sicherzustellen. Material, dass
auf Normdaten zurückgreift, kann im Anschluss potenziell mit anderen
Datenbeständen in einem Netz verknüpft werden (linked data). Das auf
dieser Basis entstehende semantische Netzwerk besitzt das Potenzial,
derzeit noch voneinander getrennte Bestände übergreifend
zusammenzuführen.\footnote{Das Semantic Web als ausgedehntes Geflecht
  von Datensätzen und anderen Webressourcen erlaubt die Verknüpfung von
  Material aus verschiedenen Wissensdomänen, das auch von Maschinen
  verarbeitet werden kann. Dadurch ist eine bessere Interaktion zwischen
  Mensch und Computer möglich.} Damit kann Material für eine
umfassendere Nutzung zugänglich gemacht und neue Zugänge zu Wissen
geschaffen werden.

Im deutschsprachigen Raum ist insbesondere die Gemeinsame Normdatei
(GND) ein zentraler Referenzpunkt für die Erschließung von Ressourcen.
Sie ist ein Zusammenschluss aus verschiedenen bereits zuvor existenten
Normdateien und wird von den deutschsprachigen Bibliotheksverbünden
redaktionell gepflegt. Sie beinhaltet mehr als neun Millionen
Datensätze, die normierte Ansetzungen unter anderem für Geografika,
Personen, Sachschlagwörter umfassen, die bis dato zumeist zur
bibliothekarischen Medienkatalogisierung genutzt wurden. Zunehmend
findet sie eine umfassendere Anwendung zur Kulturgutvernetzung in
Archiven, Museen und verschiedenen Projekt- und Webkontexten. Im Zuge
von Mapping-Projekten wurde die GND bereits mit anderen Normdateien,
beispielsweise der Library of Congress, auf Übereinstimmungen untersucht
und verknüpft. Dadurch sind mitunter auch Materialien, die mit
englischsprachigen Schlagwörtern erschlossen wurden, durch die
Verwendung deutschsprachiger GND-Schlagwörter in Katalogen zu finden.

\hypertarget{ii.}{%
\subsubsection{II.}\label{ii.}}

Viele kontrollierte Vokabulare und Normdateien sind in ihrer Struktur
und in ihrem Umfang durch eine Sichtweise auf die Welt geprägt, die in
ihrem Produktionsprozess im Globalen Norden begründet liegt. Die
Notwendigkeit von bevorzugten Ansetzungsformen als auch der
Erstellungsprozess führen dazu, dass bestimmte Begriffe und Definitionen
bevorzugt werden, beispielsweise dadurch, dass das jeweilige (nationale)
Publikationsaufkommen und Referenzeinträge in Nachschlagewerken als
Erstellungsgrundlage dienten. Die vorgeblich neutrale Bezeichnung von
Sachverhalten und das Kategoriensystem darin sind Resultat von
spezifischen, sich in Regeln manifestierenden Entscheidungen und somit
immer ideologisch-historisch eingebunden (Gartner 2016, S. 42).

Die GND basierte lange auf dem Prinzip des literary warrant
(Publikationsaufkommen), was bedeutet, dass Begriffe angelegt wurden,
wenn diese zur Beschreibung von vorliegendem Material notwendig waren.
Diese Bezugnahme auf das Publikationsaufkommen, bibliothekarische
Regelwerke, aber auch der Ressourcenmangel für eine dauerhafte,
fallspezifische Datenpflege und darüber hinausgehende Anreicherung
führten zu Lücken innerhalb dieser Normdatei. Vollständigkeit und
Abgeschlossenheit sind in diesem Zusammenhang keine realistischen Ziele.
Es handelt sich vielmehr um einen dynamischen Prozess, in dem eher die
Frage bedeutsam ist, wie eine grundsätzliche und niedrigschwellige
Offenheit sowohl bezüglich Änderungen und der Aufnahme von neuen
Begriffen als auch Anschlussfähigkeit an andere Wissenssysteme erreicht
werden kann. Diese offene Lösung sollte jedoch nicht auf die Vorteile
der Standardisierung verzichten. Innerhalb vieler
Wissensorganisationssysteme herrscht trotz universellem Anspruch ein
Mangel an Diversität beziehungsweise Multiperspektivität und
Flexibilität. Benennung, beispielsweise im Erschließungsprozess, schafft
einerseits Sichtbarkeit aber gleichzeitig auch Differenz, indem
bestimmte Sachverhalte besonders hervorgehoben werden. Vielfach bleibt
das \enquote{Selbstverständliche} (zum Beispiel weiß, europäisch,
männlich, christlich) als Norm unmarkiert, wohingegen die Abweichung
davon explizit sichtbar gemacht wird (Adler 2016, S. 632; Drabinski
2013, S. 97).\footnote{Für eine historische Einordnung von
  Erschließungspraktiken sowie Wissensorganisationssystemen und der
  Machtverwobenheit von Benennungs- und Klassifikationspraktiken sind
  die Arbeiten von Hope Olson aufschlussreich (Olson 2002, 2004). Das
  Themenfeld Rassismus und dessen Verknüpfung zu
  anthropologisch-evolutionären Theorien in Klassifikationssystemen
  arbeitet unter anderem Melissa Adler in ihrem Artikel:
  \enquote{Classification along the Color Line: Excavating Racism in the
  Stacks} näher aus (Adler 2017a). Siehe in Bezug auf Fragen der
  Geschlechterverhältnisse und -stereotype sowie der expliziten Nennung
  von Frauen und Männern beziehungsweise geschlechtergerechte
  Formulierungen und was diese für präzise Suchanfragen bedeuten die
  Arbeiten von Karin Aleksander (2014) und Sandra Sparber (2016).
  Bezüglich des Themenfelds Sexualität sei an dieser Stelle die Arbeit
  von Samuel J. Edge (2019) und für den deutschsprachigen Raum auf den
  Vortrag von Hannes Hacke: \enquote{Schlagwörter, Normdaten, Tags --
  Sexualitätsbezogenes Vokabular in Sammlungsdatenbanken}, den er im
  Februar 2021 im Deutschen Hygiene-Museum Dresden gehalten hat,
  hingewiesen. Zu finden unter:
  \url{https://www.youtube.com/watch?v=Gc-Ntsg-0m0}} Ein Teilbereich
davon ist, was Olson als \enquote{error of faulty generalization}
bezeichnet: \enquote{Books that do not specify their particularity, even
if they include only male examples, are still taken to be about the
topic in general.} (Olson 2002, S. 156) Gleichzeitig werden bestimmten
Subjektpositionen und -perspektiven und Sachverhalten, beispielsweise
durch den Fokus auf einen bestimmten Ausschnitt des
Publikationsaufkommens, mehr Aufmerksamkeit geschenkt als anderen, was
zu Lücken im Erschließungsvokabular führen kann.

In der Betrachtung sind zwei Ebenen zu unterscheiden: Es geht einerseits
um die Frage, inwiefern und bis zu welchem Grad Bias innerhalb eines
normierten Vokabulars aufgearbeitet werden kann. Andererseits steht die
Frage nach der grundsätzlichen Offenheit eines stark auf Normierung
basierenden Systems im Raum sowie der möglichen Reichweite von
Repräsentation und Vielstimmigkeit in solchen Systemen.

Die Thematisierung von problematischen Begriffen und Leerstellen
innerhalb von Normdateien und kontrollierten Vokabularen wird in den USA
seit den späten 1960er Jahren diskutiert (Drabinski 2013, S.
94).\footnote{Eine zentrale Intervention bezüglich problematischer
  Termini innerhalb der Library of Congress Subject Headings stellte das
  Buch \enquote{Prejudices and Antipathies. A Tract on the LC Subject
  Heads Concerning People} von Sandford Berman (1971) dar. Aber auch
  bezüglich einzelner Begriffe gibt es immer wieder Diskussionen. So gab
  es Veröffentlichungen zu den Begriffen \enquote{East Indians} (Biswas
  2018) und eine sehr breite und öffentlichkeitswirksame Debatte zur
  Ersetzung des Begriffs \enquote{Illegal Aliens} durch
  \enquote{Undocumented Immigrants} (Lo 2019; Aguilera 2016).} Die
Kritik beschränkt sich dabei nicht nur auf Normdateien, sondern umfasst
eine grundsätzlichere Betrachtung von verzerrten und problematischen
Erschließungspraktiken, die Ausschlüsse und Leerstellen insbesondere in
marginalisierten Wissensgebieten generieren.\footnote{Siehe hier
  exemplarisch den Artikel von Sharon Block in welchem sie in Hinblick
  auf die wichtige wissenschaftliche Datenbank JSTOR herausarbeitet, wie
  wiederholt Arbeiten zur Geschichte der Frauen, der afrikanischen
  Diaspora/Afrika, der amerikanisch-indigenen Bevölkerung und des
  Siedlerkolonialismus in der Verschlagwortung missrepräsentiert oder
  gänzlich ignoriert werden und somit der Zugang zu diesen Inhalten
  erschwert wird (Block 2020).}

Während sich ein Strang der Kritik auf die Korrektur von problematischen
Begriffen und Ergänzungen bei Leerstellen konzentriert, weisen andere
Autor*innen auf die grundsätzlichen Beschränkungen solcher
Wissensorganisationssysteme hin. Kategorienbildung gehe immer mit
Ausschlüssen einher und solche Systeme zeichneten sich vielfach durch
eine prinzipielle Unabgeschlossenheit aus (Adler 2017b, S. 157).
Drabinski hebt im Zuge dessen hervor, dass Verzerrungen in
bibliothekarischen Klassifikationsstrukturen und der Fachsprache aus
ihrer Perspektive Probleme sind, die dem Projekt der Wissensorganisation
selbst innewohnen. Sie können nicht abschließend und universell durch
vermeintlich korrektere Termini und einem mehr an Objektivität aufgelöst
werden, da Kategorien und Bezeichnungen immer kontingent und in Bewegung
sind (Drabinski 2013, S. 104). Es sei angebracht, veraltete Begriffe und
voreingenommene Beschreibungen zu bewahren, auch wenn sich die
Gesellschaft weiterentwickelt. Dies sei sinnvoll, damit die
Benutzer*innen verstehen können, wie sich die Sprache entwickelt, damit
Historizität und ideologische Eingebundenheit sichtbar bleiben und so
die Begrenzungen der Systeme und deren nur vermeintliche universelle
Objektivität den Nutzenden leichter vermittelt werden könnten (Drabinski
2013, S. 108).

Auch Turner weist darauf hin, dass das Fortbestehen von problematischen
Begrifflichkeiten innerhalb von Wissenssystemen zwar mitunter eine
\enquote{reperformance of the colonial encounter} (Turner 2020, S. 159)
darstellt, deren Existenz potenzielle Nutzende abschrecken könnten. Dies
muss jedoch nicht notwendigerweise bedeuten, dass die Informationen
gelöscht werden sollten. Sie sind mitunter notwendig, um auf diese
kolonialen Kategorien hinzuweisen und sie kontinuierlich in Frage zu
stellen, da sie andernfalls völlig getilgt und historisch unsichtbar
werden könnten. Gleichzeitig sind dies historische Spuren, die für viele
Nutzungsszenarien bedeutsam sein können. Das Abwägen zwischen
öffentlicher Zugänglichkeit und der Konservierung eines historischen
Aufzeichnungskontextes auf der einen Seite und ethischen Fragen auf der
anderen Seite, ist spannungsreich. Turner kommt zu dem Ergebnis:
\enquote{In a very real sense, there is no \enquote*{best of both
worlds} solution; there is only proper education about why the museum
maintains old terms and why they might be useful, along with the caveats
associated with this information.} (Turner 2020, S. 171)

\hypertarget{iii.}{%
\subsubsection{III.}\label{iii.}}

In dieser Auseinandersetzung spielen zugleich Fragen nach Fremd- und
Selbstbezeichnungen zum Beispiel von Gruppen,\footnote{Innerhalb der GND
  gibt es über 4000 Datensätze, die Bevölkerungsgruppen bezeichnen,
  welche unter der Systematikstelle Volks- und Völkerkunde zu finden
  sind. Trotz andauernder Umarbeitungen finden sich noch unkommentierte,
  veraltete und wertende Bezeichnungen (bei Hauptansetzungen und
  Synonymen als auch bei den beschreibenden Kurztexten), die eine
  Revision nötig machen. Gruppenbezeichnungen im Definitionstext
  umfassen beispielsweise \emph{Negrito-Volk, Indianerstamm, Primitiver
  Stamm, Naturvölker}. Dies ist unter anderem begründet durch
  zugrundeliegende alte (fachliche) Nachschlagewerke, die Schwierigkeit
  passende Oberbegriffe zu finden und das grundsätzliche Problem für
  alle zufriedenstellende Bezeichnungen zu finden sowie der schieren
  Anzahl an Datensätzen.} aber auch Sprecher*innenpositionen und die
Historizität der Begriffe eine Rolle. Hierbei werden grundsätzliche
Aspekte postkolonialer Theoriebildung in Hinblick auf die Repräsentation
(kultureller) Differenz aufgerufen. Zum einen wird gefragt, ob
\enquote{eine nicht-gewaltvolle, nicht-reduktive Repräsentation des
Anderen überhaupt möglich ist} (Do Mar Castro Varela und Dhawan 2005, S.
50), zum anderen, inwiefern die Subjekte selbstständig sprechen
beziehungsweise sich selbst repräsentieren können oder ob dies für sie
getan wird (Do Mar Castro Varela und Dhawan 2005, 68 f.). Gomes und
Frota kommentieren den Mangel an Repräsentation und Einbezug differenter
Perspektiven in vielen dieser Wissenssysteme damit, dass diese auf einer
\enquote{single voice, that is, a single discourse, usually the
scientific-academic one} (Gomes und Frota 2019, S. 640) basierten. Dies
zieht mitunter Probleme bei der Nutzung nach sich, da Nutzende nicht mit
den Systemen kommunizieren könnten und nicht in das System miteinbezogen
beziehungsweise nicht von ihm repräsentiert würden (Gomes und Frota
2019, S. 641).

Grundsätzlich sind diese Fragen der Selbst- und Fremdbezeichnung,
Un-/Sichtbarmachung, falsche Schreibweisen und Interpretationen nicht
nur auf Gruppenbezeichnungen beschränkt, sondern stellen sich
beispielsweise auch im Hinblick auf Ortsbezeichnungen und an Stellen, an
denen vorhandene Daten durch Informationen aus differenten Perspektiven
angereichert werden sollen, zum Beispiel bei Objekten. So weisen Bishop
et al.~unter Bezugnahme auf den Historiker George R. Stewart darauf hin,
dass indigene Gruppen in Nordamerika dem, was beispielsweise in
offiziellen Karten oftmals nur als ein Fluss bezeichnet worden ist,
detailliertere und mehrfache Namen gegeben haben, weil mehr Bedeutungen
notwendig waren, um Untiefen, besseres Fischen und so weiter anzuzeigen
(Bishop et al.~2015, S. 204).\footnote{An dieser Stelle sei kurz auf
  zwei Projekte in diesem Bereich verwiesen: Der Indigenous Mapping
  Workshop unterstützt und erarbeitet zusammen mit indigenen Gruppen
  umfassendes Geodaten-Material, um auf diesem Weg deren Perspektiven,
  Rechte und Interessen zu unterstützen
  (\url{https://www.indigenousmaps.com/}).

  Ein anderes Kartenprojekt stellt Coming Home to Indigenous Place Names
  in Canada dar, welches Kartenmaterial auf Basis von indigenen
  Ortsnamen in Kanada erstellt. Auf diese Weise wird die indigene
  Benennungsautorität bezüglich Ortsnamen gewürdigt und Wissen mit
  Erlaubnis der First Nations-, Métis- und Inuit-Gemeinschaften geteilt.
  Die Ortsbezeichnungen markieren unter anderem Versammlungsplätze der
  Gemeinschaften und machen die indigene Präsenz in der kanadischen
  Landschaft mittels indigener Sprachen sichtbar
  (\url{https://umaine.edu/canam/publications/coming-home-map/}).}

Nicht immer kann eindeutig festgelegt werden, was richtig und was falsch
ist. Hier herrscht ein Spannungsverhältnis zwischen einer vermeintlichen
Universalsprache, die auf Wissensbeständen sowie den kulturellen
Blickwinkeln des Globalen Nordens basiert und in diesem Kontext
möglichst allgemeinverständlich sein soll, auf der einen Seite und einer
grundsätzlichen Perspektivität und Historizität von Wissen und Begriffen
aufder anderen Seite. Dies betrifft Fragen älterer Termini, die für die
Beschreibung von Sachverhalten nicht mehr gebräuchlich sind oder für
Begrifflichkeiten, die zwischen, aber auch innerhalb von Disziplinen
unterschiedlich verwendet werden und Abstimmung mittels Kooperationen
notwendig macht. In Hinblick auf Selbstbezeichnungen gibt es oftmals
verschiedene Ansätze, die von unterschiedlichen Gruppen unterstützt
werden und notwendige Entscheidungen nach sich ziehen, mit denen
mitunter nicht alle Involvierten zufrieden sein werden.

Ein permanenter Sprachwandel sorgt dafür, dass Änderungen nicht immer
zeitnah bearbeitet werden können und erfordert zugleich Überlegungen,
inwiefern iterative Anpassungsprozesse bei der Pflege der Datensätze
integriert werden können. Dies gilt umso mehr, da die Dauerhaftigkeit
der Normdateien dazu führt, dass Begriffe über lange Zeit vorgehalten
werden. So resümiert Drabinski: \enquote{Such corrections are always
contingent and never final, shifting in response to discursive and
political and social change.} (Drabinski 2013, S. 100)

Selbstverständlich sind diese Sachverhalte nicht nur ein Problem solcher
Wissensorganisationssysteme. Sie spiegeln vielmehr
gesamtgesellschaftliche Problemstellungen. Eine globalisierte, vernetzte
Welt führt dazu, diese Wissenssysteme und Zugänge, die in eine
spezifische Wissenskultur eingebettet sind, neu zu denken. Dies umfasst
mögliche Systemanpassungen oder die Schaffung gänzlich neuer Systeme mit
neuen Funktionalitäten. Auf Grundlage dieser Problemstellungen plädieren
Autor*innen dafür, ein höheres Maß an Teilhabe mittels kooperativer
Praktiken anzustreben. Dies kann eine bedarfsspezifische Anpassung von
kontrollierten Vokabularen, zum Beispiel durch das Einbinden von
Nutzer*innentermini darin, bedeuten (Olson 2002) oder gar die
Entwicklung vollkommen neuer kontrollierter Vokabulare, um so einen
Dialog zu ermöglichen, damit verschiedene Standpunkte und Anliegen
integriert werden können (Adler 2016, S. 635).

Als Ergänzung zum \enquote{literary and philosophical warrant} schlagen
Gomes und Frota deshalb einen sogenannten \enquote{cultural warrant}
vor, der abweichende Werte und Schwerpunkte in die Systeme aufnimmt.
Dieser Prozess bringt die Nutzenden näher an das Informationssystem, da
ihre entprechenden Interessen und Bedürfnisse reflektiert werden. Dies
könnte beispielsweise durch die Integration von diversen Quellen als
Basis für Begriffe unter anderem mittels Zeitungen und Magazinen,
Mitteilungen sozialer Bewegungen, Interviews oder auch
nicht-geschriebenen Materials geschehen (Gomes und Frota 2019, 642 ff.).

Duarte und Belarde-Lewis nennen in Bezug auf indigene Kontexte vier
Praktiken, die zur Marginalisierung anderer Wissensformen führen:

\begin{quote}
\enquote{(1) misnaming, or using Western-centric terms to describe
Indigenous phenomena; (2) using parts to describe a more holistic
phenomena, or the reduction, removal, and de-linking of a piece of a
knowledge system from a greater ontology; (3) emphasis on modern
nationalist periodization, inclusive of the notion that history as it is
written by the colonizers cannot be changed; and (4) emphasis on
prohibiting changes to practices that would upset the efficiency of the
existing standardized schema.} (Duarte und Belarde-Lewis 2015, S.
683--684)
\end{quote}

Für die Praxis bedeutet dies, sich darüber Gedanken zu machen, inwiefern
eine Öffnung der Systeme zu erreichen ist, die es ermöglicht, dass
verschiedene Personenkreise den Inhalt oder mitunter auch die Struktur
der Metadaten neu strukturieren und ihre Gemeinschaftsperspektiven als
\enquote{epistemic partners} (Duarte und Belarde-Lewis 2015, S. 686)
hinzufügen können (Anderson und Christen 2019, S. 135).

Die Ergänzung oder Schaffung gänzlich neuer Systeme unter Bezugnahme auf
indigene Bedarfe und \enquote{in a community-based approach} (Littletree
et al. 2020, S. 415) ist ebenfalls möglich: So wurden das Brian Deer
Classification System für First-Nation-Gruppen in British Columbia
(Cherry und Mukunda 2015, S. 549), die in der New Zealand National
Bibliographic Database integrierten Maori Subject Headings (Adler 2016,
S. 636) oder der Mashantucket Pequot Thesaurus of American Indian
Terminology (Littletree und Metoyer 2015) entwickelt. Bone und Lougheed
beschreiben darüber hinaus, inwiefern die Library of Congress Subject
Headings mit Bezug auf indigene Themen für den kanadischen Archivkontext
innerhalb eines dialogischen Prozesses angepasst worden sind (Bone und
Lougheed 2018). Diese Projekte bezogen explizit Community-Perspektiven
ein und unterstrichen deren Bedeutung für den Erschließungs- und
Suchprozess. Sie stellen eine Reaktion auf die unzureichende Verwendung
von englischsprachigen kontrollierten Vokabularen zur Beschreibung
indigener Themen dar.

\hypertarget{iv.}{%
\subsubsection{IV.}\label{iv.}}

Zukünftig wird zu klären sein, wie diverse Wissenssysteme in einen
Dialog mit den umfassenderen, der Struktur nach eurozentrischen
Wissensorganisationssystemen treten können.\footnote{Diese mitunter neu
  zu schaffenden Systeme basieren dann eventuell auch auf anderen
  Prinzipien als die derzeit hegemonialen Systeme. Littletree et
  al.~nennen in diesem Kontext sieben Bestandteile für indigene
  Knowledge Organization Systems: \enquote{Indigenous authority,
  Indigenous diversity, wholism and interrelatedness, Indigenous
  continuity, Aboriginal user warrant, designer responsibility, and
  institutional responsibility.} (Littletree et al.~2020, S. 415)} Wie
kann damit umgegangen werden, wenn Partikularsysteme nicht in Gänze
kompatibel sind oder sich inhaltlich widersprechen?\footnote{Wikidata
  bietet beispielsweise die Möglichkeit, innerhalb der Datenstruktur
  umstrittene Bezeichnungen und Definitionen kenntlich zu machen und
  kontrastierende Angaben zu hinterlegen.} Wo liegen die Grenzen von
Öffnungsprozessen und Inklusivität? Sind diese Begrenzungen immer
problematisch? Eine weitere grundsätzliche Frage besteht darin, wie
kollaboratives Arbeiten auf Augenhöhe bei mitunter voraussetzungsreichen
Projekten möglich ist? Tiefergehend zu reflektieren ist dabei auch, wie
eine Übertragung der oben genannten Beispiele auf den deutschsprachigen
Kontext aussehen könnte. Jene Ansätze basieren insbesondere auf der
unmittelbaren Zusammenarbeit mit vor Ort lebenden indigenen
Gemeinschaften, einem Bezugsrahmen, der sich nicht unmittelbar auf die
deutschsprachigen Länder übertragen lässt.

Zugleich ist abzuwägen, inwiefern eine grundsätzliche Zugänglichmachung
von Material ebenfalls eine wissensethische Anforderung ist und bei
problematischen und nicht präzisen Begriffen eruiert werden müsste,
inwiefern der Zugang zu diesen Materialien wichtiger ist als die
Verpflichtung zur absolut genauen Bezeichnung. In der Praxis könnte dies
beispielsweise bedeuten, dass (problematische) Begriffe als Synonyme
innerhalb der Normdateien und Vokabulare, wenn auch kommentiert,
erhalten bleiben sollten, damit sie weiterhin als Zugangsvokabular für
die Recherche fungieren können. An dieser Stelle kann die
bibliothekarische Welt durchaus von Diskursen und Praktiken von Museen
profitieren, beispielsweise im Hinblick auf einen verstärkten Kontakt
mit Herkunfts-Communities, aber auch in Hinblick auf Fragen
(kulturellen) Eigentums. Zumindest am Rande muss noch darauf hingewiesen
werden, dass die beschriebenen Prozesse und Ansätze nicht auf der
diskursiven Ebene des Vokabulars verbleiben sollten, sondern eine
Einordnung in einen größeren Zusammenhang notwendig ist. Dies umfasst
die grundsätzliche Frage ökonomischer Ungleichheit und damit
einhergehend nach Zugängen zu und Teilhabemöglichkeiten an
(wissenschaftlichen) Debatten und Ressourcen (Jain 2021, S. 23), zum
Beispiel bezüglich eines westlich zentrierten wissenschaftlichen
Publikationswesens oder auch die Notwendigkeit eines ganzheitlichen
Open-Access-Ansatzes, der beispielsweise auch das Problem der
Publikationsgebühren miteinbezieht.

Trotz dieser offenen Fragen stellen kontrollierte Vokabulare und
Normdaten unbestreitbar die Grundlage für eine konsistente und präzise
inhaltliche Erschließung von Material dar. Die GND ist in diesem
Zusammenhang ein wichtiges Instrument, da sie durch ihr langes Bestehen
einen sehr umfangreichen Normdatenbestand besitzt, weithin im
deutschsprachigen Raum verankert ist und nachhaltig gepflegt wird. Unter
arbeitsökonomischen Gesichtspunkten wird es schwer möglich sein, ein
derart umfangreiches Vokabular gänzlich neu zu etablieren. Die GND
ermöglicht, wenn auch aus einem eurozentrisch-kulturellen Verständnis
heraus, umfangreiches Wissen zu organisieren, dessen Menge ansonsten gar
nicht zu bewältigen wären. Damit besitzt die GND das Potenzial, Wissen
umfassend zu bewahren, zu vernetzen und zugänglich zu machen, wobei die
technische Infrastruktur zu ihrer effektiven Anwendung mitunter noch
Limitationen unterliegen kann. So weist Wiesenmüller darauf hin, dass
viele Recherchesysteme nicht in der Lage sind, die Vorteile einer
vernetzten Normdatenstruktur umfänglich zu nutzen, da es an bequemen
Varianten mangelt, sich bei der Schlagwortsuche auch hierarchisch
verknüpfte (Unter-) Begriffe beziehungsweise thematisch-verwandte
Begriffe ausgeben zu lassen (Wiesenmüller 2018, S. 29).

Mit dem Rückgriff auf die GND können Insellösungen vermieden werden und
sie kann als eine Grundlage für die Vernetzung mit unterschiedlichsten
Projekten fungieren. Gleichzeitig ermöglicht sie durch ihre kooperative
Struktur und die multiinstitutionelle Einbindung ein hohes Maß an
Nachhaltigkeit, das bei vielen kleineren Vokabularen, die beispielsweise
nach Projektende nicht weiter gepflegt werden, nicht gegeben ist.
Zugleich bietet sie von ihrer Grundstruktur die Möglichkeit der
Modifikation, zum Beispiel durch eine Anreicherung und Modernisierung
von Begriffen. Zu überlegen ist in diesem Kontext jedoch, inwiefern eine
reine Umbenennung historischen Wandel und mit der Benennung
einhergehende eurozentrische Ideologien unsichtbar machen
würde.\footnote{Innerhalb der GND findet sich beispielsweise der
  Datensatz zur Sammelbezeichnung \enquote{Khoikhoin}
  (\url{http://d-nb.info/gnd/4025936-5}). Hier wurde der problematische
  Begriff \enquote{Hottentotten} als Synonym beibehalten, aber mit einer
  kurzen Erklärung versehen, dass dieser eine abwertende Konnotation
  besitzt. Auf diese Weise sind Sucheinstiege und Wissensspuren erhalten
  geblieben und trotzdem ist zumindest eine kurze Kontextualisierung
  beziehungsweise historische Einordnung erfolgt.} Gleichzeitig kann die
GND potenziell mit anderen (fremdsprachigen, derzeit englischen und
französischen) Vokabularen gemappt werden. Außerdem existieren
Möglichkeiten, potenzielle Satellitenvokabulare, die wiederum an die GND
angeschlossen sind, zu entwickeln (Kasprzik und Kett 2018, S. 139). Hier
wird sich erst zeigen, ob und unter welchen Bedingungen dies zukünftig
realisiert werden kann. Die GND ermöglicht es somit, unterschiedliche
Perspektive ins Gespräch miteinander zu bringen und eine Zugänglichkeit
über Sprachbarrieren hinweg zu realisieren. Dies sind Ansätze, um
zumindest in Teilen den eurozentrischen Bias der GND aufzuarbeiten.

Abschließend bleibt zu konstatieren, dass vielfach noch offene
Problemstellungen existieren und im Fortlauf Schritt für Schritt geklärt
werden muss, wie und ob sich die ethisch-theoretischen Fragen in die
Praxis übersetzen lassen. Hier zeigt sich auch, dass das im Call für
diese Ausgabe erwähnte \enquote{Bewahren und Zeigen von Kultur in ihrer
Breite} und das \enquote{restriktive(n) Streben(s) nach Ordnung und
Kontrolle} (LIBREAS-Redaktion 2021) nicht notwendigerweise in einem
Widerspruch stehen müssen, sondern vielmehr produktiv vermittelt werden
kann. Dieser Anspruch wird Insbesondere dann einlösbar, wenn die
Kontrolle in Form von Vokabularen und Normdateien eine umfangreichere
Erschließung (zum Beispiel über Massenverfahren bei
Digitalisierungsprojekten) sowie verbesserte Vernetzung von Beständen
ermöglicht, die wiederum die Basis für eine größere Zugänglichkeit
darstellt.

\hypertarget{literaturverzeichnis}{%
\subsubsection{Literaturverzeichnis}\label{literaturverzeichnis}}

Adler, Melissa (2016): The Case for Taxonomic Reparations. In:
\emph{Knowledge Organization} 43 (8), S. 630--640.
\url{https://doi.org/10.5771/0943-7444-2016-8-630}

Adler, Melissa (2017a): Classification Along the Color Line: Excavating
Racism in the Stacks. In: \emph{JournalCritLIS} 1 (1).
\url{https://doi.org/10.24242/jclis.v1i1.17} (Alternativ: \url{https://journals.litwinbooks.com/index.php/jclis/article/view/17})

Adler, Melissa (2017b): Cruising the Library. Perversities in the
Organization of Knowledge. New York City: Fordham University Press.

Aguilera, Jasmine (2016): Another Word for \enquote*{Illegal Alien} at
the Library of Congress: Contentious. Hg. v. The New York Times. Online
verfügbar unter:
\url{https://www.nytimes.com/2016/07/23/us/another-word-for-illegal-alien-at-the-library-of-congress-contentious.html},
zuletzt \linebreak geprüft am 29.06.2021.

Aleksander, Karin (2014): Die Frau im Bibliothekskatalog. In:
\emph{LIBREAS} (25). \url{https://doi.org/10.18452/9059}

Anderson, Jane; Christen, Kimberly (2019): Decolonizing Attribution:
Traditions of Exclusion. In: \emph{Journal of Radical Librarianship} 5,
S. 113--152.

Beall, Jeffrey (2008): The Weaknesses of Full-Text Searching. In:
\emph{Journal of Academic Librarianship} 34 (5), S. 438--444.

Berman, Sanford (1971): Prejudice and Antipathies. A Tract on the LC
Subject Heads Concerning People. Metuchen, New Jersey: Scarecrow Press.

Bishop, Bradley Wade; Moulaison, Heather; Burwell, Christin Lee (2015):
Geographic Knowledge Organization: Critical Cartographic Cataloging and
Place-Names in the Geoweb. In: \emph{Knowledge Organization}42 (4).

Biswas, Paromita (2018): Rooted in the Past: Use of \enquote{East
Indians} in Library of Congress Subject Headings. In: \emph{Cataloging
\& Classification Quarterly} 56 (1), S. 1--18.
\url{https://doi.org/10.1080/01639374.2017.1386253}

Block, Sharon (2020): Erasure, Misrepresentation and Confusion:
Investigating JSTOR Topics on Women's and Race Histories. In:
\emph{Digital Humanities Quarterly}14 (1). Online verfügbar unter:
\url{http://www.digitalhumanities.org/dhq/vol/14/1/000448/000448.html}

Bone, Christine; Lougheed, Brett (2018): Library of Congress Subject
Headings Related to Indigenous Peoples: Changing LCSH for Use in a
Canadian Archival Context. In: \emph{Cataloging \& Classification
Quarterly} 56 (1), S. 83--95.
\url{https://doi.org/10.1080/01639374.2017.1382641}

Cherry, Alissa; Mukunda, Keshav (2015): A Case Study in Indigenous
Classification: Revisiting and Reviving the Brian Deer Scheme. In:
\emph{Cataloging \& Classification Quarterly} 53 (5-6), S. 548--567.
\url{https://doi.org/10.1080/01639374.2015.1008717}

Do Mar Castro Varela, Maria; Dhawan, Nikita (2005): Postkoloniale
Theorie. Eine kritische Einführung. Bielefeld: transcript Verlag.

Drabinski, Emily (2013): Queering the Catalog: Queer Theory and the
Politics of Correction. In: \emph{Library Quarterly: Information,
Community, Policy} 83 (2), S. 94--111.

Duarte, Marisa Elena; Belarde-Lewis, Miranda (2015): Imagining: Creating
Spaces for Indigenous Ontologies. In: \emph{Cataloging \& Classification
Quarterly} 53 (5-6), S. 677--702.
\url{https://doi.org/10.1080/01639374.2015.1018396}

Edge, Samuel J. (2019): A Subject \enquote{Queer}-y: A Literature Review
on Subject Access to LGBTIQ Materials. In: \emph{The Serials Librarian}
75 (1-4), S. 81--90. \url{https://doi.org/10.1080/0361526X.2018.1556190}

Frank, Nina (2007): Die Renovierung der ethnologischen Terminologie in
der Schlagwortnormdatei. Institut für Bibliotheks- und
Informationswissenschaft der Humboldt-Universität zu Berlin (Berliner
Handreichungen zur Bibliotheks- und Informationswissenschaft, 177).

Gartner, Richard (2016): Metadata. Shaping Knowledge from Antiquity to
the Semantic Web. Basel: Springer.

Gomes, Pablo; Frota, Maria Guiomar da Cunha (2019): Knowledge
Organization from a Social Perspective: Thesauri and the Commitment to
Cultural Diversity. In: \emph{KO} 46 (8), S. 639--646.
\url{https://doi.org/10.5771/0943-7444-2019-8-639}

Gross, Tina; Taylor, Arlene G.; Joudrey, Daniel N. (2015): Still a Lot
to Lose: The Role of Controlled Vocabulary in Keyword Searching. In:
\emph{Cataloging \& Classification Quarterly} 53 (1), S. 1--39.
\url{https://doi.org/10.1080/01639374.2014.917447}

Jain, Rohit (2021): Beyond "The west and the rest". Eine
anthropologisch-postkoloniale Suche nach einem bedingten Universalismus.
In: Julian Warner (Hg.): After Europe. Beiträge zur dekolonialen Kritik.
Berlin: Verbrecher Verlag, S. 15--30.

Kasprzik, Anna; Kett, Jürgen (2018): Vorschläge für eine
Weiterentwicklung der Sacherschließung und Schritte zur fortgesetzten
strukturellen Aufwertung der GND. In: \emph{o-bib. Das offene
Bibliotheksjournal} (4).
\url{https://doi.org/10.5282/O-BIB/2018H4S127-140}

LIBREAS-Redaktion (2021): Call for Papers LIBREAS-Ausgabe 39, 2.
Schwerpunkt \enquote{Dekolonisierung}. Antirassistisch und/oder
dekolonial? - Bibliotheken im Spannungsfeld antirassistischer und
kritischer Auseinandersetzung mit dem eigenen kolonialen Erbe. Online
verfügbar unter:
\href{https://libreas.wordpress.com/2021/01/27/call-for-papers-libreas-ausgabe-39-2-schwerpunkt-dekolonisierung/}{https://libreas.wordpress.com/2021/01/27/call-for-papers-libreas-ausgabe-39-2-\linebreak schwerpunkt-dekolonisierung/}

Littletree, Sandra; and, Miranda Belarde-Lewis; Duarte, Marisa (2020):
Centering Relationality: A Conceptual Model to Advance Indigenous
Knowledge Organization Practices. In: \emph{Knowledge Organization} 47
(5), S. 410--426. \url{https://doi.org/10.5771/0943-7444-2020-5-410}

Littletree, Sandra; Metoyer, Cheryl A. (2015): Knowledge Organization
from an Indigenous Perspective: The Mashantucket Pequot Thesaurus of
American Indian Terminology Project. In: \emph{Cataloging \&
Classification Quarterly} 53 (5-6), S. 640--657.
\url{https://doi.org/10.1080/01639374.2015.1010113}

Lo, Grace (2019): \enquote{Aliens} vs.~Catalogers: Bias in the Library
of Congress Subject Heading. In: \emph{Legal Reference Services
Quarterly} 38 (4), S. 170--196.
\url{https://doi.org/10.1080/0270319X.2019.1696069}

Mödden, Elisabeth; Schöning-Walter, Christa; Uhlmann, Sandro (2018):
Maschinelle Inhaltserschließung in der Deutschen Nationalbibliothek.
Breiter Sammelauftrag stellt hohe Anforderungen an die Algorithmen zur
statistischen und linguistischen Analyse. In: \emph{Bub Forum Bibliothek
und Information} 70 (1), S. 30--35.

Olson, Hope A. (2002): The Power to Name. Locating the Limits of Subject
Representation in Libraries. Dordrecht, s.l.: Springer Netherlands.

Olson, Hope A. (2004): The Ubiquitous Hierarchy: An Army to Overcome the
Threat of a Mob. In: \emph{Library Trends} 52 (3), S. 604--616.

Sparber, Sandra (2016): What's the frequency, kenneth? - Eine
(queer)feministische Kritik an Sexismen und Rassismen im
Schlagwortkatalog. In: \emph{Mitteilungen der VÖB} 69 (2), S. 236--243.

Turner, Hannah (2020): Cataloguing culture. Legacies of colonialism in
museum documentation. Vancouver, Toronto: UBC Press.

Uhlmann, Sandro (2013): Automatische Beschlagwortung von
deutschsprachigen Netzpublikationen mit dem Vokabular der Gemeinsamen
Normdatei. In: \emph{Dialog mit Bibliotheken} 25 (2), S. 26--36.

Wiesenmüller, Heidrun (2018): Maschinelle Inhaltserschließung in der
Deutschen Nationalbibliothek. Breiter Sammelauftrag stellt hohe
Anforderungen an die Algorithmen zur statistischen und linguistischen
Analyse. In: \emph{Bub Forum Bibliothek und Information} 70 (1), S.
26--29.

%autor
\begin{center}\rule{0.5\linewidth}{0.5pt}\end{center}

\textbf{Moritz Strickert} (\url{https://orcid.org/0000-0001-9626-5932})
ist Ethnologe/Soziologe und wissenschaftlicher Bibliothekar an der
Universitätsbibliothek der Humboldt-Universität zu Berlin als
Mitarbeiter des Fachinformationsdienstes Sozial- und Kulturanthropologie
(FID SKA). Er arbeitet derzeit an einem Projekt zur \enquote{Ethnologisierung}
der Gemeinsamen Normdatei (GND) und ist in der Arbeitsgruppe Thesauri
des Netzwerks für nachhaltige Forschungsstrukturen in kolonialen
Kontexten aktiv.

\end{document}
