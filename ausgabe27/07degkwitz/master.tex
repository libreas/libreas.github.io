\documentclass[a4paper,
fontsize=11pt,
%headings=small,
oneside,
numbers=noperiodatend,
parskip=half-,
bibliography=totoc,
final
]{scrartcl}

\usepackage{synttree}
\usepackage{graphicx}
\setkeys{Gin}{width=.4\textwidth} %default pics size

\graphicspath{{./plots/}}
\usepackage[english]{babel}
\usepackage[T1]{fontenc}
%\usepackage{amsmath}
\usepackage[utf8x]{inputenc}
\usepackage [hyphens]{url}
\usepackage{booktabs} 
\usepackage[left=2.4cm,right=2.4cm,top=2.3cm,bottom=2cm,includeheadfoot]{geometry}
\usepackage{eurosym}
\usepackage{multirow}
\usepackage[english]{varioref}
\setcapindent{1em}
\renewcommand{\labelitemi}{--}
\usepackage{paralist}
\usepackage{pdfpages}
\usepackage{lscape}
\usepackage{float}
\usepackage{acronym}
\usepackage{eurosym}
\usepackage[babel]{csquotes}
\usepackage{longtable,lscape}
\usepackage{mathpazo}
\usepackage[flushmargin,ragged]{footmisc} % left align footnote

\usepackage{listings}

\urlstyle{same}  % don't use monospace font for urls

\usepackage[fleqn]{amsmath}

%adjust fontsize for part

\usepackage{sectsty}
\partfont{\large}

%Das BibTeX-Zeichen mit \BibTeX setzen:
\def\symbol#1{\char #1\relax}
\def\bsl{{\tt\symbol{'134}}}
\def\BibTeX{{\rm B\kern-.05em{\sc i\kern-.025em b}\kern-.08em
    T\kern-.1667em\lower.7ex\hbox{E}\kern-.125emX}}

\usepackage{fancyhdr}
\fancyhf{}
\pagestyle{fancyplain}
\fancyhead[R]{\thepage}

%meta
%meta

\fancyhead[L]{A. Degkwitz \\ %author
LIBREAS. Library Ideas, 27 (2015). % journal, issue, volume.
\href{http://nbn-resolving.de/urn:nbn:de:kobv:11-100229847
}{urn:nbn:de:kobv:11-100229847}} % urn
\fancyhead[R]{\thepage} %page number
\fancyfoot[L] {\textit{Creative Commons BY 3.0}} %licence
\fancyfoot[R] {\textit{ISSN: 1860-7950}}

\title{\LARGE{From Special Subject Collections to Discipline Driven Information Provisioning}} %title %title
\author{Andreas Degkwitz} %author

\setcounter{page}{60}

\usepackage[colorlinks, linkcolor=black,citecolor=black, urlcolor=blue,
breaklinks= true]{hyperref}

\date{}
\begin{document}

\maketitle
\thispagestyle{fancyplain} 

%abstracts

%body
\textbf{{[}Lecture given at 17\textsuperscript{th} Fiesole Collection
Development Retreat on May 6th 2015 in Berlin{]}}

There are libraries that house no collections, and it is for this reason
that they are no libraries in the `proper meaning of the word´ even if
they say they are. It is about the same as if swimming pools without
pools pass themselves off as swimming facilities because visitors may do
`dry runs´ there.

But why should readers visit libraries which have no collections to
offer? In contrast to swimming baths without pools which have nothing to
offer but dry-run facilities, libraries without collections might be
called libraries because they allow readers to do research work due to
the libraries´ virtual nature. Consequently, libraries without
collections which \enquote{provide functionality in a potentially
virtual environment} for their user communities -- to specify the
meaning of `virtual´ -- must be seen as libraries and their collections
understood as potentially or functionally existing although they cannot
be seen or touched. Therefore, readers in libraries that house no
collections may read and work on the potential or functional basis
offered by the institution.

However, there is no permanent access to the collections possible as
they only exist potentially or functionally. Hence, readers of virtual
libraries often have no other real chance than creating `their own
libraries´ in one way or the other access repositories necessary for
their current scientific work. Once the job has been done and the
particular work completed, there is no further need for the readers to
use their self-created library any longer and they simply delete their
compilations.

The outlined phenomenon is neither fictitious nor weird but illustrates
the consequences if Special Subject Collections are transformed into a
Specialised Information Services since 2012, which is a discipline
driven information provisioning with library collections that contain
expiry dates.

The policy of supra-regional, national literature supply for science and
research has existed in Germany since 1949 -- that means an
internationally unique and world-wide renowned model of cooperation
among leading German scientific libraries -- which provided
international specialist literature for research. The extension and
further development of these `treasures of knowledge´ has been organised
on a federal basis and has been focused on the priorities of the most
important German scientific libraries. This system of Special Subject
Collections, which was funded by the German Research Foundation, has
guaranteed that at least one copy of every relevant scientific
publication is available in Germany. That means there has been a
nationally defined and distributed research library whose acquisition
profile covered all sciences and comprises 23 national and university
libraries (libraries with Special Subject Collections) and three main
specialist libraries for economics (Kiel), for medicine (Cologne) and
for science and technology (Hannover). The benefit of this provisioning
model was evident: comprehensive Special Subject Collections are
available for the scientists throughout Germany via document delivery
service, interlibrary loan or as online publications. Among collections
for many disciplines the system of Special Subject Collections -- funded
by the German Research Foundation- has had its focus on humanities and
social sciences, because the majority of the STM disciplines and
economics are covered by the main specialist libraries.

Is such a system still relevant in view of the existence of Google \&
Co. and has it justified the German Research Foundation´s budget
allocation of about 15 million Euros annually over recent years? This
sum represents about 8\% of the acquisition budget of all German
university libraries whose total annual budget for the acquisition of
information and media aggregates 210 million Euros.

What explains the attractiveness of the Special Subject Collections and
what do the involved libraries have to offer? At first sight, the answer
might be that it is the topicality of the collections -- for example the
data warehousing paradigm -- to meet the demand and requirements of the
market as efficiently as possible. But this scenario is primarily
dominated by short-term provisioning motivations. To extend collections
merely on the base of topical need scenarios also implies the exclusion
of all those books and periodicals whose contents are not in the focus
of current research interests and it does not consider the fact that
those information resources might be highly significant for future
research and teaching -- a recurrent phenomenon that proves to be true.
In other words: `Treasures of knowledge´, as libraries with Special
Subject Collections are referred to, do not meet primarily temporary
requirements but go far beyond with their claim to the collection
completeness which may be entirely different across various scientific
disciplines.

With respect to this, provisioning approaches that are oriented towards
short-term demands may only be applied with restrictions to libraries
with Special Subject Collections as those approaches could not meet the
requirements and expectations predetermined by their profile and claim.
The outlined issue must be seen in the immediate context of changing
Special Subject Collections into so-called Specialised Information
Services or discipline driven information provisioning. The process of
transferring tranches has started in 2013 and is supposed to continue
over a period of three years.

The German Research Foundation has not made a hasty decision in this
matter but analysed thoroughly in advance the existing system of Special
Subject Collections. The result can be summed up as follows: The role of
the Special Subject Collections´ network has been redefined with
reference to the evaluation results in order to take greater account of
discipline-specific interests as well as improve substantially the
immediate access to digital publications. The network´s main task is the
competent provision of special interest communities with printed and
electronic resources as well as all kinds of relevant media, search
engines and reference tools. The main focus of responsibility is not the
collection completeness as such but it implies the care and enlargement
of collections pursuant to the individual subject needs. The supervision
principles for the subjects are not longer based on the same conditions
for all scientific disciplines but are autonomously defined by the
responsible libraries in their dialogue with the scientists. The system
will have to undergo a considerable restructuring if it wants to meet
the challenge of this task modification. A more appropriate name than
Special Subject Collections has to be found in order to emphasise the
differentiated subject needs in the sense of information services. The
panel of experts for the library sector which consists of librarians and
scientists agreed to change from the Special Subject Collection funding
to that of a discipline driven information provisioning and they
consequently support the associated measures.

The funding of Special Subject Collections includes an increasing number
of E-Books and E-Journals. This trend has continued over recent years
and always aimed at the integration of digital publications. Thus it is
not a new phenomenon and neither is the development of value-added
services as tools for the collection search and processing.
Considerations worthy of discussion, which come along with the funding
policy of a discipline driven information provisioning underline the
emphasis of subject specific interests, the greater importance of a
qualified supply of the user community whilst the principle of
collection primacy becomes less important. Meeting current needs and
user interests of a particular subject are becoming the precondition for
the funding by the German Research Foundation. The associated
demand-driven orientation is directly inconsistent with the
supra-regional approach of literature supply but characterises
impressively the change in user behaviour described above when library
collections exist only potentially or functionally and the user creates
`their own library´ for their particular needs. The assessment that the
specific interests of the scientific user groups has not been paid
sufficient attention to the context of the special subject collection
funding is rather paradoxical and may suggest that it has not reached
the specific target groups. But are temporary needs and demand scenarios
really able to improve the quality of the information provisioning and
hence justify the phasing out of collections?

The findings of the evaluators´ committee did not lead to application
approvals only in 2013. Several requests of important disciplines of the
humanities and social sciences were refused. These results show clearly
the incompatibility of the new German Research Foundation´s funding
approach with the considerations of the Special Subject Collections´
claim to provide media on a lasting basis. It is self-evident that the
exploitation of Special Subject Collections by other value-added
services should be improved and this holds particularly true for the
transformation process from the printed towards the digitised media
paradigm. However, sciences prioritise contents (in the form of books
and periodicals) instead of communication and processing tools and
exactly this preference is questioned in principle due to the further
implementation of the discipline driven information provisioning.

In 2013 twelve applications for Specialised Information Services have
been submitted, five of these applications were approved upon by the
German Research Foundation. In 2014 again twelve applications for
Specialised Information Services have been submitted, five of these
applications were approved. Totally more than 50 \% of the submitted
applications has been rejected. In 2015 the submission of twenty five
applications is expected. Nobody is able to envisage, what will happen
with them. If the tendency of approvals and rejections continue, smaller
and even larger subject areas of the humanities, cultural studies and
social sciences will have to face serious cuts in the literature and
information supply. This loss is not likely to be compensated by other
sources as it is definitely not balanced by the funding of the
Specialised Information Services.

Additionally, the Specialised Information Services´ three-year project
approach means a disproportionate effort for the applicant libraries.
This effort is combined with great uncertainty whether the approved
measures for the provision of information and the development and
expansion of value-added services will be sustainable beyond the
three-year funding period. Mere project funding is rather
counterproductive in a permanently operating infrastructure. This fact
raises the question why the German Research Foundation has changed their
funding policy in this supra-regional context at all.

Although this cannot be discussed in depth, the main reason may be seen
in the Foundation´s funding approach which seems incompatible with a
more than 60-year-lasting infinite funding like that of the Special
Subject Collections. In the Foundation´s point of view the support of
such an infrastructure, which is a legitimate national model, has to be
ensured by other sources than the German Research Foundation. The
transition towards the Specialised Information Services´ project
approach may respond more appropriately to the Foundation´s funding
criteria. Furthermore, the envisaged Specialised Information Services´
model may, from the Foundation´s perspective, harmonise the transition
from analogue to digital media as well as the change of scientific work
methods, and thus have greater opportunities than the funding of Special
Subject Collections.

This leads consequently to a more fundamental consideration of the term
`collection´: What are collections and on the base of which criteria may
they be characterised?

(1\textsuperscript{st}) Collections show defined profiles which are
determined by particular individuals as scientists, persons of public
interest, collectors etc. or by thematic focuses of all kinds such as
Special Subject Collections. Besides, quite a number of scientific
libraries possess material- or language-specific holdings like
handwritings, old prints, pamphlets, pictorials, bequests, children´s
books or volumes of Asiatica, Hebraica, Orientalia and so forth.

(2\textsuperscript{nd}) In the majority of cases, collections are
possessed or owned by the responsible libraries. Compilations have often
been the reason and trigger for the foundation or development and
extension of libraries. Therefore, collections often come into existence
by chance at a certain location or library. Several -- and especially
renowned, precious collections originate from the treasure chambers of
royal and princely houses.

(3\textsuperscript{rd}) Collected holdings turn into library collections
through the professional supervision and their active further
development. This includes the expansion of holdings and, besides
collection-relevant items, also comprises interdisciplinary media, both
with the prime objective to enlarge and structure the collection
systematically. In this context, it is needless to mention the key role
of the items´ long-term availability and their archiving. It is doubtful
whether these and other demands could be fully met by means of a
Specialised Information Service funding, because it is a very discipline
driven information provisioning.

Against the background of the above mentioned collection criteria, I
would like to come back once more to libraries that have no collections.
Virtual libraries must be made a subject of discussion still from
another perspective relevant in the context of revised funding policies.
Library collecting activities such as the development and profiling of
holdings undergo generally fundamental changes. These long observed
activities have been dominated by the practice of granting licences for
digital resources on the basis of which publishing houses grant access
authorisation and user as well as archiving rights to libraries.

However, in most cases this does not include the libraries´ right to the
possession or ownership of these media. Licensed content (for example
E-Books and E-Journals, data bases) -- and in particular if they are
e-only-resources -- therefore will not undergo the same collection
development as this has happened to analogue paper versions for
libraries. The obvious reason for that is the restricted exploitation as
the libraries are not the owners of those materials.

Digital libraries provide e-content based on user subscriptions and are
consequently libraries without collections of their own. According to
the above-mentioned criteria, it means in other words that libraries
lack their collecting character in the supply segment of licensed
e-books and e-journals. They are virtual libraries as they provide their
user communities \enquote{with functionality in a potentially virtual
environment}. This applies after all to e-books and e-journals as a
continually growing part of the literature and information supply which
gradually replaces the further extension of analogue collections.

The impact of libraries without own collections has been pointed out in
connection with the changed funding policies from Special Subject
Collections towards Specialised Information Services -- that means
expiry dates of collections! The increasing numbers of licenses for
electronic media, which restrict the libraries´ rights of possession,
are leading to a comparable impact. If the licensed media do not become
the libraries´ own property, it will result in virtual libraries which
have no collections of their own. At the same time, one is tempted to
say that libraries are likely to become the long arm of publishing
houses and even more so regenerate them. In view of this, libraries must
have or regain the right to the complete possession and ownership of
analogue and digital collections in order to justify their claim as
\enquote{treasures of knowledge}.

As mentioned in the beginning, swimming baths without pools fall short
of the users´ expectations and therefore are no swimming pools in the
proper sense of the word. The same applies to libraries without own
collections. However, even the most fully-packed picnic hamper will not
appease the hunger if the cutlery is missing. Hence it would be wrong
for libraries to retreat completely to their collection activities as
they are clearly under pressure to offer support and services for
content retrieval and processing. This involves services and tools for
the digital processing and structuring of content for research, for
evaluation and referencing of text- and picture corpora, for annotation,
comment and publication of research findings, and last but not least
measures that allow the long-term access to and archiving of analogue as
well as digital data, objects and texts. It is essential in this context
to provide the tools complementary to the content of the research
projects.

However, these tools must not dominate the scientific work by focusing
on social interaction and communication as well as material processing
and transformation instead of concentrating on the research content as
such. That is why libraries should not only collect items but, along
with their holdings, offer their readers specialised services and this
exactly in libraries that possess collections of their own and thus may
justifiably be called libraries.

%autor
\begin{center}\rule{0.5\linewidth}{\linethickness}\end{center}

Since 2011 \textbf{Prof.~Dr.~Andreas Degkwitz} is director of the
university library and lecturer at the School for Library and
Information Science at Humboldt-Universität zu Berlin. In 2014 he
received a honorary professorship from the University of Applied
Sciences Potsdam, Department Information Sciences.

\end{document}