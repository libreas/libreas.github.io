\documentclass[a4paper,
fontsize=11pt,
%headings=small,
oneside,
numbers=noperiodatend,
parskip=half-,
bibliography=totoc,
final
]{scrartcl}

\usepackage[babel]{csquotes}
\usepackage{synttree}
\usepackage{graphicx}
\setkeys{Gin}{width=.4\textwidth} %default pics size

\graphicspath{{./plots/}}
\usepackage[english]{babel}
\usepackage[T1]{fontenc}
%\usepackage{amsmath}
\usepackage[utf8x]{inputenc}
\usepackage [hyphens]{url}
\usepackage{booktabs} 
\usepackage[left=2.4cm,right=2.4cm,top=2.3cm,bottom=2cm,includeheadfoot]{geometry}
\usepackage[labelformat=empty]{caption} % option 'labelformat=empty]' to surpress adding "Abbildung 1:" or "Figure 1" before each caption / use parameter '\captionsetup{labelformat=empty}' instead to change this for just one caption
\usepackage{array}
\usepackage{eurosym}
\usepackage{multirow}
\usepackage[ngerman]{varioref}
\setcapindent{1em}
\renewcommand{\labelitemi}{--}
\usepackage{paralist}
\usepackage{pdfpages}
\usepackage{lscape}
\usepackage{float}
\usepackage{acronym}
\usepackage{eurosym}
\usepackage{longtable,lscape}
\usepackage{mathpazo}
\usepackage[normalem]{ulem} %emphasize weiterhin kursiv
\usepackage[flushmargin,ragged]{footmisc} % left align footnote
\usepackage{ccicons} 
\setcapindent{0pt} % no indentation in captions
\usepackage{xurl} % Breaks URLs

%%%% fancy LIBREAS URL color 
\usepackage{xcolor}
\definecolor{libreas}{RGB}{112,0,0}

\usepackage{listings}

\urlstyle{same}  % don't use monospace font for urls

\usepackage[fleqn]{amsmath}

%adjust fontsize for part

\usepackage{sectsty}
\partfont{\large}

%Das BibTeX-Zeichen mit \BibTeX setzen:
\def\symbol#1{\char #1\relax}
\def\bsl{{\tt\symbol{'134}}}
\def\BibTeX{{\rm B\kern-.05em{\sc i\kern-.025em b}\kern-.08em
    T\kern-.1667em\lower.7ex\hbox{E}\kern-.125emX}}

\usepackage{fancyhdr}
\fancyhf{}
\pagestyle{fancyplain}
\fancyhead[R]{\thepage}

% make sure bookmarks are created eventough sections are not numbered!
% uncommend if sections are numbered (bookmarks created by default)
\makeatletter
\renewcommand\@seccntformat[1]{}
\makeatother

% typo setup
\clubpenalty = 10000
\widowpenalty = 10000
\displaywidowpenalty = 10000

\usepackage{hyperxmp}
\usepackage[colorlinks, linkcolor=black,citecolor=black, urlcolor=libreas,
breaklinks= true,bookmarks=true,bookmarksopen=true]{hyperref}
\usepackage{breakurl}

%meta
%meta

\fancyhead[L]{S. Moussa\\ %author
LIBREAS. Library Ideas, 47 (2025). % journal, issue, volume.
\href{https://doi.org/10.18452/34822}{\color{black}https://doi.org/10.18452/34822}
{}} % doi 
\fancyhead[R]{\thepage} %page number
\fancyfoot[L] {\ccLogo \ccAttribution\ \href{https://creativecommons.org/licenses/by/4.0/}{\color{black}Creative Commons BY 4.0}}  %licence
\fancyfoot[R] {ISSN: 1860-7950}

\title{\LARGE{Lost in retraction: The curious case of a misidentified investigation and a missing retraction notice}}% title
\author{Salim Moussa} % author

\setcounter{page}{1}

\hypersetup{%
      pdftitle={Lost in retraction: The curious case of a misidentified investigation and a missing retraction notice},
      pdfauthor={Salim Moussa},
      pdfsubject={LIBREAS. Library Ideas, 47 (2025)},
      pdfkeywords={retraction process, transparency, academic record, publishing ethics},
      pdflicenseurl={https://creativecommons.org/licenses/by/4.0/},
      pdfcopyright={CC BY 4.0 International},
      pdfcontacturl={http://libreas.eu},
      pdfurl={https://doi.org/10.18452/34822},
      pdfdoi={10.18452/34822},
      pdflang={en},
      pdfmetalang={en}
     }



\date{}
\begin{document}

\maketitle
\thispagestyle{fancyplain} 

%abstracts
\begin{abstract}
\noindent
\textbf{Abstract}: This commentary examines the unusual retraction
record of an article in the \textit{Journal of Marketing Research}, an elite
marketing journal, that was originally retracted in 2014 but whose
retraction notice went missing from Sage Publications' scholarly record
until a second retraction in 2022. To make matters worse, the retraction
notice incorrectly identified Tilburg University as the investigating
institution, rather than Erasmus University Rotterdam. These
discrepancies raise serious questions about editorial oversight,
transparency, and academic record integrity. By analyzing this case, the
author emphasizes the need for stricter retraction protocols to maintain
scientific integrity.
\end{abstract}

%body
\section{Introduction}\label{introduction}

Retractions are a bitter pill, but they are critical to maintaining the
integrity of the scientific record (Joshi \& Minirani, 2024). They act
as a corrective mechanism for flawed or fraudulent research. (COPE,
2019; Boudry et al., 2023; NISO, 2024; Koo \& Lin, 2024; Kovacs et al.,
2024; Moussa, 2022a; Moussa, 2022b; Teixeira da Silva, 2022; Yang et
al., 2024). However, when the retraction process itself is marred by
errors, inconsistencies, or procedural failures, it raises concerns
about the reliability of academic publishing (Thorp, 2022; Moussa \&
Charlton, 2024). This commentary picks up where Moussa and Charlton
(2024), in their recent \emph{Accountability in Research} article, left
off and examines a particularly troubling case: the retraction of an
article from the \emph{Journal of Marketing Research} (JMR).

Launched in 1964, JMR is one of the five publication venues sponsored by
the American Marketing Association (AMA), the marketing discipline's
most powerful academic and professional association. JMR is a top-tier
journal included in the famous-yet-controversial Financial Times' list
of the top 50 business and economics journals\footnote{See
  \url{https://www.ft.com/content/3405a512-5cbb-11e1-8f1f-00144feabdc0}
  (Last accessed 10 March 2025).} (Moussa, 2021). The JMR was published
by AMA up to October 2018 when Sage publications started publishing
AMA's fleet of journals.\footnote{\url{https://us.sagepub.com/en-us/nam/press/sage-publishing-and-the-american-marketing-association-partner-to-publish-ama-journals}
  (Last accessed 10 March 2025).}

The JMR article under scrutiny was published online on April 1, 2010. It
was co-authored by Dirk Smeesters, a Belgian-born social psychologist
who served, until June 21, 2012, as a professor of consumer behavior and
society at the Rotterdam School of Management, Erasmus University
Rotterdam (EUR) in the Netherlands. Smeesters became the central figure
in a widely documented research misconduct scandal, first uncovered by
behavioral scientist and whistleblower Uri Simonsohn. Simonsohn
identified suspiciously improbable results in one of Smeesters' papers
and requested the raw data. Two investigations led by EUR revealed that
Smeesters manipulated datasets to produce favorable outcomes, prompting
the retraction of seven articles (Seadle, 2022; Moussa \& Charlton,
2024; Crone \& Green, 2025). The report of the first
investigation\footnote{An English version of that report is archived at:
  \url{https://web.archive.org/web/20120707062725/http://www.eur.nl/fileadmin/ASSETS/press/2012/Juli/report_Committee_for_inquiry_prof._Smeesters.publicversion.28_6_2012.pdf}}
was released on June 1\textsuperscript{st}, 2012 while the one for the
follow-up investigation\footnote{See an archived version of that report
  at:
  \url{https://web.archive.org/web/20170311224705/https://www.eur.nl/fileadmin/ASSETS/press/2014/maart/Report_Smeesters_follow-up_investigation_committee.final.pdf}}
was made public on March 5\textsuperscript{th}, 2014, explicitly
recommended the retraction of the JMR article.

\section{The long-awaited retraction
notice}\label{the-long-awaited-retraction-notice}

On December 1\textsuperscript{st}, 2022, AMA's JMR finally issued a
retraction notice for Smeesters' fraudulent article. Details about this
retraction notice are provided in Table 1.

\begin{table}[H]
\centering
\begin{tabular}{m{10em} m{30em}}
\toprule
Digital Object Identifier of the retraction notice & \url{https://doi.org/10.1177/00222437221139856} \\
\hline
Date of issuance of the retraction notice & December 1, 2022 \\
\hline
Verbatim of the retraction notice & \emph{On March 19, 2014, the Journal
of Marketing Research and American Marketing Association (AMA) issued a
retraction related to Jia (Elke) Liu and Dirk
Smeesters\textquotesingle s (2010) article, \enquote{Have You Seen the
News Today? The Effect of Death-Related Media Contexts on Brand
Preferences} (Volume 47, Issue 2, pp.~251--62), at the recommendation of
then--Editor in Chief Robert Meyer.}

\emph{The retraction was issued following the recommendation of a report
dated March 5, 2014, from Tilburg University. A Tilburg committee
conducted an intensive investigation into the data collection and
analysis of this article and concluded that it should be retracted.}

\emph{Although the retraction was made public in 2014, when the AMA
entered into a partnership with SAGE Publications to produce and
distribute the Journal of Marketing Research in 2018, the retracted
version was not included in SAGE\textquotesingle s scholarly record due
to a clerical error. This notice serves to correct the error.} \\
\bottomrule
\end{tabular}
\caption{Table 1: Retraction notice for Smeesters' fraudulent JMR article}\
\end{table}


\section{A vague claim about a previous
retraction}\label{a-vague-claim-about-a-previous-retraction}

A first troubling issue is the retraction notice's vague claim that a
retraction for Smeesters' article \enquote{was made public in 2014}
without specifying where. Was the retraction posted on JMR's website, or
was it formally published in one of its volumes? If it was merely posted
online, this would deviate from standard practice, which requires a
formal retraction notice (Xu \& Hu, 2023). If it was officially
published, the 2022 notice should provide precise details --- volume,
issue, and page numbers --- to ensure proper documentation and
accountability (Bakker et al., 2024). Such ambiguity undermines
transparency and raises questions about the integrity of the retraction
process.

\section{Institutional
misidentification}\label{institutional-misidentification}

More troubling is the fact that the retraction notice erroneously
attributes the investigation to Tilburg University. The correct
institution responsible for investigating Dirk Smeesters' research
misconduct was EUR. Given the weight that institutional investigations
carry in cases of research misconduct, naming the wrong university
introduces serious concerns about editorial diligence and
accountability. This mistake is not a minor clerical error; it
misdirects responsibility and potentially damages the reputation of an
uninvolved institution.

If the first retraction notice was indeed published on March 19, 2014,
and the 2022 notice for the same article contains this error, it is
reasonable to question whether the original \enquote{lost} notice also
misidentified the institution.

Such an error also raises concerns about the process that retraction
notices undergo. Journals, particularly those with an \enquote{elite}
status, have a duty to ensure accuracy in their official statements. If
fundamental details such as institutional involvement are incorrect, how
can readers trust that the retraction process was conducted rigorously?
This case exemplifies why retraction notices should be subject to the
same scrutiny as published research (Xu et al., 2023; Tang, 2024).

\section{Clerical error or editorial
negligence}\label{clerical-error-or-editorial-negligence}

According to the retraction notice, Smeesters' JMR article was
originally retracted in 2014, yet it did not appear in Sage
Publications' scholarly record until late 2022. This negligence ---
whether due to flawed record-keeping or a breakdown in the transition
between publishers --- raises serious concerns about the reliability of
academic documentation. For context, Smeesters' article was the first
ever retracted by JMR. Failing to communicate this lone retraction
notice to the new publisher is not just an oversight---it is a troubling
lapse in accountability. As a result, for over four years, a fraudulent
article remained publicly accessible as legitimate, persisting through
JMR's transition to Sage in October 2018. During this time, it continued
to be viewed, downloaded (after paying a fee), read, and cited,
reinforcing its false legitimacy until its retraction notice was finally
issued on December 1\textsuperscript{st}, 2022.

A striking consequence of this negligence is that a recent JMR article,
published online on August~1, 2023 (see
\url{https://doi.org/10.1177/00222437231194950}), cites Smeesters' JMR
paper as a legitimate source in its limitations and future research
section---without acknowledging its fraudulent nature/retracted status.
Hence, JMR cited its own retracted article despite officially retracting
it \enquote{twice}.

\section{No apology}\label{no-apology}

The retraction notice notably lacks any apology to JMR's readership ---
a glaring omission that raises serious ethical concerns. Readers,
researchers, and institutions relied on this article as a legitimate
contribution to the field, potentially shaping their own work based on
fraudulent findings. A formal acknowledgment of this failure, along with
an apology, would have been the bare minimum in maintaining trust and
accountability in academic publishing.

\section{Conclusion}\label{conclusion}

The case of Smeesters' JMR article exemplifies a concerning failure in
the retraction process (Thorp, 2022), allowing a fraudulent article to
remain publicly available and influential for several years. The
retraction was lost, delayed, poorly documented, and full of errors,
jeopardizing the integrity of academic publishing. The lack of an
apology raises questions about editorial responsibility and
accountability (see Meyer, 2015). If retractions are handled
incorrectly, how can scholars trust the scholarly record? This is more
than just an oversight; it is a stark reminder of the systemic flaws
that allow misinformation to thrive unchecked in environments intended
to protect research integrity. The AMA, JMR, and Sage must immediately
correct this retraction notice to ensure that it is both accurate and
comprehensive. Anything less reduces transparency and accountability in
the retraction process.

\subsection{Funding}\label{funding}

The author did not receive any financial support for this paper.

\subsection{Competing interests}\label{competing-interests}

The author declares he has no potential conflicts of interest regarding
this paper.

\section{References}\label{references}

Bakker, C. J., Reardon, E. E., Brown, S. J., Theis-Mahon, N., Schroter,
S., Bouter, L., \& Zeegers, M. P. (2024). Identification of retracted
publications and completeness of retraction notices in public health.
\emph{Journal of Clinical Epidemiology}, \emph{173}, 111427.
\url{https://doi.org/10.1016/j.jclinepi.2024.111427}

Boudry, C., Howard, K., \& Mouriaux, F. (2023). Poor visibility of
retracted articles: a problem that should no longer be ignored.
\emph{bmj}, 381, e072929. \url{https://doi.org/10.1136/bmj-2022-072929}

Committee on Publication Ethics (2019). Retraction guidelines Version 2.
\url{https://doi.org/10.24318/cope.2019.1.4} (Last accessed 13 October
2024).

Crone, G., \& Green, C. D. (2025). Tools of the data detective: A review
of statistical methods to detect data and result anomalies in
psychology. \emph{Theory \& Psychology}.
\url{https://doi.org/10.1177/09593543241311861}

Ivory, J. D., \& Elson, M. (2024). A tale of three retractions: a call
for standardized categorization and criteria in retraction statements.
\emph{Current Psychology}, \emph{43}(17), 16023--16029.
\url{https://doi.org/10.1007/s12144-023-05216-6}

Joshi, P. B., \& Minirani, S. (2024). Retractions as a Bitter Pill
Corrective Measure to Eliminate Flawed Science. In \emph{Scientific
Publishing Ecosystem: An Author-Editor-Reviewer Axis} (pp.~307--327).
Singapore: Springer Nature Singapore.
\url{https://doi.org/10.1007/978-981-97-4060-4_18}

Koo, M., \& Lin, S. C. (2024). Retracted articles in scientific
literature: A bibliometric analysis from 2003 to 2022 using the Web of
Science. \emph{Heliyon}, \emph{10}(20).
\url{https://doi.org/10.1016/j.heliyon.2024.e38620}

Kovacs, M., Varga, M. A., Dianovics, D., Poldrack, R. A., \& Aczel, B.
(2024). Opening the black box of article retractions: exploring the
causes and consequences of data management errors. \emph{Royal Society
Open Science}, \emph{11}(12), 240844.
\url{https://doi.org/10.1098/rsos.240844}

Meyer, R. J. (2015). Editorial: A field guide to publishing in an era of
doubt. \emph{Journal of Marketing Research}, \emph{52}(5), 577--579.
\url{https://doi.org/10.1509/jmr.52.5.577}

Moussa, S. (2021). Are FT50 journals really leading? A comment on
Fassin. \emph{Scientometrics}, \emph{126}(12), 9613--9622.
\url{https://doi.org/10.1007/s11192-021-04158-9}

Moussa, S. (2022a). Celebrating Failure: Learning lessons from a leading
consumer behavior journal's retractions. \emph{Consumer Behavior
Review}, \emph{6}(1).
\url{https://doi.org/10.51359/2526-7884.2022.254032}

Moussa, S. (2022b). The propagation of error: retracted articles in
marketing and their citations. \emph{Italian Journal of Marketing},
\emph{2022}(1), 11--36. \url{https://doi.org/10.1007/s43039-021-00044-7}

Moussa, S., \& Charlton, A. (2024). Retraction (mal) practices of elite
marketing and social psychology journals in the Dirk Smeesters' research
misconduct case. \emph{Accountability in Research}, \emph{31}(7),
751--766. \url{https://doi.org/10.1080/08989621.2022.2164489}

National Information Standards Organization (2024). Communication of
Retractions, Removals, and Expressions of Concern.
\url{https://doi.org/10.3789/niso-rp-45-2024} (Last accessed March 10,
2025).

Seadle, M. (2022). \emph{Quantifying research integrity}. Springer Nature. \url{https://doi.org/10.1007/978-3-031-02306-4}

Tang, B. L. (2024). Potential issues in mandating a disclosure of institutional investigation in retraction notices. \emph{Science and
Engineering Ethics}, \emph{30}(1), 1. \url{https://doi.org/10.1007/s11948-024-00468-2}

Teixeira da Silva, J. A. (2022). A synthesis of the formats for correcting erroneous and fraudulent academic literature, and associated
challenges. \emph{Journal for General Philosophy of Science}, \emph{53}(4), 583-599. \url{https://doi.org/10.1007/s10838-022-09607-4}

Thorp, H. H. (2022). Rethinking the retraction process.~\emph{Science},~\emph{377}(6608), 793. \url{https://doi.org/10.1126/science.ade3742}

Xu, S. B., \& Hu, G. (2023). What to communicate in retraction notices?.
\emph{Learned Publishing}, \emph{36}(3), 463-467.
\url{https://doi.org/10.1002/leap.1548}

Xu, S. B., Evans, N., Hu, G., \& Bouter, L. (2023). What do retraction
notices reveal about institutional investigations into allegations
underlying retractions?. \emph{Science and Engineering Ethics},
\emph{29}(4), 25. \url{https://doi.org/10.1007/s11948-023-00442-4}

Yang, S., Qi, F., Diao, H., \& Ajiferuke, I. (2024). Do retraction
practices work effectively? Evidence from citations of psychological
retracted articles. \emph{Journal of Information Science}, \emph{50}(2),
531-545. \url{https://doi.org/10.1177/01655515221097623}

%autor
\begin{center}\rule{0.5\linewidth}{0.5pt}\end{center}

\textbf{Salim Moussa} (\url{https://orcid.org/0000-0003-2589-0719}) is a
researcher and lecturer in marketing at the Institut Supérieur
d'Administration des Entreprises (ISAE), University of Gafsa, Tunisia.
His research focuses on scholarly communication, publication ethics, and
the integrity of the academic record.

\end{document}