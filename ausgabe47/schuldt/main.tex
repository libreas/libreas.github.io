\documentclass[a4paper,
fontsize=11pt,
%headings=small,
oneside,
numbers=noperiodatend,
parskip=half-,
bibliography=totoc,
final
]{scrartcl}

\usepackage[babel]{csquotes}
\usepackage{synttree}
\usepackage{graphicx}
\setkeys{Gin}{width=.4\textwidth} %default pics size

\graphicspath{{./plots/}}
\usepackage[ngerman]{babel}
\usepackage[T1]{fontenc}
%\usepackage{amsmath}
\usepackage[utf8x]{inputenc}
\usepackage [hyphens]{url}
\usepackage{booktabs} 
\usepackage[left=2.4cm,right=2.4cm,top=2.3cm,bottom=2cm,includeheadfoot]{geometry}
\usepackage[labelformat=empty]{caption} % option 'labelformat=empty]' to surpress adding "Abbildung 1:" or "Figure 1" before each caption / use parameter '\captionsetup{labelformat=empty}' instead to change this for just one caption
\usepackage{eurosym}
\usepackage{multirow}
\usepackage[ngerman]{varioref}
\setcapindent{1em}
\renewcommand{\labelitemi}{--}
\usepackage{paralist}
\usepackage{pdfpages}
\usepackage{lscape}
\usepackage{float}
\usepackage{acronym}
\usepackage{eurosym}
\usepackage{longtable,lscape}
\usepackage{mathpazo}
\usepackage[normalem]{ulem} %emphasize weiterhin kursiv
\usepackage[flushmargin,ragged]{footmisc} % left align footnote
\usepackage{ccicons} 
\setcapindent{0pt} % no indentation in captions
\usepackage{xurl} % Breaks URLs

%%%% fancy LIBREAS URL color 
\usepackage{xcolor}
\definecolor{libreas}{RGB}{112,0,0}

\usepackage{listings}

\urlstyle{same}  % don't use monospace font for urls

\usepackage[fleqn]{amsmath}

%adjust fontsize for part

\usepackage{sectsty}
\partfont{\large}

%Das BibTeX-Zeichen mit \BibTeX setzen:
\def\symbol#1{\char #1\relax}
\def\bsl{{\tt\symbol{'134}}}
\def\BibTeX{{\rm B\kern-.05em{\sc i\kern-.025em b}\kern-.08em
    T\kern-.1667em\lower.7ex\hbox{E}\kern-.125emX}}

\usepackage{fancyhdr}
\fancyhf{}
\pagestyle{fancyplain}
\fancyhead[R]{\thepage}

% make sure bookmarks are created eventough sections are not numbered!
% uncommend if sections are numbered (bookmarks created by default)
\makeatletter
\renewcommand\@seccntformat[1]{}
\makeatother

% typo setup
\clubpenalty = 10000
\widowpenalty = 10000
\displaywidowpenalty = 10000

\usepackage{hyperxmp}
\usepackage[colorlinks, linkcolor=black,citecolor=black, urlcolor=libreas,
breaklinks= true,bookmarks=true,bookmarksopen=true]{hyperref}
\usepackage{breakurl}

%meta
%meta

\fancyhead[L]{K. Schuldt\\ %author
LIBREAS. Library Ideas, 47 (2025). % journal, issue, volume.
\href{https://doi.org/10.18452/34779}{\color{black}https://doi.org/10.18452/34779}
{}} % doi 
\fancyhead[R]{\thepage} %page number
\fancyfoot[L] {\ccLogo \ccAttribution\ \href{https://creativecommons.org/licenses/by/4.0/}{\color{black}Creative Commons BY 4.0}}  %licence
\fancyfoot[R] {ISSN: 1860-7950}

\title{\LARGE{Aus der Zeit fallen. Ein Essay über die St. Galler Waldhandschrift}}% title
\author{Karsten Schuldt} % author

\setcounter{page}{1}

\hypersetup{%
      pdftitle={Aus der Zeit fallen. Ein Essay über die St. Galler Waldhandschrift},
      pdfauthor={Karsten Schuldt},
      pdfsubject={LIBREAS. Library Ideas, 47 (2025)},
      pdfkeywords={Handschrift, Ökologie, Waldsterben, Zeit, mittelalterliche Schriften, Skriptorien},
      pdflicenseurl={https://creativecommons.org/licenses/by/4.0/},
      pdfcopyright={CC BY 4.0 International},
      pdfcontacturl={http://libreas.eu},
      pdfurl={https://doi.org/10.18452/34779},
      pdfdoi={10.18452/34779},
      pdflang={de},
      pdfmetalang={de}
     }



\date{}
\begin{document}

\maketitle
\thispagestyle{fancyplain} 

%abstracts
\begin{abstract}
\noindent
\textbf{Zusammenfassung}: Die Waldhandschrift wurde 1985--1987 nach dem
Vorbild mittelalterlicher Handschriften (auf Pergament, mit
mittelalterlichen Techniken des Schreibens und Buchmalens) produziert.
Thematisch geht es in ihr um das Waldsterben. Sie unterliegt
spezifischen Nutzungsbedingungen, unter anderem darf sie nur per Hand
abgeschrieben werden. Heute liegt sie in der Stiftsbibliothek
St.~Gallen. In diesem Essay geht es darum, die Waldhandschrift möglichst
nahe am Objekt zu beschreiben und gleichzeitig eine Interpretation zu
versuchen: Wie wirkt sie heute?

\begin{center}\rule{0.5\linewidth}{0.5pt}\end{center}

\noindent
\textbf{Abstract}: The Waldhandschrift (forest manuscript) is a book
produced 1985--1987 following medieval procedures (on parchment, with
medieval techniques for writing and book illustration). Its topic is the
“Waldsterben” (forest decline). It is subject to specific
requirements, for instance handwriting as the only form of reproduction
method permitted. Today, the Waldhandschrift is deposited at the abbey
library of Saint Gall. This essay tries to describe the Waldhandschrift
as concrete as possible and attempts to interpret this manuscript: How
does it affect today?
\end{abstract}

%body
Der Zugang zur Waldhandschrift ist nicht ganz einfach -- und das ist so
gewollt. Sie ist eine in der Stiftsbibliothek St.~Gallen verwahrte
Handschrift. Um sie einzusehen, muss man sich beim Stiftsbibliothekar
anmelden, um dann, nach der Bestätigung, persönlich am vereinbarten Tag
(nicht am Wochenende und nicht in der Mittagspause) in St.~Gallen zu
erscheinen. Das gilt auch für die anderen Codices und alle vor 1900
produzierten Medien, die in der Bibliothek vorhanden sind. Aber im
Gegensatz zu diesen ist die Waldhandschrift recht aktuell. Die
Stiftsbibliothek selber ist bekannt für die mittelalterlichen
Handschriften, die sie verwahrt, und für ihren barocken Bibliothekssaal,
der in fast allen Kalendern über die schönsten Bibliotheken der Welt
abgebildet wird. Die Waldhandschrift hingegen wurde erst 1985--1987
erstellt (was jetzt allerdings auch schon fast vierzig Jahre her ist).
(Scapatetti \& Schäffel 1991)

Ihr vorangestellt ist eine Präambel, die in den vier schweizerischen
Landessprachen (Deutsch, Französisch, Italienisch, Rätoromanisch)
weitere Nutzungsbedingungen festlegt. Unter anderem soll die Handschrift
allen zugänglich sein, aber sie soll nicht in Vitrinen ausgestellt
werden (Waldhandschrift 1987: 2f.). Und sie darf nicht mechanisch
vervielfältigt werden (Waldhandschrift 1987: 2f.). Es geht darum, dass
die Handschrift gelesen, vorgelesen und abgeschrieben wird. Explizit
wird in der Präambel geschrieben, dass sie sich so verbreiten und zum
aktiven Handeln motivieren soll. Sie soll Aufwand machen. Es soll nicht
trivial sein, sie zu konsumieren.\footnote{In einer anderen Version als
  der Präambel liegen die Nutzungsbedingungen auch gedruckt vor
  (Scarpatetti \& Schäffel 1991: 167). Eigentlich soll die
  Waldhandschrift auch von der Stiftsbibliothek verliehen werden,
  allerdings unter sehr spezifischen Bedingungen, beispielsweise der,
  dass sie «bei der Benützung stets von einem Menschen betreut
  {[}wird{]}» (Scarpattit \& Schäffel 1991: 167). In der Vergangenheit
  ist dies vorgekommen (siehe unter anderem Vereinigung Bündner
  Umweltschutzorganisationen VBU (1989)), in den letzten Jahren aber
  offenbar nicht mehr.}

\subsubsection{Die Waldhandschrift als
Objekt}\label{die-waldhandschrift-als-objekt}

April 2025. Im Lesesaal der Stiftsbibliothek St.~Gallen liegt die
Waldhandschrift für mich aus. «Saal» scheint ein wenig der falsche Name
für diesen Raum zu sein. Er hat zwei grosse Tische mit acht
Arbeitsplätzen. Buch- und Zeitschriftenregale sind in den Wänden
eingelassen, und er ist auch sonst mit ausgesuchten Gegenständen
ausgestattet. Zwei Holzstatuen von Mönchen stehen auf den Schränken (ich
denke, es sind Benedikt von Nursia und St.~Gallus, aber katholische
Ikonographie ist wirklich nicht meine Expertise). Zwischen ihnen steht
ein aufgeklapptes Triptychon, welches im Mittelteil die Kreuzigung Jesus
zeigt. Es ist klar, dass ich mich in einem ehemaligen
Benediktinerkloster befinde. Aber im Gegensatz zum bekannten
Bibliotheksraum gleich nebenan, hinter einer der Türen, scheint der
Lesesaal doch recht klein. Eine Art Klause.

Die Waldhandschrift ist explizit als Codex gestaltet, nach dem Vorbild
der spätmittelalterlichen Codices, die in der Stiftsbibliothek lagern.
(Scrapatetti \& Schäffel 1991) Ihre Seiten bestehen aus Pergament, nicht
aus Papier. Der Codex ist von einer Anzahl von Skriptor*innen händisch
geschrieben worden. Er ist -- wenn man die gebräuchlichen Bezeichnungen
für mittelalterliche Schriften anlegt -- ein Folioband. (Der Katalog der
Stiftsbibliothek gibt die Grösse an als: 36 × 28 cm, also etwas kleiner
als das heutige DIN A3 mit seinen 42 × 29,7 cm.) Einige wenige Seiten
sind leer. Initialen, Verzierungen und Miniaturen sind von einem
Buchkünstler direkt auf das Pergament gemalt worden. Hier und da finden
sich Glossen. Verwendet wurden explizit Schriften, die im
Spätmittelalter auf dem Gebiet der heutigen Schweiz verbreitet waren:
vor allem gebrochene Schriften, aber auch einige Buchkursive und
Beispiele der sogenannten «humanistischen Schrift» (siehe Scarpatetti
2022: 55ff.). Für die Schrift wurden -- wie in klösterlichen
Handschriften des Mittelalters üblich -- die Farben schwarz und rot
verwendet.\footnote{Allerdings scheint schon auf den ersten Blick die im
  Mittelalter bestehende Konvention, dass der rote Text jeweils den
  «heiligen Teil» (also vor allem Texte aus der Bibel) darstellt, nicht
  beachtet worden zu sein.}

Wie von den Nutzungsbedingungen der Stiftsbibliothek und in der Präambel
vorgegeben, habe ich nur Papier und Bleistift mit in den Raum genommen.
Mache keine Photos. Lese und schreibe ab. (Aber wenn ich mich an den
Geist der Präambel halte, kann ich diese Abschriften nicht zitieren,
sondern muss sie bei Bedarf wohl wieder abschreiben und als Brief
verschicken oder direkt übergeben.)

Ich blättere die Waldhandschrift durch, lese sie. Dabei hilft mir, dass
ich gebrochene Schriften und Kursive lesen gelernt habe -- ohne das
würde ich wohl weniger entziffern können. Es hilft auch, dass ich die
Landessprachen zumindest alle lesen kann. Auffällt, dass erstaunlich
viele Texte in Rätoromanisch abgefasst sind. Als Bündner, also Bürger
des Kantons Graubünden, freut mich das. Diese Sprache ist Teil dessen,
was meinen Kanton besonders macht. Aber gleichzeitig: Wie viele Menschen
werden das verstehen können? Gleichzeitig finde ich nur wenige
italienische Texte. Eine ganze Anzahl der Texte, aber nicht alle, sind
zugleich in eine der anderen Landessprachen -- meist ins Deutsche --
übersetzt, was sie zumindest etwas mehr zugänglich macht.

Die Waldhandschrift ist ein zusammenhängendes Werk: Sie hat ein Thema,
den Wald und das Waldsterben. Das Pergament hat immer das gleiche Format
(soweit das bei Pergament möglich ist, also immer mit leichten
Unebenheiten, Kanten und auch Eigenheiten der jeweiligen einzelnen
Seite, beispielsweise leicht unterschiedlicher Farbe und Stärke). Die
Qualität des Schriftbildes und der Illustrationen ist
gleichbleibend.\footnote{Später lerne ich (in Scrapatetti \& Schäffel
  1991), dass das auch nicht ganz stimmt. Für ein Bild wurden explizit
  Farben in mittelalterlichen Verfahren hergestellt, für die anderen
  wurden industriell hergestellte, und deshalb käuflich zu erwerbende,
  genutzt. Wenn man die Orientierung an den mittelalterlichen Texten als
  Kriterium nimmt, hat also dieses eine Bild eine andere Qualität. Aber
  ich, als ungeübter Leser mittelalterlicher Codices, kann den
  Unterschied nicht erkennen, auch nicht, als ich noch einmal nach St.
  Gallen fahre und die Handschrift erneut durcharbeite.} Sie ist also
nicht einfach ein Codex, in dem ganz verschiedene Handschriften
zusammengebunden wurden. Sie ist aber auch gerade kein mittelalterliches
Werk. Es gibt zum Beispiel ein Inhaltsverzeichnis und ein Register der
Autor*innen, was bei mittelalterlichen Werken so nicht der Fall war.
(Duncan 2021)

Gleichzeitig ist die Waldhandschrift nicht «aus einem Guss». Die
Schriften und die Schriftgrössen, die Ausstattung, teilweise auch das
Layout, wechseln von Text zu Text. Es ist erkennbar das Werk mehrerer
Hände. Zudem lässt sie sich einfach blättern, einfacher als das meiner
Erfahrung nach bei anderen Codices der Fall ist. Sicherlich: Es ist
Pergament, nicht Papier. Also ist es schwerer. Das führt dazu, mit der
Handschrift bedächtiger umzugehen als mit einem Buch aus Papier. Aber
gleichzeitig ist die Bindung relativ neu, ebenso wie der lederne Einband
sowie der Lederkoffer, welcher für die Lagerung und den möglichen
Transport der Waldhandschrift geschaffen wurde.

Was mich interessiert und warum ich eigentlich hier bin: Das ist dieses
Objekt Waldhandschrift -- als Objekt. Als eine Art Kunstwerk, als Medium
in einer Bibliothek, als ökologisches Werk, kurzum: als eine Art Enigma.

\subsubsection{Im Rheintal}\label{im-rheintal}

7469 vor unserer Zeitrechnung, plus/minus zwei Jahre, stürzte ein Berg
im Rheintal. Die Trümmer stapelten sich und führten den Fluss, der heute
Vorderrhein beziehungsweise (im lokalen Dialekt des Rätoromanischen)
Rein Anteriur heisst, höher, als er zuvor floss. Die Felsen ragen jetzt
steil und schroff nach oben. Der Boden ist mit Steinen bedeckt. Ein Teil
dieser Fläche ist heute Naturschutzgebiet. Aber ein anderer Teil --
zwischen dem Bahnhof Versam-Safien und dem Bahnhof Valendas-Sagogn --
lässt sich erwandern. Auf diesem Weg läuft man grundsätzlich am Fluss
entlang, auch wenn man dabei einige Dutzend Höhenmeter überwindet.

Im Sommer ist dies, aus guten Gründen, ein beliebter Wanderweg: Er ist
leicht herausfordernd, aber nicht unüberwindbar. Der Ausblick ist
teilweise spektakulär. Es gibt eine Stelle, wo man ganz am Rand eines
Felsens steht und die ganz mit Steinen bedeckt ist. Einige Meter
darunter fliesst, an diesem Ort in ein engeres Flussbett gedrängt und
deshalb noch mächtiger als anderswo, der reissende Fluss. Gegenüber, am
anderen Ufer, steht ein Felsen, an dem die Bruchstellen des Bergsturzes
direkt sichtbar sind. So sieht es zumindest für mich aus. Ganz auf Höhe
des Flusses ist dort eine grosse, leere Höhle. Ein dunkles, fast
schwarzes Loch im grauen Felsen. Höher als ein Mensch. Wer daran glaubt,
dass die Natur belebt ist, wird diese Mächte dort fühlen. Praktisch
immer finden sich an dieser Stelle Steinhaufen, die so aufgeschichtet
sind, als wären sie Hinterlassenschaften von Ritualen -- als würden
einige Menschen sich dazu gedrängt fühlen, sie genau an diesem Ort zu
errichten. So ist die Landschaft dort.

Ich gehe diesen Weg auch zu anderen Jahreszeiten oft. Versam-Safien ist
keine halbe Stunde von meinem Wohnort entfernt. Über die Jahre ist diese
Strecke für mich zu einem Ort geworden, um den Kopf freizubekommen. An
einer Stelle, recht früh auf dem Weg, steht man praktisch direkt neben
dem gestürzten Berg. Auf der einen Seite der Felsen, gerne
hundertfünfzig Meter hoch, reiner Stein mit sichtbaren Bruchstellen. Auf
ihm wachsen Bäume, die in ihrem ganzen Leben nie von Menschen berührt
werden, weil sie so unerreichbar sind. Auf der anderen Seite der
schäumende Fluss. Wenn niemand anders auf dem Weg ist, wenn das Wetter
noch nicht oder nicht mehr perfekt ist, dann ist diese Stelle die Beste
auf dem ganzen Weg. Hier bleibe ich immer stehen und sehe dem Berg beim
Zerfallen zu.

Wenn man ruhig ist, wenn man nur steht und zuschaut, dann hört und sieht
man es nämlich: kontinuierlich weiter fallende Steine. Keine grossen
Brocken. Manchmal fast Steinstaub, aber gelegentlich auch kleine Steine.
Oft ist hier nichts zu hören, als das fliessende Wasser und die Steine,
die auf schon vor ihnen gefallenen Steinen aufschlagen. Tag und Nacht,
zu jeder Uhrzeit. Denn das ist etwas, was man in den Bergen mit ihren
sagenhaften Ausblicken oft vergisst: Sie bewegen sich, verändern sich.
Der Bergsturz im Rheintal ist auch jetzt, rund 9500 Jahre später, noch
nicht wirklich vorbei. Er ist immer noch im Gang und wenn er einmal
aufhört, wenn der Berg also ganz gestürzt ist (falls das je passieren
wird), dann werden wir alle nicht mehr da sein. Schon lange nicht mehr.

Man fühlt sich hier schnell irrelevant. Nicht nur man selber, sondern
die ganze Menschheit fühlt sich hier, am stürzenden Berg, irrelevant an.
Kein Wunder, dass Menschen Steinhaufen bauen und versuchen (vielleicht),
sich mittels Ritualen mit den hier wirkenden Kräften gut zu stellen.

\subsubsection{Eine Beschreibung der
Waldhandschrift}\label{eine-beschreibung-der-waldhandschrift}

Inhaltlich geht es in der Waldhandschrift um das Sterben der
schweizerischen Wälder -- in der damaligen Gegenwart Mitte der 1980er
Jahre, aber auch in den Jahrzehnten zuvor und in Zukunft. Sie ist auch
ein Aufruf zur Tat, zur Rettung dieser Wälder. In einigen Texten geht es
auch um die Rettung aller Wälder der Welt. Aber die Waldhandschrift, ihr
Inhalt und ihre Gestaltung sind lokal verankert: Die Verwendung der vier
Landessprachen und die Texte, welche alle von schweizerischen
Autor*innen stammen, verorten sie konkret in diesem einen
Land.\footnote{Ausnahme ist ein*e Autor*in aus der Schweiz, aber
  wohnhaft in Paris. Aber auch dies, Schweizer Autor*innen, die in Paris
  (oder Wien und Berlin) wohnen, gehört in gewisser Weise zur Schweiz.
  Auch sie -- solange sie sich weiter mit der Schweiz verbunden fühlen
  und immer wieder zurückkehren -- prägen dieses Land.} Die
Skriptor*innen, der Buchmaler, der Hersteller des Pergaments -- sie alle
sind zum Zeitpunkt der Erstellung der Waldhandschrift in der Schweiz
verortet. Auch der Inhalt ist schweizerisch: Orte, die genannt werden,
sind praktisch immer schweizerische. Verweise auf politische
Entscheidungen sind immer Verweise auf solche, die in der Schweiz
getroffen werden sollen. Sie wurde 1987 bei einer Art Ritual auf dem
Rütli bekannt gemacht. (Das Rütli ist eine Wiese mit hoher symbolischer
Bedeutung für die schweizerische nationale Identität. Auf ihm wurde dem
Mythos beziehungsweise einer alten mittelalterlichen Chronik nach 1291
durch drei Kantone die Eidgenossenschaft gegründet. Seit dem 19.
Jahrhundert wird das Rütli immer wieder als Ort von Ritualen mit Bezug
auf die Schweiz genutzt, und zwar aus fast allen politischen
Richtungen.)

Ebenso ist die Gestaltung der Handschrift als Objekt, das auf die
spätmittelalterliche Schriftkultur der mitteleuropäischen Klöster
verweist, eine lokale Verortung. Sie ist intentional: Es wäre zum
Beispiel möglich gewesen, auf die Schriftkulturen anderer Regionen
zurückzugreifen. Oder, in der Schweiz, auch auf Schriftkulturen aus
anderen Zeiten, vielleicht die des römischen Reiches oder die der
Renaissance. Aber es ging offenbar darum, ein schweizerisches Objekt zu
gestalten, kein internationales; eines, das mehr mit der Schweiz zu tun
hat als mit der «weiten Welt», welche mit einem Rückgriff auf das
römische Reich oder auf die Zeit der Omnipräsenz französischer Kultur
während der Renaissance wohl eher aufgerufen worden wäre. Deshalb liegt
sie im Kloster in St.~Gallen und nicht irgendwo anders.\footnote{Scarpatetti
  \& Schäffel (1991) stellen sie allerdings auch -- in einem Sammelband,
  der sich explizit der Pergamentforschung als historischer
  Hilfswissenschaft widmet -- explizit in den Zusammenhang von damaligen
  Versuchen, die Pergamentherstellung und -nutzung als Schreibmaterial
  selber zu beleben und über die Herstellung einer solchen Schrift -- im
  Sinne der experimental archeology, wie es heute heissen würde --,
  Wissen über das Pergament zu erarbeiten.}

Es gibt bislang keine literaturwissenschaftliche Auseinandersetzung mit
dem Inhalt der Waldhandschrift -- und angesichts dessen, dass dem Geist
der Präambel folgend nicht direkt aus ihr zitiert und sie nicht einfach
kopiert werden kann, ist es auch schwierig, sich vorzustellen, wie dies
passieren sollte. Deshalb vielleicht sind die Verweise auf den Inhalt,
die sich in einigen wissenschaftlichen Texten und vielen Presse- und
Magazinartikeln über die Waldhandschrift finden, recht oberflächlich: Es
gehe um den Wald und das Waldsterben. (Siehe zum Beispiel Ochsenbein
1987) Mehr wird sonst eigentlich nicht gesagt. Das Thema selber war
nicht so überraschend: Vielmehr stand es Mitte der 1980er Jahre (auch)
in der Schweiz auf der politischen Tagesordnung. Nicht nur bei
Umweltschutzverbänden, sondern auch in der Gesamtgesellschaft und
Politik galt das Waldsterben als eine ernsthafte Gefahr, der aktiv
begegnet werden müsse. Dazu trug auch das «Europäische Umweltjahr» 1987
bei, an dem sich die Schweiz -- obgleich nicht Mitglied der Europäischen
Union -- beteiligte. (Brändli et al.~1987) Die Beweggründe und Analysen
für das jeweilige Engagement waren damals sehr unterschiedlich. Aber
dass das Thema Bedeutung hatte, ist im Rückblick schon durch die damals
von zahlreichen Stellen publizierten Gutachten, Analysen,
Forderungskataloge und Aktionsvorschläge nachzuvollziehen. (Zum Beispiel
Aegerter \& Leder 1984, Schweizerische Gesellschaft für Umweltschutz
1984, Schweizerischer Bund für Naturschutz et al.~1985, Burkhard 1986)

Zur Waldhandschrift werden auch immer wieder die gleichen Daten
angegeben: 123 Autor*innen seien mit Texten vertreten, 33 Skriptor*innen
hätten sie geschrieben, ein Buchmaler hätte gemalt und ein Lederer hätte
das Pergament geliefert. (Siehe Scrapatetti \& Schäffel 1991, aber auch
an vielen anderen Stellen, zum Beispiel immer wieder in Meyer \& Welti
2016) Diese Zahlen klingen recht «magisch» (1-2-3, zweimal die drei) und
sind vielleicht auch so gemeint -- so wie in mittelalterlichen Quellen
eine Zahl oft etwas anderes bedeutet, als die angegebene Menge oder wie
in der Bibel «40 Jahre» nicht wirklich 40 Jahre heisst, sondern «eine
lange Zeit». Ich nämlich komme beim Nachzählen in der Waldhandschrift
auf leicht andere Zahlen, die nicht so «magisch» klingen: Es sind 138
Autor*innen, die mit Namen erwähnt werden, zudem gibt es Briefe von 30
Kindern (die teilweise mit einem Vornamen erwähnt, teilweise nur «ein
Kind» genannt werden) sowie 20 anonyme Texte. Skriptor*innen zähle ich
37 bis 39, die sich selber namentlich erwähnen.\footnote{Eine*r könnte
  einmal Initialen, einmal einen Namen verwendet haben, bei einem
  weiteren wird gesagt, er hätte den Text sowohl geschrieben als auch
  gestaltet -- was unterschiedliches heissen kann, aber halt auch, dass
  er selber als Skriptor tätig war.} Hinzu treten viele Texte, bei denen
keine Skriptor*innen erwähnt sind.

Was steht in den Texten selber? Wie sind sie zu lesen? Zuerst fällt auf,
dass sie -- mit einer Ausnahme -- allesamt recht kurz sind.
Grösstenteils sind es literarische Texte, vor allem Gedichte, sehr kurze
Erzählungen oder Reflexionen über den Zustand der Wälder und der Welt.
(So auch dargestellt in Scrapatetti \& Schäffel 1991.) Ein Beitrag
besteht aus anonymisierten Kinderbriefen an die schweizerische Politik.
({[}Das sind Briefe und Wünsche, welche um des Waldes Willen\ldots{]},
Waldhandschrift 1987: 123--128) Insgesamt geht es in den Texten vor
allem um Beschreibungen, also um das Feststellen des Status quo in den
1980er Jahren. Aber wirkliche Aufrufe zu irgendeiner konkreten Tat, eine
tiefergehende Analyse der Situation und der gesellschaftlichen
Möglichkeiten, den Wald «zu retten», findet sich fast nicht.\footnote{Das
  ist noch auffälliger, wenn man daneben die schon erwähnten anderen
  Texte aus der damaligen Zeit liest, in denen selbst bei
  repräsentativen Veranstaltungen wie der offiziellen Eröffnung des
  «Europäischen Umweltjahres» klare Ansagen über notwendige gesetzliche
  Regelungen gemacht werden (Brändli et al.~1987), ganz abgesehen von
  der Radikalität anderer Texte.}

Nur wenige Beiträge weichen davon ab. Der bei Weitem umfangreichste Text
(mit elf Textseiten) stellt in Form eines Manifests den Plan für ein
Dorf dar, welches so konstruiert wurde, dass es den Wald nutzt, ohne ihm
zu schaden. ({[}Silvania das Oekodorf{]}, Waldhandschrift 1987:
179--201) Heute würden wir es vielleicht ein «Zero Impact Dorf» nennen.
Es ist offenbar ein Verweis auf andere «Ökotopien», die in den 1980er
Jahren (und zuvor, aber auch heute\footnote{Allerdings neigt diese
  Tradition sich heute eher Dystopien zu. (Heller \& Nitzke 2022)})
populär waren: Unter Rückbezug auf das Regionale wurde damals immer
wieder versucht (nicht nur theoretisch, sondern auch in der Praxis),
lokale Gemeinschaften des Zusammenlebens und Wirtschaftens zu
etablieren, oft explizit für eine «überschaubare» Gruppe von Menschen
und deshalb nicht in einer Grossstadt. Diese Gemeinschaften sollten
möglichst umweltverträglich sein. Dabei wurde auch oft über den Sinn und
die Gefahren solcher Lebensweisen reflektiert. (Siehe für einen
zeitgenössischen Sammelband Greverus \& Haindl 1984) Der Text in der
Waldhandschrift stellt einen (weiteren) dieser Pläne dar. Gleichzeitig
ist der Beitrag aber auch ein Verweis auf den «St.~Galler Klosterplan»,
ein zwischen 819 und 830 unserer Zeit entstandener Entwurf für den
damaligen Neubau des Klosters St.~Gallen, welcher alle Funktionen eines
Reichsklosters modellhaft aufführt.\footnote{Der Rückverweis auf den
  Klosterplan ist aus der Waldhandschrift selber offensichtlich, wurde
  aber von Beat von Scarpatetti auch explizit bestätigt. (Scarpatetti
  2025)} Die letzte Seite des Textes ist eine Skizze des idealtypischen
Plans dieses «Oekodorfs», handgemalt in der Zeichensprache, welcher sich
auch der Klosterplan bediente.\footnote{Das Kloster wurde damals so
  geplant, dass es sich autark versorgen konnte und alle für das Leben
  der Mönche notwendigen Institutionen unterhielt. Insoweit ist die
  Parallele nicht unpassend. Der Klosterplan hat sich bis heute erhalten
  -- er ist in einem anderen Gebäude des Klosterbezirks in einer
  Ausstellung des Stiftsarchivs zu sehen, dabei allerdings eingelassen
  in die klaustrophobische Geschichte eines Mönchs, der mit acht Jahren
  von seinen Eltern dem Kloster übergeben wird und dann dieses nur noch
  verlässt, wenn der Abt dies anweist, sogar noch nach seinem Tod -- und
  gilt als der älteste erhaltene Plan eines europäischen Klosters.
  (Vergleiche das Digitalisat unter
  \url{http://dx.doi.org/10.5076/e-codices-csg-1092}.) Er ist auch oft
  untersucht worden, siehe beispielsweise Ochsenbein \& Schmuki (2002)
  oder Büker (2020).}

Die Texte in der Waldhandschrift sind in 14 Bücher unterteilt. Diese
Bücher beginnen immer mit einer Erwähnung in der Art: «Dieses ist das
{[}\ldots{]}te Buch, genannt {[}\ldots{]}». Einer mittelalterlichen
Handschrift gemäss ist diese Erwähnung aber nicht einheitlich, sondern
bei jedem Buch ein wenig abgewandelt und zudem immer wieder in einer
anderen Landessprache ausgeführt. Zudem wird der Titel des Buches als
Seitentitel geführt (jeweils pro Seite in einer der Landessprachen). In
jedem Buch, meist am Ende, findet sich zudem eine «Chronik», in der --
offenbar mittelalterlichen Vorbildern folgend -- bemerkenswerte
Ereignisse der (damals) letzten Jahre notiert wurden. Diese Chroniken
folgen den mittelalterlichen Vorbildern darin, dass sie auf genaue
Daten, Quellenangaben oder erkennbare Kriterien für die Auswahl
verzichten. (Alle Chroniken sind anonym, deshalb ist nicht klar, wer sie
verfasst hat.) Die Texte sind meistens, aber nicht immer, in zwei
Spalten pro Seite gesetzt. Tabelle 1 zeigt den Aufbau der
Waldhandschrift.

\begin{table}
\centering
    \begin{tabular}{llr}
\toprule
\textbf{Nummer des Buches} & \textbf{Buchtitel} & \textbf{Seiten} \\
\midrule
& Präambel & 2--3 \\
1 & Der Baum & 5--22 \\
2 & Die Zeit & 23--50 \\
3 & Der Wald & 51--82 \\
4 & Göttin und Gott & 83--94 \\
5 & Die Frau & 95--110 \\
6 & Der Mann & 111--122 \\
7 & Das Kind & 123--146 \\
8 & Der Tod & 123--146 \\
9 & Das Tier & 147--154 \\
10 & Die Erde & 167--174 \\
11 & Das Dorf & 175--194 \\
12 & Die Stadt & 195--206 \\
13 & Die Schweiz & 207--224 \\
14 & Die Tat & 225--230 \\
& {[}Inhaltsverzeichnis{]} & 231--234 \\
& {[}Autor*innenregister{]} & 234--236 \\
& {[}Abschlussgedicht{]} & Innenseite hinten \\
\bottomrule
\end{tabular}
\caption{Tabelle 1: Inhalt der Waldhandschrift}\
\end{table}

Ohne genaue Analyse ist es schwer zu begründen, aber die Texte
vermitteln immer den Eindruck, explizit in den 1980er Jahren geschrieben
worden zu sein. Die Sprache ist nicht mehr die, die man heute im
ökologischen Bereich erwarten würde. Nicht nur wird binär gedacht und
geschrieben (Mann oder Frau), sondern es wird auch immer wieder als
Lösung einfach der Verzicht auf bestimmte Dinge angepriesen (auf Konsum,
auf die Nutzung industriell produzierter Nahrungsmittel, auf das
Autofahren und auf Flugreisen, aber, anders als heute, offenbar nicht
auf das Fleischessen an sich). Oder die Rückkehr zu einer vorgeblich
früher vorhandenen Einheit von Mensch und Natur wird als Lösung
impliziert. Aber eine übergreifende soziale Utopie oder politische
Forderungen finden sich nicht. Dagegen spielt die Spiritualität eine
grosse Rolle -- in einer recht offenen, nicht unbedingt christlichen
Form, wie schon am Buchtitel «Göttin und Gott» sichtbar wird. Teilweise
führt das zu Texten, die affirmativ «Wilde» als die naturnäheren,
besseren Menschen zeichnen -- und dabei rassistische Bilder über diese
Menschen verbreiten, was heute aus guten Gründen nicht mehr opportun
ist. (Explizit im anonymen «Das ist die Anrufung der Sioux-Indianer»,
Waldhandschrift 1987: 89, aber auch an anderen Stellen.) Ebenso
zeittypisch sind dann auch die konkreten Verweise: So untersagt zum
Beispiel die Präambel explizit das Photographieren, Faksimilisieren oder
Drucken der Waldhandschrift (Waldhandschrift 1987: 2f.) -- aber nicht
die digitale Abbildung, die heute in diese Aufzählung eingefügt würde.

Es scheint so, als wenn es keine expliziten Kriterien dafür gab, wie die
Texte ausgewählt wurden. Zu diesem Thema verbleiben sowohl die Texte,
die über die Waldhandschrift veröffentlicht wurden, als auch der
eigentliche Organisator Beat von Scarpatetti in einem Gespräch mit dem
Autor, ein wenig im Unklaren.\footnote{In der Stiftsbibliothek lagern
  heute auch die archivierten Arbeitsakten von Beat von Scarpatetti,
  inklusive eines Teilbestandes zum Projekt Waldhandschrift. In Zukunft
  könnten diese durchgearbeitet und mehr über die eigentlichen
  Entscheidungsprozesse in diesem Projekt eruiert werden.} (Scarpatetti
2025) Es wurden offenbar Einladungen an Autor*innen verschickt, von
denen dann viele Texte zurücklieferten.\footnote{Die Einladungen sollen
  auf Pergament geschrieben worden sein. Diese -- oder der Text dieser
  Einladung -- sind offenbar nicht dokumentiert. Eventuell findet sie
  sich aber in den in der vorhergehenden Fussnote genannten
  Arbeitsakten.} (Scarpatetti 2025) Eine Auswahl und Bewertung scheinen
nicht stattgefunden zu haben. Soweit aus den Texten selber sichtbar
wird, scheint es sich jeweils um Arbeiten zu handeln, die explizit für
dieses Werk verfasst wurden. Die erwähnten Texte aus Kinderbriefen
stammen von einer Aktion des World Wide Fund for Nature (WWF), welcher
zudem die Waldhandschrift mitfinanzierte.

Das Ergebnis dieses Prozesses lässt sich auch in weiteren Zahlen
ausdrücken. Lässt man das Impressum und Texte, die nach 1987 hinzugefügt
wurden, aus, dann enthält die Waldhandschrift 168 Texte. Davon sind 88
Gedichte, 32 stellen Reflexionen über den Zustand der Wälder und der
Welt dar, 19 sind Kurzgeschichten. Zudem gibt es 14 Mal die schon
erwähnten Chroniken und drei explizite Lieder (zwei davon mit Noten).
Andere Textformen sind Ausnahmen: Spruch (1x), Prophezeiung (1x),
Aphorismus (1x), Gebet (1x), die erwähnten Briefe von Kindern (1x),
Reiseberichte (2x), Essays (2x), Präambeln (2x) und das schon erwähnte
Manifest (1x).

Wie zuvor dargestellt, sind die vier Landessprachen der Schweiz
vertreten.\footnote{Das Impressum (Waldhandschrift 1987: 223) ist in
  Latein gehalten, andere Sprachen wurden nicht verwendet.} Dabei
überwiegt das Deutsche: 109 Texte sind in Deutsch, zudem fünf in einem
schweizerdeutschen Dialekt und zwei explizit in Deutsch und Französisch
(im Wechsel innerhalb des Textes) verfasst. Französische Texte finden
sich 32 Mal, zwei sind zugleich in Französisch und Deutsch geschrieben
(hier überwiegt jeweils das Französische). Rätoromanische Texte finden
sich zwölf Mal, italienische vier Mal. Nur ein Text (die Präambel der
Waldhandschrift) ist in allen vier Sprachen verfasst. Schon gesagt
wurde, dass einige Texte auch übersetzt sind. Sechs deutschen Texten
folgen französische Übersetzungen, fünf französischen je eine deutsche.
Die italienischen Texte sind nicht übersetzt, aber alle rätoromanischen
liegen auch je in einer deutschen Version vor. Wer diese Übersetzungen
vornahm, ist nicht immer erkennbar. Manchmal waren es explizit die
Autor*innen, aber meist gibt es dazu keinen Hinweis. Das alles spiegelt
die Sprachlandschaft der Schweiz zum Teil wieder, insbesondere die hohe
Bedeutung des Deutschen. Aber beim Rätoromanischen und Italienischen
sind die Verhältnisse vertauscht: Das Italienische wird viel öfter
gesprochen und geschrieben, als in der Waldhandschrift repräsentiert,
das Rätoromanische weniger. (Bundesamt für Statistik 2023)

Für die Skriptor*innen und die Qualität ihrer Arbeit hingegen gab es
offenbar klarere Vorgaben. Zwar wird in der Waldhandschrift und der
dazugehörigen Literatur immer wieder betont, dass sie das Ergebnis der
Arbeit vieler Menschen war, aber treibende Kraft hinter ihr war der
schon genannte Beat von Scarpatetti, damals Handschriftenbibliothekar an
der Stiftsbibliothek St.~Gallen. Von Scarpatetti ist Mediävist und
Experte für mittelalterliche Handschriften. Aber neben seiner
eigentlichen Arbeit war er (und ist auch weiterhin, trotz Verrentung)
auch ökologisch engagiert. Zum Beispiel war er Gründer des Vereins «Club
der Autofreien» (Steiner 2016) und schrieb -- als aktualisierte
Umarbeitung der ersten helvetischen Verfassung von 1798 -- eine
«helvetische ökologische Verfassung» (Scarpatetti 1998), welche bei
Fragen der Umsetzung konkreter ist, als die Texte in der Waldhandschrift
und in der auch betont wird, dass Ökologie nicht auf ein Land beschränkt
werden kann, sondern global gedacht werden muss. Für die Waldhandschrift
entwarf von Scarpatetti den Plan, organisierte die Finanzierung und setzte
die Qualitätskriterien.

Ein Ergebnis der Handschrift war die Organisierung der Skriptor*innen in
der Schweiz selber. Diese seien zuvor untereinander nicht in Kontakt
gewesen, sondern erst für dieses Projekt zusammengebracht worden.
(Scarpatetti \& Schäffel 1991, Scarpatetti 2025) Es ist ohne weitere
Forschung schwer zu überprüfen, ob dies tatsächlich so war. Aber
offenbar gab es im Rahmen des Projektes Treffen von Skriptor*innen in
St.~Gallen, die es vorher nicht gab. Und heute gibt es tatsächlich eine
-- allerdings erst 1990 offiziell gegründete -- Schweizerische
Kalligraphische Gesellschaft.\footnote{Auffällig ist, dass diese sich
  «Kalligraphisch» nennt und gerade nicht auf Skriptorien bezieht. Aber
  sie umfasst auch Personen, die sich mit der Handschrift
  auseinandersetzen und sie praktizieren, und nicht nur solche, die sich
  auf mittelalterliche Schriften spezialisieren.} Und es gab, so wieder
die Auskunft von Beat von Scarpatetti selber, für die Waldhandschrift
Regeln dafür, was akzeptabel war und was nicht: Die Schriften in der
Waldhandschrift sollten sich an den Schriften orientieren, die im
Mittelalter genutzt wurden -- Vorbild waren die Handschriften, die sich
in der Stiftsbibliothek St.~Gallen selber befinden. Es durfte zum
Beispiel keine «Verschnörkelungen» geben -- warum, ist nicht klar
gesagt.\footnote{Dieses Verbot ist auch interessant, da andere
  Kompromisse sehr wohl eingegangen wurden. So ist nur ein Teil der
  Seiten (laut Scarpatetti \& Schäffel (1991: 160) «rund 90») mit
  Gänsekiel geschrieben worden, die anderen mit der -- für ein
  mittelalterliches Skriptorium anachronistischen -- metallenen
  Spitzfeder.} Aber zu vermuten ist, dass diese nicht «mittelalterlich»
sind, sondern erst in den Jahrhunderten später aufkamen.

\subsubsection{As slow as possible,
Halberstadt}\label{as-slow-as-possible-halberstadt}

ORGAN²/ASLSP ist ein 1987 von John Cage geschriebenes Musikstück. Cage
ist bekannt als Komponist herausfordernder, moderner Musik, insbesondere
für sein Stück 4'33. In Texten über Cage wird regelmässig geschrieben,
dass er vor allem Konzepte komponierte. (Metzger 2011, Haskins 2012)
ORGAN²/ASLSP ist ein solches Konzept und eine Herausforderung. Die
Notierung ist nur wenige Seiten lang, allerdings ist ASLSP eine
Abkürzung für \emph{as slow as possible}, was im Gegensatz zu dieser
Kürze als Aufforderung gemeint ist.

In Halberstadt, am Rand der Altstadt mit den Fachwerkhäusern, steht ein
aufgelöstes Kloster. Ein Teil seiner Gebäude ist nur noch zum Teil
erhalten, andere sind renoviert und werden von verschiedenen
Einrichtungen genutzt. Die Klosterkirche ist ebenso halb zerfallen --
eine Ruine, die so weit wiederhergestellt wurde, dass der Innenraum vor
Wetter geschützt ist. Aber ansonsten ist sie leer. Keine Kirchenbank,
kein Altar, kein Bild von Jesus, Maria oder Heiligen ist zu finden. Der
Boden ist teilweise Schotter.

In dieser Kirche stehen eine Orgel und zwei grosse Blasebälge. Es ist
keine Kirchenorgel, sondern ein extra konstruiertes Instrument, in das
im laufenden Betrieb Pfeifen ein- und ausgeschraubt werden können. Die
Orgel wird nicht über Tasten bedient. Stattdessen werden Gewichte an die
Pfeifen gehangen.

Diese Orgel spielt seit 2001 das Stück von Cage -- und zwar durchgehend
für eine Dauer von 639 Jahren. Das heisst, dass ein Ton durchgehend über
Monate oder Jahre gehalten wird. Es gab und wird auch weiterhin Pausen
geben -- weil diese in der Notierung vorgegeben sind --, in denen die
Orgel keinen Ton macht. Wann die Töne gewechselt werden, also wann
Pfeifen ein- und ausgeschraubt, Gewichte angehangen oder abgenommen
werden, ist vorhergesehen. Dies war bislang immer ein Ereignis, das von
einer Anzahl von Menschen gemeinsam begangen wurde. Aber auch sonst ist
die Orgel zugänglich.

Als wir die Orgel im Sommer 2024 besuchen, sind wir nicht alleine.
Gerade spielt sie einen dunklen, stumpfen Ton. Alle, die hier sind, sind
sehr ruhig und bedächtig. Die Kirche in ihrem halbzerfallenen Zustand
vermittelt den Eindruck einer ihr eigenen Beständigkeit. Ihre
eigentliche Funktion ist lange Vergangenheit. Das Kloster wurde 1810
aufgehoben, die Zahl der Katholik*innen liegt in Sachsen-Anhalt, wo sich
Halberstadt befindet, bei gegen 3 \%. (Statistische Ämter 2022) Es gibt
keinen Bedarf für eine Klosterkirche mehr. Und dennoch sind das Gebäude
und das alte Klostergelände da. Heute hat die Natur vieles besetzt.
Überall auf dem Gelände blühen Bäume, wachsen Gräser. Die Vögel hören
nicht auf, Laute zu machen.

Die Kirchruine und die Orgel zwingen praktisch dazu, sich langsam zu
bewegen, vorsichtig. Zuzuhören. Zuzuschauen. Es ist nur ein Ton, der
gespielt wird, ohne Veränderung. Aber die Orgel, hier in diesem Raum,
vermittelt auch ein Vertrauen auf etwas: Dass es möglich sein wird, das
Stück zu Ende zu spielen. Niemand wird das gesamte Stück in seiner
gesamten Länge erleben. Wer auch immer, wann auch immer, im Laufe dieser
639 Jahre vorbeikommt, wird immer nur einen Teil hören können. (Die Töne
werden einzeln aufgezeichnet und die schon gespielten können auch heute
auf der Homepage des Projektes abgespielt werden.\footnote{\url{https://www.aslsp.org/das-projekt.html}
  {[}Zugriff 08.04.2025{]}.} Aber das schnelle Abspielen ist ja etwas
anderes, als eigentlich vorgesehen.)

639 Jahre, an diesem Ort, sind eine lange Zeit. In den 639 Jahren vor
dem Projekt ist hier nicht nur das Kloster gewachsen, ständig ausgebaut
und dann doch aufgehoben worden, sondern es sind auch Staaten, die über
Halberstadt herrschten, entstanden und wieder untergegangen.
Religionsgemeinschaften sind hier gewachsen und wurden dann praktisch
vernichtet (die jüdische, wobei deren Spuren in den letzten Jahren
wieder sichtbar gemacht wurden) oder sind fast ganz eingegangen (die
christliche, insbesondere in der katholischen Variante). Menschen (und
Tiere) haben hier gelebt. Halberstadt ist reich geworden, dann in eine
Krise geraten -- und das mehrfach. All das ist jetzt vorbei und geht
gleichzeitig weiter. Aber die Kirchenruine steht und die Orgel in ihr,
die stupende ihren Ton spielt, zwingt den Gedanken auf, dass die Zeit
der Menschen nur eine individuelle Zeit ist. Dass ihre Probleme und
Wünsche, Ängste und Freuden lang sind, im Vergleich zum Leben der Vögel
vor dem Fenster -- aber kurz, sehr kurz im Vergleich zu anderem. Wenn
Menschen glauben, dass es eine höhere Macht gibt, dann wäre
verständlich, wenn sie hier den Eindruck hätten, diese zu spüren.

\subsubsection{Das Projekt Waldhandschrift als
Kulturkritik}\label{das-projekt-waldhandschrift-als-kulturkritik}

Warum soll man die Waldhandschrift eigentlich nur abschreiben? Fast alle
Texte über sie und ihre Präambel selber betonen dies. (Und, wie gesagt,
halte auch mich hier an diesen Wunsch.) Aber: Warum gibt es diesen
Wunsch eigentlich? Eine Antwort wäre, dass es im europäischen
Spätmittelalter, auf welches die Handschrift Bezug nimmt, vor allem die
Möglichkeiten des Lesens, Vorlesens und Abschreibens gab, um eine
Schrift wirken zu lassen. Dann könnte man die heutigen Möglichkeiten der
Reproduktion (oder die, welche es Mitte der 1980er Jahre gab), als Teil
der modernen Entwicklung der Gesellschaft sehen, die nach dem
Mittelalter einen falschen Weg genommen hat, der jetzt (in den 1980er
Jahren) dazu führt, dass der Wald stirbt. Die Präambel schweigt sich
dazu aber aus.

Dies, das Verbreiten von Literatur durch Abschreiben, Lesen und Vorlesen
in den 1980er Jahren, erinnerte mich, als ich im Lesesaal der
Stiftsbibliothek die Präambel abschrieb, an die Praxis des Samizdat in
den sozialistischen Staaten. In der DDR und anderen Ländern war dies
eine der Möglichkeiten, «unter der Hand», also ohne vom Staat beobachtet
zu werden, Literatur zu verbreiten. Sowohl solche, die verboten und
damit «gefährlich», als auch solche, die schwer zu erhalten war, da sie
zum Beispiel in zu geringer Stückzahl existierte, wurde über Jahrzehnte
abgeschrieben und weitergegeben. (Allerdings nicht nur per Hand, sondern
über alle Wege der Reproduktion, die sich damals irgendwie organisieren
liessen. In Camarade et al.~(2023) werden dafür beispielsweise
Schreibmaschinen, Stempeldruck oder Tonbänder aufgeführt.) Ist das Lesen
der Waldhandschrift also vielleicht als politischer Akt gedacht, der
Wissen unterhalb der staatlichen und gesellschaftlichen Strukturen
verbreiten will?

Oder ist dieses Gebot, die Waldhandschrift selber zu lesen und dann per
Hand abzuschreiben, eine praktische Kulturübung, die sich auf die
persönliche Lebenspraxis niederschlagen soll? Soll durch Verlangsamung
ein ökologischeres, «waldnaheres» Leben ermöglicht werden? Dadurch, dass
man sich auf den Text, auf die Schrift, konzentriert und dadurch, dass
man sie langsam reproduziert -- soll man dadurch lernen, auch langsamer,
achtsamer zu leben? Ist das also eine kulturelle Übung? Wurden deshalb
vielleicht auch die spätmittelalterlichen Schriften gewählt -- die zu
lesen für die meisten Menschen heute (oder 1987) eine Anstrengung
darstellen, weil das gerade nicht ihre alltägliche Praxis ist? Das war
meine Idee, als ich das erste Mal darauf stiess, dass es die
Waldhandschrift gibt.

Alles das stimmt nicht. Oder zumindest nicht vollständig. Es gibt
mehrere Texte, welche für das Projekt der Waldhandschrift
unterschiedliche Gründe angeben. Meine Überlegungen bestritt Beat von
Scarpatetti im Gespräch direkt (Scarpatetti 2025). Er gab mir auch einen
-- unvollständigen und undatierten, aber wohl von 1987 stammenden --
Projektantrag mit, den er bei der Stiftung «Pro Helvetia» eingereicht
hatte, um für die Finanzierung der «öffentlichen Präsentation»
(Scarpatetti 1987: 1) der Waldhandschrift anzufragen. In diesem schreibt
er, dass das Thema Wald «sicher gewählt worden {[}ist,{]} wegen des
Primats von Pflanze, Natur, Leben, aber auch, weil es am ehesten
Gegenstand eines nationalen Konsenses bildet» (Scarpatetti 1987: 9).
Sicherlich muss man bedenken, dass Pro Helvetia die Schweizer Kunst und
Kultur sowie ihre Repräsentation im Ausland fördert. Insoweit kann
dieser Satz im Antrag stehen, um der Stiftung ein Argument für die
Förderung zu unterbreiten, welches direkt in den Stiftungszweck fällt.
Aber im weiteren Antrag scheint es dann tatsächlich immer mehr, als ob
das eigentliche Thema der Waldhandschrift -- also der schweizerische
Wald und sein Sterben -- nicht unwichtig wären, aber doch nicht die
Haupttriebkraft hinter der ganzen Arbeit waren.

Es geht nämlich auch um Kultur, die bedroht sei: «In der
Massengesellschaft ist der Umgang mit Literatur heute verschlissen.
{[}\ldots{]} Der Akt des Abschreibens ist zentral als Akt der
Interiorisierung. Abschreiben ist eine Form sinnlicher und seelischer
Aneignung, die bereits eine Wiedergabe auf ebendiese Weise vorbereitet.
{[}\ldots{]} In der Schule ist heute der Zerfall von Schrift und
Formvermögen (Seitenbild) eklatant. {[}\ldots{]} {[}Die
Waldhandschrift{]} bringt {[}\ldots{]} den Versuch einer Stimulierung
der Freude an Literatur durch den spielerischen Reichtum von Form und
Farbe in einem faszinierenden Buch.» (Scarpatetti 1987: 10). Damit das
nicht falsch verstanden wird: Der Antrag spricht auch davon, dass die
Waldhandschrift eine «Kulturtat sein {[}will{]}» (Scarpatetti 1987: 10),
die auch ein ökologisches Ziel hat: «Die {[}Waldhandschrift{]} wird
heute selbst hergestellt, die Scriptoren {[}sic!{]} leben, zeigen
Handwerk und Atelier, ihr Produkt wird den Leuten in die Hände gegeben,
Mit- und Nachvollziehen sind möglich. Kreativer Umgang mit Kultur als
psychologische Brücke zur kreativen Mitgestaltung von Umwelt und
Politik.» (Scarpatetti 1987: 10)

Dem ökologischen Aktivisten Beat von Scarpatetti geht es um die
Ökologie, um aktives Handeln und um Kultur. Aber dem
Handschriftenbibliothekar und Mediävisten Beat von Scarpatetti geht es
auch um das Schreiben per Hand, um das Sammeln der Erfahrung mit
Pergament als Schreibmaterial (Scarpatetti \& Schäffel 1991), um die
mittelalterliche Schrift und Buchproduktion, um die Aura des Originals.
Um ein gewisses Zurück zu einer alten Kultur.\footnote{Was überwiegt und
  wie zum Beispiel die beteiligten Autor*innen, Skriptor*innen und
  anderen beteiligten Personen die Waldhandschrift interpretierten, wäre
  eine Frage für weitergehende Forschungen.}

Das eine Objekt, die Waldhandschrift, ist also -- wie alle Objekte der
Kultur -- offen für Interpretationen. Ich bringe meine intellektuelle
Biographie mit und interpretiere sie vor der damals, 1987,
zeitgenössischen Praxis des Samizdat. Von Scarpatetti interpretiert sie als
Kulturtat in Verbindung mit dem Kloster St.~Gallen. Wir treffen uns wohl
beide in der Sorge um die Umwelt und die Auswirkungen der
Massengesellschaft auf diese. Aber das Objekt Waldhandschrift sagt uns
nicht, wer näher an der Wahrheit ist. Dass es aus der Zeit gefallen und
doch so sehr mit seiner eigenen Zeit verbunden ist, zwingt heute zu
neuen Interpretationsleistungen. In gewisser Weise ist es unmöglich, die
Waldhandschrift nicht zu interpretieren, wenn man erst einmal von ihr
gehört hat.

Ich sitze in der Stiftsbibliothek und frage die Waldhandschrift: «Was
bist du? Was willst du? Zu welcher Zeit gehörst du?» Gut möglich, dass
in zweihundert Jahren -- wenn der Wald ganz verschwunden ist oder auch
wieder gesundet und lebendig -- jemand anders die gleichen Fragen an
dieses Objekt stellen wird, mit noch anderen Antworten.

\subsubsection{Über Sent}\label{uxfcber-sent}

Der schweizerische Winterwanderweg 355 führt vom Skigebiet Motta Naluns
am Berg entlang und dann hinab in das Dorf Sent. Man beginnt ihn, indem
man vom Bahnhof Scoul-Tarasp mit der Seilbahn (oder, wie es hier heisst,
den Pendicularas) hinauffährt, mit all den Personen, die hier auf den
Pisten Ski fahren wollen. Oben bei der ersten Bergstation angekommen --
beim Restaurant und der Information -- geht man dann zuerst an diesen
Pisten entlang. Mehrere kleinere Seilbahnen fahren dort im Skigebiet und
bringen Menschen auf noch weitere Pisten hinauf. Beim Wandern sind diese
alle zu sehen und die Menschen auf ihnen ständig zu hören. Erst nach
einer ganzen Weile, vielleicht nach dreissig oder fünfundvierzig
Minuten, lässt man die letzte Seilbahnstation hinter sich. Jetzt geht es
den Berg wirklich hinauf -- gar nicht so viel, vielleicht einige Dutzend
Meter Höhenunterschied, aber halt im Schnee, mit einem steilen Anstieg
und auf rund 2200 Meter über dem Meeresspiegel.

Noch eine lange Weile hört man hinter sich die Menschen auf den Pisten.
Aber dann beginnen sie leiser zu werden und ihre Laute ganz zu
verschwinden. Irgendwann schaut man sich um und sieht auch die Pisten
nicht mehr. Vor einem eine weite Fläche Schnee. Der Wanderweg ist mit
Maschinen geglättet und -- wie alle schweizerischen Winterwanderwege --
mit magentafarbenen Stöcken gekennzeichnet. Es ist also nicht so, als ob
man in einer verlassenen Wildnis wäre. «Der Mensch», andere Menschen,
sind hier gewesen. Aber man trifft sie nicht.

Jetzt, nach zwei, zweieinhalb Stunden Wandern, bin ich alleine. Schnee
vor und hinter mir. Manchmal ein Vogel. Links der Berg, den ich entlang
gehe. Rechts, hinter dem Abgrund und dem Tal, die nächsten Berge. Ihre
Gipfel tragen alle das rätoromanische «Piz», Gipfel oder Spitze, im
Namen: Piz Ajüz, Piz Lischana, Piz Triazza.

An einer Stelle, die sich anfühlt, als würde es danach auf dem Wanderweg
wieder hinab gehen, bleibe ich stehen. Es ist nichts zu sehen als Berge,
Schnee, Himmel und Wolken. Ich stehe am Rande einer grossen, weissen
Fläche. Drüben die Berge sehen kleiner aus, als sie sind. Aber alle
wild, abweisend und anziehend zugleich. Der Wind ist recht stark hier
oben, der Geruch des Schnees liegt über allem. Es fühlt sich an, als
wäre vieles egal. Morgen, aber auch in tausend Jahren, werden hier noch
Berge stehen. Sicher: Auch sie zerfallen, verändern sich. Bergrutsche
und Bergstürze sind eine zunehmende Gefahr in den Alpen. Doch auch ohne
sie gibt es Veränderung. Aber langsame. Wäre ich im Sommer hier, gäbe es
hier einen anderen Wanderweg, die Etappe 10 der Via Engiadina. Von dem
aus würde man auf die Bäume hinunterschauen. Die Baumgrenze liegt hier
in den Alpen bei rund 1800 Metern, auf dem Weg würde man knapp darüber
laufen. Der Wald würde nicht so gesund aussehen wie noch vor einigen
Jahren. Wohl mit immer mehr Lücken und immer mehr toten Bäumen.

Aber jetzt, im Winter, wo Schnee liegt (wo «hier noch Schnee liegt», wie
es allgemein heisst, denn die Menschen wissen, dass die Klimakatastrophe
auch hierzulande wirkt, selbst wenn sie dies im Alltag verdrängen und in
Volksabstimmungen zu grossem Teil gegen Regelungen stimmen, die die
Schweiz ökologischer machen würden), scheint es, als wäre das alles
egal. Die Natur, die Berge, sie waren hier, bevor wir da waren. Und sie
werden auch hier sein, wenn wir wieder weg sind. Es fühlt sich
gleichzeitig so an, als wäre ich Teil eines viel grösserem Ganzen, zu
dem alles gehört und als wäre ich, als wären wir alle als einzelne
Menschen, recht egal.

Ich gehöre nicht zu den Menschen, die das Gefühl haben, dass es eine
höhere Macht gäbe -- nicht, wenn ich in Klosterruinen stehe, nicht, wenn
ich Musik zuhöre, die länger klingen wird als ein Menschenleben und auch
nicht, wenn ich die Macht der Natur direkt vor mir wirken sehe. Auch
nicht, wenn ich in Büchern blättere, die daraufhin ausgelegt sind, uns
alle zu überdauern. Aber all das -- Objekte wie die Waldhandschrift, das
Projekt in Halberstadt, die Natur im Rheintal und oben auf den Alpen --
sind Erinnerungen daran, dass wir als einzelne Menschen uns in einem
anderen Zeitablauf bewegen als die menschliche Geschichte und diese noch
einmal in einem anderen Zeitlauf als die Geschichte der Natur.

355 ist ein Wanderweg. Man kann also so lange bleiben, wie man will. (Es
ist aber nicht klug, in der Dunkelheit am Berg zu sein, wenn man kein
Licht dabei hat.) Dann allerdings geht man weiter, bis zur nächsten
Hütte, von da ab eine Weile direkt auf der Piste entlang und dann ganz
hinab ins Tal, hinunter nach Sent.

\subsubsection{Die Objekte in Bibliotheken, die uns
überdauern}\label{die-objekte-in-bibliotheken-die-uns-uxfcberdauern}

Warum interessiert mich die Waldhandschrift eigentlich? Ich bin auf sie
gestossen, als ich die Festschrift für Beat von Scarpatettis' 75.
Geburtstag las. (Egli et al.~2016) Die hatte ich mir besorgt, nachdem
ich den langen Einführungstext gelesen hatte, den von Scarpatetti in seinem
kommentierten Katalog der Bibliothek des Heynlin von Stein vorangestellt
hatte. («Einleitung: Zur Person Heynlins, zum Katalog, zur
Interpretation der Glossen und zum Problem der Weltverachtung»,
Scarpatetti 2022: 11--113) Heynlin war am Ende des Mittelalters, im 15.
Jahrhundert, in Basel und im oberrheinischen Raum unter anderem
Prediger. Vorher war er aber zum Beispiel in Paris auch Professor und
dann Rektor der Sorbonne. Er selber war es, der um 1470 die
französischen Erstdrucker nach Paris holte und damit den Medienwandel
vom handgeschriebenen Codex zum gedruckten Buch mit auslöste. Diese
Einleitung von Scarpatetti geht auf die Bibliothek Heynlins, auf dessen
Biographie und Theologie ein. Aber sie vermittelt auch den Eindruck,
dass der Autor in ihr etwas über sich selbst reflektiert, nämlich die
(im Mittelalter oft diskutierte) Frage danach, ob die vita activa, das
aktive Handeln in der Welt, oder die vita contemplativa, das
zurückgezogene, intellektuelle Leben, vorzuziehen sei. Heynlin war in
seiner Zeit äusserst aktiv, zog sich aber am Ende seines Lebens ins
Kloster -- die kleine Klause in Basel -- zurück, wechselte also von der
vita activa zur vita contemplativa. In seinen Predigten rief er immer
wieder dazu auf, sich von der Welt ab- und dem geistigen Leben
zuzuwenden -- insoweit war dieser Rückzug auch konsequent. Von Scarpatetti
aber schien mir dies in seiner Einleitung abzulehnen, mit dem expliziten
Verweis auf die ökologischen Fragen, die in der heutigen Zeit drängend
seien\footnote{Das ist in den letzten Seiten der Einleitung, der
  «Konklusion» recht eindeutig: «Heynlins über Jahre unentwegte, ja
  verzweifelte Busspredigt an der Schwelle der Moderne musste u.a. an
  der Hauptbruchstelle des christlichen Weltverhältnisses scheitern, an
  der ‹Weltverachtung›. In säkularisierter Form führt das moderne
  ‹Umwelt›problem das transformierte Erbe dieser Weltverachtung der
  Öffentlichkeit drastisch vor Augen. Diese religions- und
  mentalitätsgeschichtliche Wurzel des heutigen Umweltdesasters ist noch
  nicht adäquat aufgearbeitet und bewusst.» (Scarpatetti 2022: 103) Und
  zu den Fragen, die sich ausgehend von Heynlins Bibliothek und der
  Umweltproblematik stellen: «Kann das Christentum Elemente zum Ausweg
  {[}aus der heutigen Situation, KS.{]} beitragen? Antwort: ja, aber es
  sind dann nicht die zentralen Botschaften, sondern aus prophetischem
  Charisma und der Grundstimmung der Parousie {[}die «Anwesenheit
  Gottes» in der Welt und die Erwartung der Wiederkehr Jesus, KS.{]}
  herausgewachsene wertvolle moralische Errungenschaften wie die
  Nächsten- und Feindesliebe und die Werte der Bergpredigt.»
  (Scarpatetti 2022: 103)} -- aber gleichzeitig hatte er als
Handschriftenbibliothekar einen Beruf, der gerade der vita contemplativa
zugeordnet werden könnte.

In der Festschrift wird die Waldhandschrift mehrfach erwähnt. (Mit
eigenem Kapitel: Egli et al.~2016: 43--53, aber auch immer wieder in den
anderen Texten.) Was mich an ihr faszinierte, war, dass sie einerseits
eine Antwort auf die Frage nach vita activa oder vita contemplativa zu
sein schien -- ein Objekt, in mühevoller Kleinarbeit erstellt, welches
aber Handeln in der Welt hervorbringen soll -- und gleichzeitig ein
Objekt, dem eine eigene Zeitlichkeit eingeschrieben wurde. Es soll
benutzt werden. Explizit geht es darum, dass die Waldhandschrift gelesen
und in ihr geblättert wird. Es wurden sogar, wie ich später erfuhr,
Seiten eingefügt, die nachträglich ausgefüllt werden können.
(Scarpatetti \& Schäffel 1991)\footnote{Sie wurden jetzt ausgefüllt mit
  einer Aufzählung von Taten, die einzelne Personen aus sechs Kantonen
  -- hier «Stände» genannt -- für die Umwelt vollbracht hatten. Neben
  der Aufzählung dieser Taten wurde in Deutsch und Französisch ein
  weiteres Manifest -- die «Charta Mensch Natur 1991» / «Charte de
  l'homme et de la nature» (Waldhandschrift 1987: 223 f.) -- eingefügt.}
Damit ist die Waldhandschrift ein Objekt, dem in gewisser Weise ein Ende
eingeschrieben ist. Denn egal, wie gut eine Handschrift erhalten wird:
Benutzung lässt Spuren zurück und beschleunigt den Zerfall. Schon heute,
nicht ganz vierzig Jahre nach der Erstellung, hat sie Nutzungsspuren.
Aber es ist ein Zerfall, der wohl langsam vonstattengehen wird. Weit
mehr als ein Menschenleben, eher mehrere Generationen lang. Es scheint
auch dieser Waldhandschrift eine gewisse Hoffnung mitgegeben worden zu
sein, nämlich dass sie einmal nicht mehr benutzt werden muss -- weil, um
es einfach zu sagen, der Wald gerettet ist.

Mich interessiert auch, dass als Objekt für dieses Projekt gerade ein
Buch gewählt und dann in einer Bibliothek mit einer langen Geschichte
untergebracht wurde. Sicherlich: Das hat auch mit der Biographie und der
Arbeit des Organisators, also Beat von Scarpatetti, zu tun. Aber das
wusste ich noch nicht, als ich auf die Waldhandschrift stiess und
dennoch erschien mir diese Wahl (also Buch und Bibliothek) sofort
relevant, als ich dann das erste Mal von ihr las. Zudem hat es ja nicht
nur ihn überzeugt, sondern eine ganze Reihe von Autor*innen,
Skriptor*innen und anderen Personen sowie eine Zahl von mittelgebenden
Organisationen.

\begin{center}\rule{0.5\linewidth}{0.5pt}\end{center}

Ich weiss nicht, auch jetzt, nachdem ich die Waldhandschrift gesehen und
in ihr gelesen habe, ob mich das Objekt als «Kulturtat» überzeugt.
Umweltschutz, Politik an sich, ist für mich eher das Feld von Aktion:
Organisierung, politische Strategien, Machtkämpfe, konkretes Handeln.
Ich kann den Drang zu aktiven Taten, die Andres Malm in seinem Buch «How
to blow up a pipeline» (Malm 2021) propagiert, viel eher nachvollziehen,
als die langsame, teilweise introspektive Vorgehensweise, welche die
Waldhandschrift exemplifiziert. Doch Ökologie wird immer aus sehr
verschiedenen Ansätzen heraus betrieben, insoweit ist es nicht zu
überraschend, dass ich nicht ganz mit der Waldhandschrift übereinstimme.
Nicht zuletzt, weil sie an vielen Stellen sehr zeittypisch die Schweiz
der 1980er Jahre widerspiegelt -- nicht die des Jahres 2025. Und: Es
geht bei ihr halt offenbar nicht nur um das Waldsterben. Es geht auch um
mehr. Beispielsweise um Handschrift, um den Umgang mit Pergament, um
Kultur, um Literatur und darum, etwas im lokalen Rahmen zu tun. Es geht
darum, so würde ich es jetzt interpretieren, ein wenig aus der Zeit zu
fallen, um damit zu einem anderen Handeln und Denken zu motivieren.
Dieses Fallen aus der Zeit, das ich ebenso bei Projekten wie dem in
Halberstadt verspüre oder wenn ich mich alleine in der Natur
wiederfinde, scheint mir dann auch wieder eine hilfreiche Praktik zu
sein, um mit den Katastrophen, in denen wir uns alle befinden,
umzugehen. Aber sie führt halt nicht direkt zum aktiven Handeln.

Interessant scheint mir aber, dass auch dies eine Funktion sein kann,
wie ein Buch und eine Bibliothek genutzt wird. Eine, die, wie mir
scheint, heute kaum reflektiert wird -- und wenn doch, dann eine, die
wohl schnell im Ruf des Rückwärtsgewandten oder Spirituellen steht,
obwohl sie als Praxis für alle Menschen in diesen, unseren
katastrophalen Zeiten hilfreich sein kann. Ich würde beispielsweise
beides (rückwärtsgewandt zu sein oder spirituell) persönlich von mir
weisen, und dennoch scheint mir ein temporäres «aus der Zeit fallen»
sehr hilfreich. Letztlich kann ich mich der Faszination für das Projekt
Waldhandschrift nicht entziehen.

Von der Waldhandschrift geht ein grosser Optimismus aus: Es ist möglich,
alte Techniken für aktuelle Themen einzusetzen. Es ist möglich, dass
Menschen zusammenarbeiten, um ein solches Projekt, das ja zum Beispiel
gar keinen finanziellen Gewinn haben kann, umzusetzen. Es ist möglich,
Menschen für ein solches Projekt zu gewinnen, das über ihr Menschenleben
hinauswirken soll.\footnote{Neben der Waldhandschrift findet sich im
  Bestand der Stiftsbibliothek eine «Kleine Waldhandschrift» (1989), die
  eine Abschrift der Waldhandschrift darstellt -- auch auf Pergament und
  in «mittelalterlicher Schrift», aber in einem kleineren Format. Ganz
  im Sinne von mittelalterlichen Kopien stimmt diese nicht genau mit der
  Waldhandschrift überein, sondern ist in gewisser Weise eine
  Interpretation. Die Kleine Waldhandschrift scheint viel eher ein Werk
  aus einer Hand zu sein. Die Schrift ist nicht durchgängig die gleiche,
  aber das Schriftbild ist doch einheitlicher als in der grossen
  Handschrift. Es werden mehr graphische Elemente genutzt,
  beispielsweise sind Übersetzungen in kleinerer Schrift um den
  Originaltext herum gesetzt oder aber in Zwischenzeilen. Einige Texte
  wurden ergänzt, unter anderem durch Nachträge von Autor*innen. Zudem
  wurde an den Plan des «Oekodorfes» ein weiteres Manifest angefügt, der
  «Codex Raetus» geschrieben von «Anonymus Raetus». (Wobei sich Raetus
  offensichtlich auf «Raetia» bezieht, dem Namen einer römischen
  Provinz, welcher sich selber auf einen dort lebenden «Stamm» bezog,
  der vom römischen Reich unterworfen wurde. Seit dem 19. Jahrhundert
  werden Begriffe wie Raetia, Rhätien oder rhätisch für das Kanton
  Graubünden verwendet, heute beispielsweise bei der «Rhätischen Bahn».)
  Dies sind nur einige der Unterschiede. In der Mediävistik wird heute
  bei mittelalterlichen Texten immer von «Textzeugen» gesprochen und
  davon ausgegangen, dass es bei diesen Texten niemals einen gültigen
  «Urtext» gibt, da alle diese unterschiedlichen Versionen (auch die,
  die nicht bis heute überliefert wurden) wirksam waren. Die «Kleine
  Waldhandschrift» generiert quasi eine Form dieses «Textzeugen»,
  obgleich wir mit der Waldhandschrift eigentlich ein Original vorliegen
  haben. Aber gerade, wenn man die beiden Handschriften im Sinne der
  experimental archeology als Versuche betrachtet, die Arbeit in
  mittelalterlichen Skriptorien zu reproduzieren, ist dies nur
  folgerichtig -- ein Grossteil der dortigen Arbeit bestand nicht im
  Erstellen ganz neuer Werke, sondern in der Reproduktion schon
  vorliegender. Die Skriptor*innen schrieben vor allem Werke ab, die sie
  als Objekt schon kannten und vor sich liegen hatten. Dies ist in der
  kleinen Handschrift reproduziert. Eine Untersuchung der Unterschiede
  zwischen diesen beiden Waldhandschriften steht noch aus.}

\subsubsection{Literatur}\label{literatur}

Aegerter, Irene ; Leder, Rudolf A. (1984). \emph{Waldsterben}.
(EF-Dokumentation, 26) Bern: Energieforum Schweiz, 1984

Brändli, Niklaus ; Dietrich, Elsbeth ; Kirchhofer, Markus (1987).
\emph{Umweltpolitik 1987: Standortbestimmung und Perspektiven}.
(Schriftenreihe des Stapferhauses auf der Lenzburg, 17) Aarau: Verlag
Sauerländer, 1987

Büker, Dieter (2020). \emph{In Neuem Licht -- der Klosterplan Von
St.~Gallen: Aspekte seiner Beschaffenheit und Erschaffung.} Frankfurt am
Main: Peter Lang, 2020

Burkhard, HP. ; Hanser, Ch. ; Meier, U. ; Troxler, Ch. ; Fischer, U.
(1986). \emph{Die wirtschaftlichen Folgen des Waldsterbens in der
Schweiz}. Zürich: Schweizerische Gesellschaft für Umweltschutz, 1986

Bundesamt für Statistik ({[}2023{]}). \emph{Sprachen Hauptsprachen,
Sprachen, die zu Hause und bei der Arbeit gesprochen werden.}
\url{https://www.bfs.admin.ch/bfs/de/home/statistiken/bevoelkerung/sprachen-religionen/sprachen.html}
{[}Zugriff: 08.04.2024{]}

Camarade, Hélène; Galmiche, Xavier ; Jurgenson, Luba (dir.) (2023).
\emph{Samizdat: Publications clandestines et autoédition en Europe
centrale et orientale (années 1950-1990)}. Paris: Nouveau Monde
éditions, 2023

Duncan, Dennis (2021). \emph{Index, A history of the : a bookish
adventure}. London : Allen Lane, 2021

Egli, Daniel (Hrsg.), Meyer, Kurt (Redak.), Welti, Manfred (Redak.)
(2016). \emph{Kultur und Ökologie: Festschrift zum 75. Geburtstag von
Beat von Scarpatetti}. Binningen: Verein Ökogemeinde Binningen, 2016

Greverus, Ina-Maria ; Haindl, Erika (Hrsg.) (1984). \emph{ÖKOlogie
PROvinz REGIONalismus.} (Notizen, 16) Frankfurt am Main: Institut für
Kulturantrophologie und Europäische Ethnologie, 1984

Heller, Jakob Christoph ; Nitzke, Solvejg (2022). \emph{Ökologische und
postapokalyptische Idyllen im 20. und 21. Jahrhundert}. In: Gerstner,
Jan ; Heller, Jakob Christoph ; Schmitt, Christian (Hrsg.). Handbuch
Idylle: Verfahren -- Traditionen -- Theorien. Berlin: J. B. Metzler,
2022: 271--276

Kleine Waldhandschrift (1989). \emph{Kleine Waldhandschrift}
(Stiftsbibliothek: Cod. Sang. 348a)

Metzger, Christoph (2011). \emph{John Cage: Abstract Music. Zwölf
Vorlesungen}. Saarbrücken, Pfau Verlag, 2011

Haskins, Rob (2012). \emph{John Cage}. (Critical Lives). London:
Reaktion Books, 2021

Malm, Andreas (2021). \emph{How to blow up a pipeline: learning to fight
in a world on fire}. London, New York: Verso, 2021

Ochsenbein, Peter (1987). \emph{Die St.~Galler Waldhandschrift.} In:
Schweizer Monathefte: Zeitschrift für Politik, Wirtschaft, Kultur 67
(1987) 12: 1014--1017

Ochsenbein, Peter ; Schmucki, Karl (Hrsg.) (2002). \emph{Studien zum St.
Galler Klosterplan II}. (Studien zur vaterländischen Geschichte, 52) St.
Gallen: Historischer Verein des Kantons St.~Gallen, 2002

Scarpatetti, Beat von ({[}1987{]}). \emph{Projekbeschreibung «St.~Galler
Waldhandschrift» und Finanzgesuch für öffentliche Präsentationen in den
vier Sprachgebieten der Schweiz und im benachbarten Ausland}. Basel, St.
Gallen, {[}1987{]}

Scarpatetti, Beat von ; Schäffel, Klaus-Peter (1991). \emph{33 Schreiber
auf 119 Folia Pergaments: Ein schweizerischer Erfahrungsbericht zur «St.
Galler Waldhandschrift» (1986/87)}. In: Rück, Peter (Hrsg.) (1991).
Pergament: Geschichte, Struktur, Restaurierung, Herstellung.
(Historische Hilfswissenschaften, 2) Sigmaringen: Jan Thorbecke Verlag,
1991: 159--167

Scarpatetti, Beat von (1998). \emph{Helvetische ökologische Verfassung :
constitution helvétique 1798 -- ökologische Verfassung 1998 : ein
Entwurf}. Basel: Schwabe, 1998

Scarpatetti, Beat von (Hrsg.) (2022). \emph{Bücherliebe und
Weltverachtung : die Bibliothek des Volkspredigers Heynlin von Stein und
ihr Geheimnis}. Basel: Schwabe Verlag, 2022

Scarpatetti, Beat von (2025). Persönliches Gespräch mit dem Autor, St.
Gallen, 02.04.2025

Statistische Ämter (2022). \emph{Bevölkerung nach Religionszugehörigkeit
im Zensus 2022 und im Zensus 2011 - je Bundesland},
\url{https://www.zensus2022.de/DE/Ergebnisse-des-Zensus/Sonderauswertung_Religionszugehoerigkeit.html}
{[}Zugriff: 08.04.2025{]}

Steiner, Dieter (2016). \emph{Ein wahrer «Automobilist»}. In: Egli,
Daniel (Hrsg.), Meyer, Kurt (Redak.), Welti, Manfred (Redak.) (2016).
\emph{Kultur und Ökologie: Festschrift zum 75. Geburtstag von Beat von
Scarpatetti}. Binningen: Verein Ökogemeinde Binningen, 2016: 99--106

Schweizerischer Bund für Umweltschutz ; Schweizerische Gesellschaft für
Umweltschutz ; Verkehrs-Club der Schweiz ; World Wildlife Fund Schweiz
(1985). \emph{Tut etwas Mutiges!: Abschätzung der Wirksamkeit
beschlossener, versprochener und zusätzlich notwendiger Massnahmen gegen
das Waldsterben}. Zürich: infras, 1985

Schweizerische Gesellschaft für Umweltschutz (1984). \emph{Massnahmen
gegen das Waldsterben.} Zürich: infras, 1984

Vereinigung Bündner Umweltschutzorganisationen VBU (1989). \emph{Die
Waldhandschrift in Graubünden: 11. August bis 3. November 1989.} In:
Bündner Schulblatt 49 (1989/90) 1: 98--100

Waldhandschrift (1987). \emph{St.~Galler Wandhandschrift}.
(Stiftsbibliothek: Cod. Sang. 1999)

\subsubsection{Anhang: Methodik}\label{anhang-methodik}

Die im Text angeführten Zahlen über die konkrete Waldhandschrift wurden
(ausser, wenn anders angegeben) vom Autor selber erhoben. Dazu wurde die
Waldhandschrift im April 2025 bei mehrfachen Besuchen vor Ort in der
Stiftsbibliothek St.~Gallen durchgesehen. Dabei wurden pro Beitrag
erhoben:

\begin{itemize}
\item
  Namen der Autor*innen.
\item
  Titel der Beiträge.
\item
  Seitenzahlen.
\item
  Die Sprachen, in denen der jeweilige Beitrag vorliegt. Bei Beiträgen
  in mehreren Sprachen wurde die jeweils zuerst in der Waldhandschrift
  stehende als Originalsprache, die andere (oder anderen) als
  Übersetzungssprachen gezählt.
\item
  Die Textform des jeweiligen Beitrags wurde bestimmt. Dabei wurde auf
  die Titel, den Aufbau und Inhalt der Beiträge Bezug genommen.
  Gleichzeitig fand diese Zuteilung (unter anderem wegen der begrenzten
  Zeit, die dafür zur Verfügung stand) ad hoc statt. Sie ist also eine
  mögliche Quelle von Kritik. Andere Personen könnten -- insbesondere,
  wenn sie mit klareren Definitionen und Kriterien für verschiedene
  Textformen arbeiten -- zu anderen Ergebnissen kommen. Zu vermuten ist
  aber, dass so überprüfte Ergebnisse eher in Einzelfällen, aber nicht
  vom hier im Text skizzierten Gesamtbild über die Waldhandschrift und
  deren Inhalt abweichen werden.
\item
  Die Zahl der Illustrationen und der gesondert buchmalerisch
  gestalteten Initialen und Textränder wurde ausgezählt.
\item
  Die Namen oder Initialen der Skriptor*innen wurden, wenn diese solche
  hinterliessen, ebenso ausgezählt. Dies ersetzt selbstverständlich
  keine kritische Analyse der einzelnen Texte, die aufgrund anderer
  Kriterien einzelne Skriptor*innen identifizieren können. Es ging bei
  dieser Auszählung um ein erstes Bild über die Anzahl der beteiligten
  Personen.
\item
  Ausgelassen wurden bei der Auszählung (a) das (lateinische) Impressum
  (Waldhandschrift 1987: 168) sowie die Teile, welche nach 1987 ergänzt
  wurden (Waldhandschrift 1987: 225--230). Ersteres ist nicht
  inhaltlicher Bestandteil der Handschrift, letzteres war nicht Teil der
  Handschrift, als diese verfasst wurde.
\end{itemize}

Diese Auszählung wurde mehrfach überprüft. Ebenso wurde bei
Unklarheiten, die bei der Datenaufnahme auftraten, versucht, diese bei
späteren Besuchen in der Stiftsbibliothek anhand der Waldhandschrift zu
klären. Dennoch ist klar, dass dieses Verfahren -- welches durch die
Vorgaben zur Handhabung der Waldhandschrift und die vorhandenen
Ressourcen (insbesondere die zur Verfügung stehende Zeit) bedingt waren
-- in weiteren Forschungen verbessert werden kann.

%autor
\begin{center}\rule{0.5\linewidth}{0.5pt}\end{center}

\textbf{Karsten Schuldt}, wissenschaftlicher Projektleiter am
Schweizerischen Institut für Informationswissenschaft (SII), FH
Graubünden und Redakteur der LIBREAS. Library Ideas.

\end{document}