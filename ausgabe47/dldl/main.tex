\documentclass[a4paper,
fontsize=11pt,
%headings=small,
oneside,
numbers=noperiodatend,
parskip=half-,
bibliography=totoc,
final
]{scrartcl}

\usepackage[babel]{csquotes}
\usepackage{synttree}
\usepackage{graphicx}
\setkeys{Gin}{width=.4\textwidth} %default pics size

\graphicspath{{./plots/}}
\usepackage[ngerman]{babel}
\usepackage[T1]{fontenc}
%\usepackage{amsmath}
\usepackage[utf8x]{inputenc}
\usepackage [hyphens]{url}
\usepackage{booktabs} 
\usepackage[left=2.4cm,right=2.4cm,top=2.3cm,bottom=2cm,includeheadfoot]{geometry}
\usepackage[labelformat=empty]{caption} % option 'labelformat=empty]' to surpress adding "Abbildung 1:" or "Figure 1" before each caption / use parameter '\captionsetup{labelformat=empty}' instead to change this for just one caption
\usepackage{eurosym}
\usepackage{multirow}
\usepackage[ngerman]{varioref}
\setcapindent{1em}
\renewcommand{\labelitemi}{--}
\usepackage{paralist}
\usepackage{pdfpages}
\usepackage{lscape}
\usepackage{float}
\usepackage{acronym}
\usepackage{eurosym}
\usepackage{longtable,lscape}
\usepackage{mathpazo}
\usepackage[normalem]{ulem} %emphasize weiterhin kursiv
\usepackage[flushmargin,ragged]{footmisc} % left align footnote
\usepackage{ccicons} 
\setcapindent{0pt} % no indentation in captions
\usepackage{xurl} % Breaks URLs

%%%% fancy LIBREAS URL color 
\usepackage{xcolor}
\definecolor{libreas}{RGB}{112,0,0}

\usepackage{listings}

\urlstyle{same}  % don't use monospace font for urls

\usepackage[fleqn]{amsmath}

%adjust fontsize for part

\usepackage{sectsty}
\partfont{\large}

%Das BibTeX-Zeichen mit \BibTeX setzen:
\def\symbol#1{\char #1\relax}
\def\bsl{{\tt\symbol{'134}}}
\def\BibTeX{{\rm B\kern-.05em{\sc i\kern-.025em b}\kern-.08em
    T\kern-.1667em\lower.7ex\hbox{E}\kern-.125emX}}

\usepackage{fancyhdr}
\fancyhf{}
\pagestyle{fancyplain}
\fancyhead[R]{\thepage}

% make sure bookmarks are created eventough sections are not numbered!
% uncommend if sections are numbered (bookmarks created by default)
\makeatletter
\renewcommand\@seccntformat[1]{}
\makeatother

% typo setup
\clubpenalty = 10000
\widowpenalty = 10000
\displaywidowpenalty = 10000

\usepackage{hyperxmp}
\usepackage[colorlinks, linkcolor=black,citecolor=black, urlcolor=libreas,
breaklinks= true,bookmarks=true,bookmarksopen=true]{hyperref}
\usepackage{breakurl}

%meta
%meta

\fancyhead[L]{Redaktion LIBREAS\\ %author
LIBREAS. Library Ideas, 47 (2025). % journal, issue, volume.
\href{https://doi.org/10.18452/x}{\color{black}https://doi.org/10.18452/x}
{}} % doi 
\fancyhead[R]{\thepage} %page number
\fancyfoot[L] {\ccLogo \ccAttribution\ \href{https://creativecommons.org/licenses/by/4.0/}{\color{black}Creative Commons BY 4.0}}  %licence
\fancyfoot[R] {ISSN: 1860-7950}

\title{\LARGE{Das liest die LIBREAS, Nummer \#16 (Frühling–Sommer 2025)}}% title
\author{Redaktion LIBREAS} % author

\setcounter{page}{1}

\hypersetup{%
      pdftitle={Das liest die LIBREAS, Nummer \#16 (Frühling–Sommer 2025)},
      pdfauthor={Redaktion LIBREAS},
      pdfsubject={LIBREAS. Library Ideas, 47 (2025)},
      pdfkeywords={Literaturübersicht, Bibliothekswissenschaft, Informationswissenschaft, Bibliothekswesen, Rezension, literature overview, library science, information science, library sector, review},
      pdflicenseurl={https://creativecommons.org/licenses/by/4.0/},
      pdfcopyright={CC BY 4.0 International},
      pdfcontacturl={http://libreas.eu},
      pdfurl={},
      pdfdoi={},
      pdflang={de},
      pdfmetalang={de}
     }



\date{}
\begin{document}

\maketitle
\thispagestyle{fancyplain} 

%abstracts

%body
Beiträge von Eva Bunge (eb), Najko Jahn (nj), Ben Kaden (bk), Karsten
Schuldt (ks)

\section{1. Zur Kolumne}\label{zur-kolumne}

Ziel dieser Kolumne ist es, eine Übersicht über ausgewählte in der
letzten Zeit erschienene bibliothekarische, informations- und
bibliothekswissenschaftliche sowie für diesen Bereich interessante
Literatur zu geben. Enthalten sind Beiträge, die der LIBREAS-Redaktion
oder anderen Beitragenden als relevant erschienen.

Eine Themenvielfalt sowie ein Nebeneinander von wissenschaftlichen und
nicht-wissenschaftlichen Ansätzen wird angestrebt. Auch in der Form
sollen traditionelle Publikationen ebenso erwähnt werden wie andere
Medieninhalte, beispielsweise Blogbeiträge oder Videos beziehungsweise
TV-Beiträge.

Wir freuen uns über Hinweise auf interessante Publikationen. Diese bitte
an die Redaktion richten. (Siehe
\href{http://libreas.eu/about/}{Impressum}, Mailkontakt für diese
Kolumne ist
\href{mailto:zeitschriftenschau@libreas.eu}{\nolinkurl{zeitschriftenschau@libreas.eu}}.)

Die Koordination der Kolumne liegt bei Karsten Schuldt, verantwortlich
für die Inhalte sind die jeweiligen Beitragenden. Die Kolumne
unterstützt den Vereinszweck des LIBREAS-Vereins zur Förderung der
bibliotheks- und informationswissenschaftlichen Kommunikation.

LIBREAS liest gern und viel Open-Access-Veröffentlichungen. Wenn sich
Beiträge dennoch hinter einer Bezahlschranke verbergen, werden diese
durch \enquote{{[}Paywall{]}} gekennzeichnet. Zwar macht das Plugin
\href{http://unpaywall.org/}{Unpaywall} das Finden von legalen
Open-Access-Versionen sehr viel einfacher. Als Service an der
Leserschaft verlinken wir jedoch auch direkt OA-Versionen, die wir vorab
finden konnten. Für alle Beiträge, die dann immer noch nicht frei
zugänglich sind, empfiehlt die Redaktion (neben
\href{http://unpaywall.org/}{Unpaywall}) die Browser-Plugins
\href{https://openaccessbutton.org/}{Open Access Button} oder
\href{https://core.ac.uk/services/discovery/}{CORE} zu nutzen sowie auf
dem favorisierten Social-Media-Kanal mit
\href{https://mastodon.social/tags/icanhazpdf}{\#icanhazpdf}, um Hilfe
bei der legalen Dokumentenbeschaffung zu bitten.

Die bibliographischen Daten der besprochenen Beiträge aller Ausgaben
dieser Kolumne finden sich in der öffentlich zugänglichen Zotero-Gruppe:
\url{https://www.zotero.org/groups/4620604/libreas_dldl/library}.

\section{2. Artikel und
Zeitschriftenausgaben}\label{artikel-und-zeitschriftenausgaben}

\subsection{2.1 Vermischte Themen}\label{vermischte-themen}

Lupu, Viorica ; Țurcan, Nelly ; Cujba, Robica (2025). \emph{Moldovan
academic librarians' perception on research data management.} In: IFLA
Journal 2025, Online First,
\url{https://doi.org/10.1177/03400352241304118} {[}Paywall{]}
{[}\href{http://repository.utm.md/handle/5014/29162}{OA-Version}{]}

Die Studie präsentiert die Ergebnisse einer Onlineumfrage zu Ansichten
Wissenschaftlicher Bibliothekar*innen in der Republik Moldau zum
Forschungsdatenmanagement und andere Themen der Open Science.
Grundsätzlich geht es darum, ob die Kolleg*innen sich schon aktiv mit
dem Thema beschäftigen, ob die Bibliotheken Services aufgebaut haben,
Kooperationen innerhalb ihrer Einrichtung eingegangen sind oder ob
zumindest das Thema bekannt ist. Dies ist alles zu grossen Teilen der
Fall. Zwar gibt es in vielen moldawischen Bibliotheken auch immer den
Wunsch, mehr zu tun und die Vermutung, dass es vor allem fehlende
Ressourcen wären, welche den Aufbau weiterer Services behindern würden.
(Der Text postuliert, dass dies seinen Grund in der wirtschaftlichen
Lage des Landes hätte. Aber die Situation findet sich auch in anderen
Ländern.) Gleichzeitig ist das Thema etabliert. Die Autor*innen zeigen
in der Literaturübersicht auch, dass die gleichen Fragen in den letzten
Jahren in anderen Studien schon in anderen Ländern gestellt wurden und
dort ähnliche Ergebnisse zeigten. Die Studie und ihre Ergebnisse selber
sind also wenig innovativ.

Der Text sticht aber durch eine kurzen Absatz am Anfang des
Methodenkapitels hervor. Dort steht, dass die Umfrage eigentlich für
Moldawien, Rumänien, Georgien und die Ukraine geplant war. Das hätte
zeigen können, ob es unterschiedliche Antworten bei den
Bibliothekar*innen dieser Länder gibt. Aber: \enquote{Due to the war
initiated by Russia against Ukraine and its subsequent consequences, we
were compelled to restrict our data collection to libraries in the
Republic of Moldova.} (ebenda: 4) Sicherlich hat dieser Krieg weit
schlimmere Folgen -- ganz abgesehen davon, dass er auch in Moldawien
selber zu Problemen führt --, aber offenbar zerstört er auch
Möglichkeiten der internationalen Zusammenarbeit im Bereich
Bibliothekswesen und Bibliotheksforschung. \enquote{War -- what is it
good for? Absolutely nothing.} (The Temptations / Edwin Starr) (ks)

\begin{center}\rule{0.5\linewidth}{0.5pt}\end{center}

Lim, Elisha ; Lisker, Mareike ; Hess, Lukas ; Engeler, Malte ; Friedman,
Leah ; Lingel, Jessa ; Ali, Muna-Ubdi (2025). \emph{Abolish privacy}.
In: First Monday 30 (2025) 2,
\url{https://doi.org/10.5210/fm.v30i2.13671}

Die Idee von \enquote{Privacy} -- also das Recht, die \enquote{eigenen}
Daten und deren Nutzung zu kontrollieren -- ist fest verbunden mit einer
ungerechten Gesellschaft und einer Geschichte, die es zu überwinden
gilt. Die Diskurse um Privacy und die Versuche, diese zum Beispiel durch
Gesetze sicherzustellen, sind vielleicht gut gemeint, lösen aber das
Hauptproblem nicht: Dass es überhaupt eine Notwendigkeit gibt, diese
\enquote{Privatheit} von der restlichen Gesellschaft abzugrenzen und zu
verteidigen. \enquote{Privatheit} ist das Ergebnis der aktuellen
Gesellschaftsstruktur und eine umfassende Lösung wird nur durch die
Veränderung dieser Struktur möglich sein. Deshalb muss über diese
Veränderung nachgedacht werden.

Wenn das überraschende Aussagen sind, dann ist die Hauptfunktion des
hier besprochenen Textes erfüllt. Es ist ein explizit politisches
Manifest und als solches darauf ausgelegt, Gewissheiten zu erschüttern
sowie zum utopischen Denken anzuregen, um mögliche Zukünfte
vorzuzeichnen. Dabei, das ist wichtig, geht es den Autor*innen nicht um
eine libertäre, techno-utopische Zukunft, wie sie mit prägenden
Diskursen des \enquote{Silicon Valley} verbunden sind. Privacy soll
nicht aufgegeben werden, damit einfach alle Daten über alle Menschen von
einigen wenigen Firmen und Personen genutzt werden können. Vielmehr soll
eine Gesellschaft angestrebt werden, in welcher Privacy nicht notwendig
ist, weil es keine Notwendigkeit zur Herstellung einer \enquote{privaten
Sphäre} mehr gibt. Die Autor*innen argumentieren auf der Basis von
dekolonialer und feministischer Theorie dafür, zu erkennen, dass diese
\enquote{private Sphäre} zusammen mit den modernen Gesellschaften
entstanden ist. Eingeschrieben in diese Sphäre ist also zum Beispiel die
Geschichte des Kolonialismus, der Daten über Menschen produzierte und
nutzte, um sie zu beherrschen und auszubeuten.

Den Rezenten -- vielleicht zeigt sich hier dessen intellektuelle
Geschichte -- erinnert das Manifest in seiner Argumentation vor allem an
marxistisch oder anarchistisch geprägte Gesellschaftsanalysen aus dem
frühen 20. Jahrhundert und an ebenso marxistische geprägte radikale
Texte aus der US-amerikanischen Black-Power-Bewegung in den 1960er und
1970er Jahren, welche die Frage stellten, ob bestimmte
Auseinandersetzungen der jeweiligen Zeit tatsächlich Veränderungen
ermöglichen werden oder ob sie die jeweiligen Systeme erhalten. Also
beispielsweise, ob Genossenschaften nicht die kapitalistische
Gesellschaftsstruktur reproduzieren, wenn sie auch deren Auswirkungen
\enquote{abfedern} würden, oder ob die Inklusion von schwarzen Menschen
in die Parlamente und Polizeieinheiten nicht grundsätzlich ein
rassistisches System mit erhalten, wenn auch leicht verbessert? Oder ob
halt nicht eine andere Gesellschaft -- je nach Text eine
nicht-kapitalistische, eine machtfreie oder eine nicht-rassistische --
besser wäre? Was die Autor*innen dieses Textes zu argumentieren
scheinen, ist, dass eine Gesellschaft, in der zum Beispiel Unterschiede
zwischen Menschen mit verschiedenen sexuellen Identitäten egal sind oder
in denen der Fokus darauf liegt, die notwendige Pflege für alle Menschen
zu ermöglichen (und nicht sie zu privatisieren oder staatlich zu
regeln), auch die Notwendigkeit verschwinden lassen würden, überhaupt
eine \enquote{private Sphäre} oder \enquote{private Daten} zu
verteidigen. Eine Gesellschaft, in der Daten nicht verwendet werden
(können), um Menschen auszubeuten oder zu unterdrücken, würde die
Probleme, welche der Datenschutz regeln will, besser und endgültig
regeln. Sie schliessen, dass es sinnvoller wäre, über eine solche
Gesellschaft und dann auch Datennutzung nachzudenken, als sich in der
Frage zu verlieren, wie immer wieder neue Regeln erlassen und
durchgesetzt werden können, um private Daten zu schützen. Das erinnert
an die Fragen zu Beginn des 20. Jahrhunderts und in den 1960er/1970er
Jahren, nur hier zu einem anderen Thema.

Was das Manifest schafft und empfehlenswert macht, ist, dazu zu zwingen,
die eigenen Grundannahmen zu hinterfragen. In diesem Fall gerade im
Bibliothekswesen, in dem sich stark im Bereich Daten und Fragen des
Datenschutzes engagiert wird. Was genau ist eigentlich das Ziel dieses
Engagements? Sollte man nicht ganz andere, weitergreifende Ziele
anstreben, als \enquote{nur} die Menschen zu befähigen, \enquote{ihre
Daten} zu schützen? Es gibt keine klare Antwort auf diese Fragen --
etwas, was man auch aus der Geschichte anderer radikaler Bewegungen
lernen kann. Aber es fühlt sich wie ein notwendiger Schritt an, diese
Fragen anzugehen und nicht einfach nur auf Überzeugungen zum Datenschutz
zu beharren. Sicherlich: Egal, welche Position jemand nach dem Lesen und
Durchdenken des Textes anschliessend vertritt -- das anschliessende
Handeln ist dann wichtiger als die Meinung selbst. Auch das hat dieses
Manifest mit früheren Manifesten gemein. (ks)

\begin{center}\rule{0.5\linewidth}{0.5pt}\end{center}

Eckerdal, Johanna Rivano ; Engström, Lisa ; Färber, Alexa ; Hamm, Marion
; Kofi, Jamea ; Landau-Donnelly, Friederike ; van Melik, Rianne (2024).
\emph{Social infrastructuring in public libraries:
librarians\textquotesingle{} continuous care in everyday library
practice}. In: Journal of Documentation 80 (2024) 7: 206--225,
\url{https://doi.org/10.1108/JD-12-2023-0260}

Die Arbeit stammt aus der Ethnologie und geht deshalb auch ethnologisch
vor. Sie will klären, wie in Personen in Öffentlichen Bibliotheken
handeln, damit diese Einrichtungen zu einem \enquote{place of care} werden. Im
Fokus stehen dabei die Handlungen von Bibliothekar*innen selber, welche
die Bibliothek und ihre konkrete Arbeit so gestalten, dass sich
Nutzer*innen in diesen Bibliotheken ernst genommen fühlen. Dieser Fokus
auf das Handeln von Bibliothekar*innen ist explizit ethnologisch, da es
in dieser Forschungsrichtung ja immer darum geht, zu fragen, wie
konkretes Handeln zum Entstehen und zur Reproduktion von sozialen
Strukturen, Identitäten und Kulturen führt.

Grundsätzlich referiert die Arbeit zuerst die gängigen
bibliothekarischen Diskurse darüber, dass Öffentliche Bibliotheken
aktuell \enquote{Orte} beziehungsweise \enquote{Dritte Orte} wären. Anschliessend
beschreibt der Text, wie die Forschenden in sechs Bibliotheken (je die
Hauptfiliale und eine Zweigstelle in Wien, Rotterdam und Malmö)
Bibliothekar*innen bei ihrem Arbeitsalltag begleiteten. Bei dieser
Begleitung wurde beobachtet, was die Bibliothekar*innen tun und dieses
Handeln anschliessend von den Forschenden interpretiert. Dabei wird
sichtbar, dass Bibliothekar*innen bemüht sind, die Probleme von
Nutzer*innen möglichst umfassend zu lösen und deren Interessen zu
befriedigen. Dies geht oft damit einher, konkrete Regeln zu übertreten
oder zumindest \enquote{zu dehnen}. (Ein Beispiel im Text war, dass eine
Bibliothekar*in in Wien die Nutzer*in einer Abteilung bediente, obwohl
diese Abteilung eigentlich an diesem Tag noch geschlossen war.) Die
Autor*innen betonen zwar Unterschiede zwischen den Bibliothekssystemen
in den drei Ländern, in denen sie forschten. Beispielsweise gehen sie
auf explizit von der Politik formulierte Erwartungen an die
niederländischen Bibliotheken ein, Nutzer*innen beim Umgang mit der
öffentlichen Bürokratie zu unterstützen, also zum Beispiel beim
Ausfüllen von Formularen behilflich zu sein -- was in Österreich und
Schweden nicht zum Aufgabenbereich von Bibliotheken gehört. Aber aus
diesen thematisierten Unterschieden scheint sich für die weitere
Untersuchung wenig zu ergeben.

Auffällig sind zudem zwei Dinge: Erstens beobachteten die Forschenden
eigentlich nur Interaktionen, die von Nutzer*innen initiiert wurden, die
etwas von Bibliothekar*innen \enquote{wollen} (selbst, wenn es in einem Fall
Kinder sind, die den Bibliothekar*innen einen Dankesbrief überreichen
wollen). Andere Arbeiten oder gar Interaktionen, die von den
Bibliothekar*innen ausgelöst werden, scheinen nicht in den Fokus der
Untersuchung gelangt zu sein. Angekündigt ist zwar, dass im weiteren
Projekt auch andere Fragen angegangen werden. Aber es überrascht schon,
dass die Autor*innen bereits aus der \enquote{Lösung von Problemen von
Nutzer*innen} zu schliessen scheinen, dass Bibliothekar*innen aktiv den
\enquote{Ort Bibliothek} schaffen. Ist das wirklich die wichtigste Form sozialer
Interaktion in Öffentlichen Bibliotheken? Zweitens stellen die
Autor*innen am Anfang, wie erwähnt, den zeitgenössischen
bibliothekarischen Diskurs dar, aber sie kommen dann auf ihn in ihren
Ausführungen nicht mehr zurück. Dabei wäre es interessant gewesen zu
erfahren, ob und wenn ja, wie dieser einen Einfluss auf das konkrete
Handeln von Bibliothekar*innen hat. Jetzt zumindest vermittelt der
Artikel eher den Eindruck, dass dieser Diskurs so weit von der konkreten
Praxis entfernt ist, dass er eigentlich auch ignoriert werden kann. (ks)

\begin{center}\rule{0.5\linewidth}{0.5pt}\end{center}

{[}Editorial{]}. \emph{Transparent peer review to be extended to all of
Nature's research papers}. In: Nature 642, 542 (2025)
\url{https://doi.org/10.1038/d41586-025-01880-9}

Im Juni 2025 meldet die Nature-Redaktion, dass die Zeitschrift zu den
jeweiligen Aufsätzen auch jeweils ein Peer-Review-File publizieren wird.
Dieses dokumentiert die Kommunikation zwischen den Reviewer*innen und
den Autor*innen und erhöht, so die Hoffnung, die Transparenz und damit
auch das Verständnis für wissenschaftliche Kommunikationsprozesse. Die
Peer-Review-Report-Veröffentlichung war bisher bereits per opt-in
verfügbar. Ab sofort ist es für Veröffentlichungen in Nature der
Standard. Das Verfahren der Peer Review selbst wurde bei Nature im Jahr
1973 verpflichtend eingeführt. (bk)

\subsection{2.2 Geschichte der Bibliotheken und der
Informationswissenschaft}\label{geschichte-der-bibliotheken-und-der-informationswissenschaft}

Kayak, Noyan (2025). \emph{Bibliothèques de musées: entre métier et
approche historique.} In: RESSI - Revue électronique Suisse de science
de l'information 25 (2025),
\url{https://doi.org/10.55790/journals/ressi.2025.e1885}

In gewisser Weise ist dieser Artikel ein verzweifelter Ruf. Kayak führt
seit 2022 die Bibliothek des Musée Ariana, dem schweizerischen Museum
für Keramik und Glas in Genf. Dies beinhaltete unter anderem, einen Plan
für die Weiterentwicklung der Bibliothek aufzustellen. Für Kayak war
offenbar klar, dass dies nur auf Basis einer Geschichte dieser
Einrichtung geschehen kann: Entwicklung sei als Weiterentwicklung
anzusehen, die nicht einfach mit dem Status Quo startet, sondern früher
-- in diesem konkreten Fall 1993, mit der Gründung der Bibliothek -- und
die dann unter Einbezug der bisherigen Arbeiten in die Zukunft
verlängert werden kann. Eine solche Geschichte würde Auskunft darüber
geben, was schon etabliert wurde, was sich schon entwickelt hat oder
auch nicht entwickeln konnte. Aber: Es gibt diese Geschichte nicht. Wie
Kayak feststellt, gibt es noch nicht einmal eine Übersicht über die
Entwicklung von Museumsbibliotheken im Allgemeinen.

Er hat daraufhin in einem anderen Artikel diese Geschichte für die
betreffenden Einrichtungen in Genf aufbereitet. Im vorliegenden Text
ruft er Bibliothekar*innen vor allem dazu auf, die Geschichte der
eigenen Profession zu erarbeiten, denn offenbar würden sich
Historiker*innen wenig für dieses Thema interessieren. Dabei müsse
methodisch vorgegangen werden und nicht einfach Anekdoten versammelt
werden. Eine solche Geschichte würde es der Profession ermöglichen,
selbstbewusster und realistischer in die Zukunft zu planen und zum
Beispiel auch zu wissen, was sie schon erfolgreich gemeistert hat. Ein
Anfang wäre schon, wenn Bibliotheken auf ihren Homepages eine Übersicht
über die eigene Geschichte bieten würden. (ks)

\begin{center}\rule{0.5\linewidth}{0.5pt}\end{center}

Schwerpunkt ``\emph{Libraries at the intersection of history and the
present''} (2024). In: IFLA Journal 50 (2024) 4,
\url{https://journals.sagepub.com/toc/iflb/50/4} {[}Paywall{]}
{[}\href{https://repository.ifla.org/handle/20.500.14598/3703}{OA-Version}{]}

Die IFLA -- also die International Federation of Library Associations
and Institutions -- wird 2027 hundert Jahre alt. Eine in ihr
angesiedelte Special Interest Group zur Bibliotheksgeschichte bereitet
zu diesem Jubiläum ein Buch vor. In Vorbereitung dazu führte sie 2023
eine Tagung durch. Die dort präsentierten Paper sind in dieser
Schwerpunktausgabe des IFLA Journals veröffentlicht. Grundsätzlich
sollen sie sich mit den Quellen befassen, welche die Geschichte von
Bibliotheksverbänden -- sowohl gesamthaft der IFLA als auch einzelner
Verbände -- nachvollziehbar machen können. Dabei geht es laut dem
einleitenden Editorial auch darum, ob es aktiv gepflegte Archive der
Verbände gibt, wie ihr Erhaltungszustand ist und ob sie zugänglich sind.
Mehrfach wird im Editorial und einzelnen Artikeln darauf verwiesen, dass
gerade Bibliotheksverbände und Bibliotheken schlecht darin sind, die
eigene Geschichte zu erhalten, auszuwerten und zu reflektieren. Hervor
sticht dabei gleich der erste Artikel von Alistar Black (Black, Alistar
(2024). \emph{Memory and amnesia in the archival practices of national
library and information associations}. In: IFLA Journal 50 (2024) 4:
696--705, \url{https://doi.org/10.1177/03400352241236733} {[}Paywall{]}
{[}\href{https://repository.ifla.org/handle/20.500.14598/3703}{OA-Version}
S. 8{]}), der konkret festhält, dass diese Einrichtungen sich für
Bibliotheksgeschichte eigentlich nur dann interessieren, wenn sie für
repräsentative Zwecke genutzt werden kann. Ansonsten würde sie praktisch
nicht betrieben. Aktiv würden die Dokumente der Verbände kaum
langfristig erhalten und die Möglichkeiten von Bibliotheksgeschichte --
nämlich aus ihr zu lernen, welchen konkreten Traditionen Bibliotheken
und Bibliotheksverbände folgen -- würden nicht genutzt. Das ist eine
zutreffende Beschreibung der Situation.

Insgesamt macht die Schwerpunktausgabe aber einen unausgewogenen
Eindruck. Neben dem Artikel von Black beschäftigen sich einzelne Texte
mit konkreten Archivbeständen, die von einer sehr kleinen Anzahl von
Bibliotheksverbänden tatsächlich gepflegt werden. (Unter anderem
Bertram, Cara (2024). \emph{Preserving the history of the American
Library Association}. In: IFLA Journal 50 (2024) 4: 724--731,
\url{https://doi.org/10.1177/03400352241246445} {[}Paywall{]}
{[}\href{https://repository.ifla.org/handle/20.500.14598/3703}{OA-Version}
S. 36{]}.) Dabei wird sichtbar, dass die beste Lösung im Sinne eines
nachhaltigen Betriebs und Erhalts dieser Archivalien offenbar die
konkrete Zusammenarbeit mit einem schon existierenden Archiv, welches
die Bestände erhält, sowie einer konkreten Strategie zur regelmässigen
Abgabe von Dokumenten an dieses Archiv besteht. Diese Strategie muss von
den Verbandsvorständen aktiv umgesetzt werden.

Andere Texte erzählen kleinteilig und mit einem Fokus auf Entscheidungen
ausgesuchter Personen die Entwicklung einzelner Verbände. Sichtbar wird
dabei eine andere Gefahr für die Bibliotheksgeschichte, nämlich dass
Bestände, wenn sie einmal vorhanden sind, einfach kontextlos nacherzählt
werden -- also der Blick nur auf sie fokussiert bleibt --, während
gerade dann, wenn diese Bestände nicht existieren oder nur sehr wenig
umfangreich sind, offenbar mehr Kontext dargestellt wird.

Bei einer Anzahl von Beiträgen ist der Zusammenhang zum eigentlichen
Thema des Schwerpunkts aber gar nicht klar. Diese stellen zwar auch
Bibliotheksgeschichte dar (Interessant vor allem Fontanin, Matilde
(2024). \emph{A narrative on the codebreakers\textquotesingle{} library
at Blechtley Park}. In: IFLA Journal 50 (2024) 4: 787--797,
\url{https://doi.org/10.1177/03400352241265376} {[}Paywall{]}
{[}\href{https://repository.ifla.org/handle/20.500.14598/3703}{OA-Version}
S. 99{]}.), aber beziehen sich gerade nicht auf Bibliotheksverbände.
Insgesamt lässt sich also aus diesem Schwerpunkt noch nicht
erschliessen, was genau der Fokus der für 2027 versprochenen Publikation
sein wird, ob sie also eine Ansammlung von Geschichten, eine
Nacherzählung von Entwicklungen oder aber eine -- notwendige --
kritische Reflektion der Geschichte der IFLA sein wird. (ks)

\begin{center}\rule{0.5\linewidth}{0.5pt}\end{center}

Stock, Wolfgang G. (2024). \emph{Information science in the German
Democratic Republic}. In: Journal of Documentation 80 (2024) 7:
287--305, \url{https://doi.org/10.1108/JD-03-2024-0058}

Der Autor stellt zutreffend fest, dass die bisherige
Geschichtsschreibung zur Informationswissenschaft im DACH-Raum die DDR
praktisch vollständig übergangen hat, obwohl gerade in der DDR eine
eigene Informationslandschaft und -wissenschaft bestand, die auch -- wie
implizit aus dem Text klar wird -- auf viele Ressourcen zurückgreifen
konnte. Der Text unternimmt nun den Versuch, diese Geschichte
darzustellen. Dabei muss er sich hauptsächlich auf publizierte
Zeitschriftenartikel stützen, da andere Materialien bislang kaum
aufgearbeitet wurden.

Dargestellt wird, dass die Entwicklung der Informationswissenschaft in
der DDR explizit verbunden war mit der DDR-Politik und der
Weltanschauung, auf deren Basis diese Politik betrieben wurde. Der Autor
zeigt dies zum Beispiel daran, wie das Konzept \enquote{Information} in das
marxistische Denken integriert wurde, obgleich es zum Beispiel bei den
Begründern der Kybernetik so konzipiert war -- als \enquote{nicht Materie, nicht
Ideologie} --, dass es eigentlich nicht in dieses Denksystem eingefügt
werden konnte. Gleichzeitig machte es die Struktur des
Wissenschaftssystems in der DDR möglich, ein System von
Forschungsinformation zu etablieren, bei dem die Forschenden zu aktiver
Zuarbeit verpflichtet wurden. Zu diesem System gehörten nicht nur
\enquote{Pflichtenhefte}, welche von den Forschenden zu Beginn eines Projektes
auszufüllen waren, sondern auch zahlreiche Datenbanken und
Informationsstrukturen, eine eigene Ausbildung von
Informationsspezialist*innen, und auf dem Gebiet der
Informationswissenschaft eigene Forschungen, Tagungen und
Theorieentwicklungen sowie ein eigenes Zentralinstitut.
Informationswesen und -wissenschaft waren in der DDR ein konkretes
Arbeitsfeld für eine wohl beachtliche Anzahl von Personen.

Der Autor stellt in seinem Text neben dem System selber vor allem vier
theoretische Zugänge zur Wissenschaftlichen Information und deren
Nutzung vor, welche in der DDR erarbeitet wurden. Wenig geht er auf die
konkrete Arbeit in Informationseinrichtungen ein. Vielmehr widmet er ein
recht langes Kapitel der Abwicklung dieser Strukturen nach 1989 sowie
einer Einordnung in die heutige Forschung. Dabei postuliert er, dass die
Tradition der DDR-Forschung abgebrochen und praktisch vergessen wurde.
(Der Text vermittelt aber den Eindruck, als wäre er im Original in einer
viel längeren, ausführlicheren Variante geschrieben, aber dann für die
Publikation gerade um diese Kapitel, die sich mit der konkreten Praxis
beschäftigten, gekürzt worden.) Während diese Klagen inhaltlich
berechtigt sind, tragen sie wenig zu einer weiteren Forschung bei. Dabei
drängen sich andere Fragen auf. Beispielsweise zeigt der Autor, dass es
in der DDR ein System von Forschungsinformation gab, dass in Vielem das
Gleiche zu erreichen versuchte, wie heute mit Current Research
Information Systems oder den verschiedenen Persistent Identifiers im
Wissenschaftsbereich (ORCID, DOI und so weiter) angestrebt wird. Was
liesse sich aus den Erfahrungen der DDR mit diesem System lernen?

Gleichzeitig zeigt der Artikel, wie eng Informationspraxis und
ideologische Basis miteinander verbunden waren, was für heute die Frage
aufwirft, ob das Denken über Information nicht auch eine ideologische
Basis hat -- nur eine, die nicht so explizit wie in der DDR thematisiert
wird. (ks)

\subsection{2.3 Open Access}\label{open-access}

o.A. (2025): \emph{Universiteter betaler dyrt for Open Access}. In:
Forskerforum. Dezember 2024, S. 13,
\url{https://dm.dk/media/vebn2rmg/forskerforum-nr-6-2024.pdf}

In einer kurzen Einschätzung beschreibt die dänische
Hochschulzeitschrift \emph{Forskerforum} die Herausforderungen der
Kosten für Open Access für die Universitäten in Dänemark. Auch dort
werden Publikationsgebühren zunehmend als problematisch angesehen,
besonders da die Lizenzierungskosten für wissenschaftliche Zeitschriften
ebenfalls steigen. Angegeben wird eine Kostensteigerung von 20~\% über
vier Jahre. Dies widerspricht aus Sicht des Sprechers der dänischen
Universitäten, Jesper Langergaard, der Idee des Open Access. Als
schwierig zeigt sich, dass die Universitäten einzeln mit den Verlagen
verhandeln und keine gemeinsame und damit wirksamere Position einnehmen.
Das Interesse der Forschenden ist dagegen vor allem, in bestimmten
Journalen zu erscheinen, was die Nutzung der vorhandenen Alternativen
wie Green Open Access bremst. Aus Sicht der dänischen Universitäten wäre
eine supranationale Lösung, also durch die Europäische Gemeinschaft, für
die Finanzierung des Open-Access-Publizierens und des Durchbrechens von
Monopolstrukturen wünschenswert. (bk)

\begin{center}\rule{0.5\linewidth}{0.5pt}\end{center}

Céspedes, Lucía ; Kozlowski, Diego ; Pradier, Carolina et al.~(2025).
\emph{Evaluating the linguistic coverage of OpenAlex: An assessment of
metadata accuracy and completeness.} In: Journal of the Association for
Information Science and Technology.
\url{https://doi.org/10.1002/asi.24979}

Die Studie analysierte die sprachliche Abdeckung und die Qualität der
Metadaten von OpenAlex im Vergleich zu Web of Science (WoS). Es lässt
sich festhalten, dass OpenAlex gegenüber WoS eine ausgewogenere
sprachliche Abdeckung besitzt. Damit bietet es sich prinzipiell für
linguistische Auswertungen und Untersuchungen an, wobei die Qualität der
Sprachmetadaten zum Zeitpunkt der Studie noch Qualitätsmängel aufwiesen.
Entsprechend scheint bei OpenAlex Englisch derzeit überrepräsentiert.
Auch mögliche Nachweislücken bei Publikationen, die keinen PID haben,
werden benannt. Die Autor*innen empfehlen daher den Ausbau der Forschung
zu disziplinären Klassifikationen und Publikationstypen in OpenAlex und
gehen davon aus, dass die Datenqualität unter anderem durch das Feedback
der Nutzenden steigen wird. Als Alternative zu proprietären Datenbanken
und zur wissenschaftstheoretischen und linguistischen Forschung besitzt
OpenAlex ein großes Potential. (bk)

\subsection{2.4 Bestandsmanagement und
Bibliotheksbestände}\label{bestandsmanagement-und-bibliotheksbestuxe4nde}

England, Erica (2025). \emph{Will We Ever Learn?: The Un-Enlightenment
of Selector Librarians}. In: Journal of Radical Librarianship 11 (2025):
1--32,
\url{https://journal.radicallibrarianship.org/index.php/journal/article/view/114}

Der Studie geht es darum, ob und wie Bibliothekar*innen in
US-amerikanischen Bibliotheken das Ziel umsetzen, den jeweiligen Bestand
diverser zu gestalten, also Medien aufzunehmen, die eine breitere,
gesellschaftlich repräsentativere Auswahl darstellen. Der Text hält sich
lange mit der Herleitung dieses Themas auf, aber grundsätzlich basiert
er auf der Auswertung von semi-strukturierten Interviews mit 19
Bibliothekar*innen, welche Bestandsmanagement betreiben. Diese Auswahl
war durch eine vorhergehende Umfrage schon einigermassen gesteuert: Es
waren alles Personen, die ein Interesse daran hatten, den jeweiligen
Bestand divers zu gestalten.

Das wichtige Ergebnis ist, dass allen die Notwendigkeit dafür bewusst
ist, um einen gesellschaftlich sinnvollen, fairen Bestand zu entwickeln,
aber dass es strukturelle Hürden gibt. Weder in der Ausbildung noch in
der Weiterbildung würde vermittelt werden, wie dieses Ziel praktisch
umsetzbar ist. Wenn überhaupt, dann sei vermittelt worden, dass es
wichtig wäre. Eigentliche Werkzeuge oder Vorgehensweisen würden dagegen
nicht thematisiert. Zum anderen müsse man sich beim Bestandsmanagement
immer mehr auf den Einsatz von Tools von externen Anbietern verlassen.
Die Medienauswahl erfolgte immer weniger direkt durch die
Bibliothekar*innen. Stattdessen verliessen sie sich auf die Auswahl von
Verlagen, dem Buchhandel oder mehr noch spezifischen Anbietern.

Der Text ruft zwar dazu auf, die Aus- und Weiterbildung praktischer zu
orientieren, also direkt zu vermitteln, wie Bibliothekar*innen
diversitätsorientiert Bestände auswählen oder Medien so einschätzen und
bewerten können, dass sie am Ende zu diversen Beständen führen. Was
nicht diskutiert wird, ist, ob nicht auch die anderen Strukturen
verändert werden könnten, beispielsweise durch (wieder mehr)
bibliothekarische Besprechungen oder aber dadurch, dass Anbieter
verpflichtet werden, ihre Auswahlkriterien transparent darzulegen und
gegebenenfalls anzupassen. (ks)

\begin{center}\rule{0.5\linewidth}{0.5pt}\end{center}

Orner, Sylvia (2024). \emph{Analyzing and Assessing a Library Collection
Using Faculty Via OpenAlex and R}. In: Evidence Based Library and
Information Science 19 (2024) 4: 39--52,
\url{https://doi.org/10.18438/eblip30493}

Der Artikel stellt dar, wie eine Universitätsbibliothek -- die
University of Scranton, Pennsylvania -- versuchte, die Nutzung des
Bibliotheksbestandes durch die Forschenden der Universität mittels einer
Auswertung von Zitations- und anderen Daten zu erheben. Das Ganze war
Teil einer Evaluation des gesamten Bibliotheksbestandes. Es wurde also
unter anderem gefragt, ob die Forschenden Literatur zitieren, die von
der Bibliothek lizenziert oder gekauft worden war.

Solche Versuche, die Publikationen von Forschenden zu nutzen, um die
Bestände von Bibliotheken zu evaluieren, sind nicht grundsätzlich neu.
Interessant an diesem Beispiel ist aber -- neben dem, dass dies hier
einmal publiziert wurde und nicht intern in einer Bibliothek verblieb
--, dass dafür OpenAlex und R genutzt wurden. OpenAlex ist als Anbieter
von Forschungsinformation relativ neu, scheint sich aber sehr schnell
als offene Alternative zu den bekannten kommerziellen Anbietern zu
etablieren. R ist ebenso eine offene Variante des Angebots eines
kommerziellen Anbieters, aber schon länger etabliert. Was der Artikel
zeigt, ist, dass es für Bibliotheken jetzt tatsächlich relativ einfach
möglich ist, ohne grossen finanziellen oder personellen Aufwand solche
Analysen durchzuführen. (Der Artikel stellt die Schritte der Analyse
dar, die verwendeten R-Skripte sind ebenso publiziert.) Nicht diskutiert
wird allerdings, ob diese ausgewerteten Daten tatsächlich etwas über den
Bibliotheksbestand aussagen. (ks)

\begin{center}\rule{0.5\linewidth}{0.5pt}\end{center}

René, Will (2024). \emph{Displaying sound: the National Poetry
Library\textquotesingle s vinyl collection, 2019--2024}. In: Art
Libraries Journal 49 (2024) 4: 126--135,
\url{https://doi.org/10.1017/alj.2024.21} {[}Paywall{]}

In diesem Artikel geht es um einen sehr speziellen Bestand in einer sehr
spezifischen Bibliothek: Die National Poetry Library in London sammelt,
erschliesst und präsentiert die Lyrik Grossbritanniens. Das beinhaltet
auch das Veranstalten von Lesungen oder Workshops sowie das Ausrichten
von Ausstellungen. Im Bestand der Bibliothek finden sich -- allerdings
zu Beginn der im Artikel diskutierten Periode noch sehr verstreut --
aufgezeichnete Lesungen von Lyrik, zumeist von den Dichter*innen selber,
auf Vinyl.

Der Text beschreibt nun, wie dieser Bestand in den letzten Jahren
zusammengezogen, bibliothekarisch bearbeitet und neu präsentiert wurde.
Es ist ein sehr praxisorientierter Artikel, der aber einmal direkt auf
die konkrete Arbeit von Bibliothekar*innen mit den Medien eingeht.
Angesichts dessen, dass in der bibliothekarischen Fachliteratur sonst
gerne über abstraktere Themen berichtet wird, ist er erfrischend nerdig.
(ks)

\begin{center}\rule{0.5\linewidth}{0.5pt}\end{center}

Bobrow, Evan (2025). \emph{The Role of Academic Libraries in the
Shifting Landscape of Zines.} In: College \& Research Libraries, 86
(2025) 2: 204--208, \url{https://doi.org/10.5860/crl.86.2.204}

Zines -- also selber hergestellte, gedruckte Hefte, die vor allem seit
dem Aufkommen von Kopierern eine Rolle in fast allen Subkulturen im
Globalen Norden spielten -- sind Teil von Beständen US-amerikanischer
Bibliotheken geworden. Es gab in den letzten Jahren verstärkt
Publikationen zu diesem Thema und auch Projekte, diese Medien nicht nur
in die Bestände aufzunehmen, sondern auch aktiv zu vermitteln. Das
Editorial von Bobrow reflektiert kurz den Stand und die möglichen
Entwicklungen.

Dabei wird betont, dass Zines weiterhin das Potential haben, eine
Gegenöffentlichkeit herzustellen und auch Personen zu motivieren, selber
Zines zu erstellen. Gerade in Zeiten, in denen auf der einen Seite alles
online gestellt werden könnte, aber auf der anderen Seite immer mehr der
Eindruck entsteht, dass online alles immer weniger sinnvoll und
selbstbestimmt ist, stellten Zines ein Gegenmedium dar. Wer ein Zine
erstellt, entscheidet sich explizit für das materielle Objekt.
Gleichzeitig gäbe es aktuell eine Entwicklung dahin, Zines als grosse
Projekte zu planen und zu realisieren, inklusive künstlerischer
Ausgestaltung und gutem Druck. Dies vermittle aber vielen Menschen den
Eindruck, dass es für die Herstellung eines Zines bestimmter Fähigkeiten
bedürfe. Bobrow führt dies darauf zurück, wie vor allem junge Leute
heute Medien kennenlernen würden -- nämlich zuerst digital, also auch in
\enquote{professionellem} Layout. Hingegen sei es wichtig, dass sie auch
einfach hergestellte Zines als Medium kennenlernen würden, also lieber
als selber kopierte und geklammerte Hefte, anstatt als fertiges,
\enquote{professionelles} Druckerzeugnis. Dies vermittelte, dass
wirklich alle etwas sagen könnten und auch die Mittel dazu hätten.
Bestände solcher Zines in Bibliotheken, so die implizite Aussage, würden
dazu beitragen. (ks)

\section{3. Monographien und
Buchkapitel}\label{monographien-und-buchkapitel}

\subsection{3.1 Vermischte Themen}\label{vermischte-themen-1}

Keyes, Kelsey ; Dworak, Ellie (2024). \emph{Supporting Student Parents
in the Academic Library: Designing Spaces, Policies, and Services}.
Chicago: Association of College and Research Libraries, 2024
{[}gedruckt{]}

Wie im Titel angegeben, beschäftigt sich das Buch aus Sicht von
Hochschulbibliotheken mit Eltern, die studieren und damit, wie diese
durch die Bibliotheken beim Studium unterstützt werden können. Die
Autor*innen (beide arbeiteten in der Bibliothek der Boise State
University in Idaho) zeigen schon im Vorwort anhand von selber erlebten
Situationen, dass dies grundsätzlich sinnvoll und notwendig ist. Sie
berichten davon, dass die meisten Hochschulbibliotheken nicht einmal
klare Regeln dazu hätten, ob Eltern ihre Kinder mit in die Bibliotheken
bringen können und wenn ja, mit welchen Auflagen. Es sei oft nicht
einmal klar, ob Kinder \enquote{ruhig} sein müssten und was
\enquote{Ruhe} genau heisst. Dabei gibt es eine grosse Anzahl von
Studierenden, die Kinder in allen Altersstufen betreuen und gleichzeitig
Studium, Elternschaft und Arbeit managen.

Die Universitäten in den USA und auch deren Bibliotheken seien für
Studierende eingerichtet, die es immer weniger gäbe: Nämlich solche,
welche direkt nach der Schule auf den Campus ziehen, sich dann dort
vollkommen auf das Studium konzentrieren könnten und keine
Verpflichtungen für Kinder (oder andere Angehörige) hätten. Die
Autor*innen argumentieren, dass es richtig sei, diesen Blick zu weiten.
Obwohl die Daten nicht eindeutig wären, ist doch davon auszugehen, dass
an jeder Hochschule eine grosse Minderheit der Studierenden selber
Eltern sind. Die Autor*innen betonen zudem, dass davon auch die
Universitäten selber profitieren werden, weil sie so für mehr
Studierende offen werden und gleichzeitig dazu beitragen können, dass
mehr Studierende das Studium auch abschliessen und nicht vorzeitig
abbrechen müssen.

Am Ende laufen ihre Empfehlungen darauf hinaus, klare Regeln zu
etablieren, eigene Bereiche für Studierende und ihre Kinder einzurichten
sowie im Idealfall Services wie Lesestunden oder Hausaufgabenhilfe
anzubieten und dies auch in Zusammenarbeit mit anderen Einrichtungen,
wobei unter den besprochenen erstaunlicherweise die Public Libraries
fehlen.

Das Buch ist aber mehr als diese Argumentation. Es will daneben Daten
zusammentragen, um diese Argumentation zu unterstützen, gleichzeitig
zeigen, wie Bibliotheken studierende Eltern unterstützen können und
zudem eine ganze Anzahl von Beispielen aus Bibliotheken dokumentieren.
Diese verschiedenen Ziele erweisen sich als Schwäche der Publikation.
Sie lässt sich als Potpourri beschreiben: Lange Abschnitte zu einem
spezifischen Thema stehen neben Auswertungen einzelner Datensätze (zum
Beispiel zu den Policies von Universitätsbibliotheken). Daneben stehen
Beschreibungen von Studien. Gleichzeitig finden sich detaillierte
Beschreibungen dazu, welche Daten es zu studierenden Eltern gibt oder
wie eine Policy geschrieben werden könnte, die studierende Eltern in
einer Bibliothek willkommen heisst. Mitten im Text stehen dann noch Case
Studies -- sogar im Kapitel zur Boise State University Library, das
selber eine Case Study ist --, die nicht immer eine Verbindung zum
Kapitel zu haben scheinen, in denen sie stehen. Das alles vermittelt
einen ungeordneten, unklaren Eindruck.

Zudem ist dies alles, wie zu erwarten, immer direkt auf die USA bezogen.
Für den DACH-Raum, wo ja zum Beispiel das \enquote{Leben auf dem Campus}
praktisch nicht zum Studienalltag gehört, kann es nicht direkt genutzt
werden -- auch wenn das Thema \enquote{studierende Eltern} für
Hochschulbibliotheken hierzulande ebenfalls ein relevantes Thema sein
sollte. (ks)

\begin{center}\rule{0.5\linewidth}{0.5pt}\end{center}

Jürgens, Moritz ; Sander, Julia ; Werner, Sybille (Hrsg.) (2022).
\emph{Leseförderung in der Ganztagsschule}. (Lesesozialisation und
Medien) Weinheim, Basel: Beltz Juventa, 2022 {[}gedruckt{]}

In der bibliothekarischen Fachpresse wird erstaunlich selten auf die
Forschung zur Leseförderung zurückgegriffen, welche im Rahmen der
Erziehungswissenschaft produziert wird. Angesichts dessen, dass
Öffentliche Bibliotheken sich gerne selber als Ort der Leseförderung
begreifen, wäre das eigentlich zu erwarten.

Dieser aktuelle Band mit Forschungsergebnissen und Praxisbeispielen
deutet allerdings an, dass es auch Schulen selber schwerfällt,
konsequent und nachvollziehbar auf die vorliegende Forschung zu
reagieren. Interessant ist allerdings, dass Bibliotheken in diesem Band
tatsächlich auch direkt thematisiert werden. In dem Buch werden in
einigen Beiträgen theoretische Konzepte der Leseförderung erklärt, dann
Ergebnisse aus Schulleistungsvergleichsstudien (PISA, Nationaler
Bildungsbericht) und vergleichbarer weitflächiger Untersuchungen der
letzten Jahre referiert. Dann wird eine Studie, die genau eine Schule
betrachtet, dargestellt und anschliessend eine Anzahl von
Praxisbeispielen zur Leseförderung in Schulen. Die letzten beiden
Beiträge beschäftigen sich mit Schulbibliotheken. Der Fokus des gesamten
Bandes liegt, ausser bei einem Text über die Situation in Finnland,
immer auf Deutschland.

Im Gesamtblick fallen zwei Dinge auf: Es gibt keine richtige Verbindung
zwischen den Ergebnissen aus den \enquote{grossen} Studien und der
Praxis in den Schulen. Und: Die Fördermassnahmen der vergangenen fast
zwanzig Jahre hatten wenig nachweisbaren Erfolg. Diese Massnahmen der
jüngeren Vergangenheit zielten darauf, Lesekompetenz als Set von
Kompetenzen und Fähigkeiten zu definieren und diese dann gezielt zu
fördern. Es ging also nicht mehr darum, Lesen an sich zu promoten,
sondern beispielsweise das Sinn-entnehmende Lesen zu üben. In
kontrollierten Settings konnten damit messbare Erfolge nachgewiesen
werden, aber offenbar nicht mehr, wenn dies breitflächig im gesamten
Schulsystem angewendet wurde. Gleichwohl zeigen die Daten, dass die
Probleme, die schon bei der ersten PISA-Studie 2003 sichtbar waren,
weiter bestehen: Im Vergleich zu anderen Ländern ist die Lesekompetenz
von Schüler*innen in Deutschland schwach, wobei vor allem auffällt, dass
es zwar viele leistungsstarke, aber auch viele leistungsschwache
Schüler*innen gibt. Diese Spreizung ist weit grösser als in
vergleichbaren Ländern. Zudem sind die Ergebnisse bei der Lesekompetenz
weit mehr als in anderen Ländern an den sozialen Status des Elternhauses
gebunden. Oder anders gesagt: Im deutschen Bildungssystem werden soziale
Unterschiede stärker reproduziert als in anderen Ländern und dies hat
sich, trotz einiger versprechender Ansätze, nicht gross geändert.

Die Beiträge zu den übergreifenden Daten sind interessant, wenn auch
desillusionierend zu lesen. Aber die Praxisbeispiele beziehen sich dann
gar nicht darauf, die Probleme -- also die soziale Ungerechtigkeit --
anzugehen, sondern setzen andere Schwerpunkte bei der Leseförderung.
Während in den einführenden Beiträgen des Buches argumentiert wird, dass
es eine Aufgabe der Schulen sein muss, sozial gerechter zu werden, indem
der soziale Hintergrund des Lesenlernens betrachtet wird, wird in den
meisten Praxisbeiträgen behauptet, dass es darum ginge, möglichst viele
Kinder zu erreichen und beispielsweise mittels Partizipation zum Lesen
zu motivieren. Im Ganzen vermitteln diese Beiträge den Eindruck, dass in
der Leseförderung in Schulen in Deutschland zwar engagiert gehandelt
wird, aber unterhalb dieser ganzen Aktivitäten gerade die tatsächlichen
Probleme ignoriert und damit reproduziert werden. In diesem scheinen
Schulen den Bibliotheken zu gleichen, deren Leseförderungsaktivitäten
auch oft die vorliegenden Daten nicht wahrzunehmen scheinen und
stattdessen versuchen, andere Probleme anzugehen.

Um auf Bibliotheksthemen zu sprechen zu kommen: In einer ganzen Reihe
der datenbasierten Beiträge wird darauf insistiert, dass es wichtig
wäre, Kindern in der Schule direkt Lesezeiten zur Verfügung zu stellen
und zwar möglichst selbstbestimmt sowie in gesonderten Leseräumen, aber
auch in Leseecken in den Schulzimmern. Man verspricht sich davon, dass
das Lesen so durch kontinuierliche Praxis besser gelernt werden kann.
Genannt werden dabei mehrfach Schulbibliotheken. Allerdings sind auch
hier zwei Dinge sichtbar: Zum einen beziehen sich diese Nennungen nie
auf die bibliothekarische Literatur oder bibliothekarische Vorstellungen
von Schulbibliotheken. Man stellt sich letztere meist nur als Leseräume
mit Büchern vor, die einfach einzurichten wären. Alle weiteren
Funktionen von Schulbibliotheken werden offenbar nicht angedacht.
Ergänzt wird dieser reine \enquote{Schulblick} noch dadurch, dass es ein
Blick nur auf Deutschland ist. Mit der Schweiz gibt es nämlich ein Land,
dessen Schulsystem dem deutschen sehr ähnlich ist, aber in dem freie
Lesestunden für Schüler*innen in recht gut ausgestatteten
Schulbibliotheken Teil des Unterrichtsalltags in Primarschulen sind.
Wenn dies eine Lösung für die Probleme in Deutschland wäre, müsste sich
das zum Beispiel in den PISA-Studien zeigen. Aber das tut es nicht
wirklich. Ein Grossteil der besseren Ergebnisse in der Schweiz scheint
sich aus dem allgemein höheren sozio-ökonomischen Stand der Haushalte zu
erklären. Ansonsten sind die Leistungen und die Spreizung zwischen
\enquote{guten} und \enquote{schlechten} Leser*innen fast gleich gross.
Insoweit ist nicht klar, warum dieser Vorschlag eine bessere Lösung als
die jetzige Praxis darstellen soll.

Die beiden Beiträge zu Schulbibliotheken am Ende des Buches helfen
wenig, diesen grundsätzlich verwirrenden Gesamteindruck des Bandes zu
mildern. Im ersten wird dargestellt, wie in Bibliotheken Bestände
systematisiert werden, um dann eine mit Kindern erarbeitete Systematik
in einer konkreten Schulbibliothek darzustellen. Im zweiten stellen
Kolleg*innen der schulbibliothekarischen Arbeitsstelle in Frankfurt am
Main überblicksartig dar, was Schulbibliotheken angeblich alles
ermöglichen würden und dann, wie sie in Frankfurt unterstützt werden.
Dieser Text liest sich wie ein Werbetext, der am Ende zudem die ganzen
Daten aus den ersten Beiträgen des Bandes ignoriert. Auch hier wird ohne
jede Herleitung behauptet, dass eine Schulbibliothek das Lesen aller
Schüler*innen fördert. Alles in allem ist dieser Band also eher
verwirrend und zeigt, wie inkonsequent die Leseförderung in Deutschland
bezogen auf das Ziel von Bildungsgerechtigkeit eigentlich ist. (ks)

\begin{center}\rule{0.5\linewidth}{0.5pt}\end{center}

Lo, Patrick ; Sutherland, Robert ; Hsu, Wei-En ; Girsberger, Russ
(edit.) (2022). \emph{Stories and Lessons from the World's Leading
Opera, Orchestra Librarians, and Music Archivists, Volume 1: North and
South America}. Bingley: Emerald Publishing Limited, 2022 {[}gedruckt{]}

Lo, Patrick ; Sutherland, Robert ; Hsu, Wei-En ; Girsberger, Russ
(edit.) (2022). \emph{Stories and Lessons from the World's Leading
Opera, Orchestra Librarians, and Music Archivists, Volume 2: Europa and
Asia}. Bingley: Emerald Publishing Limited, 2022 {[}gedruckt{]}

Die beiden Bände versammeln Interviews mit \enquote{Opera Librarians}
und Personen mit ähnlichen Positionen. Dies stellt ein eigenes
Aufgabenfeld dar und -- so argumentieren die Herausgebenden in den
kurzen Vorworten der beiden Bände -- praktisch einen eigenen
Bibliotheks- und Archivtyp. Opera Librarians haben in den Orchestern,
für die sie angestellt sind, verschiedene, unterstützende Aufgaben: Sie
verwalten die Notenblätter, welche im Besitz der Orchester sind, üben
praktisch Lizenzmanagement aus (obgleich sie die Lizenzen mit den
Musikverlagen oft nicht selber schliessen, sondern jemand anders im
Orchester dafür zuständig ist), bereiten Noten für die Musiker*innen vor
und bearbeiten sie zum Teil in Zusammenarbeit mit den Dirigent*innen.
Aber sie sind auch zum Teil verantwortlich dafür, Instrumente anzuordnen
oder Notenhalter aufzustellen sowie die Aktivitäten der Orchester zu
dokumentieren. Bei den Orchestern, die in diesen beiden Bänden
thematisiert werden, handelt es sich immer um grosse, professionelle und
wohl durchgehend staatlich finanzierte: Die Berliner Philharmonie, die
Wiener Staatsoper oder die Canadian Opera Company sind einige der hier
vertretenen.

Auch die Herausgeber*innen, welche die Interviews durchführten, arbeiten
in solchen Positionen. Es scheint sich -- wie die Opernwelt an sich --
um eine eigene, international stark vernetzte Szene zu handeln. (Wobei
die Internationalität auch eingeschränkt ist, offenbar hier auf
Personen, die Englisch sprechen und unter Auslassung von Afrika und
Ozeanien.) Das Erstaunlichste ist wohl, dass es tatsächlich so viele
Opera Librarians gibt, dass sich so viele rund je zehnseitige Interviews
führen lassen, um zwei Bände zu füllen. Man hat beim Lesen nicht den
Eindruck, dass hier lauter Ausnahmen gemacht werden mussten, um mehr
Personen zu finden, die inkludiert werden konnten. Bis auf eine Person,
die in der Musikabteilung der Library of Congress arbeitet, sind alle
wirklich fest an Orchestern angestellt. Allerdings zeigt sich in den
Interviews auch die grosse Vertrautheit miteinander. Obgleich alle
Personen am Anfang vorgestellt werden, hat es immer den Eindruck, als
wenn Personen miteinander reden würden, die sich alle kennen. Ständig
werden persönliche Anekdoten erzählt oder es gibt Verweise auf andere
Personen, deren Interviews sich oft auch in den Bänden finden.

Die Interviews geben Einblick in den Arbeitsalltag der Opera Librarians.
Dieser ist abwechslungsreich, direkt an die Arbeit der Orchester
gebunden und beispielsweise von Probe- und Aufführungsaisons geprägt. Er
lässt sich nicht auf das Verwalten von Noten reduzieren, sondern ist
tatsächlich Teil der \enquote{Infrastrukturarbeit} für die Orchester.
Dies geschieht alles in recht offenen, wertschätzenden Gesprächen,
welche oft durch Bildmaterial unterstützt werden. Aber was kaum sichtbar
wird, ist die konkrete bibliothekarische Arbeit. Man erfährt weit mehr
über die Orchester selber, über herausragende Aufführungen oder die
Zusammenarbeit von Opera Librarians und anderen Personen, als darüber,
wie der Bestand gemanagt wird oder wie genau das Lizenzmanagement
geschieht. Es sind unterhaltsame Einblicke, aber sie sind wohl am Besten
als eine Selbstvergewisserung der \enquote{Szene} der Opera Librarians
zu verstehen, die sich hier gegenseitig präsentieren und damit
bestätigen, dass es sie als eigene Profession gibt und sie nicht jeweils
allein stehen. (ks)

\begin{center}\rule{0.5\linewidth}{0.5pt}\end{center}

\pagebreak

Wiley, Claire Walker ; Click, Amanda B. ; Houlihan, Meggan (edit.)
(2023). \emph{Everyday Evidence-Based Practice in Academic Libraries:
Case Studies and Reflections}. Chicago: Association of Research and
College Libraries, 2023 {[}gedruckt{]}

Evidence Based Library and Information Practice ist ein Sammelbegriff
für den Einsatz wissenschaftlicher Methoden und Denkweisen beim Treffen
von Entscheidungen in Bibliotheken, also beispielsweise der Planung der
Bestandsentwicklung oder der Evaluierung von Bibliotheksservices. Anfang
des 21. Jahrhunderts wurde es insbesondere in Kanada als eigene Richtung
-- inklusive eigener Zeitschrift, Einbindung in die bibliothekarische
Ausbildung und Praxis -- entwickelt. Damals wurde sich stark an die
Evidence Based Medicine angelehnt, welche den Einsatz jeweils der besten
wissenschaftlichen Fakten im Medizinbereich, also zum Beispiel bei der
Wahl von Therapien, anstrebt.

Seitdem hat sich diese \enquote{Bewegung} ausgebreitet, eigene Modelle
etabliert und dabei auch Veränderungen vollzogen. So ist es heute auch
in anderen englisch-sprachigen Bibliothekswesen im Globalen Norden (vor
allem Grossbritannien und Australien) verbreitet. Es wurde sich auch vom
medizinischen Vorbild abgewandt und eine der Praxisorientierung von
Bibliotheken angepasste Variante entwickelt. Ein wichtiges Modell ist
dabei ein Modell, das mit der Abkürzung \enquote{5As process} benannt
wurde: Articulate (Formulieren einer praxisorientierten Forschungsfrage,
zum Beispiel: \enquote{Was denken unsere Nutzer*innen über die neue
Möblierung der Bibliothek?}), Assemble (Zusammentragen von Evidenzen zum
Thema, beispielsweise der wissenschaftlichen Forschung oder anderen
Bibliotheken, und durch eigene Datenerhebungen), Assess (Übertragung der
Evidenz auf die lokale Situation und gemeinsame Interpretation,
Beantwortung der Frage), Agree (Gemeinsames Treffen von Entscheidungen
über das weitere Vorgehen in der Bibliothek, auf Basis der Ergebnisse),
Adapt (Umsetzen der Entscheidung und Überprüfung der Ergebnisse). Diese
Schritte sind als Kreislauf vorgesehen. Es soll am Ende also immer
wieder mit neuen Fragen angeschlossen werden. Ziel ist es, die
Entscheidungen in Bibliotheken grundsätzlich auf Evidenzen aufzubauen
und nicht auf \enquote{Bauchgefühl} oder rein strukturellen Zwängen. Das
Modell wurde 2016 in einem auf Forschungen zum tatsächlichen Einsatz von
\enquote{Evidence Based Practices} in Bibliotheken basierenden Buch
vorgeschlagen. (Brettle, Allison ; Koufogiannakis, Denise (edit.)
(2016). \emph{Being Evidence Based in Library and Information Practice}.
London: Facet Publishing, 2016 {[}gedruckt{]}) Die beiden damaligen
Herausgeber*innen führen das Modell in diesem Buch im ersten Beitrag
noch einmal aus und schreiben darüber, wie es seitdem von anderen
weiterentwickelt wurde.

Der Rest des Buches besteht aber aus Beiträgen, in denen der
\enquote{originale} 5As process von Hochschulbibliotheken in den USA
angewendet wurde. Es liest sich ein wenig wie eine Ergänzung zum Buch
von 2016. Jeder dieser Beiträge beginnt mit einer Darstellung der
jeweiligen Bibliothek und folgt dann dem Prozess selber. In jedem
Beitrag wird eine praxisorientierte Forschung vorgestellt. Zudem ist
praktisch jedem Beitrag ein Anhang mit den jeweiligen
Forschungsinstrumenten beigefügt, die erstellt wurden (also zum Beispiel
die Fragebögen oder Interviewleitfäden). Was damit gezeigt wird, ist,
dass der Prozess genutzt werden kann, um Entscheidungen zu treffen.
Gleichzeitig wird vermittelt, dass praxisorientierte Forschung in
Bibliotheken möglich und sinnvoll ist. Ansonsten sind die Beispiele alle
immer wieder spezifisch für die jeweilige Bibliothek. Wenn das Buch
einen Einfluss haben kann, dann wohl vor allem, andere Bibliotheken dazu
anzuspornen, den Prozess bei eigenen Fragen einzusetzen. (ks)

\subsection{3.2 Schrift- und
Bibliotheksgeschichte}\label{schrift--und-bibliotheksgeschichte}

Ferrara, Silvia (2021). \emph{La Fabuleuse Histoire de
l\textquotesingle invention de l\textquotesingle écriture}. (Traduit de
l\textquotesingle italien par Jacques Dalarun) Paris: Éditions du Seuil,
2021 {[}gedruckt{]}

Ferrara, Silvia (2023). \emph{Avant l\textquotesingle écriture: Signes,
figures, paroles. Voyage au source de l\textquotesingle imagination}.
(Traduit de l\textquotesingle italien par Jacques Dalarun) Paris:
Éditions du Seuil, 2023 {[}gedruckt{]}

Diese beiden Bücher sind aufeinander bezogen. Beide erschienen je zwei
Jahre vor ihrer französischen Übersetzung (die hier für die Besprechung
herangezogen werden) im italienischen Original (also 2019 respektive
2021). Die Autorin ist Professorin für mykenische Philologie oder anders
gesagt: Expertin für eine vorantike Kultur im heutigen Griechenland. In
ihren Büchern geht es ihr um die Frage, wie Menschen weltweit -- und
also nicht nur im heutigen Griechenland -- begannen, Schrift und
Schriftkultur zu entwickeln sowie wie sie zuvor Erfahrungen
aufzeichneten. Während es im ersten Buch darum geht, wie, wo, wieso und
von welchen Menschen in früheren Kulturen die Schrift \enquote{erfunden} wurde,
geht es im zweiten Buch um Zeichen und Symbole, die von
\enquote{vorschriftlichen} Kulturen hinterlassen wurden.

Zum Thema passend kann Ferrara für diese Themen fast nicht mit
schriftlichen Quellen arbeiten, sondern muss erste Symbolsysteme, die
Schriften sein könnten oder aber auch \enquote{noch nicht wirklich Schriften
sind}, interpretieren oder -- im zweiten Buch -- Symbole (Einritzungen
in Berge, Höhlenzeichnungen, intentional angeordnete Steine und so
weiter), die teilweise mehrere 10.000 Jahre alt sind, zum Ausgang ihrer
Überlegungen nehmen. Beide Bücher zeichnet aus, dass hier die Ergebnisse
historischer Forschung mit expliziten Spekulationen und Mutmassungen
verbunden werden. Dabei wird nicht versucht, die Grenzen zwischen Fakten
und Interpretationen zu vermischen. Aber es liest sich nicht wie ein
historisches Fachbuch, sondern ein wenig so, als würde Ferrara in
geselliger Runde begeistert von ihrem Thema, von archäologischen Funden
und dann von ihren Vermutungen erzählen. Dabei werden im Text sowohl
recht unbekannte Funde präsentiert als auch prominente wie zum Beispiel
die Bauten in Göbekli Tepe.

Was Ferrara offenbar fasziniert, ist, dass Menschen (und nicht nur Homo
sapiens) über mehrere Jahrtausende immer wieder dazu angetrieben wurden,
entweder (im zweiten Buch) Spuren, Zeichnungen und ähnliches zu
hinterlassen oder (im ersten Buch) von diesen Zeichen ausgehend
Zeichensysteme zu entwickeln, die dann zu Schriften wurden. Immer wieder
versucht Ferrara, sich vorzustellen, in welchen Situationen und zu
welchen Zwecken das passierte. Und sie fragt auch, warum es immer wieder
neu passierte. An den Grenzen zur Schriftlichkeit scheint Ferrara einen
Willen der damaligen Menschen zu vermuten, ihre Gedanken zu hinterlassen
und zu kommunizieren. Gleichzeitig scheinen Schriftsysteme auch wieder
eingegangen zu sein. In einem Kapitel geht Ferrara auf die Geschichte
der chinesischen Schrift ein, die gleichsam \enquote{fertig} die Welt betritt --
in einem Grab (von \enquote{Lady} Fu Hao), das rund 1.200 Jahre vor unserer
Zeitrechnung angelegt wurde, findet sich der erste Text in dieser
Schrift, ohne dass wir (bislang) Vorläufersysteme kennen, aus denen sie
sich entwickelt hat. Auch dieses Beispiel lässt die gleichen Fragen
offen: Wie wurde dieses komplexe Schriftsystem erfunden, ohne weitere
Spuren hinterlassen zu haben? Warum so und nicht anders? Was sagt diese
Form der Schrift über die Menschen, die sie benutzt haben? Das Thema
steht also immer an der Grenze von Verständlichkeit und
Nicht-Verständlichkeit, von Vermutungen und Fakten.

Beide Bücher sind erfrischend zu lesen und ähneln einem langem Gespräch
mit einer Person, die sich in ein interessantes Spezialthema
eingearbeitet hat und die hier hinter die sonst so alltägliche
scheinende \enquote{Schrift} zu schauen versucht. Wobei Ferrara dabei ein
erfrischend humanistisches Interesse an allen Menschen hat, welches die
ganzen Vermutungen und Darstellungen durchzieht: Immer geht es darum,
die \enquote{früheren Menschen} in ihrer ganzen Komplexität zu verstehen
und wertzuschätzen. (ks)

\begin{center}\rule{0.5\linewidth}{0.5pt}\end{center}

Charles, Sara J. (2024). \emph{The Medieval Scriptorium: Making Books in
the Middle Ages}. London: Reaktion Books, 2024 {[}gedruckt{]}

Das Buch gibt einen gut lesbaren, wenn auch an Stellen vielleicht etwas
zu detaillierten, Überblick zur Buchherstellung in der Spätantike und im
(europäischen) Mittelalter, beginnend mit den ersten christlichen
Schriften im zweiten Jahrhundert unserer Zeitrechnung und endend mit der
Verbreitung des Buchdrucks. Dabei geht Sara J. Charles nicht nur auf die
Bücher selber, sondern auch auf das ein, was über die Herstellung der
Handschriften und der Materialien für diese (Pergament, Tinten und
Farben, Schreibwerkzeuge, das Buchbinden) bekannt ist. Zudem stellt sie
dar, was überhaupt über die \enquote{Skriptorien} an den europäischen Klöstern
nachgewiesen ist. Dieser Fokus auf die Herstellungsprozesse hebt das
Buch von anderen Werken zu diesem Thema ab. Während Beispiele
mittelalterlicher Handschriften in vielen Büchern -- von \enquote{Coffee Table
Books} bis zu wissenschaftlichen Arbeiten -- oft präsentiert werden und
auch die historische Entwicklung der verwendeten Schriften gerne
dargestellt wird, werden die Fragen, wie diese Bücher überhaupt
hergestellt werden, zumeist nur kurz angeschnitten -- dann oft mit
Zitaten von Schreibenden, die sich in den Manuskripten selber beklagen,
wie anstrengend ihre Arbeit sei.

Was in diesem Buch zu lernen ist, ist, dass die Herstellungsprozesse
recht anstrengend und unhygienisch waren, beispielsweise wenn bei der
Produktion von Pergament Urin eingesetzt wurde. Gleichzeitig zeigen die
verwendeten Materialien auch, wie stark die mittelalterlichen Klöster
notwendigerweise für einige der Inhaltsstoffe in weitreichende
Handelsnetzwerke eingebunden waren. Gleichzeitig zeigt Charles, dass die
Vorstellung von kontinuierlich arbeitenden, extra eingerichteten
Skriptorien wohl falsch ist. Es gibt nur einige Hinweise (auf dem St.
Galler Klosterplan und in einigen Texten), aber keine archäologischen
Funde, die solche nachweisen würden. Wenn, dann existierten sie wohl nur
für eine bestimmte Zeit. Aber eher scheinen \enquote{schreibende} Mönche
und Nonnen oft allein gearbeitet zu haben.

Jedes Kapitel des Buches beginnt mit einer erfundenen Geschichte über
eine Person, die an der spätantiken oder mittelalterlichen
Buchproduktion beteiligt ist -- immer zum jeweiligen Thema des Kapitels
passend und basierend auf Personen, deren Existenz nachgewiesen ist,
also beispielsweise Schreibern und Illustratorinnen, die sich in
Handschriften selber genannt haben oder, im letzten Kapitel zum Ende der
Buchproduktion, Johannes Gutenberg. Nach dieser Geschichte wird jeweils
in einer erzählenden Weise das Thema des Kapitels dargestellt. Dabei ist
zu merken, dass Charles ein persönliches Interesse an den konkreten
Herstellungsprozessen hat. Die dazu vorhandenen Kapitel lesen sich
lebhafter als die anderen und sind teilweise mit Bildern aus der
Herstellung illustriert, die Charles selber experimentell durchgeführt
hat. Eine anzubringende Kritik ist, dass in anderen Kapiteln oft auf
Handschriften als Beispiele für jeweils geschilderte Fakten verwiesen
wird, ohne dass diese abgebildet werden (oder alternativ auf
Digitalisate verwiesen wird). Dies scheint eher dem Verlag anzulasten zu
sein, da diese Manuskripte selbstverständlich alle gemeinfrei sind.
Davon abgesehen kann man dieses Buch aber sehr als Einführung ins Thema
empfehlen. (ks)

\begin{center}\rule{0.5\linewidth}{0.5pt}\end{center}

Nikolaizig, Andrea ; Kohl, Annika (2024). \emph{Helene Petrenz und die
Ernst-Abbe-Bücherei Jena 1865 -- 1899 -- 2023}. Leipzig: Leipziger
Universitätsverlag, 2024 {[}gedruckt{]}

Helene Petrenz war die Bibliothekarin, welche in Jena in den ersten
Jahren des 20. Jahrhunderts als Direktorin die dortige \emph{Lesehalle
und Volksbibliothek} -- Vorgängerin der heutigen \emph{Ernst Abbe
Bücherei} -- aufbaute. Diese war damals eine der grössten und modernsten
Volksbibliotheken in Deutschland. Über die Jahre wurde sie deshalb zum
Beispiel in der volksbibliothekarischen Presse immer wieder
thematisiert. Und dennoch ist Petrenz selber kaum bekannt, so wie die
Arbeit der meisten Bibliothekar*innen ausserhalb ihres jeweiligen
Wirkungskreises für die Nachwelt meist unbekannt bleibt. Dass der
Förderkreis der heutigen Bücherei diese Publikation über Petrenz
veranlasste, ist zu begrüssen: Petrenz leistete in einer Zeit, in der
die Gesellschaft sich rapide veränderte und (bürgerliche) Frauen sich
ihren Platz in der Arbeitswelt erst erobern mussten, eine
kontinuierliche, selbstbewusste und wirksame Arbeit.

Die reich bebilderte Publikation basiert auf zahlreichen Quellen von und
über Petrenz sowie der Volksbibliothek selber. Dabei, so wird im Text
sichtbar, ist aber auch vieles heute einfach nicht mehr bekannt. Die
beiden Autorinnen können Einiges über das Leben und Wirken Petrenz
zeigen, aber wirklich an die Person Helene Petrenz kommen sie nicht mehr
heran. Diese hat vor allem Quellen über ihre Arbeit, nicht aber über ihr
Leben hinterlassen. Wir wissen, wo sie geboren wurde, wo sie starb, mit
wem sie verheiratet war und auch, wie viele Kinder sie hatte. Aber was
sie dachte, welche Ziele sie hatte, welche Träume und welche Erfolge
oder Misserfolge -- das wissen wir praktisch nicht mehr. Auch das wird
für das Leben vielen Bibliothekar*innen gelten. Die
Bibliotheksgeschichte ist halt fast immer eine Geschichte der
Einrichtungen und nicht der einzelnen Bibliothekar*innen.

Dennoch ist die Publikation mit Vorsicht zu lesen. Zuerst fällt auf,
dass der Text nicht so inhaltsreich ist, wie es auf den ersten Blick
erscheint. Nicht nur ist der Band, wie schon gesagt, mit sehr vielen
Abbildungen ausgestattet. Auch zitiert der Text an vielen Stellen
unnötig lang einzelne Statistiken oder Briefe, die zum eigentlichen
Inhalt wenig beitragen. Am erstaunlichsten ist aber, dass an vielen
Stellen der Eindruck auftaucht, als wären die Autorinnen -- obgleich
beide im Bibliothekswesen aktiv -- praktisch nicht mit der Geschichte
der Volksbüchereien vertraut. Immer dann, wenn es um die Einordnung der
professionellen Aktivitäten von Petrenz geht, wird der zeitgenössische
Kontext, welcher für eine solche Bewertung nötig wäre, nicht wirklich
dargestellt. Die Autorinnen präsentieren zum Beispiel den
Richtungsstreit -- eine heftig geführte ideologische Auseinandersetzung
über die Aufgaben und Arbeitsweisen der deutschen und österreichischen
Volksbüchereien, welche in den 1910er bis 1930er Jahren geführt wurde --
praktisch nur als eine Auseinandersetzung um das Verhalten einer Person
(Walter Hofmann) und das auch nur, weil Petrenz einmal einen Offenen
Brief unterzeichnet hatte, der sich an Hofmann richtete. Sie verorten
auch die Arbeit der Volksbibliothek überhaupt nicht in der
\enquote{normalen} Arbeit der damaligen Zeit, können also nirgends
zeigen, was an der Bibliothek in Jena besonders modern oder gerade nicht
modern war. Einzig, dass sie als eine der ersten eine Frau als
Direktorin einsetzte (wenn auch gerade nicht als \enquote{erste Wahl},
wie im Text dargestellt wird), können sie hervorheben. Erstaunlich ist
auch, dass sie sich offenbar gar nicht die Frage stellen, wie es in der
damals aufstrebenden Industriestadt um andere Büchereien (vor allem um
Arbeiterbibliotheken und Leihbuchhandlungen) stand, so dass bei ihnen
die Volksbücherei als eine alleinstehende Einrichtung erscheint, nicht
als Teil einer Landschaft von Büchereien, die sie mit hoher
Wahrscheinlichkeit war. Sie scheinen bei der -- wie sie schreiben --
\enquote{Spurensuche} vergessen zu haben, Quellen kritisch zu lesen und
zu kontextualisieren.

Insoweit ist die Publikation zwar zu begrüssen und vermittelt mit den
Abbildungen auch tatsächlich ein \enquote{Bild} der damaligen Arbeit in
der Volksbibliothek in Jena. Das Leben von mehr Bibliothekar*innen
sollte so sichtbar gemacht werden. Es ist gewiss auch ein Beitrag zur
Stadtgeschichte Jenas. Aber als Beitrag zur Bibliotheksgeschichte ist
sie kritisch zu sehen. (ks)

\begin{center}\rule{0.5\linewidth}{0.5pt}\end{center}

Wolcott, Renée (2022). \emph{Preserving Useful Knowledge: A History of
Collection Care at the APS Library}. (Transactions of the American
Philosophical Society, Volume 111, Part 1.) Philadelphia: American
Philosophical Society Press, 2022 {[}gedruckt{]}

Die Autorin ist aktuell eine der Bibliothekar*innen, welche mit dem
Erhalt und der Reparatur der Bestände in der Bibliothek der American
Philosophical Society beauftragt ist. Dies ist eine der ältesten
wissenschaftlichen Vereinigungen in den USA, gegründet einige Jahre vor
der US-amerikanischen Revolution, unter anderem von Benjamin Franklin.
Die ältesten Bücher und Manuskripte der Sammlung stammen aus dieser
Gründungszeit. Das Buch stellt die Ausarbeitung eines längeren Vortrags
vor, in dem die Autorin die Geschichte ihre Vorgänger*innen und deren
Arbeit darstellte. Unterteilt in vier Hauptkapitel tut sie dies dann
auch in dieser Publikation. Im ersten Kapitel werden Buchbindearbeiten
im ersten Jahrhundert der Bibliothek bis Mitte des 19. Jahrhunderts
vorgestellt, im zweiten einzelne Buchbinder*innen und Restaurator*innen
sowie deren Arbeit. Nicht zu allen liegen heute viele biographische
Angaben vor. Schwierig ist auch, dass sich erst in den letzten
Jahrzehnten etabliert hat, dass bei solchen Arbeiten jeweils genau
dokumentiert wird, warum ein Buch oder Manuskript repariert und wie
dabei vorgegangen wurde. Daher muss die Autorin immer wieder Spuren
früherer Arbeiten an den Medien durch Autopsie interpretieren. Sie ist
dabei immer wertschätzend gegenüber ihren Vorgänger*innen, stellt aber
auch mehrfach dar, dass diese viele Entscheidungen trafen, die bei der
heutigen Restaurierungsarbeit nicht mehr getroffen werden würden. Dann,
im dritten Teil, werden relativ aktuelle Reparaturen vorgestellt,
inklusive bildlicher Darstellungen. Hier können die Leser*innen
praktisch einen Einblick in die Werkstatt von Buch- und
Papierrestaurator*innen erhalten. Im abschliessenden vierten Kapitel
gibt die Autorin nochmal einen Überblick darüber, welche Entscheidungen
getroffen werden müssen, wenn ein Bestand, wie der von ihr betreute,
langfristig erhalten werden soll.

Das Buch erschien in der wissenschaftlichen Publikationsreihe der
Society und richtet sich offenbar an Personen, die auf der einen Seite
wohl keinen professionellen Bezug zu dieser Arbeit haben, aber auf der
anderen Seite nicht davon überzeugt werden müssen, dass sie notwendig
ist. Sicherlich ist diese Arbeit heute ein Spezialthema des
Bibliothekswesens. Dieses Buch bietet einen guten Einblick in deren
Status Quo, wenn auch am Beispiel einer ausserordentlich gut mit
Ressourcen ausgestatteten Einrichtung. (ks)

\begin{center}\rule{0.5\linewidth}{0.5pt}\end{center}

Pavillon, Olivier (2024). \emph{Les Maisons du Peuple de Lausanne
(1899-1945).} (Collection Histoire) Lausanne: Éditions Antipodes, 2024
{[}gedruckt{]}

Maisons du Peuple -- in deutsch \enquote{Volkshaus}, in italienisch
\enquote{Casa del Popolo} -- waren vor allem in der ersten Hälfte des
20. Jahrhunderts Orte der Selbsthilfe und Bildungsbestrebungen der
Arbeiter*innenbewegungen. Oft von Gewerkschaften und linken Parteien,
aber auch von Einkaufsgenossenschaften und Einzelmitgliedern getragen,
stellten sie Einrichtungen dar, in denen verschiedene Bildungs- und
Freizeitaktivitäten stattfanden, Genossenschaften Läden betrieben,
Gesundheitspraxen für Arbeiter*innen unterhalten wurden und so weiter.
Sie waren auch Orte der gewerkschaftlichen und politischen Organisation.
Teilweise waren sie aber auch von Personen aus dem damaligen
\enquote{bürgerlichen} Spektrum getragen, die etwas für Arbeiter*innen
unternehmen wollten. Solche Einrichtungen gab es in verschiedenen
europäischen Städten. In der Schweiz existieren sie heute, wenn auch oft
mit anderen Funktionen -- vor allem als Gewerkschaftshäuser -- teilweise
weiter, unter anderem in Lausanne. Dieses kurze Buch beschäftigt sich
mit der Geschichte der Maison du Peuple in dieser Stadt.

Dabei zeigt der Autor, dass das heutige Maison du Peuple nur eines von
zweien war. Während das heute noch bestehende in den 1930er Jahren
direkt von sozialistischen und gewerkschaftlichen Gruppen gegründet
wurde, war das andere eines, das auf genossenschaftlicher Basis auch von
\enquote{bürgerlichen} Gruppen und Personen getragen wurde. Der Autor
schildert, immer in sehr kurzen Kapiteln und unterstützt von vielen
Abbildungen, die Geschichten der beiden Einrichtungen, der Aktivitäten,
die in ihnen stattfanden und auch der politischen Auseinandersetzungen
in ihnen und um sie. Es ist eine sehr lokale Geschichte, die auch
dadurch unterstützt wird, dass ein Grossteil der Strassen und Adressen,
die genannt werden, heute noch existieren. Wer sich in Lausanne
auskennt, kann die ganze Zeit nachvollziehen, auf was für einer doch
kleinen Fläche sich diese Geschichte abspielte.

Der Rezensent hat dieses Buch nicht nur aus einem lokalhistorischen
Interesse gelesen, sondern auch, um zu erfahren, ob sich etwas über die
Bibliotheken der beiden Maison du Peuple lernen lässt. In dieser
Hinsicht ist das Buch enttäuschend: Schon auf dem Cover findet sich ein
Bild des ersten Volkshauses. Auf diesem Bild ist unter dem Namen der
Einrichtung der explizite Verweis auf eine Bibliothek zu finden. Auch im
Buch taucht diese Bibliothek immer wieder auf. Schon bei der Gründung
des ersten Hauses wird eine solche angedacht und ein Etat für sie
bereitgestellt. Immer wieder wird erwähnt, wie wichtig sie war und dass
die Arbeitsgruppe, welche die Bibliothek betreute, teilweise die grösste
war. Auch wird erwähnt, dass die Bibliothek mehrfach erweitert und
modernisiert wurde. Aber es gibt keine konkreteren Informationen. Der
Autor schreibt über einzelne Vorträge, die in beiden Häusern gehalten
wurden, über Abstimmungen, über handgreifliche Auseinandersetzung in der
Bar des zweiten Maison und so weiter. Aber die Arbeit der Bibliothek --
die ja offenbar im Alltag wichtig war -- thematisiert er nicht. Wie so
oft scheint diese Bibliothek (oder, so klar ist es nicht, ob es sie im
zweiten Haus auch gab, vielleicht dieser beiden Bibliotheken) schnell in
den Hintergrund zu geraten. Irgendwie wichtig, aber offenbar für die
Geschichtsschreibung doch nicht spannend.

Dabei ist noch heute auffällig, dass die Stadtbibliothek in Lausanne
eine Filiale direkt neben dem heute noch existierenden Maison du Peuple
führt. Es wäre interessant gewesen, herauszufinden, ob das Zufall ist,
ob sie einst eine \enquote{Arbeiterbibliothek} gewesen war, die zur
städtischen umgewidmet wurde oder ob sie von der Stadt explizit dort
angesiedelt wurde, um die Bildung der Arbeiter*innen nicht dem
selbstorganisierten Maison du Peuple zu überlassen. Aber leider schweigt
das -- sonst recht kurzweilige -- Buch zum Thema Bibliothek. (ks)

\begin{center}\rule{0.5\linewidth}{0.5pt}\end{center}

Hui, Andrew (2025). \emph{The Study: The Inner Life of Renaissance
Libraries}. Princeton ; Oxford: Princeton University Press, 2025
{[}gedruckt{]}

In diesem Buch geht es, wie im Titel genannt, um die Bibliotheks-, Lese-
und Arbeitsräume von Humanisten (immer Männer), um die Gedanken und
Ängste, die sie in diesen Räumen formulierten und die Netzwerke, die sie
aus diesen Räumen heraus knüpften. Es geht um Träume und Alpträume.
Aber: Es ist ein kulturwissenschaftliches Werk, kein rein historisches.
Es gibt schon frühere Arbeiten, welche die privaten Bibliotheken der
Humanisten untersuchten, auf die Hui hier aufbaut. Was ihn interessiert,
sind die Welten, die in den Köpfen der Humanisten entstanden, als sie in
diesen Privatbibliotheken arbeiteten.

Das alles ist ein Parforceritt durch die Aufklärung und gleichzeitig
Interpretation von Schlüsseltexten sowie Bildern der damaligen Zeit,
immer wieder zurückgeführt auf den Raum Bibliothek oder die Sammlung von
Büchern, mit denen sich die Aufklärer umgaben. Gleichzeitig schreibt
sich Hui persönlich mit seinen Gedanken in den Text. Innovativ sind die
Verweise auf ähnliche Diskurse, Denk- und Bildprogramme, wie sie in
asiatischen Kulturen (hier China und Japan) existierten, die Hui an
verschiedenen Stellen (aber nicht systematisch) einflicht.

Hui beschreibt sein Buch als Essay und, wohl getreu der
kulturwissenschaftlichen Herangehensweise, muss man es auch so lesen. Es
ist keine neue Geschichte der genannten Bibliotheken, sondern eine -- in
vielen Teilen auch anregende -- Denkübung über Bibliotheken und
Humanismus. Für diese bietet die Geschichte, hier als chronologische
Abfolge verstanden, ein Organisationsprinzip, an dem sich entlang
gehangelt wird. (ks)

\begin{center}\rule{0.5\linewidth}{0.5pt}\end{center}

Blume, Patricia F. (2024). \emph{Die Geschichte der Leipziger Buchmesse
in der DDR: Literaturtransfer, Buchhandel und Kulturpolitik in
deutsch-deutscher Dimension}. Berlin, Boston: Walter de Gruyter, 2024,
\url{https://doi.org/10.1515/9783111317076}

Diese Arbeit stellt die Entwicklung der Leipziger Buchmesse und ihrer
Vorläufer von 1946 bis 1990, einschliesslich einem Ausblick auf das
\enquote{Überleben} und \enquote{Neuerfinden} in der Zeit danach, dar, also die Zeit
nach dem Ende des Zweiten Weltkrieges bis zum Ende der DDR. Das alles
passiert umfassend, detailliert, teilweise ausschweifend. Quellen sind
Archivmaterialien, offizielle Veröffentlichungen, Berichte des
Ministeriums für Staatssicherheit, Interviews mit Beteiligten und mit
damaligen Besucher*innen. Einige dieser Quellen -- mindestens die
Interviews mit ehemaligen Besucher*innen -- wurden auch mit Studierenden
in einem Seminar er- und bearbeitet. Insgesamt ist die Arbeit
gleichzeitig eine Dissertation, für die sich Detailgenauigkeit anbietet.
Aber in gewisser Weise stellt diese Fülle an Quellen auch das Problem
der Arbeit dar: Sie ist über 700 Seiten lang und äusserst genau. Im
Grunde stellt sie mehrere Bücher dar. Gerade der Abschnitt zur
Überwachung der Messe durch die Staatssicherheit ist so fokussiert
geschrieben und umfassend, dass er auch hätte selbstständig erscheinen
können. Die Detailtreue führt auch dazu, dass bei der Darstellung nicht
immer klar wird, ob umfassende Entwicklungen geschildert werden oder ob
es um die inhaltlichen Positionen und Entscheidungen von Einzelpersonen
geht. Teilweise scheint es in der Darstellung, als wären bestimmte
Entwicklungen nur aufgrund von Einzelpersonen geschehen. Das alles führt
dazu, dass die Arbeit die Lesenden stellenweise mit Fakten, Daten und
Einschätzungen überhäuft. Es stellt sich dann der Eindruck ein, dass es
notwendig ist, sich \enquote{durch den Text zu kämpfen}, um zwischen den Details
nicht die grosse Geschichte zu verpassen. Dass eine historische
Dissertation keine für die breite Öffentlichkeit geschriebene
Darstellung ist, ist dieser Arbeit anzumerken.

Nichtsdestotrotz ist sie selbstverständlich relevant. Sie zeigt, wie die
Buchmesse als \enquote{Mustermesse} gegründet wurde, welche die Produktion der
sowjetischen Besatzungszone widerspiegeln sollte und dabei für den
Buchbereich an die traditionelle Leipziger Messe für die Buchbranche
(aber nicht für ein allgemeines Publikum) anknüpfte, die während des
Zweiten Weltkrieges eingestellt worden war. Es wird sichtbar, wie die
Messe in der Folge weg entwickelt wurde von einer Veranstaltung, bei der
vor allem Verlage und Sortimente (also der Buchhandel) miteinander
verhandelten und Geschäfte machten, hin zu einer Publikumsmesse. Das
geschah unter anderem, weil sich andere Formen des Buchhandels (neue
Bestellverfahren) entwickelten. Aber bis zum Ende der DDR geschah es
auch immer im Rahmen der gesamten Leipziger Messe. Alle Versuche, die
Buchmesse als eigenständige Veranstaltung zu etablieren, schlugen fehl.

Die Arbeit ist grösstenteils chronologisch aufgebaut -- nur der
Abschnitt zur Überwachung durch die Staatssicherheit weicht davon ab,
weil er einem Zeitraum zugeordnet wurde, aber eigentlich für die gesamte
Zeit der DDR gilt -- und folgt Einteilungen, die sich aus der
Entwicklung der Buchmesse selber ergeben. Beispielsweise ist der Umzug
der Messe in ein eigenes Haus der Punkt, an dem ein Kapitel endet.
Gleichzeitig zeigt die Arbeit, dass die Messe sich auch immer mit der
allgemeinen Kulturpolitik der DDR (und, da diese immer auch auf die BRD
bezogen war, auch die der BRD) zusammenhing. Die Schliessungen gegenüber
Westdeutschland, der Mauerbau, die Ausweisung von Wolf Biermann, die
Niederschlagung des Prager Frühlings und darauffolgende kulturpolitische
Entscheidungen, die Angst vor dem \enquote{Übergreifen} der Solidarność sowie
der Perestroika und auch die vorsichtige Annäherung an die BRD ab den
1970er Jahren, spielten alle eine Rolle für die Buchmesse. Sie war immer
als Veranstaltung gedacht, die eine Aussenwirkung haben sollte und
deshalb zum Beispiel immer daran interessiert war, möglichst viele
unterschiedliche Länder zu präsentieren und zudem aus der BRD möglichst
viele Verlage. Gleichzeitig geriet die Messe immer wieder neu in Krisen,
die regelmässig angegangen werden mussten.

Was in der Arbeit aber auch klar wird, ist, dass der Messe und der
Literatur von Seiten der DDR-Führung (und der Staatssicherheit) eine
erstaunlich hohe Wirkung zugeschrieben wurde. Sie musste ständig
überwacht, ausgehandelt und auf ihren möglichen Einfluss auf die
DDR-Bevölkerung hin abgeklopft werden. Bibliotheken kommen im Werk
allerdings nur am Rand vor -- teilweise als Einrichtung, welche die
Zensur unterstützen, teilweise als Teil der Kund*innen. Ansonsten ist
dies Literatur-, aber keine Bibliotheksgeschichte. (ks)

\section{4. Weitere wissenschaftliche Medien (Konferenzberichte,
Abschlussarbeiten)}\label{weitere-wissenschaftliche-medien-konferenzberichte-abschlussarbeiten}

{[}Diesmal keine Beiträge{]}

\section{5. Populäre Medien (Social Media, Zeitungen, Radio,
TV)}\label{populuxe4re-medien-social-media-zeitungen-radio-tv}

Knibbs, Kate (2023). \emph{The Battle Over Books Could Change AI
Forever.} In: Wired, 04.09.2023,
\url{https://www.wired.com/story/battle-over-books3/}

Der Artikel beschreibt, wie es dazu kam, dass ein Datensatz von circa
196.000 digitalen Büchern als Trainingsgrundlage für eine Vielzahl von
Large Language Model KI-Anwendungen verwendet wurde. Dabei kommen
verschiedene Personen zu Wort, die diese Verwendung ablehnen oder
befürworten. Die ablehnende Position wird vor allem durch
Urheberrechtsverstöße und die Verletzung der Rechte der Autor:innen
begründet. Befürworter:innen sehen durch die freie Verfügbarkeit des
Datensatzes insbesondere die Chancengleichheit kleiner und
nicht-kommerzieller Anbieter von KI-Anwendungen gewahrt, die sonst durch
finanzstarke Großkonzerne übervorteilt würden. Der Artikel gibt damit
einen interessanten Einblick in die Sichtweisen der verschiedenen
Akteure. (eb)

\begin{center}\rule{0.5\linewidth}{0.5pt}\end{center}

Savin, Serge (März 2025). \emph{Biblioteket som en varmestue}. In:
Akademikerbladet.
\url{https://dm.dk/akademikerbladet/magasinet/2025/dm-akademikerbladet-nr-1-2025/biblioteket-som-en-varmestue/}

In der Mitgliederzeitschrift \emph{Akademikerbladet} der dänischen
Gewerkschaft DM wird über die Herausforderungen im Umgang mit
auffälligen Personen in öffentlichen Bibliotheken aus Sicht des
Bibliothekspersonals berichtet. Ausgangspunkt ist eine Umfrage der
Gewerkschaft, nach der 90 \% der Beschäftigten Unruhe, Geschrei und
unangemessenes Verhalten erlebt haben, davon 50 \% mindestens einmal
täglich oder wöchentlich. Diese gehen vor allem von Obdachlosen,
psychisch Erkrankten und anderen benachteiligten Personengruppen aus.
Sie treffen in der Regel auf unvorbereitetes Bibliothekspersonal, das
nicht dafür ausgebildet ist, mit solchen Konfliktsituationen umzugehen.
Am Beispiel der Bibliothek Næstved werden Anpassungsstrategien
diskutiert, die sowohl die Sicherheit des Personals (Dienstausweise mit
integrierten Notrufknöpfen, Fluchtwege für das Personal,
Videoüberwachung) als auch bauliche Veränderungen zur Verhinderung von
Vandalismus (Wegfall der Waschbecken in den Toiletten zugunsten einer
zentralen Waschgelegenheit auf dem Flur) umfassen. Besonders
erfolgversprechend erscheint der Einsatz von sogenannten
\enquote{relationsmedarbejder}, die über eine sozialpädagogische
Ausbildung verfügen. Sie sprechen Personen mit auffälligem Verhalten an,
vermitteln sozialpsychologische Unterstützungsangebote der Kommune und
ziehen gegebenenfalls die Polizei hinzu. In den Kopenhagener
Stadtteilbibliotheken in Vesterbro konnte so die Zahl der Vorfälle um 60
\% gesenkt werden, so dass nun auch in Nørrebro
\enquote{relationsmedarbejder} eingesetzt werden. Auch in Næstved wird
auf die Zusammenarbeit mit Streetworkern und Polizei gesetzt. Die
Gewerkschaft DM unterstützt ihre Mitglieder mit virtuellen
Konfliktseminaren und setzt sich bei den Bibliotheksleitungen für mehr
Sicherheit ein. (nj)

\begin{center}\rule{0.5\linewidth}{0.5pt}\end{center}

Redaktion back-aktuell, js (15.04.2025). \emph{\enquote{Dies ist ein Werk mit
umstrittenem Inhalt}: Autor muss Warnhinweis einer Bücherei dulden.} In:
beck-aktuell. Heute im Recht.
\url{https://rsw.beck.de/aktuell/daily/meldung/detail/vg-muenster-1l5925-einordnung-hinweis-buch-gerechtfertigt-stadtbuecherei}

Im Blogeintrag wird kurz über einen Beschluss des Verwaltungsgerichts
Münster berichtet. Ein Autor hatte in einem Eilverfahren verlangt, dass
ein Hinweis entfernt wird, den die Stadtbücherei Münster in einem von
ihm verfassten Buch angebracht hatte. Es wurde darin auf den
umstrittenen Inhalt des Buches, das insbesondere auch die
Atombombenabwürfe in Hiroshima und Nagasaki bestreitet, hingewiesen.

Das Verwaltungsgericht entschied, dass dieser Hinweis zulässig ist. In
der Begründung wurde dabei auf die folgenden Punkte verwiesen: 1)
Bibliotheken können inhaltlich durch Empfehlungen oder durch Kritik zu
ihren Medien Stellung beziehen. 2) Bibliotheken haben einen
Bildungsauftrag, der durch das reine Verleihen von Büchern nicht gedeckt
sei. 3) Die Bibliothek sei nicht verpflichtet, gegenüber dem Autor
neutral zu sein. 4) Die Bibliothek sei dem Sachlichkeitsgebot gefolgt,
da der Hinweis auf Tatsachen beruhe. (eb)

\section{6. Weitere Medien}\label{weitere-medien}

\emph{Case Tracker: Artificial Intelligence, Copyrights and Class
Actions},
\url{https://www.bakerlaw.com/services/artificial-intelligence-ai/case-tracker-artificial-intelligence-copyrights-and-class-actions/}
(abgerufen: 16.12.2024)

Auf dieser Seite bietet eine amerikanische Anwaltskanzlei eine laufend
aktualisierte, detailreiche Übersicht über laufende Gerichtsverfahren
zum Thema KI und Copyright. Dies beschränkt sich zwar auf die
Vereinigten Staaten, aber da die einschlägigen Konzerne dort ihren Sitz
haben, gibt es potentiell Auswirkungen für den gesamten KI-Markt. (eb)

\begin{center}\rule{0.5\linewidth}{0.5pt}\end{center}

Harington, Robert (2024). \emph{A Dissonance of Ideals: Openness,
Copyright, and AI}. In: The Scholarly Kitchen, 25.11.2024,
\url{https://scholarlykitchen.sspnet.org/2024/11/25/robert-harington-attempts-to-reveal-inherent-conflicts-in-our-drive-to-be-as-open-as-possible-authors-need-to-understand-their-rights-and-a-librarys-mandate-to-provide-their-patron/}

Der Artikel diskutiert -- unter dem Stichwort der kognitiven Dissonanz
-- eine Zwickmühle, in der sich Bibliotheken befänden, die mit den neuen
Tatsachen der KI-Anwendungen konfrontiert sind. Auf der einen Seite
setze man sich im Zuge der Open-Access-Bewegung dafür ein, dass
Autor:innen ihre Publikationen so offen wie möglich verbreiten und
gleichzeitig die Rechte an ihren Werken weitestmöglich wahren. Auf der
anderen Seite würden die Anbieter von KI-Anwendungen und eben auch das
der KI zugrundeliegende technische Prinzip eben diese Rechte der
Autor:innen routinemäßig verletzen. Gleichzeitig sollen Bibliotheken
ihren Nutzer:innen bestmögliche Unterstützung bei Forschung und
Recherche bieten und kämen dabei früher oder später nicht um
KI-Anwendungen herum.

Der Artikel bietet keine Lösung für diese widersprüchlichen Prinzipien,
verweist aber auf erste, punktuelle Anstrengungen zur Behebung des
Problems und weitere Rahmenbedingungen und Informationsressourcen. (eb)

\begin{center}\rule{0.5\linewidth}{0.5pt}\end{center}

Jasper Franz T. Mapa: \emph{Pirates of the Academe: A Critical
Criminological Analysis of Intellectual Property Laws Criminalizing
Filipino College Students Using Pirated Papers from Sci-Hub.} Manila: De
La Salle University Publishing House. 2024. Asia-Pacific Intellectual
Property Management and Innovation Book Series. 3.
\url{https://animorepository.dlsu.edu.ph/apipmibookseries/3}

Die Studie untersucht am Beispiel von Sci-Hub die Nutzung von
Schattenbibliotheken durch Studierende in den Philippinen. Sie kommt zu
dem Ergebnis, dass die Nutzung aus juristischer Sicht zwar gesetzlich
untersagt ist (\emph{malum prohibitum}), aber nicht inhärent unrecht ist
(\emph{malum in se}). Die Angebote tragen sogar zu einer mitunter
gewünschten besonderen Verbreitung von Inhalten aus Kunst und
Wissenschaft bei, so dass sie aus der Perspektive unterschiedlicher
Interessen reflektiert werden müssen. Sie erfolgt in der Praxis der
Studierenden aus dem Bedarf heraus, für die wissenschaftliche Ausbildung
relevante Quellen einsehen und zitieren zu können. Der Bedarf ist
besonders ausgeprägt, wenn aus sozioökonomischen Gründen keine andere
Möglichkeit zum Zugriff auf diese Inhalte besteht. Dies betrifft
überproportional Studierende in Ländern des Globalen Südens. Als
Lösungen für die sich aus dem Gebrauch von Schattenbibliotheken
ethischen ergebenden Problemen werden unterschiedliche Ansätze
vorgeschlagen, unter anderem der Ausbau von Open-Access- beziehungsweise
alternativen Publikationsmodellen; gegebenenfalls verzögerte
Open-Access-Publikation nach einem zeitlichen Embargo, eine stärkere
auch forschende Problematisierung von Zielkonflikten zwischen
Urheberrecht beziehungsweise Intellectual Property und den
Zugangsinteressen. (bk)

\begin{center}\rule{0.5\linewidth}{0.5pt}\end{center}

Rhea Nayyar (2025): \emph{Historic Buildings Destroyed in Southern
California Blazes}. In: Hyperallergic / News. 08. Januar 2025.
\url{https://hyperallergic.com/982535/it-looks-like-a-bomb-exploded-la-artists-grapple-with-loss-as-fires-rage/}

Während des Palisades Fire brannte im Januar 2025 in Pacific Palisades,
Los Angeles die dortige Palisades Branch Library, Zweigstelle der Los
Angeles Public Library, vollständig nieder. (bk)

\begin{center}\rule{0.5\linewidth}{0.5pt}\end{center}

Lisa Yin Zhang (2025): \emph{Raquel Rabinovich, Artist of Submerged
Worlds, Dies at 95}. In: In: Hyperallergic / Obituaries. 10.01.2025.
\url{https://hyperallergic.com/982551/raquel-rabinovich-artist-of-submerged-worlds-dies-at-95/}

Am 05. Januar 2025 starb die aus Argentinien stammende Künstlerin Raquel
Rabinovich, zu deren Zentralwerk die Langzeitserie \enquote{River
Library} gehört, eine Serie von Zeichnungen auf handgeschöpftem Papier,
für die sie den Schlamm aus verschiedenen Flüssen als Material
verwendet. Im Schlamm und seiner Eigenschaft als Schichtung sowohl als
Lebensraum als auch von totem Material sah sie die Erdgeschichte und die
der Menschheit kumuliert. (bk)

\begin{center}\rule{0.5\linewidth}{0.5pt}\end{center}

o.A. (2025): \emph{\enquote{Bibliothek der Dinge} für blinde Kinder
eröffnet}. In: Deutschlandfunk / deutschlandfunk.de. 07.01.2025
\url{https://www.deutschlandfunk.de/bibliothek-der-dinge-fuer-blinde-kinder-eroeffnet-104.html}

Die in Marburg ansässige Deutsche Blindenstudienanstalt e. V. (blista)
eröffnete, wie verschiedene Medien melden, Anfang Dezember 2024 die
erste \enquote{Deutsche Blinden-Mediathek}. (bk)

%autor

\end{document}