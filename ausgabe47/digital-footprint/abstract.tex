\textbf{Zusammenfassung}: Publizieren ist für Wissenschaftler*innen
essentiell. Dabei ist es heutzutage kaum möglich, an digitalen Angeboten
vorbeizukommen. Nur wenigen Forschenden ist dabei bewusst, dass auch bei
digitalen Verlagsangeboten persönliche Daten getrackt und missbraucht
werden können.

Ein Auftrag von Bibliotheken ist es, Forschende zu unterstützen und
Informationskompetenz zu stärken. Deshalb fällt das Stärken von
Bewusstsein für Data Tracking durch wissenschaftliche Verlage in ihr
Aufgabengebiet. Dies kann unter anderem durch Workshops passieren.
Unsere Frage hierbei ist, wie Wissenschaftler*innen, speziell Early
Career Researchers in den Naturwissenschaften, Datentracking durch
wissenschaftliche Verlage, die negativen Auswirkungen und
Abwehrmöglichkeiten gegen Datentracking vermittelt werden können. In
unserem Workshop-Konzept haben wir zusätzlich zu Kurzvorträgen ein Spiel
als eine simulationsbasierte Aufgabe mit Rollenspiel-ähnlichen Merkmalen
entwickelt, das auf einem Entdecken-lassenden-Ansatz basiert. Mithilfe
dessen wird ein realitätsnaher Publikationsprozess durchgespielt. Hier
werden, abhängig von den getroffenen Entscheidungen, personenbezogene
Daten getrackt.

Dieses Konzept wurde in einem Pre-ISI-Workshop (ISI = Internationales
Symposium für Informationswissenschaft) am 17. März 2025 vorgestellt und
durchgeführt; das prototypische Kartenspiel wurde getestet und Feedback
eingeholt. Dieses Kartenspiel wird für die Weiterverwendung (und
Weiterentwicklung) als Open Educational Resource freigegeben. Mögliche
Anpassungen sind beispielsweise, den Workshop auch auf Studierende
auszurichten und ihnen so den Publikationsprozess oder den
Schreibprozess von wissenschaftlichen Arbeiten spielerisch darzustellen.

\begin{center}\rule{0.5\linewidth}{0.5pt}\end{center}

\textbf{Abstract}: No researcher can avoid publishing. Nowadays, it is
almost impossible to avoid digital publication services. Not many people
are aware that personal data can be tracked by academic publishers
without the knowledge of the user and, in the worst case, misused. One
of the missions of libraries is to support researchers and foster
information literacy. Awareness of data tracking in scholarly publishing
houses falls within this scope. This can, for instance, be communicated
through workshops. The question here is, how scientists, especially
early career researchers in natural sciences, can be educated about data
tracking by scholarly publishing houses, along with its negative effects
and ways to defend against data tracking.

In our workshop concept, in addition to short presentations, we have
developed a game as a simulation-based task with role-playing game-like
features based on a discovery approach. This is used to play through a
reality-based publication process. Depending on the decisions made,
personal data will be tracked.

This concept was presented and implemented in the pre-ISI workshop (ISI
= International Symposium on Information Science) on March 17th, 2025,
where the prototype card game was tested, and feedback was obtained. The
card game will be released for further use (and adaptations) as an open
educational resource. Possible adaptations include targeting the
workshop towards students and thus presenting the publication process or
the writing process of academic papers to them in a playful way.
