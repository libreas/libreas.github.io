\documentclass[a4paper,
fontsize=11pt,
%headings=small,
oneside,
numbers=noperiodatend,
parskip=half-,
bibliography=totoc,
final
]{scrartcl}

\usepackage[babel]{csquotes}
\usepackage{synttree}
\usepackage{graphicx}
\setkeys{Gin}{width=.4\textwidth} %default pics size

\graphicspath{{./plots/}}
\usepackage[ngerman]{babel}
\usepackage[T1]{fontenc}
%\usepackage{amsmath}
\usepackage[utf8x]{inputenc}
\usepackage [hyphens]{url}
\usepackage{booktabs} 
\usepackage[left=2.4cm,right=2.4cm,top=2.3cm,bottom=2cm,includeheadfoot]{geometry}
\usepackage[labelformat=empty]{caption} % option 'labelformat=empty]' to surpress adding "Abbildung 1:" or "Figure 1" before each caption / use parameter '\captionsetup{labelformat=empty}' instead to change this for just one caption
\usepackage{eurosym}
\usepackage{multirow}
\usepackage[ngerman]{varioref}
\setcapindent{1em}
\renewcommand{\labelitemi}{--}
\usepackage{paralist}
\usepackage{pdfpages}
\usepackage{lscape}
\usepackage{float}
\usepackage{acronym}
\usepackage{eurosym}
\usepackage{longtable,lscape}
\usepackage{mathpazo}
\usepackage[normalem]{ulem} %emphasize weiterhin kursiv
\usepackage[flushmargin,ragged]{footmisc} % left align footnote
\usepackage{ccicons} 
\setcapindent{0pt} % no indentation in captions
\usepackage{xurl} % Breaks URLs

%%%% fancy LIBREAS URL color 
\usepackage{xcolor}
\definecolor{libreas}{RGB}{112,0,0}

\usepackage{listings}

\urlstyle{same}  % don't use monospace font for urls

\usepackage[fleqn]{amsmath}

%adjust fontsize for part

\usepackage{sectsty}
\partfont{\large}

%Das BibTeX-Zeichen mit \BibTeX setzen:
\def\symbol#1{\char #1\relax}
\def\bsl{{\tt\symbol{'134}}}
\def\BibTeX{{\rm B\kern-.05em{\sc i\kern-.025em b}\kern-.08em
    T\kern-.1667em\lower.7ex\hbox{E}\kern-.125emX}}

\usepackage{fancyhdr}
\fancyhf{}
\pagestyle{fancyplain}
\fancyhead[R]{\thepage}

% make sure bookmarks are created eventough sections are not numbered!
% uncommend if sections are numbered (bookmarks created by default)
\makeatletter
\renewcommand\@seccntformat[1]{}
\makeatother

% typo setup
\clubpenalty = 10000
\widowpenalty = 10000
\displaywidowpenalty = 10000

\usepackage{hyperxmp}
\usepackage[colorlinks, linkcolor=black,citecolor=black, urlcolor=libreas,
breaklinks= true,bookmarks=true,bookmarksopen=true]{hyperref}
\usepackage{breakurl}

%meta
%meta

\fancyhead[L]{Ulrike Wuttke\\ %author
LIBREAS. Library Ideas, 47 (2025). % journal, issue, volume.
\href{https://doi.org/10.18452/x}{\color{black}https://doi.org/10.18452/x}
{}} % doi 
\fancyhead[R]{\thepage} %page number
\fancyfoot[L] {\ccLogo \ccAttribution\ \href{https://creativecommons.org/licenses/by/4.0/}{\color{black}Creative Commons BY 4.0}}  %licence
\fancyfoot[R] {ISSN: 1860-7950}

\title{\LARGE{\enquote{Das ist ja nur die Spitze des Eisbergs!}}}
\subtitle{Bericht vom Workshop zum Datentracking in der Wissenschaft im Vorfeld der ISI 2025, 17.03.2025 (TU Chemnitz)}% title
\author{Ulrike Wuttke} % author

\setcounter{page}{1}

\hypersetup{%
      pdftitle={"Das ist ja nur die Spitze des Eisbergs!" Bericht vom Workshop zum Datentracking in der Wissenschaft im Vorfeld der ISI 2025, 17.03.2025 (TU Chemnitz)},
      pdfauthor={Ulrike Wuttke},
      pdfsubject={LIBREAS. Library Ideas, 47 (2025)},
      pdfkeywords={Datentracking, ISI, Wissenschaftliches Publizieren, Wissenschaftstracking},
      pdflicenseurl={https://creativecommons.org/licenses/by/4.0/},
      pdfcopyright={CC BY 4.0 International},
      pdfcontacturl={http://libreas.eu},
      pdfurl={},
      pdfdoi={},
      pdflang={de},
      pdfmetalang={de}
     }



\date{}
\begin{document}

\maketitle
\thispagestyle{fancyplain} 

%abstracts

%body
\section{Einleitung}\label{einleitung}

In den letzten Jahren hat sich der Fokus der internationalen,
kommerziellen Verlagslandschaft von klassischen
Publikationsdienstleistungen hin zu Data-Analytics-

Dienstleistungen verschoben. Nicht zuletzt seitdem sich der
DFG-Ausschuss für Wissenschaftliche Bibliotheken und Informationssysteme
(AWBI) 2021 in einer Stellungnahme\footnote{Ausschuss für
  Wissenschaftliche Bibliotheken und Informationssysteme (AWBI) (28.
  Oktober 2021) Datentracking in der Wissenschaft: Aggregation und
  Verwendung bzw. Verkauf von Nutzungsdaten durch Wissenschaftsverlage.
  Ein Informationspapier des Ausschusses für Wissenschaftliche
  Bibliotheken und Informationssysteme der Deutschen
  Forschungsgemeinschaft. Deutsche Forschungsgemeinschaft (DFG).
  \url{https://doi.org/10.5281/zenodo.5900759} (zugegriffen: 8. April
  2022).} bezüglich der Problematik dieser Entwicklung, die Risiken für
die digitale Souveränität der Wissenschaft und den Datenschutz mit sich
bringt, positioniert hat, nimmt das Thema im deutschsprachigen Raum und
darüber hinaus an Fahrt auf. In der AWBI-Stellungnahme wird u.~a.
kritisiert:

 \glqq Im Einzelnen kann unreguliertes bzw. unerkanntes Datentracking
\begin{itemize}
\item eine Verletzung der Wissenschaftsfreiheit und der Freiheit von Forschung und Lehre bedeuten; 
\item eine Verletzung des Rechts auf den Schutz der eigenen Daten darstellen; 
\item eine potenzielle Gefährdung von Wissenschaftlerinnen und Wissenschaftlern darstellen, da die Daten auch
ausländischen Regierungen und autoritären Regimes zugänglich werden können; 
\item einen Eingriff ins Wettbewerbsrecht darstellen, da neue Teilnehmer kaum eine Chance auf einen Markteintritt haben; 
\item eine Wertminderung öffentlicher Forschungsinvestitionen begünstigen, da im Rahmen von Wirtschaftsspionage wissenschaftliche Aktivitätsdaten von kommerziellen Forschungskonkurrenten erhoben oder ihnen gegen Bezahlung zugänglich gemacht werden können.\grqq \footnote{AWBI 2021, S. 8.}
\end{itemize}

Eines der Verlags-Oligopole, das wegen unlauterer Tracking-Praktiken im
Fokus steht, ist Elsevier: \enquote{Being both the main suppliers of
research \emph{and} the main suppliers of data analytics about research
gives Elsevier outsized power to shape science. The research oligopoly
controls both the input (funding) and the output (research findings) of
the knowledge enterprise.}\footnote{Lamdan, S. (2023). \emph{Data
  cartels: the companies that control and monopolize our information}.
  Stanford, California: Stanford University Press, S. 63.} Das führt
nicht nur zu wissenschaftsethischen Problemen sondern -- verbunden mit
der umfangreichen Sammlung von digitalen Nutzungsspuren (\emph{data
tracking}) durch die Verlage und andere Akteure sowie ihrem Verkauf
(\emph{data brokering}) -- zu handfesten Datenschutzproblemen. Diese
sind in grundsätzliche Fragen bezüglich der Digitalen Souveränität, der
Wahrung der Wissenschaftsfreiheit im Kontext demokratischer Systeme und
Staaten oder der Digitalen Teilhabe eingebettet, um nur einige zu
nennen. Lamdan fordert daher: \enquote{Scholarly journals shouldn't be
mixed up in data brokering and metrics businesses. Academic research is
meant to be a public good, not a data-collection tool for private data
broker companies.}\footnote{Lamdan (2023), S. 71.}

Während die wissenschaftlichen Großverlage versuchen, ihre
Vormachtstellung im Bereich Datenanalyse auszubauen, regt sich zunehmend
Widerstand. Verschiedene nationale und internationale Initiativen wie
SPARC und Stop Tracking Science haben sich des Themas angenommen und
motivieren die eingehende Befassung mit diesem Problemfeld. Auch in der
deutschsprachigen Bibliotheks-Community ist das Thema angekommen. Es
wurde im Kontext verschiedener fachspezifischer Tagungen wie vBib (2021,
2022, 2024) oder BiblioCon (2023, 2024) sowie in
Fachpublikationen\footnote{Lauer, G. (2022): Datentracking in den
  Wissenschaften: Wissenschaftsorganisationen und die bizarre Asymmetrie
  im wissenschaftlichen Publikationssystem. In: O-Bib. Das Offene
  Bibliotheksjournal, 9:1, \url{https://doi.org/10.5282/o-bib/5796}}
aufgegriffen.

Bei einem unter Beteiligung der Autorin durchgeführten Hands- on
Lab\footnote{Seltmann, M., Siems, R., Steyer, T., \& Wuttke, U. (2023).
  „Alles DSGVO-konform! -- Wirklich?\enquote{: Bericht vom Hands-On
  Lab}Datentracking im wissenschaftlichen Workflow''. O-Bib. Das Offene
  Bibliotheksjournal, 10:4, \url{https://doi.org/10.5282/o-bib/5958}}
während der BiblioCon 2023 wurde deutlich, dass eine der
Herausforderungen die Vermittlung des Themas in die wissenschaftlichen
und bibliothekarischen Communities ist. Es gilt einerseits mittels
Methoden der Computerforensik den aktuellen Stand des Datentrackings und
allgemeine Awareness für das Thema zu vermitteln und andererseits
Aktionsräume (im Sinne von Empowerment, wie Schutzmechanismen,
vertragliche Regelungen et cetera) für verschiedene Zielgruppen zu
eröffnen.\footnote{Siehe Altschaffel, R., Wuttke, U., Steyer, T.,
  Dittmann, J., \& Kiltz, S. (2025). Der ungewollte Blick über die
  Schulter: Datentracking im wissenschaftlichen Bereich. 18.
  Internationales Symposium für Informationswissenschaft (ISI 2025),
  Chemnitz, Deutschland. Zenodo.
  \url{https://doi.org/10.5281/zenodo.14925626}} Im Sommersemester 2024
hat die Autorin mit Timo Steyer daher mit Bachelorstudierenden der
Bibliothekswissenschaft an der Fachhochschule Potsdam das Thema
Wissenschaftstracking in einem Seminar zur Informationsdidaktik
bearbeitet und die Erstellung von Lehr-Lern-Konzepten
betreut.\footnote{Steyer, T. und Wuttke, U. (2024): Data Tracking in the
  Classroom?! Wissenschaftstracking als Seminarthema. vBIB 2024,
  4.-5-12-2024, online.
  \url{https://www.vbib.net/vbib24-programm/programmdetail/vbib24-2-1/}}

\section{\texorpdfstring{Der Pre-ISI2025-Workshop
\enquote{Datenethik --
Datentracking}}{Der Pre-ISI2025-Workshop ``Datenethik -- Datentracking''}}\label{der-pre-isi2025-workshop-datenethik-datentracking}

Vor dem Hintergrund dieser allgemeinen Vorbemerkungen und des Seminars
aus dem Sommersemester 2024 fand 2025 ein halbtägiger Workshop zum Thema
\enquote{Datenethik -- Datentracking} an der TU Chemnitz
statt.\footnote{Für weitere Impressionen siehe Wuttke, U. (25.04.2025),
  Movetia-Workshop zum Datentracking im Vorfeld der ISI 2025.
  Blogbeitrag.
  \url{https://ulrikewuttke.wordpress.com/2025/04/25/movetia-workshop-zum-datentracking-im-vorfeld-der-isi-2025/}}
Der Workshop wurde im Rahmen des Movetia-Projekts \enquote{Datenkompetenzen
vermitteln an Hochschulen im D-A-CH Raum}\footnote{Fördernummer
  2024-1-CH01-IP-0018} federführend von Vera Husfeldt (FH Graubünden,
Schweiz) und ihrem Team gemeinsam mit den Movetia-Projektpartnern Stefan
Dreisiebner (FH Kärnten, Österreich) und der Autorin (Ulrike Wuttke, FH
Potsdam, Deutschland) ausgerichtet. Der Fokus des an der Fachhochschule Graubünden bei Vera Husfeldt als Lead angesiedelten Projekts ist die
Förderung von Datenkompetenzen von Studierenden, insbesondere durch die
Organisation von Workshops, Schulungen oder Seminaren und die
Entwicklung innovativer Lehrmethoden und -inhalte zur Förderung von
Datenkompetenzen, der Austausch von Forschungsergebnissen und bewährten
Praktiken sowie die Stärkung der Zusammenarbeit zwischen der
Fachhochschule Graubünden und den Fachhochschulen Potsdam und Kärnten.

Im Mittelpunkt des Workshops standen daher die Auseinandersetzung und
Diskussion möglicher Weiterentwicklungsansätze. Dieses wurde im Vorfeld
durch die Dozierenden mit den Projektpartnern ausgewählt und die
konkrete Umsetzung im Workshopkontext im Vorfeld mit der betreffenden
Projektgruppe besprochen (siehe Beitrag in dieser Ausgabe von Ha Thao
Suong Vu, Ioanna Danai Katsougiannoupoulou, Nadja Hartwich \url{https://libreas.eu/ausgabe47/digital-footprint}). Neben der
Projektgruppe nahmen weitere Studierende der Fachhochschule Potsdam im
Rahmen einer studentischen Exkursion zur des Internationalen Symposiums
für Informationswissenschaft (ISI) 2025 am Workshop teil, sowie dank der
Förderung von Movetia als weitere Expert*innen Timo Steyer (Technische
Universität Braunschweig, Universitätsbibliothek) und Robert Altschaffel
(Otto-von-Guericke-Universität Magdeburg Universität Magdeburg) sowie
Maia Lenherr und Urban Kalbermatter aus dem Team von Vera Husfeldt.

Nach einer kurzen Einführung zum Hintergrund des Movetia-Projekts durch
Vera Husfeldt und einer kurzen Vorstellungsrunde, bei der die sehr
diversen Vorkenntnisse der Gruppe zum Thema deutlich wurden, gab die
Autorin gemeinsam mit Timo Steyer und Robert Altschaffel eine
thematische Einleitung zum Seminar-Kontext, zu Datentracking im
Allgemeinen und zu Methoden der Computerforensik mit dem Ziel,
First-Party- und Third-Party-Tracking auf Verlagswebseiten oder anderen
Angeboten aufzuspüren. Nach einer kurzen Pause stellte die Projektgruppe
eine für den Workshop entworfene Spielidee sowie den dazugehörigen
Prototyp vor und leitete eine Spielrunde an, um Feedback zu erhalten.
Als Lehrende war die Autorin begeistert zu sehen, wie souverän die
Studierenden die Workshopleitung übernahmen.

Die Grundidee des Spiels mit dem Titel \enquote{Choose your own
adventure -- Digital Footprint} besteht darin, dass die Spielenden als
Gruppe für eine Persona eine Publikation planen und veröffentlichen und
das während bestimmter Entscheidungsschritte immer wieder Daten
\enquote{getrackt} werden -- oder auch nicht. Im Spiel müssen sie immer
wieder diskutieren, welche Entscheidungen sie fällen, wie, um ein paar
Beispiele zu nennen \enquote{online recherchieren}, \enquote{in die
Bibliothek fahren}, \enquote{Googlen}, \enquote{den Betreuenden fragen}.
Es gibt am Ende auch keine Gewinner*innen oder Verlierer*innen im
klassischen Sinne, sondern es geht darum, dass sich die Spielenden über
Publikations- und Forschungspraktiken und digitale Nutzungsspuren
unterhalten. Das geschah dann auch während des gut eine dreiviertel
Stunde langen Testspiels.

Im Anschluss an die Spielrunde entspann sich unter den Teilnehmenden
eine angeregte Diskussion. Es wurde nicht nur deutlich, dass durch die
Gruppe eine Menge Potenzial für die Weiterentwicklung des Spiels gesehen
wurde. Es kam auch ein grundsätzlicher Punkt zur Sprache, in dessen
Kontext auch das Titelzitat gefallen ist, nämlich der Unterschied
zwischen Datentracking und Datentracking in der Wissenschaft. Die
Problematik ist die gleiche, nämlich das unerlaubte beziehungsweise
unbewusste Tracking von Daten (und die problematische Verarbeitung oder
der Verkauf). Datentracking in der Wissenschaft ist nur die Spitze des
Eisbergs, denn (unerlaubtes) Datentracking findet erstaunlich häufig
statt. Aber während es Vielen vielleicht bewusst ist, dass bei
kommerziellen Tools Datentracking stattfindet (zu denken ist an die
omnipräsenten Cookie-Banner), ist vielleicht weniger bekannt, dass dies
auch massiv auf den Webseiten einiger Verlage stattfindet oder bei
\enquote{Access through your institution}-Zugängen. Dazu kommt, dass
Wissenschaftler*innen und Studierende keine Wahl haben (oder aber eben
doch?), weil sie Literatur brauchen -- im Gegensatz dazu haben sie eine
Wahl, ob sie beispielsweise Social Media nutzen. Als wichtig wurde auch
die Frage der institutionellen Verantwortung empfunden, zum Beispiel
dass Bibliotheken eine (rechtliche) Verantwortung haben, wenn
Datentracking stattfindet, oder der Wissenschaftler*innen, wo sie
publizieren und was sie zum Lesen empfehlen (Lektürelisten). Und das ist
nicht nur eine technische, sondern auch eine politische Diskussion.

\section{Fazit und Ausblick}\label{fazit-und-ausblick}

Der internationale Austausch zwischen den Studierenden und den
Teilnehmenden aus der Schweiz, Deutschland und Österreich zum Thema
Datentracking wie das anschließende gemeinsame Abendessen, das den
intensiven Dialog abrundete und weitere Gelegenheit für das Schmieden
von Plänen für zukünftige Kooperationen und Projekte gab, war ebenso
bereichernd wie perspektiverweiternd. Der Workshop legte einen wichtigen
Grundstein für weitere gemeinsame Lehr- und Lernaktivitäten der
Movetia-Partner und die intensivere Verschränkung mit den Aktivitäten
des Magdeburger Teams um Prof.~Dr.-Ing. Jana Dittmann (u.~a.
Arbeitsgruppe Multimedia and Security), zu dem u.~a. Robert Altschaffel
gehört.

\section{Bibliografie}\label{bibliografie}

Altschaffel, R., Wuttke, U., Steyer, T., Dittmann, J., \& Kiltz, S.
(2025): Der ungewollte Blick über die Schulter: Datentracking im
wissenschaftlichen Bereich. 18. Internationales Symposium für
Informationswissenschaft (ISI 2025), Chemnitz, Deutschland. Zenodo.
\url{https://doi.org/10.5281/zenodo.14925626}

Ausschuss für Wissenschaftliche Bibliotheken und Informationssysteme
(AWBI). 2021. Datentracking in der Wissenschaft: Aggregation und
Verwendung bzw. Verkauf von Nutzungsdaten durch Wissenschaftsverlage.
Ein Informationspapier des Ausschusses für Wissenschaftliche
Bibliotheken und Informationssysteme der Deutschen
Forschungsgemeinschaft, 28. Oktober 2021. Deutsche
Forschungsgemeinschaft (DFG).
\url{https://doi.org/10.5281/zenodo.5900759} (zugegriffen: 8. April
2022).

Lamdan, S. (2023): Data cartels: the companies that control and
monopolize our information. Stanford, California: Stanford University
Press.

Lauer, G. (2022): Datentracking in den Wissenschaften:
Wissenschaftsorganisationen und die bizarre Asymmetrie im
wissenschaftlichen Publikationssystem. In: O-Bib. Das Offene
Bibliotheksjournal, 9:1, \url{https://doi.org/10.5282/o-bib/5796}

Steyer, T. und Wuttke, U. (2024): Data Tracking in the Classroom?!
Wissenschaftstracking als Seminarthema. vBIB 2024, 4.-5-12-2024, online.
\url{https://www.vbib.net/vbib24-programm/programmdetail/vbib24-2-1/}

Wuttke, U. (25.04.2025), Movetia-Workshop zum Datentracking im Vorfeld
der ISI 2025. Blogbeitrag.
\url{https://ulrikewuttke.wordpress.com/2025/04/25/movetia-workshop-zum-datentracking-im-vorfeld-der-isi-2025/}

%autor
\begin{center}\rule{0.5\linewidth}{0.5pt}\end{center}

\textbf{Ulrike Wuttke} (\url{https://orcid.org/0000-0002-8217-4025}) ist
Professorin für Bibliothekswissenschaft, Strategien, Serviceentwicklung
und Wissenschaftskommunikation an der Fachhochschule Potsdam. Seit April
2025 hat sie an der FH Potsdam eine Transferprofessur zum Thema
Datentracking.

\end{document}