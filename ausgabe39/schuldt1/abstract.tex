Computer wurden in den Wissenschaftlichen Bibliotheken des DACH-Raumes
in den 1960er Jahren eingeführt. Der Text vollzieht dies nach und zeigt
unter anderem, dass die Grundprobleme, die am Ende mit den
Rechenmaschinen gelöst werden sollten, zuvor auch anders angegangen
wurden. Es ging um die Beherrschung einer wachsenden Anzahl von
Literatur. Auffällig an der Geschichte ist, dass die Entwicklung sich
relativ schnell vollzogen, dass sie im DACH-Raum relativ ähnlich
verliefen und das es nicht per se um die Einführung von Technik, sondern
um das Lösen von Problemen ging.
