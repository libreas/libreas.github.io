\begin{center}\rule{0.5\linewidth}{\linethickness}\end{center}

\textbf{Beate Rajski} ist Bibliothekarin und arbeitet seit 30(!) Jahren
an der TU Hamburg. Sie ist Fachreferentin für Forschungsdaten und
Abteilungsleiterin der Digitalen Dienste der Universitätsbibliothek.
2018 bis 2020 leitet sie das Projekt Forschungsdatenmanagement für
Hamburg Open Science, ein Programm, in dem DSpace sowohl für
Open-Access- und Forschungsdatenrepositorien als auch für FIS-Funktionen
zum Einsatz kommt. Die Idee des DSpace-Konsortiums hat sie von Anfang an
unterstützt und ist seit der Gründung Sprecherin und Mitglied der DSpace
Leadership Group.

\textbf{Pascal Becker} hatte die Idee, ein nationales DSpace-Konsortiums
zu gründen und hat diese Idee seit 2016 verfolgt. Er hat Informatik
studiert und arbeitet seit über 13 Jahren in bibliothekarischen
Umgebungen und an Open-Access-Repositorien, zuletzt an der TU Berlin. Er
ist Geschäftsführer des von ihm gegründeten DSpace Service Provider The
Library Code GmbH (\url{https://www.the-library-code.de/}), zweiter
Sprecher des DSpace-Konsortiums Deutschland, organisiert seit 2014
jährlich das DSpace Anwendertreffen, ist Mitglied in der DSpace
Committer, DSpace Leadership, DSpace Steering Group und in der DataCite
Services and Technology Steering Group.
