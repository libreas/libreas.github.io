\begin{center}\rule{0.5\linewidth}{0.5pt}\end{center}

Nach dem Abitur studierte \textbf{Carmen Krause} im Magisterstudiengang
Neuere und Neueste Geschichte sowie Neuere deutsche Literatur an der
Humboldt-Universität zu Berlin. Dort konnte sie als studentische
Beschäftigte der Zweigbibliothek Philosophie bereits während des
Studiums erste Berufserfahrungen im Bibliotheksbereich sammeln. Nach dem
Studium folgten Beschäftigungen in Bibliotheken von Unternehmen und
wissenschaftlichen Einrichtungen. Aus diesem Grund beschloss sie, ein
Studium im Bachelorstudiengang Bibliotheksmanagement an der
Fachhochschule Potsdam aufzunehmen, welches sie 2018 mit einer Arbeit
über die Potenziale des Internets der Dinge für Bibliotheken abschloss.
Für diese Arbeit wurde sie mit dem b.i.t.online Innovationspreis 2019
sowie mit dem Best Presentation Award des 9. Studierenden-Workshops für
informationswissenschaftliche Forschung ausgezeichnet. Derzeit studiert
sie im Masterstudiengang Informationswissenschaften an der
Fachhochschule Potsdam.
