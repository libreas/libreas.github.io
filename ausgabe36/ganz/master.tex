\documentclass[a4paper,
fontsize=11pt,
%headings=small,
oneside,
numbers=noperiodatend,
parskip=half-,
bibliography=totoc,
final
]{scrartcl}

\usepackage[babel]{csquotes}
\usepackage{synttree}
\usepackage{graphicx}
\setkeys{Gin}{width=.4\textwidth} %default pics size

\graphicspath{{./plots/}}
\usepackage[ngerman]{babel}
\usepackage[T1]{fontenc}
%\usepackage{amsmath}
\usepackage[utf8x]{inputenc}
\usepackage [hyphens]{url}
\usepackage{booktabs} 
\usepackage[left=2.4cm,right=2.4cm,top=2.3cm,bottom=2cm,includeheadfoot]{geometry}
\usepackage{eurosym}
\usepackage{multirow}
\usepackage[ngerman]{varioref}
\setcapindent{1em}
\renewcommand{\labelitemi}{--}
\usepackage{paralist}
\usepackage{pdfpages}
\usepackage{lscape}
\usepackage{float}
\usepackage{acronym}
\usepackage{eurosym}
\usepackage{longtable,lscape}
\usepackage{mathpazo}
\usepackage[normalem]{ulem} %emphasize weiterhin kursiv
\usepackage[flushmargin,ragged]{footmisc} % left align footnote
\usepackage{ccicons} 
\setcapindent{0pt} % no indentation in captions

%%%% fancy LIBREAS URL color 
\usepackage{xcolor}
\definecolor{libreas}{RGB}{112,0,0}

\usepackage{listings}

\urlstyle{same}  % don't use monospace font for urls

\usepackage[fleqn]{amsmath}

%adjust fontsize for part

\usepackage{sectsty}
\partfont{\large}

%Das BibTeX-Zeichen mit \BibTeX setzen:
\def\symbol#1{\char #1\relax}
\def\bsl{{\tt\symbol{'134}}}
\def\BibTeX{{\rm B\kern-.05em{\sc i\kern-.025em b}\kern-.08em
    T\kern-.1667em\lower.7ex\hbox{E}\kern-.125emX}}

\usepackage{fancyhdr}
\fancyhf{}
\pagestyle{fancyplain}
\fancyhead[R]{\thepage}

% make sure bookmarks are created eventough sections are not numbered!
% uncommend if sections are numbered (bookmarks created by default)
\makeatletter
\renewcommand\@seccntformat[1]{}
\makeatother

% typo setup
\clubpenalty = 10000
\widowpenalty = 10000
\displaywidowpenalty = 10000

\usepackage{hyperxmp}
\usepackage[colorlinks, linkcolor=black,citecolor=black, urlcolor=libreas,
breaklinks= true,bookmarks=true,bookmarksopen=true]{hyperref}
\usepackage{breakurl}

%meta

%meta

\fancyhead[L]{K. Ganz, M. Wrzesinski, M. Rauchecker \\ %author
LIBREAS. Library Ideas, 36 (2019). % journal, issue, volume.
\href{http://nbn-resolving.de/}
{}} % urn 
% recommended use
%\href{http://nbn-resolving.de/}{\color{black}{urn:nbn:de...}}
\fancyhead[R]{\thepage} %page number
\fancyfoot[L] {\ccLogo \ccAttribution\ \href{https://creativecommons.org/licenses/by/4.0/}{\color{black}Creative Commons BY 4.0}}  %licence
\fancyfoot[R] {ISSN: 1860-7950}

\title{\LARGE{Nachhaltige Qualitätssicherung und Finanzierung von non-APC, scholar-led Open-Access-Journalen}} % title
\author{Kathrin Ganz, Marcel Wrzesinski, Markus Rauchecker} % author

\setcounter{page}{1}

\hypersetup{%
      pdftitle={Nachhaltige Qualitätssicherung und Finanzierung von non-APC, scholar-led Open-Access-Journalen},
      pdfauthor={Kathrin Ganz, Marcel Wrzesinski, Markus Rauchecker},
      pdfcopyright={CC BY 4.0 International},
      pdfsubject={LIBREAS. Library Ideas, 36 (2019).},
      pdfkeywords={Bibliothek, Open Access, Nachhaltigkeit, Zeitschrift, Finanzierungsstruktur, platin Open Access},
      pdflicenseurl={https://creativecommons.org/licenses/by/4.0/},
      pdfcontacturl={http://libreas.eu},
      baseurl={http://libreas.eu},
      pdflang={de},
      pdfmetalang={de}
     }



\date{}
\begin{document}

\maketitle
\thispagestyle{fancyplain} 

%abstracts
\begin{abstract}
\noindent
\textbf{Kurzfassung}: In inter- und transdisziplinären Feldern sowie
kleinen Fächern der Sozial- und Geisteswissenschaften hat sich ein
spezifisches Segment von Open-Ac\-cess-Zeitschriften entwickelt: Sie
arbeiten scholar-led, verlagsunabhängig und ohne Artikelgebühren (Platin
Open Access). Diese Journale füllen Lücken im Publikationsmarkt und
wirken als Multiplikatorinnen für die Open-Access-Idee. Am Beispiel von
fünf Zeitschriften werden die strukturellen Herausforderungen
herausgearbeitet, vor denen Zeitschriften dieses Segments hinsichtlich
der nachhaltigen Finanzierung und Qualitätssicherung stehen, und
Empfehlungen erarbeitet, um qualitativ hochwertige Publikationsorgane
auf Basis offener Publikationsinfrastrukturen zu ermöglichen.

\begin{center}\rule{0.5\linewidth}{\linethickness}\end{center}

\noindent\textbf{Abstract}: A specific segment of Open Access journals has been
developed in inter- and transdisciplinary fields and small subject areas
in the social sciences and humanities: they work scholar-led,
independently of publishers and without article processing charges (that
is: platinum open access). Those journals fill gaps in the publication
market and act as multipliers for the Open Access idea. Using the
example of five journals, the structural challenges facing journals in
this segment in terms of sustainable financing and quality assurance are
highlighted, and recommendations are developed to enable high-quality
modes of publication based upon open publication infrastructures.
\end{abstract}

%body
\hypertarget{einleitung}{%
\section{Einleitung}\label{einleitung}}

Die Open-Access-Transformation verändert Rezeptions- und
Publikationsgewohnheiten wissenschaftlicher Literatur auf vielfältige
Weise. Für Leser*innen bedeutet Open Access (OA) zunächst, dass
wissenschaftliche Ergebnisse, die meist durch öffentliche Gelder
ermöglicht werden, als Publikationen dauerhaft frei verfügbar sind.
Autor*innen stehen vor der Herausforderung, sich in dieser verändernden
Publikationslandschaft zu orientieren. Wenn sie sich für freie
Publikationsformen entscheiden, die auch die Nachnutzung ihrer Werke
ermöglichen, sind sie zunehmend mit für Open-Access-Publikationen
typischen Artikelgebühren (Article Processing Charges (APC))
konfrontiert.

Für die Forschung insgesamt eröffnet das elektronische Publizieren mit
Open Access neue Möglichkeiten, eigene Zeitschriften und
Publikationsreihen zu gründen und so Lücken im Publikationsmarkt zu
füllen. Viele Wissenschaftler*innen arbeiten dabei bewusst nicht mit
Wissenschaftsverlagen oder APCs, sondern nutzen Infrastrukturen, die
Bibliotheken und die Open-Access-Community im Bereich des elektronischen
Publizierens anbieten: Redaktionssoftware wie Open Journal Systems,
DOI-Registrierung, Unterstützung bei der Einbindung der Angebote in
Suchmaschinen und Langzeitarchivierung.

Neue, unabhängige Open-Access-Zeitschriften ins Leben zu rufen, ist
besonders für Wissenschaftler*innen aus Forschungsbereichen attraktiv,
in denen es noch keine etablierten Publikationsorgane gibt. Dazu gehören
neue Forschungsfelder etwa im Bereich der Internetforschung und
Digitalisierung, interdisziplinäre Forschungsfelder wie die
Geschlechterforschung und Regionalstudien sowie kleine Fächer\footnote{\url{https://www.kleinefaecher.de/}.}.
Die Gründungsredaktionen dieser Journale engagieren sich für die
Entwicklung des jeweiligen Forschungsfeldes und sind zugleich
Advokat*innen und Multiplikator*innen für die Open-Access-Idee. Vor
diesem Hintergrund definieren jene Redaktionskollektive die Bedingungen
und Richtlinien des wissenschaftlichen Publizierens von Anfang an unter
den Bedingungen von Platin Open Access\footnote{Hierunter ist eben der
  Verzicht auf jedwede Artikel- oder Publikationsgebühren zu verstehen.
  Es gibt hier diverse neue Farbschemata, die die klassische
  Grün/Gold-Klassifizierung im Open-Access-Kontext erweitern.} und
ausgehend von den Bedürfnissen ihres jeweiligen Feldes. Auf diese Weise
verkörpern sie die Idee des \emph{scholar-led publishing} in besonderem
Maße.

In dem vorliegenden Beitrag werden die Funktionen dieser Journale in der
Publikationslandschaft analysiert und es wird nach den spezifischen
Herausforderungen gefragt, die hinsichtlich einer nachhaltigen
Finanzierung dieser Zeitschriften bei gleichzeitig hohen
Qualitätsstandards für den Inhalt bestehen. Anhand von fünf Beispielen,
die im Rahmen eines Workshops auf den Open Access Tagen 2019 in
Hannover\footnote{Workshop 11: \enquote{Was darf Qualität kosten?
  Geschäftsmodelle für neue, nicht-APC-finanzierte
  Open-Access-Journals}, vergleiche
  \url{https://open-access.net/community/open-access-tage/open-access-tage-2019/programm/}.}
vorgestellt wurden, zeigen wir, dass die aktuell vorherrschenden
Finanzierungsmodelle für Open Access nicht geeignet sind, um die
nachhaltige Finanzierung und eine angemessene Qualitätssicherung dieser
Journale zu gewährleisten. Daraus folgt, dass wissenschaftliche
Institutionen (Universitäten, Forschungseinrichtungen und \hbox{-bibliotheken)}
zusammen mit Fachgesellschaften und Forschungsförderern eine aktivere
Rolle im Open-Access-Publishing einnehmen müssen, um auch jene
Publikationsangebote nachhaltig zu sichern, die auf Basis der zur
Verfügung gestellten Infrastrukturen in den letzten Jahren aufgebaut
worden sind.

\hypertarget{fuxfcnf-beispiele-verlagsunabhuxe4ngiger-scholar-led-open-access-journale}{%
\section{Fünf Beispiele verlagsunabhängiger, scholar-led
Open-Access-Journale}\label{fuxfcnf-beispiele-verlagsunabhuxe4ngiger-scholar-led-open-access-journale}}

Im Folgenden zeigen wir anhand von fünf Beispielen Funktionsweisen,
Workflows, Chancen und Herausforderungen von non-APC-, scholar-led
Open-Access-Journalen. Dabei geht es nicht nur um gemeinsame
strukturelle Merkmale, sondern auch um die spezifische Funktion, die sie
in den jeweiligen Fachcommunities erfüllen. Die anschließenden
Schlussfolgerungen sind entsprechend auf andere Zeitschriften dieses
Segments übertragbar.

Die fünf Zeitschriften -- CROLAR, Internet Policy Review, META,
On\_Culture und Open Gender Journal -- sind relativ junge Zeitschriften,
die seit ihrer Gründung im Open Access erscheinen und fachbegutachtete
Beiträge veröffentlichen. Seit 2012 veröffentlicht CROLAR\footnote{\url{https://www.crolar.org/}.}
(Critical Reviews on Latin American Research), angesiedelt am
Lateinamerika-Institut der Freien Universität Berlin, vor allem
Rezensionen und Review Articles zur Lateinamerikaforschung. Ebenfalls
2012 erschien die erste Ausgabe von Internet Policy Review\footnote{\url{https://policyreview.info/}.}
(IPR) am Alexander von Humboldt Institut für Internet und Gesellschaft,
einer Zeitschrift, die sich mit Internet-Regulierung und den Effekten
auf europäische Gesellschaften befasst. Middle East -- Topics and
Arguments\footnote{\url{https://meta-journal.net/}.} (META) wurde 2012
am Center for Near and Middle Eastern Studies (CNMS) der Universität
Marburg gegründet; die erste Ausgabe erschien 2013. Seit 2016 erscheint
am Gießener International Graduate Centre for the Study of Culture
(GCSC) die Zeitschrift On\_Culture\footnote{\url{https://www.on-culture.org/}.}
mit wissenschaftlichen Artikeln, Essays und anderen Formaten, die sich
aus interdisziplinärer Perspektive mit kulturwissenschaftlichen
Konzepten befassen. Das Open Gender Journal\footnote{\url{https://opengenderjournal.de/}.}
(OGJ) erscheint seit 2017 und veröffentlicht ausschließlich
Fachbeiträge. An OGJ sind verschiedene Zentren und Einrichtungen der
Geschlechterforschung sowie die Fachgesellschaft Geschlechterstudien
beteiligt.

Die Zeitschriften nutzen die Online-Publikationsplattformen nicht nur
für Open-Access-Ver\-öf\-fent\-lichungen, sondern auch, um die textbasierte
Form der traditionellen Zeitschriften aufzubrechen und damit neue
inhaltliche Wege zu gehen. CROLAR beschränkt sich bewusst nicht auf
Rezensionen wissenschaftlicher Publikationen, sondern veröffentlicht
auch Besprechungen zu anderen Formaten wie etwa Blogs, Webseiten, Filmen
oder Ausstellungen. Darüber hinaus werden Interviews mit Expert*innen
des jeweiligen thematischen Fokus geführt. Einen anderen Ansatz wählt
On\_Culture mit der Rubrik \_Perspectives\footnote{Rubrik \_Perspectives
  der Zeitschrift On\_Culture:
  \url{https://www.on-culture.org/journal/perspectives}.}, die
publizistische Räume für Formate jenseits des schriftlichen Beitrages
eröffnet. Im Rahmen der thematischen Ausgaben von On\_Culture erscheinen
regelmäßig Video-Clips, Interviews und visuelle Arbeiten, die den
Begriff des kulturwissenschaftlichen Forschungsbeitrags erweitern. In
ähnlicher Weise erweitert Internet Policy Review das Spektrum an
möglichen Formaten: Neben Artikeln und Editorials finden sich Essays,
News- und Meinungsbeiträge; mit den Open Abstracts\footnote{Rubrik Open
  Abstracts der Zeitschrift Internet Policy Review
  \url{https://policyreview.info/open-abstracts}.} schließlich sollen
Form und Ablauf der klassischen Fachbegutachtung hinterfragt werden.
Beim Open Gender Journal schließlich ist die Öffnung für neue Formate im
Kontext von Open Access damit verbunden, die Spielräume frei
lizenzierter Veröffentlichungen stärker auszunutzen als bisher üblich.
In Zusammenarbeit mit der Fachgesellschaft Geschlechterstudien werden in
OGJ neben freien Beiträgen auch Beiträge der Jahrestagungen der
Fachgesellschaft erstveröffentlicht. Aufgrund der fortlaufenden
Publikationsweise wird es so möglich, Tagungsbeiträge vergleichsweise
schnell und begutachtet zu publizieren. Die Beiträge werden anschließend
im Rahmen der Open Gender Collections in Form von Tagungsbänden
zusammengestellt.\footnote{So zum Beispiel den Tagungsband 2017 der
  Fachgesellschaft Geschlechterstudien: \enquote{Aktuelle
  Herausforderungen der Geschlechterforschung},
  \url{https://opengenderplatform.de/open-gender-collections/tagungsbande-fgg/tagungsband-2017}.}
All diese Bemühungen zeigen, wie produktiv das kritische Hinterfragen
etablierter Publikationsmechanismen und Formatkriterien ist; zumal vor
dem Hintergrund weitergehender Debatten zur \enquote{Bibliodiversität}
im Open Science Zusammenhang.\footnote{Vergleiche hierzu den Jussieu
  Call von 2017 (\url{https://jussieucall.org/jussieu-call/}, den wir
  inhaltlich voll unterstützen.}

Eine Gemeinsamkeit der fünf Zeitschriften ist, dass sie in inter- oder
transdisziplinären Kontexten beheimatet sind beziehungsweise sich
kleineren Fachzusammenhängen zugehörig fühlen: Regionalstudien (META und
CROLAR), Kulturwissenschaften (On\_Culture), Internetforschung (IPR) und
Geschlechterforschung (OGP). Aufgrund dieser Verankerung ist es
erklärtes Ziel und wissenspolitisches Anliegen, dass die Zeitschriften
Übersetzungsleistungen zwischen verschiedenen wissenschaftlichen
Disziplinen, regional segmentierten Wissenschaftscommunities sowie
zwischen Wissenschaft und Zivilgesellschaft erbringen. Gerade in den
Regionalstudien (CROLAR \& META) ist dies ein Element einer kooperativen
Haltung gegenüber den Akteuer*innen in der Region und damit einem post-
beziehungsweise dekolonialen Anspruch von Open Access
verpflichtet.\footnote{Vergleiche Piron (2018) zur unidirektionalen Idee
  des \enquote{freien} Wissenstransfers von \enquote{Nord} nach
  \enquote{Süd}; sowie Neylon (2019) zur Kritik an Qualität und
  Exzellenz im globalen Open-Access-Zusammenhang.} Jedoch sind
Wissenschaftler*innen, die in diesen Feldern aktiv sind, oft mit den
engen fachlichen Korsetts klassischer Fachzeitschriften konfrontiert.
Sie benötigen interdisziplinaritätsfreundliche und fachkundige
Publikationsorte, die beispielsweise über Kontakte zu geeigneten
Gutachter*innen und tragfähige Distributionsnetzwerke verfügen. Insofern
dienen die neuen OA-Journale nicht nur der Profilierung von
Forschungseinrichtungen oder -richtungen, sondern schließen auch Lücken
im Publikationsmarkt und intervenieren in die Wissensproduktion und ihre
Strukturen.

Die Kommunikation wissenschaftlicher Ergebnisse hinein in Praxisfelder
ist ein weiteres Anliegen vieler Redaktionen. So versteht sich Internet
Policy Review als Ort, an dem unabhängige Analysen der europäischen
Internetregulierung zu finden sind, die über die Wissenschaft hinaus
auch für Akteur*innen aus der Zivilgesellschaft, Medienvertreter*innen,
Unternehmer*innen und die Politik von hoher Relevanz sind.\footnote{\url{https://policyreview.info/about}.}
Zugleich sollen die zivilgesellschaftlichen Potentiale und Ideen
kollaborativ Einfluss finden, sowohl über den Einbezug spezifischer
Aspekte des Citizen Science als auch die Öffnung vielfach hermetischer
und langwieriger Prozesse der Qualitätssicherung.\footnote{Vergleiche
  hierzu die Ausführungen zum Open-Abstracts-Modell von Riechert/Dubois
  (2017).}

Auch kann die inklusive und öffnende Perspektive der
Open-Access-Bewegung Gestaltungsspielräume für Wissenschaftler*innen in
frühen Karrierestadien bieten. Aufgrund der verlagsunabhängigen
Strukturen der Zeitschriften nimmt die redaktionelle Arbeit viel Zeit
ein und wird zu einem wichtigen Bestandteil der wissenschaftlichen
Arbeit und Ausbildung einzelner Akteuer*innen, die Aufgaben aus den
Bereichen Redaktionsmanagement, Lektorat, Layout und Webadministration
übernehmen. Auf diese Weise werden die Zeitschriften vielfach zu
Qualifizierungsprojekten und bieten die Möglichkeit, sich innerhalb der
Open-Access-Bewegung zu profilieren.

Der hohen Qualität und Innovationskraft der Zeitschriften stehen meist
prekäre und nicht nachhaltige Finanzierungsmodelle gegenüber.
Institutionelle Anschubfinanzierung, zum Teil in Form von geringen
Eigenmitteln der Universität (CROLAR), Drittmitteln (META), Mitteln der
deutschen Exzellenzinitiative (On\_Culture), Eigenmitteln des Instituts
(Internet Policy Review) oder die Integration als Pilotjournal in ein
drittmittelgefördertes Forschungsprojekt (Open Gender Journal im
BMBF-Projekt Open Gender Platform) ermöglichte die Umsetzung der
initialen Idee und teilweise den weiteren Betrieb. Die Zeitschriften
stützen sich zudem auf einen großen Anteil ehrenamtlicher Mitarbeit. Ob
und wie diese Arbeit Teil der jeweiligen (wissenschaftlichen) Anstellung
sein kann, musste bei den oben genannten Beispielen individuell
verhandelt werden. Vielfach fallen diese redaktionellen Tätigkeiten
jedoch \enquote{extra} an, sind also unentgeltlich und verschärfen die
ohnehin meist prekäre Arbeitssituation von befristet Angestellten
zusätzlich.

Auch die Suche nach geeigneten Geschäfts- beziehungsweise
Finanzierungsmodellen liegt in den Händen der beteiligten
Wissenschaftler*innen.\footnote{Das laufende DFG-Projekt Innovatives
  Open Access im Bereich Small Science am Alexander von Humboldt
  Institut für Internet und Gesellschaft unterstützt Zeitschriften bei
  dieser Suche unter anderem mit einer skalierbaren Budget-Toolbox
  (Veröffentlichung: Herbst 2020), vergleiche
  \url{https://www.hiig.de/project/innovatives-open-access-im-bereich-small-science-innoaccess/}.}
Nach Ablaufen von verschiedenen Formen der Anschubfinanzierung aus
institutionellen oder Drittmitteln stehen die Redaktionen oft vor der
Frage, wie die nachhaltige Finanzierung der Zeitschrift unter den
Bedingungen von Open Access realisiert werden kann. Dabei leisten gerade
die oben genannten und ähnliche Zeitschriften einen wichtigen Service
für ihre jeweiligen Communities.

\hypertarget{geschuxe4ftsmodelle-fuxfcr-open-access-zeitschriften}{%
\section{Geschäftsmodelle für
Open-Access-Zeitschriften}\label{geschuxe4ftsmodelle-fuxfcr-open-access-zeitschriften}}

Die Diskussion über mögliche Finanzierungsmodelle von
Open-Access-Zeitschriften wird derzeit von einem Modell dominiert: APCs,
die im Zuge der Veröffentlichung von den Autor*innen bezahlt und meist
institutionell gegenfinanziert werden.\footnote{\enquote{Publish-and-Read}-Verträge,
  wie sie etwa im Rahmen von Project-DEAL ausgehandelt werden, stellen
  eine Sonderform des APC-Modell dar, insofern mit den Artikelgebühren
  zugleich Zugang zum gesamten Verlagsangebot für die teilnehmenden
  Institutionen erkauft wird.}

Dass das APC-Modell den Diskurs um Open-Access-Finanzierung derzeit
dominiert, hat zwei Gründe: Erstens liegt der Fokus darauf, \enquote{das
Subskriptionssystem mit seinen Barrieren zu überwinden} (Pampel 2019:
1). Davon sind vor allem die Zeitschriftenangebote der
wissenschaftlichen (Groß-)Verlage betroffen. Die Verlage haben ein
Interesse am APC-Modell, da es ermöglicht, dass Verlage weiterhin eine
zentrale Funktion im Publikationswesen einnehmen. Verlagsleistungen
werden weiterhin aus öffentlichen Mitteln finanziert, indem
Erwerbungsmittel in Publikationsfonds umgeschichtet werden. Zweitens
wird dadurch auch aus Sicht der Wissenschaft -- angefangen von
individuellen Forschenden über die Forschungsbibliotheken bis zu den
Förderinstitutionen -- die Komplexität der Open-Access-Transformation
reduziert. Das Umschichten von Finanzmitteln und das Aufstellen neuer
Workflows ist keine einfache Aufgabe, aber doch überschaubar im
Vergleich zu einer kompletten Neuorganisation des wissenschaftlichen
Publikationswesens. Auch die Bemühungen um Transparenz seitens der
OA-Beauftragten und Projekten wie OA-Monitoring sind nicht gering zu
schätzen. Dennoch geraten andere Finanzierungsmöglichkeiten, die sich
möglicherweise auf lange Sicht als besser herausstellen könnten, durch
die Favorisierung des organisatorisch vergleichsweise einfachen
APC-Modell aus dem Blickfeld.

Aktuelle Zahlen zeigen, dass APC \enquote{für natur- und
lebenswissenschaftliche, international sichtbare und in einschlägigen
bibliometrischen Datenbanken indexierte Open-Access-Zeitschriften das
dominierende Geschäftsmodell sind} (Schönfelder 2019) -- gemessen am
Publikationsaufkommen. In den Geistes- und Sozialwissenschaften liegt
der Fall anders: 78 Prozent der zwischen 2013 und 2018 erschienen
Open-Access-Artikel wurden in Journalen publiziert, die keine APCs
erheben. Bei den restlichen 22 Prozent war die durchschnittliche APC
zudem deutlich geringer als in anderen wissenschaftlichen Feldern
(Crawford 2019: 3). Auch der hohe Anteil der APC-freien Journale unter
den im DOAJ gelisteten Zeitschriften zeigt, dass das Modell nicht
vorherrschend ist: Derzeit erheben lediglich 26,7 Prozent der 13.769 im
DOAJ gelisteten Open-Access-Zeitschriften APCs (keine APC: 72,91
Prozent; Rest: keine Angaben, Stand 16.09.2019). Offenbar favorisieren
viele Open-Access-Redaktionen nach wie vor Geschäftsmodelle, die ohne
APC auskommen.

Auch hierfür gibt es mehrere Gründe: Zum einen kann -- vor allem bei
Journalen, die \emph{OA-born} sind -- sicherlich von einer
Pfadabhängigkeit gesprochen werden. Ein einmal eingeschlagener Weg wird
nicht so schnell verlassen, auch wenn sich die Alternative im Nachhinein
als wirtschaftlich tragfähiger erweisen sollte. Dafür, den Pfad des
APC-freien Modells beizubehalten, sprechen aber oft auch praktische
Gründe. Durch die Einführung von APC würden hohe Overheadkosten
entstehen, die einzelne verlagsunabhängige Zeitschriften nicht ohne
Weiteres stemmen können. Oftmals sehen sich die Herausgeber*innen
beziehungsweisen deren Institutionen nicht in der Lage, APCs für
einzelne Publikationsprojekte wirtschaftlich sinnvoll abzurechnen. Und
auch wenn dies rechtlich in vielen Fällen möglich wäre, fehlt es an
geeigneten, nachhaltigen Workflows in den oft kleinen institutionellen
Zusammenhängen.

Zum anderen sprechen sich viele Zeitschriftenmacher*innen aber auch aus
Überzeugung gegen das APC-Modell aus. Zum einen gehen mit APCs neue
Ausschlüsse einher: Sie benachteiligen Wissenschaftler*innen, die keinen
Zugriff auf Publikationsfonds oder vergleichbare Angebote zur
Refinanzierung der Gebühren haben. Dies betrifft stärker bereits
marginalisierte Stimmen zum Beispiel aus dem Globalen Süden oder
Wissenschaftler*innen ohne institutionelle Anbindung. Dadurch
verfestigen sich die Dominanzverhältnisse der bestehenden
Wissensproduktion. Weiterhin verbindet sich mit dem APC-Modell die
Gefahr, dass die Kostensteigerung im wissenschaftlichen Publizieren
langfristig nicht unterbunden werden kann (Czymborska 2017). Die drei
wissenschaftlichen Großverlage, die mittlerweile mehr als die Hälfte des
akademischen Publikationsaufkommens verlegen, orientieren sich als
gewinnorientierte Unternehmen zuvorderst an der Mehrung des
Shareholder-Values -- auch in Zeiten von Open Access. Wenn sich das
APC-Modell durchsetzt, ist davon auszugehen, dass die Gebühren weiter
steigen werden, und das, obwohl die redaktionelle Arbeit und das
Peer-Review-Verfahren weiterhin unentgeltlich von Wissenschaftler*innen
geleistet wird (vergleiche die Problematisierung bei Keller 2017). Vor
diesem Hintergrund sind viele in der Open-Access-Community überzeugt
davon, dass es Alternativen zum APC-Modell geben muss, und setzen sich
dafür ein, wissenschaftliches Publizieren nach den Prinzipien von Fair
Open Access\footnote{\url{https://www.fairopenaccess.org/}.} zu
gestalten.

Eine Alternative zu APC-Modellen ist die direkte institutionelle
Förderung von Publikationsprojekten. Diese kann entweder in konsortialen
Modellen oder direkt über einzelne Institutionen oder kleine Verbünde
von Institutionen realisiert werden. Für die in den Beispielen
betrachteten Zeitschriften, die mit kurzfristiger institutioneller
Förderung und Anschubfinanzierung durch Drittmittel als
Open-Access-Projekte gegründet wurden und erst seit kurzer Zeit
bestehen, ist allerdings der Zugang zu alternativen
Finanzierungsstrukturen schwierig. Konsortien in den
Geisteswissenschaften wie Knowledge Unlatched und die Open Library of
Humanities legten bislang den Schwerpunkt auf etablierte Zeitschriften
und Zeitschriften, die zu Open Access transformiert werden
sollen.\footnote{Voraussetzung für eine Bewerbung bei Knowledge
  Unlatched war bislang, dass Journale mindestens zehn Jahre bestehen.
  Die Open Library of Humanities nimmt weiterhin nur Zeitschriften in
  die Förderungen auf, die bisher im Subskriptionsmodell erschienen
  sind.}

Ein weiteres Problem verbindet sich mit der interdisziplinären
Ausrichtung der Journale: Die fehlende Anbindung an eine große Disziplin
hat zur Folge, dass eindeutige Zuordnungen und damit Budgetierungswege
fehlen. Disziplinäre Förderstrukturen wie etwa die
Fachinformationsdienste fallen häufig als mögliche Kooperationspartner
für OA-Projekte aus. Fachgesellschaften sind -- wenn vorhanden -- klein
und verfügen nur über begrenzte finanzielle Mittel. Auch in Bezug auf
institutionelle Förderungen seitens der Hochschulen macht sich die
fehlende disziplinäre Zuordnung bemerkbar, gerade wenn die Zielgruppe
der Zeitschriften auf unterschiedliche Fachzusammenhänge und Fakultäten
verteilt ist, so im Falle der Geschlechterforschung, der
Regionalstudien, der Internetforschung oder der Kulturwissenschaften.

Die begründete Skepsis gegenüber dem APC-Modell als allgemeingültige und
nachhaltige Lösung für die Open-Access-Finanzierung und die
Schwierigkeiten, die sich aus der spezifischen Situation von
interdisziplinären, \emph{OA-born}-Journalen in den Geistes- und
Sozialwissenschaften ergeben, führt dazu, dass die Zeitschriften ihren
Auftrag häufig auf einer sehr unsicheren finanziellen Grundlage
ausführen.

\hypertarget{qualituxe4t-kostet}{%
\section{Qualität kostet}\label{qualituxe4t-kostet}}

Reputation und Anerkennung wissenschaftlicher Publikationen werden
gemessen an der Qualität des Inhalts und der Form. In besonderer Weise
sehen sich Open-Access-Journale dabei mit der Aufgabe betraut,
Prozessqualität durch Transparenz und Ressourceneinsatz zu verbessern.
Welche Mittel und Prioritäten sollten Redaktionen aber angesichts des
hohen Kostendrucks einem kompetenten Redaktionsmanagement, einem
aufwendigen Lektorat oder professionellem Design beziehungsweise Layout
einräumen? Was muss ein Journal investieren, um professionelle
Arbeitsabläufe und somit qualitativ hochwertige Ergebnisse zu
ermöglichen? Wie kann oder soll das Verhältnis von unbezahlter zu
bezahlter Arbeit sein?

An den fünf Beispielen zeigt sich, dass wissenschaftliche Zeitschriften
in den Geistes- und Sozialwissenschaften zwar unterschiedlich
ausgestattet sind, zugleich aber ähnliche Prioritäten für stets
begrenzte Mittel setzen. Dazu gehören einerseits grundständige,
langfristige technische Infrastrukturen, meist unter Nutzung freier
Redaktions- und Content-Management-Systeme (Open Journal Systems,
Wordpress, Drupal). Auch sind dank der Kooperationen mit Bibliotheken
und Repositorien einfache informationswissenschaftliche und
bibliometrische Services zu erhalten (Langzeitarchivierung, Indizierung
und Referenzierung, Metadatenpflege und -distri\-bution). Darüber hinaus
gehender Bedarf für erfolgreiches (Reputation!) und sichtbares (Impact!)
Publizieren wird jedoch von institutioneller Seite oft als
vernachlässigbar eingestuft: Hierzu gehören die Entwicklung einer Marke
und eines Corporate Designs, Öffentlichkeitsarbeit und Social Media,
Suchmaschinenoptimierung sowie die Pflege alternativer Metriken.

Die personellen Ressourcen sind in den Geschäftsmodellen unserer
Beispiele -- und vermutlich bei den meisten Journalen dieses Segmentes
-- nicht eindeutig budgetiert. Die Organisation des redaktionellen
Alltags, das professionelle Lektorat und Korrektorat,
Übersetzungsleistungen\footnote{Dies scheint für Zeitschriften in den
  Regionalstudien ein besonderer Posten zu sein, da vielfach das
  Englische als Wissenschaftssprache hinterfragt wird.} sowie ein
effizientes Layout sind vielfach Aufgabe der wissenschaftlichen
Redakteur*innen, die hierfür weder ausreichend qualifiziert sind noch
bezahlt werden. Gern euphemistisch als \enquote{ehrenamtliches
Engagement} (Keller 2017: 32) bezeichnet, werden Wissenschaftler*innen
über Stellenanteile oder im Rahmen ihrer wissenschaftlichen
Weiterqualifikation für diese Tätigkeiten eingesetzt.\footnote{Neben
  redaktionellen Tätigkeiten fallen auch Tätigkeiten als Gutachter*innen
  oder Gastherausgeber*innen an, die ebenfalls \enquote{ehrenamtlich}
  geleistet werden.} Auch wenn die Aufbereitung und Verbreitung
wissenschaftlicher Ergebnisse essenzieller Bestandteil
wissenschaftlicher Tätigkeiten sind, besteht zwischen solchen als
eigentlichen Begleiterscheinungen des Forschens und den berechtigten
Anforderungen an wissenschaftliches Publizieren im Verlagskontext ein
Qualifikationsgefälle, das einerseits zu Überforderung und Frustration
führt; andererseits scheint das Journalsegment unserer Beispiele trotz
des hohen Engagements der jeweiligen Redaktionen durch das
Qualifikationsgefälle an Reputation oder Relevanz einzubüßen.\footnote{Wo
  bei Verlagen Lektorat, kaufmännische Aufgaben, Marketing und
  Mediengestaltung idealiter jeweils von ausgebildeten Fachkräften
  übernommen werden, haben die Journale in unserem Segment nur begrenzte
  Möglichkeiten, arbeitsteilig vorzugehen. Viele Aufgaben fallen in den
  Bereich der wissenschaftlichen Redakteur*innen.} Beides führt -- wie
durch unsere Beispiele belegt und leicht zu extrapolieren -- zu einer
Rekrutierungproblematik: Neue und innovative Open-Access-Journale stehen
vor dem Aus, wenn beispielsweise das Gründungspersonal abtritt,
Qualifikationsstellen oder Drittmittel auslaufen und zugleich die
Strahlkraft für neue Ehrenamtler*innen fehlt. Auch wenn eine pauschale
Entlohnung der redaktionellen Tätigkeiten keinen Reputationsgewinn per
se bedeutet, so ist Planungssicherheit doch eine Voraussetzung für die
anvisierte hohe Prozessqualität, durch die langfristig die Wahrnehmung
innerhalb der \enquote{peer community} verbessert werden kann.

Unsere fünf Beispiele und sicher auch andere Open-Access-Zeitschriften
scheinen somit in einen verhängnisvollen Zirkel aus Anspruch und
Umsetzung gezwungen: Wenn die langfristige Förderung von neuen,
innovativen Open-Access-Zeitschriften durch ihre eigentlichen
Stakeholder (Universitäten, Forschungszentren, Bibliotheken und
Fachgesellschaften)~auch von der Qualität im Sinne professionellen
Publizierens abhängt,\footnote{Zu Recht wird nur gefördert, was den
  hohen Standards der Wissenschaft und des Publikationswesens
  entspricht.} für diese Qualität aber eben langfristige Förderung
benötigt wird, dann können sie nur scheitern. Damit müssen sich die
privaten und öffentlichen Fördereinrichtungen die unbequeme Frage
gefallen lassen: Wie soll jemals \enquote{Verlagsniveau} erreicht
werden, wenn es an Mitteln fehlt, um Workflows und Services ähnlich wie
Verlage auf eine verlässliche Grundlage zu stellen? Zudem stellt sich
die förderpolitische Sinnfrage, wenn nur Anschubfinanzierungen (zum
Beispiel durch DFG und BMBF) bereitgestellt werden, aber nachhaltige
Geschäftsmodelle bisher fehlen.

\hypertarget{empfehlungen-fuxfcr-nachhaltige-publikationsmodelle}{%
\section{Empfehlungen für nachhaltige
Publikationsmodelle}\label{empfehlungen-fuxfcr-nachhaltige-publikationsmodelle}}

Unsere Beispiele zeigen, dass das Segment der verlagsunabhängigen
\emph{scholar-led} Open-Access-Journale, die im Laufe der letzten Jahren
gegründet worden sind, eine wichtige Rolle im wissenschaftlichen
Publizieren spielt: Diese Journale füllen gerade in interdisziplinären
Forschungsfeldern Lücken auf dem Publikationsmarkt, setzen
wissenspolitische Anliegen publizistisch um und schärfen das Bewusstsein
für die Open-Access-Idee. Darüber hinaus sind sie gut geeignet, um
alternative Finanzierungswege zum APC-Modell zu erproben. Dies ist für
die gesamte Open-Access-Community wichtig, weil mit dem APC-Modell die
Gefahr einhergeht, dass die Kosten für wissenschaftliches Publizieren
weiter steigen und sich Dominanzverhältnisse in der Wissensproduktion
verschärfen.

Die aktuelle Open-Access-Förderung erfolgt zweigleisig: Auf der einen
Seite werden Publikationsinfrastrukturen gefördert, die unabhängiges
wissenschaftliches Publizieren ermöglichen sollen. Auf der anderen Seite
werden Gelder in Publikationsfonds gelenkt, aus denen die von Verlagen
erhobenen APC finanziert werden können. Zugleich fehlt es den
Redaktionen aber an finanzieller Unterstützung über die
Anschubfinanzierung hinaus -- sie haben keinen Zugriff auf die neuen
Finanzierungstöpfe für Open Access und somit laufen die beiden Gleise
der Open-Access-Förderung auseinander. Der Effekt: Innovative
Open-Access-Projekte müssen eingestellt werden, die
Publikationsinfrastrukturen verlieren nach und nach ihre Nutzer*innen,
und das wissenschaftliche Publizieren verbleibt wesentlich Aufgabe von
(Groß-)Verlagen.

Für die Redaktionskollektive der vorgestellten Journale gehört
Open-Access-Publishing in die Hand der Wissenschaftler*innen -- und
damit sind Fachgesellschaften, Forschungseinrichtungen, Hochschulen und
Forschungsförderer auch für ihre nachhaltige Finanzierung
verantwortlich. Konkret könnte dabei an drei Punkten angesetzt werden:

Erstens: Wir brauchen Stellen(-anteile) für Redaktionsarbeit und
Editorial Management. Verlagsunabhängiges Open Access ist eine Chance,
endlich anzuerkennen, dass auch redaktionelle Publikationstätigkeiten
zum Aufgabenspektrum von Wissenschaftler*innen gehören und entsprechend
entlohnt werden müssen. Die Open-Access-Community und Netzwerke, die
sich gegen prekäre Arbeitsbedingungen in der Wissenschaft einsetzen,
haben hier gemeinsame Interessen, die sie auch politisch zur Geltung
bringen könnten.

Zweitens: Hochschulen und Forschungseinrichtungen müssen flexiblere
Möglichkeiten anbieten können, sich an Publikationsprojekten finanziell
oder mit Stellen(-anteilen) zu beteiligen. Eine Möglichkeit wäre,
Publikationsfonds so zu gestalten, dass sie neben APCs auch andere
Geschäftsmodelle unterstützen. Einzelne Wissenschaftler*innen,
Lehrstühle oder Einrichtungen könnten sich darauf bewerben, ihr
Engagement für ein Publikationsprojekt zu finanzieren. Dazu müssen
Qualitäts- und Transparenzkriterien aufgestellt werden, vergleichbar mit
den Kriterien für APC-Erstattung.\footnote{Die Universität Leipzig hat
  zu diesem Zweck einen PublikationsfondsPlus eingerichtet, allerdings
  beträgt die Förderhöchstsumme pro Projekt nur 3.000 € (siehe
  https://www.ub.uni-leipzig.de/open-science/publikationsfonds plus/).
  Auch wenn dies ein erster, kleiner Schritt in die richtige Richtung
  ist, bleibt es beim \enquote{Gießkannenprinzip}: punktuelle, eben
  nicht nachhaltige Förderung.} Auch wäre die Einrichtung eines
bundesweiten Bibliothekskonsortiums zur Unterstützung von
Platin-OA-Zeitschriften zu überlegen, das qualitativ hochwertige
Journale abhängig von regelmäßiger Evaluation dauerhaft finanziert. Das
Ziel muss eine Finanzierungsstruktur sein, in der redaktionelle Aufgaben
langfristig abgesichert werden und befristete Mittel für innovative
Weiterentwicklungen genutzt werden können.

Drittens: Hochschulen, Forschungseinrichtungen und
Forschungsbibliotheken können die Redaktionskollektive konkret
unterstützen, indem sie Services bündeln, die für ein qualitativ
hochwertiges Publishing nötig sind. Lektorats- und Korrektoratsdienste,
Übersetzungen, Layout und Design sollten von ausgebildeten Fachkräften
übernommen werden. Eine Möglichkeit wäre, diese Dienstleistungen
zusammen mit der Publikationsinfrastruktur in Open-Access-Zentren oder
Hochschulverlagen anzubieten. Dafür könnten Gelder von Subskriptions-
und APC-Modellen zu Open-Access-Zentren oder Hochschulverlagen
umgeschichtet werden. So würden nicht nur Redaktionskollektive
profitieren, sondern auch die Bibliothekscommunity, die auf diese Weise
enger mit den Nutzer*innen der Publikationsinfrastrukturen
zusammenarbeiten und diese gemeinsam weiterentwickeln kann. Schließlich
brauchen Redaktionen nachhaltige Publikationsinfrastrukturen ebenso, wie
jene Publikationsinfrastrukturen Redaktionen brauchen, die sie für
qualitativ hochwertigen Inhalte nutzen.

\hypertarget{danksagung}{%
\section{Danksagung}\label{danksagung}}

Wir danken Max Bergmann, Maike Neufend, Simon Ottersbach und Anita Runge
für die engagierten Diskussionen und zahlreichen Hinweise zu diesem
Artikel.

\hypertarget{literatur}{%
\section{Literatur}\label{literatur}}

Crawford, Walt 2019: \enquote{Gold Open Access 2013--2018: Articles in
Journals (GOA4)}. Cites \& Insights Books: Livermore, California.
\url{https://waltcrawford.name/goa4.pdf}

Czymborska, Anita 2017: \enquote{Open-Access-Ideologie und nachteilige
Systemwirkungen. Einige Überlegungen}. LIBREAS. Library Ideas, 32
(2017). \url{https://libreas.eu/ausgabe32/anonym/}

Keller, Alice 2017: \enquote{Finanzierungsmodelle für
Open-Access-Zeitschriften}. Bibliothek Forschung und Praxis 41(1).
\url{https://doi.org/10.1515/bfp-2017-0012}

Neylon, Cameron (im Erscheinen): \enquote{Research excellence is a
neo-colonial agenda (and what might be done about it)} in Kraemer-Mbula,
Tijssen, Wallace \& McLean. Transforming Research Excellence. African
Minds: Cape Town.

Pampel, Heinz 2019: \enquote{Open Access an wissenschaftlichen
Einrichtungen in Deutschland. Ergebnisse einer Erhebung im Jahr 2018}.
\url{http://doi.org/10.2312/os.helmholtz.005}

Piron, Florence 2018: \enquote{Postcolonial Open Access} in Herb,
Schöpfel. Open Divide: Critical Studies on Open Access. Litwin Books:
Sacramento, California. \url{http://hdl.handle.net/20.500.11794/16178}

Riechert, Patrick/Dubois, Frédéric 2017: \enquote{Open Abstracts: a new
peer review feature that helps scholars develop connections and
encourages transdisciplinarity}. LSE Impact Blog.
\url{https://blogs.lse.ac.uk/impactofsocialsciences/2017/07/27/open-abstracts-a-new-peer-review-feature-that-helps-scholars-develop-connections-and-encourages-transdisciplinarity/}

Schönfelder, Nina 2019: \enquote{Sind APCs das dominierende
Geschäftsmodell bei Open-Access-Zeit\-schriften?}.
\url{https://oa2020-de.org/blog/2019/08/19/APCs-dominierendes-Modell/}

%autor
\begin{center}\rule{0.5\linewidth}{\linethickness}\end{center}

\textbf{Kathrin Ganz} ist wissenschaftliche Mitarbeiterin am
Margherita-von-Brentano-Zentrum der Freien Universität Berlin. Sie
arbeitet im BMBF-Projekt \enquote{Open Gender Platform} und ist
Gründungsredakteurin des Open Gender Journals. Zuvor promovierte sie zum
politischen Diskurs des netzpolitischen Aktivismus an der Technischen
Universität Hamburg. \url{https://orcid.org/0000-0003-3968-3470}

\textbf{Marcel Wrzesinski} ist Open-Access-Officer am Alexander von
Humboldt Institut für Internet und Gesellschaft und wissenschaftlicher
Mitarbeiter im DFG-Projekt „InnOAccess``. Zuvor betreute er als
Fachredakteur die Open-Access-Aktivitäten am International Graduate
Centre for the Study of Culture (Gießen) und war wissenschaftlicher
Mitarbeiter in einem BMBF-Projekt zur Open-Access-Transformation in der
Geschlechterforschung. Er ist zudem Mitgründer des Open Gender Journal.
\url{https://orcid.org/0000-0002-2343-7905}

\textbf{Markus Rauchecker} ist Postdoc am Lateinamerika-Institut der
Freien Universität Berlin. Er ist wissenschaftlicher Mitarbeiter im
BMBF-Projekt „Integrative Biodiversitätsforschung in der kolumbianischen
Karibik`` (ColCari) und Gründungsredakteur von CROLAR -- Critical
Reviews on Latin American Research. Kontakt:
markus.rauchecker@fu-berlin.de

\end{document}
