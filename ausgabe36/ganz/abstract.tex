\textbf{Kurzfassung}: In inter- und transdisziplinären Feldern sowie
kleinen Fächern der Sozial- und Geisteswissenschaften hat sich ein
spezifisches Segment von Open-Access-Zeitschriften entwickelt: Sie
arbeiten scholar-led, verlagsunabhängig und ohne Artikelgebühren (Platin
Open Access). Diese Journale füllen Lücken im Publikationsmarkt und
wirken als Multiplikatorinnen für die Open-Access-Idee. Am Beispiel von
fünf Zeitschriften werden die strukturellen Herausforderungen
herausgearbeitet, vor denen Zeitschriften dieses Segments hinsichtlich
der nachhaltigen Finanzierung und Qualitätssicherung stehen, und
Empfehlungen erarbeitet, um qualitativ hochwertige Publikationsorgane
auf Basis offener Publikationsinfrastrukturen zu ermöglichen.

\begin{center}\rule{0.5\linewidth}{\linethickness}\end{center}

\textbf{Abstract}: A specific segment of Open Access journals has been
developed in inter- and transdisciplinary fields and small subject areas
in the social sciences and humanities: they work scholar-led,
independently of publishers and without article processing charges (that
is: platinum open access). Those journals fill gaps in the publication
market and act as multipliers for the Open Access idea. Using the
example of five journals, the structural challenges facing journals in
this segment in terms of sustainable financing and quality assurance are
highlighted, and recommendations are developed to enable high-quality
modes of publication based upon open publication infrastructures.
