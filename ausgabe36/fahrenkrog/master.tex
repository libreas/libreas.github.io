\documentclass[a4paper,
fontsize=11pt,
%headings=small,
oneside,
numbers=noperiodatend,
parskip=half-,
bibliography=totoc,
final
]{scrartcl}

\usepackage[babel]{csquotes}
\usepackage{synttree}
\usepackage{graphicx}
\setkeys{Gin}{width=.4\textwidth} %default pics size

\graphicspath{{./plots/}}
\usepackage[ngerman]{babel}
\usepackage[T1]{fontenc}
%\usepackage{amsmath}
\usepackage[utf8x]{inputenc}
\usepackage [hyphens]{url}
\usepackage{booktabs} 
\usepackage[left=2.4cm,right=2.4cm,top=2.3cm,bottom=2cm,includeheadfoot]{geometry}
\usepackage{eurosym}
\usepackage{multirow}
\usepackage[ngerman]{varioref}
\setcapindent{1em}
\renewcommand{\labelitemi}{--}
\usepackage{paralist}
\usepackage{pdfpages}
\usepackage{lscape}
\usepackage{float}
\usepackage{acronym}
\usepackage{eurosym}
\usepackage{longtable,lscape}
\usepackage{mathpazo}
\usepackage[normalem]{ulem} %emphasize weiterhin kursiv
\usepackage[flushmargin,ragged]{footmisc} % left align footnote
\usepackage{ccicons} 
\setcapindent{0pt} % no indentation in captions

%%%% fancy LIBREAS URL color 
\usepackage{xcolor}
\definecolor{libreas}{RGB}{112,0,0}

\usepackage{listings}

\urlstyle{same}  % don't use monospace font for urls

\usepackage[fleqn]{amsmath}

%adjust fontsize for part

\usepackage{sectsty}
\partfont{\large}

%Das BibTeX-Zeichen mit \BibTeX setzen:
\def\symbol#1{\char #1\relax}
\def\bsl{{\tt\symbol{'134}}}
\def\BibTeX{{\rm B\kern-.05em{\sc i\kern-.025em b}\kern-.08em
    T\kern-.1667em\lower.7ex\hbox{E}\kern-.125emX}}

\usepackage{fancyhdr}
\fancyhf{}
\pagestyle{fancyplain}
\fancyhead[R]{\thepage}

% make sure bookmarks are created eventough sections are not numbered!
% uncommend if sections are numbered (bookmarks created by default)
\makeatletter
\renewcommand\@seccntformat[1]{}
\makeatother

% typo setup
\clubpenalty = 10000
\widowpenalty = 10000
\displaywidowpenalty = 10000

\usepackage{hyperxmp}
\usepackage[colorlinks, linkcolor=black,citecolor=black, urlcolor=libreas,
breaklinks= true,bookmarks=true,bookmarksopen=true]{hyperref}
\usepackage{breakurl}

%meta

%meta

\fancyhead[L]{G. Fahrenkrog, A. Jobmann, Redaktion LIBREAS \\ %author
LIBREAS. Library Ideas, 36 (2019). % journal, issue, volume.
\href{http://nbn-resolving.de/}
{}} % urn 
% recommended use
%\href{http://nbn-resolving.de/}{\color{black}{urn:nbn:de...}}
\fancyhead[R]{\thepage} %page number
\fancyfoot[L] {\ccLogo \ccAttribution\ \href{https://creativecommons.org/licenses/by/4.0/}{\color{black}Creative Commons BY 4.0}}  %licence
\fancyfoot[R] {ISSN: 1860-7950}

\title{\LARGE{Radikale Openness -- wie Bibliotheken mit Open Educational Resources und Open Access die UN-Agenda 2030 unterstützen können}} % title
\author{Gabriele Fahrenkrog, Alexandra Jobmann, Redaktion LIBREAS} % author

\setcounter{page}{1}

\hypersetup{%
      pdftitle={Radikale Openness -- wie Bibliotheken mit Open Educational Resources und Open Access die UN-Agenda 2030 unterstützen können},
      pdfauthor={Gabriele Fahrenkrog, Alexandra Jobmann, Redaktion LIBREAS},
      pdfcopyright={CC BY 4.0 International},
      pdfsubject={LIBREAS. Library Ideas, 36 (2019).},
      pdfkeywords={Bibliothek, Open Access, Open Educational Resources, UN-Agenda 2030, Nachhaltigkeit, Interview, Openness},
      pdflicenseurl={https://creativecommons.org/licenses/by/4.0/},
      pdfcontacturl={http://libreas.eu},
      baseurl={http://libreas.eu},
      pdflang={de},
      pdfmetalang={de}
     }



\date{}
\begin{document}

\maketitle
\thispagestyle{fancyplain} 

%abstracts

%body
Gabriele Fahrenkrog (GF) und Alexandra Jobmann (AJ) stellten bei den
Open-Access-Tagen 2019 in Hannover die 17 Nachhaltigkeitsziele
(Sustainable Development Goals) vor, die das Kernstück der UN-Agenda
2030 bilden. Sie sind dabei der Frage nachgegangen, wie Bibliotheken
insbesondere in den Bereichen Open Educational Resources (OER) und Open
Access (OA) die Ziele der UN-Agenda 2030 unterstützen können.\footnote{Gabriele
  Fahrenkrog, Alexandra Jobmann: Mit Open Educational Resources und Open
  Access die UN-Agenda 2030 unterstützen. Open-Access-Tage 2019.
  \url{https://doi.org/10.5281/zenodo.3492183}.} Sie wurden vor Ort in
Hannover von Maxi Kindling und Michaela Voigt interviewt.

\begin{figure}[h!]
\centering
\includegraphics[width=30em]{SDG.jpg}
\caption{UNDP
(\url{https://commons.wikimedia.org/wiki/File:Sustainable_Development_Goals.jpg}),
\enquote{Sustainable Development Goals}, als gemeinfrei gekennzeichnet,
Details auf Wikimedia Commons:
\url{https://commons.wikimedia.org/wiki/Template:PD-UN-doc}}
\end{figure}

\emph{LIBREAS: Wie seid ihr beide auf das Thema gekommen? Gibt es
institutionelle Anknüpfungspunkte? Wie seid ihr dazu gekommen, an diesem
Thema zusammen zu arbeiten?}

GF: Wir kennen uns schon lange und sind an ähnlichen Themen
interessiert. Wir arbeiten zudem auf vergleichbaren Positionen --
Alexandra für den Nationalen Open Access Kontaktpunkt OA2020-DE und ich
arbeite für die Informationsstelle OER. Wir versuchen jeweils die
Themen, die Menschen und die Projekte in unseren Bereichen zu
koordinieren.

AJ: Ich bin vor einer Weile über Gabriele auf das Thema gekommen. Sie
hatte mich darauf aufmerksam gemacht, dass auch die IFLA (Anm. der Red.:
International Federation of Library Associations and Institutions, der
bibliothekarische \enquote{Weltverband}) in diesem Bereich in Form der
UN-Agenda aktiv ist und Empfehlungen für Bibliothekar*innen formuliert
hat. Gabriele hat die Zusammenarbeit initiiert und eine erste gemeinsame
Aktivität gab es dann beim OER Camp 2019 in Lübeck.\footnote{\#OERcamp
  2019 Lübeck siehe \url{https://www.oercamp.de/19/lubeck/}.}

\emph{LIBREAS: Welche Maßnahmen können Bibliotheken konkret für OA
beziehungsweise für OER ergreifen, um die UN-Agenda 2030 beziehungsweise
die 17 abgeleiteten Ziele für nachhaltige Entwicklung der IFLA
umzusetzen?}

GF: Radikale Openness ist hier das Stichwort. Das ist leicht gesagt, das
ist mir bewusst. Nur mit mehr Openness werden wir auch den Zugang
verbessern können -- und damit Chancengerechtigkeiten beim
Informationszugang für alle Menschen, in allen Ländern erreichen. Das
kann nur über freie Lizenzen erreicht werden, die eine Nachnutzbarkeit
ermöglichen -- das gilt für wissenschaftliche Veröffentlichungen ebenso
wie für Bildungsmaterialien. Das ist meine feste Überzeugung. Mit
Openness unterstützen wir das vierte Ziel \enquote{Hochwertige
Bildung}\footnote{\enquote{SDG 4 Hochwertige Bildung -- Inklusive, gerechte
  und hochwertige Bildung gewährleisten und Möglichkeiten des
  lebenslangen Lernens für alle fördern: Relevante Teilziele für
  deutsche Kommunen sind unter anderem die Sicherstellung, dass alle
  Mädchen und Jungen eine hochwertige Grund- und Sekundarschulbildung
  abschließen, die Sicherstellung, dass alle Mädchen und Jungen einen
  Zugang zu hochwertiger frühkindlicher Betreuung und Bildung sowie alle
  Frauen und Männer einen Zugang zu hochwertiger fachlicher, beruflicher
  und tertiärer Bildung erhalten, die Förderung der Bildung für
  nachhaltige Entwicklung sowie den Bau und Ausbau von
  Bildungseinrichtungen, die kinder-, behinderten- und
  geschlechtergerecht sind.} (siehe \url{https://sdg-portal.de/}).})
der UN-Agenda.

AJ: Im Kontext von Open Access kann es relativ klar auf zwei Ebenen
heruntergebrochen werden, welche Maßnahmen Bibliotheken konkret
unternehmen können: Das ist zum einen klassisch der Zugang zu
Information, wie es Elena Šimukovič auch in ihrer Keynote\footnote{Elena
  Šimukovič: Eine Erfolgsgeschichte? Open Access zwischen kollektivem
  Handeln, (un-)sichtbaren Infrastrukturen und neoliberalen
  Verwandlungen. Open-Access-Tage 2019.
  \url{https://doi.org/10.5281/zenodo.3482831}. Aufzeichnung im TIB
  AV-Portal: \url{https://doi.org/10.5446/44025}.} dargestellt hat, und
zum anderen die Partizipation an Information, was natürlich mit dem
Zugang zu Information einhergeht. Wer auf die Materialien, die
Forschungsergebnisse nicht zugreifen kann, kann nicht partizipieren oder
sie einsetzen. Man hat also den Kontext von Informationskompetenz --
etwa in Schulungen den Umgang mit Information zu vermitteln -- und den
Bereich Verfügbarmachung oder Zugänglichmachung von
Forschungsergebnissen, um bestimmte Ziele der UN-Agenda zu erreichen. Um
es an einem Beispiel zu erläutern: Welche Informationen brauche ich zur
Umsetzung des zweiten Ziels der UN-Agenda \enquote{Ernährungssicherheit}?
Welche davon sind Open Access und wie kann ich sie zur Verfügung
stellen, damit die Leute, die dieses Ziel umsetzen wollen, sie auch
nutzen können? Zum Beispiel indem man Repositorien aufbaut oder eine
Collection wie etwa bei ScienceOpen anlegt. Über diesen Weg kann man
Menschen befähigen, sich auch selbständig für die Ziele einzusetzen.
Bibliotheken können also ganz aktiv im infrastrukturellen Bereich und im
Kompetenzbildungsbereich arbeiten. In beiden sind sie bereits aktiv --
nur dann auch explizit auf die UN-Agenda 2030 bezogen.

GF: Bei Open Access liegt der Fokus eher auf Wissenschaft. Das Thema OER
zielt hingegen sehr viel mehr auf Bildung ab. Bibliotheken bieten hier
beste Voraussetzungen. Sie sind Lernorte und sie werden auch als solche
wahrgenommen. Hier liegt für Bibliotheken eine große Chance, ihre
Angebote weiter auszubauen. Sie können Menschen auf ihrem individuellen
Lebensweg helfen -- mit der Unterstützung von freien
Bildungsmaterialien, der Vermittlung von Kursen, Materialien, Menschen,
die ihnen auf ihrem persönlichen Lebens- und Lernweg weiterhelfen.
Überall wird darüber gesprochen, dass wir in Zukunft lebenslang lernen
müssen. Aber wer hilft uns dabei? Wer zeigt uns die Möglichkeiten auf?
Wer zeigt uns die Zugänge zu den Materialien, die wir brauchen? Wer
stellt uns vielleicht unser individuelles Programm zusammen? Ich glaube,
dass Bibliotheken dabei noch sehr viel mehr leisten können, als sie es
bisher tun. Und man muss es immer wieder betonen: Die Voraussetzung
dafür ist, dass die Materialien frei zur Verfügung gestellt werden.

\emph{LIBREAS: Gabriele, du hattest es gestern in eurem Vortrag auch so
schön gesagt: \enquote{Wir dürfen da ruhig noch etwas selbstkritischer sein,
auch die Bibliotheken.} Könnt ihr das konkret benennen? Wie ist der
Stand der Umsetzung seitens der Bibliotheken? Du hattest es ein wenig
angedeutet -- es passiert schon einiges, aber es geht noch sehr viel
mehr. Gibt es eine Erhebung dazu? Bei den OA-Tagen verweisen Vortragende
zunehmend darauf, dass auch die Daten publiziert wurden. Aber eigentlich
wäre es schön, systematisch den Status Quo zu erheben.}

GF: Nein, es gibt keine systematische Bestandsaufnahme. OER ist auch
noch ein vergleichsweise junges Thema. Mit OERinfo sind wir jetzt im
dritten Jahr der Förderung.\footnote{Die Informationsstelle OER
  (OERinfo) -- \url{https://open-educational-resources.de/} -- wird
  gefördert vom BMBF im Rahmen des Projekts \enquote{Informationsstelle OER}
  (Richtlinie zur Förderung von Offenen Bildungsmaterialien (Open
  Educational Resources -- OERinfo). Bundesanzeiger vom 15.01.2016 --
  \url{https://www.bmbf.de/foerderungen/bekanntmachung-1132.html}} Wir
haben dafür schon relativ viel erreicht. Aber es gibt kaum
Begleitforschung. Das ist auch ein Problem aus meiner Sicht.

\emph{LIBREAS: Das bedeutet ja, dass wir die Entwicklung nicht
beobachten können. Wir starten jetzt in 2019 mit eurem Vortrag zu diesem
Thema. Wo sind wir vielleicht in zehn Jahren?}

GF: Der Bereich OER betrifft vor allem Öffentliche Bibliotheken.
Öffentliche Bibliotheken sind in der Pflicht sich hier zu engagieren.
Für mein Verständnis sind Öffentliche Bibliotheken vielfach noch zu sehr
auf Bestand ausgerichtet. Lernen ist etwas Individuelles. Dann muss man
sich auf den Menschen konzentrieren statt auf Bestände. Vor diesem
Hintergrund ist noch viel Arbeit zu leisten, um ein Umdenken zu
erreichen. Wir stehen hier noch relativ am Anfang.

\emph{LIBREAS: Das ist dann auch ein charmanter Weg, um Themen von
Wissenschaftlichen und Öffentlichen Bibliotheken wieder zusammen zu
bringen. Eigentlich sind beides Themen für beide Bibliothekssparten.}

GF: Ja, das sehe ich auch so.

\emph{LIBREAS: Welche Bedeutung haben Infrastrukturen und im Speziellen
auch Forschungs- oder Bildungsinfrastrukturen für die 17
Nachhaltigkeitsziele der UN?}

AJ: Das gilt zum einen für Infrastruktur als Plattform, als
Inhaltsbehälter für die Forschungsergebnisse. Sie sind notwendig, um
gewisse Prozesse in Gang zu setzen. Das gilt nicht nur für die UN Agenda
2030, das gilt für alles mit wissenschaftlichem oder gesellschaftlichem
Bezug. Daneben sind Infrastrukturen auch Ermöglichungsplattformen für
wissenschaftliche Kommunikation. Dass es eine Möglichkeit gibt für das
aktive Betreiben und Gestalten von Wissenschaftskommunikation -- zum
Beispiel ein Blog, ein Podcast \ldots{} Für beide Ebenen braucht es eine
entsprechende Einbindung in institutionelle Zusammenhänge. Etwa eine
Einbindung in die Lehre -- bezogen auf Universitätsbibliotheken wäre
darauf zu achten, dass es für Studierende eine Verbindung zwischen
Lernmanagementsystemen und Repositorien gibt, dass beide Systeme
gemeinsam gedacht werden. Das schlägt dann auch die Brücke zu OER.
Lehrveranstaltungen werden zunehmend unter freien Lizenzen zur Verfügung
gestellt. Auch dafür braucht man wiederum Infrastruktur, die das
ermöglicht. Und man braucht entsprechendes Personal, das damit umgehen
und dazu schulen kann. Und natürlich auch Personal, das die
Infrastruktur betreut und so gestaltet, dass einzelne Nutzer*innen die
Technik ohne Probleme anwenden können.

\emph{LIBREAS: Also Infrastruktur auch als personelle Ressource
gedacht?! Das wird häufig vergessen. Wenn man Infrastruktur denkt, denkt
man in der Regel an Technik.}

AJ: Genau. Im Sinne von \enquote{wir stellen eine Plattform zur Verfügung, die
Nutzer*innen kommen von allein und nutzen sie zweckgemäß} funktioniert
es bekanntermaßen nicht. Das wäre schön, aber ...

GF: So wie Alexandra es beschrieben hat, gilt das vor allem für Open
Access. Für OER sieht das Bild sehr anders aus. Jede*r kann Materialien
unter freien Lizenzen veröffentlichen und dann ist es OER. Es heißt zwar
immer, dass es einen Bildungshintergrund oder -zusammenhang geben muss.
Aber wer will das beurteilen? Ich kann im Prinzip aus allem etwas
lernen, wenn es mein Thema betrifft. Das bedeutet, die Materialien
liegen wild verstreut im Internet. Und das Problem besteht darin sie zu
finden. Hier können Bibliotheken mit ihren Rechercheprofis natürlich
helfen.

Ich bin keine Freundin von Silos, also von Repositories und anderen
geschlossenen Systemen. Für bestimmte Bereiche ist das sinnvoll. Für OER
insgesamt halte ich es für kontraproduktiv, weil der freie Fluss, die
Veränderbarkeit, die Dynamik dadurch ausgebremst wird. Infrastruktur
braucht es natürlich dennoch -- wir brauchen Metadaten, wir brauchen
Expert*innen. Wir brauchen Menschen, die wissen, wie wir Lizenzen
maschinenlesbar in die Materialien einbetten, damit sie auch gefunden
werden. Und es braucht die Vermittlung, so wie es Alexandra auch
geschildert hat. Ohne Infrastruktur wird es nicht gehen. Ich engagiere
mich in diesem Bereich, da ich großes Potential für Bibliotheken sehen.
Bibliotheken sind prädestiniert dafür. Sie sollten sich darauf
einlassen, denn OER sind ein großes Feld, auf dem Bibliotheken aktiv
wirken und Menschen mit ihren Bedarfen in Bezug auf Bildung und Lernen
konkret helfen können.

AJ: Dafür müssen sich Bibliothekar*innen auch nicht lange weiter
qualifizieren. Metadatenmanagement und Lizenzvergabe sind klassische
bibliothekarische Aufgabenfelder. Die Kompetenzen sind bereits
vorhanden, sie müssen nur auch dafür eingesetzt werden.

\emph{LIBREAS: Noch eine konkrete Nachfrage an Gabriele. Du bist nicht
für Silos und den Einsatz von Repositorien für OER. Was würdest du
empfehlen, wenn jemand seine Materialien verfügbar machen will.
Irgendeine Infrastruktur muss dafür ja genutzt werden und es wird sicher
nicht die eigene Homepage sein.}

GF: Es kann aber auch die eigene Homepage sein. Sehr viele Lehrer*innen
stellen Materialien über eigene Webseite zur Verfügung. Dann sind die
Erschließung mit Metadaten und maschinenlesbare Lizenzinformationen
besonders wichtig, sonst findet die Materialien nur, wer es weiß. Dann
müsste sehr viel Eigenwerbung gemacht werden, damit andere Menschen
darauf aufmerksam werden. Aber so ist es einfach. Wir müssen den
Tatsachen ins Auge sehen. OER sind zu einem großen Teil auf
irgendwelchen Seiten im Internet frei verfügbar. Es gibt viele Schätze,
die gehoben werden könnten, wenn sie auffindbar wären.

Ich schließe auch nicht aus, dass es zu bestimmten Themen oder
Bildungsbereichen Repositorien gibt. In manchen Fällen macht es Sinn!
Ich persönlich glaube, dass wir mehr auf das Netz setzen sollten, auf
die offene Zirkulation. Wir sollten häufiger über Lösungen nachdenken,
um die vorhandenen Schätze zu heben.

\emph{LIBREAS: Wie gehen wir dann mit Nachhaltigkeit um? Oder ist das
gar nicht die Idee bei OER? Forschungsdaten sollen mindestens zehn Jahre
verfügbar sein. Bei Publikationen ist es eigentlich nicht in konkreten
Zahlen zu benennen -- Repositorien sollen die Materialien so lange wie
möglich vorhalten. Sind solche Vorgaben auf OER übertragbar?}

GF: Das Wesen von OER ist die dynamische Veränderbarkeit. Das schränkt
die Nachhaltigkeit der einzelnen Ressource ein. Auf der anderen Seite
ergibt sich die Nachhaltigkeit aber genau daraus. Es ist nicht die
eigene Arbeit gefragt; ich kann immer auf die Arbeit anderer aufbauen --
ich darf sie benutzen, weiterbearbeiten, remixen\ldots{} Ich darf mit
den Materialien alles tun, was die freien Lizenzen hergeben. Das ist der
Beitrag, den OER zur Nachhaltigkeit leisten kann.

\emph{LIBREAS: Die Ideen in den Materialien sind nachhaltig. In welcher
Form sie weitergegeben werden, ist dynamisch. In diesem Sinne?}

GF: Genau, Veränderbarkeit ist inhärent. Der Beitrag von OER ist
womöglich nicht die konkrete Manifestation \emph{eines} bestimmten
Werken, sondern die Idee kann weitergegeben werden. Ich kann ja auch
konkret eine Datei weiter bearbeiten. Ich muss nicht jedes Mal etwas neu
erfinden. Das ist sicher eine andere Form von Nachhaltigkeit, als wenn
wir auf Open Access schauen und den wissenschaftlichen Bezug dazu.

\emph{LIBREAS: Wir würden gern noch über ein Thema sprechen, dass auch
bei den OA-Tagen sehr präsent ist -- Finanzierung. Im OA-Bereich werden
Finanzierungsmodelle immer wichtiger. Mitunter muss man sich die Frage
stellen, ob wir aktuell an einem Punkt sind, der vor 15 Jahren so nicht
angedacht war -- das hatte Elena Šimukovič in ihrer Keynote auch
angesprochen und uns noch einmal an die eigentlichen Ziele der BOAI
Declaration}\footnote{Budapest Open Access Initiative (BOAI) siehe
  \url{https://www.budapestopenaccessinitiative.org/}.} \emph{erinnert.
Welche Rolle spielt Finanzierung, insbesondere für OER? Welche
Finanzierungsmodelle gibt es, welche sind denkbar? Welche Modelle sind
aus heutiger Sicht am vielversprechendsten in puncto Nachhaltigkeit?}

GF: Ich bin in der komfortablen Situation, dass ich die OA-Genese
mitbekommen haben. Ich kann also gut vergleichen und aus den guten
Entwicklungen sowie aus den Fehlern lernen. Aber es gibt auch
grundsätzliche Unterschiede zwischen OA und OER. Was wir im Bereich OER
lernen und dadurch womöglich von Anfang an besser machen können, ist die
Vernetzung von Communities, das heißt von vornherein verstärkt auf
Kommunikation und Austausch zu setzen. Das ist anfangs im OA-Bereich
nach meiner Wahrnehmung etwas verpasst worden -- und klappt bei OER
besser. Hier fließt viel Energie hinein und wird auch finanziell vom
BMBF (Anm. der Red.: Bundesministerium für Bildung und Forschung) stark
gefördert, zum Beispiel mit den OER-Camps. Wenn wir wollen, dass
Menschen das verinnerlichen und dass sie gemeinsam Projekte ins Leben
rufen, dann müssen wir sie zusammenbringen und das auch unterstützen.

In Bezug auf Geschäftsmodelle ist ein Vergleich schwierig. Dafür sind
die Materialien zu unterschiedlich. Wir können OA nicht als Vorbild
nehmen. Es gibt zwar Schulbuchverlage, die aber ganz anders arbeiten als
Herausgeber wissenschaftlicher Journale. Es gibt kaum Parallelen, außer
dass alle Geld verdienen wollen und die Aktivitäten auf dem
Verlagsgeschäft basieren. Aber selbst die Verlagstätigkeiten sind so
unterschiedlich, dass man sie nicht vergleichen kann. Bildung ist
Ländersache. Die Bundesländer setzen vor allem auf die Schaffung von
Infrastrukturen. Es gibt mittlerweile in Hamburg ein Repository für OER
und auch Menschen, die OER redaktionell bearbeiten, um sie für Lehrende
zur Verfügung zu stellen. In Berlin ist Ähnliches im Aufbau. Solche
Strukturen gibt es aktuell in Bayern, Hamburg und Berlin. Und auch
andere Bundesländer sind an dem Thema dran.\footnote{Ein vergleichsweise
  frisches Beispiel ist das im August 2019 gestartete Projekt
  \enquote{OER-Portal Niedersachsen}:
  \url{https://www.tib.eu/de/forschung-entwicklung/projektuebersicht/projektsteckbrief/oer-portal-niedersachsen/}.}
Das heißt, es wird schon aus den bestehenden Infrastrukturen heraus ein
Raum geschaffen und es werden Menschen dafür abgestellt, die tatsächlich
mit und an OER arbeiten, um sie frei zur Verfügung zu stellen. Menschen,
die sowieso dafür bezahlt werden -- in der Regel sind das Lehrer*innen,
die über ihr Gehalt bereits bezahlt werden und etwa Stundenkontingente
bekommen, um an OER zu arbeiten. Ein Übertragen dieser Ansätze von OER
zu OA wird nicht funktionieren; die Materialien und Verfahren sind zu
unterschiedlich.

GF: Wir erhoffen uns, dass zum einen das Bewusstsein steigt und zum
anderen auch die Nachhaltigkeit sichergestellt wird. Dass nachhaltige
Infrastrukturen und überhaupt Strukturen geschaffen werden, damit das
Thema weitergetrieben wird. Momentan sind es alles Pilotprojekte. Es
hängt natürlich viel davon ab, wie es angenommen wird. Insgesamt denkt
man in andere Richtungen und ich glaube, dass die Erfahrungen der
letzten Jahre aus dem OA-Bereich dabei eine große Rolle spielen.

AJ: Zu Open Access ist schon relativ viel gesagt worden. Es gibt eine
Studie von Information Power aus UK\footnote{Wise, A., \& Estelle, L.
  (2019). Society Publishers Accelerating Open Access and Plan S - Final
  Project Report. \url{https://doi.org/10.6084/m9.figshare.9805007.v1}.},
die festgestellt hat, das von den existierenden OA-Geschäftsmodellen nur
ein geringer Teil auf APC abzielt -- mehrheitlich liegt der Fokus auf
kollektiven Finanzierungsmöglichkeiten. Es gibt eigentlich nicht
\emph{die} Open-Access-Bewegung, wie uns Elena Šimukovič in der Keynote
erinnert hat -- sondern es gibt mehrere Strömungen oder Aktionsfelder.
Es wäre wünschenswert, wenn sich die Personen, die sich mit
OA-Geschäftsmodellen und Finanzierungsstrukturen intensiv auseinander
setzen, diese Studie zu Herzen nehmen. Mein Plädoyer wäre, APC-basierten
Ansätzen den Rücken zu kehren und den Fokus darauf zu legen \enquote{collective
action} durchzusetzen. Und über das \emph{community building}, was bei
OER ja schon besser läuft als im OA-Bereich, stärker die
Finanzierungsmöglichkeiten anzuschieben. Denn wir haben zum Beispiel die
Bibliothekar*innen, die im Bereich elektronisches Publizieren aktiv sind
und wir haben die Bibliothekar*innen, die in der Medienerwerbung tätig
sind. Warum liegt der Fokus der Erwerbung auf gedruckten Büchern und
nicht von Open-Access-Publikationen -- oder gern auch beides? Da fehlt
es an stärkerer Zusammenarbeit in der Ebene. Und wenn man elektronisches
Publizieren und Medienerwerbung zusammendenkt, kann das kollektive
Open-Access-Finanzieren mit bestehendem Personal und bestehenden Mitteln
ermöglicht werden.

GF: Mir ist wichtig, eine Sache zu ergänzen. In der OER-Bewegung sind
sehr viele Personen intrinsisch motiviert sich zu engagieren. Es gibt
viel unbezahlte Arbeit -- das gibt es natürlich auch im OA-Bereich. Ich
arbeite zum Beispiel selbst ehrenamtlich bei der OA-Zeitschrift
Informationspraxis mit und weiß das, was das beinhaltet. Aber nach
meiner Wahrnehmung ist ehrenamtliches Engagement im OER-Bereich noch
weiter verbreitet. Viele Lehrende haben ein starkes Sendungsbewusstsein
-- sie wollen ihre Materialien verfügbar machen und das gern auch unter
offenen Lizenzen. Weil sie davon überzeugt sind. Im OA-Bereich ist das
weniger stark ausgeprägt.

\emph{LIBREAS: Gibt es hier Parallelen zur Reputationsidee?}

GF: Ja, durchaus.

AJ: Die Reputationsproblematik, die es im wissenschaftlichen Bereich
gibt, ist bei Lehrkräften in dem Sinne nicht existent. Die meisten
Lehrer*innen sind freier. Sie wollen Wissen vermitteln -- sonst wären
sie nicht Lehrer*innen geworden. Und zur Wissensvermittlung gehört die
Bereitstellung von Lehrmaterialien. Wissenschaftler*innen arbeiten
hingegen anders -- logischerweise, denn das Wissenschaftssystem
funktioniert ganz anders.

\emph{LIBREAS: Ist es so anders? Zwischen Sendungsbewusstsein und
Reputation ist gar kein so großer Unterschied.}

GF: Dem würde ich widersprechen. Es ist oft eine altruistische Haltung,
aus der heraus OER entstehen. Reputation ist zunächst einmal etwas
Egoistisches. Es ist selbstbezogen. Es ist eine andere Haltung, die
dahinter steht.

\emph{LIBREAS: In der Tat, der Einfluss auf Karrieren ist im
Wissenschaftsbereich ein anderer als im Bildungsbereich. Das ist sicher
ein essentieller Faktor.}

AJ: Wenn eine Open-Access-Publikation der Karriere schaden kann, dann
wird man sich vermutlich nicht dafür entscheiden. Im Bildungsbereich
hingegen spielt OER-Aktivität keine Rolle für die Karriere. Bei OER ist
es vielleicht eher ein Zeitproblem. Ob Lehrer*innen Zeit dafür
eingeräumt bekommen, weil es einen Mehrwert für andere bietet\ldots{}

\emph{LIBREAS: Sicher wollt ihr an dem Thema UN-Agenda 2030
weiterarbeiten. Gibt es schon konkrete Pläne -- zum Beispiel in Richtung
Vortrag, Barcamp\ldots?}

GF: Nein, ich habe keine konkreten Pläne. Mich würde Begleitforschung
interessieren -- belegen zu können, was funktioniert und was nicht, und
zu ermitteln, an welchen Stellen es noch Bedarf gibt. Ich werde selbst
keinen Einfluss darauf nehmen können; das kann man nicht in der Freizeit
machen.

\emph{LIBREAS: Da wären wir wieder beim Thema.}

GF: Genau \emph{(lacht)} -- das finanziert niemand.

AJ: Mir geht es ähnlich. Ich beobachte das natürlich weiter. Häufig gibt
es im beruflichen Alltag Anknüpfungspunkte. Bei den verschiedenen
Initiativen, die sich mit der Umsetzung der einzelnen
Nachhaltigkeitsziele beschäftigen und festgestellt haben, dass Zugang zu
Information essentiell ist für die Umsetzung, gibt es tatsächlich
Bestrebungen, das mit Open Access zu verknüpfen. Also etwa konkret ein
Repositorium für Agrarwissenschaften auf den Philippinen zu starten --
damit Informationen in dem Bereich zugänglich sind. Aber das ist häufig
noch ungebündelt, Wildwuchs. Es geht mir also ähnlich wie Gabriele --
ich würde mir wünschen, dass sich jemand mit solchen Fragen beschäftigt
\emph{(lacht)}. Und vielleicht auch Beratungsstrukturen aufbaut. Damit
gewisse Standards eingehalten werden. Und damit Personen, die wenig
Erfahrung mit dem Aufbau derartiger Informationszugänge haben, Hilfe
bekommen. Aber so etwas gehört nicht zu meinem regulären Aufgabenbereich
und es ist doch begrenzt, was man in der Freizeit und ehrenamtlich
verfolgen kann. Wir werden das Thema aber auf jeden Fall weiter
verfolgen.

GF: Ja, auf jeden Fall.

\emph{LIBREAS: Wir können uns eigentlich nur strukturelle Änderungen
wünschen, damit wir uns nicht alle nur in der Freizeit mit diesen Themen
beschäftigen können.}

\emph{GF und AJ stimmen lachend zu.}

\emph{LIBREAS: Herzlichen Dank, dass ihr euch für das Interview Zeit
genommen habt.}

%autor
\begin{center}\rule{0.5\linewidth}{0.5pt}\end{center}

\textbf{Gabriele Fahrenkrog} (ORCiD:
\url{https://orcid.org/0000-0002-7835-5114}) ist Mitarbeiterin im Team
OER der Agentur J\&K -- Jöran und Konsorten
(\url{https://joeran.de/oer/}), das die Blog-Redaktion von OERinfo
verantwortet. Sie ist Bibliotheks- und Informationswissenschaftlerin und
interessiert sich besonders für alle Aspekte des Zugangs zu offenen
Informationen und Ressourcen im Internet. Sie ist Herausgeberin und
Redaktionsmitglied bei der informationswissenschaftlichen Open Access
Zeitschrift Informationspraxis (\url{http://informationspraxis.de/}) und
schreibt als Co-Autorin über Bibliotheken und OER im Blog biboer
(\url{https://open-educational-resources.de/blog/}).

\textbf{Alexandra Jobmann} (ORCiD:
\url{https://orcid.org/0000-0001-6464-4583}) ist für den Bereich
Kommunikation und Öffentlichkeitsarbeit des Nationalen
Open-Access-Kontaktpunkt OA2020-DE (\url{https://oa2020-de.org/})
zuständig. Sie ist Bibliothekarin und Informationswissenschaftlerin und
beschäftigt sich neben Open Access auch mit anderen Aspekten von
Offenheit wie Open Science und OER. Sie ist außerdem Mitglied im
Editoral Board der Open-Access-Zeitschrift Informationspraxis
(\url{http://informationspraxis.de/}).

\textbf{Maxi Kindling} (ORCiD:
\url{https://orcid.org/0000-0002-0167-0466}) ist Referentin im
Open-Access-Büro Berlin. Sie ist Mitbegründerin und -herausgeberin von
LIBREAS. Library Ideas.

\textbf{Michaela Voigt} (ORCiD:
\url{https://orcid.org/0000-0001-9486-3189}), Open-Access-Team der TU
Berlin, Redakteurin LIBREAS. Library Ideas.

\end{document}
