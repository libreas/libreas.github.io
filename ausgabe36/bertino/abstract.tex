\textbf{Kurzfassung}: Die Nachhaltigkeit einer digitalen
Forschungsinfrastruktur hängt nicht nur von Technik, sondern vor allem
von dem Vertrauen ihrer NutzerInnen ab. Dieses Vertrauen wiederum
resultiert insbesondere aus dem erkennbaren Mehrwert, den die Angebote
der Forschungsinfrastruktur den NutzerInnen bieten können. Um diesen
Mehrwert dauerhaft bieten zu können, müssen die Dienste einer Virtuellen
Infrastruktur nicht nur kontinuierlich weiterentwickelt werden, sondern
es ist auch notwendig, dass die NutzerInnen bei allen mit der Verwendung
der Angebote verbundenen Fragen unterstützt werden. Der vorliegende
Beitrag skizziert ein Modell, um fortlaufende Weiterentwicklung und
nutzerfreundliche Unterstützungsangebote der Infrastruktur zu vereinen.
Die Grundlage dafür bildet eine Neubetrachtung des Helpdesks von
DARIAH-DE, der deutschen Beteiligung an der europäischen
Forschungsinfrastruktur DARIAH-EU (Digital Research Infrastructure for
the Arts and Humanities).

\begin{center}\rule{0.5\linewidth}{\linethickness}\end{center}

\textbf{Abstract}: The sustainability of a digital research
infrastructure relies not only on technology, but first and foremost on
the trust of its users. In turn, this trust results in particular from
the tangible added value that the services provided by the research
infrastructure offer its users. In order to be able to offer this added
value on an ongoing basis, the services of a virtual research
infrastructure must not only be continuously developed, but it is also
necessary that the users are supported in all questions regarding the
use of the services. This article outlines a model for integrating
continuous development and user-friendly infrastructure support
services. The basis for this outline is a reassessment of the helpdesk
of DARIAH-DE, Germany's participation in the European research
infrastructure DARIAH-EU (Digital Research Infrastructure for the Arts
and Humanities).
