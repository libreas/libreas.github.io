Am Beispiel der Erd- und Umweltwissenschaften (einschließlich der
landschafts- und standortbezogenen Teilgebiete der Agrarwissenschaften)
zeigt dieser Beitrag, dass auch in scheinbar „unverdächtigen``
Disziplinen personenbezogene Forschungsdaten vorkommen. Eine Auswertung
der Literatur zeigt, dass allgemeine Handreichungen zum Datenschutz in
der Forschung kaum Unterstützung bei der Arbeit mit den für diese
Disziplinen besonders relevanten Fällen bieten. Für die in den Erd- und
Umweltwissenschaften besonders relevanten raumbezogenen Daten kommt
hinzu, dass selbst unter Fachjuristinnen Uneinigkeit über die
datenschutzrechtliche Bewertung herrscht. Die Ergebnisse einer
empirischen Vorstudie zeigen eine ganze Reihe verschiedener Arten
personenbezogener Forschungsdaten auf, die in der Forschungspraxis der
Erd- und Umweltwissenschaften eine Rolle spielen. Sie legen außerdem
nahe, dass der Umgang mit personenbezogenen Daten in der
Forschungspraxis der Erd- und Umweltwissenschaften auf Grund der
mangelnden Vertrautheit mit dem Datenschutz nicht immer den rechtlichen
Anforderungen entspricht. Auch Unterstützung durch Fachgesellschaften
und Infrastruktureinrichtungen -- etwa in Form disziplinspezifischer
Handreichungen, qualifizierter Beratung oder institutionalisierten
Möglichkeiten, Daten sicher zu archivieren und gegebenenfalls
zugangsbeschränkt zu publizieren -- bestehen kaum. Aus dieser Situation
ergeben sich Herausforderungen an die Weiterentwicklung der
disziplinären Datenkultur und Dateninfrastruktur, beispielsweise im
Rahmen des Prozesses zum Aufbau einer Nationalen
Forschungsdateninfrastruktur (NFDI). Zu den Möglichkeiten für
Infrastruktureinrichtungen, diese Weiterentwicklung zu unterstützen,
zeigt dieser Beitrag Handlungsoptionen auf.
