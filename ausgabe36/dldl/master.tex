\documentclass[a4paper,
fontsize=11pt,
%headings=small,
oneside,
numbers=noperiodatend,
parskip=half-,
bibliography=totoc,
final
]{scrartcl}

\usepackage[babel]{csquotes}
\usepackage{synttree}
\usepackage{graphicx}
\setkeys{Gin}{width=.4\textwidth} %default pics size

\graphicspath{{./plots/}}
\usepackage[ngerman]{babel}
\usepackage[T1]{fontenc}
%\usepackage{amsmath}
\usepackage[utf8x]{inputenc}
\usepackage [hyphens]{url}
\usepackage{booktabs} 
\usepackage[left=2.4cm,right=2.4cm,top=2.3cm,bottom=2cm,includeheadfoot]{geometry}
\usepackage{eurosym}
\usepackage{multirow}
\usepackage[ngerman]{varioref}
\setcapindent{1em}
\renewcommand{\labelitemi}{--}
\usepackage{paralist}
\usepackage{pdfpages}
\usepackage{lscape}
\usepackage{float}
\usepackage{acronym}
\usepackage{eurosym}
\usepackage{longtable,lscape}
\usepackage{mathpazo}
\usepackage[normalem]{ulem} %emphasize weiterhin kursiv
\usepackage[flushmargin,ragged]{footmisc} % left align footnote
\usepackage{ccicons} 
\setcapindent{0pt} % no indentation in captions

%%%% fancy LIBREAS URL color 
\usepackage{xcolor}
\definecolor{libreas}{RGB}{112,0,0}

\usepackage{listings}

\urlstyle{same}  % don't use monospace font for urls

\usepackage[fleqn]{amsmath}

%adjust fontsize for part

\usepackage{sectsty}
\partfont{\large}

%Das BibTeX-Zeichen mit \BibTeX setzen:
\def\symbol#1{\char #1\relax}
\def\bsl{{\tt\symbol{'134}}}
\def\BibTeX{{\rm B\kern-.05em{\sc i\kern-.025em b}\kern-.08em
    T\kern-.1667em\lower.7ex\hbox{E}\kern-.125emX}}

\usepackage{fancyhdr}
\fancyhf{}
\pagestyle{fancyplain}
\fancyhead[R]{\thepage}

% make sure bookmarks are created eventough sections are not numbered!
% uncommend if sections are numbered (bookmarks created by default)
\makeatletter
\renewcommand\@seccntformat[1]{}
\makeatother

% typo setup
\clubpenalty = 10000
\widowpenalty = 10000
\displaywidowpenalty = 10000

\usepackage{hyperxmp}
\usepackage[colorlinks, linkcolor=black,citecolor=black, urlcolor=libreas,
breaklinks= true,bookmarks=true,bookmarksopen=true]{hyperref}
\usepackage{breakurl}

%meta
\expandafter\def\expandafter\UrlBreaks\expandafter{\UrlBreaks%  save the current one
  \do\a\do\b\do\c\do\d\do\e\do\f\do\g\do\h\do\i\do\j%
  \do\k\do\l\do\m\do\n\do\o\do\p\do\q\do\r\do\s\do\t%
  \do\u\do\v\do\w\do\x\do\y\do\z\do\A\do\B\do\C\do\D%
  \do\E\do\F\do\G\do\H\do\I\do\J\do\K\do\L\do\M\do\N%
  \do\O\do\P\do\Q\do\R\do\S\do\T\do\U\do\V\do\W\do\X%
  \do\Y\do\Z}
%meta


%meta

\fancyhead[L]{Redaktion LIBREAS \\ %author
LIBREAS. Library Ideas, 36 (2019). % journal, issue, volume.
\href{http://nbn-resolving.de/}
{}} % urn 
% recommended use
%\href{http://nbn-resolving.de/}{\color{black}{urn:nbn:de...}}
\fancyhead[R]{\thepage} %page number
\fancyfoot[L] {\ccLogo \ccAttribution\ \href{https://creativecommons.org/licenses/by/4.0/}{\color{black}Creative Commons BY 4.0}}  %licence
\fancyfoot[R] {ISSN: 1860-7950}

\title{\LARGE{Das liest die LIBREAS, Nummer \#5 (Sommer/Herbst 2019)}} % title
\author{Redaktion LIBREAS} % author

\setcounter{page}{1}

\hypersetup{%
      pdftitle={Das liest die LIBREAS, Nummer \#5 (Sommer/Herbst 2019)},
      pdfauthor={Redaktion LIBREAS},
      pdfcopyright={CC BY 4.0 International},
      pdfsubject={LIBREAS. Library Ideas, 35 (2019).},
      pdfkeywords={Open Access},
      pdflicenseurl={https://creativecommons.org/licenses/by/4.0/},
      pdfcontacturl={http://libreas.eu},
      baseurl={http://libreas.eu},
      pdflang={de},
      pdfmetalang={de}
     }



\date{}
\begin{document}

\maketitle
\thispagestyle{fancyplain} 

%abstracts

%body
\emph{Beiträge von Karsten Schuldt (ks), Viola Voß (vv), Ben Kaden (bk),
Michaela Voigt (mv), Alexander Struck (as), Eva Bunge (eb)}

\hypertarget{zur-kolumne}{%
\section{1. Zur Kolumne}\label{zur-kolumne}}

Das Ziel dieser Kolumne ist, eine Übersicht über die in der letzten Zeit
erschienene bibliothekarische, informations- und
bibliothekswissenschaftliche sowie für diesen Bereich interessante
Literatur zu geben. Enthalten sind Beiträge, die der LIBREAS-Redaktion
oder anderen Beitragenden als relevant erschienen.

Themenvielfalt sowie ein Nebeneinander von wissenschaftlichen und
nicht-wissenschaftlichen Ansätzen wird angestrebt. Auch in der Form
sollen traditionelle Publikationen ebenso erwähnt werden wie
Blogbeiträge oder Videos beziehungsweise TV-Beiträge.

Gern gesehen sind Hinweise auf erschienene Literatur oder Beiträge in
anderen Formaten. Die Redaktion freut sich über entsprechende Hinweise
(siehe \url{http://libreas.eu/about/}, Mailkontakt für diese Kolumne ist
\href{mailto:zeitschriftenschau@libreas.eu}{\nolinkurl{zeitschriftenschau@libreas.eu}}).
Die Koordination der Kolumne liegt bei Karsten Schuldt. Verantwortlich
für die Inhalte sind die jeweiligen Beitragenden. Die Kolumne
unterstützt den Vereinszweck des LIBREAS-Vereins zur Förderung der
bibliotheks- und informationswissenschaftlichen Kommunikation.

LIBREAS liest gern und viel Open-Access-Veröffentlichungen. Wenn sich
Beiträge doch einmal hinter eine Bezahlschranke verbergen, werden diese
durch \enquote{{[}Paywall{]}} gekennzeichnet. Zwar macht das Plugin
Unpaywall (\url{http://unpaywall.org/}) das Finden von legalen
Open-Access-Versionen sehr viel einfacher. Als Service an der
Leserschaft verlinken wir OA-Versionen, die wir vorab finden konnten,
jedoch nach Möglichkeit auch direkt. Für alle Beiträge, die nicht frei
zugänglich sind, empfiehlt die Redaktion Werkzeuge wie den Open Access
Button (\url{https://openaccessbutton.org/}) zu nutzen oder auf Twitter
mit \#icanhazpdf (\url{https://twitter.com/hashtag/icanhazpdf?src=hash})
um Hilfe bei der legalen Dokumentenbeschaffung zu bitten.

\hypertarget{artikel-und-zeitschriftenausgaben}{%
\section{2. Artikel und
Zeitschriftenausgaben}\label{artikel-und-zeitschriftenausgaben}}

Dass sich Open Access durch APCs, Offsetting und anderen Verträge in
eine falsche Richtung entwickeln würde, wurde jetzt schon oft
postuliert. Aber wie hätte sich Open Access sonst entwickeln sollen?
Samuel A. Morre gibt zu dieser Frage einen informativen Einblick in die
frühen Entwicklungen, Hoffnungen und Projekte von Open Access zu Beginn
der 1990er Jahre. Was er vorfindet, sind engagierte Diskussionen
darüber, wie sich das wissenschaftliche Publikationswesen auch ohne
Verlage gestalten lassen würde und viele -- aber nicht ungebrochene --
Hoffnungen auf die kommenden Entwicklungen des Internets. Aus dieser
Sicht lässt sich zumindest sagen, dass die Entwicklung ganz anders
verlief, als zu Beginn diskutiert. (Samuel A. Moore (2019). Revisiting
\enquote{the 1990s debutante}: Scholar-led publishing and the prehistory
of the open access movement. In: \emph{JASIST -- Journal of the
Association for Information Science and Technology} 2019 (latest
articles), \url{https://doi.org/10.1002/asi.24306} {[}Paywall{]},
OA-Postprint: \url{https://doi.org/10.17613/41h8-j423}) (ks)

Eine kurze Übersicht zu den Erfahrungen von Bibliotheken, welche
Open-Source-Bibliotheks\-systeme einsetzen, liefert ein Artikel von
Vandana Singh. Sie befragte neun US-amerikanische Bibliotheken (sieben
ÖB, eine WB, eine Schulbibliothek), welche sich bei einer vorherigen
Studie zum gleichen Thema bereit erklärt hatten, Follow-Up Interviews zu
führen. Alle berichten auf der Basis schon länger gesammelter
Erfahrungen. Ein Ergebnis ist, dass der Grossteil der Bibliotheken die
eigentliche Arbeit des Betriebs und der Pflege der Software ausgelagert
hat, entweder an Vendors oder IT-Abteilungen von Verbünden. Nur einige
halten dies In-house. Mit der Zeit würden sie aber mutiger,
insbesondere, wenn sie die ersten Lernkurven überwunden haben, würden
sie mehr und mehr auch selber IT-Funktionen übernehmen. Die Beteiligung
an der eigentlichen Community um die jeweilige Software ist gering;
viele nehmen sie wahr, nutzen sie mit positiven Ergebnissen bei
Rückfragen, liefern aber nur selten eigene Beiträge. Wichtig sei, dass
Bibliotheken verstehen, dass sie federführend seien, was heisst, dass
sie diese Verantwortung auch annehmen und klar kommunizieren müssten,
was sie von der Software erwarten und Anpassungen vornehmen. Wie der
Artikel aber auch zeigt, ist das gut möglich. (Singh, Vandana (2019).
Open source integrated library systems migration: Librarians share the
lessons learnt. In: \emph{Journal of Librarianship and Information
Science} 51 (2019) 2, \url{https://doi.org/10.1177/0961000617709059}
{[}Paywall{]}) (ks)

Eine sehr kurzen, aber relevanten Überblick von Modellen, E-Books in die
Fernleihe zu integrieren, gibt Julie A. Murphy. Der Texte ist eine
Kolumne, also ist bei ihr keine Vollständigkeit angestrebt. Die
Beispiele stammen alle von US-amerikanischen Wissenschaftlichen
Bibliotheken und deren Konsortien. Aber schon dieser kurze Text zeigt,
(a) dass sich Bibliotheken aktiv darum bemühen, diese Integration zu
erreichen und dies auch in Lizenzverhandlungen einfordern und, (b) dass
es dafür verschiedene Modelle gibt. Das Feld ist, wie Murphy betont, (c)
weiterhin stark in Bewegung. (Julie A. Murphy (2019). Ebook Sharing
Models in Academic Libraries. In: \emph{Serials Review} 45 (2019) 3:
176--183, \url{https://doi.org/10.1080/00987913.2019.1644934}
{[}Paywall{]}) (ks)

Bibliotheken, gerade im englischsprachigen Raum, haben in den letzten
Jahren immer wieder einmal Verspätungsgebühren abgeschafft. Dafür wurden
verschiedene Gründe angegeben -- der Aufwand würde sich nicht rechnen,
zu strafen sei nicht Aufgabe der Bibliothek, die Gebühren würden vor
allem Menschen mit wenig Einkommen treffen und davon abhalten,
Bibliotheken zu nutzen und andere -- und Erwartungen an eine steigende
Nutzung von Bibliotheken und Beständen ausgesprochen. Studien, welche
daraufhin versuchten, diese Effekte zu untersuchten, kamen zu
durchwachsenen Ergebnissen. Conrad Helms legt für die Bibliothek des St.
Mary's College of Maryland eine weitere dieser Studien vor, ebenso mit
durchwachsenen Ergebnissen. Sechs Jahre, nachdem die Gebühren
abgeschafft wurden, sind zum Beispiel die Anzahl und auch die Länge der
überzogenen Leihfristen gestiegen, gleichzeitig ist die Bibliothek nicht
unzufrieden. Hervorzuheben ist die Literaturübersicht des Artikels,
welche weitere Studien der letzten Jahre zum Thema übersichtlich
zusammenfasst. (Conrad Helms (2019). Eliminating overdue fines for
undergraduates: A six-year review. In: \emph{Journal of Access Services}
16 (2019) 4: 173--189,
\url{https://doi.org/10.1080/15367967.2019.1668793} {[}Paywall{]}) (ks)

Da Bibliothekar:innen vermutlich nicht oft in den \enquote{Mitteilungen
des Deutschen Germanistenverbandes} blättern, sei auf das Heft 66 (2019)
3 hingewiesen, das auch für andere Fachrichtungen und eben Bibliotheken
interessante Lektüre bietet: \enquote{Die Digitalisierung der
Wissenschaftskommunikation in der Germanistik. Informieren --
Recherchieren -- Publizieren -- Partizipieren}. Aus der Einführung:
\enquote{Das vorliegende Heft beschäftigt sich mit den Chancen, aber
auch den Risiken der Digitalisierung in der Germanistik und ist dabei
nicht auf den häufig diskutierten Bereich der \enquote{Digital
Humanities} fokussiert, sondern auf jene Praktiken und Konzepte, die im
Ergebnis eine \enquote{Digitalisierung} der wissenschaftlichen
Arbeitstechniken und Kommunikationsformen bedeuten. Es geht uns also
gerade nicht um digitalisierte Forschung selbst, sondern um
Veränderungen der Fachkultur, die z.B. in der veränderten Rolle der
Bibliotheken und überhaupt der \enquote{Literaturversorgung}, in neuen
Recherche- und Publikationsformen, aber auch in neuen Praktiken und
Möglichkeiten wissenschaftlichen Interagierens und Kommunizierens im
Netz zum Ausdruck kommen. {[}\ldots{]} Zu Wort kommen insbesondere
Fachvertreter\_innen, die sich den Herausforderungen einer digital
orientierten Germanistik widmen und durch Projekte, institutionelle
Anbindungen oder persönliche Motivationen in die Thematik involviert
sind. {[}\ldots{]} Die Beiträge des Heftes zeigen: Der Prozess der
Digitalisierung germanistischer Wissenschaftspraxis ist in vollem Gang,
vieles ist noch unabgeschlossen und ungeklärt. Ob z.B. die
Wissenschaftsverlage ihre dominierende Stellung im Anerkennungssystem
der Germanistik behalten werden, ob etablierte Instanzen der Verteilung
von \enquote{Prestige} zugunsten eher partizipativer Strukturen
aufgelöst oder nur durch andere ersetzt werden -- all das ist noch nicht
wirklich abzusehen. Man darf gespannt sein.} (Mitteilungen des Deutschen
Germanistenverbandes, 66 (2019) 3,
\url{https://www.vr-elibrary.de/toc/mdge/66/3\%5D}) (vv)

\enquote{This research serves, in part, to confirm what librarians of
color and those who are gender nonconforming have been saying for
decades: there remains, in the LIS profession, material benefits to
performing gender in socially predictable ways.} (S. 818) Bryant, Bussel
und Halpern interviewten in ihrer Studie 29 US-amerikanische,
wissenschaftliche Bibliothekar*innen, die sich zudem unterschiedlichen
ethnischen Hintergründen zuordneten und verschiedene sexuelle
Identitäten hatten, dazu, wie diese Identitäten ihre Arbeit prägen oder
nicht prägen. Die halbstrukturierten Interviews wurden codiert und
ergaben vier Hauptthemen: (1) Visibility and Connection to Library Users
(vor allem, weil Bibliothekarinnen als Frauen mit \enquote{mütterlichen
Werten} wahrgenommen wurden, aber gleichzeitig auch Bibliothekar*innen,
die von Angehörigen von Minderheiten als ebensolche angesehen und ihnen
mit spezifischen Fragen mehr vertraut wurde), (2) Credibility and
Presumed Competence (die in \enquote{erwartbaren} Wegen zugeschrieben
oder nicht zugeschrieben wurden, je nach Geschlecht und Identität), (3)
Lack of Awareness and Hyperawareness (im Bezug auf die eigenen
Identitäten), (4) Being Your Authentic Self and Concealing Yourself.
Durch die Methode bedingt, zeigt dies Studie vor allem, wie die
Bibliothekar*innen selber die Situation wahrnehmen und auch, welche
Themen (die genannten vier) sie selber als wichtig erachten. Die
Autorinnen stellen am Ende -- wie im Zitat gesehen -- klar, dass weithin
die zu erwartenden Strukturen vorherrschen -- \enquote{(...) it pays to
be white and cisgender} (S. 819) -- auch wenn sich einiges geändert
haben mag. Sie betonen, dass diese Strukturen durch die alltäglichen
Interaktionen zwischen Personal und Studierenden, zwischen Personal und
Leitung und innerhalb des Personals reproduziert (und dann wohl auch
dort geändert) werden. (Tatiana Bryant, Hilary Bussell, Rebecca Halpern
(2019). Being Seen: Gender Identity and Performance as a Professional
Resource in Library Work. In: College \& Research Libraries 80 (2019) 6,
805--826, \url{https://doi.org/10.5860/crl.80.6.805}) (ks)

Eine aktuelle Ausgabe der Fachzeitschrift für französische
Schulbibliotheken, interCDI, hat als Schwerpunkt das Spielen als Angebot
von Schulbibliotheken. Dabei geht es um alle Formen von Spielen
(Brettspiele bis Computerspiele) und verschiedene Funktionen, die diesen
Spielen zugeschrieben werden (Lernen, Unterhaltung, Marketing). Die
Artikel stehen selbstverständlich alle dem Thema positiv gegenüber,
einige sind eher theoretischer Natur, einige berichten direkt aus
Schulbibliotheken (beziehungsweise Centres de Documentation et
d'Information -- CDI). Interessant zu sehen ist dabei, wie anders das
Thema in Frankreich behandelt wird: Die Überlegungen beziehen sich auf
das gesamte Schulbibliothekswesen des Landes. Gleichzeitig ist auch
ersichtlich, wie sehr es sich nur auf Frankreich bezieht. Über die
Grenze in die Schweiz mit ihren vielen Ludotheken wird nicht geschaut,
stattdessen werden Ludotheken nur als Angebot von Bibliotheken
beschrieben und nur aus den Erfahrungen im eigenen Land gelernt.
(interCDI 48 (2019) 280--281, septembre-octobre 2019 {[}Print{]}) (ks)

\hypertarget{uxf6ffentliche-bibliotheken-nutzung}{%
\subsection{2.1 Öffentliche Bibliotheken:
Nutzung}\label{uxf6ffentliche-bibliotheken-nutzung}}

Die Nutzung einer Bibliothek im ländlichen Raum durch Seniorinnen und
Senioren erhob Everette Scott Sikes mittels Fokusgruppen und Interviews
mit dem Bibliothekspersonal. Die Bibliothek befindet sich in den
Appalachen, also einer eher durch Armut und schwierige Lebensumstände
gekennzeichneten Gegend der USA. Zu Beginn des Artikels stellt Skies auf
der Basis von teilweise schon älterer Literatur fest, dass (a) die
Arbeit von Bibliotheken im ländlichen Raum viel weniger in der
bibliothekarischen Literatur vorkommt als die anderer Gegenden und dass
(b) die Angebote von Bibliotheken für ältere Menschen eher am Ende der
Aufmerksamkeitsketten bibliothekarischer Literatur und Praxis stehen.
Gleichwohl zeichnen die Interviews wieder einmal ein sehr positives Bild
der Bibliothek: Sie wird gelobt, die Arbeit des Personals wird positiv
hervorgehoben. Für diejenigen Personen, welche die Bibliothek nutzen,
ist sie wichtig für ihr Leben, ihre Alltagsgestaltung, intellektuelle
Anregung und als Ort der \enquote{Gemeinschaft}. Wichtig ist die
Erreichbarkeit: Diese Bibliothek wird genutzt, weil sie räumlich
erreichbar ist; die anderen Bibliotheken des Systems werden kaum
genutzt, weil sie weniger gut zu erreichen sind. Der Text bestätigt
viele Annahmen von Bibliotheken über ihre eigene Wirkung für Nutzerinnen
und Nutzer. Gleichwohl ist am Ende wieder einmal nicht ersichtlich, ob
und wie sich Bibliotheken ändern sollten, wenn sie schon so gut
funktionieren. (Scott Sikes (2019). Rural Public Library Outreach
Services and Elder Users: A Case Study of the Washington County (VA)
Public Library. In: \emph{Public Library Quarterly} (Latest Articles)
\url{https://doi.org/10.1080/01616846.2019.1659070} {[}Paywall{]})
(Siehe auch: Everette Scott Sikes (2018). \emph{The Impact of Library
Outreach Services on Elder Users in Rural Virginia: A Case Study of the
Washington County Public Library}. (Master's Thesis) Knoxville:
University of Tennessee,
\url{https://trace.tennessee.edu/utk_gradthes/5069/}) (ks)

Eine der Obsessionen des Öffentlichen Bibliothekswesens in verschiedenen
Staaten ist bekanntlich die Vorstellung, Öffentliche Bibliotheken mögen
\enquote{sozial inklusiv} wirken und ihre lokale Community stärken. Der
Diskurs um den \enquote{Dritten Ort}, zahlreiche Umbauten,
Neukonzeptionen von Veranstaltungen, Veränderung im Personal und den
Aufgaben des Personals zielen bekanntlich darauf, das Bibliotheken dies
ermöglichen sollen. Lo, He und Liu untersuchten für Bibliotheken in
Shanghai -- die recht neu ausgestattet sind und ähnlichen Überzeugungen
dienen --, ob dies zutrifft: Warum nutzen Menschen diese? Kommunizieren
sie untereinander? Bilden sie Communities? Die Daten einer auf diese
Fragen bezogenen Umfrage werden in ihrer Studie etwas sehr positiv
gedeutet, sprechen aber eine deutliche Sprache: Die Bibliotheken werden
intensiv genutzt, aber nicht so Community-bildend, wie sich das erhofft
oder vorgestellt wird. Weiterhin nutzen junge Menschen die Bibliothek
vor allem, um dort eigenständig zu lernen. Andere, gerade Senior*innen
oder Menschen ohne festen Wohnsitz, nutzen sie um \enquote{Zeit
totzuschlagen} oder sich selbstständig zu unterhalten. Das ist wichtig,
aber gleichzeitig wird die Bibliothek nicht genutzt, um Kontakte zu
anderen Personen herzustellen oder aufrechtzuerhalten. Nutzende bilden
durch die Bibliothek kein neues soziales Kapital aus. Die Autor*innen
schliessen aus ihren Daten zwar, dass die Bibliotheken einen Beitrag zum
Gemeinwohl leisten, aber offenbar keinen \enquote{Community-bildenden}
in dem Sinne, wie er gerne in der bibliothekarischen Literatur
diskutiert wird. Über den eigentlich untersuchten Fall hinaus ist die
Studie auch ein Hinweis darauf, öfter und genauer zu überprüfen, ob die
oft vorgetragenen Hoffnungen auf \enquote{starke Communities durch
Bibliotheken} tatsächlich erfüllt werden -- oder ob sie vor allem eine
bibliothekarische Obsession darstellen. (Patrick Lo; Minying He; Yan Liu
(2019). Social inclusion and social capital of the Shanghai Library as a
community place for self-improvement. In: \emph{Library Hi Tech} 37
(2019) 2: 197--218, \url{https://doi.org/10.1108/LHT-04-2018-0056}
{[}Paywall{]}) (ks)

Die Effekte, welche Veranstaltungen zur Leseförderung in Öffentlichen
Bibliotheken haben können, wurden auch schon mehrfach untersucht. Sie
können vor allem Eltern beziehungsweise Erziehungspersonen Hinweise
darauf geben, wie diese im Alltag ihre Kinder unterstützen können, um
oft, freiwillig und gewinnbringend zu lesen. Zudem können sie Eltern
selber motivieren, dass auch tatsächlich zu tun. Wichtig ist dies, weil
sich offenbar vor allem die Zeiten vor der Einschulung und während der
Schulferien auf die Lesekompetenz von Schüler*innen auswirken: Sie
machen in der Schulzeit ungefähr gleiche Fortschritte, kommen aber mit
unterschiedlichen Voraussetzungen in die Schule und entwickeln sie in
den Ferien unterschiedlich weiter: Je höher in der Sozialstruktur, um so
eher mehr. Insoweit helfen Veranstaltungen in Bibliotheken zur
Leseförderung vor allem Eltern und Kindern aus niedrigeren
Sozialschichten. All dies ist eigentlich bekannt, aber Crist et al.
haben das Gleiche nochmal in einer Studie mit acht Bibliotheken in Ohio
untersucht und nachgewiesen. Dazu wurde noch ein weiteres Rahmenwerk für
solche Veranstaltungen entworfen, angewandt und dann dessen Wirkung
untersucht. Offenbar ist das immer wieder neu nötig. (Beth Crist;
Courtney Vidacovich Donovan; Miranda Doran-Myers; Linda Hofschire
(2019): Supporting Parents in Early Literacy through Libraries (SPELL):
An Evaluation of a Multi-Site Library Project. In: \emph{Public Library
Quarterly} (latest articles).
\url{https://doi.org/10.1080/01616846.2019.1622070} {[}Paywall{]}) (ks)

In einer, vor allem was die Zahl der Beitragenden und den
Betrachtungszeitraum angeht, sehr breit angelegten Literaturschau
betrachten Ragnar Audunson et al.~die oft betonte Bedeutung öffentlicher
Bibliotheken als, wenn man so will, Vollzugsort einer demokratischen
Öffentlichkeit beziehungsweise Public Sphere. Dabei untersuchen sie die
Entwicklung und Differenzierung entsprechender Forschungsfragen, die
Themen und die identifizierten Herausforderungen sowie den Einfluss von
Social Media auch im Kontrast zur Krise traditioneller Medienformen. Bei
der Definition des Begriffs von Öffentlichkeit stützen sie sich auf ein
Konzept nach Jürgen Habermas. Öffentlichkeit ist ein für alle
Bürger*innen offener Ort, in dem mittels des Austauschs zwischen
privaten Individuen öffentliche Meinung entsteht. Jeder solche Austausch
konstituiert Öffentlichkeit. Für den Diskurs über die öffentliche
Bibliothek als Ort, an dem Öffentlichkeit ermöglicht und gepflegt wird,
lassen sich mehrere Komplexe benennen:

\begin{enumerate}
\def\labelenumi{\arabic{enumi}.}
\item
  Soziale Inklusion (im Sinne einer Bibliothek für Alle) und das Angebot
  eines allgemeinen Zugangs zu Information, wozu auch das Schließen des
  \enquote{Digital Gap} zählt;
\item
  Die Rolle der Bibliotheken zur Pflege und Verbreitung von Community,
  Demokratie, Diversität, dem Angebot der Möglichkeit zum Aufbau von
  sozialen Kapital und ihrer Funktion als Baustein der Stadtentwicklung;
\item
  Die Rolle von Bibliotheken als Anbieterinnen unterschiedlicher
  Sichtweisen und Meinungsbilder, also Meinungsfreiheit und
  Informationsfreiheit, und damit auch die Frage von Neutralität und
  Zensur, was allerdings mit einer starken Gewichtung auf die USA
  (Stichwort: Patriot Act) ausgewertet wird;
\item
  Der Einfluss von Social Media, der, wie bereits das Internet an sich,
  zum einen als potentielle Krise für das Bibliothekswesen durch
  Disintermediations-Effekte (Nutzende haben für ihr
  Informationsverhalten Alternativen zu Bibliotheken) angesehen wird.
  Zugleich gibt es erwartungsgemäß die Position, neue Technologien als
  Möglichkeit zur Positionierung zu sehen, beispielsweise über gezielten
  Kompetenzaufbau und die Mitgestaltung der Entwicklung.
\end{enumerate}

Diskurstheoretisch stellen die Autor*innen für die
Bibliothekswissenschaft eine Erweiterung der Perspektiven und Themen im
Zusammenhang mit Bibliotheken und Öffentlichkeit fest, ausgehend von
zunächst einer Konzentration auf Informationsfreiheit hin zu Aspekten
der Inklusion und der Demokratievermittlung. Die einschlägigen Arbeiten
von Jürgen Habermas wurden, wie sie feststellen, recht spät und zum
Beispiel in der deutschen Bibliothekswissenschaft eher zurückhaltend
rezipiert. Viele der ausgewerteten Arbeiten sind dabei weniger empirisch
und stärker normativ ausgerichtet, was die Autor*innen als Desiderat
benennen. Die Effekte der Digitalisierung wirken dabei in zwei
Richtungen: Einerseits senken sie die Schwelle einer möglichen Teilhabe
an Öffentlichkeit, andererseits schaffen sie möglicherweise neue
Zugangsschwellen. (Ragnar Audunson, Svanhild Aabø, Roger Blomgren,
Sunniva Evjen, Henrik Jochumsen, Håkon Larsen, Casper Hvenegaard
Rasmussen, Andreas Vårheim, Jamie Johnston, Masanori Koizumi (2019).
Public libraries as an infrastructure for a sustainable public sphere: A
comprehensive review of research. In: \emph{Journal of Documentation}.
75.4:773--790. \url{https://doi.org/10.1108/JD-10-2018-0157}
{[}Paywall{]}) (bk)

\enquote{Working Together} war ein vierjähriges Projekt einiger
kanadischer Bibliotheken, welche versuchten, durch eigene Kontakte und
Forschung zu verstehen, wie sozial ausgegrenzte Personen Bibliotheken
wahrnehmen und wie Bibliotheken entwickelt werden könnten, um diesen
Personen mehr zu nutzen. Ken Williment berichtet darüber in einem kurzen
Text. Hauptergebnisse -- die wohl auch durch den Autor herausgestellt
werden, weil sie an frühere Arbeit von ihm {[}u.a. Pateman, John ;
Williment, Ken (2013). \enquote{Developing community-led public
libraries : evidence from the UK and Canada}. Surrey: Ahsgate, 2013{]}
anschliessen -- waren: (1) Es gibt strukturelle Gründe dafür, wie
Bibliotheken wahrgenommen und genutzt werden. Deswegen ist der
\enquote{individualisierte Blick} auf Bibliotheksnutzung, bei der
angenommen wird, Menschen würden einfach individuell entscheiden, ob sie
Bibliotheken nutzen und müssten nur mit dem richtigen Marketing richtig
angesprochen werden, falsch. (2) Dass Bibliotheken der Meinung sind,
inklusive und soziale Einrichtungen zu sein, aber dass sozial
ausgegrenzte Personen das überhaupt nicht so spiegeln. Diese sehen eher
viele Barrieren -- einige aus ihrem eigenen Leben, einige von
Bibliotheken aufgebaute -- zur eigenen Bibliotheksnutzung. (3) Der
Hauptgrund, Bibliotheken nicht zu nutzen, sind anfallende
Versäumnisgebühren, wobei schon der Glaube, Gebühren zu schulden -- auch
wenn das gar nicht der Fall ist -- ausreicht, Bibliotheken nicht zu
nutzen. (4) Dass soziale ausgegrenzte Personen sich in nicht
Bibliotheken wohlfühlen und auch den Eindruck haben, dass Bibliotheken
keine Rolle in ihrem Alltag spielen. An diese Ergebnisse anschliessend
präsentiert Williment das weitere Vorgehen der Bibliotheken beim
Erarbeiten neuer Angebote, welches genau dem im schon genannten Buch
entwickelten Vorgehen entspricht. (Ken Williment. \enquote{It Takes a
Community to Create a Library}. In: Public Library Quarterly (latest
articles) \url{https://doi.org/10.1080/01616846.2019.1590757}
{[}Paywall{]}) (ks)

\hypertarget{uxf6ffentliche-bibliotheken-bibliotheksentwicklung}{%
\subsection{2.2 Öffentliche Bibliotheken:
Bibliotheksentwicklung}\label{uxf6ffentliche-bibliotheken-bibliotheksentwicklung}}

Mit wem kann eine Bibliothek kollaborieren? Nicht selten werden sich
dazu Gedanken gemacht, Pläne erstellt und Versuche angestellt, die dann
nicht immer erfolgreich sind. Ein Beispiel aus Chapel Hill, USA, zeigt
allerdings: Wenn die Bibliothek sich nur dahinter stellt und Arbeit
hineinsteckt, kann sie mit allen anderen Einrichtungen kollaborieren. In
diesem Beispiel wird die Zusammenarbeit mit einer Radio-Sendung
präsentiert, welche wöchentlich eine Stunde lang über Gesundheitsthemen
berichtet. Die Health Science Library der örtlichen Universität
unterstützt diese seit 2009 erfolgreich, beispielsweise durch den
Betrieb eines Blogs zu den Themen der Sendung, durch Zuarbeit bei
Recherchen, durch das Hervorheben von Medien zu den Themen der
jeweiligen Sendung. Gleichzeitig wird in der Sendung immer wieder auf
die Bestände und Angebote der Bibliothek hingewiesen, für den Fall, dass
Interesse an der Vertiefung einzelner Themen besteht. Das ist für beide
Seiten Arbeit, die sich aber offenbar lohnt -- sonst würde diese nicht
seit jetzt zehn Jahren geleistet werden. (Lee M. Richardson ; Barbara
Rochen Renner ; Terri Ottosen ; Adam O. Goldstein (2019). A Library and
a Radio Show: The Story of a Successful Partnership at 10 Years and
Counting. In: \emph{Journal of Library Administration} 59 (2019) 4:
395--408, \url{https://doi.org/10.1080/01930826.2019.1593713}
{[}Paywall{]}) (ks)

Mit ihrer Forschungsfrage \enquote{What is innovative to public
libraries in the United States?} stellen Potnis et al.~schon das
interessanteste Ergebnis ihrer Studie vor: Es gibt im Bibliothekswesen
offenbar unterschiedliche Vorstellungen davon, was eine Innovation ist,
wozu sie notwendig ist, wie sie durchgeführt wird und so weiter. Diese
decken sich nicht unbedingt mit Vorstellungen von Innovation aus anderen
Bereichen, beispielsweise der Ökonomie. Letzteres ist nicht negativ, da,
wie Potnis et al.~zeigen, auch in diesen Bereichen unterschiedliche
Vorstellungen vorherrschen. Ersteres ist allerdings problematisch, denn
wenn alle von Innovation sprechen, was heisst das dann noch? Die
Autor*innen erhoben in ihrer Studie deshalb per Umfrage und per Analyse
von Innovationen, die von Bibliotheken selber als solche benannt wurden,
welche Formen es (beschränkt auf die USA) gibt. Sie benennen vier
Kategorien (Progamm, Prozess, Partnerschaft, Technology), aber diese
sind auf die untersuchten Bibliotheken bezogen. In anderen Kontexten
würde eine ähnliche Studie gewiss andere Kategorien hervorbringen.
(Devendra Dilip Potnis ; Joseph Winberry ; Bonnie Finn ; Courtney Hunt
(2019). \emph{What is innovative to public libraries in the United
States? A perspective of library administrators for classifying
innovations}. In: Journal of Librarianship and Information Science
(Online First), \url{https://doi.org/10.1177/0961000619871991}
{[}Paywall{]} {[}OA-Preprint:
\url{https://trace.tennessee.edu/utk_infosciepubs/61}{]})
(ks)

\hypertarget{verluxe4ngerte-brexit-edition-zur-krise-der-bibliotheken-in-grossbritannien}{%
\subsection{2.3 Verlängerte Brexit-Edition: Zur Krise der Bibliotheken in
Grossbritannien}\label{verluxe4ngerte-brexit-edition-zur-krise-der-bibliotheken-in-grossbritannien}}

Die Krise der Öffentlichen Bibliotheken in Grossbritannien {[}siehe auch
die letzte Ausgabe dieser Kolumne in LIBREAS 35 (2019),
\url{https://libreas.eu/ausgabe35/dldl/}{]} beinhaltet auch, dass eine
wachsende Anzahl von Bibliotheken von den Gemeinden an andere
Trägereinrichtungen zum Betrieb übergeben werden. Bekanntlich ist das
einer der Alpträume des deutschen Bibliothekswesens: Sollte dies Schule
machen, so die öfter formulierte Befürchtung, würde das von der Politik
genutzt, um Bibliotheken zu de-professionalisieren und ehrenamtlich
betreiben zu lassen. Charlie Smith untersuchte an fünf solcher
Bibliotheken in Liverpool, was mit diesen nach der Übergabe tatsächlich
passiert. Die Ergebnisse sind nicht so negativ, wie das vielleicht zu
erwarten wäre. Drei der Bibliotheken werden von einer Community-Agency
betrieben, welche in lokalen Freizeitheimen auch weitere Angebote macht,
eine vom National Health Service, eine von einem weiteren, lokal
verankerten Verein. Vier der Bibliotheken wurden nach der Übernahme mit
anderen Angeboten verbunden, beispielsweise in Freizeitheime umgezogen.
Sie sind nun mit anderen Aufgaben verbunden (was Öffentliche
Bibliotheken auch so anstreben), die Öffnungszeiten wurden verlängert,
die Entleihungs- und Besuchszahlen sind -- allerdings nach Jahren des
ständigen Etatabbaus -- wieder merklich gestiegen. Gemessen an diesen
war die Übergabe erfolgreich. Smith warnt aber auch vor
Verallgemeinerungen: Die Vereine und die Quartiere, in denen sich die
Bibliotheken finden, sind alle sehr gut aufgestellt. Strukturell
bevorzugte Gegenden werden weiter bevorzugt, eine Bibliotheksstruktur
für alle sei so nicht erreichbar. Zudem sei sichtbar, dass die eine
Bibliothek, für welche die Stadt weiterhin eine Unterstützung für die
Medienaufbau zahlt, merklich besser ausgestattet ist als die anderen.
Der Artikel endet mit vielen Bedenken über die Skalierbarkeit der
Ergebnisse. (Charlie Smith (2019). An evaluation of community-managed
libraries in Liverpool. In: \emph{Library Management} 40 (2019) 5:
327--337, \url{https://doi.org/10.1108/LM-09-2018-0072} {[}Paywall{]}
{[}OA-Postprint:
\url{http://researchonline.ljmu.ac.uk/id/eprint/10713}{]}) (ks)

Zur gleichen Frage wie Smith, nämlich den Einsatz von Freiwilligen in
Öffentlichen Bibliotheken in Grossbritannien, forschten Casselden et
al., allerdings mit einem negativeren Ergebnis. Sie fragten nach den
Meinungen verschiedener Stakeholder: Wie verändern Freiwillige die
Bibliotheken? Sind sie eine Lösung und wenn ja, wofür? Das alles im
britischen Kontext (was die Übertragung der Ergebnisse auf den DACH-Raum
einschränkt). Für diesen Kontext stellen sie fest, dass der Einsatz von
Freiwilligen von der Ideologie einer \enquote{Big Society} motiviert
ist. Diese konservative Weltsicht postuliert, dass staatlichen Aufgaben
wenn möglich von der Bevölkerung selber übernommen werden sollte, die so
die Aufgabe übernehmen würde, die Gesellschaft konkret zu erhalten, zu
revitalisieren und gleichzeitig auch anzuzeigen, welche Angebote sie
tatsächlich relevant finden und welche nicht. Diese Idee wurde seit 2015
aktiv vertreten und umgesetzt, nicht nur in Bibliotheken. Casselden et
al.~zeigen nun, dass es anfänglich in einigen Fällen positive Effekte
(value-added roles) gab, auf die auch immer wieder einmal
zurückverwiesen wird. Dies vor allem, wenn durch Freiwillige neue
Angebote aufgebaut werden konnten. Allerdings sind diese Effekte jetzt
fast vollständig verschwunden. Heute werden Freiwillige tatsächlich vor
allem eingesetzt, um überhaupt Bibliotheken geöffnet zu halten. Diese
planen den Einsatz von Freiwilligen ein, obgleich sich damit versteckte
Kosten (zum Beispiel für das Management der Freiwilligen oder auch die
Aushandlung der unterschiedlichen Vorstellungen zwischen Personal und
Freiwilligen) ergeben. Zugleich zeigt sich der Effekt der Ideologie der
Big Society, so ein Grossteil der Befragten Stakeholder, konkret in
einer Verstärkung negativer Effekte wie sozialer Exklusion (weil nur
bestimmte Bibliotheken mit Freiwilligen erhalten werden können, selbst
wenn diese in andere Gegenden pendeln, um dort auszuhelfen), sinkender
Professionalität der bibliothekarischen Angebote (unter anderem weil
verschiedene Vorstellungen davon, was eine Bibliothek ist und tun
sollte, aufeinandertreffen) oder schlechter werdenden Arbeitskulturen
(weil sich die internen Auseinandersetzungen häufen). Obwohl sich
theoretisch auch positive Effekte durch die Einbindung der
Öffentlichkeit ergeben könnten, werden diese kaum wahrgenommen. (Ob dies
so in den DACH-Raum zu übertragen ist, ist schwer zu sagen: Im
britischen Kontext ging diese Entwicklung mit massiven Sparprogrammen
einher, die es in diesem Masse im DACH-Raum lange nicht gab.) {[}Biddy
Casselden; Alsion Pickard; Geoff Walton; Julie McLeod (2019). Keeping
the doors open in an age of austerity? Qualitative analysis of
stakeholder views on volunteers in public libraries. In: \emph{Journal
of Librarianship and Information Science}. 51 (2019) 4,
\url{https://doi.org/10.1177/0961000617743087} {[}Paywall{]},
{[}OA-Postprint: \url{http://nrl.northumbria.ac.uk/32413/}{]}) (ks)

Ebenso die in der letzte Ausgabe dieser Kolumne thematisierte Krise der
Öffentlichen Bibliotheken in Grossbritannien bespricht ein Artikel von
Lennart Güntzel in der aktuellen Ausgabe der Kollegen der 027.7 zum
Thema Sparen.
(\url{https://0277.ch/ojs/index.php/cdrs_0277/issue/view/148}) Güntzel
diskutiert, wie Öffentliche Bibliotheken, Nationalbibliotheken und
Universitätsbibliotheken in den letzten Jahren von Sparmassnahmen
betroffen waren. Dabei stellt er ganz klar fest, dass gerade die
Öffentlichen Bibliotheken heute ihre einst unbestrittene Vorbildfunktion
für Bibliotheken aus anderen Staaten verloren haben. Gleichzeitig stellt
er fest, dass den Sparprogrammen (auch) politische Entscheidungen
zugrunde liegen, wenn in England und Wales Bibliotheken viel mehr vom
Abbau betroffen sind, als in Schottland. (Lennart Güntzel (2019).
Endstation Fitnessstudio? Die Situation der britischen Bibliotheken im
Nachhall der Finanzkrise. In: \emph{027.7 - Zeitschrift für
Bibliothekskultur} 6 (2019) 1,
\url{https://doi.org/10.12685/027.7-6-1-183}) (ks)

\hypertarget{wissenschaftliche-bibliotheken-nutzung}{%
\subsection{2.4 Wissenschaftliche Bibliotheken:
Nutzung}\label{wissenschaftliche-bibliotheken-nutzung}}

\enquote{Academic librarians spend a considerable amount of time and
resources on activities, events, and initiatives they call outreach.}
(S. 184). Was aber ist dieser \enquote{outreach} eigentlich genau?
Dieser Frage ist Stephanie A. Diaz von der Pennsylvania State University
nachgegangen. Auf Basis einer Literaturauswertung mit Concept Analysis
und einiger Beispielfälle stellt sie folgende \enquote{working
definition} auf: \enquote{In academic librarianship, outreach is work
carried out by library employees at institutions of higher education who
design and implement a variety of methods of intervention to advance
awareness, positive perceptions, and use of library services, spaces,
collections, and issues (e.g.~various literacies, scholarly
communication, etc.). Implemented in and outside of the library,
outreach efforts are typically implemented periodically throughout the
year or as a single event. Methods are primarily targeted to current
students and faculty, however, subsets of these groups, potential
students, alumni, surrounding community members, and staff can be
additional target audiences. In addition to library-centric goals,
outreach methods are often designed to support shared institutional
goals such as lifelong learning, cultural awareness, student engagement,
and community engagement.} (S. 191) Diese Synthese und die Leitfragen,
die Diaz zur Auswertung durchgegangen ist, können Anregung für die
Planung neuer oder das Überdenken bestehender eigener
Outreach-Aktivitäten bieten. (Diaz, Stephanie A. (2019): Outreach in
academic librarianship: A concept analysis and definition. In: \emph{The
Journal of Academic Librarianship} 45 (2019) 3: 184--194.
\url{https://doi.org/10.1016/j.acalib.2019.02.012} {[}Paywall{]}) (vv)

Eine der Obsessionen in Hochschulbibliotheken in verschiedenen Staaten
ist bekanntlich der Trend, bei Neu- und Umbauten immer mehr Lern- und
Arbeitsplätze einzurichten, sowohl für das individuelles Lernen als auch
für Gruppen. Die Erwartung ist, dass so die Nutzung der jeweiligen
Bibliothek gesteigert wird. Allison et al.~untersuchten diese Erwartung
anhand ihrer Bibliothek an der University of Nebraska-Lincoln. Zu Beginn
verweisen Sie darauf, dass es bislang erstaunlich wenig Untersuchungen
zu diesen Effekten gäbe, obgleich immer mehr dieser Lernräume
eingerichtet würden. Insbesondere würde kaum nach dem Zusammenhang
zwischen diesen \enquote{Learning Commons} und den restlichen Angeboten
der Bibliotheken gefragt. Sie überprüfen nach der Einrichtung eines
solchen an ihrer Universität die Entwicklung der verschiedenen
Nutzungszahlen und versuchen, solche Zusammenhänge herauszuarbeiten.
Diese sind recht eindeutig: Learning Commons werden sehr stark genutzt.
Aber vor allem werden sie für das individuelle Arbeiten genutzt, nicht
-- wie oft angenommen -- sehr stark für die Arbeit in Gruppen. Einen
Einfluss dieser erhöhten Nutzungszahlen auf andere Angebote (Ausleihen,
Nutzung von Datenbanken et cetera) lässt sich nicht nachweisen. Auch
extra für den Learning Commons bereitgestellte Bestände, in denen zuvor
ausgesondert wurde, um nur aktuelle und relevante Literatur zur
Verfügung zu stellen, wurden nicht relevant mehr oder weniger genutzt.
Offenbar werden die Arbeitsplätze als Arbeitsplätze genutzt, aber nicht
für die Nutzung von Medien, zu denen die Bibliothek Zugang schafft. Das
scheinen zwei voneinander getrennte Aktivitäten zu sein, die sich weder
verstärken noch behindern. Die Autor*innen weisen auch darauf hin, dass
bei Entscheidungen über Learning Commons mehr mit solchen Fakten und
weniger mit Behauptungen gearbeitet werden sollte. (Deeann Allison;
Erica DeFrain; Brianna D.Hitt; David C.Tyler (2019). Academic library as
learning space and as collection: A learning commons' effects on
collections and related resources and services. In: \emph{The Journal of
Academic Librarianship} 45 (3): 305--314,
\url{https://doi.org/10.1016/j.acalib.2019.04.004} {[}Paywall{]})
{[}OA-Postprint:
\url{https://digitalcommons.unl.edu/libraryscience/384}{]}) (ks)

\enquote{{[}S{]}ome students came dressed specifically to use the Bike
Desks}. {[} 98{]} Bekanntlich wollen sich viele Wissenschaftliche
Bibliotheken heute als \enquote{Landschaften} begreifen, in denen sich
Studierenden und Forschende über lange Zeiträume hinweg aufhalten.
Dieser Lern- und Arbeitsraum will gestaltet sein, dazu gehört
Infrastruktur -- was immer wieder zu der Frage führt, was alles zu
dieser gehört und warum. Für die Bibliotheken der Texas A\&M University
zählen aktuell \enquote{Bike Desks} -- Arbeitsplätze, an denen
gleichzeitig Fahrrad gefahren werden kann -- dazu, in Zukunft wohl auch
Fitness-Räume, in denen gleichzeitig gelesen und gearbeitet wird. Das
mag etwas absurd klingen, aber in einer Studie argumentieren
Kolleg*innen aus der Bibliothek, dass für viele Nutzer*innen Bewegung
das Lernen besser macht, Müdigkeit vorbeugt und die Gesundheit fördert.
Sie haben sechs Bike Desks angeschafft und deren Nutzung mit einer
Umfrage sowie anderen Daten erstaunlich tiefgehend untersucht. Sie
werden gut genutzt, eher allein, manchmal in Paaren. Der Grossteil der
Antworten in der Umfrage ist positiv, auch die Presse und andere
Nutzende reagieren positiv. Deshalb erscheint der Bibliothek jetzt auch
die Ausweitung des Angebots als sinnvoll. (Hoppenfeld, Jared ; Graves,
Stepahnie J. ; Sewell, Robin R. ; Halling, T. Derek (2019). Biking to
Academic Success: A Study on a Bike Desk Implementation at an Academic
Library. In: \emph{Public Services Quarterly} 15 (2019) 2, 85--103,
\url{https://doi.org/10.1080/15228959.2018.1552229} {[}Paywall{]}
{[}OA-Postprint:
\url{https://oaktrust.library.tamu.edu/handle/1969.1/175383}{]}) (ks)

Der Titel der Studie \enquote{Performing Arts Library Patron Behavior}
ruft Bilder von Künsterler*innen hervor, die in der Bibliothek
Performance Art durchführen, dabei Raum und Institution Bibliothek
erschliessen. Dieses Versprechen wird leider nicht erfüllt. Stattdessen
handelt es sich um eine drei Bibliotheken umfassende Untersuchung der
Nutzung derselben mittels Beobachtungen und Umfragen. Die Studie
reproduzierte eine frühere Untersuchung, zudem wurden in allen drei
Einrichtungen die gleichen Methoden genutzt und damit die Ergebnisse
vergleichbar gemacht. Alle drei Bibliotheken sind Hochschulbibliotheken,
die sich vor allem an Studierende in den Performing Arts richten. Die
Ergebnisse -- das betonen die Autor*innen auch -- sind vergleichbar mit
zahlreichen anderen Studien dieser Art, die in den letzten Jahren
durchgeführt wurden: Die Bibliotheken wurden alle umgebaut oder neu
eingerichtet mit den erwarteten Veränderungen in Lehre und Nutzung, also
mit Fokus auf Gruppenarbeiten und Soziale Räume. Die Nutzung in allen
drei Bibliotheken ist allerdings ähnlich: Studierenden kommen zum
Ausleihen von Medien und zum Lernen in die Bibliothek, nicht für soziale
Kontakte. Gruppenarbeit findet eher selten statt; wenn, dann in
Gruppenarbeitsräumen. Zumeist wird aber für sich alleine gearbeitet,
sowohl in diesen Räumen als auch an den anderen Arbeitsplätzen.
Personen, die die Bibliothek eher kurz besuchen (bis zu zehn Minuten)
nutzen die Bestände; Personen, welche die Bibliothek länger nutzen,
lernen und arbeiten in ihr, nutzen aber kaum die vorhandenen Bestände.
Die lange vorhergesagten Veränderungen in der Nutzung von
Hochschulbibliotheken sind nicht im erwarteten Masse eingetreten,
sondern eher Nebenerscheinungen. Hauptsächlich werden die Bibliotheken
-- und das recht massiv -- \enquote{traditionell} genutzt. Was die
Studie zeigt, ist, dass dies auch für angehende Künstler*innen gilt.
Relevant ist, dass noch einmal gezeigt wurde, dass es bei den
regelmässigen Nutzer*innen offenbar zwei Gruppen gibt: Die, welche die
Bestände nutzen und die, welche die Bibliothek zum Lernen und Arbeiten
nutzen. (Clark, Joe C.; Newcomer, Nara L.; Avenarius, Christine B.;
Hursh, David W. (2019). Performing Arts Library Patron Behavior: An
Ethnographic Multi-Institutional Space Study. In: \emph{Music Reference
Services Quarterly} (Latest Articles)
\url{https://doi.org/10.1080/10588167.2019.1660127} {[}Paywall{]}) (ks)

Eine kurze Studie mit relativ wenigen Interviewten (sieben) von Pionke,
Knight-Davis und Brantley versucht einen ersten Überblick zu geben für
die Nutzung von (US-amerikanischen) Hochschulbibliotheken durch
Studierende mit Autismus. Die Eastern Illinois University, um die es in
der Studie geht, hat eine spezielles Programm zur Unterstützung solcher
Studierender, welches unter anderem das Lernen in der Bibliothek als
Gruppe beinhaltet. Studierende dieser Gruppe wurden dazu befragt, wie
sie die Bibliothek nutzen, wahrnehmen und verändern würden. Benutzt wird
sie vor allem für \enquote{Homework} und Studium. Wichtig ist dabei,
dass die Atmosphäre der Bibliothek das Arbeiten unterstützt. Routine ist
für Studierende mit Autismus wichtig, dies bietet die Bibliothek.
Gleichwohl sind sie oft schneller abgelenkt und haben oft ein höheres
Level von Angst -- was einhergeht mit einer ständigen Beobachtung der
Umgebung, um mögliche Gefahren zu erkennen. Das trifft auch in der
Bibliothek zu. Die Studierenden kommunizieren eher weniger mit dem
Bibliothekspersonal, dafür mit dem Betreuer der Gruppe, was auch damit
zu tun hat, dass ihnen dieser eher bekannt ist. Veränderungen forderten
die Befragten von der Bibliothek nicht, erst auf Nachfrage
thematisierten sie, dass die Notwendigkeit, ausserhalb der Bibliothek
Essen zu besorgen und zu konsumieren, für sie ablenkend ist. Der Text
schliesst mit einigen Vorschlägen, wie Bibliotheken reagieren können.
Zudem fordert er, dass in der bibliothekarischen Fachliteratur mehr
Menschen mit Autismus zu Wort kommen. {[}Wobei allerdings eine schnelle
Recherche eine ganze Reihe schon vorhandener Texte zu diesem Thema
anzeigt, welche im Text nicht erwähnt wurden.{]} (Pionke, J.J.;
Knight-Davis, Stacey; Brantley, John S. (2019). Library involvement in
an autism support program: A case study. In: \emph{College \&
Undergraduate Libraries} 26 (2019) 3: 221--233,
\url{https://doi.org/10.1080/10691316.2019.1668896} {[}Paywall{]}) (ks)

\hypertarget{skills-und-kompetenzen}{%
\subsection{2.5 Skills und Kompetenzen}\label{skills-und-kompetenzen}}

Etwas aus der Zeit gefallen scheint die Studie von Demasson, Partridge
und Bruce, die ein weiteres Modell von Informationskompetenz erstellten.
Man könnte der Meinung sein, dies sei zu Anfang des Jahrtausends -- auch
in Australien, wo die Autor*innen situiert sind -- schon zur Genüge
getan worden. Die Basis dieses Modells ist jetzt aber die Wahrnehmung
von Bibliothekar*innen aus Öffentlichen Bibliotheken, die in Interviews
nicht direkt nach dem Konzept \enquote{Informationskompetenz} befragt
werden, sondern indirekt. (Die Autor*innen nennen die Methode
Phänomenographie, die Analyse der Konzepte, die von Menschen generiert
und wahrgenommen werden.) Relevant an dem schliesslich erstellten Modell
mit den vier Kategorien (1) Intellectual process, (2) Technical skills,
(3) Navigating the social world, (4) Getting the desired result, ist,
dass es das Verständnis davon, was als Informationskompetenz
wahrgenommen wird, kontextuell verortet, also bei den Nutzer*innen
selber. Das verkompliziert die Frage, was Bibliotheken eigentlich in
diesem Bereich fördern oder beibringen können, ungemein. Es ist ein
Modell, das die anderen Modelle und Praktiken in Bibliotheken -- die
Schulungen, Kurse und so weiter, die oft auf spezifisches Wissen hin
konzipiert sind -- hinterfragt. (Demasson, Andrew; Partridge, Helen;
Bruce, Christine (2019). How do public librarians constitute information
literacy?. In: \emph{Journal of Librarianship and Information Science}
51 (2019) 2: 473--487, \url{https://doi.org/10.1177/0961000617726126}
{[}Paywall{]}) (ks)

\enquote{What knowledge and skills do humanities librarians require to
effectively provide support to postgraduate students in the digital
age?} -- Diese Forschungsfrage hat sich Glynnis Johnson 2016 in einer
Masterarbeit gestellt. Die Erkenntnis \enquote{a combination of
discipline-specific knowledge and skills, generic skills and personal
attributes are required} ist nicht wirklich neu, und auch die
ermittelten Fähigkeiten und Wissensgebiete sind meines Erachtens weder
überraschend noch auf den Bereich der Geisteswissenschaften beschränkt:
\enquote{In order for humanities librarians to effectively support
postgraduate students in the current digital age, a combination of
discipline-specific knowledge and skills, generic skills and personal
attributes are required. Bibliometrics (including altmetrics),
collection development, information finding skills, metadata,
referencing management tools, research data management and using
electronic information resources are considered to be the main
discipline-specific knowledge and skills (also referred to as
professional knowledge and skills) for postgraduate support. The most
dominant generic skills include: communication skills; customer service
skills; general ICT skills; management and supervisory skills; teaching,
training and coaching skills; and teamwork skills. In order to
effectively support postgraduate students, personal attributes required
include being: able to display good morals and ethics; approachable;
flexible and adaptable; patient; self-motivated; and willing to continue
learning.}

Die Sammlung kann aber vielleicht dazu anregen, sich mal über das eigene
\enquote{Kompetenzen-Set} Gedanken zu machen, vielleicht auch in Form
einer Grafik wie Figure 2 im Artikel, um bereits Vorhandenes, noch
Fehlendes und gegebenenfalls noch zu Ergänzendes zu identifizieren.
(Johnson, Glynnis ; Raju, Jaya. Knowledge and Skills Competencies for
Humanities Librarians Supporting Postgraduate Students. In: \emph{Libri}
68 (2018) 4, 331--344. \url{https://doi.org/10.1515/libri-2018-0033}
{[}Paywall{]}) (vv)

Wissenschaftler:innen und Social Media -- das ist bekanntlich ein weites
Feld. Zur Untersuchung dieses Verhältnisses am Beispiel der Uni Glasgow
haben zwei Kolleg:innen ein Modell herangezogen, das vier
\enquote{sources of self-efficacy} definiert: (1) performance
accomplishments or personal mastery experiences refer to the positive or
negative past experiences that influence researchers' ability to use
social media for sharing knowledge; (2) vicarious experience refers to
the mimicry of other researchers who effectively use social media for
knowledge sharing by observing their performance and successes, and then
attempting to replicate their behaviours; (3) verbal persuasion refers
to encouragement and discouragement from colleagues or institutions that
influence the researchers' decisions surrounding whether to use social
media for knowledge sharing; and (4) emotional arousal refers to
psychological reactions based on researchers' positive and negative
experiences of this use. (S. 1275)

Sie setzen zwei Forschungsfragen an: Welche dieser Arten von
Selbstwirksamkeit ist für Wissenschaftler:innen bei der Nutzung von
Social Media relevant, und welchen Einfluss haben sie auf die Verwendung
von Social Media zum fachlichen Austausch? Die Ergebnisse könnten
interessant sein für Institutionen, die die Nutzung von Social Media
unter \enquote{ihren} Wissenschaftler:innen fördern wollen, aber auch
für jeden selbst mit Blick auf das eigene Nutzungsverhalten.
(Alshahrani, Hussain / Rasmussen Pennington, Diane (2018): \enquote{Why
not use it more?} Sources of self-efficacy in researchers' use of social
media for knowledge sharing. In: \emph{Journal of Documentation} 74
(2018) 6: 1274--1292. \url{https://doi.org/10.110/JD-04-2018-0051}
{[}Paywall{]} {[}OA-Postprint:
\url{https://strathprints.strath.ac.uk/id/eprint/64909}{]}) (vv)

\hypertarget{materialien-zur-dekolonisierung-des-bibliothekswesens}{%
\subsection{2.6 Materialien zur Dekolonisierung des
Bibliothekswesens}\label{materialien-zur-dekolonisierung-des-bibliothekswesens}}

Am Selbstverständnis von Bibliotheken als offene und neutrale Räume
kratzen sollte eigentlich auch die Erkenntnis, dass die Systeme, mit
denen in ihnen Wissen geordnet wird, nicht neutral und aus sich selber
heraus verständlich sind, sondern immer kulturell geprägt. Sichtbar wird
das immer wieder, wenn in Bibliotheken, die vor allem von Angehörigen
von First Nations genutzt werden, andere Klassifikationen entworfen
werden als die im englischsprachigen Raum verbreitete Dewey Decimal
Classification. Anhand solcher Klassifikationen wird schnell sichtbar,
dass auch die vorgeblich logischen Klassifikationen ein bestimmtes
Denksystem und eine bestimmte Weltsicht vermitteln. Während in Kanada
und den USA oft auf das Brain Deer Classification System hingewiesen und
dieses auch in einigen Bibliotheken genutzt wird, beschreibt der Text
von Masterson, Stableford und Tait, wie in einer Öffentlichen Bibliothek
in einer vor allem von Aboriginals bewohnten, abgelegenen australischen
Gemeinde ein solches System erstellt wurde. Es handelt sich offenbar um
die erste für eine Öffentliche Bibliothek entwickelte Klassifikation,
über die in der Literatur berichtet wird. Dabei wurde auch darauf
aufgebaut, dass in dieser und einer Reihe ähnlicher Bibliotheken
offenbar eh von der Dewey Decimal Classification abgewichen wurde, weil
deren Unterteilungen und Logik für die Nutzerinnen und Nutzer nicht
nachvollziehbar war. Gleichzeitig wurde an eine Tendenz im australischen
Bibliothekswesen, kleine Bibliotheken mit lokalen \enquote{Living
Room}-Klassifikationen und -Aufstellungen auszustatten (offenbar mit
Interessenskreis-Aufstellungen vergleichbar) angeschlossen. Die
Grundstruktur der gewählten Klassifikation wird im Anhang des Artikels
mitgeliefert; interessanter ist aber die theoretische Herausforderung,
die sich aus solchen Klassifikationen ergibt. (Maeva Masterson ; Carol
Stableford ; Anja Tait (2019). Re-imagining Classification Systems in
Remote Libraries. In: \emph{Journal of the Australian Library and
Information Association} 68 (2019) 3: 278--289,
\url{https://doi.org/10.1080/24750158.2019.1653611} {[}Paywall{]}) (ks)

Katalogisierungsstandards werden verstärkt als internationale Standards
verstanden, ebenso die auf diesen aufbauende Services, obwohl sie
geprägt sind vom Bibliothekswesen in einigen wenigen Staaten im globalen
Norden. Bei diesen Standards wird eine Neutralität behauptet, die sie
nicht haben. Darauf weist der Text von Hollie White und Songphan
Choemprayong zur Katalogisierungspraxis in Thailand explizit hin. Mit
Fokusgruppe und einer Reihe von Interviews erhoben sie bei Personen, die
in unterschiedlichen Bibliotheken in Thailand für die Katalogisierung
zuständig sind, nicht nur deren Wissen über diese Standards, sondern
auch Angaben über ihre alltägliche Praxis. Es wird schnell eine
Wissenshierarchie sichtbar. So nutzen beispielsweise die meisten
Bibliotheken Daten aus dem Worldcat-Katalog, aber kopieren diese oft
händisch. Nur eine kleine Zahl grosser Wissenschaftlicher Bibliotheken
aus Thailand sind auf den Worldcat abonniert, was heisst, dass die
meisten Titel in Thai, deren Metadaten dann in den Katalogen
thailändischer Bibliotheken zu finden sind, ausserhalb Thailands
erstellt werden und diese fast keine Möglichkeit haben, darauf Einfluss
zu nehmen. Es zeigt sich auch, dass eine ganze Reihe von Bibliotheken
andere Katalogisierungspraktiken entwickelt hat, als sich an den
Standards zu orientieren -- aber auch weder Infrastruktur, Zeit noch
Wissen haben, sich an einer Entwicklung dieser (oder anderer, vielleicht
rein thailändischer) Standards zu beteiligen. So bleiben sie immer ein
Werk von ausserhalb der thailändischen Bibliothekscommunity. Zu vermuten
ist, dass das nicht nur für Thailand zutrifft. (Hollie White ; Songphan
Choemprayong (2019). Thai Catalogers' use and Perception of Cataloging
Standards. In: Cataloging \& Classification Quarterly (Latest Articles),
\url{https://doi.org/10.1080/01639374.2019.1670767} {[}Paywall{]}) (ks)

\hypertarget{monographien}{%
\section{3. Monographien}\label{monographien}}

Bei seinem Erscheinen 2008 wurde, soweit das ersichtlich ist, die Studie
von Arnulf Kutsch im Bibliothekswesen nicht wahrgenommen, obgleich er --
so der zweite Untertitel der Arbeit -- \enquote{Studien zur
Frühgeschichte der Bibliothekswissenschaft und der Zeitungskunde}
vorlegt. Dies soll hier nachgeholt werden. Grundthese der Studie ist,
dass Anfang des 20. Jahrhunderts formative Jahre sowohl für die
Zeitungswissenschaft als auch die Bibliothekswissenschaft gewesen seien,
in denen Grundparadigma zur Wirkungsforschung von Medien etabliert
wurden. Der Fokus liegt bei Kutsch auf zwei Personen: Douglas Waples für
die USA und Walter Hofmann für Deutschland. Beide hätten daran
gearbeitet, eine Leseforschung zu fundieren, die danach fragte, wie
Medien tatsächlich genutzt würden. Während Waples damit eine Basis für
die Bibliothekswissenschaft in der USA gelegt hätte, sei Hofmann in
Deutschland daran gescheitert. Hofmann, der aus dem
\enquote{Richtungsstreit} der Volksbibliotheken und Lesehallen in den
1920ern bekannt ist, gründete in Leipzig unter anderem das
\enquote{Institut für Leser- und Schrifttumskunde}. Der Fokus dieser
Einrichtung lag auf den Leseinteressen der Arbeiter*innen. Damit wollte
er eine wissenschaftliche Basis für die Bibliotheksarbeit legen, fand
aber nie Anschluss an die Wissenschaft selber und blieb dann, trotz
Annäherungsversuchen an das Regime, im Nationalsozialismus auch vom
Bibliothekswesen ausgeschlossen. Beide betrieben in den 1930er Jahren
gemeinsam eine Studie -- genauer führten sie eine von Waples noch einmal
in Leipzig durch --, allerdings ohne grossen Einfluss. Kutsch
interessiert an diesen Beiden vor allem die theoretischen Ansätze,
welche er als Grundlagen für die Zeitungswissenschaften ansieht, die
nicht genutzt wurden. Aber auch für die Bibliothekswissenschaft im
DACH-Raum ist seine Darstellung vor allem eine Erinnerung an mögliche
Wege, die nicht gegangen wurden. Das heisst nicht per se, dass sie
hätten gegangen werden müssen: Das Denken Hofmanns war zeittypisch
autoritär sowie völkisch und rassistisch genug, dass er sich begründet
Hoffnungen machen konnte, damit auch nach 1933 akzeptiert zu werden.
Dieses Denken ist selbstverständlich auch Teil seiner theoretischen
Überlegungen. Aber dennoch ist es eine Erinnerung daran, dass die
Entwicklung der Bibliothekswissenschaft und der theoretischen Basis des
Bibliothekswesens nicht so sein und bleiben muss, wie sie ist, sondern
dass immer andere mögliche Weiterentwicklungen existieren. (Arnulf
Kutsch (2008). \emph{Leseinteresse und Lektüre. Die Anfänge der
empirischen Lese(r)forschung in Deutschland und den USA am Beginn des
20. Jahrhunderts. Studien zur Frühgeschichte der Bibliothekswissenschaft
und der Zeitungskunde}. {[}Presse und Geschichte -- Neue Beiträge, 35{]}
Bremen: edition lumière, 2008 {[}Print{]}) (ks)

Selbstverständlich ist der Titel \enquote{European Origins of Library
and Information Science} viel zu gross für das, was Fidelia Ibekwe in
ihrem Buch unternimmt. Es geht nicht um ganz Europa, sondern um sieben
Staaten (Frankreich, Portugal, Spanien, Dänemark, Norwegen, Schweden und
Jugoslawien), deren Entwicklungen in der Bibliotheks- und
Informationswissenschaften sie zusammenführt. Wie im Vorwort bemerkt,
hat dies mit den Sprachen, welche die Autorin spricht, und der
Verfügbarkeit von Expertinnen und Experten, die interviewt werden
konnten, zu tun. Im gleichen Vorwort wird aber davon gesprochen, dass
dieses Buch der Beginn einer ganzen Reihe von vergleichenden Studien
dieser Art sein soll -- insoweit kann es gut sein, dass Entwicklungen in
anderen Staaten noch besprochen werden. Die Autorin hat aber
selbstverständlich Recht damit, dass die Geschichte dieser (unserer)
Wissenschaft bislang fast nur auf nationaler Ebene beschrieben wurde und
es sinnvoll wäre, sie vergleichend zu betreiben. Leider geht das Buch
dabei kleinteilig, aber unkritisch vor. Zu den einzelnen Ländern werden
die Ereignisse (wie bestimmte wichtige Publikationen, Gründung von
Lehrgängen, Verbänden, Lehrstühlen oder auch politische Entscheidungen,
die sich auf die Wissenschaft auswirkten) einfach aufgezählt. Einzelne
Texte werden genauer referiert. So lesen sich diese nationalen
Geschichten oft einfach wie Aneinanderreihungen von Entscheidungen
einzelner Personen und Institutionen. Interpretiert wird dies nicht, in
Kontexte eingebunden auch nicht. Teilweise erscheinen die besprochenen
Einzelpersonen wie einsam handelnde Heldengestalten. Es ist also eher
eine veraltete Form der Geschichtsschreibung, die hier versucht wird. Am
Ende stellt die Autorin fest, dass sich die untersuchten
romanisch-sprachigen Staaten an den französischen Entwicklungen
orientieren; die anderen an der US-amerikanischen. Zudem gab es in allen
immer wieder Versuche, das Feld eng zu führen und zu definieren.
Letztlich aber entwickelt sich die Wissenschaft immer wieder zu einer
thematisch offenen, interdiziplinären. (Fidelia Ibekwe (2019).
\emph{European Origins of Library and Information Science} {[}Studies in
Information, 13{]}. Bingley : Emerald Publishing Limited, 2019
{[}Print{]}) (ks)

\enquote{Herrschaftswissen: Bibliotheks- und Archivbauten im Alten
Reich} ist ein Tagungsband der Arbeitsgemeinschaft für Geschichtliche
Landeskunde am Oberrhein, also keines bibliothekarischen Vereins, auch
wenn die beiden Herausgeber Bibliothekare sind. Die Tagung beschäftigte
sich mit Bibliotheken und Archiven in Schlössern und Klostern des
Barocks, vornehmlich in der Region des Oberrheins. Der Fokus lag ganz
auf Baugeschichte, Bau- und Bildprogramme der Räume. Teilweise werden
die Ziele der Bauherrn thematisiert. Ganz dem barocken Raumverständnis
nach geht es immer um den Gesamtraum als Gesamtkunstwerk, nicht um den
konkreten Bestand -- der oft als Teil des Gesamtraumes begriffen wurde
-- und auch nicht um die Nutzung dieses Bestandes. Es ist ein Blick in
ein sehr anderes Verständnis von Bibliotheken und Archiven als
Repräsentationsräume und nicht als \enquote{Nutzungseinrichtungen}. Das
Buch ist durchzogen von Bildern schöner Räume, die Texte sind zu grossen
Teilen vergnüglich zu lesen, teilweise sind es aber Aufzählungen von
Fakten. Das Buch ist sehr schön anzusehen und damit unterhaltsam, auch
wenn -- oder gerade weil -- es für heutige Bibliotheksentwicklung wenig
zu sagen hat. (Konrad Krimm ; Ludger Syré (Hrsg.) (2018).
\emph{Herrschaftswissen : Bibliotheks- und Archivbauten im Alten Reich}
{[}Oberrheinische Studien, 37{]} Ostfildern : Jan Thorbecke Verlag, 2018
{[}Print{]}) (ks)

If there ever was a coffee table book for librarians this may be it. The
format is 40x29x7 cm and it clocks in at about 8 kg. Thanks to the very
welcome cardboard case you could also carry it to some place else (or to
your home when it was delivered to some pickup station). The typography
connoisseur will be pleased by the exquisite use of fonts. Book
aficionados will enjoy the different kinds of paper used when the
tactile senses will report something you hardly get from other books.
And there is no pungent stench when you open it for the first time. Of
course there is a silk-like bookmark to put in any of the 560 pages.
Short informative texts in English, German and French by Georg Ruppelt
and other competent librarians complete the information about some
remarkable historic libraries with more context on librarianship. Europe
and the Americas are covered in this book. The high-resolution
photography by Listri is just gorgeous. The librarian in me would have
appreciated some data points about the camera and its settings but we
are digressing. Every library is introduced with some short text and
metadata. In the full-size pictures the architecture, book display and
overall atmosphere really come across. Hashtag LibraryPorn. Rather
paradoxically the impressive detail in the photos just makes you want to
have even larger pictures to study. But that would require a different
kind of exhibition. The publishing house 'Taschen' claims carbon
footprint: zero. The book has been featured on \url{https://nerdcore.de}
-- a decades old non-commercial output that enlarges the horizon of most
interested readers. The book's selling price has risen since the
reviewer has bought it a year ago.\footnote{This link is an Amazon
  Partner link which helps the blog creator to pay his bills:
  \url{https://amzn.to/2FS0c0D}.} (Massimo Listri (2018). \emph{The
world's most beautiful libraries = Die schönsten Bibliotheken der Welt}.
Köln : TASCHEN, 2018 {[}Print{]}) (as)

2016 erregte die Nachricht, dass unter anderem der umstrittene
Bibliothekar und Bibliothekshistoriker Uwe Jochum sich an der Herausgabe
eines neuen \enquote{Jahrbuch für Buch- und Bibliotheksgeschichte}
beteiligte, einiges Aufsehen im Bibliothekswesen. Während die erste
Ausgabe zum Teil noch öffentlich kommentiert wurde, wurde es
anschliessend um dieses Jahrbuch sehr ruhig. Es ist aber nicht
eingegangen, sondern 2019 in der vierten Ausgabe erschienen. Der
Rezensent nimmt für diese Kurzbesprechung alle vier Ausgaben zur Basis.
Auffällig an ihnen ist, dass sie -- auch erklärtermassen, wie in den
Vorworten zu lesen ist -- in einer eher älteren Publikations- und
Wissenschaftskultur verortet werden. Das im Jahrbuch vertretene Ideal
ist das Wissenschaftlicher Bibliothekar*innen, welche sich in ihrer
Arbeit auch wissenschaftlich untersuchend mit dem Medium Buch
beschäftigen. Bettina Wagner -- eine der Herausgeber*innen -- betont
dies im Vorwort der vierten Ausgabe explizit. Dieses Bild scheint eher
dem 19. Jahrhundert verpflichtet zu sein: Geschichtliche und
altphilologische Kenntnisse scheinen bei den Lesenden vorausgesetzt zu
werden, ebenso wird ein eher konservatives Geschichtsbild bedient.
Selbstverständlich erscheint das Jahrbuch gedruckt.

Unterteilt ist es jeweils in Vorwort, Forschungsartikel (dem jeweils
grössten Teil), eher freie Essays (unter dem Rubrikennamen
\enquote{Kritik}) und Fundberichte (über von Bibliotheken neu erworbene
alte Schriften und Drucke oder solche, die in Beständen von Bibliotheken
und Archiven aufgefunden wurden). Die Forschungsartikel beschäftigen
sich zumeist mit ganz expliziten Beständen alter Drucke, der Geschichte
einzelner Bibliotheken oder sehr spezifischen Fragen mit Bezug auf Buch-
und Druckgeschichte (zum Beispiel -- Christine Sauer im ersten Band --
dem Nachweis, wie gross die Druckwerkstatt Anton Kobergers gewesen sei.)
Es gibt keine Einschränkung der Zeit, über die geforscht wird, es findet
sich also auch ein Artikel zu Bibliotheken in der Antike (lydia Glorius,
erster Band) oder der Wahrnehmung der Revolution 1918 durch Georg
Leidinger (Gerhard Hölzle, vierter Band). Der Schwerpunkt liegt aber
erkennbar auf dem Spätmittelalter und der frühen Neuzeit, räumlich
scheint der süddeutsche Raum im Fokus zu liegen.

Alle Artikel sind eingehend recherchiert und gut dokumentiert.
Geschrieben sind sie meist von Bibliothekar*innen aus Einrichtungen mit
historischen Beständen. Sie scheinen, wie gesagt, zumeist oft einem
Modell der Geschichtsschreibung des späten 19. Jahrhunderts zu folgen:
Sie setzen ein profundes geschichtliches Wissen voraus (es gibt fast nie
eine Kontextualisierung, sondern es wird einfach angenommen, dass die
Lesenden zum Beispiel wissen, wie der bayerische Staat gefestigt wurde
oder warum bestimmte theologische Auseinandersetzungen relevant waren),
gleichzeitig wird bei den Untersuchungen zumeist ganz kleinteilig
vorgegangen. Das Erkenntnisinteresse wird eigentlich nie benannt, auch
kaum die Relevanz der jeweiligen Ergebnisse diskutiert. Vor allem wird
im Jahrbuch Geschichte offenbar sehr konservativ begriffen: Als Ablauf
von Ereignissen, die zu berichten seien, wie sie vorfiehlen -- nicht als
eine Möglichkeit der Entwicklung unter vielen, die auch hätten sein
können. So, als wäre Geschichte nicht gestaltbar.

Polemiken, die von Uwe Jochum offenbar erwartet wurden und weshalb das
Jahrbuch zuerst Beachtung fand, finden sich nur verstreut in den ersten
Ausgaben. Vielmehr ist das Jahrbuch, bei allem Respekt vor der
Arbeitsleistung hinter den Artikeln (und bei einigen Artikeln, die aus
dem Rahmen fallen), aber vor allem eines: Bieder. (Uwe Jochum, Bernhard
Lübbers, Armin Schlechter, Bettina Wagner (Hrsg.) (2016 -- ).
\emph{Jahrbuch für Buch- und Bibliotheksgeschichte}. Heidelberg:
Universitätsverlag Winter {[}Print{]}) (ks)

\hypertarget{social-media}{%
\section{4. Social Media}\label{social-media}}

Peter Delin ist bekannt dafür, dezidierte Meinungen zur
Bibliotheksentwicklung in Deutschland und vor allem in Berlin zu haben.
Insbesondere gegen die Utopien beziehungsweise Dystopien der Ersetzung
von gedruckten durch elektronische Medien, welche eine Grundlage für
viele aktuelle Bibliotheksstrategien bilden, ist er kritisch
eingestellt. Garantiert sind viele Bibliotheksdirektionen, die sich
Gedanken machen wollen über die Zukunft ihrer Bibliotheken, genervt von
diesen Einwürfen -- würden sie doch, wenn man sie ernst nimmt, oft
bedeuten, dass heute schon viel richtig gemacht wird, also der ganze
Veränderungsdruck, den man behauptet, um Entwicklungen zu begründen, gar
nicht so gross ist. Das könnte bedeuten, dass man mehr darüber
nachdenken müsste, warum man bestimmte Transformationen angeht. Und
dennoch -- beziehungsweise gerade deshalb -- lohnt es sich, die Beiträge
von Delin wahrzunehmen. Sie sind nämlich, bei aller Polemik, oft
fundierter als die meisten Strategiekonzepte. In einer Nachricht, Anfang
August, in der Inetbib liefert Delin dafür wieder ein Beispiel: Anhand
vorliegender statistischer Daten zeigt er, dass die Ablösung von
gedruckten Büchern durch E-Books in den Öffentlichen Bibliotheken in
Deutschland und in Skandinavien in den letzten Jahren gerade nicht
stattgefunden hat, trotz allem Diskurs darum. Erstaunlich ist dabei
selbstverständlich, dass diese Daten offen liegen, also alle, die sich
um die Entwicklung von Bibliotheken Gedanken machen, auf diese hätten
schon zugreifen können. Zu wünschen wäre, wenn zukünftige Debatten um
die Entwicklung von Bibliotheken auf diesem Niveau geführt würden.
{[}\url{http://www.inetbib.de/listenarchiv/msg66644.html}{]} (ks)

\hypertarget{konferenzen-konferenzberichte}{%
\section{5. Konferenzen,
Konferenzberichte}\label{konferenzen-konferenzberichte}}

{[}Diesmal keine Hinweise.{]}

\hypertarget{populuxe4re-medien-zeitungen-radio-tv-etc.}{%
\section{6. Populäre Medien (Zeitungen, Radio, TV
etc.)}\label{populuxe4re-medien-zeitungen-radio-tv-etc.}}

Katrin Passig wundert sich in der Frankfurter Rundschau ein wenig über
das Phänomen, dass die unzulässige digitale Vervielfältigung von
Textinhalten, und damit auch die Schattenbibliotheken, ein
vergleichsweise spätes Phänomen des doch Filesharing begünstigenden
Internets war und erst seit den Gründungen von Library Genesis im Jahr
2008 und Sci-Hub im Jahr 2011 eine Art Schub erfuhr. Zugleich weiß sie
aus persönlichen Gesprächen und Erfahrungen zu berichten, dass an
Hochschulen eine Nutzung solcher Bezugsquellen für Literatur weit
verbreitet -- \enquote{knapp 100 Prozent der {[}anekdotisch von ihr{]}
Befragten} -- ist. Zwei Gründe kristallisieren sich heraus: Einerseits
ist es meist weniger aufwendig, über Schattenbibliotheken einen Text
abzurufen als beispielsweise über den Gang in die Bibliothek.
Andererseits haben Universitätsbibliotheken sehr viele Inhalte nicht im
Bestand, Schattenbibliotheken jedoch schon -- die Möglichkeit einer
Fernleihe wird nicht thematisiert, verständlicherweise, siehe
Bequemlichkeitargument. Offen jedoch spricht kaum jemand über diese
Praxis einer auf digitale Kopien setzenden ergänzenden und rechtlich
hochproblematischen Literaturversorgung, was, so ihre interessante und
unerwartete Perspektive, für eine Wissenschaftsgeschichtschreibung in
der Zukunft neue Herausforderung für die Rekonstruktion
wissenschaftlicher Arbeitsweisen in den ersten Jahrzehnten des 21.
Jahrhunderts aufwerfen wird. (Kathrin Passig: Die Buch-Kopisten. In:
\emph{Frankfurter Rundschau}, 13.07.2019, S. 10) (bk)

Im Interview mit der Süddeutschen Zeitung betont Elisabeth Niggemann,
Generaldirektorin der Deutschen Nationalbibliothek, dass ein zentrales
Problem für Bibliotheken (oder die Deutsche Nationalbibliothek, die
Interviewsituation klärt das nicht eindeutig auf) ist,
\enquote{kompetente Mitarbeiterinnen und Mitarbeiter zu finden}. Dies
gilt insbesondere für IT-Kompetenzen. (Caspar Busse: \enquote{Ohne
Geschichte kann man nicht leben}. Montagsinterview mit Elisabeth
Niggemann. In: \emph{Süddeutsche Zeitung}, 22. Juli 2019, S. 16) (bk)

Die öffentliche Bibliothek in Remscheid setzt auf eine Reihe neuer
Dienste, berichtet die Solinger Morgenpost. Dazu gehört ein
\enquote{freier Marktplatz} beziehungsweise \enquote{Market Space} zum
Ausprobieren digitaler Technik (unter anderem 3D-Drucker), die
Einrichtung einer Sitz- und Kaffeeecke und eine Bühne für
Veranstaltungen -- alles Entwicklungen, die unter das Programm eines
Umbaus zu einem \enquote{Dritten Ort} im Sinne Ray Oldenburgs gehören.
Maßgeblich geprägt ist die Entwicklung, so der Beitrag, durch eine
mittlerweile im Haus aktive \enquote{neue Generation an Bibliothekaren
zwischen 25 und 35 Jahren}. (Christian Peiseler: \emph{Die Bibliothek
der Zukunft startet}. In: Solinger Morgenpost, 25.07.2019, S. D1) (bk)

Die Süddeutsche Zeitung begeistert sich en passant in einem Bericht über
das Architekturbüro MVRDV, deren Bibliothek in der südholländischen
Stadt Spijkenisse, gesetzt \enquote{auf einen Sockel, in dem ein
Lesecafé, ein Umweltzentrum und Geschäfte einzogen, was wirkt, als hätte
sich hier ein Bücherberg über den Rest aufgetürmt.} Das Büro selbst
führt das Projekt denn auch unter der Bezeichnung Book Mountain
(\url{https://www.mvrdv.nl/projects/126/book-mountain}). (Laura
Weissmüller: Jedes Haus ein Tausendsassa. In: \emph{Süddeutsche
Zeitung}, 11.07.2019, S. 11) (bk)

Eine von vielen Medien übernommene Agenturmeldung der dpa berichtet über
Einschätzungen von Cornelia Poenicke vom Deutschen Bibliotheksverband
Sachsen-Anhalt zur Zukunft von Bibliotheken. Sie betont den Schwenk in
der Ausrichtung von der Orientierung auf Medien hin zur einer stärkeren
Diversifikation der Angebote, der sich in dem Slogan \enquote{Weniger
Regale, mehr Begegnung} bündelt. Während in Sachsen-Anhalt die Zahl der
Einrichtungen zurückgeht (2018: 186 ÖBs, 2010: 233), bleiben die
Nutzungshäufigkeiten auf gleichbleibendem Niveau (2018: 1,98 Millionen
Besucher*innen). Herausforderungen sieht die Bibliotheksleiterin der
Stadtbibliothek Magdeburg vor allem für kleinere Einrichtungen in
schrumpfenden Gemeinden. Den Bibliotheken fallen im Zuge dieses
demographischen Wandels auch Nutzer*innen weg. Zugleich treten
Streamingdienste in direkte Versorgungskonkurrenz. Und drittens sind
öffentliche Bibliotheken freiwillige und damit bei Bedarf und
Haushaltsbeschluss auch einsparbare Leistungen von Kommunen. Dennoch
beziehungsweise umso mehr bleibt die Versorgung in der Fläche eine
Aufgabe für das Bibliothekswesen, wofür in Sachsen-Anhalt vier
Fahrbibliotheken betrieben werden. (dpa/sa: Bibliotheken verändern sich:
große Häuser werden attraktiver. In: \emph{Volksstimme /
volksstimme.de}, 07.07.2019,
\url{https://www.volksstimme.de/sachsenanhalt/bibliotheken-veraendern-sich-grosse-haeuser-werden-attraktiver/1562484244000})
(bk)

Auch in den Niederlanden gäbe es einen Trend von Bibliothek als
\enquote{mehr als Medienausleihstation}, wie einer kurzen Meldung zur
Fusion der Bibliotheken von VANnU (Zundert, Roosendaal, Halderberge,
Moerdijk und Rucphen) und von Het Markiezaat (Bergen op Zoom,
Steenbergen und Woensdrecht) zum 01. Januar 2020. Stattdessen werden
Bibliotheken stärker in der Rolle als Bildungseinrichtungen, also
beispielsweise der Medienerziehung und Alphabetisierung, verstanden.
(o.A.: Bibliotheken VANnU en Het Markiezaat gaan in januari fuseren. In:
nu.nl, 10.07.2019,
\url{https://www.nu.nl/west-brabant/5963281/bibliotheken-vannu-en-het-markiezaat-gaan-in-januari-fuseren.html}))
(bk)

Ebenfalls in den Niederlanden werden Bibliotheken als Anlaufpunkte für
Informationen zur digitalen Verwaltung ausgebaut. In Venlo eröffnete im
Sommer zu diesem Zweck der erste Informatiepunt Digital Overheid.
Weitere folgen. Das Programm wird vom Regierungskabinett mit 7,5
Millionen Euro und von der Königlichen Bibliothek mit weiteren 5,5
Millionen Euro gefördert. Zielgruppen sind Teile der Bevölkerung mit
geringeren digitalen Kompetenzen. Für diese werden neben den
Informationsangeboten auch Weiterbildungen finanziert. (Peter
Wintermann: Bibliotheken krijgen loket voor digibeten. In: Het Parool /
parool.nl, 01. Juli 2019,
\url{https://www.parool.nl/nederland/bibliotheken-krijgen-loket-voor-digibeten/~bfb8d260})
(bk)

Der Bonner Generalanzeiger stellte die neue Leiterin der
Zentralbibliothek der Stadt Bonn, die gebürtige Dessauerin und
Wahl-Wuppertalerin Sylvia Gladrow, vor. Ein Schwerpunkt ihrer Arbeit
liegt auf der Entwicklung eines \enquote{Makerspaces}. Außerdem vertritt
sie die Position, dass Stadtbibliotheken in den Dialog zu aktuellen
gesellschaftlichen und politischen Fragen treten sollten. (Nicole
Garten-Dölle: Zentralbibliothek unter neuer Leitung. In:
\emph{General-Anzeiger}, 05. Juni 2019, S. 21,
\url{http://www.general-anzeiger-bonn.de/bonn/stadt-bonn/Bonner-Zentralbibliothek-hat-neue-Leiterin-article4119971.html})
(bk)

In Dormagen ist im Sommer und bis Oktober ein Bookbike unterwegs, das
eine Auswahl an Büchern sowie Bastel- und Malmaterialien an
verschiedenen Standorten der Stadt, unter anderem an Spielplätzen, in
einer Art Outreach-Programm für Kinder anbietet. (Red: Bookbike macht
auch am City-Beach Station. In: Westdeutsche Zeitung, Lokalteil
Rhein-Kreiss Neuss, 18.07.2019, S. 25) (bk)

Die Katholische Nachrichten Agentur (KNA) meldet, dass sowohl die
Katholische als auch die Evangelische Kirche in Nordrhein-Westfalen die
Sonntagsöffnung öffentlicher Bibliotheken unterstützt. Sie erhoffen sich
dabei auch eine Stärkung kirchlicher Bibliotheken. Öffentliche
Bibliotheken werden in der Meldung als \enquote{niederschwellige{[}r{]},
konsumfreie{[}r{]} und öffentliche{[}r{]} Begegnungs- und Kulturraum}
beschrieben. (KNA: Kirchen unterstützen Sonntagsöffnung öffentlicher
Bibliotheken. \emph{KNA Landesdienst} NRW, 28.06.2019) (bk)

Aus der FAZ erfahren wir zwei Fakten über die Bibliothek im Deutschen
Haus bei der UNO in New York: Erstens, dass die Bestände nicht allzu
üppig sind, was aber nur bedingt eine Rolle spielt (\enquote{Dass der
Bestand an Büchern in dem Raum sehr überschaubar ist, wird durch die
wache Präsenz des Gastgebers {[}i.e.~Christoph Heusgen, Ständiger
Vertreter der Bundesrepublik Deutschland bei den Vereinten Nationen{]}
kompensiert.}) Und zweitens, dass man von \enquote{der richtigen
Längsseite des Tisches in der Bibliothek} aus der Bibliothek über den
East River hinweg blicken und am anderen Ufer Brooklyn sehen kann.
(Ewald Hetrodt: Schwierige Gespräche in hohen Türmen. In:
\emph{Frankfurter Allgemeine Zeitung}, 10.07.2019, S. 29) (bk)

In Mainz planen dieser Tage Studierende des Fachbereichs Architektur an
der dortigen Hochschule ein Computer- und Medienzentrum. (Tim
Blumenstein: Auf ein Wort. Stefanie Freitag. Ist Fan der Stadt Mainz.
In: \emph{Frankfurter Allgemeine Zeitung, Rhein-Main-Zeitung},
09.07.2019, S. 32) (bk)

An der Bibliothek des Georgia Institute of Technology (Georgia Tech)
werden Drohnen für den Buchtransport eingesetzt. (John Tinnell: Op-Ed:
Are scooters a transit solution or a Trojan Horse for big tech to
colonize our public spaces? In: Los Angeles Times / latimes.com,
18.07.2019,
\url{https://www.latimes.com/opinion/story/2019-07-18/scooters-bird-uber-airbnb-tech-public-space?}
) (bk)

Die Schriftstellerin Katja Petrowskaja berichtet, dass sie einmal nach
dem Besuch offenbar der Buchhandlung Walther König in der Berliner
Burgstraße nach einer Art Fotobildbandüberforderung in der gleich
nebenan gelegene Bibliothek des Polnischen Instituts Zuflucht suchte und
sich dort denkbar sinnlich den Büchern näherte. \enquote{Es ähnelte
einem Traum.} Sie fand Czeslaw Milosz' \enquote{Die Straßen von Wilna}.
(Katja Petrowskaja: Was wir sehen. In: \emph{Frankfurter Allgemeine
Sonntagszeitung}, 07.07.2019, S. 34) (bk)

Der Karikaturist Plantu (Jean Plantureux) übergab der \emph{Bibliothèque
nationale de France (BnF)} einen Großteil seiner Zeichnungen. (o.A.:
lantu confie ses dessins de presse à la BnF. In: Liberation /
liberation.fr, 04.07.2019,
\url{https://www.liberation.fr/direct/element/plantu-confie-ses-dessins-de-presse-a-la-bnf_99791/}
) (bk)

Bereits im September 2018 berichtete die Lokalzeitung von Rockland
County, NY, The Journal News, dass eine Sammelaktion für das
Bibliothekstier der Katonah Village Library, eine Schildkröte namens
Tina the Turtle, 5000\$ für ein neues Becken zusammenbrachte. Bei der
Gelegenheit werden weitere Bibliothekstiere erwähnt: Katzen in der
Bibliothek von Blauvelt und ein Hamster namens Jasmine in Orangeburg.
Hinter dem Konzept der Bibliothekstiere steckt mehr als reine Sensation.
Ihr Ziel ist die Vermittlung von sozialen Kompetenzen wie Empathie und
der kommunikative Brückenbau zu schüchternen Kindern. (Julie Moran
Alterio: Hamsters, turtles, dogs and kittens: Does your library have a
resident pet? In: \emph{The Journal News / lohud.com}, 29.08.2018,
\url{https://eu.lohud.com/story/life/2018/08/29/animal-fans-raise-5-k-katonah-librarys-resident-turtle-tina/949387002/})
(bk)

Die Wiener Zeitung zitiert im Rahmen einer Besprechung von Massimo
Listris im Taschen Verlag erschienen Buch \enquote{Die schönsten
Bibliotheken der Welt} (siehe dazu auch die Kurzrezension in dieser
Kolumne) die bekannte, die Berliner Bibliothekswissenschaft lange
grundsätzlich prägende Definition \enquote{Die Bibliothek ist eine
Einrichtung, die unter archivarischen, ökonomischen und synoptischen
Gesichtspunkten publizierte Information für die Benutzer sammelt, ordnet
und verfügbar macht} aus dem Lehrbuch der Bibliotheksverwaltung von
Gisela Ewert und Walter Umstätter aus dem Jahr 1997, findet sie im Klang
sehr trocken und ergänzt als synonyme Beschreibung:
\enquote{Schatzkammer des Wissens}. (Christina Mondolfo: Paradies für
Bibliophile. In: \emph{Wiener Zeitung}, 16.11.2018, S. 4ff) (bk)

In der Ausgabe der Tageszeitung Freiheit vom 31.03.1983 findet sich auf
Seite 11 ein Bild des 79-jährigen Wissenschaftlers Prof.~Dr.~med.
Herbert Urban, der als ältester Leser der Universitätsbibliothek Leipzig
vorgestellt wird. Er arbeitet, so ist zu erfahren, an einer Trilogie
\enquote{Ethno- und Theo-Psychiatrie}. Der Katalog der DNB weist bis
heute kein Buch mit einem solchen Titel nach. (bk)

Das Medium Buch und seine mediale Logik stehen erwartungsgemäß im
Zentrum eines lesenswerten Berichts über den Berliner Verlag Brinkmann
\& Bose in der von hundert \#33 (September 2019). Die große Stärke des
Handmediums sieht die Mitinhaberin des Verlages, Rike Felka, darin, dass
es im Gegensatz zu digitalen Medien einen \enquote{medienkritischen
Diskurs} direkt \enquote{auf gestalterischer und inhaltlicher Ebene}
führbar macht. Dadurch wird das Buch selbst zu mehr als einem
Übertragungsmedium, nämlich konkret zu einem diskursiven Objekt. (Birgit
Szepanski: Über das Lesen und Verlegen von Büchern - Ein Besuch beim
Berliner Verlag Brinkmann \& Bose. In: von hundert \#33 / September
2019, S. 6--10) (bk)

\hypertarget{abschlussarbeiten}{%
\section{7. Abschlussarbeiten}\label{abschlussarbeiten}}

{[}Diesmal keine Hinweise.{]}

\hypertarget{weitere-medien}{%
\section{8. Weitere Medien}\label{weitere-medien}}

Die Carpentries-Familie umfasst drei Sektionen: Daten
\url{https://datacarpentry.org/}, Software
\url{https://software-carpentry.org/} und Bibliotheken
\url{https://librarycarpentry.org/}. In dem Webinar stellt Chris
Erdmann, Bibliothekar an der California Digital Library und Library
Carpentry Community and Development Director, das Format Library
Carpentry vor: Was ist der Hintergrund für diese Initiative? Wie
arbeitet die non-profit Organisation? Was ist Gegenstand in den
Workshops und wie werden diese organisiert? Warum sind die Workshops
relevant, insbesondere für Mitarbeiter*innen Wissenschaftlicher
Bibliotheken? Und wie können sich Interessierte selbst in der Community
einbringen? Ein aufschlussreicher Blick hinter die Kulissen... (Webinar:
The Carpentries: Teaching data science skills. YouTube, 7.2.2019.
\url{https://youtu.be/o4aZ5f74zGU}) (mv)

Die American Library Association (ALA) hat unter URL
\url{https://ebooksforall.org/} eine neue Protestwebseite gegen die
E-Book-Lizenzierungspraxis des Verlags Macmillan, der zur deutschen
Verlagsgruppe Georg von Holtzbrinck gehört, aufgesetzt. Macmillan ist
einer der sogenannten Big Five, also einer der fünf großen
Publikumsverlage der USA. Seit November 2019 kann jede amerikanische
Bibliothek während der ersten acht Wochen nach erscheinen eines
Macmillan-Buches nur ein einziges E-Book erwerben und verleihen --
völlig unabhängig davon, wie viele Nutzerinnen und Nutzer die Bibliothek
tatsächlich hat. Begründet wird dies vom CEO von Macmillan unter anderem
damit, dass die Onleihe der Bibliotheken die Verkaufszahlen
\enquote{kannibalisiere}
(\url{https://www.publishersweekly.com/binary-data/ARTICLE_ATTACHMENT/file/000/004/4222-1.pdf}).
Die Webseite bietet einen Überblick über die Sachlage sowie die Aktionen
der ALA gegen dieses Embargo und hat über eine Unterschriftensammlung
inzwischen mehr als 200.000 Stimmen gegen diese Praxis gesammelt.
(Webseite: \#ebooksForAll: Tell Macmillan Publishers that you demand
\#eBooksForAll, 2019, \url{https://ebooksforall.org/} (eb)

%autor

\end{document}
