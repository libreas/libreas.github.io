\begin{center}\rule{0.5\linewidth}{0.5pt}\end{center}

\textbf{Vanessa Hannesschläger} hat Germanistik an der Universität Wien
studiert. Zu ihren Interessensgebieten zählen unter anderem das Arbeiten
mit Archivmaterialien in digitalen Kontexten, Standardisierung und
Langzeitarchivierung im Bereich der Digital Humanities sowie
Rechtsfragen im digitalen Raum. Am ACDH leitet sie die Taskforce zum
Thema „legal issues" und ist an den Infrastrukturprojekten
\href{https://www.oeaw.ac.at/de/acdh/projects/clarin/}{CLARIN},
\href{https://www.oeaw.ac.at/de/acdh/projects/dariah-eu/}{DARIAH} und
\href{https://www.oeaw.ac.at/de/acdh/projects/dha-digital-humanities-austria/}{dha}
beteiligt. Vanessa war außerdem
\href{https://en.wikiversity.org/wiki/Wikimedia_Deutschland/Open_Science_Fellows_Program}{Open
Science Fellow} im gleichnamigen Programm von Wikimedia Deutschland,
Stifterverband und Volkswagen-Stiftung (Programmjahr 20117/18) und ist
Mitglied des Programmkommittees der Open-Access-Tage 2020.

\textbf{Maxi Kindling} (ORCiD:
\url{https://orcid.org/0000-0002-0167-0466}) ist Referentin im
Open-Access-Büro Berlin. Sie ist Mitbegründerin und -herausgeberin von
LIBREAS. Library Ideas.

\textbf{Michaela Voigt} (ORCiD:
\url{https://orcid.org/0000-0001-9486-3189}), Open-Access-Team der TU
Berlin, Redakteurin LIBREAS. Library Ideas.
