\documentclass[a4paper,
fontsize=11pt,
%headings=small,
oneside,
numbers=noperiodatend,
parskip=half-,
bibliography=totoc,
final
]{scrartcl}

\usepackage[babel]{csquotes}
\usepackage{synttree}
\usepackage{graphicx}
\setkeys{Gin}{width=.4\textwidth} %default pics size

\graphicspath{{./plots/}}
\usepackage[ngerman]{babel}
\usepackage[T1]{fontenc}
%\usepackage{amsmath}
\usepackage[utf8x]{inputenc}
\usepackage [hyphens]{url}
\usepackage{booktabs} 
\usepackage[left=2.4cm,right=2.4cm,top=2.3cm,bottom=2cm,includeheadfoot]{geometry}
\usepackage{eurosym}
\usepackage{multirow}
\usepackage[ngerman]{varioref}
\setcapindent{1em}
\renewcommand{\labelitemi}{--}
\usepackage{paralist}
\usepackage{pdfpages}
\usepackage{lscape}
\usepackage{float}
\usepackage{acronym}
\usepackage{eurosym}
\usepackage{longtable,lscape}
\usepackage{mathpazo}
\usepackage[normalem]{ulem} %emphasize weiterhin kursiv
\usepackage[flushmargin,ragged]{footmisc} % left align footnote
\usepackage{ccicons} 
\setcapindent{0pt} % no indentation in captions

%%%% fancy LIBREAS URL color 
\usepackage{xcolor}
\definecolor{libreas}{RGB}{112,0,0}

\usepackage{listings}

\urlstyle{same}  % don't use monospace font for urls

\usepackage[fleqn]{amsmath}

%adjust fontsize for part

\usepackage{sectsty}
\partfont{\large}

%Das BibTeX-Zeichen mit \BibTeX setzen:
\def\symbol#1{\char #1\relax}
\def\bsl{{\tt\symbol{'134}}}
\def\BibTeX{{\rm B\kern-.05em{\sc i\kern-.025em b}\kern-.08em
    T\kern-.1667em\lower.7ex\hbox{E}\kern-.125emX}}

\usepackage{fancyhdr}
\fancyhf{}
\pagestyle{fancyplain}
\fancyhead[R]{\thepage}

% make sure bookmarks are created eventough sections are not numbered!
% uncommend if sections are numbered (bookmarks created by default)
\makeatletter
\renewcommand\@seccntformat[1]{}
\makeatother

% typo setup
\clubpenalty = 10000
\widowpenalty = 10000
\displaywidowpenalty = 10000

\usepackage{hyperxmp}
\usepackage[colorlinks, linkcolor=black,citecolor=black, urlcolor=libreas,
breaklinks= true,bookmarks=true,bookmarksopen=true]{hyperref}
\usepackage{breakurl}

%meta
%meta

\fancyhead[L]{V. Hannesschläger, Redaktion LIBREAS \\ %author
LIBREAS. Library Ideas, 36 (2019). % journal, issue, volume.
\href{http://nbn-resolving.de/}
{}} % urn 
% recommended use
%\href{http://nbn-resolving.de/}{\color{black}{urn:nbn:de...}}
\fancyhead[R]{\thepage} %page number
\fancyfoot[L] {\ccLogo \ccAttribution\ \href{https://creativecommons.org/licenses/by/4.0/}{\color{black}Creative Commons BY 4.0}}  %licence
\fancyfoot[R] {ISSN: 1860-7950}

\title{\LARGE{Infrastrukturen, Kulturwandel und Nachhaltigkeit – eine Open Policy für das Austrian Centre for Digital Humanities der Österreichischen Akademie der Wissenschaften}} % title
\author{Vanessa Hannesschläger, Redaktion LIBREAS} % author

\setcounter{page}{1}

\hypersetup{%
      pdftitle={Infrastrukturen, Kulturwandel und Nachhaltigkeit – eine Open Policy für das Austrian Centre for Digital Humanities der Österreichischen Akademie der Wissenschaften},
      pdfauthor={Vanessa Hannesschläger, Redaktion LIBREAS},
      pdfcopyright={CC BY 4.0 International},
      pdfsubject={LIBREAS. Library Ideas, 36 (2019).},
      pdfkeywords={Digital Humanities, Nachhaltigkeit, Interview, ACDH, Österreich, Austrian Centre for Digital Humanities, Österreichische Akademie der Wissenschaften, Openness, Policy, Open Policy},
      pdflicenseurl={https://creativecommons.org/licenses/by/4.0/},
      pdfcontacturl={http://libreas.eu},
      baseurl={http://libreas.eu},
      pdflang={de},
      pdfmetalang={de}
     }



\date{}
\begin{document}

\maketitle
\thispagestyle{fancyplain} 

%abstracts

%body
Vanessa Hannesschläger stellte bei den Open-Access-Tagen 2019 in
Hannover die Open Policy des Austrian Centre for Digital Humanities der
Österreichischen Akademie der Wissenschaften vor.\footnote{Vanessa
  Hannesschläger. (2019). Nachhaltigkeit durch Institutionalisierung:
  Die Open Policy des Austrian Centre for Digital Humanities der
  Österreichischen Akademie der Wissenschaften. Open-Access-Tage 2019.
  \url{https://doi.org/10.5281/zenodo.3465505}.} Sie thematisierte den
Entstehungsprozess der Richtlinien (bottom-up Prozess) und die Grenzen,
die der Institutionalisierung von Openness an einem derartigen Institut
gesetzt sind. Interviewt wurde sie vor Ort von Maxi Kindling und
Michaela Voigt.

\emph{LIBREAS: Kannst du deine Institution kurz vorstellen? Welche Rolle
spielt sie innerhalb Österreichs?}

VH: Ich bin für das Austrian Centre for Digital Humanities der
Österreichischen Akademie der Wissenschaften (ACDH-OeAW) tätig. Den
Namen des Instituts muss man tatsächlich so umfänglich nennen, denn der
Name \enquote{Austrian Centre for Digital Humanities} würde glauben lassen,
dass es \emph{ein} österreichisches Zentrum für digitale
Geisteswissenschaften gibt. Dem ist aber nicht so. Es gibt vielmehr drei
Zentren: Das sind zum einen wir an der Akademie der Wissenschaften und
zum anderen unser Schwesterinstitut in Graz -- das Zentrum für
Informationsmodellierung -- sowie zum dritten ein Zentrum an der
Universität Wien. Diese drei formieren sich gemeinsam mit weiteren
Institutionen zum Konsortium Digital Humanities Austria, welches CLARIN
ERIC und DARIAH-EU in Österreich vertritt.

Unserem Institut, also dem ACDH-OeAW, obliegt die Koordination der
Aktivitäten von CLARIN und DARIAH in Österreich, denn einer unser
Direktor*innen ist in Personalunion auch der nationale Koordinator
beider Projekte. Wir unterstützen die geisteswissenschaftlichen
Institute an der Akademie bei der Transformation zum digitalen
Paradigma, vor allem bei der Durchführung digitaler
geisteswissenschaftlicher Forschungsprojekte. Die Forschungsinstitute
entwickeln Fragestellungen und Projektideen; wir entwickeln mit ihnen
gemeinsam deren digitale Umsetzung. Konkret kümmern wir uns um die
Datenmodellierung, die Programmierung, die Erstellung von
Datenmanagementplänen, die Datenarchivierung,
Langzeitverfügbarkeit\ldots{} Wir betreiben auch das Repository
ARCHE\footnote{A Resource Centre for the HumanitiEs (ARCHE), siehe
  \url{https://arche.acdh.oeaw.ac.at/}.} -- ein Repositorium für
geisteswissenschaftliche Forschungsdaten in Österreich. Publikationen
hingegen werden auf Zenodo oder auf dem Repository epub\footnote{epub.oeaw,
  siehe \url{http://epub.oeaw.ac.at/}.} gehostet, welches der Verlag der
Akademie betreibt. Wir sind außerdem sehr stark im Soft-Skill-Bereich
der Infrastrukturarbeit involviert. Aktivitäten, die landläufig unter
\enquote{outreach} fallen, wie zum Beispiel Trainings oder Workshops, in denen
Geisteswissenschaftler*innen bestimmte Werkzeuge und Methoden der
digitalen Geisteswissenschaften kennen- und benutzen lernen können. Dazu
gehören aber auch eher Keynote-artige Vorträge, die ACDH
Lectures\footnote{Siehe auch
  \url{https://www.oeaw.ac.at/acdh/events/event-series/acdh-lectures/}.},
die zur digitalen Arbeit im Allgemeinen inspirieren sollen. Das
Schlagwort lautet hier kultureller Wandel; die Inspiration zum
kulturellen Wandel gehört sozusagen zu unseren Aufgaben.

\emph{LIBREAS: Kurz zu CLARIN und DARIAH -- ihr wirkt also in die
Einrichtung, in die Akademie und die Institute. Wie ist dabei die
Verbindung zu den großen Forschungsinfrastrukturprojekten?}

VH: Wir sind die Koordinationsstelle für die österreichischen
Manifestationen dieser Forschungsinfrastrukturen, die es ja immer auf
zwei Ebenen gibt -- als das EU-Projekt und in den nationalen Gremien. So
ist es bei uns auch. Zum einen gibt es die innerösterreichischen
Projekte, in denen Infrastruktur im Kleineren aufgebaut werden soll. Zum
anderen gibt es natürlich Mitwirkung an den größeren, von den
europäischen Infrastrukturen selbst initiierten Projekten. Für CLARIN
könnte man zum Beispiel das Virtual Language Observatory\footnote{CLARIN
  Virtual Language Observatory (VLO), siehe
  \url{https://vlo.clarin.eu/}.} nennen, bei dem wir stark im Bereich
Datenkuratierung involviert sind. Für DARIAH könnte man die
Arbeitsgruppe ELDAH\footnote{Ethics and Legality in the Digital Arts and
  Humanities (ELDAH), siehe
  \url{https://www.dariah.eu/activities/working-groups/ethics-and-legality-in-the-digital-arts-and-humanities-eldah/}.}
nennen, die sich mit \enquote{ethics and legality in arts and humanities}
auseinandersetzt.

\emph{LIBREAS: Und der Content von kleineren Projekten in Österreich
wird in die großen Infrastrukturen eingebunden?}

VH: Ja, genau. Natürlich nicht auf Punkt und Komma. Nicht jede
Kleinigkeit, die für Österreich relevant ist, wird in die europäische
Infrastrukturen überführt. Aber grundsätzlich werden die Inhalte
weitergegeben und zentral gesammelt.

\emph{LIBREAS: Einiges wurde schon kurz genannt. Kannst du etwas
detaillierter ausführen, welche Art von Forschungsinfrastruktur am
ACDH-OeAW betrieben beziehungsweise benutzt wird?}

VH: Wie schon erwähnt, es gibt das Repositorium ARCHE\footnote{A
  Resource Centre for the HumanitiEs (ARCHE), siehe
  \url{https://arche.acdh.oeaw.ac.at/}.} für Forschungsdaten. Das macht
einen großen Teil aus. Daneben gibt es einen inzwischen recht gut
ausgebauten Technologie-Stack für geisteswissenschaftliche Arbeit, die
vor allem textfokussiert ist. Das umfasst sowohl Datenbanken für Texte
und TEI-kodierte XML-Daten. Andererseits arbeiten wir sehr viel mit
kontrollierten Vokabularen. Wir hosten SKOS-Vokabulare, die sogenannten
ACDH Vocabs.\footnote{ACDH Vocabs, siehe
  \url{https://www.oeaw.ac.at/acdh/tools/acdh-vocabularies/}.} Diese
werden in den verschiedenen Projekten verwendet. Ein weiteres
herausragendes Angebot ist APIS, ein prosopographisches
Informationssystem.\footnote{Austrian Prosopographical Information
  System, siehe \url{https://www.oeaw.ac.at/acdh/projects/apis/}.} APIS
ist ursprünglich aus einer Kooperation mit dem Österreichischen
Biographischen Lexikon entstanden, das auch digital werden sollte. Hier
werden vor allem Personen aber auch Institutionen erfasst und mit
vorhandenen Normdatensätzen verknüpft. APIS ist inzwischen sehr
elaboriert und man kann damit quasi die ganze Welt abbilden, was es
besonders für historische Projekte spannend macht. Das sind drei der
großen Komponenten in unserem Technologie-Stack, die meist
zusammenspielen und auf die die verschiedenen Projekte zurückgreifen
können.

Technische Infrastruktur ist das Eine. Unser Institut legt viel Wert
darauf zu betonen, dass Infrastruktur nicht nur aus den technischen
Ebenen besteht, in denen digitale Objekte gespeichert werden können.
Dazu gehört natürlich auch die kulturelle Transformation, die eine neue
Art des Arbeitens, eine neue Art der kollaborativen und offenen Kultur
ermöglicht. Deswegen gehören Vermittlung, Outreach und Advocacy ebenso
sehr zu unseren Kernaufgaben.

\emph{LIBREAS: Welche Rolle spielt Nachhaltigkeit für diese Komponenten
und in der Institution insgesamt?}

VH: Das ist eine große Frage. Schauen wir uns zuerst die soziale Ebene
an. Wir streben vor allem an, die Forschenden zum Umdenken zu
inspirieren. Man kann natürlich die Wissenschaftler*innen nicht einfach
\enquote{flippen} -- wie im Open-Access-Bereich die Zeitschriften. Einen
Denkprozess anzustoßen dauert lange; es ist mühsam. Und es gibt immer
Personen, die man nicht erreicht. Aber denen, die wir erreichen, ist
klar, dass der Wandel unwiderruflich ist. Quasi ein Schneeballeffekt.
Wer einmal verstanden hat, wie sich diese Veränderung vollzieht, hört
nicht einfach wieder auf und kehrt zum ursprünglichen Arbeiten zurück.
Damit befinden wir uns im Soft-Skill-Bereich, den wir gerade schon
angesprochen haben.

Bei der Ebene der Daten und der Werkzeuge stehen wir vor einer enormen
Herausforderung. Das Repositorium ist der erste Versuch. Es heißt nicht
ohne Grund ARCHE, denn es soll wirklich eine sein: Es geht tatsächlich
darum, die Daten zu \enquote{retten} und so lange wie möglich am Leben zu
halten. Wobei selbst in der Open Policy des ACDH bereits steht, dass wir
uns als Institut und die hier beschäftigten Wissenschaftler*innen sich
dazu verpflichten, die Daten nachhaltig zu sichern, solange das
technologisch sinnvoll möglich ist.\footnote{Zu Vorgaben der Policy zum
  Umgang mit Forschungsdaten und Software (Ablage, Lizenz und
  Aufbewahrungsdauer) siehe Folie 14 bei Hannesschläger (2019)
  \url{https://doi.org/10.5281/zenodo.3465505}: \enquote{Forschungsdaten und
  Aufzeichnungen sind so lange aufzubewahren und zugänglich zu halten,
  wie ohne Verlust des Informationsgehalts der Daten (z.B. durch häufige
  Formatmigrationen) möglich ist. Mindestens aufzubewahren sind sie so
  lange, wie es nach einschlägigen gesetzlichen oder vertraglichen
  Vorschriften, insbesondere nach einer Vorgabe des Drittmittelgebers
  erforderlich ist.}} Die Daten sollen transformiert werden, solange
dies ohne Verlust des Informationsgehalts machbar ist. Es wird also
schon mitgedacht, dass die Daten realistisch betrachtet in 50 Jahren
kaputt sein könnten. Wobei 50 Jahre schon utopisch sind. Die digitale
Langzeitarchivierung ist bekanntlich noch ein ungelöstes Problem. Selbst
die Sicherung in Form von Screenshots, die dann als TIFF oder JPEG
abgelegt werden, wäre nur die absolut minimale Verlustvariante, um im
Internet zugängliche Materialien zu sichern. Und selbst das ist kein
Versprechen, denn auch diese Formate können kaputt gehen. Man müsste die
Materialien schlussendlich doch wieder ausdrucken und in den Schrank
legen\ldots{} Man muss also realistisch bleiben. Wir versuchen so weit
wie möglich vorauszudenken, aber wir haben, wie alle anderen auch, keine
wirklich befriedigenden Antworten hinsichtlich der digitalen
Langzeitarchivierung.

\emph{LIBREAS: Wie ist das Finanzierungsmodell für die
Forschungsinfrastrukturen am ACDH-OeAW?}

VH: Das Repository ARCHE etwa ist in einem ersten Schritt in einem
Projektkontext aufgebaut worden -- einerseits mit Mitteln aus der
Nationalstiftung und andererseits mit eigenen Mitteln aus dem
Institutsbudget. So finanziert sich der Aufbau und aktuell auch der
Betrieb des Repositoriums. Die Mittel dafür sind im Moment noch
vergleichsweise luxuriös und erlauben auch Experimente. Die Frage der
dauerhaften Finanzierung ist allerdings noch nicht abschließend geklärt.

Bei den anderen Infrastrukturen ist die Lage ähnlich, etwa auch bei
CLARIN und DARIAH. Das klingt erst einmal nicht gut. Aber man muss das
im Kontext sehen: Eine Regierung wird auch nur für fünf Jahre gewählt
und kann keine Versprechen darüber abgeben, was in zehn Jahren
finanziert wird. Niemand kann versprechen, dass in 100 Jahren noch die
gleiche Summe in die gleichen Bereiche fließt. Dinge verändern sich
kontinuierlich, die Aufgaben und Anforderungen wandeln sich.

\emph{LIBREAS: Die langfristige Finanzierung für eine konkrete
Infrastruktur wie das Repositorium eines bestimmten Instituts zu klären
ist natürlich etwas leichter als für große, auch internationale
Infrastrukturen. Zum Beispiel, dass sich das Institut etwa verpflichtet,
langfristig Personalmittel für den Betrieb und auch die
Weiterentwicklung bereitzustellen.}

VH: Genau, ein Repositorium muss man eher denken wie eine Bibliothek. Es
gehört zur Institution und Punkt.

\emph{LIBREAS: Welche Rolle spielt Kollaboration mit anderen
Einrichtungen für die Nachhaltigkeit der Forschungsinfrastrukturen?}

VH: Es gibt das Digital Humanities Austria Konsortium\footnote{dha -
  digital humanities austria, siehe
  \url{http://www.digital-humanities.at/}.}, dem sich immer wieder neue
Partnerinstitutionen anschließen. Dieses Konsortium entspricht in etwa
den österreichischen Sparten von CLARIN und DARIAH. In diesem Konsortium
kommen die Arbeiten der verschiedenen Partner zusammen. Es ist eine Art
Operationalisierung der Kollaboration, die zwischen verschiedenen
Institutionen bereits stattfand -- weil es inhaltliche Vorteile bietet.
Das Konsortium erleichtert die Zusammenarbeit, weil es als Struktur eine
gewisse Repräsentativität birgt. Es erleichtert die Durchsetzung
bestimmter Entscheidungen und auch die Mittelbeantragung.

\emph{LIBREAS: In Deutschland gibt es im Bereich Digital Humanities eine
viel kleinteiligere Landschaft, in der auch mehr Konkurrenz stattfindet.
Hier fehlt ein solcher nationaler Rahmen.}

VH: Konkurrenz gibt es in Österreich natürlich in gewissem Umfang auch.

\emph{LIBREAS: Vom Konsortium abgesehen, welche Rolle spielt
Kollaboration? Stellt Kollaboration einen Wert an sich dar um
Nachhaltigkeit zu sichern? Oder ist das ein Fehlschluss?}

VH: Eine gute Frage. Aus dem Bauch heraus würde ich sagen, dass
Kollaboration prinzipiell förderlich ist -- für die Entwicklung ebenso
wie für die Absicherung der Nachhaltigkeit. Aber richtig
argumentieren\ldots{} Das ist eher ein Gefühl. Kollaboration ist gut für
den Forschungsinhalt -- seien es Daten, seien es Ergebnisse oder
Gedanken. Es ist gut für einen Forschungsinhalt, wenn sich mehr als
Eine*r dafür interessiert. Das ist nicht unbedingt eine Eigenheit der
Wissenschaft, sondern gilt generell: Wenn sich mehrere Köpfe zusammen
tun, bekommen Dinge eine andere Dynamik und die Chance ist höher, dass
noch jemand eine Idee einbringt. Insofern unterstützt Kollaboration die
Nachhaltigkeit.

\emph{LIBREAS: Die Frage ist zugegeben auch etwas schwammig. Konsortiale
Strukturen bergen ja auch gewisse Nachteile, und es gibt auch Formen der
Kollaboration, die nicht institutionalisiert sind...}

VH: Ich glaube, dass formale Strukturen im schlechtesten Fall sogar
Nachhaltigkeit aktiv verhindern können. Mitunter arbeiten formalisierte
Gruppen auch gegen sich selbst. Bei informeller Zusammenarbeit, die sich
aus inhaltlichen Interessen und Gemeinsamkeiten speist, ist es
umgekehrt. Eher stoppt die Kollaboration als das Projekt, wenn es
Uneinigkeiten gibt.

\emph{LIBREAS: Wir würden gern noch über freie Lizenzen sprechen. Welche
Rolle spielen sie und andere Kriterien offener Wissenschaft -- sowohl
für die Infrastrukturen, über die wir sprachen, als auch für die
Policy?}

VH: Die Policy besagt, dass freie Lizenzen prinzipiell gewünscht sind.
Sie sollen grundsätzlich genutzt werden und sollen so \emph{frei} sein,
wie es in dem konkreten Zusammenhang und auf Grundlage der jeweiligen
rechtlichen Gegebenheiten möglich ist. Auch das Digital Humanities
Austria Konsortium bekennt sich zu freien Lizenzen und zu offener
Wissenschaft, wenn auch die Nutzung freier Lizenz nicht ganz so klar,
wie in der Policy des ACDH-OeAW, angemahnt wird. Es wird immer dann
kompliziert, wenn die Forschenden selbst ins Spiel kommen. \enquote{Geistiges
Eigentum} ist ein schwieriges Konzept. Das kann das Arbeiten
erschweren. Das weiß jede*r, der/die programmiert. Wir brauchen
Offenheit und freie Lizenzen, sonst kann kein sinnvoller Code entstehen
-- gerade wenn man nicht für einen großen Konzern arbeitet. Wenn es um
die digitalen und technologischen Komponenten geht, ist die Antwort ganz
klar: Sie sind nicht sinnvoll umsetzbar, wenn man nicht mit offenen
Materialien arbeitet. Für die technologischen Grundkomponenten sind
offene Lizenzen also eine Grundvoraussetzung. Und das wird in der Praxis
auch so umgesetzt. Im Code selbst ist die Angabe der Lizenz eine Zeile.
Die Programmierer*innen stellen das auch gar nicht in Frage; sie
verstehen häufig die Aufregung nicht.

\emph{LIBREAS: Warum ist die Situation mit Blick auf die
Wissenschaftler*innen schwieriger? Woran liegt das?}

VH: Es ist wohl ein althergebrachtes Verständnis von Wissenschaft. Und
in gewisser Hinsicht ist das auch nachvollziehbar -- es geht um unsere
Karrieren und um Anerkennung für unsere Arbeit. Ich selbst bin
Wissenschaftlerin und keine Programmiererin. Ich kann das schon
nachvollziehen. Ein Beispiel: Digitale Edition haben eine
Herausgeber*innenschaft und die Herausgeber*innen werden auch benannt.
An einer digitalen Edition sind aber viele Personen beteiligt. Die
wissenschaftlichen Herausgeber*innen sind in der Regel die
Projektleiter*innen. Dass die digitale Edition entsteht, ist aber
letztendlich auch der Verdienst der Programmierer*innen und
Datenkurator*innen -- die jedoch nicht als Herausgeber*innen gelistet
werden. Aber welche Rolle haben diese beiden Gruppen? Und an welcher
Stelle werden sie in einer digitalen Edition aufgeführt? Wie früher bei
Büchern ist es wichtig, dass der Name an der richtigen Stelle steht.
Dabei wird aber nicht berücksichtigt, dass eine digitale Edition ganz
anders entsteht als ein Buch.

\emph{LIBREAS: Namensnennung als solches ist kein unlösbares Problem.
Diese Thematik verhindert doch nicht, dass ein Werk zum Beispiel unter
einer CC-BY-Lizenz verbreitet wird. Es klingt so, als ob die Kontroverse
nicht ist,} dass \emph{die digitale Edition unter freier Lizenz
erscheint -- sondern vielmehr wer an welcher Stelle aufgeführt wird.
Stimmt das?}

VH: Ja, durchaus. Lizenzierung als solches ist einzelnen
Wissenschaftler*innen mitunter fremd. Wenn man ihnen die Grundthematik
erläutert, ist es in der Regel nicht schwierig, sie von den Vorteilen
einer offenen Lizenzierung zu überzeugen. Ein Problem für die
Lizenzierung ergibt sich häufig eher durch die Materialien, mit denen
die Wissenschaftler*innen arbeiten. Wenn diese noch Urheberrechtsschutz
genießen, dann stehen die Rechte der Urheber*innen beziehungsweise der
Rechteinhaber*innen einer freien Lizenzierung der digitalen Edition
entgegen. Digitale Editionen sind eine besondere Kategorie. Ab wann ist
bei einer digitalen Edition die Schöpfungshöhe erreicht? Welche Rechte
hat der/die Arbeitgeber*in? Und können auch Metadaten urheberrechtlich
geschützt sein? Die Abgrenzung ist nicht trivial, weder im analogen noch
im digitalen Bereich.

\emph{LIBREAS: Lass uns noch einen Blick in die Zukunft werfen. Was
kommt nach der Policy?}

VH: \emph{(lacht)} Die richtige Arbeit beginnt eigentlich mit der
Verabschiedung der Policy. Die Implementierung einer so weitgreifenden
Richtlinie ist ein langer Weg durch die Instanzen, der mit der
Verabschiedung der Policy durch das Direktor*innengremium des Instituts
im Februar 2019 erst begonnen hat. Aber es gilt nun, die Policy-Inhalte
im Alltag umzusetzen. Die Policy sieht keine Sanktionierung bei
Nichteinhaltung vor. Sie soll vor allem motivieren. Aber eine positive
Wirkung entfalten kann sie nur, wenn sie auch bekannt ist. Der nächste
Schritt wird also sein, die Policy im ACDH-OeAW bekannt zu machen und
der/den Einzelnen zu vermitteln, welche Vorteile die Einhaltung der
einzelnen Punkte birgt.

\emph{LIBREAS: Wir danken dir herzlich für die vielen Anregungen und das
Gespräch!}

%autor
\begin{center}\rule{0.5\linewidth}{0.5pt}\end{center}

\textbf{Vanessa Hannesschläger} hat Germanistik an der Universität Wien
studiert. Zu ihren Interessensgebieten zählen unter anderem das Arbeiten
mit Archivmaterialien in digitalen Kontexten, Standardisierung und
Langzeitarchivierung im Bereich der Digital Humanities sowie
Rechtsfragen im digitalen Raum. Am ACDH leitet sie die Taskforce zum
Thema „legal issues" und ist an den Infrastrukturprojekten
\href{https://www.oeaw.ac.at/de/acdh/projects/clarin/}{CLARIN},
\href{https://www.oeaw.ac.at/de/acdh/projects/dariah-eu/}{DARIAH} und
\href{https://www.oeaw.ac.at/de/acdh/projects/dha-digital-humanities-austria/}{dha}
beteiligt. Vanessa war außerdem
\href{https://en.wikiversity.org/wiki/Wikimedia_Deutschland/Open_Science_Fellows_Program}{Open
Science Fellow} im gleichnamigen Programm von Wikimedia Deutschland,
Stifterverband und Volkswagen-Stiftung (Programmjahr 20117/18) und ist
Mitglied des Programmkommittees der Open-Access-Tage 2020.

\textbf{Maxi Kindling} (ORCiD:
\url{https://orcid.org/0000-0002-0167-0466}) ist Referentin im
Open-Access-Büro Berlin. Sie ist Mitbegründerin und -herausgeberin von
LIBREAS. Library Ideas.

\textbf{Michaela Voigt} (ORCiD:
\url{https://orcid.org/0000-0001-9486-3189}), Open-Access-Team der TU
Berlin, Redakteurin LIBREAS. Library Ideas.

\end{document}
