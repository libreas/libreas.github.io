\documentclass[a4paper,
fontsize=11pt,
%headings=small,
oneside,
numbers=noperiodatend,
parskip=half-,
bibliography=totoc,
final
]{scrartcl}

\usepackage{synttree}
\usepackage{graphicx}
\setkeys{Gin}{width=.6\textwidth} %default pics size

\graphicspath{{./plots/}}
\usepackage[ngerman]{babel}
%\usepackage{amsmath}
\usepackage[utf8x]{inputenc}
\usepackage [hyphens]{url}
\usepackage{booktabs} 
\usepackage[left=2.4cm,right=2.4cm,top=2.3cm,bottom=2cm,includeheadfoot]{geometry}
\usepackage{eurosym}
\usepackage{multirow}
\usepackage[ngerman]{varioref}
\setcapindent{1em}
\renewcommand{\labelitemi}{--}
\usepackage{paralist}
\usepackage{pdfpages}
\usepackage{lscape}
\usepackage{float}
\usepackage{acronym}
\usepackage{eurosym}
\usepackage[babel]{csquotes}
\usepackage{longtable,lscape}
\usepackage{mathpazo}
\usepackage[flushmargin,ragged]{footmisc} % left align footnote

\usepackage{listings}

\urlstyle{same}  % don't use monospace font for urls

\usepackage[fleqn]{amsmath}

%adjust fontsize for part

\usepackage{sectsty}
\partfont{\large}

%Das BibTeX-Zeichen mit \BibTeX setzen:
\def\symbol#1{\char #1\relax}
\def\bsl{{\tt\symbol{'134}}}
\def\BibTeX{{\rm B\kern-.05em{\sc i\kern-.025em b}\kern-.08em
    T\kern-.1667em\lower.7ex\hbox{E}\kern-.125emX}}

\usepackage{fancyhdr}
\fancyhf{}
\pagestyle{fancyplain}
\fancyhead[R]{\thepage}

%meta
%meta

\fancyhead[L]{M. Metzendorf, A. Kellersohn \\ %author
LIBREAS. Library Ideas, 25 (2014). % journal, issue, volume.
} % urn
\fancyhead[R]{\thepage} %page number
\fancyfoot[L] {\textit{Creative Commons BY 3.0}} %licence
\fancyfoot[R] {\textit{ISSN: 1860-7950}}

\title{\LARGE{Frauen in bibliothekarischen Führungspositionen – Ein Gespräch im Mai 2014}} %title %title
\author{Maria-Inti Metzendorf \& Antje Kellersohn} %author

%\setcounter{page}{}

\usepackage[colorlinks, linkcolor=black,citecolor=black, urlcolor=blue,
breaklinks= true]{hyperref}

\date{}
\begin{document}

\maketitle
\thispagestyle{fancyplain} 

%abstracts

%body
\begin{quote}
\emph{Version vom 08. August 2014. Bitte beachten Sie folgenden Hinweis:
\href{https://libreas.wordpress.com/2014/08/09/interview_antje_kellersohn/}{In
eigener Sache: De- und Republikation des Gesprächs mit Antje Kellersohn
(LIBREAS \#25)}}
\end{quote}

\textbf{Frau Kellersohn, wir haben uns vor etwa 12 Jahren kennengelernt,
als ich mitten im Studium steckte und Sie an der Fachhochschule
Darmstadt im Studiengang Informations- und Wissensmanagement als
Gastdozentin das Seminar \enquote{Bibliotheksmanagement} anboten. Dieses
ist mir sehr positiv in Erinnerung geblieben, da es von Ihnen
abwechslungsreich und praxisnah gestaltet wurde. Sie stellten sich
damals als promovierte Chemikerin vor, die Leiterin der Bielefelder
Fachhochschulbibliothek war. Seit dem 1. Oktober 2008 sind sie nun
Leiterin der Universitätsbibliothek Freiburg.}

Ich habe relativ kurz entschlossen mit meiner Familie 2008 den Wechsel
nach Freiburg angetreten. Das war nicht von langer Hand vorbereitet und
ich war nicht aktiv auf Stellensuche;~ aber als ich von der Universität
Freiburg angesprochen wurde, habe ich mich beworben -- und eine Zusage
bekommen.~

\textbf{Mit Familie.}

Ja, das war mir auch wichtig. Ich habe von Anfang an klar gemacht, dass
es mich nicht alleine gibt. Ich hatte schon einmal jahrelang eine
Wochenendbeziehung mit meinem Mann gepflegt. Die Uni Freiburg mit ihrem
Dual Career Service hat meine Familie und mich damals sehr unterstützt.
Da konnten beispielsweise Kontakte zum Schulamt hergestellt werden --
mein Mann ist im Schuldienst tätig -- sodass die Versetzung nach
Freiburg reibungslos vonstatten ging. Auch bei der Suche einer
Kinderbetreuung für unsere Tochter war man uns behilflich. Das war für
uns eine sehr positive Erfahrung. Wenn man spürt, dass dem neuen
Arbeitgeber dieses Thema am Herzen liegt und er Unterstützung bietet,
dann wird vieles leichter. In den vergangenen Jahren haben wir immer
wieder neue Kolleginnen und Kollegen der Universitätsbibliothek vor
ihrem Wechsel nach Freiburg an den Dual Career Service vermittelt, die
dort ebenfalls Hilfestellung bekommen haben. Dual Career ist in
Deutschland, ganz besonders im Wissenschaftsbereich immer noch nicht
allzu weit verbreitet, in den USA und vielen anderen Ländern ist das
hingegen seit Jahrzehnten gang und gäbe.~

\textbf{Man sieht den Menschen also nicht nur in seiner Funktion als
Arbeitskraft, sondern auch in seinem Umfeld.~}

Genau. Und man weiß, wenn man ihm den Weg in das Unternehmen oder in die
Hochschule ebnet, dann arbeitet er produktiver, motivierter und er
bleibt auch länger da.~

\textbf{Wie kamen Sie eigentlich dazu als Chemikerin ins
Bibliothekswesen zu wechseln? Was hat Sie an dem Ort Bibliothek
gereizt?~}

Das ist die klassische Frage. Ich frage mich immer, ob sie mir auch
gestellt würde, wenn ich beispielsweise Historikerin wäre. In meinem
Beisein ist diese Frage jedenfalls noch nie einem Geisteswissenschaftler
gestellt worden\ldots{}~

Die Antwort ist eigentlich ganz einfach: Ich habe schon immer viel in
Bibliotheken gearbeitet. Schon als Kind habe ich in „meiner``
Stadtbibliothek quasi lesen gelernt und dort viel Zeit verbracht. In der
Schulzeit, ab der 8. oder 9. Klasse, war ich dann auch regelmäßige
Nutzerin der Universitätsbibliothek vor Ort, die mich wohl ein stückweit
ins Studium gebracht hat. Was man heute so unter dem Stichwort Übergang
Schule -- Studium propagiert, habe ich damals schon erlebt. Denn ich
habe die Bibliotheken als sehr offene und mir zugewandte Orte erlebt, in
denen ich stets willkommen war. Das war für mich ein nahezu grenzenloser
Ort des Eintauchens und des Entdeckens. Keiner hat mir vorgeschrieben,
welche Bücher ich aus dem Regal ziehen darf. Das war einfach großartig!~

\textbf{Sozusagen eine Oase.~}

Ja, eine Oase! Bibliotheksarbeit war auch in unserem Studium fest
verankert. Es wurde erwartet. dass man abends nach den
Lehrveranstaltungen, dem Praktikum und später nach der Laborarbeit in
die Bibliothek ging: Um Zeitschriftenartikel zu lesen, sich auf dem
Laufenden zu halten, in den Chemical Abstracts Recherchen zu machen,
„den Beilstein`` und „den Gmelin`` regelmäßig durchzuackern\ldots{} Das
ließ sich damals noch nicht online machen und so habe ich jede Woche
einige Abende in der Bibliothek verbracht. Wohl anders, als viele
Menschen das von Naturwissenschaftlern erwarten würden. Und als ich dann
mit meiner Dissertation begann, habe ich mich in ein völlig neues
Arbeitsfeld einarbeiten, oder besser gesagt einlesen müssen. Ich habe in
meinem Promotionsprojekt nanokristallines Kupfer hergestellt und mit
diesem viele Untersuchungen durchgeführt. Über die dafür nötigen
Apparaturen fand man in den frühen 90ern noch kaum Informationen in der
gängigen Fachliteratur. Ich war deshalb auf japanische Patentschriften
angewiesen (!). Wenn Sie jetzt mal denken, das alles vor 20
Jahren\ldots{}~

\textbf{Ich verstehe.~}

Ja, ich habe das Personal mit meinen exotischen Bestellwünschen damals
wahrscheinlich sehr in Beschlag genommen. Dennoch war die Bibliothek für
mich wieder einmal ein Ort, an dem mir geholfen wurde. Wo mir Wissen und
Information zur Verfügung gestellt wurde, das für mich in keiner anderen
Weise verfügbar war, das mir geholfen und mir wahrscheinlich viele
Fehlschläge erspart hat. Da bin ich dann endgültig neugierig geworden
und habe mir überlegt: Wer arbeitet da eigentlich? Welche
Qualifikationen haben diese Leute? Von dem für die Chemie zuständigen
Fachreferenten erfuhr ich, dass man ein Referendariat machen kann. Das
Auswahlverfahren sei hart, weil man in Konkurrenz mit anderen
Fachdisziplinen stehe. Ich solle es aber einfach ausprobieren, auch wenn
ich meine Promotion noch nicht abgeschlossen habe. Im ersten Anlauf
klappe das ohnehin nie. Aber es hat dann doch geklappt und ich bekam
1993 einen Referendariatsplatz in Heidelberg zugewiesen. Meine
experimentellen Arbeiten im Labor waren glücklicherweise kurz vorher
abgeschlossen, so bin ich an die Bergstraße gezogen und habe acht Monate
später meine Doktorarbeit abgegeben. Ich haben meinen Schritt in die
Bibliothekslaufbahn nie bereut.~

Damals hat mich auch ein Bauchgefühl umgetrieben: Da tut sich was im
IT-Bereich! Es gab ja damals noch kein Internet, wie wir es heute
kennen. Aber durch meine Forschung in einem stark IT-affinen Umfeld --
in der Physikalischen Chemie haben wir nicht nur auf Großrechnern wie
z.B. der Cray \emph{(einem Supercomputer)} gerechnet, sondern damals
auch schon ganz selbstverständlich E-Mails verschickt oder FTP-Dienste
genutzt -- heutzutage ganz selbstverständliche und für jedermann
nutzbare Features.

Schon damals habe ich gedacht, das muss reizvoll sein -- wenn man im
Informationsbereich tätig werden kann, an vorderster Front mitschwimmen
und den Veränderungsprozess in den Wissenschaftsdisziplinen mit
gestalten kann. Glücklicherweise hat sich auch das bewahrheitet. Die
E-Medien haben Einzug gehalten. Im E-Science-Kontext entwickeln wir
völlig neue Arbeitsweisen, die vor einigen Jahren noch völlig undenkbar
gewesen wären. Das macht einfach Spaß!~

\textbf{Ja, das ist gut nachvollziehbar. Und wenn Sie ein
promovierterChemikergewesen wären, hätten Sie sich dann auch für das
Bibliothekswesen entschieden?}

Das muss ich mit einem uneingeschränkten Ja beantworten. Denn das war
definitiv keine frauenspezifische Entscheidung.~

\textbf{Sie haben sich nicht dafür entschieden, weil sie den
Öffentlichen Dienst als reizvollen Arbeitgeber empfunden haben?}

Nein.~

\textbf{In einem Artikel über Sie in der Badischen Zeitung, der kurz
nach Ihrem Wechsel nach Freiburg erschien, habe ich gelesen, Sie wollten
auch nicht unbedingt in der chemischen Industrie landen.~}

Das ist richtig, hat aber ebenfalls nichts mit der Tatsache zu tun, dass
ich Frau bin. Das war vielmehr eine sehr frühe Entscheidung: Schon im
Grundstudium haben wir eine Exkursion zu einem großen Chemiekonzern
gemacht, waren einen ganzen Tag lang dort. Das Schlüsselerlebnis war für
mich der Besuch der Essigsäurestraße, die heißt wirklich so und ist eine
riesige Produktionsanalage zur Synthese von Essigsäure. Der
Betriebsleiter dieser Produktion -- ein gestandener Mann mit vielen
Jahrzehnten Berufserfahrung -- stand kurz vor der Rente und erzählte uns
von seinem Lebenswerk. Er hatte die Produktionsausbeute im Laufe vieler
Jahre um ein oder zwei Prozentpunkte erhöht. Darauf er sehr stolz --
zurecht, denn er hat damit dem Unternehmen wahrscheinlich immense
Gewinne beschert. Aber ich bin abends sehr nachdenklich nach Hause
gefahren. Denn ich habe mir die Frage gestellt, ob mich eines Tages
meine Kinder oder Enkelkinder verstehen würden, wenn ich ihnen etwas
ähnliches erzählen würde. Nach Jahrzehnten mühevoller Arbeit ein solches
Ergebnis für herauszuziehen, das war mir zu wenig. Und fachlich auch zu
sehr fokussiert.~

Das ist in meiner jetzigen Position ganz anders, ich komme mit den
unterschiedlichsten Fachdisziplinen zusammen, oftmals an einem
Arbeitstag: Mediziner, Philosophen, Juristen, Theologen\ldots{} Da ist
viel mehr Vielfalt und Abwechslung drin, man muss sich auf sehr
unterschiedliche Denk- und Arbeitsweisen einstellen. Das macht es für
mich so reizvoll, in einer wissenschaftlichen Bibliothek zu arbeiten.~

\textbf{Seit 2008 leiten Sie nun die Universitätsbibliothek Freiburg.
Können Sie zwischen sich und männlichen Kollegen in einer ähnlichen
Position einen unterschiedlichen Führungsstil feststellen?~}

Man muss ja erst einmal die Frage stellen: Was ist eher weiblich und
eher männlich? Ich persönlich tue mich damit immer ein bisschen schwer.
Es gibt zweifelsohne Unterschiede zwischen Männern und Frauen in
Führungspositionen, die ich selbst als typisch männlich oder typisch
weiblich wahrnehme. Ich nehme aber auch Unterschiede zwischen den
Generationen wahr. Es gab ja in unserer Branche in den vergangenen
Jahren einen regelrechten Generationenwechsel, der nach meiner
Wahrnehmung einiges verändert hat. Führungskräfte verhalten sich meiner
Meinung nach heute oftmals anders.~

\textbf{Vermutlich war es früher hierarchischer?}

Ja, das sehe ich so. Heute in der IT-gestützten, aber auch --geprägten
Arbeitswelt laufen viele Prozesse einfach anders. Ein teamorientierter
Arbeitsstil ist unumgänglich, er wird Jugendlichen schon in der Schule
mit auf den Weg gegeben und das ist auch richtig so.~

Wenn ich Kolleginnen und Kollegen beispielsweise in Besprechungen
erlebe, sehe ich durchaus unterschiedliche persönliche Ausprägungen,
aber ich kann sie in aller Regel nicht spezifisch auf das Geschlecht
zurückführen. Dennoch, ich habe natürlich auch schon ganz
geschlechtsspezifische Verhaltensweisen im positiven, wie auch im
negativen Kontext erlebt.~

\textbf{Geschlechtsspezifische Verhaltensweisen kommen also vor, sind
aber nicht die einzigen Unterschiede.} \textbf{Und hinsichtlich der
inhaltlichen Gestaltung, gibt es da Unterschiede zwischen Frauen und
Männern?}

Das kann ich ebenfalls nicht abschließend beurteilen und weiß auch
nicht, ob es dazu Erhebungen gibt. Ich würde sagen, technisch
orientierte Fragestellungen könnten in der öffentlichen Meinung
vielleicht eher Männern zugeschrieben werden als Frauen. Aber ich kenne
zahlreiche Kolleginnen in unserer Branche mit einem stark
(IT-)technischen Profil und ebenso diverse Kollegen, die ihren Fokus
eben nicht in diesen Bereichen sehen. Man müsste wahrscheinlich eine
statistisch valide Erhebung zu dieser Fragestellung durchführen. Mich
würde interessieren, ob dabei etwas aussagekräftiges herauskommt.~

\textbf{Vielleicht wäre das etwas für eine Masterarbeit.}

Ja, vielleicht. Ich habe bereits in sehr unterschiedlichen,
geschlechtsspezifischen Kontexten gearbeitet. Die Chemie war zu meiner
Zeit stark männerdominiert. In meiner Arbeitsgruppe, in der ich
promoviert habe, war ich jahrelang die einzige Frau. Mein Doktorvater
begrüßte mich damals mit den Worten: \enquote{Hallo, wir sind hier nicht
frauenfeindlich. Wir hatten schon einmal eine Frau, die ist aber leider
direkt nach dem Diplom gegangen.}~

Wir haben letzte Woche wieder einmal intensiv über dieses Thema
diskutiert. Carl Djerassi, der chemische Vater der Pille, war bei uns in
Freiburg. Djerassi vertritt die Ansicht, dass der Stand der Forschung in
den Naturwissenschaften heute ein anderer wäre, wenn schon seit
Jahrzehnten mehr Frauen dort vertreten wären. Er macht dies am Beispiel
der immer noch nicht etablierten Pille für den Mann deutlich. Aus
heutiger Rückblende muss auch ich sagen: Manches war für Frauen nicht
gerade einfach. Es gab damals die klare Aussage: Wer seine Promotion
noch nicht unter Dach und Fach hat, der braucht sich keine Gedanken über
Familie oder gar Nachwuchs zu machen -- aus heutiger Sicht ganz klar
eine Form von Diskriminierung. Ich habe es aber damals als solche noch
nicht wahrgenommen.~

\textbf{Sie kamen bereits aus einem männerreichen Berufsumfeld. Also
haben Sie nicht erst als Führungskraft angefangen, als typisch männlich
geltende Verhaltensmuster zu entwickeln, wie zum Beispiel
Durchsetzungskraft, Risikobereitschaft, Selbstbeherrschung \footnote{http://de.wikipedia.org/wiki/Männlichkeit~}?~}

Sie haben drei Beispiele genannt. Wer diese Attribute nicht hat, der
kann ein Studium in einer „harten`` Naturwissenschaft nicht überstehen.
Ich habe einige Kommilitonen erlebt, an denen mir sehr viel lag, die
aber aufgrund mangelnder Selbstdisziplin, Beharrungsvermögen und
Durchsetzungskraft das Studium nicht geschafft haben. Auch eine gewisse
Risikobereitschaft muss man haben, sonst kann man nicht mit gefährlichen
Substanzen und Apparaturen arbeiten. Ich wäre wohl nicht in einen
Forschungsreaktor gegangen und hätte wahrscheinlich die
Selbstbeherrschung verloren, als die Türen bei einem Alarm für eine
halbe Stunde zugingen -- ein übrigens für mich sehr eindrucksvolles
Erlebnis. Aber das sind letztendlich Eigenschaften, die auch ein
Leistungssportler haben muss oder schlichtweg alle Personen, die in
anspruchsvollen beruflichen Kontexten arbeiten wollen. Diese
Eigenschaften werden leider auch heutzutage noch häufig Männern
zugesprochen. Ich finde aber, sie sollten \textbf{allen} Personen -~
Männern und Frauen -- zugesprochen und von ihnen auch erwartet werden
können, die in herausgehobener Position arbeiten.~

\textbf{Man liest und erlebt leider nach wie vor, dass diese
Eigenschaften geschlechtsspezifisch zugeordnet werden und nicht in
erster Linie mit verantwortungsvollen Positionen in Verbindung gebracht
werden.~}

Fragen der Gleichberechtigung waren für mich im Studium und am Anfang
meiner Berufstätigkeit noch kein Thema. Für mich galt von Anfang an das
Prinzip: Ich benachteilige und ich bevorzuge niemanden wegen seines
Geschlechts. Ich fördre diejenigen, die gute Arbeit machen. Und ich habe
lange Zeit geglaubt, dass auch ich stets einzig und allein gemäß dieser
Prinzipien bewertet werde. Im Laufe der Jahre habe ich aber durchaus
Situationen erlebt, in denen ich ins Zweifeln kam. Es war
interessanterweise ein männlicher Kollege, mit dem ich über dieses Thema
sprach und der sagte: \enquote{Das was hier passiert, ist ein
Geschlechterproblem. Es gibt Männer, die haben ein Problem mit Frauen in
Führungspositionen.`` Er empfahl mir ein Buch:}Das Arroganzprinzip" von
Peter Modler\footnote{http://www.fischerverlage.de/buch/das\_arroganz-prinzip/9783596184330}.
Modler ist ein Unternehmensberater, der seit vielen Jahren Beratungen
und Workshops speziell für Frauen in Führungspositionen anbietet. Er
schildert in diesem Buch Berichte von Frauen, die Diskriminierung in
ihrem Berufsalltag erlebt haben. Beim Lesen sind mir viele Lichter
aufgegangen und ich bewerte viele Erlebnisse heute ganz neu. Seitdem
suche ich auch aktiv das Gespräch mit Frauen in Leitungspositionen und
ich stelle immer wieder fest, sie haben ähnliches erfahren und das
ebenfalls nicht bewusst zuordnen können. Es gibt also sehr wohl
geschlechtsspezifische Verhaltensweisen und Diskriminierung, übrigens
auch in unserer Branche. ~

\textbf{Man muss es aber erst einmal wahrnehmen.}

Frauen (und selbstverständlich auch Männern!) muss man die notwendige
Sensibilisierung dafür mit auf den Weg geben und die Frauen darin
stärken, Diskriminierung jedweder Art nicht hinzunehmen.~

\textbf{Meiner eigenen Beobachtung zufolge gibt es in den öffentlichen
Bibliotheken mehr weibliche Führungskräfte als in den wissenschaftlichen
Bibliotheken. Sind Ihnen statistische Zahlen zum Frauenanteil in
bibliothekarischen Leitungspositionen bekannt?}

Bei der Vorbereitung auf unser Interview bin ich auf eine sehr
interessante Grafik\footnote{http://infobib.de/blog/2013/03/18/frauen-in-fuhrungspositionen-bei-dbv-mitgliedern/}
über Führungspositionen gestoßen. Insgesamt liegt der Frauenanteil in
Bibliotheken durch alle Ebenen hinweg nach einer Erhebung des Instituts
für Arbeitsmarkt- und Berufsforschung im Bibliotheksbereich bei fast
75\%. Auf der Basis des Adressverzeichnisses des Deutschen
Bibliotheksverbands wurde, auf die Sektionen des DBV aufgeschlüsselt,
der Anteil an Frauen in Führungspositionen ermittelt. Beim
Führungspersonal in den öffentlichen Bibliotheken nimmt der Anteil an
Männern mit der Größe der Bibliothek zu. In den wissenschaftlichen
Bibliotheken liegt der Anteil der Frauen wie bei den
Großstadtbibliotheken bei etwa 50\% und damit deutlich niedriger als in
den anderen Sektionen.~

\textbf{Frauen sind also, gemessen an ihrem Gesamtanteil im
Bibliothekswesen, deutlich geringer in Führungspositionen vertreten.
Woran könnte das liegen?}

Ganz spontan könnte man vermuten, dass der Männeranteil mit der Größe
der Einrichtung korreliert, wie auch in vielen anderen Branchen und
insbesondere in der freien Wirtschaft. Beim Vergleich zwischen den
öffentlichen und~ wissenschaftlichen Bibliotheken wird es dann aber
schwieriger. Hier stelle ich mir die Frage: Wer sitzt da in den
Auswahlgremien? An vielen Universitäten und Hochschulen sind
Leitungsebenen und Rektorate ja oft noch in hohem Maße von Männern
dominiert. Empirische Erhebungen in den unterschiedlichsten Branchen
belegen, dass Männer noch immer bevorzugt Männer einzustellen.~

\textbf{Das dürfte sich über die Generationen hinweg etwas auflösen.~}

Ja, das hoffe ich sehr. In der Bachelorarbeit \enquote{Die gläserne
Decke in Schweizer Bibliotheken}\footnote{http://www.htwchur.ch/uploads/media/CSI\_53\_Stadler.pdf}
an der Hochschule für Technik und Wirtschaft in Chur. Laura Stadler hat
darin Erhebungen aus dem internationalen Bereich aufgeführt, die einen
steigenden Anteil an Frauen, auch in den Führungsebenen, nachweisen. Wir
haben in der Bibliotheksbranche insgesamt einen hohen Anteil an Frauen
und der Anteil an Führungspositionen ist im Vergleich zu anderen
Branchen ebenfalls hoch.~

\textbf{Also sieht es doch gar nicht so schlecht aus.}

Das finde ich auch. Und wenn man bedenkt, dass Frauen erst seit gerade
einmal ca. 100 Jahren im Bibliothekswesen arbeiten, dann hat sich das
durchaus gut entwickelt. Bei uns an der Universitätsbibliothek Freiburg
gibt es einen deutlichen Frauenüberschuss in der Belegschaft. Das führt
dazu, dass unsere Gleichstellungsbeauftragte bei vielen
Stellenbesetzungsverfahren gar nicht mehr einbezogen werden möchte.
Ausnahme: Im höheren Dienst, also ab Entgeltstufe 13, denn da liegt der
Frauenanteil noch bei unter 50 \%. Aber der Trend ist aber auch hier
glücklicherweise steigend.

In der Churer Arbeit gibt es auch Erhebungen, die sich mit der
Arbeitssituation von Frauen in Leitungspositionen beschäftigen. Eine der
großen Herausforderungen für Frauen ist die Vereinbarkeit von Beruf und
Familie. Die Verantwortung für die Familie liegt nach wie vor im hohen
Maße bei Frauen. Familienplanung wird in der Regel auch heutzutage auf
die Zeit nach dem Studium geschoben. Dann kommen der Berufseinstieg und
die ersten Stufen auf der Karriereleiter. Und dann geht Frau schon
langsam auf die 40 zu und es wird Zeit für Nachwuchs. Genau zu diesem
Zeitpunkt ist Frau aber in einem Alter, in dem es richtig spannend wird
für Leitungspositionen. Aber die Arbeitsbedingungen sind häufig nicht
familienkonform: z.B. Dienstbesprechungen abends um 18 oder 19 Uhr. Wo
doch jeder wissen sollte, dass für diese Zeit eine Kinderbetreuung nicht
ohne weiteres auf die Beine gestellt werden kann. Hier erlebe ich
glücklicherweise seit einigen Jahren eine zunehmende Sensibilisierung.

\textbf{Die UB Freiburg wird zur Zeit saniert und voraussichtlich im
Wintersemester 2014/15 neu eröffnet. Sie waren in den letzten Jahren
also auch stark im Bauumfeld involviert, und damit erneut in einem sehr
männerdominierten Bereich. Haben Sie dies als in irgendeiner Weise
nachteilhaft empfunden?}

Vor wenigen Jahren haben mein Mann und ich ein Haus gebaut. Der eine
oder andere Handwerker wollte nicht so gerne mit mir verhandeln und
verlangte stattdessen nach dem Bauherrn\ldots{} Im dienstlichen Umfeld
ist mir so etwas dagegen noch nie passiert. Ich kann mich aber an eine
sehr lange Baubesprechung erinnern, in der es um die klimatechnischen
Voraussetzungen für unsere Lesesäle ging. Außer einer UB-Kollegin nahmen
nur Männern daran teil und es gab sehr unschöne Diskussionen. Die
Ingenieure wollten kein zusätzliches Geld in die Hand nehmen und
vertraten die Ansicht, es könne im Sommer auch einmal warm werden, auch
im Handschriftenlesesaal. Glücklicherweise hatten meine Kollegin und ich
aber die einschlägigen Normen sehr genau gelesen -- im Gegensatz zu den
Bauprofis. Bei mir entstand der Eindruck, dass sie nicht damit gerechnet
hatten, dass wir uns als Bibliothekarinnen mit derartigen technischen
Fragestellungen beschäftigen würden. Die zusätzliche Lüftungsanlage
wurde dann jedenfalls nach unseren Anforderungen und normenkonform
gebaut.

\textbf{Sie haben sie also durch fachliche Argumente überzeugen
können.~}

Ja. Aber ich kann nicht genau sagen: Haben sie uns unterschätzt, weil
wir Frauen waren oder weil wir vermeintlich technisch nicht versierte
Bibliothekare waren.

\textbf{Ja, es bleibt unklar, welches Klischee da wohl eine Rolle
gespielt hat. Ertappen Sie sich denn manchmal dabei, dass Sie mit Ihren
männlichen Mitarbeitern anders umgehen als mit Ihren Mitarbeiterinnen?~}

Auf unbewusster Ebene kann sich davon, glaube ich, niemand frei machen.
Frauen gehen mit Frauen anders um als mit Männern -- und umgekehrt auch.
Ich habe für mich den Anspruch, damit so professionell umzugehen, dass
ich zumindest \emph{bewusst}keinen Unterschied mache. Ganz generell muss
ich im Beruf auch mit Menschen, die mir sympathisch sind, genauso gut
zusammenarbeiten wie mit solchen, die ich nicht so gut leiden mag. Das
gilt für meine Mitarbeiterinnen und Mitarbeiter, meine Kolleginnen und
Kollegen genauso wie für meine Vorgesetzen.~

Wenn ein männlicher Mitarbeiter einen Antrag auf Elternzeit stellt, dann
sage ich: Prima, das unterstütze ich! Denn wir werden bei der
Gleichstellung nur dann wirklich etwas erreichen, wenn Männer sich
aktiver an der Familienarbeit beteiligen. Glücklicherweise ist die Zahl
der Anträge inzwischen steigend. Vielleicht bevorzuge in dieser Frage
sogar hin und wieder einen Mann, aber dazu stehe ich auch und sehe dies
als eine Maßnahme zur Förderung der Gleichberechtigung.

\textbf{Welche Frauen empfinden Sie als positive Beispiele im
Bibliothekswesen und warum?}

Da kann und möchte ich keine Namen nennen. Es gibt viele Kolleginnen,
deren Arbeit ich sehr schätze, die ich auch als Person schätze und gerne
und intensiv mit ihnen zusammenarbeite. Wirklich hervorzuheben sind
sicherlich Frauen wie Bona Peiser. Es ist ja wie schon gesagt noch nicht
lange so, dass Frauen in Bibliotheken überhaupt arbeiten können und eine
Ausbildung absolvieren dürfen. Und dann sind da die Frauen, die auch
heute noch unter sehr schwierigen politischen Rahmenbedingungen ihren
Beruf ausüben müssen. Wer in einer Bibliothek arbeitet, Wissen und
Informationen verwaltet, ist für totalitäre Regime per se angreifbar.
Frauen, die trotz derartiger widriger Umstände \emph{(lacht)} „ihren
Mann stehen`` -- da sieht man wieder an unserer Alltagssprache, wie wir
eben doch noch in einer männlich geprägten Berufswelt leben -- gebührt
meiner Meinung nach größter Respekt.~

\textbf{Zum Abschluss möchte ich Sie noch fragen: Was sind Ihre
persönliche Empfehlungen für eine junge Schulabgängerin, die sich
beruflich ins Bibliothekswesen hin orientieren möchte?~}

Für mich war und ist Grundprinzip: Wenn man Leidenschaft für etwas
empfindet, wenn man sich für etwas interessiert, etwas gerne macht,
lernen und wissen möchte, dann soll man genau das machen. Dann findet
man seinen Weg. Leider lese ich häufig in Bewerbungsschreiben:
\enquote{Ich möchte in der Bibliothek arbeiten, weil ich so gerne lese.}
Das ist gewiss eine sehr gute Grundvoraussetzung, greift aber meines
Erachtens nach zu kurz. Man muss sich vielmehr mit dem modernen
Berufsbild und den Perspektiven für die Zukunft auseinandersetzen. Und
man muss für sich klären, ob sich eine solche Tätigkeit für die nächsten
Jahr(zehnt)e vorstellen kann. Unser Berufsbild ist einer hohen
Veränderung unterworfen, das war sicherlich vor zwanzig Jahren noch
anders. Und wenn man Karriere machen will, muss man zielstrebig und
engagiert daran arbeiten, man muss sich etwas zutrauen, Netzwerke
aufbauen und sich stetig weiterbilden. Und last but not least: Keine zu
langen Berufspausen einlegen! Dazu muss man sich auch im privaten
Bereich so aufstellen, dass dies möglich ist.

\textbf{Das bringt uns wieder zum Anfang zurück: Den Menschen als
Arbeitskraft auch in seinem Umfeld betrachten.}

Ganz genau!

\textbf{Liebe Frau Kellersohn, vielen Dank für das ausführliche Gespräch
und für Ihre Offenheit.}

%autor

\end{document}