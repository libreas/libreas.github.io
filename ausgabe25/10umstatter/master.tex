\documentclass[a4paper,
fontsize=11pt,
%headings=small,
oneside,
numbers=noperiodatend,
parskip=half-,
bibliography=totoc,
final
]{scrartcl}

\usepackage{synttree}
\usepackage{graphicx}
\setkeys{Gin}{width=.6\textwidth} %default pics size

\graphicspath{{./plots/}}
\usepackage[ngerman]{babel}
%\usepackage{amsmath}
\usepackage[utf8x]{inputenc}
\usepackage [hyphens]{url}
\usepackage{booktabs} 
\usepackage[left=2.4cm,right=2.4cm,top=2.3cm,bottom=2cm,includeheadfoot]{geometry}
\usepackage{eurosym}
\usepackage{multirow}
\usepackage[ngerman]{varioref}
\setcapindent{1em}
\renewcommand{\labelitemi}{--}
\usepackage{paralist}
\usepackage{pdfpages}
\usepackage{lscape}
\usepackage{float}
\usepackage{acronym}
\usepackage{eurosym}
\usepackage[babel]{csquotes}
\usepackage{longtable,lscape}
\usepackage{mathpazo}
\usepackage[flushmargin,ragged]{footmisc} % left align footnote

\urlstyle{same}  % don't use monospace font for urls

\usepackage[fleqn]{amsmath}

%adjust fontsize for part

\usepackage{sectsty}
\partfont{\large}

%Das BibTeX-Zeichen mit \BibTeX setzen:
\def\symbol#1{\char #1\relax}
\def\bsl{{\tt\symbol{'134}}}
\def\BibTeX{{\rm B\kern-.05em{\sc i\kern-.025em b}\kern-.08em
    T\kern-.1667em\lower.7ex\hbox{E}\kern-.125emX}}

\usepackage{fancyhdr}
\fancyhf{}
\pagestyle{fancyplain}
\fancyhead[R]{\thepage}

%meta
%meta

\fancyhead[L]{W. Umstätter \\ %author
LIBREAS. Library Ideas, 25 (2014). % journal, issue, volume.
\href{http://nbn-resolving.de/urn:nbn:de:kobv:11-100219316}{urn:nbn:de:kobv:11-100219316}} % urn
\fancyhead[R]{\thepage} %page number
\fancyfoot[L] {\textit{Creative Commons BY 3.0}} %licence
\fancyfoot[R] {\textit{ISSN: 1860-7950}}

\title{\LARGE{Rezension zu: Bredemeier, Willi (2014) – Ein Anti-Heimat-Roman. Bildungsreisen durch ein unbekanntes Land. Simon Verlag für Bibliothekswissen, Berlin (2014)}} %title %title
\author{Walther Umstätter} %author

\setcounter{page}{94}

\usepackage[colorlinks, linkcolor=black,citecolor=black, urlcolor=blue,
breaklinks= true]{hyperref}

\date{}
\begin{document}

\maketitle
\thispagestyle{fancyplain} 

%abstracts

%body
Auch wenn die 28 Jahre PASSWORD im Vergleich zu \enquote{Nachrichten für
Dokumentation} (NfD) beziehungsweise der Information in Wissenschaft und
Praxis (IWP) im 65. Jahrgang kaum vergleichbar sind, so ist es trotzdem
eine Leistung, diese Zeitschrift, in der Willi Bredemeier für die
Information Professionals kämpft, mit Leben erfüllt zu haben. Nun
schrieb er seine Falschnamen-Memoiren als Anti-Heimat-Roman, in denen
sich nur betroffene Insider wiedererkennen können.

Wer den ersten oder zweiten Weltkrieg überlebt, vielleicht auch
Gefangenschaft überstanden hat, wusste nicht nur was Heimweh ist. Die
Heimat und alle die dort blieben, erschienen insbesondere den jungen
Soldaten verteidigungswürdig -- ob sie wollten oder nicht, denn sonst
wären ihre Opfer sinnlos gewesen. Was den meisten Menschen in der
heutigen Globalisierung begrifflich nur schwer definierbar und
nachvollziehbar ist, war den Heimatvertriebenen und Entwurzelten völlig
selbstverständlich. Der heimatliche Werteverlust begann, als man anfing
sich über Heimatfilme lustig zu machen. Der Hölle gegenüber, die nicht
nur die jungen Landser an der jeweiligen Front durchlebten, erschien die
Erinnerung an die Heimat wie ein Paradies. Natürlich war sie für die
Verbliebenen keinesfalls paradiesisch. Insbesondere im Ruhrgebiet war
sie das weder vor, während, noch nach dem zweiten Weltkrieg, wo sich~
insbesondere beim Wiederaufbau alle freuten, dass die Schornsteine
endlich wieder rauchten. Da war die Heimatidylle bei genauer Betrachtung
eher eine andere Hölle, die sich hier aus der recht sarkastischen Sicht
des Pseudonyms Gerd Arntz in Plattdeutsch entwickelt und nicht im
bekannteren österreichischen Anti-Heimat-Roman-Stil. Außerdem ging es
früher in den Anti-Heimat-Romanen meist um die Industrialisierung,
während es hier bereits um die Folgen der Postindustrialisierung und des
wachsenden Bedarfs nach Schulbildung für die kommende Informations- und
Wissenschaftsgesellschaft geht.~

Die Bildungsreise von Gerd Arntz von der Zwergschule in Grotebühl bis
zum Doktor, der sich mit dem \enquote{kritischen Irrationalismus}
beschäftigt, macht deutlich, welch ein Vabanquespiel die Bildungspolitik
in Deutschland bislang war. Arntz hat es bis zum Verleger einer
Zeitschrift gebracht, während unzählige seiner Wegbegleiter auf der
Strecke bleiben mussten. Auch bei ihm hing es, wie bei allen, die von
unten kommen, zeitweise am seidenen Faden, wenn er schreibt:
\enquote{Vielen Dank, Herr Physiklehrer, Sie haben mir das Leben am
Abendgymnasium unserer Stadt gerettet. (S. 293) So ist Bredemeiers Fazit
auf S. 480, dass er in}eine extrem bildungsfeindliche Bundesrepublik, in
der um jeden Lesestoff gekämpft werden musste" hinein geboren wurde.~

Ein bisschen erinnert Gerd Arntz in seiner fanatischen Liebe zu Büchern
an den blechtrommelnden Oskar Matzerath, auch wenn Oskar sein Lesen und
Schreiben für sich behält, und recht begrenzt auf die Vorbilder Rasputin
und Goethe beschränkt ist, während Gerd dafür gehänselt wird, alles zu
lesen, was ihm \enquote{unter die Finger kommt}. (S. 145) Sicher wird
das vorliegende Buch nicht den Bekanntheitsgrat der Blechtrommel
erreichen, denn \enquote{Die explizite Beschreibung des
Geschlechtsverkehrs stellt dieses Buch {[}Die Blechtrommel, nach Meinung
seiner Klassenlehrerin{]} außerhalb jeder Literatur.} (S. 295), während
Bredemeiers Anti-Heimat-Roman noch eher im Rahmen des Normalen bleibt,
und für wirkliche Bestseller braucht man viel mehr Sex and Crime.~

Arntz liest querbeet, alles was ihm vor die Flinte kommt, so dass er im
Laufe der Zeit erkennt, was in Goethes Sinne mehr zu Genuss und Belebung
dient, und welche Autoren er als \enquote{Säulenheilige} wie
beispielsweise Thomas Kuhn identifiziert. Wenn er sich aber berechtigt
wundert, warum Kuhns Werk in etlichen Disziplinen nicht zur Kenntnis
genommen wird, so kann hier angemerkt werden, dass Kuhn 1965 eine Little
Science beschrieben hat, die inzwischen weitgehend von der Big Science
abgelöst worden war, was viele Wissenschaftstheoretiker bis heute nicht
zur Kenntnis nehmen wollen.

Unter dem Aspekt, den Bildungserwerb einer neuen Generation zu
beobachten und zu hinterfragen, ist dieses Buch höchst interessant. Ob
im Elternhaus, in der Schule, dem Freundeskreis oder Poppers Welt 3,
jede Generation wächst unter neuen Bedingungen auf. Schon der Versuch
Gerhard Hauptmanns, mit den Webern deutlich zu machen, wie ein
technologischer Umbruch menschliche Existenzen zerstört, kann trotz
seiner ganzen Dramatik als misslungener Versuch angesehen werden, Fehler
der Vergangenheit zu vermeiden, wenn man sich daran erinnert, wie sich
in unseren Schulen Lehrer und Schüler immer wieder über die skrupellosen
Fabrikanten ereiferten, obwohl sie damit am eigentlichen Drama völlig
vorbei diskutierten. Denn die Ursache der Mechanisierung und ihrer
Folgen traf nicht nur die Weber selbst, sondern auch die, die ihre
Erzeugnisse zu verkaufen versuchten. Die Automatisierung mit Hilfe der
Webstühle war bereits die Vorstufe von Jacquard, von Lochkarten und
Computern bis hin zu unseren smarten Robotern von heute. Die Schuldigen
am Leid der Weber waren weder die Erfinder der Dampfmaschine, noch die
Fabrikanten, sondern diejenigen, die die Weber in diesen sinnlosen
Wettbewerb mit den mechanischen Webmaschinen schickten. So wie der
Berufsberater, der auf die Bemerkung, \enquote{Der Junge liest gern}
sagt, \enquote{Dann sollte er Schriftsetzer werden}. (S. 160) Damit
wurden viele Jungen dieser Generation ebenso in die absehbare
Arbeitslosigkeit geschickt wie die im Kohlebergbau, dessen Zechen bald
danach reihenweise schlossen.~

Bei Bredemeier geht es um die grundsätzlich gleiche Problematik, wie bei
den Webern und ihrer fehlenden Umschulung, aber nun zum Beginn des
Informationszeitalters. Mit Recht hatte schon Norbert Wiener XE
\enquote{Wiener, N.} , der Begründer der Kybernetik XE
\enquote{Kybernetik}~ (1943), die Gewerkschaften nach dem zweiten
Weltkrieg vor den Veränderungen in unserer Gesellschaft durch die
Robotik gewarnt XE \enquote{Robotik} . Denn schon damals war klar, was
viele Menschen heute noch immer nicht wahr haben wollen: Roboter
übernehmen schrittweise immer mehr Aufgaben der Agrar- und
Industriegesellschaft, der Medizin, der Altenpflege, der Massenmedien
und wahrscheinlich auch der Rentnerversorgung. Auch wenn es immer zwei
Parteien gibt, die, die sich dieser Herausforderung stellen und die, die
sie so lange bremsen und klein reden, solange es ihr Vorteil ist.~

Was Gerd Arntz als analytischer Beobachter dieses Umbruchs dabei sieht,
sind die Interessenverschiebungen und die strukturellen Veränderungen,
auch wenn das auf rund fünfhundert Seiten differenzierter geschieht als
bei Hauptmann. Außerdem geht es thematisch nebeneinander darum:
\enquote{In der Landwirtschaft hat die Maschinisierung eingesetzt.
Mähmaschine, Kartoffelroder und Traktor werden in kürzestmöglichen
Abständen eingeführt.} (S. 101); \enquote{Ohne dass? dies einer zur
Kenntnis genommen hätte, hat die Bildungsrevolution sogar auf dem Land
eingesetzt.} (S. 117); \enquote{Glücklicherweise hat die Maschinisierung
der Haushalte eingesetzt. Jetzt muss die gnädige Frau selbst in die
Küche.} (S. 118) -- und sie muss oder darf immer öfter auch den Männern
den Beruf streitig machen, können wir heute rückblickend hinzufügen.

Eigentlich konnte man schon damals erkennen, dass die größte Revolution
darin liegt, dass immer mehr Mädchen und Jungen Abitur machen und
studieren, während die Klassenlehrerin von G. Arntz meint: \enquote{In
der Bundesrepublik besitze weniger als jeder zehnte Bürger das Abitur.}
(S. 283) und auch erkennen lässt, dass das so bleiben müsse. Nun konnte
damals nicht jeder wissen, dass Derek John de Solla Price bereits
erkannt hatte, dass sich die Zahl der Menschheit mit nur 50 Jahren
verdoppelt, während die der Wissenschaftler mit einer Verdopplungsrate
von nur 20 Jahren wächst. Damit kann man sich ausrechnen, wann in dieser
Welt fast alle Menschen Wissenschaftler sind beziehungsweise sein
müssten. Auffällig ist dabei nur, dass immer mehr Bildungspolitiker von
Elite, von Exzellenz und von Spitzenforschung sprechen, je weiter die
Big Science nun auch den geistigen Durchschnittsbürger unausweichlich in
sich aufsaugt.~

Arntz hat \enquote{die Lust am fröhlichen Fabulieren} (S. 290), die er
im Lesen und Schreiben auslebt, wobei vieles dessen, was hier
beschrieben wird, zu realistisch ist, um es als Satire zu bezeichnen.
Trotzdem liest sich diese Bildungsreise unterhaltsam wie eine Satire.

Apropos Bildungsreise: Dass Goethes Faust in der jungen Bundesrepublik
Deutschland unumstritten an der Spitze des deutschen Bildungskanons
stand, ist sicher richtig, aber insofern bemerkenswert, als etliche der
damaligen Deutschlehrer inzwischen gern etwas Moderneres an seine Stelle
gestellt hätten. So entsinne ich mich, dass auch meine Deutschlehrerin
das Essentielle des Faust nicht besonders interessierte. Sie hatte ihn
nicht verstanden. Denn es war eine der wichtigsten Erkenntnisse Goethes,
dass Menschen im Wissensgewinn vier Phasen durchlaufen, das Streben, den
Genuss, die Resignation und die Gewohnheit. Er hat das in einem Brief
1801 an Schiller angesprochen, wobei Schiller mit der Weitsicht des
Historikers erkannte, dass das nicht nur für Einzelpersonen gilt,
sondern auch für historische Abläufe insgesamt. Der faustische Mensch,
als Gedankenexperiment, ist der ewig nur nach Erkenntnis Strebende, nach
dem er in der Osternacht seine tiefste Resignation überlebt hatte, weil
er erkannte, dass er mit immer mehr Wissen auf ein immer größeres Meer
der Unwissenheit hinaus blickte. Bei Arntz lautet diese Erkenntnis so:
\enquote{Je mehr wissenschaftliche Marktnischen ich mir zu Eigen mache
und je häufiger ich einen Zeh in andere Fachbereiche setze, desto mehr
Löcher und Widersprüche entdecke ich.} (S. 370) Fausts Pendant, Wagner,
glaubt dagegen in seiner Naivität noch eines Tages alles wissen zu
können, und macht sich mit dem Satz lächerlich \enquote{Zwar weiß ich
viel, doch möcht ich alles wissen.} Es ist ein großes Problem, in den
Schulen und Hochschulen, dass sich die Lehrer leichter auf das Niveau
der Schüler, als die Schüler auf das der Lehrer begeben. In den Worten
Bredemeiers heißt das zum Beispiel, \enquote{dass der Lehrer die Rolle
eines Fähnleinführers übernimmt.} (S. 287) Es ist richtig,
\enquote{Faust ist eine Liebesgeschichte} (S. 297), weil Goethe wusste,
\enquote{Wer vieles bringt, wird manchem etwas bringen; Und jeder geht
zufrieden aus dem Haus.}. Und so bringt auch Bredemeier vieles, was
manchem etwas bringen dürfte.

\begin{center}\rule{3in}{0.4pt}\end{center}

\textbf{Walther Umstätter} ist emeritierter Professor des Instituts für
Bibliotheks- und Informationswissenschaft (IBI) der Humboldt-Universität
zu Berlin.~

%autor

\end{document}