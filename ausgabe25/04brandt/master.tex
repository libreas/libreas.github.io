\documentclass[a4paper,
fontsize=11pt,
%headings=small,
oneside,
numbers=noperiodatend,
parskip=half-,
bibliography=totoc,
final
]{scrartcl}

\usepackage{synttree}
\usepackage{graphicx}
\setkeys{Gin}{width=.6\textwidth} %default pics size

\graphicspath{{./plots/}}
\usepackage[ngerman]{babel}
%\usepackage{amsmath}
\usepackage[utf8x]{inputenc}
\usepackage [hyphens]{url}
\usepackage{booktabs} 
\usepackage[left=2.4cm,right=2.4cm,top=2.3cm,bottom=2cm,includeheadfoot]{geometry}
\usepackage{eurosym}
\usepackage{multirow}
\usepackage[ngerman]{varioref}
\setcapindent{1em}
\renewcommand{\labelitemi}{--}
\usepackage{paralist}
\usepackage{pdfpages}
\usepackage{lscape}
\usepackage{float}
\usepackage{acronym}
\usepackage{eurosym}
\usepackage[babel]{csquotes}
\usepackage{longtable,lscape}
\usepackage{mathpazo}
\usepackage[flushmargin,ragged]{footmisc} % left align footnote

\urlstyle{same}  % don't use monospace font for urls

\usepackage[fleqn]{amsmath}

%adjust fontsize for part

\usepackage{sectsty}
\partfont{\large}

%Das BibTeX-Zeichen mit \BibTeX setzen:
\def\symbol#1{\char #1\relax}
\def\bsl{{\tt\symbol{'134}}}
\def\BibTeX{{\rm B\kern-.05em{\sc i\kern-.025em b}\kern-.08em
    T\kern-.1667em\lower.7ex\hbox{E}\kern-.125emX}}

\usepackage{fancyhdr}
\fancyhf{}
\pagestyle{fancyplain}
\fancyhead[R]{\thepage}

%meta
%meta

\fancyhead[L]{S. Brandt \\ %author
LIBREAS. Library Ideas, 25 (2014). % journal, issue, volume.
\href{http://nbn-resolving.de/urn:nbn:de:kobv:11-100219254
}{urn:nbn:de:kobv:11-100219254}} % urn
\fancyhead[R]{\thepage} %page number
\fancyfoot[L] {\textit{Creative Commons BY 3.0}} %licence
\fancyfoot[R] {\textit{ISSN: 1860-7950}}

\title{\LARGE{Lieder und Gedichte sammeln gegen das Verstummen. Erinnerungen an die Bibliothekarin Anna Kvapilová (1905-1992)}} %title %title
\author{Susanne Brandt} %author

\setcounter{page}{26}

\usepackage[colorlinks, linkcolor=black,citecolor=black, urlcolor=blue,
breaklinks= true]{hyperref}

\date{}
\begin{document}

\maketitle
\thispagestyle{fancyplain} 

%abstracts

%body
Vermutlich gibt es viele vergessene oder verloren gegangene Geschichten
wie diese, Geschichten von Bibliothekarinnen, die durch Krieg und
politische Verfolgung aus ihrem Beruf gerissen wurden, aber doch etwas
davon gerettet haben: Die Kraft der gehüteten und weitergegebenen
Gedichte und Lieder für die Rettung der Identität und gegen das
Verstummen.

Aufmerksam geworden bin ich auf die Lebensgeschichte von Anna (Anička)
Kvapilová (geb. 19.3.1905 bei Sedlcany/Böhmen -- gest. 28.6.1992 in
Norwegen) eher zufällig bei Recherchen zu einem Lagerlieder-Projekt
(Ausländer 2006), die zunächst eine eigentlich ganz alltägliche
bibliothekarische Eigenschaft von ihr ans Licht brachten: Sie war eine
engagierte Sammlerin. Mit großer Sorgfalt und Leidenschaft für die
Wirksamkeit der Worte und der Musik trug sie Lieder und Gedichte
zusammen -- in den1930er Jahren als Musikbibliothekarin in Prag, vor
allem aber in den 1940er Jahren im Konzentrationslager Ravensbrück. Über
ihre Tätigkeit an der Musikabteilung der Prager Stadtbibliothek von Juni
1936 bis Juli 1939 (andere Quellen lassen auf eine bibliothekarische
Tätigkeit bis zu ihrer Verhaftung 1941 schließen) lässt sich nicht viel
in Erfahrung bringen. Sie soll in dieser Zeit mehrere Artikel über das
Bibliothekswesen publiziert sowie eine Ausstellung zusammengestellt
haben, die dem Komponisten Antonín Dvořák gewidmet war. Als Mitglied der
sozialistischen Partei beteiligte sie sich dann am antifaschistischen
Widerstand gegen die Besetzung von Böhmen und Mähren durch die
Nationalsozialisten. Der Widerstandsgruppe \enquote{Úvod} stellte sie
ihre Prager Wohnung für konspirative Treffen zur Verfügung. Im April
1941 wurde sie von der Gestapo verhaftet und im Herbst 1941 in das
Konzentrationslager Ravensbrück gebracht. (Knapp 2003, S. 234-236)

Dort traf sie im so genannten tschechischen Block Milena Jesenská
wieder, die sie bereits aus Prag von einer Begegnung in der Redaktion
der später verbotenen Zeitschrift \emph{Přítomnost} her kannte -- jene
Milena, deren Name durch die Briefe Kafkas in die Literaturgeschichte
eingegangen ist (Kafka, 1993). Margarete Buber-Neumann erwähnt die
Freundschaft der beiden Frauen später in ihrem Buch über Milena
(Buber-Neumann, 1977). Anna Kvapilová selbst äußert sich dazu so, wie
sie in einem Beitrag der ZEIT aus dem Jahr 1983 zitiert wird:

\begin{quote}
\enquote{Ich las alles, was Milena schrieb, ich habe mir alle ihre
Artikel ausgeschnitten und aufbewahrt. Ich konnte eben nicht anders,
machte mich auf den Weg, um Milena persönlich für ihre Tapferkeit zu
danken. Zum zweiten Mal begegneten wir uns am 15. Oktober 1941, als ich
gemeinsam mit 20 Frauen, die aktiv am Widerstand gegen Hitler
teilgenommen haben, ins KZ Ravensbrück\footnote{\url{http://www.zeit.de/schlagworte/orte/ravensbrueck}}
eingeliefert wurde. Ich werde diesen Abend nie vergessen. Wir mußten
nackt an der hell beleuchteten Tür zum Krankenrevier vorbeilaufen, und
da erblickte ich Milena. Sie stand in der offenen Tür, und mir schien
es, als trüge sie rund um ihren blonden Kopf eine Gloriole.
,Willkommen`, rief sie uns zu.}(Filip, 1983)
\end{quote}

Annas kulturelles und musikalisches Engagement fand in Ravensbrück eine
besondere Fortsetzung. Frauen, die sie kannte und die neu ins Lager
kamen, bat sie darum, ein Gedicht in ihrer jeweiligen Muttersprache in
ein Heft einzutragen, das sie gut verwahrte. Aus ihrer Zeit als
Bibliothekarin hatte sie auch die Fähigkeit zum Buchbinden mitgebracht
und nutzte diese nun im Lager, um für Mithäftlinge kleine Lieder- und
Gedichtbücher als Geschenke herzustellen. Daneben führte sie selbst
Tagebuch und schrieb Gedichte, in denen sie die quälende Monotonie des
Lagerlebens verarbeitete:~

\emph{Der Tag\\\\ Wir stehen morgens nur deshalb auf,\\ um abends
wiederum schlafen zu gehen.\\ Vielleicht bringt die Nacht uns dann im
Traum,\\ was die Wirklichkeit nicht zu geben vermag.\\\\ Wir stehen auf,
erwarten den nächsten Tag,\\ ein Meer ungeweinter Tränen rinnt
hernieder.\\ An uns bricht sich der Sturm der Zeit.\\ Worauf wartet ein
jeder von uns?\\\\ Eine treue Wiederholung des Gestern nur,\\ wieder Not
und Erniedrigung,\\ alles ist so zum Verrücktwerden gleich,\\ nur Wandel
prägt ins Gesicht uns die Zeit.\\\\Anička Kvapilová (Ravensbrück 1944),
Nachdichtung von Jan-Peter Abrahami~ (Jaiser, 2005)}

Dieses wie auch andere Gedichte aus ihrer Sammlung halfen ihr und den
anderen Frauen, sich vor der inneren Erstarrung zu retten, das Gefühl
der Machtlosigkeit mit der Macht der Worte zu durchbrechen -- und mit
der Macht der Musik. Vermutlich wird sie aus den zusammengetragenen
Volksliedern im Lager wie aus den Erinnerungen an den Notenbestand der
Musikbibliothek in Prag geschöpft haben, als sie im Lager einen
Frauenchor gründete, der eben diese Volkslieder wie auch bekannte Lieder
von Dvorak und Smetana sang.

Anna Kvapilová überlebte das Lager. In einem Transport mit Norwegerinnen
wurde sie durch das Schwedische Rote Kreuz aus Ravensbrück evakuiert.
Sie kehrte 1945 nach Prag zurück, war als Offizier der
\enquote{Vereinigung der Nationalen Revolution} tätig und publizierte
einige Zeitschriftenaufsätze und Bücher über das kulturelle Leben in
Ravensbrück. Nachdem 1948 die kommunistische Partei die Regierung
übernommen hatte, wurde sie aus dem öffentlichen Dienst entlassen. Um
einer erneuten Verhaftung zu entgehen, floh sie im August 1948 ins
norwegische Exil und engagierte sich fortan in der Hilfe für
tschechische und slowakische Flüchtlinge. 1985 erhielt sie in Norwegen
eine Auszeichnung für ihre Verdienste.

Nach ihren Erinnerungen an Ravensbrück befragt, erzählte sie nicht nur
von geretteten Texten und Büchern, sondern auch von verlorenen: Milena
hatte Anna ihre Tagebücher anvertraut, bevor sie am 17. Mai 1944 an
einer schweren Nierenentzündung in Ravensbrück starb. Ihre Hoffnung war,
dass Anna die Tagebücher heimlich verwahren und irgendwann nach Prag
zurück bringen könnte. Das aber gelang nicht. Noch 40 Jahre später
fühlte sich Anna für den Verlust der Tagebücher verantwortlich. Sie
erzählt:

\begin{quote}
\enquote{Ich konnte aber Milenas Tagebücher nicht ständig unter dem Rock
versteckt tragen, das war zu riskant, ich habe sie im Lager unter den
Fußboden geschoben und ständig die Verstecke gewechselt. Als im April
1945 die deutschen Sozialdemokratinnen, mit ihnen auch Grete
Buber-Neumann, gemeinsam mit den norwegischen und dänischen Frauen
entlassen wurden, hätte ich Milenas Tagebücher Grete oder einer der
vielen Frauen, zu denen ich volles Vertrauen hatte, geben sollen. Ich
hatte aber Angst, daß sie noch gefilzt würden. Und dann kamen die
hektischen Tage kurz vor dem Zusammenbruch, ich lebte in ständiger
Unsicherheit und habe Milenas Tagebücher schließlich ganz einfach
verloren \ldots{}}(Filip, 1983)
\end{quote}

Zwei Jahre vor ihrem Tod nutzte Anna Kvapilová 1990 noch einmal die
Gelegenheit, in ihre Heimat reisen zu können. Sie starb 1992 in Norwegen
und wurde im Grab ihres Kindes, das schon vor dem Krieg in Prag
gestorben war, beigesetzt.

\section*{Literatur}\label{literatur}

Ausländer, Fietje; Brandt, Susanne; Fackler, Guido (2006): O bittre
Zeit. Lagerlieder 1933-1945. Hrsg. vom Dokumentations- und
Informationszentrum Emslandlager (DIZ). Papenburg : DIZ, 3 CD's mit
Beiheften.

Buber-Neumann, Margarete (1977): Milena, Kafkas Freundin. München :
Müller.

Bundeszentrale für politische Bildung (2006): Ravensbrück -- Überlebende
Erzählen.~\url{http://www.bpb.de/geschichte/nationalsozialismus/ravensbrueck/60697/frauenlager-ravensbrueck}

Filip, Ota (1983): Wer war Milena? Auf Spurensuche in Oslo. In: Die
Zeit. 2 (7. Januar)
1983.~\url{http://www.zeit.de/1983/02/wer-war-milena/seite-8}

Jaiser, Constanze; Pampuch, Jacob David (2005): Europa im Kampf
1939--1944. Internationale Poesie aus dem Frauen-Konzentrationslager
Ravensbrück. CDs mit Begleitbuch, Berlin : Metropol.

Kafka, Franz (1983):\emph{Briefe an Milena}, erweiterte und neu
geordnete Ausgabe, hrsg. von Jürgen Born und Michael Müller, Frankfurt
am Main : Fischer.

Knapp, Gabriele (2003): Frauenstimmen. Musikerinnen erinnern an
Ravensbrück. Berlin : Metropol, 2003, S.234-236.

%autor

\end{document}
