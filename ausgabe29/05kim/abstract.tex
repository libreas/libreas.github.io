\small
\textbf{Zusammenfassung:} Im Folgenden soll aufgezeigt werden, wie
derzeit das Literaturverwaltungsprogramm Zotero innerhalb des Index
Theologicus genutzt wird, um unselbstständige Literatur in einem
bibliothekarischen Katalogisierungssystem zu erfassen. Die modulare und
flexible Architektur der Open Source Software erlaubt es, die bereits
kollaborativ zusammengetragene Programmierarbeit zur Datenextraktion
mitzunutzen. Das vorgestellte semiautomatische Verfahren bringt auch bei
der Verknüpfung von Normdaten erhebliche Vorteile für die
Medienbearbeitung.

\textbf{Schlüsselwörter:} Literauturverwaltungsprogramm, Zotero,
Katalogisierung, Unselbständige Werke, Aufsatzliteratur, Index
Theologicus, Online-Bibliographie, WinIBW, Fachinformationsdienst
Theologie

\begin{center}\rule{0.5\linewidth}{\linethickness}\end{center}

\textbf{Abstract:} This article presents an approach to use the
reference management software Zotero within the theological article
database Index Theologicus to catalogue article metadata for a library
management system. Zotero's Open Source nature and flexible architecture
allowed us to seamlessly reuse the vast amount of data extraction
routines collaboratively developed for the software. We will show how
the semi-automatic workflow we developed will make authority linking fun
again.

\textbf{Keywords:} Reference Management System, Zotero, Cataloguing,
Journal articles, Index Theologicus, Theological database, Academic
Information Services for Theology
