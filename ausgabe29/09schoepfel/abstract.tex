The paper presents results from a campus-wide survey at the University
of Lille (France) on research data management in social sciences and
humanities. The survey received 270 responses, equivalent to 15\% of the
whole sample of scientists, scholars, PhD students, administrative and
technical staff (research management, technical support services); all
disciplines were represented. The responses show a wide variety of
practice and usage. The results are discussed regarding job status and
disciplines and compared to other surveys. Four groups can be
distinguished, i.e.~pioneers (20-25\%), motivated (25-30\%), unaware
(30\%) and reluctant (5-10\%). Finally, the next steps to improve the
research data management on the campus are presented.
