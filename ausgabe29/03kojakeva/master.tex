\documentclass[a4paper,
fontsize=11pt,
%headings=small,
oneside,
numbers=noperiodatend,
parskip=half-,
bibliography=totoc,
final
]{scrartcl}

\usepackage{synttree}
\usepackage{graphicx}
\setkeys{Gin}{width=.4\textwidth} %default pics size

\graphicspath{{./plots/}}
\usepackage[ngerman]{babel}
\usepackage[T1]{fontenc}
%\usepackage{amsmath}
\usepackage[utf8x]{inputenc}
\usepackage [hyphens]{url}
\usepackage{booktabs} 
\usepackage[left=2.4cm,right=2.4cm,top=2.3cm,bottom=2cm,headheight=25.60228pt,includeheadfoot]{geometry}
\usepackage{eurosym}
\usepackage{multirow}
\usepackage[ngerman]{varioref}
\setcapindent{1em}
\renewcommand{\labelitemi}{--}
\usepackage{paralist}
\usepackage{pdfpages}
\usepackage{lscape}
\usepackage{float}
\usepackage{acronym}
\usepackage{eurosym}
\usepackage[babel]{csquotes}
\usepackage{longtable,lscape}
\usepackage{mathpazo}
\usepackage[flushmargin,ragged]{footmisc} % left align footnote

%%url brekas grrr
\def\UrlBreaks{\do\a\do\b\do\c\do\d\do\e\do\f\do\g\do\h\do\i\do\j\do\k\do\l%
\do\m\do\n\do\o\do\p\do\q\do\r\do\s\do\t\do\u\do\v\do\w\do\x\do\y\do\z\do\0%
\do\1\do\2\do\3\do\4\do\5\do\6\do\7\do\8\do\9\do\-}%

\usepackage{listings}

\urlstyle{same}  % don't use monospace font for urls

\usepackage[fleqn]{amsmath}

%adjust fontsize for part

%% geometry
\clubpenalty = 10000 
\widowpenalty = 10000 
\displaywidowpenalty = 10000
%% tightlist

\providecommand{\tightlist}{%
  \setlength{\itemsep}{0pt}\setlength{\parskip}{0pt}}

\usepackage{sectsty}
\partfont{\large}

%Das BibTeX-Zeichen mit \BibTeX setzen:
\def\symbol#1{\char #1\relax}
\def\bsl{{\tt\symbol{'134}}}
\def\BibTeX{{\rm B\kern-.05em{\sc i\kern-.025em b}\kern-.08em
    T\kern-.1667em\lower.7ex\hbox{E}\kern-.125emX}}

\usepackage{fancyhdr}
\fancyhf{}
\pagestyle{fancyplain}
\fancyhead[R]{\thepage}

%meta

%meta

\fancyhead[L]{K. J. Kojakeva \\ %author
LIBREAS. Library Ideas, 29 (2016). % journal, issue, volume.
\href{http://nbn-resolving.de/urn:nbn:de:kobv:11-100238146
}{urn:nbn:de:kobv:11-100238146}} % urn
\fancyhead[R]{\thepage} %page number
\fancyfoot[L] {\textit{Creative Commons BY 3.0}} %licence
\fancyfoot[R] {\textit{ISSN: 1860-7950}}

\title{\LARGE{Bibliographien: Einsichten eines ihrer möglichen Leser - ein Rundgang
}} %title %title
\author{Krisztof Ján Kojakeva} %author

\setcounter{page}{}

\usepackage[colorlinks, linkcolor=black,citecolor=black, urlcolor=blue,
breaklinks= true]{hyperref}

\date{}
\begin{document}

\maketitle
\thispagestyle{fancyplain} 

%abstracts

%body
\section*{I. Bibliographie und
(Selbst-)Referenzialität}\label{i.-bibliographie-und-selbst-referenzialituxe4t}

\emph{Wenn ein Begriff im Beginn sein Einiges erkennen und aus sich
selbst das Viele sehen, beleuchten und verstehen will, in diesem Prozess
zur Zweiheit, ja eigentlich zum Vielen und Vielfachen fortschreiten
möchte und zunächst auf sich selber stößt}, mag zunächst befremdlich
wirken, den Gedanken der Bibliographie direkt an den Begriff des
Rekursiven, des Selbstreferenziellen zu binden. Dies kann dem Denken
geschehen, indem es annimmt, nur ein Text (im Sinne einer Darlegung von
schriftlich niedergelegten und logisch geführten Gedankengängen in der
Form sich bedingender und fortlaufender Zeichen) könne sich auf Quellen
beziehen und diese Form des Textes (der damit in Sinne einer wie auch
immer gearteten mathematischem Menge als jene der
\enquote{Nicht-Quellen} erscheint) im Wege stehen würde, gedanklich
unmittelbar eine Rekursion beziehungsweise eine Selbstzuwendung zu
ermöglichen Beziehungsweise diese auf sich selbst bezogene Hinwendung
überhaupt erkennbar werden zu lassen, da die Zitate und/oder ihre
verschiedenen Formen der Benennung ihrer Quellen (etwa im \emph{Chicago
Citation Style}, \emph{Fuß- und Endnoten}) sich immer auf
\enquote{Außentexte} beziehen und spontan eine Art Beziehung meinen, des
\emph{\enquote{entre-deux}} (zwischen Zweien).

Doch gerade die Bibliographie bedarf doch der Quellen, die außerhalb
ihrer selbst liegen, welche sie dann ohne Verwendung von Zitaten in
einen größeren Zusammenhang stellt. Letztlich: die Bibliographie stellt
das einzige abgeschlossene Werk dar, welches \emph{zugleich und allein}
aus der Nennung von Quellen besteht und aus diesem Zugleich und diesem
Allein und ihres Aufeinanderbezogenseins, das gedankliche Element des
Rekursiven entstehen lassen kann oder -- Kritiker würden behaupten --
dem Denken vorgegaukelt wird.\footnote{Ein kleines sprachlich
  motiviertes Denkexperiment: nimmt man an, der letzte Teil dieses
  Satzes würde heißen: {[}\ldots{}{]} \enquote{wurde aus diesem Zugleich
  und diesem Allein und ihres gegenseitigen Bezuges} {[}\ldots{}{]},
  besteht die Annahme, dass aufgrund des Begriffes der Gegenseitigkeit
  auch damit immer auch irgendwie die Zweiheit anklingt, die den Begriff
  der Rekursion spontan aufscheinen zu lassen vermeidet. (Vgl. Auch Anm.
  2, unten)} Die gestellte Frage, würde man meinen, klärt sich darin,
die Bibliographie wäre die einzige Art eines Verzeichnisses, welches
\enquote{seinen Text} zugleich als ihre Quellen in einer strukturierten
Ordnung anlegt und sie somit ihre Quellen alleine als Rechtfertigung
dieses, nämlich \enquote{ihres Textes} vorlegen kann, um gewissermaßen
(in der Intension eines Buches überhaupt) als ein abgeschlossener Teil
oder ein ganzes Werk zu erscheinen. Wie können die \emph{Denkvorgänge},
welche sich hinter dem \emph{Rekursiven} hervorgezupft, eingefangen,
beleuchtet werden, wie -- in einem Anflug einer allgemeinsten Bedeutung
(\emph{meaning}) -- dessen mannigfaltige Struktur erhellen, die sich
zunächst nur als eine Zeichenfolge begreifbaren Inhalts, als einen (auch
deskriptivistischen) Eintrag zu erkennen gibt?\footnote{In der
  Feststellung einer Synonymie des Referenziellen, das nicht
  grundsätzlich und immer von der Selbstbezüglichkeit spricht, kann
  geklärt werden, welche Begriffe der Form der Bibliographie stringent
  anheimgestellt werden können, das heißt von welchen Begriffen zu
  behaupten ist, sie würden den Begriff der Referenzialität einer
  Bibliographie richtigerweise veranschaulichen können. Schaut man sich
  die Kaskade von Worten an, die vermeintlich von Ähnlichem sprechen -
  Autopoiesis, Rückkoppelung, Rekursion, (Selbst-)Referenzialität,
  (Rück- oder Selbst-) Bezüglichkeit, gegenseitige Verschränkung,
  (Rück-)Verweis, labyrinthische Spiegelung, logische Paradoxie,
  Solipsismus, Zirkularität - kann intuitiv erfasst werden, welches von
  ihnen, die spezielle Bedeutung im Betreffnis einer Bibliographie würde
  aufnehmen können. Im wesentlichen ging es mir um begriffliche Grenzen
  spielerisch auszuloten, denn die Bedeutungen von referenziell und
  rekursiv -- wie hier experimentell zusammengehörig angenommen --
  können selbstverständlich in keinesfalls in eins gesetzt werden,
  obwohl sie aus der Perspektive einer Bezugnahme gedacht,
  außerordentlich nahe beieinander liegen. Zu den Rollen der
  Deskriptoren beziehungsweise Vorzugsbezeichnungen. Vergleiche:
  Expertengruppe RSWK des Deutsches Bibliotheksinstituts/Arbeitsstelle
  für Standardisierung et. al. (Hsgr.): Regeln für den Schlagwortkatalog
  {[}RSWK{]}, Fünfte Ergänzungslieferung (20093) Leipzig und Frankfurt
  am Main: DNB, p.~143, A 34 , A 65 (beide Anlage 6), Grundregeln, § 2,
  Pt. 7.}

In zweifacher Hinsicht ist dieses Gedankenspiel von Rekursion und
Bibliographie weiterzuführen unsinnig, denn:

\begin{itemize}
\tightlist
\item
  Die Bibliographie ist, wie alle anderen wissenschaftlichen
  Veröffentlichungen,auf die Nennung von \enquote{Außenquellen}
  angewiesen, und in Hinsicht dieses Aspekts nicht nur rekursiv
\end{itemize}

ebenso:

\begin{itemize}
\tightlist
\item
  Die Bibliographie ist auch dann nicht rekursiv zu nennen, wenn sie als
  eine Quelle in einer Fortsetzung ihrer selbst oder in einer weiteren
  Bibliographie erscheinen würde.\footnote{Die Frage bleibt also nicht,
    inwiefern die Logik der Rekursivität im Zusammenhang mit
    Bibliographien erhalten bleiben würde, (i) wenn eine Bibliographie
    Werke der Rekursivität verzeichnet, (ii) eine bestehende
    Bibliographie weitergeführt wird und sie sich selbst erneut zitiert
    bzw. als Quelle und Fortsetzung bestimmen muss und/oder (iii) als
    Werk in einer Bibliographie der Bibliographien aufgenommen wird.}
\end{itemize}

ebenso:

\begin{itemize}
\tightlist
\item
  Ich wiederholende: Hinwendung ist nicht zwingend rekursiv, sondern
  lediglich eine mögliche Form des Referenziellen.
\end{itemize}

In diesem Zusammenhang ist eine erste persönliche (und sicherlich
bleibende) Begegnung mit einem mehrbändigen, bibliographischen Werk,
hier nicht allein erwähnenswert, sondern Verständnisvoraussetzung, um
die von mir oben zur Darstellung gebrachte Verbindung zwischen
Bibliographie und Rekursion einsichtig aufzuweisen: Bevor weitere Jahre
persönlicher philosophischer Lektüre ins Land gehen sollten, beschloss
ich, dass eine Überblicksdarstellung der Philosophie doch zwingende
Vorbedingung sein müsse, die sich nicht aus einem bestimmten Blickwinkel
darstellt, sondern ein Versuch einer möglichst umfassenden,
geschichtlichen Zusammenstellung wäre. So kam es zu Beginn der 1990er
Jahre zur Subskription des mehrbändigen Werkes \emph{Handbuch der
Geschichte der Philosophie} von Wilhelm Totok; mit erster Lieferung
erreichten mich die 1964 in erster Auflage veröffentlichten Bände.

Als Gesamtwerk weisen sie einige Besonderheiten auf: erst der zuletzt
abgeschlossene Teil entspricht im engeren Sinne einer klassischen
Bibliographie, insofern, als dieser durchgehend auf Kommentare
verzichtet, während in den vorhergehenden Teilen einige Erläuterungen,
fast ausschließlich biographischer Natur, dem jeweiligen Philosophen im
Haupteintrag mitgegeben wurden (weshalb der Herausgeber sein Werk dann
auch unter dem Titel des \emph{Handbuches} und nicht der
\emph{Bibliographie} veröffentlicht wissen wollte). Interessant ist auch
das anzutreffende, ja verborgene, stetige evolutionäre Prinzip zu nennen
(oder wäre es als eine immer wieder auftretende Selbstbezüglichkeit
herauszuarbeiten?), welches/welche sich darin darstellt, dass eine
genannte Persönlichkeit mehrmals in Erscheinung treten kann, einmal als
früher Kritiker und in einem späteren Band unter dem eigenen
Haupteintrag. Dies ist sicher eine sehr angemessene Möglichkeit des
Aufweises, da sie \emph{eo ipso} das Zustandekommen der
Philosophiegeschichte zeigt; \emph{formaliter} hatte Totok das Problem
der strengen Chronologie damit ebenfalls elegant gelöst, indem später
erschienene Aufsätze und Werke immer in zweiter und dritter Ebene eines
im Haupteintrag genannten Namens erfasst, und falls vorhanden, die
Replik (im Sinne weiterer Sekundärliteratur) eines Drittautoren -- oder
eben eines neuen frühen Kritikers - ebenfalls dort würde untergebracht
werden können.\footnote{Dieses Lebenswerk -- aufgrund seines Umfanges
  und seiner zeitlichen Dimension sicherlich als solches zu erkennen --
  ist auch vor dem Hintergrund einer Ablösung der verschiedenen
  Zettelkataloge (Autoren-, Titel-. Systematischer bzw. Schlagwort- und
  Zugangskatalog) durch einen einzigen Onlinekatalog zu sehen, dessen
  Einführung viele Wissenschaftler damals auch als Bedrohung oder
  zumindest als eine große Einschränkung wahrnahmen.} Ab dem zweiten
Band erhielt jeder Eintrag eine fortlaufende Nummer, um eine präzise
Referenzierung im Index und in Querverweisen zu ermöglichen.\footnote{(Entgegen
  dieser referenziellen Klarheit und den jedem Band einleitenden
  Vorworten, gab es für mich als junger Leser Prozesse der Erstellung,
  die anfangs nicht vollständig einsichtig waren. In einem Brief an
  Wilhelm Totok berichtete ich davon, dass eine Zeitschrift wie
  TUMULT-Zeitschrift für Verkehrswissenschaft es durchaus verdient
  hätte, aufgenommen zu werden, zumal sie immer wieder in die Geschichte
  der Philosophie weisende Beiträge von JACQUES DERRIDA und GILLES
  DELEUZE publizierte (damals als Beispiele angeführt). DELEUZE fand
  lediglich in seiner Auseinandersetzung mit HENRI BERGSON Eingang im
  sechstem Band des Handbuch, DERRIDA wurde dort etwas breiter angelegt,
  so auch als Übersetzer. Der Verlag leitete das sehr nette
  Antwortschreiben, datiert vom 11. August 1992 an mich weiter, worin
  TOTOKs Bedauern ausgedrückt wurde, nicht vollständig aufnehmen zu
  können, insbesondere bei Zeitschriften wie TUMULT, welche sich nicht
  ausschließlich mit der Geschichte der Philosophie auseinander setzen
  würden. Rückschauend werte ich es nicht als gravierend, in der ersten
  Auflage des Handbuch, diese Zeitschrift nicht verzeichnet zu finden,
  jedoch dokumentierte TUMULT eingehend den Beginn des späteren
  \enquote{philosophischen Hype} um DERRIDA und DELEUZE in Deutschland.
  Wohl gab es einen weiteren, sehr verständlichen Grund: Wilhelm TOTOK
  stand nach vierzigjährigen Arbeit am Handbuch der Geschichte der
  Philosophie an dessen Abschluss. Mit einem auch tränenden Auge und im
  Verständnis um diese wirklich großen Aufwendungen für dieses Werk, wie
  es das Handbuch darstellt, muss gesagt werden, dass heute in einem
  Verbundkatalog mit wenigen Klicks beinahe jedweder Autor und fast
  jeder Beitrag gefunden werden kann: als permanent verlinkte
  Katalogseite und Positionen (gemeinsam als Permalink erfasst) die
  weitere Zugangsoptionen zu Buchreihen eröffnen, oder -- wenn auch
  seltener -- zu Rezensionen.}

\section*{II. Findelkind Bibliographie / Quellenangaben / Keine
Aufnahme
erfolgte?}\label{ii.-findelkind-bibliographie-quellenangaben-keine-aufnahme-erfolgte}

Die Methode erster Wahl findet sich in der Annäherung an meine
Fragestellung als adäquate Beschreibung bestehender Formen der
Bibliographie beziehungsweise des Bibliographierens, darin ist auch ihre
Ansprache als der eines Findelkindes zu verstehen.

In der Sichtung der schriftlichen Zeugnisse aus Literatur, Wissenschaft
und Kunst, können folgende Materialtypen als grundlegend herausgestellt
werden:

\begin{itemize}
\item
  \emph{Zeitung und Zeitschriften:} Sie geben in den Buchrezensionen
  meist nur minimale bibliographische Angaben der besprochenen Werke
  mit; Preisangabe und Seitenumfang sind immer Bestandteil der Nennung
  eines Buchtitels. Zeitungen und die sogenannten Publikumszeitschriften
  werden selbst lediglich in Ausnahmen als Quelle herangezogen;
  einschlägige, etwa soziologische Untersuchungen nehmen selbstredend
  als Quelle auf sie Bezug. In jährlichen Verzeichnissen der
  Medienbranche werden Titel und Auflagenzahlen angegeben; sie können
  auch in Bibliographien mit spezialwissenschaftlicher Themenausrichtung
  genannt werden.
\item
  \emph{Wissenschaftliche Bulletins}, (\emph{Tagungs-, Forschungs- und
  Kolloquiums-})Berichte, \emph{Rundbriefe} und (weitere)
  wissenschaftliche Fachzeitschriften, gegebenenfalss
  \emph{Aufsatzsammlungen} und oft auch \emph{Festschriften} führen die
  Werkangaben ihrer rezensierter Werke etwas vertiefter, indem etwa die
  ISBN / ISSN angegeben wird. Wissenschaftliche Zeitschriften
  beziehungsweise die Verfasser einzelner Beiträge werden stets im Sinne
  einer vollständigen Quelle bezeichnet / bibliographiert.
\end{itemize}

\begin{itemize}
\item
  Die extensive Verschlagwortung, welche in Zeitschriftenbeiträgen
  angetroffen wird, erinnert nicht selten an die
  \emph{Abstracts/Zusammenfassungen}, wie sie auch teilweise in
  Rezensionen für Bücher erstellt werden.
\item
  \emph{Romane, Erzählungen, Novellen} können Angaben zu einem Autor
  oder einem Titel des Werkes enthalten, die jedoch auch frei erfunden
  sein können, wie etwa in der Romanerzählung \emph{Das Glasperlenspiel}
  von HERMANN HESSE (eine erfundene Buchquelle als einführende
  Herausstellung versunkener Buchwelten, manchmal \enquote{geheimer
  Bücher}\footnote{Eine sehr sachliche, kurz gehaltene Zusammenfassung
    findet sich in Wikipedia unter
    http://de.wikipedia/wiki/Das\_Glasperlenspiel, während GetAbstract
    eine kostenpflichtige Onlineversion bereitstellt, vgl. http://
    www.get-abstract.com/de/zusammenfassung/klassiker/das-glasperlenspiel/3414
    ; (beide zuletzt abgerufen 22.März 2016) Die Zusammenstellung von
    MARIA-FELICITAS HERFORTH: Erläuterungen zu Hermann Hesse, Das
    Glasperlenspiel/Königs Erläuterungen und Materialien, Bd. 316
    (20063) Hollfeld: Bange, beinhaltet außerdem Referenzen zur
    Entstehungsgeschichte dieses Romans im Überblick.}) oder im Nachweis
  von Zitaten, wie etwa bei DONNA LEONs \emph{Tod zwischen den
  Zeilen}.\footnote{DONNA LEON: Tod zwischen den Zeilen. Comissario
    Brunettis dreiundzwanzigster Fall (2015) Zürich: Diogenes, p.~278
    {[}umnummerierte Seite{]}.}
\item
  Im wissenschaftlichen Gegenpart der Prosa, insbesondere des
  Sachbuches, sind gelegentlich unvollständige Angaben zu einem Werk
  oder als beschreibende Elemente im Text zu finden (unter anderem
  findet man im Sachbuch auch \emph{Auswahlbibliographien} oder
  \emph{Weiterführende Literatur}), wohingegen das Fachbuch (oft auch
  Monographie) und der sorgfältig editierte Ausstellungskatalog fast
  durchgängig immer vollständige bibliographische Angaben zu Quellwerken
  enthalten. Im Gegensatz zu Fachbüchern wird das Sachbuch wenig bis
  nicht als Quelle in selbständigen Bibliographien verzeichnet.
\item
  Als themenbezogene Sammlung in Buchform, kann die \emph{Anthologie}
  wissenschaftlichen Charakter haben; in diesem Fall werden ihr präzise
  bibliographische Nachweise mitgegeben und diese, andernfalls sind
  mindestens die Quellenwerke der zusammengetragenen Materialien benannt
  -- zusammen mit einer sie erläuternden Notiz oder umfangreichen
  Einführung / Nachwort des Herausgebers bedacht.\footnote{Vgl. etwa
    folgende mit dem Formschlagwort Anthologie versehene Ausgabe: HERWIG
    GÖRGEMANNS (Hsgr.): Die Griechische Literatur in Text und
    Darstellung, 5 Bde. (1998 /20042) Stuttgart: Reclam.}
\item
  Das \emph{Kunstbuch} operiert mitunter experimentell, indem eine als
  chic geltende größtmögliche Verkürzung der bibliographischen Angaben
  erfolgt, die sich zudem gleich in der Marginalie einer Seite
  beziehungsweise seitlich des Lauftextes befinden, analog zu den
  Quellen und Beschreibungen der Abbildungen (mitunter der
  Bildlegenden). Als ein besonderes Werk dieser Klasse dürfte - nicht
  ohne Augenzwinkern -- UMBERTO ECOs \emph{Die unendliche Liste}
  Erwähnung finden.\footnote{UMBERTO ECO: Die unendliche Liste (1985)
    München: Hanser. Die unendliche Liste wurde gleichzeitig als
    Ausstellung unter dem Motto LE LOUVRE INVITE UMBERTO ECO: MILLE E
    TRE (7. November 2009 -- 8. Februar 2010) gemeinsam mit dem Louvre
    realisiert und sie kann durchaus ebenso als \enquote{Ein begehbares
    Wörterbuch} interpretiert werden; vgl. den gleichnamigen Artikel,
    allerdings zum Museum Grimmwelt, Kassel, aus: NEUE ZÜRCHER ZEITUNG
    v. Samstag, 28. November 2015, p.~48.}
\item
  In Auktionskatalogen findet man, entsprechend ihrer Aufgabe, zwei
  Formen der bibliographischen Notation, zum einen wird diese zu einer
  bibliographisch-bibliophilen Gegenstandbeschreibung der für den
  jeweiligen Termin als Lose gelisteten Werke (seltenes Buch,
  mehrbändiges (seltenes/antiquarisches) Werk, Sammlung, Konvolut,
  Autograph, Manuskript, Typoskript, Notizbuch und so weiter); zum
  anderen können die zur Verifikation benutzten (Antiquariats- und/oder
  Kunstkataloge) im Sinne einer bibliographischen Verzeichnung am
  Schluss des jeweiligen Auktionskataloges angeführt werden. Kataloge
  von Auktionshäusern finden nur in seltenen Fällen Eingang in
  wissenschaftlichen Bibliographien (da sie meist eine von Kunstmuseen
  unabhängige, also auf einer eigenständigen kunstgeschichtlichen
  Bewertung beruhen und Quellen/Besitzerwechsel -- aus Gründen der
  gewollten Anonymität von Einlieferer und Käufer -- eigentlich immer
  verschwiegen werden), jedoch sind die in ihnen angeführten Gegenstände
  beziehungsweise ihre Buchbeschreibungen, bisweilen in andere
  Verzeichnisse aufgenommen, unter hinreichender Nennung des
  Auktionskataloges, um den Lesern die Konsultation der darin
  enthaltenen Quelltexte zu ermöglichen.
\end{itemize}

\begin{itemize}
\item
  Kartenwerke (\emph{Bildbände, zahlreich} und/oder \emph{durchgehend
  illustrierte Kinderbücher, Atlanten}, frühe geographische Literatur
  oder auch \emph{Stiche} nur imaginärer Landschaften, wie etwa
  \enquote{die Karte des Landes der Zärtlichkeit}, i.e.
  \emph{\enquote{La carte du Pays de Tendre}} von FRANÇOIS CHAUVEAU, ca.
  1754), \emph{synoptische Tafeln} (bzw. die sie begleitenden Textbände)
  weisen ihre Herstellung beziehungsweise die damit verbundenen
  Personen, Auftraggeber und/oder befasste Institutionen aus und
  enthalten selbst oft keine bibliographisch erfasste Quellenliteratur.
\item
  \emph{Thesauri} und Wortzusammenstellungen innerhalb fortgesetzter
  \emph{Lieferungen} (etwa ERICH MATTERs \emph{Deutsche Verben 1-10},
  aus dem Jahr 1966, verlegt bei VEB Bibliographisches Institut Leipzig)
  verzeichnen die durchgearbeiteten (Bild-)Wörterbücher, Lexika und
  berufsbezogene Literatur, wie etwa technische Spezifikationen und
  ähnliches. Das \emph{Post Skriptum} in seinen vielfältigen Formen,
  insbesondere der bibliographische Dank (\emph{\enquote{bibliographic
  aknowledgement}}) mancher neuer, besonders englischer und
  amerikanischer Fach- und Sachliteratur, kann auf einen Mangel der
  deutschen Handhabung bei der Erwähnung von Quellen hinweisen, eine
  Handhabung, die es vernachlässigt oder gar verbietet, den äußeren und
  inneren, gedanklichen Prozess der Buchentstehung vollständig
  chronologisch zu dokumentieren, da in diesem Fortgang oftmals eine
  Zeitungsmeldung, ein Name, eine Institution, eine Internetseite oder
  ein mündlich gegebener Hinweis auf ein Werk in der realen Abfolge --
  man würde meinen: wild durcheinander -- als immer auch inspirierende
  Quellen vom Autor genannt werden dürfen. Dies bedeutet, eine
  Unterscheidung zwischen einer Sekundärliteratur, die -- e\emph{x otio
  academicum} -- zitiert werden soll und jener der Tertiärliteratur
  (Lexikon, Handbuch, Bibliographie, Synchronopse), welche in
  Monografien und in Druckwerken, teils auch in einem Band vereinigter
  \emph{Aufsätze, Materialien} bzw. \emph{Beiträge} (analog zur
  \emph{Anthologie}), \emph{Loseblattsammlungen} als nicht immer
  zitierfähig gelten, wird diese Herangehensweise im \emph{bibliographic
  aknowledgment} aufgegeben.Ironischerweise kann man feststellen, dass
  hervorragende (oftmals jene als \enquote{Klassiker} bezeichneten)
  Werke, vornehmlich amerikanischer und englischer Forscher, diese
  Unterscheidung schon seit den 1970er Jahren für hinfällig bewerten und
  sie heute mittlerweile auch in Europa eine (mitunter noch)
  unausgesprochene Norm darstellt, nämlich jede in Anspruch genommenen
  Quelle zumindest im bibliographischen Teil auch vollständig zu
  rubrizieren. Die einst feine Trennlinie zwischen wissenschaftlicher
  (Primär-, Sekundär- und Tertiär-)Literatur in Bezug auf ihre Aufnahme
  oder Nichtaufnahme von vornherein festzulegen, hat sich nicht bewährt
  und vermag heutigen wissenschaftlichen Standards auch nicht mehr zu
  genügen.
\end{itemize}

Alle bis hierhin aufgeführten Medien und Darstellungen enthalten oder
können bibliographische Einträge enthalten; diese Form der Bibliographie
ist immer eine unselbstständige oder auch kryptische. Bibliographische
Zwischen- oder Übergangstypen finden sich nicht nur historisch sondern
auch in modernen, eben vielleicht selbständigen oder scheinselbständigen
Buchausgaben. Die zwei für dieses Kapitel hier abschließend gewählten
Beispiele betreffen einerseits einen Jubiläumsband für EDITION LEIPZIG
und eine Zusammenstellung referenzierter psychologischer Literatur aus
dem 16. Jahrhundert aus dem Verlage Georg Olms.\footnote{ANGELIKA
  BÖTTCHER (Zus.): Zwanzig Jahre Edition Leipzig - eine Bibliographie
  (1983) Leipzig: Edition, von der Begrifflichkeit der selbständigen
  Bibliographie ausgegangen, ist dieser und der zweite, untenstehende
  Titel ebenso dazugehörig, wie sie es beide auch nicht sind. ANGELIKA
  BÖTTCHERs Zusammenstellung aus dem Jahre 1983 ist, obwohl in sich
  abgeschlossen, einerseits lediglich einer der Bände, welche Edition
  Leipzig zur Darstellung in Buchform der Verlagsproduktion zu Jubiläen
  herausbrachte und welche zugleich einen im Umfang ansehnlichen,
  präliminierenden bzw. fortführenden bibliographischen Teil besitzen
  (jeweils enthalten in: Zehn Jahre Edition Leipzig, 1969, und
  Fünfundzwanzig Jahre Edition Leipzig (das ist: \enquote{Ansichten zu
  einer Verlagsgeschichte}, 1985). Jubiläumsveröffentlichungen sind wie
  hier, als Fundus zwischen fortgesetzter Lieferung in unregelmäßiger
  Erscheinungsweise und Reihentitel anzusehen, was sich in ihrer
  ambivalenten Bewertung widerspiegelt. Ein weiteres Beispiel bildet das
  Buch von HERMANN SCHÜLING: Bibliographie der psychologischen Literatur
  des 16. Jahrhunderts / Studien und Materialien zur Geschichte der
  Philosophie, Bd. 4 (1967) Hildesheim: Olms Innerhalb der Reihe Studien
  und Materialien zur Geschichte der Philosophie, welche 1965 mit der
  Publikation WILHELM RISSEs einer Bibliographia Logica in vier Teilen
  beginnt, entspricht auch SCHÜLINGs Zusammenstellung zwar einer in sich
  abgeschlossenen Bibliographie, dennoch stellt sie zugleich einen Band
  dieser Reihe dar; auch hier muss die Frage betreffend ihrer
  Abgeschlossenheit oder eben Nichtabgeschlossenheit unbeantwortet
  bleiben oder man geht den Weg des Kompromisses indem man behauptet,
  sie wären jeweils in sich selbst abgeschlossen (sequelled).}

\section*{III. Näherungen an die Definition der
Bibliographie}\label{iii.-nuxe4herungen-an-die-definition-der-bibliographie}

\paragraph{A. - Arbeiten auf Papier (in
Auszügen)}\label{a.---arbeiten-auf-papier-in-auszuxfcgen}

Zedler: Grosses vollständige Universal-Lexikon in 25 Bänden (1993 -1999
\textsuperscript{2}) Graz: Akademische Druck- und Verlagsanstalt
(Photomechanischer Nachdruck d. Ausgabe Halle und Leipzig, 1732-1754) (
- keine Verzeichnung)

J. S. Ersch / J. G. Gruber: Allgemeine Enzyklopädie der Wissenschaften
und Künste (1818-1889) Graz: Akademische Druck- und Verlagsanstalt
(Nachdruck, 1970)

\begin{itemize}
\item
  {[}\ldots{}{]} Bibliographie ist der Name derjenigen Wissenschaft,
  welche sich mit der Kenntnis der Schriftsteller-Erzeugnisse aller
  Zeiten und Völker, sowohl an sich, als auch einzelnen dieser Umstände
  beschäftigt.{[}\ldots{}{]}
\item
  {[}\ldots{}{]} weniger üblich Bibliognosie, Bibliologie
\item
  {[}\ldots{}{]} in älterer Zeit {[}\ldots{}{]} Schreiber, später auch
  Abschreiber, zuweilen auf Buchdrucker übertragen (Montfaucon, Paläogr.
  Graeca, p.~351) {[}\ldots{}{]}
\item
  reine Bibliographie:

  \begin{itemize}
  \item
    innere Bibliographie
  \item
    angewandte, äußere, beschreibende, historische Bibliographie
    {[}\ldots{}{]} (J. Ersch: Handbuch der deutschen Literatur, 1750)
  \item
    äußere: Umstände, Meinungen und Bedürfnisse des Sammlers
    {[}\ldots{}{]} dass Bücher schätzbar werden {[}\ldots{}{]} äußerer
    Gründe und Bedingungen
  \end{itemize}
\item
  Hilfswissenschaft der Bibliographie
\item
  Biblioteca italiana
\item
  Lit. Anzeiger
\item
  Bibliographie, orientalisch: d.i. arab{[}ische{]}, pers{[}ische{]},
  türk{[}ische{]} Bücherkunde{[} \ldots{}{]}
\end{itemize}

Meyers enzyklopädisches Lexikon -- mit 100 signierten Sonderbeiträgen
(1971-1985 \textsuperscript{9}) Mannheim: Bibliographisches Institut
{[}eckige Klammern im Original, außer Auslassungsvermerke{]}

\begin{itemize}
\item
  Buchbeschreibung
\item
  {[}\ldots{}{]} ursprünglich (bereits im 5. vorchristl. Jh)
  {[}\ldots{}{]} :

  \begin{itemize}
  \item
    Bezeichnung für das Abschreiben mitunter für das Schreiben von
    Büchern
  \item
    im Gegensatz zu Bibliothekskatalogen unabhängig von einer
    Büchersammlung
  \end{itemize}
\item
  Lehre von den Literaturverzeichnissen, ihrer Benutzung und Herstellung
\item
  Hilfsmittel für Bibliotheken
\item
  Erscheinungsform: selbständig

  \begin{itemize}
  \item
    unselbstständig, versteckte, kryptische Bibliographie
  \item
    Literaturübersichten in Büchern
  \item
    als Anhang
  \item
    regelmäßige Beiträge in Zeitschriften
  \end{itemize}
\item
  Erscheinungshäufigkeit: laufende / periodische

  \begin{itemize}
  \tightlist
  \item
    abgeschlossene / retrospektive, ergänzt durch Register
  \end{itemize}
\item
  Kumulativ-Bibliographie
\item
  Inhalt: Allgemein-Bibliographie

  \begin{itemize}
  \item
    Fach-Bibliographie
  \item
    Regional-Bibliographie
  \item
    Lokal-Bibliographie
  \item
    Personal-, Biobibliographie
  \item
    auch: buchhändlerische, bibliophile, wissenschaftliche Verzeichnisse
  \end{itemize}
\item
  Umfang: national, international,

  \begin{itemize}
  \item
    {[}\ldots{}{]} vollständig {[}in den verschieden definierten
    Grenzen{]} der nationalen Literatur
  \item
    Auswahl-Bibliographie
  \end{itemize}
\item
  Material kann geordnet sein: alphabetisch nach Autor, Sachtitel und
  anonyme Schriften

  \begin{itemize}
  \item
    oder Sammelwerken oder Schlagworten, in systematischen Sachgruppen
  \item
    chronologisch, topographisch
  \end{itemize}
\item
  Stufenordnungen: keine, kombiniert
\item
  besondere Listen: Zeitungen, Zeitschriften, Hochschulschriften,
  amtliche Druckschriften
\item
  Primärbibliographie, Nationalbibliographie, Pflichtexemplargesetze
\end{itemize}

Die große Bertelsmann Lexikothek / Grundbände (1984-1989) Gütersloh:
Bertelsmann

\begin{itemize}
\item
  Bücherverzeichnisse nach Schlagworten
\item
  Methodik u. Herstellung von Schriftverzeichnissen
\item
  {[}\ldots{}{]} im Unterschied zu Bibliothekskatalogen:
  Dokumententation {[}\ldots{}{]}
\item
  Auswahl-, Fach-, Spezial-, Personal-Bibliographie
\item
  annotierte, empfehlende Bibliographie (bibliogr. Raisonnée)
\item
  versteckte (unselbständige) Bibliographie
\item
  {[}\ldots{}{]} als bibliographisches Hilfsmittel leisten
  Zettelkataloge d. Bibliotheken

  \begin{itemize}
  \tightlist
  \item
    große Dienste {[}\ldots{}{]}
  \end{itemize}
\item
  Gesamtkatalog d. Prorussischen Bibliographie (später Deutscher
  Gesamtkatalog)
\item
  {[}\ldots{}{]} Zusammenfassung älterer Buchverzeichnisse ist das
  Gesamtverzeichnis des

  \begin{itemize}
  \tightlist
  \item
    deutschsprachigen Schrifttums (GV)
  \end{itemize}
\item
  bibliographieren: den Titel einer Schrift verzeichnen oder genau
  feststellen
\end{itemize}

Brockhaus Enzyklopädie in 30 Bänden (2006\textsuperscript{2}) Leipzig:
F.A. Brockhaus

\begin{itemize}
\item
  Bücherbeschreibung
\item
  Notitia librorum (hist.)
\item
  Buchgeschichte
\item
  Verzeichnis von Literaturnachweisen
\item
  Teilgebiet der Bibliothekswissenschaft
\item
  {[}\ldots{}{]} im Unterschied zu Katalogen von Bibliotheken
  {[}\ldots{}{]}
\item
  ISBD
\item
  {[}\ldots{}{]} Primär-Bibliographie {[}\ldots{}{]} Autopsie
  {[}\ldots{}{]}
\item
  {[}\ldots{}{]} Sekundär-Bibliographie {[}\ldots{}{]} aus vorliegenden
  Bibliographien erstellt
\item
  Allgemein-Bibliographie: internationale allg. Bibliographie

  \begin{itemize}
  \tightlist
  \item
    National- und Regional-Bibliographie
  \end{itemize}
\item
  laufende, periodische, retrospektive Bibliographien,
  Epochen-Bibliographie
\item
  chronologische, systematische, Indices, Stich- /
  Schlagwortbibliographie
\item
  Register
\item
  laufende, periodische Bibliographie
\item
  Fachbibliographie
\item
  Auswahlverzeichnisse nach Publikationsform
\item
  Inkunabel-Bibliographie
\item
  Hochschulschriftenverzeichnisse
\item
  Zeitschriften-Bibliographie
\item
  Bibliographien nach bestimmten Medien: Tonträger-Bibliographien

  \begin{itemize}
  \tightlist
  \item
    Bibliographie Online-Publikationen
  \end{itemize}
\item
  Personal-Bibliographie
\item
  {[}\ldots{}{]} Nachschlagen wird als bibliographieren bezeichnet
  {[}\ldots{}{]}
\end{itemize}

\paragraph{B. - Eklektik der
Elektrik?}\label{b.---eklektik-der-elektrik}

In der Deutschen \textbf{Wikipedia}
(http://de.wikipedia.org/wiki/Bibliographie, abgerufen: 11. Januar 2016)
führte dies zur folgenden Umschreibungen:

\begin{itemize}
\item
  {[}\ldots{}{]} ist ein eigenständiges Verzeichnis von
  Literaturnachweisen {[}\ldots{}{]}
\item
  {[}\ldots{}{]} sind ein unerlässliches Hilfsmittel in der Wissenschaft
  zur Erschließung von

  \begin{itemize}
  \tightlist
  \item
    Literatur {[}\ldots{}{]}
  \end{itemize}
\item
  {[}\ldots{}{]} zum Beispiel in Bibliotheken (dort gesammelt in Form
  von Bibliothekskatalogen) {[}\ldots{}{]}
\item
  Allgemeinbibliographien, Fachbibliographien
\item
  Nationalbibliographien (Universitätsschriften und andere)
\item
  Mundaneum {[}\ldots{}{]}\footnote{Vgl. EVGENIJ IVANOVICČ ŠAMURIN:
    Geschichte der bibliothekarisch-bibliographischen Klassifikation
    (1967) Leipzig: VEB Buch- und Bibliothekswesen. Im zweiten Band
    finden sich zahlreiche Fundstellen zu PAUL OTLET im Zusammenhang mit
    der FID und IIB.}
\item
  Literaturangabe, Literaturdatenbank, Literaturverzeichnis,
  Mediagrafie,

  \begin{itemize}
  \tightlist
  \item
    Regionalbibliographie
  \end{itemize}
\end{itemize}

Bei \textbf{Woxikon} (http:// www.woxikon.de , aufgerufen am 13. Februar
2016, Verkürzungen von mir) finden sich im Bereich der Synonymabfrage:

\begin{itemize}
\item
  Bibliographie, Buchkunde, -wissenschaft
\item
  Literaturverzeichnis, -angabe, -hinweis, -nachweis, Quellen, -angabe,
\item
  Schrifttumsnachweis
\item
  Titelangabe, -verzeichnis
\end{itemize}

In \textbf{Google Bücher}, einem Dienst von Google, der eine Sicht in
die Werke via sogenannter Snippet-, Seitenansicht oder Titelbeschreibung
erlaubt, fanden sich folgende Fundstellen: (https://books.google.com)

Andreas Lawaty, Wiesłav Mincer (unter Mitwirkung von Anna Domańska)
{[}Hsgr.{]}:

Deutsch-polnische Beziehungen in Geschichte und Gegenwart, Bibliographie
1900-1998, Bd. 1 (2000) Wiesbaden: Harrassowitz (p.~5):

\begin{itemize}
\item
  {[}\ldots{}{]} Die Bibliographie ist gleichermaßen als eine
  wissenschaftliche Dokumentation

  \begin{itemize}
  \tightlist
  \item
    von Forschungstraditionen, {[}\ldots{}{]} ; {[}ist ein{]}
    Hilfsmittel {[}\ldots{}{]}
  \end{itemize}
\end{itemize}

Moshe Zuckermann / Institut f. deutsche Geschichte, Tel Aviv
{[}Hsgr.{]}: Ethnizität, Moderne und Enttraditionalisierung (2002)
München: Wallstein (p.~429):

\begin{itemize}
\tightlist
\item
  {[}\ldots{}{]} Der Bibliographie ist naturgemäß ein Index angegliedert
  {[}\ldots{}{]}
\end{itemize}

Maria-Luise Mayr-Caputo/Julius M. Herz (Hgr.): Franz Kafka -
Internationale Bibliographie der Primär- und Sekundärliteratur (dt. Und
engl.), {[}Bd.1 , Bd. 2 in zwei Teilen{]} (2001 \textsuperscript{2})
München: Saur (Titel)

\begin{itemize}
\tightlist
\item
  {[}\ldots{}{]} Franz Kafka: Internationale Bibliographie der Primär-
  und Sekundärliteratur, Bd. 1{[}\ldots{}{]} Bd. 2 {[}\ldots{}{]}
\end{itemize}

Steffen Wenig (Hsgr.): Neueste Feldforschungen im Sudan und in Eritrea
(Akten des Symposiums vom 13. bis 14. Oktober 1999 in Berlin) (2004)
Wiesbaden: Harrassowitz (p.~217)

\begin{itemize}
\tightlist
\item
  {[}\ldots{}{]} Die Bibliographie ist gegliedert {[}\ldots{}{]}
\end{itemize}

Gewiss sind die Ergebnisse der Internetaufrufe lediglich als Blitzstudie
beabsichtigt, als ein nur kurzer Moment des Lichterns einer vielleicht
inspirierenden Flamme, indem Gegenüberstellungen, Abgrenzungen durch die
Logik / Konjunktionen der Sprache der oder Gliederungen in ihr
aufleuchten und die anhand der hier noch undefinierten Nähe zum Suchwort
innerhalb des Textes aufgetreten sind.\footnote{Betreffen alle
  Internetquellen und ihre Aufrufe zwischen dem 11. / 12. Januar -
  1./2./3. März 2016. im Wechsel von ca. sieben Stunden unter
  Berücksichtigung des veränderten Algorithmus und im Wechsel zwischen
  Bibliothek, Internetcafé und Zuhause; vgl. zu diesem Vorgehen: DIRK
  LEWANDOWSKI /GDI: Web Information Retrieval -- Technologien zur
  Informationssuche im Internet (2005) Frankfurt am Main: Deutsche
  Gesellschaft für Informationswissenschaft und Informationspraxis. Zu
  den hier ebenfalls relevanten Themen des Big Data in den Belangen Text
  Mining natürlicher Sprachen, Information Retrieval, Statistische
  Methoden ist die Literatur zu sehr im Wandel, als dass die ungestüme
  technische Veränderung ausreichend in ihr abgebildet werden könnte.
  Digital Humanities ( engl. etwa: rechnergestützte Humanwissenschaften)
  bildet jedoch ein begriffliches Gerüst, welches diesen Wandel seit
  einigen Jahren in der englischsprachigen Fachliteratur in hohem
  qualitativen Anspruch begleitet und gleichzeitig zu einem kritisches
  Sammelbecken für die verschiedenen Strömungen von der Biologie und
  Medizin über Ökologie bis hin zu Geschichte und Wissenschaftstheorie
  sich entwickelt.}

\section*{IV. Wünschbare
Bibliographien?}\label{iv.-wuxfcnschbare-bibliographien}

Ganz zu Beginn, sei es, man erklimme einen Berg und hoffe auf eine
möglichst aufschlussreiche Rundsicht -- ja und möge der Blick sich im
Land verlieren -- ; ganz zu Beginn steht das Aufleuchten eines
Gedankens, von dem man glaubt, er sei der eigene, der einem ganz allein
gehöre; so beginnt jedes Werk, insbesondere das künstlerische -- doch,
weshalb jetzt schon von Werk sprechen, wo man doch vorerst und nun im
sprichwörtlichen Boot sitzt, das einen auf den Fluss der Gedanken einmal
erst auf die Reise nimmt?

So ging ich durch die Stadt und stieß in einer Auslage auf einen
interessanten Fund: \enquote{Leporellos im Schuber}, ging es ganz
offensichtlich um Kunst oder doch um Architektur, um Mathematik? - war
es, weil in viele Richtungen zugleich deutbar und daher flüchtig, allein
deswegen schon ein Kunstwerk, dass nun, in der Reproduktion eines
unvollendet gebundenen Werkes (aus dem Blickwinkel der traditionellen
japanischen Buchbinderei entspricht ein Leporello immer einem
unvollendeten Buch), welches etwas verlassen im Schaufenster lag.:
\textbf{Topographies} von MONICA URSINA JÄGER. Ich stelle mir vor:
Doppelseite für Doppelseite durchblättern -- aufspannen wie ein Himmel,
nur für mich allein, so wie einst eine Wolldecke das Stubenhaus war --
es um sich herum aufstellen, wie die Lichtspielpanoramen umherziehender
Schausteller mit Wochenschauen, in den Zeiten des frühen Films -- wie
ein Bund Worte des Feststehenden, die einen (oder den?) Rhythmus des
Wassers einfangen wollen?

Ich habe mir erlaubt, einen kurzen Ausflug ins Imaginäre zu unternehmen
und die Frage zu stellen, wenn von Prozessen die Rede ist, wie wären sie
beschreibbar oder würden es werden, in einem Augenblick, da mein Auge
noch nicht einmal die Zeile oder das Wort bewusst erblickt, bereits mit
der Idee in Gedanken und später, nachdem mein Auge den Spuren der
Zeichen auf der Seite über viele Zeilen hinweg gefolgt ist, jenen
Begriff erblicke, der sich doch gerade noch als mein eigener Einfall
ausgab und mir jetzt ein Anderer entgegentritt. Doch auch wenn mein Auge
später das Wort nicht auf der Buchseite stehen sieht, hat dieses mich
oft schon gesucht und in einem selbst, eine Wohnung gebaut oder eben
einen Himmel, von dem ich immer fühle (und deshalb zu wissen vermeine):
er sei diese möglichst aufschlussreiche Rundsicht, von der ich eingangs
sprach.\footnote{Vergleiche in diesem Zusammenhang: \enquote{Wenn ich
  dieses Blatt wegwerfe / werfe ich nicht das Gedicht weg / dein Kopf
  wird zum Haupt / falls du dich verneigst vor ihm} / {[}\ldots{}{]},
  aus: Der Astronaut, in: JÜRGEN THEOBALDY, Der Nachtbildsammler (1992)
  Köln: Palmenpresse. (http://www.lyriklover.de, abgerufen am 25.
  3.2016)}

Stelle mir vor, eine Bibliographie würde einen solchen Prozess
versinnbildlichen, dieser könnte individuell in einer Zusammenstellung
aller bisher gelesenen Bücher seinen Niederschlag finden, auch als
individualpsychologische Gesprächsvariante auftreten, oder einem ganz
anderen Mäander folgen, wo eine spätere, einst mögliche Ordnung gefunden
geglaubt wird. Doch: ist da nicht schon das Handbuch, welches -- einmal
mit mehr, einmal mit weniger Begleittext -- den Leser erreicht? Ist da
nicht auch schon das Lexikon mit umfangreichen Artikeln oder kurzen
Worterklärungen und Literaturangaben am Textende? Sind es nicht
Verzeichnisse, die allen Werken dieser Literaturgattung schon zugrunde
liegen?

Wahrscheinlich ist eine andere, tiefer liegende Frage vornehmlich zu
klären: sind die Bibliographien im Sinne eines \emph{Leitfadens}, eines
\emph{Vademekums} oder mehr als die einer \emph{Ikonographie} oder nur
als \emph{Requisit} oder eben, als den eines Wandlungsprozesses zu
denken? Können abgeschlossene Werke dies und mehr noch leisten, wenn sie
zugleich eine allgemeine Teilhabe versprechen, wie dies gerade in und
mit den \emph{Diensten im Web} geschieht?

Einigen Beispielen bin ich begegnet, die auf die eine oder andere Weise
einen Prozess abbilden:

\begin{enumerate}
\def\labelenumi{\arabic{enumi}.}
\item
  Textiles Bibliography - a joint production of the Textile Society of
  America and the Textile Museum (1998 -- 2008) Earlville, Md: Textile
  Society (pers. Einordnung: Bibliographie als regelmäßige Lieferung
  beziehungsweise jährlich erscheinende Zeitschrift)
\item
  JONAS FANSA: Dem Geschmack auf der Spur. Eine Gewürzweltreise.
  Handbuch (2008) Berlin: Die Werkstatt (kulinarische und
  kulturhistorische Artikel zu den Gewürzen, umfangreiches
  Literaturverzeichnis)
\item
  Arbeitsgemeinschaft hist. Forschungseinrichtungen (AHF): Historische
  Bibliographie Online und Jahrbuch der historischen Forschung (1990 --
  ) {[}Impressum:{]} Berlin: Walter de Gruyter / Oldenbourg
  (Onlinebibliographie, dadurch stetiger Wandel bzw. Hinzufügungen,
  klassisches Erscheinungsbild, Artikelformat ähnlich zu Wiki,
  Bio-bibliographisches Literaturverzeichnis)
\item
  Centre Pompidou, {[}et al.{]}: Encyclopédie nouveaux médias / New
  media encyclopedia / Enzyklopädie neue medien ({[}wahrsch.{]} 1998-)
  Paris, et al. (http://www.newmedia-art.org) \emph{(Div. Zugänge wie 3,
  extensive mehrsprachige Bibliographie)}
\end{enumerate}

Die Form der Bibliographie hat sich selbstverständlich mit den
Trägermaterialien im Gespann stark gewandelt. Zunächst ist in
internetbasierten Dokumenten und Plattformen immer unklar, welche
Aufgabe sie wirklich erfüllen wollen und können, das heißt der
dargestellte Prozess ließe sich grundsätzlich in zwei Formen mit dem
traditionellen Druck verwirklichen: Annotierte Bibliographie und des
Handbuch mit einem mächtigen Dokumentenverzeichnis.

Von einer wesentlich vereinfachten Form ist bei einer synchronoptischen
Darstellung auszugehen. Hier stellt sich die Frage, wie sehr diese
letztlich nur ein Stichwortverzeichnis darstellt, indes, auch diese Form
der zeitlichen Verortung von Stichwörtern könnte wenig heterogen mit
einer bibliographischen Veröffentlichung gemeinsam in Erscheinung
treten, wie vor nicht allzu langer Zeit es Lexika mit instruktiven,
didaktischen Postern taten.\footnote{Auf eine besondere Publikation
  eines Faltbuches sei noch aufmerksam gemacht -- mit Fokus Stichwörter
  in ihrer Geschichte: ESOMAR (Celebrating 60 Years): Insight Track --
  The Evolution of Market Research (2007) Amsterdam / London: ESOMAR
  World Research, Fortune Street. Beschreibung: Bilden die vorderen
  Falzkanten des Leporellos den ersten Zugang zu einer auf verschiedenen
  Ebenen (zugleich Vorder- und Hintergrund betreffend des aufgespannten
  Rectos, das in themenbezogenen ungesättigten Farben erscheint) der
  teils technischen Einflüsse auf die Marktrecherche bezogenen
  Zeitlinien mit stichwortartig beschriebenen Ereignissen, ist idas
  Veraso künstlerisch illustrativ gestaltet mit verschiedenen
  historischen Themenmotiven aus dem Alltag ab 1790 bis in die
  Jetztzeit; frontseitig ist der Buchstabe i in den Umschlag gestanzt,
  wodurch ein Teil der Rückseite des Leporellos selbst wiederum sichtbar
  bleibt. Auf den Innenseiten des Umschlages werden Statistiken der
  weltweiten Markt- und Meinungsforschungsumsätze vereinfacht
  dargestellt. Der Umschlag (wie ein Schutzumschlag um das Faltbuch
  gelegt) ist in den Farben Weiß und Blau gehalten.}

Stelle ich mir vor -- und wage nochmals den Schritt ins Imaginäre --
eine Bibliographie der Lieferwege von Gütern und/oder Nahrungsmitteln,
welche Grenzen würde man ziehen? Wären für die verschiedenen Länder
jeweils separat angegebenen Themen, wie Qualitätsmanagement,
Betriebslehre und Buchführung, Zertifizierungen, Normen, Transportmittel
bereits so ins Allgemeine hineinreichend, dass sie -- schon des Umfanges
wegen -- ausgelassen werden müssten? Oder -- auf die Sprachen bzw.
Regiolekte Chinas schauend -- welche Werke würden sinnvollerweise
einzubeziehen sein, wenn von den zweihundertvierzehn Radikalen
ausgegangen wird? Wäre chinesische Fachliteratur zur Linguistik
einbezogen, Autoren genannt, die zu den Begründungen verschiedener
europäischen Schrifttranskriptionen der chinesischen Zeichen Stellung
bezogen haben? Sind Tabellenwerke der verschiedenen, weltweit zur
Anwendung gelangenden Transkriptionssysteme und -sprachen aufzunehmen,
oder wären es Werke, die sich mit der Kunst der Stempel- und
Druckverfahren befassend, ausgeschlossen bleiben müssen?

Diese Fragen sind mehr als nur rhetorisch. Sie deuten darauf hin, dass
auch Bibliographien selbst die Landschaft der Wissenschaft(en)
mitprägen, sie gegeneinander abgrenzen oder sie ergänzen. Die Didaktik
der Bibliographie entspricht ein Stück weit auch der Wahlfreiheit
der/des zusammenstellenden Autorin/Autors und des publizierenden
Verlages.\footnote{Hier werde ich dann doch an die Aussage GEORG CANTORs
  erinnert: \enquote{Das Wesen der Mathematik ist ihre Freiheit}.
  (Quelle: Wissenschaftler- Würfel , Halle (Saale).}

\section*{V. Möglichkeiten der Dynamik und Belege von Prozessen in
der
Bibliographie?}\label{v.-muxf6glichkeiten-der-dynamik-und-belege-von-prozessen-in-der-bibliographie}

\paragraph[1. Vorschläge zur Verfeinerung der
Verschlagwortung]{\texorpdfstring{1. Vorschläge zur Verfeinerung der
Verschlagwortung\footnote{Der folgende Abschnitt enthält mögliche
  Vorschläge zu Möglichkeiten der Nutzung der Herangehensweisen und
  Mitteln der qualitativen Sozialforschung einhergehenden
  Ergebniserreichung zu diesem Thema. Natürlich gibt es auch die
  Recherche innerhalb der Netzwerke des Social Media, welche für
  Wissenschaftler außerdem attraktiv sein kann, die sich jedoch zeitlich
  ebenso intensiv gestalten können wie das Aufsuchen und den eigenen
  Blick in Bibliographien. Ein weiterer Aspekt findet im folgenden
  Aspekt keine weitere Berücksichtigung: die ebenso als dynamisch zu
  bezeichnenden Denk- und Gestaltungsprozesse wie sie den Bibliographien
  zugrunde liegen, weshalb hier -- nur andeutend -- zwei Werke genannt
  werden, die -- wie ich finde - den iterativen Charakter für die
  Erstellung einer Bibliographie in nuce veranschaulichen: Zum einen sei
  genannt: CHRISTOPHER HAMLIN: More than Hot. A short History of
  Fever/Johns Hopkins Biographies of Deseases, vol 4 (2014) Baltimore:
  Johns Hopkins University Press. Diese medizinhistorische Buchreihe
  verfährt in Längsschnitten als Wegweiser und kann als eine Variante
  des externalisierten Prozesses für die Erstellung von Bibliographien
  genannt werden. Ein zeitlich viel weiter zurückliegendes Werk von
  JOHANNES HESSEN, Religionsphilosophie in zwei Bänden (1: Methoden und
  Gestalten und 2: System der Religionsphilosophie, beide 1948 und
  19552) veranschaulicht dies ebenso; während Hessen im ersten Band die
  Differenzen der einzelnen Persönlichkeiten und Schulen in diesem Fach
  herausarbeitet und die anzutreffenden Schwierigkeiten genauer
  Abgrenzungen im Bereich der Religion und Philosophie verständlich
  macht, bildet der zweite Band den Versuch, die gewonnenen Ergebnisse
  in ein Gesamtsystem, wissenschaftlich zu verorten. Gerade bei den hier
  erwähnten Randdisziplinen Medizin und Religionsphilosophie, können
  diese Verfahrensschritte gut beobachtet werden, wie sie ähnlich in
  Diskussion und Bestimmung bei der Findung geeigneter Deskriptoren aus
  dem Sprachkorpus der linguistischer Forschung verlaufen.}}{1. Vorschläge zur Verfeinerung der Verschlagwortung}}\label{vorschluxe4ge-zur-verfeinerung-der-verschlagwortung16}

Bereits mit dem Beginn dieses Textes lässt sich ein mehr auf die Belange
stetiger und fortlaufender gedanklicher Prozesse gerichtetes Augenmerk
erkennen. Nachgespürt habe ich -- wenn auch nur andeutend -- den
vorauseilenden referenziellen Problemen (Verschlagwortung,
Literaturtypen), nicht zuletzt, um damit die Zusammenhänge zu erhellen,
in welchen diese Fragen überhaupt auftreten. Begonnen hat dieser
Rundgang mit der Zerbrechlichkeit von Gedanken und Begriffen zur
Rekursivität, denen ich schier idiomatischen Charakter zugesprochen und
-- in unserem Falle -- den jeweils verschiedenen Bezugsrahmen und
-linien des Rekursiven, wie einem Selbst, dass zu sich selbst gekommen,
seine Umschau beginnt, dann die Bezüge nicht mehr nur als
Selbstbezogenheit erkennt. Es war zu verstehen, dass eine feine, ebenso
in der Geschichte des Denkens verwurzelte Semantik im Wesen des
Bibliographischen eingewoben ist.\footnote{\ldots{} (und eine, jedoch
  hier nicht weiter zu begründenden) Semiotik, die Wort- Satz- und
  symbolische Systeme auf das Denken und Beurteilen zurückwirken und
  ihre Veränderung auch selbst immer wieder sich verändernde
  Begründungsstrukturen entstehen lassen.}

Es war ebenso zu verstehen, dass in einer Welt der Selbst- und
Fremdbezüglichkeit, in der Welt der Ebenen und Netze (\enquote{in a
world of levels and meshes}) natürlich darauf geachtet werden muss,
Strukturen, will man ihnen nicht einfach mit Gleichgültigkeit begegnen,
eben immer auch zu konservieren und mit hohem Sachverstand
problembezogen weiter zu entwickeln, auch nur, um damit eine Rückkehr in
ein \enquote{wüstes Land} zu vermeiden. In der Erfahrung der
Bibliotheken, ihren Mitarbeitern und ihrer Trägerschaften, welche diese
Strukturen bauen, vertreten und weiterentwickeln, hat es sich meist auch
gezeigt, dass beispielsweise eine öffentliche Verschlagwortung
(\enquote{public tagging}) rasch an Grenzen stößt und Anonymität das
Verfahren eher stört als fundiert und beflügelt.\footnote{würde man den
  Vorgang des Rezensierens und der Verschlagwortung auch
  bibliothekarisch an die Bestellung von Fernleihen (ILL) binden, wäre
  schon viel gewonnen: das Unternehmen Amazon hatte dies früh erkannt
  und die Bewertungsmöglichkeit des Käufers an das verkaufte Buchexempar
  bedungen. Bewertungen von leihenden Bibliotheksbenutzern, wie sie
  beispielsweise in MyBookshelf abgegeben werden, können in
  Onlinekatalogen erscheinen, mit entsprechender Kenntlichmachung oder
  als weiterführender Link, wo dann Leseerfahrungen/sachliche
  Bewertungen hinterlegt sind.}

Kehre ich zurück zu weiteren Anliegen dieses Textes, lässt sich ein im
Druckvermerk in der und Auflagenbezeichnung das längst gelöste Problem
der Versionierung einer Publikation erkennen, welches im Bereich des
Hypertextdokuments noch schmerzlich einer Lösung entgegenzusehen hofft.
Sind \emph{Permalink, Document Object Identifier (DOI), Unified Resource
Number (URN), E-Publishing, Wikis} alles Formate, die trotz ihrer zum
Teil starker Anlehnung an die analoge Welt, nicht ausreichen sollen, ein
zeitlich verankertes Nacheinander zustande zu bringen, die allen Seiten
genügen kann, das heißt den Bibliotheken selbst, ihren Benutzern, den
Bibliotheksverbänden und den Gremien des Internets (W3C)? Ich zweifle,
sehe aber auch, dass es daran liegen mag, für jeden neuen technischen
Ansatz, immer wieder neue Standards geltend zu machen, die untereinander
so wenig kompatibel, geschweige interoperabel funktionieren.

Mein Gedanke geht nun dahin, dass, wäre ein Standard für die
Onlineerstellung einer interbibliothekarischen Bibliographie festgelegt,
in einem Format jedoch, in der Inhalte und Standards jederzeit
ausbaufähig blieben, es für die Bibliotheken grundsätzlich möglich wäre,
im Bereich der Erarbeitung von Bibliographien gemeinsam tätig zu werden;
ich stelle mir vor, etablierte Verfahren wie \emph{Peer Reviewing} von
studentischen Arbeiten oder \emph{qualitative Inhaltsforschung} mit
einzelnen Bibliotheksbesuchern unter freiwilliger Teilnahme in Bezug
einer Offenlegung von Suchvorgängen möglich zu machen, die für eine
Studienarbeit notwendig werden und so die Zugänge über \emph{Wort, Bild}
und \emph{Musik} zu beleuchten, welche zur Erreichung eines Ergebnisses
durchschritten wurden; schließlich wären auch verworfene Suchergebnisse
und die Gründe des Verwerfens zu ermitteln.\footnote{Um ein richtiges
  Verständnis für das hier Gesagte zu erlangen, muss angemerkt werden,
  dass es eben nicht darum geht, das Denken des Menschen den Funktionen
  von Maschinen zu unterwerfen. Intelligenz wird immer trans- und
  metakategoriales Denken sein und viel mehr enthalten als programmierte
  Verzweigungen oder Entscheidungsbäume: Denken bedeutet und ist auch
  heute noch Überschreiten (Ernst Bloch), während Maschinen
  Programmfunktionen folgen, bedarf das Denken grundlegend eines
  individuell gefassten Motives, eines Anlasses, des Abwägens und ebenso
  des Zweifelns, dies und mehr leisten Maschinen nicht, hingegen sind
  sie in Berechnungen und im Prozess des Suchens und des
  \enquote{Erkennens des Gleichen} unangefochten schneller. Auch in
  dieser Hinsicht sind Bibliotheken angehalten, weiterhin Institutionen
  forschenden Denkens zu bleiben, ebenso, um unlauterem
  wissenschaftlichen Verhalten die Stirn zu bieten: Bibliotheken, einmal
  mit Instrumentarien ausgestattet, die es erlauben, große Mengen an
  Texten zu analysieren, wären prädestinierte Orte für Plagiatsprüfungen
  der Arbeiten von Absolventen, sie verfügten über die Möglichkeiten in
  house aufgrund ihres Bestandes und der Onlinedatenbanken allfälligen
  Verdachtsmomenten professionell zu begegnen.} Die Leistung des
bibliothekarischen \emph{Peer-Reviewing} kann von der orthographischen
Korrektur vor Abgabe bis hin -- nach Fertigstellung durch die
Studierenden - zur Erkundung der Gründe und Textorte zitierter Literatur
umfassen (mit oder ohne Anwesenheit des Studierenden), um sprachliche
Bedeutungsstrukturen in Wortketten semantisch exakt zu bestimmen; auch
damit ließe sich eine verfeinerte Verschlagwortung erreichen (ebenso für
die Erstellung von Fachbibliographien), zudem wäre es möglich, diese
Wissensbestände an in- und ausländische Institutionen der
Sprachforschung gegen Entgelt oder Tauschleistungen zu überlassen (etwa
dort zur Überarbeitung von Thesauri und Wörterbüchern).

Die qualitative Inhaltsforschung geht in die gleiche Richtung, würde
allerdings nur jenen Teil betreffen, der darauf abzielt, gemeinsam mit
Studierenden die semantische Struktur der gewählter Aussagen innerhalb
ihrer eigenen Arbeiten zu erhellen; indes, wenn Incentives anfallen,
sollten diese an den Bezug von bibliothekarischen Dienstleistungen
gebunden werden (etwa kostenlose Voucher für Fernleihen). Es könnte auch
ein automatisiertes Verfahren geschaffen werden, welches den freiwillig
teilnehmenden Personen, jeweils bei Besuch der Bibliothek, einen smarten
Controller mit an die Hand zu geben, der es erlaubt, jeden Griff zu
einem Buch im frei erreichbaren Bestand oder die Einempfangsnahme an der
Theke mittels Radio-Frequency Identification (RFID) einzulesen und zu
dokumentieren, daneben setzt die qualitative Inhaltsforschung voraus,
die Suchvorgänge im Online-Katalog und den Datenbanken im Sinne eines
Track- and Trace durchgehend zu loggen. Eine Ausgestaltung des
technischen Settings muss natürlich immer auch mit dem vollen Respekt
der Privatsphäre einhergehen, selbst da, wo es sich um ein Verfahren
handelt, das lediglich im Rahmen einer zweckorientierter Suche, die für
eine mit dem Studium verbundenen Arbeit erfolgt und von Beginn an für
Außenstehende nicht einsehbar und deshalb - für die Beteiligten
gesprochen - anonym durchgeführt wird.\footnote{Bis dahin stehen
  Lesevorgänge und ihre Unterbrechungen auf iPads dokumentbezogen nur
  den Geräteproduzenten zur Verfügung; dies müsste für eine solche
  automatisierte Forschungsanwendung eine Veränderung erfahren. Es geht
  um BigData jedoch in vertretbarem Rahmen: während der Zeit, in der die
  qualitative Studie stattfindet, könnten -- wie in sozialen Netzen
  üblich, hier jedoch anonymisiert -- Vorschläge zu Suchbegriffen
  mittels Pop-Ups auf das entsprechende von einer Bibliothek zur
  Verfügung gestellte Gerät erfolgen. In extremis lässt sich ein
  VR-/SmartGlass mit gleichartiger Pop-Up-Funktion vorstellen wie auf
  den iPads, das selbstverständlich eine umfassende Schrift- und
  Spracherkennung voraussetzt und alle Such- , Lese- und Schreibvorgänge
  im Blickfeld freiwilliger Probanden mitschneidet. Die damit verbundene
  Vorschlagsfunktion mittels Pop-Ups wäre dann einer vollständig viralen
  und virtualisierten und zugleich formlosen Bibliographie
  gleichzusetzen, ebenso bedeutete sie eine überaus gefährliche
  Unterwerfung menschlichen Denkens, da Worte und Begriffe immer in das
  System auch des kritischen Überlegens eingreifen und einen eigenen
  Bedeutungsraum entfalten, der zwar verschiedentlich mit einer
  kreativen Erweiterung des eigenen Hirns gleichgesetzt wird, auf Dauer
  jedoch mehr Last und Bürde statt der verheißenen Befreiung darstellt.}

Eine Herangehensweise, die sich auf technisch-instrumenteller Ebene für
die Erarbeitung von neuen (Fach-)Bibliographien für die Bibliotheken
selbst anbietet, beinhaltet demnach: die gewählten/aufzunehmenden Titel
in einem Format zu hinterlegen, welches einerseits erlaubt, als jeweils
zeitlich abgeschlossene Version zu gelten, ähnlich wie dies heute in der
Wikisoftware bereits geschieht, was jedoch strukturell stark vereinfacht
werden könnte (gewissermaßen die technische Seite der International
Standard Book Description: ISBD); hier wie dort wäre die Versionierung
lediglich für die Kontrolle, die es bei Hard- oder Softwareproblemen
erlaubt, auf eine Sicherungskopie und zugleich ihren Entstehungsverlauf
mittragend, zurückgreifen zu können. Der interbibliothekarische
Austausch der Dokumente bedingt eine einheitliche, \emph{interoperable}
(und nicht eine wie heute nur inter\emph{kompatible}) technische
Struktur; diese Formate stellen -- so die Vorstellung - gewissermaßen
\enquote{Zeitdokumente} mit Fußnoten dar, die etwaige Angaben zur
Verschlagwortung / Systematik und zu den Gründen enthalten könnte,
weshalb Titel überhaupt für eine jeweils vorgesehene Bibliographie
erfasst wurden. Schließlich lässt sich annehmen, den bibliographischen
Teil bei Neuzugängen in Zukunft unmittelbar zu scannen wie dies heute
mit Buchtiteln/Inhaltsverzeichnissen bereits in Deutschland und der
Schweiz geschieht) und diese Datenbank an ein Benutzerlogin und ein
\emph{Pay-per-Click-Verfahren} oder an die \emph{Dauer ihrer
Konsultation} zu binden, da sie nicht zu der durch das Grundgesetz
beziehungsweise die Verfassung garantierten frei zugänglichen
Information gehören muss, indem sie zum Bestand vor allem der sehr
vertieften wissenschaftlichen Forschung gezählt werden kann.

Ein geringer an die scannenden Bibliotheken zurückfließender Obolus ist
nicht nur denkbar sondern zu fordern, etwa in einer monatlichen
Abrechnung geringer Höhe an die Benutzer. Eine Datenbank in dieser Form
muss zwingend einer Volltextsuche ausgerüstet sein, unter der
Voraussetzung, dass der einzelne Scan an das ursprüngliche Werk
zurückgebunden bleibt.\footnote{Von den Suchmöglichkeiten her gedacht,
  kann zurückverfolgt werden, mit welchen Grenzen und Schwierigkeiten
  die Gestaltung der neuen Onlinekataloge verbunden war bzw. es heute
  noch sein kann, die aber immer eine Entlastung der Serverstruktur
  bieten, anders sind die Unterschiede zwischen einer Mehrfeldsuche, wie
  sie sehr umfassend der Karlsruher Virtuelle Katalog (KVK) bietet im
  Unterschied zu den Katalogen mit einer Einfeldsuche (wie etwa kobv
  {[}Kooperativer Bibliotheksverbund Berlin-Brandenburg{]} oder Swissbib
  {[}Gesamtonlinekatalogwissenschaftlicher Bibliotheken der Schweiz{]}).
  Mit der Volltextsuche werden heute in kleinen Onlinekatalogen die
  einst zahlreichen Suchfelder abgelöst: Titel (Title, Work, Journal)
  oder {[}dt./engl.{]}: beginnt mit/starts with, enthält/contains),
  Autor, Erscheinungsjahr/pubishing date, Verlag/edition, ISBN, Freitext
  (alle Felder/all Fields), Phrasensuche (exact, phrase, exact phrase),
  Stichwortsuche (nur Schlagwörter und Zusammenfassungen /Abstracts
  umfassend, werden im Englischen mit all keywords, any in Datenbanken
  angelegt. Ein für das Verständnis bezüglich Struktur und Leistung von
  Suchmaschinen, ihren kommerziellen Verbandelungen, strategischer
  Entwicklung, bietet eine noch heute gute Übersicht: ALEXANDER
  HALAVAIS: Search Engine Society (2009) Cambridge {[}et al.{]}: Polity
  Press, passim.} Die rechtlichen Fragen des Copyrights werden im
Bereich eingescannter unselbständiger Bibliographien aus Autorenwerken
-- davon ist auszugehen -- weit weniger berührt sein als dies bei den
Scans von Titeln, Inhaltsverzeichnissen und Verlagstexten bereits der
Fall ist und wie sie heute in den Onlinekatalogen für die Öffentlichkeit
schon frei zugänglich sind. Diese Scans haben rechtliche Hürden bereits
hinter sich gelassen. Juristische Fragen werden sich viel
voraussichtlich weniger bei gebührenpflichtigen Katalogen stellen, die
darüber hinaus den Benutzern allein in der Bibliothek zur Verfügung
stehen. Allerdings -- und das wäre eventuell auch ein Nachteil -- würden
die effektiven Ausleihen möglicherweise stark zurückgehen. Doch könnte
dieser Umstand für die Bibliotheken bedeuten, einen weiteren gewichtigen
Schritt in Richtung Informationszentrum zu tun.

\section*{VI. Ein Zwischenhalt}\label{vi.-ein-zwischenhalt}

Schaue ich nochmals zurück auf die Ausführungen, ist auffallend, dass
ich hier nicht nur auf (eingangs auch den sehr hermeneutischen) Prozess
des Erstellens von Bibliographien beleuchte und ihre über die Zeit
hinweg verschiedenen Ausformungen, welchen sie teilweise heute noch
unterliegt, ebenso war mir die Aufhellung der Struktur und die
Möglichkeiten der Verfeinerung der Verschlagwortung in den
Bibliographien selbst wichtig -- als Suchort teils spezialisierter
Recherchen -- und ich bin der Frage des Suchprozesses selbst
nachgegangen in dessen Zusammenhang künftige Formen der Bibliographie
gesehen werden könnten, um in ihrer Struktur und Erscheinung weiterhin
an Aktualität zu besitzen. Zuletzt ist die Finanzierung zu einem Thema
geworden, dem man sich letztlich nicht entziehen kann. Die Möglichkeiten
der Finanzierung öffentlicher Bibliotheken sind beschränkt, dies spüren
nach wie vor private Trägerschaften, welche für ihre Angebote die
Mitgliederkarten ihrer Benutzer mit einer jährlich erhobenen Gebühr
belegen müssen. Die Finanzierung von öffentlichen wissenschaftlichen
Bibliotheken sollte -- wenn auch nur gering -- dem Benutzer etwas
bewusster gemacht werden, insbesondere wenn über die weiteren
Entwicklungen von Bibliographien nachgedacht wird. Es ist auch deshalb,
dass ich vielmehr von einem \emph{Zwischenhalt} sprechen möchte denn von
einem Abschluss.

%autor
\begin{center}\rule{0.5\linewidth}{\linethickness}\end{center}

\textbf{Krisztof Ján Kojakeva} (eigentlich Christoph Kujawa), geb. 1962,
begann als Versicherungskaufmann und war viele Jahre in Verlagen,
Buchhandlungen und der papierverarbeitenden Industrie im In- und Ausland
tätig, später zählten ebenso Mandate in der qualitativen
Meinungsforschung mit Jugendlichen und Erwachsenen zu seinen Aufgaben.
Heute lebt er in Zürich und schreibt u.a. auch für die Wikipedia zu
Musik und zur Soziologie der Industrialisierung. Mit Bibliographien und
Verzeichnissen pflegt er immer wieder längere Begegnungen, wenn
extensive Recherchen für die Erschließung interdisziplinärer
Zusammenhänge anstehen, weshalb dieser Beitrag konsequent aus der Sicht
des Lesers geschrieben wurde.

\end{document}