\documentclass[a4paper,
fontsize=11pt,
%headings=small,
oneside,
numbers=noperiodatend,
parskip=half-,
bibliography=totoc,
final
]{scrartcl}

\usepackage{synttree}
\usepackage{graphicx}
\setkeys{Gin}{width=.4\textwidth} %default pics size

\graphicspath{{./plots/}}
\usepackage[ngerman]{babel}
\usepackage[T1]{fontenc}
%\usepackage{amsmath}
\usepackage[utf8x]{inputenc}
\usepackage [hyphens]{url}
\usepackage{booktabs} 
\usepackage[left=2.4cm,right=2.4cm,top=2.3cm,bottom=2cm,headheight=25.60228pt,includeheadfoot]{geometry}
\usepackage{eurosym}
\usepackage{multirow}
\usepackage[ngerman]{varioref}
\setcapindent{1em}
\renewcommand{\labelitemi}{--}
\usepackage{paralist}
\usepackage{pdfpages}
\usepackage{lscape}
\usepackage{float}
\usepackage{acronym}
\usepackage{eurosym}
\usepackage[babel]{csquotes}
\usepackage{longtable,lscape}
\usepackage{mathpazo}
\usepackage[flushmargin,ragged]{footmisc} % left align footnote

%%url brekas grrr
\def\UrlBreaks{\do\a\do\b\do\c\do\d\do\e\do\f\do\g\do\h\do\i\do\j\do\k\do\l%
\do\m\do\n\do\o\do\p\do\q\do\r\do\s\do\t\do\u\do\v\do\w\do\x\do\y\do\z\do\0%
\do\1\do\2\do\3\do\4\do\5\do\6\do\7\do\8\do\9\do\-}%

\usepackage{listings}

\urlstyle{same}  % don't use monospace font for urls

\usepackage[fleqn]{amsmath}

%adjust fontsize for part

%% geometry
\clubpenalty = 10000 
\widowpenalty = 10000 
\displaywidowpenalty = 10000
%% tightlist

\providecommand{\tightlist}{%
  \setlength{\itemsep}{0pt}\setlength{\parskip}{0pt}}

\usepackage{sectsty}
\partfont{\large}

%Das BibTeX-Zeichen mit \BibTeX setzen:
\def\symbol#1{\char #1\relax}
\def\bsl{{\tt\symbol{'134}}}
\def\BibTeX{{\rm B\kern-.05em{\sc i\kern-.025em b}\kern-.08em
    T\kern-.1667em\lower.7ex\hbox{E}\kern-.125emX}}

\usepackage{fancyhdr}
\fancyhf{}
\pagestyle{fancyplain}
\fancyhead[R]{\thepage}

%meta

%meta

\fancyhead[L]{K. Schuldt \\ %author
LIBREAS. Library Ideas, 29 (2016). % journal, issue, volume.
\href{http://nbn-resolving.de/urn:nbn:de:kobv:11-100238216
}{urn:nbn:de:kobv:11-100238216}} % urn
\fancyhead[R]{\thepage} %page number
\fancyfoot[L] {\textit{Creative Commons BY 3.0}} %licence
\fancyfoot[R] {\textit{ISSN: 1860-7950}}

\title{\LARGE{Textarbeit und nervöses Pochen darauf, dass sich alles ändern wird.
Muss. Rezension zu: Stefan Alker ; Achim Hölter (Hrsg.). Literaturwissenschaft
und Bibliotheken. (Bibliothek im Kontext; 2) Wien: Vienna University Press, 2015. 30.00 €. (Auch als Open Access \url{http://doi.org/10.14220/9783737004541})}} %title %title
\author{Karsten Schuldt} %author

\setcounter{page}{}

\usepackage[colorlinks, linkcolor=black,citecolor=black, urlcolor=blue,
breaklinks= true]{hyperref}

\date{}
\begin{document}

\maketitle
\thispagestyle{fancyplain} 

%abstracts

%body
Es ist nicht falsch, so wird im hier besprochenen Band
\emph{Literaturwissenschaft und Bibliotheken} klar, die
Literaturwissenschaft als eine besonders bibliotheksaffine Wissenschaft
zu betrachten. Sie basiert auf der Analyse von Texten, beschreibt die
Bibliothek als Arbeitsort und kann auf einige literarische Bilder der
Bibliothek verweisen, welche in ihrem fachlichen Diskurs immer wieder
aufgerufen werden. Gleichzeitig ist es, wie jede lebendige Wissenschaft,
keine, die eine einheitliche Meinung hätte. Vor allem aber, und das ist
für das Bibliothekswesen wohl relevant, sind
Literaturwissenschaftlerinnen und -wissenschaftler zumeist keine
Praktikerinnen und Praktiker der Bibliothek, sondern versierte und mit
bestimmten Vorstellungen von Bibliotheken ausgestattete Nutzende.

\section*{Zugang über die
Textarbeit}\label{zugang-uxfcber-die-textarbeit}

\emph{Literaturwissenschaft und Bibliotheken} unternimmt es, eine Anzahl
von Autoren -- keine Autorinnen -- aus der Literaturwissenschaft, einige
davon mit Bibliothekserfahrungen oder bibliothekarischen Ausbildungen,
zum Themenkomplex Bibliothek zu versammeln. Es gibt dabei einen starken
Bezug zur Universität Wien, bei deren Presse das Buch erschien, sowie
zur Stadt Wien. Fast alle Beitragenden sind dort angesiedelt oder haben
Zeiten dort verbracht. Ohne einen weiten Blick auf die restliche
Literaturwissenschaft ist nicht zu sagen, ob dies eine Fokussierung auf
einen bestimmten Diskurs bedeutet oder ob der Band tatsächlich
Positionen der gesamten (zumindest deutschsprachigen)
Literaturwissenschaft abdeckt.

Der Band ist nicht für das Bibliothekswesen erstellt worden, sondern
offenbar für die Literaturwissenschaft. Insoweit erstaunt es, dass eine
Anzahl von Texten sich dennoch daran wagt, die Zukunft der Bibliotheken
als Institution zu bearbeiten. Im Vorwort betonen die beiden
Herausgeber, Stefan Alker und Achim Hölter, dass die
Bibliothekswissenschaft die Bibliothek als Institution untersuchen
würde, während die Literaturwissenschaft eher die Metapher Bibliothek im
Blick hätte; ebenso postuliert dies der erste Beitrag
\enquote{Literaturtheorie als Bibliothekstheorie} von Dirk Werle. Diese
Grenze überschreiten einige Beiträge aber und das in nicht immer
nachvollziehbarer Weise.

Weiterhin ist über die gesamten Beiträge hin auffällig, dass die
Literaturwissenschaft als Wissenschaft anhand von Texten arbeitet, weit
mehr, als dies in der Bibliothekswissenschaft der Fall ist. Es ist den
Beitragenden offenbar einsichtig, den Zugang zu fast allen gestellten
Fragen über die vorhandenen bibliothekarischen Texte -- und dabei
überwiegen, schaut man die verwendete Literatur durch, die Monographien
und Sammelbände, nicht etwa die Zeitschriftenartikel -- anzugehen. Dies
führt dazu, dass in einigen Beiträgen ganz selbstverständlich Texte als
aktueller bibliothekarischer Diskurs verstanden werden, die sich in der
aktuellen bibliothekarischen Literatur nur noch äusserst selten finden,
wie die von Walther Umstätter, Paul Raabe, Uwe Jochum oder Friedrich
Nestler; an einer Stelle wird \enquote{Bibliotheken '93} als aktuelle
Literatur bezeichnet. Offensichtlich schaut die Literaturwissenschaft
anders auf die bibliothekarische Literatur, als es das Bibliothekswesen
mit seiner Bevorzugung von Artikeln und dem eher kürzeren
\enquote{Gedächtnis} bibliothekarischer Debatten selber tut. Es wäre zu
einfach, diesen Zugang aus Bibliothekssicht als falsch zu bezeichnen --
er ist sehr umfassend, erstaunlich belesen, aber vor allem anders.

\section*{Die Bibliothek ist positiv, aber
verwirrend}\label{die-bibliothek-ist-positiv-aber-verwirrend}

In einer ganzen Reihe von Texten wird festgestellt, dass es einen
kleinen Kanon an Zitaten (Jorge Luis Borges, Goethe) und theoretischen
Zugängen (Michel Foucault, Walter Benjamin, Niklas Luhmann) gibt, auf
denen beim Schreiben über Bibliotheken eher unsystematisch
zurückgegriffen wird, ohne dass klar wird, was genau der Wert dieser
Rückgriffe ist. Es scheint, dass dieser Kanon vor allem die
\enquote{richtigen Bilder} hervorruft, auf denen aufgebaut werden kann.

Stefan Alker untersucht zum Beispiel die Darstellung der Bibliothek in
den Einführungen in das literaturwissenschaftliche Studium, wobei für
ihn die Sozialisation in die Literaturwissenschaft immer auch die
Sozialisation in eine spezifische Bibliothekserfahrung ist. Äussern sich
diese Einleitungen zu Bibliotheken, dann würden sie es zumeist mit
weitfassenden Narrativen über Bibliotheken im Rückgriff auf den
genannten Kanon tun, um dann die konkrete Bibliothek selber als
grundsätzlich wichtig und positiv, aber auch verwirrend und nicht ganz
vertrauenswürdig zu beschreiben. Bei den Beschreibungen, die Alker
referiert, aber auch in anderen Beiträgen, dringt die Ansicht durch,
dass die Bibliothekarinnen und Bibliothekare zwar wichtige Arbeit tun,
aber doch nicht unbedingt die, die eine Literaturwissenschaftlerin oder
ein Literaturwissenschaftler bräuchte. Gerade der Katalog und die
Klassifikationen (insbesondere im Beitrag von Peter Blume zu
bibliothekarischen und philologischen Systematiken) gelten als
unvollständig und \enquote{anders} (zum Beispiel als zu eng und zu
streng für die literaturwissenschaftliche Arbeit). Wichtig sei es
deshalb für Literaturwissenschaftlerinnen und -wissenschaftler, sich in
den Bibliotheken schnell zurechtzufinden und eigene Recherchefähigkeiten
zu erlernen.

Gerade Michael Pilz stellt sich in seinem sehr belesenen Beitrag zur
\enquote{Literaturvermittlung} in Wissenschaftlichen Bibliothek gegen
den bibliothekarischen Diskurs. Er weiss von den Bemühungen und Debatten
um die \enquote{Informationskompetenz}, empfindet sie aber für die
Literaturwissenschaft als falsch. Die Bibliotheken hätten sich von einem
\enquote{weiten Literaturbegriff}, den sie einmal hatten, getrennt und
sich eines Informationsbegriffes bemächtigt, der die literarische
Perspektive verlässt. Dadurch wäre auch das, was die
Literaturwissenschaft als Literaturvermittlung begreift, unmöglich. Es
würde in Bibliotheken nicht mehr von ästhetischen Fragen gesprochen,
auch nicht mehr von Bestandsvermittlung, sondern von
Informationsvermittlung; statt der Vermittlung des Bestandes sei eine
Reduktion auf das Katalogisat als Vermittlung vorgenommen worden. Die
Vermittlung des Inhalt selber würde jetzt als Aufgabe der Leserinnen und
Leser verstanden. Bibliothek und Literaturwissenschaft würden aneinander
vorbei handeln und sich kaum wahrnehmen. (Eine Einschätzung, die im Buch
mehrfach geäussert wird.) Genau genommen würde die Position von Pilz im
Bibliothekswesen wohl schnell als unmodern bezeichnet werden, da er auf
Aufgaben verweist, die Bibliotheken sich heute nicht mehr zuschreiben
würden (zumindest die, die den Diskurs bestimmen), aber seine Herkunft
aus der Literaturwissenschaft macht klar, dass ein solcher Vorwurf wohl
am Kern seines Argumentes vorbei zielt. Sein Insistieren darauf, dass
Literaturvermittlung etwas anderes ist als
Informationskompetenzvermittlung eröffnet einen Blick auf die Frage, was
eigentlich durch diese Wende im Bibliothekswesen nicht mehr gemacht wird
oder nicht mehr gemacht werden kann.

\section*{Trotzdem: Die Bibliothek muss modern werden.
Muss.}\label{trotzdem-die-bibliothek-muss-modern-werden.-muss.}

Eine ganze Reihe von Texten, unternimmt es dann trotzdem -- wie weiter
oben angedeutet -- sich zur Zukunft der Bibliotheken zu äussern. Das
sind die schwachen Teile des Bandes. Während andere Aussagen offenbar
auf einer intensiven Textarbeit beruhen und denn bibliothekarischen
Allgemeinplätzen andere Perspektiven gegenüberstellen, verfallen solche
Aussagen immer wieder selber in Allgemeinplätze.

Ein Beispiel dafür ist der Beitrag von Andreas Brandtner, der sich mit
dem immer wieder aufgerufenen Bild von den (Wissenschaftlichen)
Bibliotheken als Labore der Geisteswissenschaft (spezifisch der
Literaturwissenschaft) beschäftigt. Der grösste Teil des Textes
beschreibt diese Analogie als eine in Krisensituationen aufgerufene
Behauptung, der keine bibliothekarische Praxis gegenüberstehe, welche
diesen Laborcharakter tatsächlich herstellen würde. Dieser Teil des
Textes greift weit in die Bibliotheksgeschichte zurück und dabei wieder
auch auf Quellen, die im zeitgenössischen bibliothekarischen Diskurs
nicht bekannt scheinen. Kommt der Autor inhaltlich zur Jetztzeit,
behauptet er einfach, dass die Offenheit des Internets dazu führen
würde, dass sich Bibliotheken ändern müssten. Es gäbe neue Akteure am
Informationsmarkt, die Bibliotheken müssten darauf reagieren. Ob sie das
schaffen würden, wäre noch ungeklärt. Dieser Teil des Beitrags scheint
nicht auf intensiver Textarbeit zu basieren, sondern auf Behauptungen.
Dabei könnte sich der Autor auf mehr als diese Vermutungen stützen, um
seine Aussagen zu überprüfen. Gleichzeitig wäre es auch möglich, diese
Position in bibliothekarischen Texten zu finden, er müsste sie nicht
selber erstellen. Brandtner folgt in der Abschlussargumentation in toto
dem, was Raphael Ball beispielsweise ausführlicher in seinem \emph{Was
von Bibliotheken wirklich bleibt (Wiesbaden, 2013)} dargestellt hat,
inklusive der Terminologie von den vorgeblichen Informationsmonopolen,
die sie verloren hätten -- allerdings ohne Ball anzuführen.

Dieses Vorgehen findet sich in mehreren Texten: Wenn es um die
Bibliotheken der Zukunft geht, wird mit Behauptungen gearbeitet. Es
fehlen klare Darstellungen, was wieso das passieren soll. Auffällig ist
auch, dass im Gegensatz dazu heutige Bibliotheken gleichzeitig als
sinnvolle Arbeitsorte für Literaturwissenschaftlerinnen und
-wissenschaftler beschrieben werden. Zudem wird im gleichen Buch die
Verantwortung für die Entwicklung den Bibliotheken selber und der
Bibliothekswissenschaft zugeschrieben, und gerade nicht der
Literaturwissenschaft, die als Feld \enquote{daneben} beschrieben wird.
Es scheint aber bei vielen der Autoren den Drang zu geben, diese
Bibliotheken als überholt zu bezeichnen, ohne es genauer darzustellen.
Gleichzeitig scheint es auch keine Vorstellung davon zu geben, wie diese
zukünftigen Bibliotheken aussehen sollten. Dies alles verbleibt am
Ungefähren.

Erstaunlich ist, dass bei all diesen Behauptungen die Digitial
Humanities im Band kaum thematisiert werden. Wenn, dann geht es vor
allem darum, dass die Bibliotheken Digitalisate anbieten sollten. Der
Drang bezieht sich offenbar auf die Bibliotheken, die sich verändern
sollen, aber nicht auf die literaturwissenschaftliche Praxis in den
Bibliotheken selber.

\section*{Bibliothek als Text}\label{bibliothek-als-text}

Daneben finden sich im Band selbstverständlich auch Texte zur Bibliothek
als literarischem Sujet. Eine wenig erstaunliche Feststellung, die sich
in diesen immer wieder findet, ist die, dass sich die Bilder von
Bibliotheken stark von der tatsächlichen Bibliothek unterscheiden.
Insbesondere werden mit Bibliotheken in literarischen Texten
Vorstellungen von der Verfügbarkeit unendlicher Informationen und
unendlich vieler Bücher verbunden, die eine reale Bibliothek so gar
nicht erfüllen könnte. Bibliotheken scheinen oft ein Symbol für die
Verbindung zum Denken, eine Repräsentation des Denkens, zum Teil auch --
wie Daniel Syrovy in seinem Beitrag zum \enquote{Berufsfeld Bibliothek}
festhält -- für einen denkfördernden Arbeitsort. Syrovy zeigt, dass die
Vorstellung, als Bibliothekar und Bibliothekarin würde man vor allem
viel lesen und ein solcher Brotberuf sei deshalb für eine
schriftstellerische oder literaturwissenschaftliche Karriere perfekt,
verbreitet ist und immer wieder reproduziert wird, obwohl
selbstverständlich die bibliothekarische Professionalität anders
aussieht.

\section*{Fazit}\label{fazit}

Die Beiträge in \emph{Literaturwissenschaft und Bibliotheken}, bei denen
sich die Autoren auf literaturwissenschaftliche Themen fokussieren, sind
die lesbarsten des Bandes. Beiträge, in denen die Autoren die
literaturwissenschaftliche Arbeitspraxis auf den bibliothekarischen
Diskurs anwenden, sind die inhaltlich stärksten und, durch den
Perspektivwechsel, für das Bibliothekswesen auch die interessantesten.
Die Beiträge -- oder Teile von Beiträgen --, in denen die Autoren die
literaturwissenschaftliche Arbeitspraxis suspendieren und sich daran
machen, zu diskutieren, dass die Bibliotheken anders werden müssten,
sind die schwächsten, sowohl was die Argumentation als auch die
Überzeugungskraft betrifft. Was die Literaturwissenschaft für einen
Gewinn aus dem Band ziehen kann, muss sie selber klären. Für das
Bibliothekswesen scheint vor allem die unterschiedliche Bewertung
bibliothekarischer Arbeit -- zum Beispiel der Arbeit im Bezug auf
\enquote{Informationskompetenz} oder der Katalogisierung -- anregend.
Nicht, dass die Ansichten der Autoren des Bandes übernommen werden
müssten, aber sie stellen eine sehr bibliotheksaffine Gruppe dar,
insoweit sollten ihre Anmerkungen als Hinweis darauf genommen werden,
wie die Nutzerinnen und Nutzer die Bibliothek tatsächlich sehen.

Zu begrüssen wäre es, wenn sich bibliothekarische Beiträge an der
Quellenarbeit, die bei allen Beiträgen geleistet wurde, orientieren
würden. Stellenweise bieten sie Überblicke zur Bibliotheksgeschichte und
-entwicklung, die so eigentlich von der Bibliothekswissenschaft
geleistet werden müssten. Das Vorwort und der erste Beitrag des Bandes
postulieren allerdings darüber hinaus, dass die Bibliotheksforschung von
literaturwissenschaftlichen Methoden profitieren würde. Dies ist im
Band, über die Quellenarbeit hinaus, nicht überzeugend dargelegt. Es
fehlt an Hinweisen auf bibliothekarische Fragestellungen, die wirklich
von diesen Methoden profitieren würden.

%autor
\begin{center}\rule{0.5\linewidth}{\linethickness}\end{center}

\textbf{Karsten Schuldt} (Chur / Berlin) ist Wissenschaftlicher
Mitarbeiter am Schweizerischen Institut für Informationswissenschaft,
HTW Chur und Redakteur der LIBREAS. Library Ideas.

\end{document}