\documentclass[a4paper,
fontsize=11pt,
%headings=small,
oneside,
numbers=noperiodatend,
parskip=half-,
bibliography=totoc,
final
]{scrartcl}

\usepackage{synttree}
\usepackage{graphicx}
\setkeys{Gin}{width=.4\textwidth} %default pics size

\graphicspath{{./plots/}}
\usepackage[ngerman]{babel}
\usepackage[T1]{fontenc}
%\usepackage{amsmath}
\usepackage[utf8x]{inputenc}
\usepackage [hyphens]{url}
\usepackage{booktabs} 
\usepackage[left=2.4cm,right=2.4cm,top=2.3cm,bottom=2cm,headheight=25.60228pt,includeheadfoot]{geometry}
\usepackage{eurosym}
\usepackage{multirow}
\usepackage[ngerman]{varioref}
\setcapindent{1em}
\renewcommand{\labelitemi}{--}
\usepackage{paralist}
\usepackage{pdfpages}
\usepackage{lscape}
\usepackage{float}
\usepackage{acronym}
\usepackage{eurosym}
\usepackage[babel]{csquotes}
\usepackage{longtable,lscape}
\usepackage{mathpazo}
\usepackage[flushmargin,ragged]{footmisc} % left align footnote

%%url brekas grrr
\def\UrlBreaks{\do\a\do\b\do\c\do\d\do\e\do\f\do\g\do\h\do\i\do\j\do\k\do\l%
\do\m\do\n\do\o\do\p\do\q\do\r\do\s\do\t\do\u\do\v\do\w\do\x\do\y\do\z\do\0%
\do\1\do\2\do\3\do\4\do\5\do\6\do\7\do\8\do\9\do\-}%

\usepackage{listings}

\urlstyle{same}  % don't use monospace font for urls

\usepackage[fleqn]{amsmath}

%adjust fontsize for part

%% geometry
\clubpenalty = 10000 
\widowpenalty = 10000 
\displaywidowpenalty = 10000
%% tightlist

\providecommand{\tightlist}{%
  \setlength{\itemsep}{0pt}\setlength{\parskip}{0pt}}

\usepackage{sectsty}
\partfont{\large}

%Das BibTeX-Zeichen mit \BibTeX setzen:
\def\symbol#1{\char #1\relax}
\def\bsl{{\tt\symbol{'134}}}
\def\BibTeX{{\rm B\kern-.05em{\sc i\kern-.025em b}\kern-.08em
    T\kern-.1667em\lower.7ex\hbox{E}\kern-.125emX}}

\usepackage{fancyhdr}
\fancyhf{}
\pagestyle{fancyplain}
\fancyhead[R]{\thepage}

%meta

%meta

\fancyhead[L]{M. Herzer \\ %author
LIBREAS. Library Ideas, 29 (2016). % journal, issue, volume.
\href{http://nbn-resolving.de/urn:nbn:de:kobv:11-100238171
}{urn:nbn:de:kobv:11-100238171}} % urn
\fancyhead[R]{\thepage} %page number
\fancyfoot[L] {\textit{Creative Commons BY 3.0}} %licence
\fancyfoot[R] {\textit{ISSN: 1860-7950}}

\title{\LARGE{Kurze Geschichte der Numa-Forschung
}} %title %title
\author{Manfred Herzer} %author

\setcounter{page}{}

\usepackage[colorlinks, linkcolor=black,citecolor=black, urlcolor=blue,
breaklinks= true]{hyperref}

\date{}
\begin{document}

\maketitle
\thispagestyle{fancyplain} 

%abstracts

%body
Numa Praetorius war das Pseudonym des Straßburger Juristen und
Schriftstellers Eugen Wilhelm (1866-1951). Er benutzte es in den Jahren
1899 bis 1932 für seine Beiträge zu deutschen, französischen und
italienischen Zeitschriften, Jahrbüchern und Serien, in denen er sich
fast stets zu homosexuellen Themen äußerte.

Am Anfang der 1970er Jahre entstand in der BRD und in Westberlin eine
Emanzipationsbewegung der Schwulen, bald auch der Lesben, die aus der
studentischen Protestbewegung gegen den Vietnamkrieg hervorgegangen war.
Schon früh entdeckte sie ihr Interesse an der Geschichte der
Schwulenverfolgung und -emanzipation, fand in Bibliotheken und Archiven
die Bände des \emph{Jahrbuchs für sexuelle Zwischenstufen} als beste
Quelle zum Einstieg in die Thematik, und darin die
\enquote{Bibliographie der Homosexualität}, die ein \enquote{Dr.~jur.
Numa Praetorius} beinahe in jedem Jahrgang geschrieben hatte.

Diese Bibliographie war eigentlich keine. Es handelt sich vielmehr um
eine Sammlung manchmal sehr umfangreicher kritischer Referate zu Büchern
und Zeitschriftenaufsätzen zur Homosexualität. Im Fall des Buches
\emph{Homosexualität und Strafgesetz}, das ein Rostocker Juraprofessor
Wachenfeld als Warnung vor der Schwulenemanzipation verfasst hatte,
füllte die Besprechung mehr als hundert Seiten im \emph{Jahrbuch} von
1902. Das war Numas Rekord, dabei war Wachenfelds Buch nicht viel dicker
(150 Seiten).

Damals war Wilhelm noch Amtsrichter in Straßburg, doch schon seit etwa
1898 in Magnus Hirschfelds Wissenschaftlich-humanitärem Komitee in
Charlottenburg bei Berlin aktiv. Bei seinen Kommentaren zur aktuellen
wissenschaftlichen Literatur zu schwuler oder lesbischer Belletristik
und zu politischen Traktaten kam es ihm stets darauf an, Hirschfelds
Theorie der Homosexualität zu verteidigen und zu erklären. Er hatte sie
sich vollständig zu Eigen gemacht. Sie geht hauptsächlich von zwei
Annahmen aus (Homosexualität sei genauso natürlich und angeboren wie
Heterosexualität, und genau wie bei dieser handele es sich nicht um eine
Krankheit) und folgert daraus, dass der Paragraph 175 des
Reichsstrafgesetzbuches, der homosexuelle Handlungen der Männer
bestrafte, abgeschafft werden müsse. Das Wissenschaftlich-humanitäre
Komitee hatte den Zweck, für dieses Ziel zu werben und die Bevölkerung
über die moralische Berechtigung der Homosexualität aufzuklären. Numa
Praetorius stellte sich nicht allein mit seiner \enquote{Bibliographie}
in den Dienst an dieser Sache.

Das Buch \emph{The Homosexual Emancipation Movement in Germany}, das der
amerikanische Germanist James D. Steakley 1975 vorlegte, könnte man,
obwohl in New York in englischer Sprache erschienen, als
Gründungsmanifest einer schwullesbischen Geschichtsforschung hierzulande
bezeichnen. Steakley hatte an der Freien Universität studiert, war in
der Homosexuellen Aktion Westberlin aktiv und recherchierte für sein
kommendes Buch vor allem in der damals noch existenten Berliner
Medizinischen Zentralbibliothek, die unter anderem alle 23 Jahrgänge des
\emph{Jahrbuchs für sexuelle Zwischenstufen} besaß. In seinem Buch, das
einen ersten skizzenhafte Überblick über die Schwulenbewegung von ihren
Anfängen im 19. Jahrhundert bis zu ihrer Vernichtung im Hitlerfaschismus
bot, erwähnte Steakley zwar die \enquote{reviews of fictional and
non-fictional publications and complete annual bibliographies of
relevant works} (S. 24). Vom Autor, seinem wirklichen Namen und seinem
Leben wussten er und die Mitglieder des Geschichtsarbeitskreises in der
Homosexuellen Aktion Westberlin aber noch nichts.

Steakleys Buch bewirkte einen Aufschwung der schwulen
Geschichtsforschung und noch in den 1970er Jahren konnte ein erster
Hinweis auf die wahre Identität des offensichtlich pseudonymen Numa
Praetorius aufgefunden werden. In einer Broschüre von 1924, \emph{Die
deutsche Bewegung zur Aufhebung des § 175 R.St.G.B.}, verfasst von
Ferdinand Karsch-Haack, einem langjährigen Mitarbeiter Hirschfelds heißt
es, ebenfalls auf Seite 24:

\begin{quote}
\enquote{Von dauerndem Wert {[}im Jahrbuch für sexuelle
Zwischenstufen{]} ist unter anderem die durch fast alle Bände verstreute
,Bibliographie der Homosexualität` von Eugen Wilhelm (Numa Praetorius),
die bedauerlicher Weise mit dem 22. Jahrbuch 1922 ihren Abschluß
gefunden zu haben scheint.}
\end{quote}

Bald schon sollte sich die Richtigkeit dieser Enthüllung von Numas
wirklichem Namen erweisen, die offensichtlich ohne Zustimmung des
Betroffenen als eine Art Zwangsouting erfolgte. Ob Wilhelm davon je
erfahren hat, wissen wir nicht.

Ein großer Schritt nach vorn in der Numa-Forschung gelang 1984 Hartmut
Walravens, einem leitenden Bibliothekar der damals noch Westberliner
Staatsbibliothek, mit seinem 111 Nummern umfassenden Verzeichnis
\emph{Eugen Wilhelm, Jurist und Sexualwissenschaftler. Eine
Bibliographie.} Walravens konnte, wie er im Vorwort mitteilt, unter
anderem mittels stilistischer Textvergleichung eigenständig Numas
Pseudonym auflösen. Wilhelms Lebensdaten hatte er ebenfalls erstmals in
einem Straßburger Archiv ermittelt.

Im folgenden Jahr wurde in Berlin-Kreuzberg das Schwule Museum
gegründet. Es entstand die Idee einer Museumszeitschrift, die
\emph{Capri} heißen sollte, nach der italienischen Sehnsuchtsinsel
schwuler Bildungsbürger im Europa des 20. Jahrhunderts. Dem Vorbild der
Numa Praetoriusschen \enquote{Bibliographie} folgend, sollte
\emph{Capri} ein Referateorgan sein, in dem die neue Literatur zur
Homosexualität besprochen wird. Die Verlage würden die
Rezensionsexemplare gratis zur Verfügung stellen und so den Aufbau einer
Museumsbibliothek ermöglichen. Dieser Plan gelang nur zum Teil.

Walravens` Numa-Bibliographie war in einem Hamburger Kleinverlag
erschienen, so dass wir im Museumsverein erst Jahre später davon
erfuhren. Für das \emph{Capri}-Heft vom November 1990 habe ich eine
Besprechung geschrieben, in der ich das sehr schmale, in blaues Leinen
gebundene und mit Goldschrift auf dem Rücken verzierte Buch wegen seiner
pioniermäßigen Neuheit lobte. Der Besprechung konnte ich einen gerade
erst in der Handschriftenabteilung der Staatsbibliothek entdeckten Brief
Kurt Hillers beifügen, in dem dieser von seiner jahrzehntelangen
Bekanntschaft mit Wilhelm erzählt und Details aus dessen Leben
berichtet.

Ein weiterer Forschungsfortschritt war im Jahr 2003 erreicht, als
Hillers Nachlass für die Öffentlichkeit freigegeben und darin elf Briefe
und Postkarten entdeckt wurden, die Wilhelm zwischen 1936 und 1947 an
Hiller geschickt hatte. Man erfuhr jetzt, dass Wilhelm 1941 von den
deutschen Besatzern Frankreichs wegen seiner Homosexualität ins KZ
gesperrt wurde. Wenngleich nun allmählich immer mehr Einzelheiten aus
Wilhelms Leben und schriftstellerischem Werk bekannt wurden, herrschte
doch noch weitgehende Unkenntnis über Wilhelms Persönlichkeit. Nicht
einmal eine Fotografie war bekannt.

Die Wende trat ein, als 2009 in Paris die Entdeckung der Tagebücher
Wilhelms und des Fotoalbums seiner Familie gelang. Das Tagebuch hatte er
ohne Unterbrechung während der Jahre 1885 bis 1951, dem Jahr seines
Todes, geschrieben. Bisher hat die philologisch-historische Untersuchung
der Tagebücher, die eine kommentierte, transkribierte und ins Deutsche
übersetzte Edition des gesamten Werks zum Ziel hat, eine Fülle neuer
Erkenntnisse gezeitigt. Der Abschluss dieser Arbeit ist noch nicht
abzusehen.

Wilhelm hat in den Jahrgängen 1900 bis 1922 des \emph{Jahrbuchs für
sexuelle Zwischenstufen} den Rezensionenteil betreut und weitgehend
selbst verfasst. Der Titel \enquote{Bibliographie der Homosexualität}
benennt die tatsächlich vorliegende Sammlung von Rezensionen nicht
angemessen. Er wurde aus dem ersten Jahrgang des \emph{Jahrbuchs}
übernommen, in dem es tatsächlich eine solche gab. Sie verzeichnete
etwas mehr als dreihundert Werke aller literarischen Gattungen und
historischen Epochen von Platon bis Krafft-Ebing nach dem
Autorenalphabet geordnet und unannotiert. Richard Meienreis, ein
Mitarbeiter Hirschfelds, hatte sie zusammengestellt, ohne dass sein Name
genannt wurde. Erst viele Jahre später gab Hirschfeld den Namen des
Autors bekannt.

Meienreis hat die Bibliographie der Homosexualität nicht erfunden. Das
vermutlich erste einschlägige Verzeichnis hat 1869 Karl Heinrich
Ulrichs, die Mutter aller schwulen Emanzipationsbewegungen, vorgelegt.
Als Anhang zum neunten Band seiner \emph{Forschungen über das Räthsel
der mannmännlichen Liebe} listet er die ihm bekannten \enquote{Schriften
über Urningsliebe} auf. Diese Liste beginnt mit seinen eigenen
Schriften, die \enquote{durch A. Serbe's Buchhandlung in Leipzig gegen
Einsendung der betreffenden Beträge direct und franco versandt} werden.
Dann folgt ein nach Ländern geordnetes, mit \enquote{I. Griechenland}
beginnendes und mit \enquote{VIII. Deutschland} endendes Verzeichnis von
neunzehn durchnumerierten manchmal annotierten Schriften. Unter
\enquote{III. Italien} nennt er bloß \enquote{Alcibiade fanciullo a
scola, 1652; nach Baseggio's vager Vermuthung von Ferrante Pallavicini;
übersetzt ins Französische: Alcibiade enfant à l'école; 1866}. Er merkt
dazu an: \enquote{Neben naturwissenschaftlich wichtigen Stellen enthält
das Buch so viel schlüpfriges, dass ich es hier nicht aufführen möchte.}
Er führt es aber doch auf und fügt hinzu: \enquote{Wer zu wissen
wünscht, wo und wie es zu erhalten sei, wende sich an mich.}\footnote{Das
  Werk liegt seit 2002 in einer Übersetzung von Wolfram Setz unter dem
  Titel Der Schüler Alkibiades im Hamburger Männerschwarmverlag vor.
  Ziemlich eindeutig konnte inzwischen der römische Arzt und Philosoph
  Antonio Rocco als Verfasser ermittelt werden.}

Meienreis` Bibliographie von 1899 verzeichnet \emph{Alcibiade fanciullo}
als anonymes Werk und hält den vermeintlichen Autor Pallavicini für den
Übersetzer aus dem Italienischen. Die Liste der Schriften zur
Urningsliebe scheinen weder Meienreis noch Wilhelm zur Kenntnis genommen
zu haben.

\textbf{Literatur}

Dubout, Kevin (2011): Eugen Wilhelms Tagebücher. Editorische Probleme,
Transkriptions- und Kommentarprobe, in: Officina Editorica, hrsg. von
Jörg Jungmayr und Marcus Schotte. Berlin, S. 214-304.

Dubout, Kevin (2014): Aufklären, vernetzen, entgegnen. Zur unmittelbaren
Vorgeschichte des WhK (1894-1897), in: Capricen. Momente schwuler
Geschichte, zusammengestellt von Rüdiger Lautmann. Hamburg, S. 15-39.

Herzer, Manfred (1990): {[}Rezension zu Walravens 1984{]}, in: Capri,
Jg. 3, Nr. 3, S. 31-33.

Herzer, Manfred (2004): \enquote{Ich freue mich sehr, dass Sie den Krieg
gut überstanden haben.} Zu einem Brief von Eugen Wilhelm an Kurt Hiller
in London, in: Capri, Nr. 35, S. 32-35.

Karsch-Haack, Ferdinand (1924): Die deutsche Bewegung zur Aufhebung des
§ 175 R.St.G.B. Berlin-Pankow.

{[}Meienreis, Richard{]} (1899): Bibliographie der Homosexualität, in:
Jahrbuch für sexuelle Zwischenstufen unter besonderer Berücksichtigung
der Homosexualität, Jg. 1, S.215-238.

Steakley, James D. (1975): The Homosexual Emancipation Movement in
Germany. New York.

Ulrichs, Karl Heinrich (1869): \enquote{Argonauticus.} Zastrow und die
Urninge des pietistischen, ultramontanen und freidenkenden Lagers.
Leipzig. (Reprint Berlin 1994: Ulrichs, Forschungen über das Räthsel der
mannmännlichen Liebe. VIII. Incubus. IX. Argonauticus.

Walravens, Hartmut (1984): Eugen Wilhelm, Jurist und
Sexualwissenschaftler. Eine Bibliographie. Hamburg.

%autor

\end{document}