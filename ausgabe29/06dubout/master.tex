\documentclass[a4paper,
fontsize=11pt,
%headings=small,
oneside,
numbers=noperiodatend,
parskip=half-,
bibliography=totoc,
final
]{scrartcl}

\usepackage{synttree}
\usepackage{graphicx}
\setkeys{Gin}{width=.4\textwidth} %default pics size

\graphicspath{{./plots/}}
\usepackage[ngerman]{babel}
\usepackage[T1]{fontenc}
%\usepackage{amsmath}
\usepackage[utf8x]{inputenc}
\usepackage [hyphens]{url}
\usepackage{booktabs} 
\usepackage[left=2.4cm,right=2.4cm,top=2.3cm,bottom=2cm,headheight=25.60228pt,includeheadfoot]{geometry}
\usepackage{eurosym}
\usepackage{multirow}
\usepackage[ngerman]{varioref}
\setcapindent{1em}
\renewcommand{\labelitemi}{--}
\usepackage{paralist}
\usepackage{pdfpages}
\usepackage{lscape}
\usepackage{float}
\usepackage{acronym}
\usepackage{eurosym}
\usepackage[babel]{csquotes}
\usepackage{longtable,lscape}
\usepackage{mathpazo}
\usepackage[flushmargin,ragged]{footmisc} % left align footnote

%%url brekas grrr
\def\UrlBreaks{\do\a\do\b\do\c\do\d\do\e\do\f\do\g\do\h\do\i\do\j\do\k\do\l%
\do\m\do\n\do\o\do\p\do\q\do\r\do\s\do\t\do\u\do\v\do\w\do\x\do\y\do\z\do\0%
\do\1\do\2\do\3\do\4\do\5\do\6\do\7\do\8\do\9\do\-}%

\usepackage{listings}

\urlstyle{same}  % don't use monospace font for urls

\usepackage[fleqn]{amsmath}

%adjust fontsize for part

%% geometry
\clubpenalty = 10000 
\widowpenalty = 10000 
\displaywidowpenalty = 10000
%% tightlist

\providecommand{\tightlist}{%
  \setlength{\itemsep}{0pt}\setlength{\parskip}{0pt}}

\usepackage{sectsty}
\partfont{\large}

%Das BibTeX-Zeichen mit \BibTeX setzen:
\def\symbol#1{\char #1\relax}
\def\bsl{{\tt\symbol{'134}}}
\def\BibTeX{{\rm B\kern-.05em{\sc i\kern-.025em b}\kern-.08em
    T\kern-.1667em\lower.7ex\hbox{E}\kern-.125emX}}

\usepackage{fancyhdr}
\fancyhf{}
\pagestyle{fancyplain}
\fancyhead[R]{\thepage}

%meta

%meta

\fancyhead[L]{K. Dubout \\ %author
LIBREAS. Library Ideas, 29 (2016). % journal, issue, volume.
\href{http://nbn-resolving.de/urn:nbn:de:kobv:11-100238160
}{urn:nbn:de:kobv:11-100238160}} % urn
\fancyhead[R]{\thepage} %page number
\fancyfoot[L] {\textit{Creative Commons BY 3.0}} %licence
\fancyfoot[R] {\textit{ISSN: 1860-7950}}

\title{\LARGE{Durch Rezensionen zur Emanzipation? Die \enquote{Bibliographie der Homosexualität}(1900-1922) im \textit{Jahrbuch für sexuelle Zwischenstufen}}} %title %title
\author{Kevin Dubout} %author

\setcounter{page}{}

\usepackage[colorlinks, linkcolor=black,citecolor=black, urlcolor=blue,
breaklinks= true]{hyperref}

\date{}
\begin{document}

\maketitle
\thispagestyle{fancyplain} 

%abstracts

%body
Im Mai 1897 wurde in Charlottenburg bei Berlin das
Wissenschaftlich-humanitäre Ko\-mi\-tee\linebreak (WhK) gegründet. Die weltweit erste,
bis 1933 bestehende Homosexuellenorganisation wurde langjährig vom Arzt
und Sexualforscher Magnus Hirschfeld (1868--1935) geleitet und trat für
die Entkriminalisierung gleichgeschlechtlicher Handlungen zwischen
Männern (\enquote{widernatürliche Unzucht} nach § 175 des deutschen
Reichsstrafgesetzbuches) ein. Zu diesem Zweck legte das Komitee eine
beständige Petitionstätigkeit an den Tag und leistete eine breit
angelegte Aufklärungs- und Beratungsarbeit.\footnote{Überblicke über die
  wechselvolle Geschichte des WhK: Herzer 1997; Herzer
  \textsuperscript{2}2001, S. 98-152; Lehmstedt 2002, S. 71-154;
  Keilson-Lauritz 1997, S. 30-61; Dose 2005, S. 50-63.} Entsprechend dem
Motto \enquote{Durch Wissenschaft zur Gerechtigkeit} war das Komitee
außerdem als wissenschaftliche Forschungsstätte tätig, die sich der
Erforschung aller \enquote{sexuellen Zwischenstufen}, besonders der
Homosexualität, widmete, und ihre Forschungsergebnisse geltend machen
wollte, um Sexualreform voranzutreiben.

Die \enquote{Zwischenstufenlehre} wurde von Hirschfeld als
\enquote{Einteilungsprinzip} entwickelt und über die Jahre immer weiter
ausgearbeitet. Demnach treten \enquote{alle menschlichen Eigenschaften,
seien sie körperlicher oder seelischer Art, {[}\ldots{}{]} in weiblicher
oder in männlicher Form auf} und lassen sich nach vier Hauptkriterien
(Bildung der Geschlechtsorgane, weitere körperliche Eigenschaften,
Richtung des Geschlechtstriebes und sonstige psychische Eigenschaften)
einordnen: Während \enquote{Vollmann} und \enquote{Vollweib} bloße
Idealtypen seien, charakterisiere sich im Grunde jeder Mensch durch eine
einzigartige Mischung aus weiblichen und männlichen
Eigenschaften.\footnote{Dose 2005, S. 97f. (Zitat S. 97).} Vor diesem
Hintergrund wurde Homosexualität zu einer
\enquote{gemischtgeschlechtlichen} Kombination unter vielen anderen --
zu einer Zwischenstufe auf der Ebene des Geschlechtstriebes. Als
angeborene, unveränderliche, natürliche und weitgehend
entpathologisierte Veranlagung sollte sie nicht strafrechtlich verfolgt
werden.

Zwischen 1899 und 1923 wurde von Hirschfeld im Auftrag des WhK das
\emph{Jahrbuch für sexuelle Zwischenstufen} herausgegeben. Die 23
Jahrgänge legen mit ihren zahlreichen medizinischen, juristischen,
kulturhistorischen, literarischen Studien und Materialsammlungen ein
beeindruckendes Zeugnis von der beharrlichen Bewegungsarbeit und der
wissenschaftlichen Tätigkeit des WhK ab.\footnote{Zur Geschichte des
  Jahrbuchs für sexuelle Zwischenstufen: Dobler (Hrsg.) 2004.} Innerhalb
des \emph{Jahrbuchs} nahm die von 1900 bis 1922 von \enquote{Dr.~jur.
Numa Praetorius} verfasste \enquote{Bibliographie der Homosexualität}
einen besonderen Platz ein. Hinter diesem Pseudonym verbarg sich der
Straßburger Jurist Eugen Wilhelm (1866--1951), den Erinnerungen
Hirschfelds zufolge \enquote{der weitaus produktivste
Mitarbeiter}\footnote{Hirschfeld 1986, S. 63.} des \emph{Jahrbuchs}.
Wilhelm beteiligte sich von Anfang an maßgeblich an den Bestrebungen des
Komitees, das er nicht nur schriftstellerisch, sondern auch
organisatorisch und finanziell unterstützte.\footnote{Überblick über
  Leben und Werk Eugen Wilhelms in: Herzer 1993 und 2009.} Neben
juristischen und biographischen Arbeiten stellt die
\enquote{Bibliographie der Homosexualität} wohl seine mit Abstand
umfangreichste Leistung im Rahmen der damaligen Homosexuellenbewegung
dar. Bemerkenswert ist, dass es sich beim Bibliographen um einen
Juristen handelte, der sich als zuständig erachtete, über sein
ursprüngliches Kerngebiet hinaus auch medizinische, belletristische und
kulturhistorische Untersuchungen zu besprechen und zu bewerten. Deshalb
zeichnen sich die Bibliographien durch vielfache Grenzüberschreitungen
und einen gemischten Charakter aus, der der Hybridität des
\emph{Jahrbuchs} -- Fachzeitschrift und Kampforgan zugleich --
entsprach.

Durch seine Bibliographien positionierte sich Wilhelm/Praetorius in
sexualtheoretischen und strategischen Debatten inner- und außerhalb der
Homosexuellenbewegung: Er registrierte nicht nur das vorhandene Wissen,
sondern griff dadurch performativ in die Wissensproduktion über
Homosexualität ein. Den Stellenwert der \enquote{Bibliographie der
Homosexualität} für das WhK verdeutlicht die Tatsache, dass sie
regelmäßig einen beträchtlichen Anteil (bis zu einem Drittel) des
Gesamtumfangs des \emph{Jahrbuchs} beanspruchte. Die systematischen
Besprechungen mehrerer hundert Titel auf insgesamt ca. 2300 Druckseiten
haben sich bis heute als ergiebige Quellen für die Geschichte der
Homosexualitäten bewährt. Dies wirft die Frage auf, worin der Mehrwert
der Bibliographie für das WhK bestand und welche Funktionen sie im
\enquote{Befreiungskampf} erfüllte. Welche Ziele wurden mit ihr vom
Komitee und insbesondere von Praetorius verfolgt? Inwiefern verstand sie
sich als \enquote{emanzipatorisch}? Welchen Status beanspruchte der
Bibliograph auf dem Gebiet der damaligen
\enquote{Homosexualitätsforschung} und welche Ressourcen machte er
geltend?

Drei Hauptanliegen stechen hervor: Numa Praetorius lag erstens daran,
die Bibliographien als wissenschaftliches Forum für die Diskussion um
Homosexualität zu gestalten. Zweitens stellten sie eine
Legitimierungsarbeit dar, durch welche die Expertise des WhK auf diesem
Gebiet etabliert werden sollte. Praetorius war ein unermüdlicher
Verfechter und Vermittler der Hirschfeldschen Zwischenstufenlehre.
Drittens entwarf und propagierte er durch seine Besprechungen das in
seinen Augen emanzipatorische Bild eines \enquote{normalen}
beziehungsweise \enquote{durchschnittlichen} männlichen Homosexuellen,
das Hirschfelds eigene Normalisierungsbestrebungen\footnote{Hierzu:
  Herrn 2009, S. 289f.} unterstützte und ergänzte. Weibliche
Homosexualität wurde hingegen viel seltener thematisiert, vor allem
deshalb, weil der Schwerpunkt der Arbeit des WhK im Allgemeinen der
\enquote{Bibliographie der Homosexualität} im Besonderen doch auf
männlicher Homosexualität lag. Das Augenmerk wird im Folgenden
hauptsächlich auf die \enquote{nichtbelletristischen}, das heißt
\enquote{wissenschaftlichen} Besprechungen aus den Bibliographien von
1900 bis 1910 gerichtet, welche dennoch als repräsentativ gelten
dürfen.\footnote{Eine systematische Untersuchung der Rezensionen zur
  Belletristik und der \enquote{Erfindung einer schwulen
  Literaturkritik} durch die Homosexuellenbewegung liegt bereits vor:
  Keilson-Lauritz 1997.}

\newpage

\section*{1.) Ein wissenschaftliches
Forum}\label{ein-wissenschaftliches-forum}

\subsection*{Aufbau der Bibliographien und Strukturierung des
Felds}\label{aufbau-der-bibliographien-und-strukturierung-des-felds}

Die Bibliographien des \emph{Jahrbuchs} waren keineswegs die erste
bibliographische Unternehmung auf dem Gebiet der Homosexualität. Schon
1897 hatte der französische Schriftsteller Marc-André Raffalovich
(1864--1934) -- wohlgemerkt ebenfalls kein Arzt -- in der
kriminologischen Zeitschrift \emph{Archives de l'anthropologie
criminelle} sogenannte \emph{Annales de l'unisexualité}
(\enquote{Jahrbücher der Unisexualität}) vorgelegt: Eine umfangreiche,
mehrsprachige Dokumentation wurde bereitgestellt und kommentiert, die
von der Literatur über aktuelle Zeitungsausschnitte bis hin zu
historischen Dokumenten reichte.\footnote{Cardon 2008, insb. S. 181-243.}
Auf den ersten Jahrgang folgten zwei weitere \emph{Chroniques de
l'unisex\-ualité} 1907 und 1909. Während Raffalovich die freie Form einer
Chronik auswählte und sich ausdrücklich vom wissenschaftlichen Stil
abgrenzte\footnote{Cardon 2008, S. 75f.}, ging Praetorius systematisch,
möglichst erschöpfend und betont wissenschaftlich vor. Seine
Bibliographien setzten sich zum Ziel, einen möglichst lückenlosen
Überblick über das Forschungsfeld zu geben, wenngleich der Anspruch auf
Vollständigkeit aufgrund der unüberschaubare Zahl der Neuerscheinungen
bald aufgegeben werden musste.

Da ihr Anliegen nicht nur darin bestand, relevante Publikationen zu
registrieren, sondern diese kritisch einzuordnen und damit
unterschiedliche Positionen zu markieren, das heißt, das Forschungsfeld
insgesamt zu strukturieren, erscheint es lohnend, die Entwicklung des
Aufbaus der Bibliographien nachzuzeichnen. Grundlegend war die über den
gesamten Untersuchungszeitraum fortbestehende Unterscheidung zwischen
\enquote{wissenschaftlichen} und \enquote{belletristischen}
(beziehungsweise \enquote{nichtwissenschaftlichen}) Schriften.
Einleuchtend sind in diesem Zusammenhang die von Jahr zu Jahr
schwankenden Unterteilungen des wissenschaftlichen Teils und die damit
einhergehenden Begründungen. In den frühen Bibliographien war die
Medizin noch ein unhinterfragter Bezugspunkt: So unterschied Praetorius
bis 1902 zwischen \enquote{Schriften der Mediziner} und
\enquote{Schriften der Nicht-Mediziner (Juristen, Ethiker, Philosophen,
etc.)}.\footnote{Praetorius 1900, S. 345-349 (Zitat S. 345).} Nachdem
sich jedoch die ursprünglich nicht näher definierte Kategorie des
\enquote{Nichtmedizinischen} ausdifferenziert hatte, fand 1904 zum
ersten Mal eine thematische Gliederung Anwendung. Unterschieden wurde
nun zwischen \enquote{Homosexualität und Angeborensein} (mit den meisten
besprochenen Titeln), \enquote{Die neueste Richtung},
\enquote{Homosexualität und Erwerbung}, \enquote{Die Anhänger der
Strafe} und \enquote{Der Geschlechtstrieb an und für sich. (Ohne
Berücksichtigung der Homosexualität.)}\footnote{Praetorius 1904, S.
  457-594.} Beim \enquote{Mapping} des Forschungsstands orientierte sich
Praetorius somit an zwei Kriterien: An der theoretischen Ausrichtung
(Position zur \enquote{Entstehungsart der Homosexualität}) und an der
praxisbezogenen Haltung zur Strafbarkeit. Emanzipationspolitisch wurde
überprüft, ob die Autoren \enquote{mehr zu den neueren oder mehr zu den
älteren Anschauungen über Homosexualität neig{[}t{]}en}.\footnote{Praetorius
  1904, S. 452, Note 2.} Das deutliche quantitative Überwiegen der
Schriften, die der Angeborenheitstheorie anhingen, deute auf die
(Praetorius zufolge) wissenschaftlich nun tonangebende Richtung hin,
während abweichende Positionen innerhalb der Bewegung (die
\enquote{neueste}, das heißt maskulinistische\footnote{In Anlehnung an
  Claudia Bruns werden im Folgenden diejenigen Positionen und
  Forderungen innerhalb der Homosexuellenbewegung als
  \enquote{maskulinistisch} bezeichnet, welche durch \enquote{die
  Betonung {[}der{]} Zugehörigkeit zum männlichen Geschlecht und dessen
  besondere Leistungen für den Staat eine kulturelle Höherwertigkeit
  mannliebender Männer postulier{[}t{]}en} und sich der
  Zwischenstufenlehre widersetzten (Bruns 2008, S. 16 und 138, Zitat S.
  138).} \enquote{Richtung}) und außerhalb (die \enquote{Anhänger der
Strafe}) ausgesondert wurden. Bemerkenswert ist auch das erstmalige
Vorkommen einer allgemeinen Kategorie (\enquote{Der Geschlechtstrieb an
und für sich}), das die ansetzende Verselbständigung der
Sexualwissenschaft als Disziplin markierte. Diese Einteilung wurde mit
einer ähnlichen quantitativen Gewichtung der unterschiedlichen Ansätze
1905 beibehalten; allein der Abschnitt \enquote{Die Anhänger der Strafe}
fiel weg, \enquote{da sich keine Befürworter der Aufrechterhaltung der
Strafe in den besprochenen Schriften mehr vorfanden} \footnote{Praetorius
  1905, S. 673.}. 1906 bestätigte sich die grundlegende Trennlinie
zwischen \enquote{Homosexualität und Angeborensein} (3/4 des
Gesamtumfangs) und \enquote{Homosexualität und Erwerbung}.\footnote{Praetorius
  1906, S. 706-834 (\enquote{Homosexualität und Angeborensein} 41
  besprochene Titel) und S. 834-866 (\enquote{Homosexualität und
  Erwerbung}, 13 Titel).} Daraus wird ersichtlich, dass die Einteilung
einen kontinuierlichen Aufwärtstrend der Angeborenheitstheorie sowie
einen unaufhaltsamen Fortschritt der Bewegung begünstigte. Die infolge
des Eulenburg-Skandals\footnote{Als Eulenburg-Skandal wird eine Reihe
  von skandalträchtigen Enthüllungen und Prozessen zwischen 1906/07 und
  1909 bezeichnet, in die enge Berater und Freunde des Kaisers Wilhelm
  II., darunter Philipp Fürst zu Eulenburg, verwickelt waren. Es ging um
  ihre vermeintliche oder tatsächliche Homosexualität und ihren Einfluss
  auf Kaiser und Politik. Die Prozesse bedeuteten einen herben
  Rückschlag für die damalige Homosexuellenbewegung. Hierzu: Domeier
  2010.} 1907 bis 1909 eingetretene Krise des WhK spiegelte sich
freilich auch in der Gliederung des wissenschaftlichen Teils der
Bibliographie wieder: Unterschieden wurde nicht mehr zwischen
wissenschaftlichen Richtungen, sondern zwischen \enquote{vor} und
\enquote{nach} dem Beginn der \enquote{Skandalprozesse}, deren
einschneidende Bedeutung damit deutlich zum Ausdruck kam.\footnote{Praetorius
  1908, S. 431-578.} Ab 1909 wurde die Bibliographie vor allem aufgrund
des neuen Erscheinungsformats des \emph{Jahrbuchs} (in
Vierteljahreshefte bescheideneren Umfangs) nicht mehr thematisch,
sondern nur noch alphabetisch gegliedert.

\subsection*{Eine wissenschaftliche Streitkultur
einfordern}\label{eine-wissenschaftliche-streitkultur-einfordern}

In der \enquote{Bibliographie der Homosexualität} wurde an der
Etablierung einer wissenschaftlichen und sachlichen Streitkultur um das
Thema Homosexualität gearbeitet. Die besprochenen Schriften wurden
zuallererst auf ihren wissenschaftlichen Gehalt hin geprüft; von allen,
besonders aber von den Gegnern verlangte Praetorius zumindest das
Einhalten wissenschaftlicher Umgangsformen und Maßstäbe. Wenn ein Autor
diesen Standards und einem Minimum an korrektem Ton nicht genügte,
erübrigte sich für ihn eine eingehende Besprechung. Dass eine solche
Forderung gerade in den ersten Jahren des Komitees alles andere als
selbstverständlich war, zeigt der Umstand, dass umgekehrt auch
gegnerische Schriften positiv hervorgehoben wurden, wenn sie zumindest
einen höflichen Ton an den Tag legten: \enquote{Das Schriftchen zeichnet
sich trotz seiner von der des Komitees abweichenden Ansichten doch
vorteilhaft von anderen Gegenschriften durch den ruhigen, würdigen Ton
und die ernste Diskussionsweise aus. {[}\ldots{}{]} Mit anständigen
Gegnern kämpft man gern.}\footnote{Praetorius 1905, S. 851f.}

Die Definition eines angemessenen Umgangs mit der Homosexualitätsfrage
durch die Sicherung von wissenschaftlichen Qualitätsstandards diente dem
Zweck, die Bibliographien als Ort des kritischen, aber konstruktiven und
sachlichen Dialogs zu gestalten und dadurch dem \emph{Jahrbuch} und dem
WhK zur wissenschaftlichen Legitimität zu verhelfen. Dies war u.a. daran
sichtbar, dass Besprechungen zu den \emph{Jahrbuch}-Rezensionen bis 1906
ein fester Bestandteil der Bibliographien waren. Praetorius hob dabei
vor allem die Stellen hervor, an denen die Objektivität, der
wissenschaftliche Wert und daher der überzeugende Charakter des
\emph{Jahrbuchs} unterstrichen wurden. Entsprechend dem
Wissenschaftsverständnis Wilhelms und des WhK überhaupt (\enquote{Die
Aufsätze sollen der Propaganda und Belehrung dienen, aber in erster
Linie sollen sie wissenschaftlichen Charakter aufweisen und durch diesen
Charakter der Wissenschaftlichkeit und Wahrhaftigkeit wirken}\footnote{Praetorius
  1903, S. 1145.}) kam eine zunehmende wissenschaftliche Anerkennung des
\emph{Jahrbuchs} einem Fortschritt der Bewegung gleich. Freilich scheint
es, dass die explizite Verknüpfung von Wissenschaft und Bewegungsarbeit
gegen das Gebot der \enquote{Objektivität} und
\enquote{Unparteilichkeit} verstieß und eine zwiespältige, oft aber
schließlich ablehnende Haltung der Fachwelt begründete.

Nichtsdestoweniger wurde in den frühen Jahrgängen des \emph{Jahrbuchs}
der anscheinend unaufhaltsame Fortschritt der Aufklärung suggeriert und
die Lernfähigkeit beziehungsweise die zunehmende Einsicht der Fachwelt
begrüßt. So erschien 1901 einer der letzten Arbeiten des einflussreichen
Psychiaters Richard von Krafft-Ebing (1840--1902), in dem er die
\enquote{konträre Sexualempfindung} erstmals weder für ein
Degenerationszeichen noch für eine Krankheit hielt, an prominenter
Stelle.\footnote{Krafft-Ebing 1901.} Nachahmenswerte, beinahe als
Bekehrungsgeschichten stilisierte Beispiele von Gesinnungswandel bei
ehemaligen Gegnern wurden in den Bibliographien ebenfalls
herausgestellt. Nachdem etwa der Psychiater Paul Näcke (1851--1913)
unter Hirschfelds Führung im Oktober 1903 den Berliner Homosexuellen
einen Besuch abgestattet, \enquote{zahlreiches lebendiges Material}
persönlich kennengelernt und sich damit auf \enquote{das Studium der
Wirklichkeit} eingelassen hatte, ereignete sich \enquote{eine heilsame
Umwandlung} in ihm.\footnote{Praetorius 1905, S. 752.} Zu den
wissenschaftlichen Qualitäten gehörte demnach auch die Bereitschaft,
sich eines Besseren zu besinnen, frühere irrtümliche Ansichten
aufzugeben und durch selbständiges, kritisches Denken zu einem neuen,
nämlich \enquote{so ziemlich auf den von Dr.~Hirschfeld und mir
eingenommenen Standpunkt}\footnote{Praetorius 1903, S. 1007.} zu
gelangen.

\section*{2.) Legitimierungsarbeit nach außen und nach
innen}\label{legitimierungsarbeit-nach-auuxdfen-und-nach-innen}

Ein wesentliches Ziel der Bibliographien Praetorius' bestand darin, den
sexualtheoretischen und -reformerischen Standpunkt des WhK zu vertreten,
zu verteidigen und zu legitimieren. Sie entwickelten sich somit zu einem
Ort, an dem die Diskursproduktion über Homosexualität einer
systematischen kritischen Kontrolle unterzogen wurde. Einerseits wurde
überprüft, ob eine Übereinstimmung mit den Ansätzen und
Schlussfolgerungen der Zwischenstufenlehre (konstitutionelle Grundlage
der Homosexualität durch die bisexuelle Embryonalanlage, unzählige
Geschlechtsübergänge, Straffreiheit für den einvernehmlichen
Sexualverkehr unter Erwachsenen), oder zumindest die Anerkennung einer
angeborenen gleichgeschlechtlichen Sexualempfindung vorhanden war,
andererseits wurden gegnerische Ansichten unermüdlich widerlegt. Von
übergeordneter Bedeutung war für Praetorius aber letzten Endes die
Frage, ob trotz aller theoretischen Unterschiede für die Aufhebung,
Beibehaltung oder Reform des § 175 eingetreten wurde. Von
wissenschaftstheoretischen oder politischen Abweichungen konnte
abgesehen werden, solange in strafrechtlicher Beziehung Übereinstimmung
bestand. In der Rezension zu Otto Weiningers \emph{Geschlecht und
Charakter} (1903) rügte Praetorius zum Beispiel zwar die
Frauenfeindlichkeit und den Antisemitismus des Autors, im Mittelpunkt
standen jedoch Gesichtspunkte, die für das Komitee von unmittelbarer
Relevanz waren, wie die Anerkennung der Natürlichkeit der Zwischenstufen
und des angeborenen Sexualtriebes sowie die Kritik am § 175:
\enquote{Hoch erfreulich} sei bei allen sonstigen Vorbehalten, dass
\enquote{Weininger sich in den Bahnen der neuesten Spezialwissenschaft
über Homosexualität bewegt}.\footnote{Praetorius 1904, S. 520-527 (Zitat
  S. 526).}

\subsection*{Empirische
Erkenntnisgewinnung}\label{empirische-erkenntnisgewinnung}

Seit seiner Gründung war das WhK bestrebt, zum einen die Erforschung der
sexuellen Zwischenstufen als eigenständiges Forschungsgebiet zu
etablieren, zum anderen sich selbst durch seine Fachzeitschrift als
Autorität zu profilieren. Insbesondere in den Bibliographien wurde der
Kampf um die Deutungshoheit über Homosexualität ausgetragen, die Grenze
zwischen Expertise und Laienhaftigkeit gezogen, Kriterien für
\enquote{wirkliche} Sachkenntnis und legitime Herangehensweisen
ausgehandelt. Praetorius nahm für sich in Anspruch, mitzubestimmen, wer
in dieser Expertenfrage als kompetent gelten durfte:

\begin{quote}
Für die Entscheidung der Frage, wo Irrtum und Wahrheit auf homosexuellem
Gebiet liegt, sind nicht nur die Laien inkompetent, sondern auch viele
Juristen und Aerzte. Nur diejenigen sind als Sachverständige zu
betrachten, welche ausser dem wissenschaftlichen Studium der
Homosexualität praktische Erfahrung in dieser Materie besitzen, d. h.
Homosexuelle kennen gelernt und beobachtet haben.\footnote{Praetorius
  1902a, S. 675.}
\end{quote}

Praetorius wies somit die Diskussion um Homosexualität zwar als ein
Expertengebiet aus, nicht jedoch zwangsläufig als ärztliches Gebiet. Die
Unterscheidung ist insofern von Relevanz, als dass sie ihm ermöglichte,
auch als Nichtarzt seinen Standpunkt, den er freilich als medizinisch
gestützt darstellte, geltend zu machen. Umgekehrt war in seinen Augen
eine berufliche Tätigkeit als Arzt beziehungsweise Psychiater keineswegs
ausreichend, um als kompetent zu gelten. Als ab 1900 eine
\enquote{Entspezialisierung des sexuellen Wissens} stattfand, also
sexualwissenschaftliche Kategorien sich rasch verbreiteten und teilweise
zu Alltagskategorien wurden,\footnote{Müller 1991, S. 267-325; Bruns
  2008, S. 184-186.} war Praetorius zugleich bemüht, den
fachwissenschaftlichen Status dieses Wissens zu sichern und für den
\enquote{Befreiungskampf} wirksam einzusetzen.

Zwingende Voraussetzung für die Sachkenntnis war in seinen Augen eine
möglichst umfangreiche Empirie: \enquote{Zur Kenntnis der Homosexualität
gehört persönliche Untersuchung der Homosexuellen, und zwar vieler
Homosexueller}.\footnote{Praetorius 1904, S. 554.} Waren bei einem Autor
keine empirischen Kenntnisse nachgewiesen, so wurde ihm das
Mitspracherecht verweigert. Dementsprechend schreckte der Jurist
Praetorius nicht davor zurück, Ärzte zu berichtigen oder als
dilettantisch darzustellen. Der Vorwurf der mangelnden Sachkenntnis und
ungenügenden praktischen Erfahrung entwickelte sich zum
meisteingesetzten Disqualifizierungsmittel in den Bibliographien. Oft
hätte das WhK mit realitätsfremden Gegnern zu tun, die in sturer
Unkenntnis der neuesten Forschungsergebnisse ein \enquote{borniertes},
\enquote{rückständiges} Verhalten an den Tag legten. Der Umstand, dass
alle \emph{Jahrbuch}-Mitarbeiter für die Aufhebung des § 175 waren,
sprach also nicht gegen ihre Wissenschaftlichkeit und Objektivität, da
\enquote{{[}s{]}ämtliche deutschen Spezialforscher auf dem Gebiet}
ebenfalls \enquote{auf Grund ihrer wissenschaftlichen
Forschungen}\footnote{Praetorius 1901, S. 478.} dasselbe verlangten. So
ließ sich für Praetorius der vermeintliche Zirkelschluss erklären: Nur
diejenigen, die der Anlagetheorie anhingen und sich für die Aufhebung
der Strafbestimmungen aussprachen, konnten als Sachverständige gelten;
wer umgekehrt ein wirklicher Sachverständiger war, konnte aufgrund
seiner praktischen Erfahrungen nur zu den gleichen Schlussfolgerungen
kommen.

Praetorius stellte sich nicht nur sexualtheoretisch auf den Standpunkt
Hirschfelds, sondern\linebreak schlug auch den gleichen resolut empirischen Weg
ein. Auch er reklamierte für sich eine empirisch gestützte Sachkenntnis.
Da er als Nichtarzt nicht in erster Linie medizinisch argumentierte,
stand bei ihm die Darstellung der \enquote{sozialen Facetten der
Homosexualität} deutlicher im Vordergrund: Bei den von ihm mitgeteilten
Beispielen, die die Stichhaltigkeit des Standpunkts des WhK belegen
sollten, handelte es sich nicht um Patienten, sondern um
\enquote{individuelle Menschen inmitten einer spezifischen
gesellschaftlichen Situation}\footnote{Benkel 2014, S. 399.}. \footnote{}Das
Heranziehen einer reichhaltigen \enquote{persönlichen Erfahrung} war
allerdings nicht ohne Gefahr, verstieß Praetorius doch damit gegen das
Gebot wissenschaftlicher \enquote{Objektivität}.

\subsection*{\texorpdfstring{Die Mehrdeutigkeit der \enquote{persönlichen
Erfahrung}}{Die Mehrdeutigkeit der persönlichen Erfahrung}}\label{die-mehrdeutigkeit-der-persuxf6nlichen-erfahrung}

In den Bibliographien war die Berufung auf eigene Erfahrungen
allgegenwärtig. Mit solchen Belegen ließen sich insbesondere
unbegründete Befürchtungen beschwichtigen und irrtümliche Ansichten
widerlegen, sei es bezüglich der Hoffnung auf Heilung durch Hypnose
(\enquote{Die mir bekannten Homosexuellen, die sich der Hypnose
unterzogen, sind unverändert homosexuell geblieben}), der Verbreitung
von Homosexualität (\enquote{Nach meinen Erfahrungen kommt schlimmsten
Falles einer auf 200-300 Männer}), der Verführung durch die Lektüre
einschlägiger Schriften (\enquote{Unter den Homosexuellen der Mittel-
und Volksklassen habe ich so gut wie nie solche gefunden, welche irgend
etwas über Homosexualität gelesen hatten}) oder der Erwerbung durch
Übersättigung (\enquote{ich {[}habe{]} überhaupt heterosexuelle Roués
{[}das heißt Lebemänner, K.D.{]}, die auf den gleichgeschlechtlichen
Verkehr als ein letztes Reizmittel verfielen, noch nicht kennen
gelernt}).\footnote{Praetorius 1901, S. 382 und 433; 1903, S. 979 und
  999.} Unter Berufung auf die eigene Erfahrung konnten umgekehrt
sexualtheoretische Hypothesen empirisch bestätigt werden (\enquote{Auch
mir ist ein ganz ähnlich fühlender psychischer Hermaphrodit
bekannt}).\footnote{Praetorius 1905, S. 701.} Außerdem kam die
persönliche Erfahrung zum Einsatz, um die in allen von Praetorius
besuchten Ländern vorhandene, von der jeweiligen Gesetzeslage
unabhängige Verbreitung der Homosexualität unter Beweis zu stellen.

Da die Expertise auf homosexuellem Gebiet am Umfang der
\enquote{persönlichen Erfahrungen} gemessen wurde, drängt sich die Frage
auf, was genau unter \enquote{persönlich} zu verstehen war. Wenn von
konkreten sexuellen Handlungen die Rede war, stützte sich Praetorius
tendenziell auf \enquote{Mitteilungen} von (realen oder frei erfundenen)
\enquote{Gewährsmännern}. Mitunter kann nachgewiesen werden, dass eigene
Erlebnisse als Mitteilung eines \enquote{zuverlässigen Gewährsmannes}
getarnt wurden, wie etwa im \enquote{interessante{[}n{]} Fall} eines
\enquote{akademisch gebildeten, den höheren Gesellschaftskreisen
angehörigen, auch schriftstellerisch bekannten, durchaus
vertrauenswürdigen Mann{[}es{]}}.\footnote{Praetorius 1906, S. 796-798
  (Zitat S. 796); 1905, S. 757-759 (Zitat S. 757).} Hier rekurrierte
Praetorius auf das, was Patrick Cardon das \enquote{Alibi der
Begleitung} nennt: die Heranziehung einer fiktiven Gewährsperson, die
den Autor bei seinen Erkundungen angeblich begleitete und einerseits für
dessen Seriosität und Wissenschaftlichkeit, andererseits für dessen
sexuelle Unbeteiligtheit bürgte.\footnote{Cardon 1994, S. 100.} Ob mit
\enquote{persönlicher Erfahrung} eigenes \enquote{Betroffensein} gemeint
war, ließ Praetorius oft in der Schwebe und entzog sich damit einer
klaren Einordnung.

Waren hingegen die Ausführungen nicht direkt sexuellen, sondern
allgemeinen Charakters, etwa über städtische Subkulturen im In- und
Ausland, männliche Prostitution oder Erpressung, so konnte sich
Praetorius -- auch dank der Verwendung eines Pseudonyms -- durchaus als
Erkenntnissubjekt präsentieren. Da \enquote{Dr.~jur. Numa Praetorius}
eindeutig kein Arzt war, lasen sich die geschilderten Erfahrungen
keineswegs als ärztliche Patientenbeobachtungen. Ein wichtiger
Unterschied zu den autobiographischen Patientengeschichten aus der
\emph{Psychopathia sexualis} Krafft-Ebings im späten 19. Jahrhundert
bestand folglich darin, dass die in den Bibliographien mitgeteilten
Erfahrungen nicht als ein klinisch auszuwertendes Material ins Feld
geführt wurden, sondern ausdrücklich als Korrektiv und Ergänzung zu dem
begrenzten Blickfeld des Fachmanns. Gerade deshalb, weil sie dem
\enquote{authentischen}\footnote{Das Kriterium der
  \enquote{Authentizität} diente in seinen Rezensionen zur Belletristik
  ebenfalls als Gegengewicht zur Fachperspektive des Arztes oder
  Juristen: Keilson-Lauritz 1997, S. 259-268, insb. S. 266f.}
homosexuellen Alltagsleben entnommen waren, beanspruchten sie keinen
bloß individuellen, dokumentarischen Wert, sondern wissenschaftliche
Beweiskraft. Es genügte daher Praetorius zufolge die Anführung eines
Beispiels aus dem eigenen Alltag, um dadurch den gegnerischen Standpunkt
als widerlegt, weil realitätsfremd, zu erachten. Gegen die Vorstellung
einer ärztlichen Deutungshoheit in letzter Instanz redete Praetorius der
Berechtigung bewanderter Homosexueller, auf Augenhöhe mit Spezialisten
über Homosexualität zu sprechen, offen das Wort: Zwar würden manche
\enquote{beschönigen und übertreiben, überhaupt zu subjektiv färben},
jedoch sei es auch möglich, \enquote{vom ernsten, gebildeten,
zuverlässigen Homosexuellen am besten die sichersten und massgebendsten
Aufschlüsse zu erhalten}.\footnote{Praetorius 1902b, S. 869.}

Wie strittig der vermeintlich ausschlaggebende Stellenwert der
\enquote{persönlichen Erfahrung} war, lässt sich am Beispiel der Kritik
nachzeichnen, die Praetorius gegen die Psychoanalyse richtete. Seine
1906 erschienene Rezension zu den \emph{Drei Abhandlungen} Sigmund
Freuds zählt zu den ersten gründlichen Auseinandersetzungen mit der
frühen psychoanalytischen Bewegung.\footnote{Praetorius 1906, S.
  729-748. Zu den frühen, letztendlich gescheiterten
  Annäherungsversuchen der organisierten Homosexuellenbewegung mit der
  Psychoanalyse zwischen 1905 und 1911: Herzer \textsuperscript{2}2001,
  S. 153-197 (zur Rolle Wilhelms insb. S. 158-161); Sigusch 2008, S.
  268-273.} Doch trotz der wohlwollenden Anerkennung des Neuartigen an
der Methode hatte Praetorius (genauso wie Hirschfeld) durch das
Festhalten an der Zwischenstufenlehre letztendlich kein Verständnis für
psychoanalytische Grundkonzepte und Erklärungsansätze. Keinesfalls
hinnehmbar war für ihn die Infragestellung der autobiographischen
Anamnese als Instrument der Wahrheitsfindung zugunsten der
psychoanalytischen Methode, die er als Enteignung und Entmachtung
verstand, weil sie den Verlust der Deutungshoheit markierte und die
Homosexuellen wieder zu Patienten machte:

\begin{quote}
Ich muß gestehen, daß es mir unverständlich ist, wie das Ausfragen und
Herausanalysieren seitens eines Dritten bei solchen Menschen mehr
Wahrheit herausbefördern soll als das eigene Selbsterforschen und
Selbstnachdenken über die seelischen Regungen und Empfindungen seitens
eines klar denkenden, in der eigenen Selbstzergliederung gewandten
Mannes.\footnote{Praetorius 1909, S. 210.}
\end{quote}

Als entscheidend hatten daher nach wie vor die Aussagen der \enquote{in
der Analyse ihres Ichs geschulte{[}n{]} Homosexuelle{[}n{]}}\footnote{Praetorius
  1910, S. 438.} zu gelten, deren Wissen über sich selbst fundierter war
als dasjenige noch so ausgewiesener Außenstehender. Für Praetorius
sollte die Wissensproduktion über Homosexualität im Rahmen eines
medizinisch gestützten Fachdiskurses erfolgen, in dem jedoch die
\enquote{Betroffenen} selbst aufgrund ihrer reflektierten
\enquote{Erfahrungen} weiterhin einen Expertenstatus für sich
beanspruchen durften.

In den Bibliographien wurde die Vielfalt der legitimen Herangehensweisen
immer wieder betont. Im Einklang mit dem Selbstverständnis des
\emph{Jahrbuchs} definierte Praetorius die
\enquote{Homosexualitätsforschung} von Vornherein als
interdisziplinär.\footnote{Sigusch 2008, S. 108.} Galt also die Medizin
als der aussichtsreichste Weg, um den Wegfall der Strafbestimmungen
herbeizuführen, so waren andere Zugänge zur homosexuellen Frage genauso
berechtigt, solange sie wissenschaftlich informiert und dem
\enquote{Emanzipationskampf} nicht hinderlich waren. Zu den Aufgaben der
Bibliographien gehörte demnach die Verteidigung nichtmedizinischer
Zugänge, in erster Linie der \enquote{homosexuellen
Belletristik},\footnote{Keilson-Lauritz 1997, S. 213-268.} der
Volksaufklärung, die sowohl ein erklärtermaßen wichtiges Ziel des WhK
als auch einer der Programmschwerpunkte des dem Komitee nahestehenden
Max Spohr Verlags war,\footnote{Lehmstedt 2002, S. 118-125.} und der
kulturgeschichtlichen beziehungsweise biographischen Herangehensweise,
mit der sich eine emanzipationspolitisch einsetzbare
\enquote{Ahnengalerie} -- \enquote{große Männer} der Geschichte, die als
Homosexuelle dargestellt wurden -- zusammenstellen ließ.\footnote{Micheler
  / Michelsen 2001; Micheler 2005, S. 130-133.}

\subsection*{Ein Machtmittel innerhalb der
Homosexuellenbewegung}\label{ein-machtmittel-innerhalb-der-homosexuellenbewegung}

Numa Praetorius vertrat eine aufklärerisch-liberale Haltung und legte in
seiner Argumentation eine betonte Sachlichkeit und Nüchternheit an den
Tag, die der erstrebten Einhaltung von gemäßigten und selbstbeherrschten
Verhaltensregeln in seiner privaten Lebensführung entsprachen. Deshalb
nimmt es nicht wunder, dass auch in seinen Bibliographien ausgehandelt
wurde, wie der \enquote{Emanzipationskampf} auf angebrachte und
zielgerichtete Weise zu führen sei. Das WhK hielt in der Kaiserzeit
stets an den Grundsätzen fest, \enquote{sich bescheiden um Akzeptanz zu
bemühen}, auf Wissenschaft zu vertrauen und Homosexuelle als
\enquote{ordentliche und nützliche Bürger} darzustellen: Diese Strategie
des \enquote{umsichtige{[}n{]} Taktieren{[}s{]}}\footnote{Keilson-Lauritz
  2005, S. 84 und 93.} verkörperte Wilhelm/Praetorius geradezu
musterhaft.

Durch seine Bibliographien profilierte er sich als ein nüchterner und
besonnener Kämpfer, der sich vor jeglicher \enquote{Übertreibung}
hütete. Unsicher ist deshalb, ob sich Wilhelm überhaupt als
\enquote{Aktivist} wahrnahm. Als sich der Arzt und Sexualforscher Albert
Moll (1862--1939), zunächst dem WhK wohlgesonnen, ab 1905 immer mehr zum
Gegner entwickelte und dem Komitee wiederholt eine fragwürdige
Verknüpfung von Forschung und \enquote{Agitation} vorwarf, erwiderte
Praetorius, dass sich das Komitee lediglich um die \enquote{Verbreitung
von wissenschaftlichen Ergebnissen in weitere Volkskreise} bemühe, da
die Wissenschaft schließlich keine \enquote{scholastische{[}\ldots{}{]}
Luxusgelehrsamkeit} sei.\footnote{Praetorius 1906, S. 783. Zu diesem
  Kritikpunkt Albert Molls vor dem Hintergrund seiner wissenschaftlichen
  Rivalität mit Magnus Hirschfeld: Sigusch 2008, S. 197-233, insb. S.
  209-214 und 218-227.} Dass das WhK dennoch keine
\enquote{Verherrlichung} der Homosexualität bezweckte, wurde in der
Bibliographie oft beteuert: Angestrebt seien nur \enquote{Duldung} für
die Homosexuellen und \enquote{Aufhebung des Strafgesetzes}, fehl am
Platz hingegen grundlose \enquote{Glorifizierung},
\enquote{Überschätzung und höhere{[}{]} Schätzung ihres Liebesgefühls}
sowie \enquote{übertriebene{[}{]} Forderungen und exaltierte{[}{]}
Anschauungen}.\footnote{Praetorius 1901, S. 480; 1902a, S. 774.}

Zweifellos sind diese Beteuerungen auch als taktisches Zugeständnis zu
verstehen. Durch seine Distanzierung von \enquote{überspannte{[}n{]} und
maßlose{[}n{]}} beziehungsweise \enquote{übertriebene{[}n{]} Forderungen
gewisser Homosexueller}\footnote{Praetorius 1904, S. 542 und 541.} bezog
er Stellung zu den zunehmenden Flügelkämpfen innerhalb der organisierten
Homosexuellenbewegung und ließ damit die bescheideneren Ziele des WhK in
einem günstigen Licht erscheinen. Dies geht deutlich aus seiner
Bekämpfung maskulinistischer Positionen hervor, die er nicht nur als
sexualtheoretisch falsch (das heißt mit der Zwischenstufenlehre
inkompatibel), sondern genauso sehr als strategisch kontraproduktiv
zurückwies. Insbesondere nach der Gründung der eher maskulinistisch
angehauchten \emph{Gemeinschaft der Eigenen}\footnote{Zur Gemeinschaft
  der Eigenen: Keilson-Lauritz / Lang (Hrsg.) 2000.} 1903 machte sich in
den Besprechungen zu den Schriften der von Praetorius so genannten
\enquote{neuesten Richtung} ein viel schärferer Ton bemerkbar: Diese
gefährdete nämlich durch ihr Gebaren das bisher mühsam Erreichte, obwohl
oder gerade weil sie sich als Teil der Emanzipationsbewegung verstand.
Nicht hinnehmbar war für Praetorius die strategisch gefährliche
Vermengung von Freundschaft und Sexualität, insbesondere der Begriff der
\enquote{physiologischen Freundschaft} (Benedict Friedlaender), durch
den die Vorstellung eines angeborenen unabänderlichen Geschlechtstriebes
in Frage gestellt wurde. Das Festhalten an einem \enquote{fundamentalen
Unterschied zwischen Freundschaft und homosexueller Liebe} war
Praetorius (wieder in Einklang mit Hirschfeld) insofern äußerst wichtig,
als sonst der These der Verführung zur Homosexualität wieder Vorschub
geleistet würde.\footnote{Praetorius 1905, S. 793 und 1908, S. 492.
  Hierzu: Keilson-Lauritz 1997, S. 329-333.}

Über strategische und theoretische Divergenzen hinaus unterschieden sich
die Maskulinisten von Praetorius letzten Endes im Hinblick auf den
Umfang der anzustrebenden gesellschaftlichen Reformen, die mit der
Homosexualitätsfrage zusammenhingen. Während die ersteren eine
grundlegende Umgestaltung im Sinne der Wiederbelebung einer
\enquote{männlichen Kultur} befürworteten, maß Praetorius dem Problem
eine viel begrenztere Bedeutung bei. Ihm zufolge zog die Lösung dieser
Frage keine grundsätzliche Infragestellung gesellschaftlicher Ordnungen
oder kultureller Normen nach sich: Aufgrund der kleinen Zahl der
Homosexuellen entfaltete sie eine nur sehr beschränkte Wirkung. Nicht
auf einen tiefgreifenden Umbruch, sondern auf die Integration einer
kleinen Minderheit in die bestehende Mehrheitsgesellschaft sei
hinzuarbeiten.\footnote{Praetorius 1908, S. 480.} Dementsprechend
sollten bescheidene und realistische Kampfziele gesetzt werden.
Praetorius betonte, dass das WhK nicht weniger, aber auch nicht mehr als
die \enquote{Gleichwertung} der Homo- mit der Heterosexualität anstrebte
-- die Aufhebung des § 175 sei nur ein Schritt in diese Richtung. Unter
\enquote{Gleichwertung} seien nicht etwa \enquote{Ehen zwischen
Homosexuellen oder überhaupt {[}\ldots{}{]} öffentlich anerkannte
Liebesverhältnisse}, \enquote{offene Werbung Homosexueller um erkorene
Lieblinge} oder \enquote{Umwälzung der Kultur auf Grund einer
Anerkennung der homosexuellen Liebe} zu verstehen, sondern vielmehr die
aufklärende Verbreitung der \enquote{richtigen Erkenntnis des Wesens der
Homosexualität} und die Bekämpfung der gesellschaftlichen
Diskriminierung.\footnote{Praetorius 1904, S. 479f.}

\section*{\texorpdfstring{3.) Der normative Entwurf eines
\enquote{normalen}
Homosexuellen}{3.) Der normative Entwurf eines normalen Homosexuellen}}\label{der-normative-entwurf-eines-normalen-homosexuellen}

Gekämpft wurde, so Praetorius, für die
\enquote{Durchschnittshomosexuellen}, das heißt \enquote{Männer aus
allen Stellungen, sehr bedeutende, weniger bedeutende und eine große
Durchschnittsmasse}.\footnote{Praetorius 1905, S. 694; 1908, S. 484.}
Der Entwurf eines \enquote{durchschnittlichen},
\enquote{normalen}\footnote{\enquote{Normal} und
  \enquote{Normalisierung} sind hier so zu verstehen, dass die
  Homosexuellenbewegung (in diesem Fall durch die Bibliographie
  Wilhelms/Praetorius') daran arbeitete, \enquote{die Normalitätsgrenzen
  in ihrem Sinne zu verschieben} (Bruns 2008, S. 118).} Homosexuellen
bildete demnach eine der Hauptleistungen der Bibliographien. Normativ
war dieser Entwurf insofern, als bestimmte Verhaltens- und Kampfweisen
als die passendsten empfohlen wurden. Da die mitunter gestreifte
weibliche Homosexualität insgesamt doch nur ein Randthema im
wissenschaftlichen Teil der Bibliographien bildete, sei an dieser Stelle
betont, dass die identitätsstiftende Arbeit in erster Linie auf
homosexuelle Männer gerichtet wurde und somit einen Versuch darstellte,
die Möglichkeit einer homosexuellen Männlichkeit auszuloten. In den
Blick werden vier zentrale Aushandlungspunkte genommen, über die der
\enquote{normale} männliche Homosexuelle in den Bibliographien entworfen
und legitimiert wurde: Seine Einordnung in eine Männlichkeits- und
Weiblichkeitsskala, sein Stellenwert gegenüber den Heterosexuellen,
seine Einhaltung bürgerlicher Ideale und die Rolle sexueller Handlungen.

\subsection*{\texorpdfstring{Ein unbestreitbarer \enquote{weiblicher
Einschlag}}{Ein unbestreitbarer weiblicher Einschlag}}\label{ein-unbestreitbarer-weiblicher-einschlag}

Die Hirschfeldsche Zwischenstufenlehre war zugleich Ausgangspunkt der
Argumentation des Komitees und dauernder Streitpunkt innerhalb der
organisierten Bewegung, deren Umgang mit Männlichkeitsentwürfen stets
eine wichtige strategische Bedeutung hatte. Auch innerhalb des WhK war
diese Strategie nicht unumstritten.\footnote{Keilson-Lauritz 2005; Herrn
  2005, S. 38-42; Bruns 2008, S. 138-162.} Wilhelm hingegen übernahm
Hirschfelds Zwischenstufenlehre vorbehaltlos und verteidigte sie als
Numa Praetorius gegen alle Infragestellungen. Dies bedeutet, dass er
einerseits an der Vorstellung polarer Geschlechtercharaktere festhielt
und geschlechtsspezifische Zuschreibungen vornahm, andererseits aber
unter Berufung auf die unzähligen dazwischen liegenden \enquote{Stufen}
diese binäre Opposition potentiell überwinden konnte.\footnote{Herrn
  2008, S. 181f.}

Durch die konsequente Anwendung der Zwischenstufenlehre, vor allem durch
die durchgängige Thematisierung der Mischgeschlechtlichkeit des
homosexuellen Mannes, konnte etwa gegen maskulinistische Ansprüche, eine
Sonder- beziehungsweise Höherstellung für \enquote{virile} Homosexuelle
geltend zu machen, wirksam vorgegangen werden. Praetorius wies den auf
den Mann gerichteten Sexualtrieb eines Mannes als weibliche Eigenschaft
\emph{per se} aus und duldete dabei keine Ausnahme: Mochte ein
\enquote{Päderast} -- damit war ein \enquote{männlich geartete{[}r{]}
Homosexuelle{[}r{]}} gemeint -- ein noch so männliches Gebaren an den
Tag legen, so hatte er trotzdem nicht \enquote{den Charakter vollster
Männlichkeit bewahrt.}\footnote{Praetorius 1904, S. 526; 1906, S. 733.}
Die bewegungspolitische Schlussfolgerung lag auf der Hand: Da
\enquote{jeder Homosexuelle, selbst der scheinbar virilste, sich doch
durch eine Anzahl charakteristischer Merkmale vom Heterosexuellen
unterscheidet oder wie ich richtiger sagen möchte, vom
Durchschnittsheterosexuellen}, bestand kein grundsätzlicher Gegensatz
\enquote{zwischen virilen und femininen Homosexuellen}.\footnote{Praetorius
  1908, S. 476.} Solche Binnenhierarchien wurden für bewegungspolitisch
irrelevant erklärt. Damit widersetzte sich Praetorius der in den
Homosexuellenbewegungen der Kaiserzeit und Weimarer Republik vielfach
belegten Tendenz, \enquote{Effeminierte} für die gesellschaftliche
Diskriminierung verantwortlich zu machen.\footnote{Micheler 2005, S.
  181-186; Herrn 2008, S. 179; Lücke 2008, S. 106f.; Eder 2014, S. 27.}
Daraus ergab sich außerdem, dass entgegen jeglicher idealisierenden oder
beschönigenden Darstellung nicht davor gescheut werden sollte,
\enquote{das Vorkommen von weibischem Wesen bei Homosexuellen in
Sprache, Bewegungen, Neigungen usw.} anzusprechen.\footnote{Praetorius
  1905, S. 752.}

Mit der Vorrangstellung der Kategorie \enquote{Homosexualität} verloren
andere Kriterien an Bedeutung: Praetorius betonte die Zugehörigkeit der
\enquote{Effeminierten} zum Kollektiv der Homosexuellen ausdrücklich,
gleichwohl er sie tendenziell weiter stigmatisierte. Hirschfeld zufolge
waren Homosexuelle allein durch ihren \enquote{weiblichen Einschlag}
\enquote{nicht minderwertig} und \enquote{den Heterosexuellen zwar nicht
gleichartig, aber doch gleichwertig}.\footnote{Das Zitat ist dem
  Gutachten Hirschfelds im ersten Moltke-Harden-Prozess entnommen
  (zitiert nach: Domeier 2010, S. 169).} In Einklang mit ihm beteuerte
Praetorius ebenfalls, dass dies \enquote{gar keine Minderwertigkeit}
beziehungsweise \enquote{keine Herabsetzung} bedeutete.\footnote{Praetorius
  1908, S. 439 und 500.} Gleichzeitig machte er jedoch deutlich, dass
ein Übermaß an weiblichen Eigenschaften nicht gerade wünschenswert sei:
Gepriesen wurde weder Weiblichkeit noch Verweiblichung an sich, sondern
vielmehr das produktive \enquote{Gemisch männlicher und weiblicher
Charaktere} beziehungsweise die \enquote{Mischung der Vorzüge beider
Geschlechter in ausgeprägtem, eigenartigem Maße}\footnote{Praetorius
  1903, S. 1092; 1905, S. 709.} -- unter der stillschweigenden
Voraussetzung, dass die wichtigeren Eigenschaften (in erster Linie die
Intelligenz) doch \enquote{männlich} blieben. Dies verrät das Festhalten
Praetorius' an einem \enquote{geschlechterhierarchischen Denken}, das
tendenziell doch zur \enquote{Abwertung der Weiblichkeit Homosexueller}
führte.\footnote{Herrn 2005, S. 38.}

\subsection*{Gleichwertigkeit mit der
Heterosexualität}\label{gleichwertigkeit-mit-der-heterosexualituxe4t}

Ein zentrales Argument in den Bibliographien war die Betonung der
Parallelität gleich- und gegengeschlechtlicher Sexualtriebe, die zwar
bei anderen WhK-Mitstreitern ebenfalls zu finden war, bei Praetorius
dennoch eine besonders prominente Stellung einnahm. Das wiederkehrende
Heranziehen der Heterosexualität als Vergleichsfolie erfüllte mehrere
Zwecke. Es war Bestandteil seiner Strategie, jegliche
\enquote{Übertreibung} im Sinne einer Höherstellung zu verhindern,
zugleich aber eine strikte Gleichbehandlung einzufordern:
\enquote{{[}Z{]}wischen den zwei Extremen: Krankhaftigkeit und höhere
Wertung, gibt es einen Mittelweg, nämlich Gleichwertung von Homo- und
Heterosexualität.}\footnote{Praetorius 1905, S. 851.} Betrachtete man
den homosexuellen Trieb als \enquote{Aequivalent des normalen Triebes},
so war die \enquote{Duldung des homosexuellen Auslebens in den auch den
Heterosexuellen gezogenen Schranken}\footnote{Praetorius 1901, S. 386;
  1906, S. 838.} nur folgerichtig.

Die Hervorhebung einer die Gleichbehandlung rechtfertigenden
Parallelität diente zudem dem Zweck, durch wirklichkeitsgetreue
Darstellung und nüchternen Ton den \enquote{durchschnittlichen}
Homosexuellen zu normalisieren. Demnach war jegliche Verklärung
überflüssig. Es genügte schon, Heterosexuelle auf die Ähnlichkeit der
jeweiligen Verhältnisse aufmerksam zu machen. Das Heranziehen der
Heterosexualität als Vergleichsfolie sollte vertraute Bilder beim
männlich-heterosexuell vermuteten Leser hervorrufen, an dessen
Einfühlungsvermögen appelliert wurde: \enquote{Es braucht sich nur jeder
Heterosexuelle zu fragen, ob Strafandrohungen ihn von der Bethätigung
seines Geschlechtstriebes abhalten würden.}\footnote{Praetorius 1902b,
  S. 868.} Durch das Verständnis schaffende Sich-Hineinversetzen wurden
die Probleme, mit denen der durchschnittliche Homosexuelle konfrontiert
war, für eine männlich-heterosexuelle Vorstellungswelt begreiflich
gemacht. Auf diese Weise widerlegte Praetorius gängige Einwände, führte
etwa die Absurdität der Erwerbungstheorie vor Augen (\enquote{auch der
heterosexuelle Trieb macht sich ja regelmäßig erst nach der
Pubertätszeit geltend, man kann deshalb doch auch nicht von ihm sagen,
er sei erworben}).\footnote{Praetorius 1905, S. 689.}

Die Betonung der Parallelität und Gleichwertigkeit mit dem
heterosexuellen Trieb eignete sich außerdem dafür, die Vorstellung eines
angeborenen, natürlichen, gesunden und unveränderlichen Sexualtriebes zu
plausibilisieren, die eines notwendigen Zusammenhangs von Homosexualität
und Krankheit hingegen entschieden zurückzuweisen. Praetorius wies auf
die Unzulänglichkeit eines rein ärztlichen beziehungsweise klinischen
Blickes hin, dem sich die Homosexualität als soziale Erscheinung ohnehin
weitgehend entzog: \enquote{Die Sache ist ähnlich, als wenn man alle
Heterosexuellen als \enquote{Kranke} bezeichnen wollte, weil fast alle
Heterosexuellen, welche den Nervenarzt aufsuchen, nervenkrank
sind.}\footnote{Praetorius 1905, S. 769.} Eindeutiger als Hirschfeld hob
er auf die Nutzlosigkeit und ethische Fragwürdigkeit von
Heilungsversuchen ab. Als Anhänger der Anlagetheorie sah er kaum
Aussicht auf \enquote{Erfolg} und konnte Heilungsversprechen, sei es
durch Hypnose, Psychoanalyse oder Kastration, grundsätzlich nicht
gutheißen.

\subsection*{Die Pflege männlich-bürgerlicher
Tugenden}\label{die-pflege-muxe4nnlich-buxfcrgerlicher-tugenden}

Einen weiteren Bestandteil der Strategie, den männlichen Homosexuellen
als \enquote{normal} zu entwerfen, bildete die Herausstellung des
kulturellen Wertes und sozialen Nutzens der Homosexualität. Praetorius
hob auf die für den Staat nutzbringende Gleichberechtigung aller
gewöhnlichen Bürger ab. Beinahe bilanzmäßig ergebe sich aus der
Verfolgung ein Nachteil, aus der Akzeptanz ein Gewinn für Staat und
Gesellschaft: \enquote{{[}A{]}lles in allem genommen, {[}wird{]} das
Defizit der körperlichen Fortpflanzung durch ein Plus von geistiger
Zeugung bei der Homosexualität ersetzt}.\footnote{Praetorius 1908, S.
  479.} Dem durchschnittlichen Homosexuellen, so Praetorius, sollte es
daran liegen, seinen Beitrag dazu zu leisten. Angestrebt sei keine
Infragestellung der sozialen Ordnung, sondern deren Verfestigung durch
Einbeziehung bisher ungerecht ausgestoßener Mitglieder: Es galt,
\enquote{dem Staate an Stelle eines durch Mutlosigkeit, Zaghaftigkeit,
Niedergeschlagenheit entkräfteten und oft zum Selbstmord getriebenen
Mitgliedes einen in Harmonie mit sich und der Welt lebenden, tüchtigen,
arbeitsfrohen Staatsbürger {[}zu{]} geben}.\footnote{Praetorius 1905, S.
  849f. Bereits angesprochen in: Praetorius 1902a, S. 755.}

Dass der so entworfene durchschnittliche Homosexuelle deutlich
bürgerliche Züge trug, war sicherlich eine Reaktion auf die
zeitgenössisch gängige Konstruktion der Homosexuellen als
Ausnahmemenschen im guten (\enquote{Geistesgrößen}) wie im schlechten
Sinne (\enquote{Verbrecher}, \enquote{Kranke}): Bürgerliche Tugenden und
Lebensweisen boten ein naheliegendes Identifikationsmodell zur
Normalisierung des homosexuellen Mannes. Die Adressaten dieser
normativen Konstruktion waren daher nicht nur Fachwelt und Gesetzgeber,
sondern auch homosexuelle Männer selbst: War eine anständige und sozial
nützliche Lebensführung imstande, Akzeptanz zu schaffen, so wurden sie
aufgefordert, daran aktiv mitzuwirken und \enquote{nicht nur an ihr
Recht auf sinnliche Befriedigung zu denken, sondern auch an ihre Pflicht
einer ethischen Ausgestaltung ihrer Liebesrichtung.}\footnote{Praetorius
  1900, S. 387. Die Hervorhebung der \enquote{ethischen Aufgaben der
  Homosexuellen} war keineswegs nur bei Praetorius zu finden: vgl. den
  gleichnamigen Aufsatz Kurt Hillers im \emph{Jahrbuch} (Hiller 1913).
  Die Notwendigkeit einer anständigen, arbeitsamen Lebensführung wurde
  in den Freundschaftszeitschriften der Weimarer Zeit ebenfalls betont
  (Micheler 2008, S. 211-215).} Entworfen wurde ein Verhaltenskodex, der
Empfehlungen sowohl für das Alltagsleben als auch für das
emanzipationspolitische Auftreten enthielt und sich hauptsächlich mit
der Frage der Sichtbarkeit auseinandersetzte.

Praetorius zufolge zwang die Notwendigkeit der \enquote{Agitation} die
Homosexuellen, sichtbarer und lauter zu werden, als es ihnen genehm
wäre: Paradoxerweise veranlasste sie ehrbare und sonst unauffällige
Bürger, an die Öffentlichkeit zu gehen, um ihr Recht auf Ehrbarkeit und
Unauffälligkeit einzuklagen. Doch die \enquote{Agitation} hatte
Entwicklungen in Gang gebracht, die nicht mehr rückgängig gemacht werden
konnten. Schrieb Praetorius den Strafbestimmungen eine
identitätsstiftende Wirkung zu, indem sie \enquote{das Gefühl der
Zusammengehörigkeit und Solidarität bei den Konträren geweckt und sie zu
engerem Zusammenschluss bewogen}\footnote{Praetorius 1902a, S. 773.}
hätten, so erkannte er dem \enquote{Emanzipationskampf} an sich
ebenfalls eine charakterbildende Funktion im Sinne einer Vermännlichung
zu. Zum einen begriff er die wissenschaftliche und
emanzipationspolitische Befassung mit Homosexualität als Teil einer
männlichkeitsstärkenden Lebensführung, die auf Selbstkontrolle und
Zügelung der Triebe abzielte: \enquote{Das Bestehen der Strafandrohung
hat ihnen eine Aufgabe gestellt und ideale Ziele geschaffen und dadurch
auch eine Verinnerlichung des Triebes und Ablenken von der sinnlicheren
Seite bewirkt.}\footnote{Praetorius 1902a, S. 773.} Zum anderen kam
seine Definition eines würdigen öffentlichen Auftretens einer etwas
martialischen Aufforderung zur Gefühlskontrolle gleich: Das bei
Homosexuellen allzu oft vorkommende \enquote{fortwährende Jammern und
Klagen} sowie das \enquote{behäbige{[}{]} Breittreten ihrer Leiden und
Schmerzen} sollten männlicheren Verhaltensweisen weichen -- von
\enquote{Handeln}, \enquote{Kämpfen} und \enquote{Streiten} war die
Rede:

\begin{quote}
Handeln soll der Uranier, nicht jammernd verzagend die Hände in den
Schoß legen.
\end{quote}

Kämpfen soll er dafür, daß die Verachtung und der Spott, die ihm die
Heterosexuellen entgegenbringen, verstummen, streiten soll er dafür, und
als sein Recht verlangen, daß ein die Betätigung seines angeborenen
Triebes bestrafendes schimpfliches Gesetz beseitigt werde.\footnote{Praetorius
  1908, S. 440. Ähnlich kämpferische Töne bei Kurt Hiller: Hiller 1913,
  S. 400-403.}

So entwarf Praetorius die Mitwirkung an der Homosexuellenbewegung als
eine männliche Ermächtigungsstrategie, die den bei jedem homosexuellen
Mann vorhandenen \enquote{weiblichen Einschlag} nicht beseitigen,
sondern ergänzen beziehungsweise ausgleichen sollte.

\subsection*{Der Stellenwert des
Geschlechtsverkehrs}\label{der-stellenwert-des-geschlechtsverkehrs}

Die erste Homosexuellenbewegung neigte dazu, die rein geistige, ideale
Dimension gleichgeschlechtlicher Beziehungen zu (über)betonen. Es
handelte sich zweifellos um \enquote{eine der Schutzbehauptungen, die
damals üblich waren und zumindest anfangs auch von Hirschfeld verwendet
wurden}, dennoch um eine inkonsequente, ja widersprüchliche
Schutzbehauptung, denn der § 175 stellte keine Veranlagung, sondern
bestimmte sexuelle Handlungen unter Strafe.\footnote{Herzer
  \textsuperscript{2}2001, S. 96. In diesem Punkt aber änderte
  Hirschfeld später seine Strategie (ebd. S. 139).} Von vornherein hegte
Praetorius Bedenken gegen eine solche Strategie, hielt sie zum einen für
wirklichkeitsfremd, zum anderen -- vom Grundsatz der strikten
Parallelität zum heterosexuellen Trieb ausgehend -- für nutzlos. Der von
ihm entworfene durchschnittliche Homosexuelle hatte demnach durchaus ein
Geschlechtsleben, dem die gleichen Schranken gesetzt werden sollten wie
dem heterosexuellen:

\begin{quote}
Man braucht die Homosexuellen, wenn man sie verteidigen will, nicht als
Engel zu malen, sie verlieren nichts an Achtung, sie werden nicht zu
Lüstlingen gestempelt, wenn man der Wirklichkeit entsprechend zugesteht,
daß die meisten -- ebenso wie die Heterosexuellen -- des sinnlichen
Verkehrs bedürfen und ihn ausüben.\footnote{Praetorius 1904, S. 516.}
\end{quote}

Daraus ergab sich die Abwegigkeit sowohl der Forderungen nach
Enthaltsamkeit als auch der Charakterisierung des gleichgeschlechtlichen
Triebs als rein \enquote{platonischer} Neigung. Freilich war für
Praetorius das Einfordern eines Rechts auf straffreie sexuelle
Betätigung kein Selbstzweck, sondern stets an Ideale der Mäßigung und
Selbstbeherrschung sowie an das Streben nach sozialem Nutzen gebunden.
Das maßvolle sexuelle Ausleben des \enquote{Normalmensch{[}en{]} -- ob
homo- oder heterosexuell --} sei eine der Grundlagen funktionierender
Gesellschaften.\footnote{Praetorius 1905, S. 790.} Die Legitimierung
eines gemäßigten Sexuallebens auch für Homosexuelle ging mit zwei
ergänzenden, emanzipationspolitisch brisanten Bedingungen
beziehungsweise Einschränkungen einher.

Einerseits lehnte Praetorius die Hierarchisierung gleichgeschlechtlicher
Sexualhandlungen, also die Einteilung in legitime und illegitime
Sexualhandlungen ab. Unzulässig war für ihn die moralisch und rechtlich
unterschiedlich fallende Beurteilung der Homosexuellen je nach dem,
welche sexuellen Handlungen sie vornahmen oder bevorzugten. Mochte der
Oral- und Analverkehr von den \enquote{meisten} abgelehnt und in
\enquote{ästhetischer} Hinsicht wenig anziehend sein, so ließ sich
allerdings aus der relativen Seltenheit einer Handlung keine größere
Verwerflichkeit herleiten: \enquote{Ethisch halte ich diese
verschiedensten Formen für ziemlich gleichwertig, ebenso hinsichtlich
ihrer strafrechtlichen Beurteilung.}\footnote{Praetorius 1901, S. 433
  und 1902b, S. 856.} Die Ähnlichkeit mit der Argumentation der
Vorkämpfers Karl Heinrich Ulrichs (1825--1895), der seinerzeit ebenfalls
die Seltenheit dieser Praktiken betonte, gleichzeitig jedoch
\enquote{keine Scheu vor einer generellen Verteidigung}\footnote{Kennedy
  \textsuperscript{2}2001, S. 202f., Zitat S. 202; Herzer
  \textsuperscript{2}2001, S. 103, Fußnote 19.} zeigte, ist
bemerkenswert. Praetorius zufolge waren alle Befriedigungsarten
gleichwertig beziehungsweise gleich moralisch irrelevant.

Andererseits grenzte er die \enquote{gesunden}\enquote{normalen}
Homosexuellen als Kollektiv von diskursiv benachbarten Gruppen ab, denen
er verstärkt eine pathologische Sexualität zuschrieb: Der Integration
nach innen (keine Diskriminierung aufgrund von bestimmten
Sexualpraktiken) standen klare Grenzziehungen nach außen gegenüber. In
der Frage, wer Teil der Gruppe war und verteidigt werden sollte, spielte
die Distanzierung von der \enquote{Pädophilie} eine zentrale Rolle.
Praetorius verwendete dabei den 1896 von Krafft-Ebing erstmals in diesem
Sinne theoretisierten Begriff der \emph{Paedophilia erotica}, unter dem
der Psychiater eine Perversion, das heißt eine dauerhafte, krankhafte,
auf das Kind ausgerichtete sexuelle Veranlagung verstand. Aufgrund der
Ähnlichkeiten, die beide Kategorien in ihrer psychiatrischen
Konstruktion aufwiesen, und um emanzipationspolitisch kontraproduktiven
Verwirrungen und Missverständnissen in der Öffentlichkeit vorzubeugen,
erschien Praetorius eine grundsätzliche terminologische Klärungsarbeit
dringend nötig. Demnach wurde \enquote{echte} Homosexualität durch die
betonte Abgrenzung von einer auf den ersten Blick verwandten, aber mit
ihr nicht zu verwechselnden Erscheinung definiert: Die wesentliche
Unterscheidung zwischen \enquote{Liebe zu unreifen Knaben} und
\enquote{Liebe zu Jünglingen} sei \enquote{zur Vermeidung falscher
Vorstellungen des Volks über Homosexualität {[}\ldots{}{]} scharf
aufrecht zu halten}.\footnote{Praetorius 1908, S. 536.} Hierin stimmte
Praetorius mit der überwiegenden Mehrheit im WhK überein, welche stets
darauf bedacht war, beide Fragen grundsätzlich auseinanderzuhalten und
an der Strafbarkeit sexueller Handlungen mit Kindern
festzuhalten.\footnote{Zur diesbezüglichen Diskussion in der ersten
  deutschen Homosexuellenbewegung: Herzer 1995; Keilson-Lauritz 1997, S.
  142f.; Hekma 2014, S. 118-120 und 135f.} Auch bezüglich der bereits in
der Petition befürworteten Festlegung eines Schutzalters bis zum 16.
Lebensjahr, das für beide Geschlechter gelten sollte, war Praetorius mit
Hirschfeld einer Meinung.\footnote{Praetorius 1906, S. 846; 1908, S.
  447. Vgl. die Petition des WhK in: Hirschfeld 1914, S. 977.} Der
\enquote{normale}, \enquote{durchschnittliche} Homosexuelle war demnach
nicht nur gemäßigt und auf Selbstbeherrschung bedacht, sondern richtete
seinen Sexualtrieb ausschließlich auf Erwachsene.

\section*{4.) Fazit}\label{fazit}

Die ab 1900 von \enquote{Numa Praetorius} betreute
\enquote{Bibliographie der Homosexualität} im \emph{Jahrbuch für
sexuelle Zwischenstufen}, der Zeitschrift des
Wissenschaftlich-humanitären Komitees, verfolgte auf verschiedenen
Ebenen ambitionierte Ziele. Sie bemühte sich um die Etablierung der
\enquote{Homosexualitätsforschung} als eigenständiges, eine
wissenschaftliche Streitkultur verdienendes Forschungsfeld, um die
Verteidigung der Hirschfeldschen Zwischenstufenlehre und um die
Profilierung des Komitees als sachkundige Instanz auf diesem Gebiet. Die
Bibliographie stellte außerdem nicht nur eine Legitimierungsarbeit nach
außen, sondern auch ein Machtmittel nach innen dar: Mit ihr verfügte
Wilhelm/Praetorius über ein Sprachrohr, das der eigenen Positionierung
diente und ihm eine Einflussnahme auf Ausrichtung und Prioritätensetzung
der Bewegung ermöglichte. Durch seine Bibliographie entwarf er
schließlich den in seinen Augen emanzipationswirksamen Typus eines
\enquote{normalen} homosexuellen Mannes. Dabei befasste er sich durchaus
explizit mit Themen wie Analverkehr, Effemination oder
\enquote{Knabenliebe}, die anzusprechen sich die Homosexuellenbewegung
aufgrund ihres vermeintlich abträglichen Charakters tendenziell
scheute.\footnote{Hierzu: Hekma 2014.} Kennzeichnend für Praetorius war
jedoch das Bestreben, dem \enquote{normalen} Homosexuellen einen
bürgerlichen Anstrich zu geben. Konstruiert wurde eine Normalität im
Sinne einer unspektakulären Durchschnittlichkeit. Der
\enquote{gewöhnliche Homosexuelle}\footnote{Dannecker / Reiche 1974.}
bewältigte seinen \enquote{weiblichen Einschlag} und führte ein
arbeitsames, nach Möglichkeit anständiges und (auch in sexueller
Hinsicht) gemäßigtes Leben, war kurzum ein respektabler, nützlicher
Bürger, der unter Berufung auf eine strikte Parallelität zur
Heterosexualität einen Anspruch auf Gleichberechtigung erhob. Hier
führte er Argumente (maßvolle und selbstbeherrschte Lebensweise, Wunsch
nach Respektabilität und Unauffälligkeit) ins Feld, die in manchen
Autobiographien der \emph{Psychopathia sexualis} bereits zum Ausdruck
gekommen waren und von späteren homosexuellen Emanzipationsbewegungen
wie den Freundschaftsverbänden der Weimarer Zeit und der
Homophilenbewegungen der Nachkriegszeit wieder aufgegriffen werden
sollten.\footnote{Müller 1991, S. 242f.; Micheler 2005, S. 169-175;
  Delessert 2013, S. 67-73; Jackson 2009, S. 130-157; Pretzel / Weiß
  (Hrsg.) 2010.}

\section*{Literaturverzeichnis}\label{literaturverzeichnis}

Benkel, Thorsten (2014): Stigma, Sex und Subkultur. Zur soziologischen
Beobachtung von Homosexualität. In: Evans, Jennifer / Lautmann, Rüdiger
/ Mildenberger, Florian / Pastötter, Jakob (Hrsg.): \emph{Was ist
Homosexualität? Forschungsgeschichte, gesellschaftliche Entwicklungen
und Perspektiven}. Hamburg: Männerschwarm, S. 391-426.

Bruns, Claudia (2008): \emph{Politik des Eros. Der Männerbund in
Wissenschaft, Politik und Jugendkultur (1880-1934)}. Köln {[}u.a.{]}:
Böhlau.

Cardon, Patrick (1994): Les relations homosexuelles en Algérie en 1900.
Suivi de Poésies homosexuelles arabes par le Dr.~Numa Praetorius. In:
Mendès-Leite, Rommel (Hrsg.): \emph{Sodomites, invertis, homosexuels.
Perspectives historiques}. Lille: GKC, S. 99-119.

Cardon, Patrick (2008): \emph{Discours littéraires et scientifiques
fin-de-siècle. Autour de Marc-André Raffalovich}. Paris: L'Harmattan
Orizons.

Dannecker, Martin / Reiche, Reimut (1974): \emph{Der gewöhnliche
Homosexuelle. Eine soziologische Untersuchung über männliche
Homosexuelle in der Bundesrepublik}. Frankfurt am Main: Fischer.

Delessert, Thierry (2013): Straflosigkeit in Grenzen. Zur politischen
und rechtlichen Geschichte männlicher Homosexualität in der Schweiz in
der ersten Hälfte des 20. Jahrhunderts. In: \emph{Invertito. Jahrbuch
für die Geschichte der Homosexualitäten}, 15, S. 45-74.

Dobler, Jens (Hrsg.) (2004): \emph{Prolegomena zu Magnus Hirschfelds
Jahrbuch für sexuelle Zwischenstufen (1899 bis 1923). Register,
Editionsgeschichte, Inhaltsbeschreibungen}. Hamburg: von Bockel.

Domeier, Norman (2010): \emph{Der Eulenburg-Skandal. Eine politische
Kulturgeschichte des Kaiserreichs}. Frankfurt am Main {[}u.a{]}: Campus.

Dose, Ralf (2005): \emph{Magnus Hirschfeld. Deutscher -- Jude --
Weltbürger}. Teetz: Hentrich \& Hentrich.

Eder, Franz (2014): Homo- und andere gleichgeschlechtliche Sexualitäten
in Geschichte und Gegenwart. In: Evans, Jennifer / Lautmann, Rüdiger /
Mildenberger, Florian / Pastötter, Jakob (Hrsg.): \emph{Was ist
Homosexualität? Forschungsgeschichte, gesellschaftliche Entwicklungen
und Perspektiven}. Hamburg: Männerschwarm, S. 17-39.

Hekma, Gert (2014): Sodomie -- Unmännlichkeit -- Knabenliebe. Male
Same-sexual Practices and Identifications in Occidental Societies. In:
Evans, Jennifer / Lautmann, Rüdiger / Mildenberger, Florian / Pastötter,
Jakob (Hrsg.) (2014): \emph{Was ist Homosexualität?
Forschungsgeschichte, gesellschaftliche Entwicklungen und Perspektiven}.
Hamburg: Männerschwarm, S. 113-140.

Herrn, Rainer (2005): \emph{Schnittmuster des Geschlechts.
Transvestitismus und Transsexualität in der frühen Sexualwissenschaft}.
Gießen: Psychosozial-Verlag.

Herrn, Rainer (2008): Magnus Hirschfelds Geschlechterkosmos: Die
Zwischenstufentheorie im Kontext hegemonialer Männlichkeit. In:
Brunotte, Ulrike / Herrn, Rainer (Hrsg.): \emph{Männlichkeiten und
Moderne. Geschlecht in den Wissenskulturen um 1900.} Bielefeld:
Transcript, S. 173-196.

Herrn, Rainer (2009): Magnus Hirschfeld (1868--1935). In: Sigusch,
Volkmar / Grau, Günter (Hrsg.): \emph{Personenlexikon der
Sexualforschung.} Frankfurt am Main: Campus, S. 284-294.

Herzer, Manfred (1993): Eugen Wilhelm (Numa Praetorius). In: Lautmann,
Rüdiger (Hrsg.): \emph{Homosexualität. Handbuch der Theorie- und
Forschungsgeschichte}. Frankfurt am Main, New York: Campus, S. 130-133.

Herzer, Manfred (1995): Stimmen aus dem Wissenschaftlich-humanitären
Komitee zum Sex mit Kindern. Nachträge zu den \enquote{Ungewöhnlichen
Liebesgeschichten}. In: \emph{Capri. Zeitschrift für schwule
Geschichte}, Nr. 19, S. 26-29.

Herzer, Manfred (1997): Das Wissenschaftlich-humanitäre Komitee. In:
Schwules Museum / Akademie der Künste Berlin (Hrsg.): \emph{Goodbye to
Berlin? 100 Jahre Schwulenbewegung}. Berlin: rosa Winkel, S. 37-48.

Herzer, Manfred (\textsuperscript{2}2001): Magnus Hirschfeld: Leben und
Werk eines jüdischen, schwulen und sozialistischen Sexologen. Hamburg:
MännerschwarmSkript.

Herzer, Manfred (2009): Eugen Wilhelm (Numa Praetorius) (1866--1951).
In: Sigusch, Volkmar / Grau, Günter (Hrsg.): \emph{Personenlexikon der
Sexualforschung.} Frankfurt am Main: Campus, S. 764-766.

Hiller, Kurt (1913): Ethische Aufgaben der Homosexuellen. In:
\emph{Jahrbuch für sexuelle Zwischenstufen}, 13, S. 399-410.

Hirschfeld, Magnus (1914): \emph{Die Homosexualität des Mannes und des
Weibes}, Berlin: Marcus.

Hirschfeld, Magnus (1986): \emph{Von einst bis jetzt. Geschichte einer
homosexuellen Bewegung. 1897-1922}. Hrsg. und mit einem Nachwort
versehen von Manfred Herzer und James Steakley. Berlin: rosa Winkel.

Jackson, Julian (2009): \emph{Arcadie: La vie homosexuelle en France, de
l'après-guerre à la dépénalisation}. Aus dem Englischen übersetzt von
Arlette Sancery. Paris: Autrement.

Keilson-Lauritz, Marita (1997): \emph{Die Geschichte der eigenen
Geschichte. Literatur und Literaturkritik in den Anfängen der
Schwulenbewegung am Beispiel des} Jahrbuchs für sexuelle Zwischenstufen
\emph{und der Zeitschrift} Der Eigene. Berlin: rosa Winkel.

Keilson-Lauritz, Marita (2005): Tanten, Kerle und Skandale. Flügelkämpfe
der Emanzipation. In: zur Nieden, Susanne (Hrsg.): \emph{Homosexualität
und Staatsräson. Männlichkeit, Homophobie und Politik in Deutschland
1900-1945}. Frankfurt am Main {[}u.a.{]}: Campus, S. 81-99.

Keilson-Lauritz, Marita / Lang, Rolf F. (Hrsg.) (2000):
\emph{Emanzipation hinter der Weltstadt. Adolf Brand und die
Gemeinschaft der Eigenen}. Berlin-Friedrichshagen: Müggel-Verlag Rolf F.
Lang.

Kennedy, Hubert (\textsuperscript{2}2001): \emph{Karl Heinrich Ulrichs.
Leben und Werk}. Hamburg: MännerschwarmSkript.

Krafft-Ebing, Richard von (1901): Neue Studien auf dem Gebiete der
Homosexualität. In: \emph{Jahrbuch für sexuelle Zwischenstufen}, 3, S.
1-36.

Lehmstedt, Mark (2002): \emph{Bücher für das \enquote{dritte
Geschlecht}. Der Max-Spohr-Verlag in Leipzig. Verlagsgeschichte und
Bibliographie (1881-1941)}. Wiesbaden: Harrassowitz.

Lücke, Martin (2008): Komplizen und Klienten. Die Männlichkeitsrhetorik
der Homosexuellen-Bewegung in der Weimarer Republik als hegemoniale
Herrschaftspraktik. In: Brunotte, Ulrike / Herrn, Rainer (Hrsg.):
\emph{Männlichkeiten und Moderne. Geschlecht in den Wissenskulturen um
1900.} Bielefeld: Transcript, S. 97-110.

Micheler, Stefan (2005): \emph{Selbstbilder und Fremdbilder der
\enquote{Anderen}. Männerbegehrende Männer in der Weimarer Republik und
der NS-Zeit}. Konstanz: UVK.

Micheler, Stefan (2008): \enquote{Männer} und \enquote{Tanten}.
Identitätsmodelle und Geschlechterkonzepte in den Zeitschriften Männer
begehrender Männer der Weimarer Republik. In: Tuider, Elisabeth (Hrsg.):
\emph{QuerVerbindungen. Interdisziplinäre Annäherungen an Geschlecht,
Sexualität, Ethnizität}. Berlin: LIT, S. 203-225.

Micheler, Stefan / Michelsen, Jakob (2001): Von der \enquote{schwulen
Ahnengalerie} zur Queer Theory. Geschichtsforschung und
Identitätsbildung, in: Heidel, Ulf / Micheler, Stefan / Tuider,
Elisabeth (Hrsg.): \emph{Jenseits der Geschlechtergrenzen. Sexualitäten,
Identitäten und Körper in Perspektiven von Queer Studies}. Hamburg:
MännerschwarmSkript, S. 127-143.

Müller, Klaus (1991): \emph{Aber in meinem Herzen sprach eine Stimme so
laut}. \emph{Homosexuelle Biographien und medizinische Pathographien im
neunzehnten Jahrhundert}. Berlin: rosa Winkel.

Praetorius, Numa {[}d.i. Wilhelm, Eugen{]} (1900): Die Bibliographie der
Homosexualität für das Jahr 1899, sowie Nachtrag zu der Bibliographie
des ersten Jahrbuchs. In: \emph{Jahrbuch für sexuelle Zwischenstufen},
2, S. 345-445.

Praetorius, Numa (1901): Die Bibliographie der Homosexualität für das
Jahr 1900, sowie Nachtrag zu der Bibliographie des ersten u. zweiten
Jahrbuches. In: \emph{Jahrbuch für sexuelle Zwischenstufen}, 3, S.
326-519.

Praetorius, Numa (1902a): Bibliographie I. Teil. Homosexualität und
Strafgesetz von Dr.~F. Wachenfeld. In: \emph{Jahrbuch für sexuelle
Zwischenstufen}, 4, S. 670-774.

Praetorius, Numa (1902b): II. Teil. Die Bibliographie der Homosexualität
für das Jahr 1901 mit Ausschluss der Belletristik. In: \emph{Jahrbuch
für sexuelle Zwischenstufen}, 4, S. 775-920.

Praetorius, Numa (1903): Bibliographie der Homosexualität. In:
\emph{Jahrbuch für sexuelle Zwischenstufen}, 5, S. 943-1155.

Praetorius, Numa (190b): Die Bibliographie der Homosexualität für das
Jahr 1903. In: \emph{Jahrbuch für sexuelle Zwischenstufen}, 6, S.
449-645.

Praetorius, Numa (1905): Die Bibliographie der Homosexualität für das
Jahr 1904. In: \emph{Jahrbuch für sexuelle Zwischenstufen}, 7, S.
671-948.

Praetorius, Numa (1906): Die Bibliographie der Homosexualität für das
Jahr 1905. In: \emph{Jahrbuch für sexuelle Zwischenstufen}, 8, S.
701-886.

Praetorius, Numa (1908): Die Bibliographie der Homosexualität. Nicht
belletristische Werke aus den Jahren 1906 und 1907. Belletristik aus den
Jahren 1905, 1906 u. 1907. In: \emph{Jahrbuch für sexuelle
Zwischenstufen}, 9, S. 425-620.

Praetorius, Numa (1909): Die Bibliographie der Homosexualität aus den
Jahren 1908 und 1909. In: \emph{Jahrbuch für sexuelle Zwischenstufen},
10 {[}=\emph{Vierteljahrsberichte des Wissenschaftlich-humanitären
Komitees}, 1 (1909/10){]}, S. 36-106, 194-230, 313-339, 431-439.

Praetorius, Numa (1910): Die Bibliographie der Homosexualität aus dem
Jahre 1908 und 1909. In: \emph{Jahrbuch für sexuelle Zwischenstufen}, 11
{[}=\emph{Vierteljahrsberichte des Wissenschaftlich-humanitären
Komitees}, 2 (1910/11){]}, S. 67-111, 201-226, 319-341, 409-442.

Pretzel, Andreas / Weiß, Volker (Hrsg.) (2010): \emph{Ohnmacht und
Aufbegehren. Homosexuelle Männer in der frühen Bundesrepublik}. Hamburg:
Männerschwarm.

Sigusch, Volkmar (2008): \emph{Geschichte der Sexualwissenschaft}.
Frankfurt am Main: Campus.

%autor
\begin{center}\rule{0.5\linewidth}{\linethickness}\end{center}

\textbf{Kevin Dubout} ist Doktorand im Fach Geschichte an der Humboldt
Universität zu Berlin und schreibt seine Dissertation zum Leben und Werk
Eugen Wilhelms / Numa Praetorius'. Kontaktadresse: kevin.dubout@gmx.de

\end{document}