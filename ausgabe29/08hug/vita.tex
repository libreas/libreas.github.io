\begin{center}\rule{0.5\linewidth}{\linethickness}\end{center}

\textbf{Marius Hug} hat von 2000--2006 an der Humboldt-Universität zu
Berlin Kulturwissenschaft und Philosophie studiert. Er war von
2007--2009 wissenschaftlicher Mitarbeiter im Forschungsprojekt
``Geschichte der technischen Bildübertragung (1843--1923)'' an der
Universität Konstanz und assoziiertes Mitglied des Zukunftskollegs. Von
01--07/2008 war er zudem Stipendiat am ``Center for Knowledge
Architecture'' an der TU Dresden. Von 2009--2013 koordinierte er das von
der DFG geförderte Projekt ``Digitalisierung des Polytechnischen
Journals'' (\url{http://www.polytechnischesjournal.de}). Von 2013--2015
war er Koordinator des weiterbildenden Masterstudiengangs
Psychoanalytische Kulturwissenschaft an der Humboldt-Universität zu
Berlin und ist seither wissenschaftlicher Mitarbeiter im aus Mitteln der
Exzellenz-Initiative geförderten Projekt ``Hidden Kosmos: Reconstructing
A. v. Humboldt's ›Kosmos-Lectures‹''
(\url{http://www.culture.hu-berlin.de/hidden-kosmos}).

\textbf{Benjamin Fiechter} studiert seit 2011 an der
Humboldt-Universität zu Berlin Deutsche Literatur und Geschichte, seit
2015 im Masterstudiengang Deutsche Literatur. Er war studentische
Hilfskraft im DFG-geförderten Projekt ``Deutsches Textarchiv'' (DTA,
\url{http://www.deutschestextarchiv.de}). Seit 2014 arbeitet er im
Projekt ``Hidden Kosmos: Reconstructing A. v.
Humboldt's''Kosmos-Lectures"
(\url{http://www.culture.hu-berlin.de/hidden-kosmos}), gefördert aus
Mitteln der Exzellenz-Initiative, als studentische Hilfskraft. Außerdem
ist er aktiver Beiträger der freien Quellensammlung ``Wikisource''
(Wikimedia Foundation, \url{https://de.wikisource.org/}).

\textbf{Christian Thomas} hat Neuere deutsche Literatur und Philosophie
an der Humboldt-Universität zu Berlin studiert und ist seit 2010
wissenschaftlicher Mitarbeiter im DFG-geförderten Projekt ``Deutsches
Textarchiv'' (DTA, www.deutschestextarchiv.de) und im Projekt
``CLARIN-D'' (\url{http://www.clarin-d.de}) an der
Berlin-Brandenburgischen Akademie der Wissenschaften. Im Rahmen von
CLARIN-D koordinierte er ein Kurationsprojekt zur Aufwertung und
Integration historischer Textressourcen des 15.--19. Jahrhunderts in die
Korpora des DTA bzw. von CLARIN-D
(\url{http://www.deutschestextarchiv.de/clarin_kupro}). Seit Juni 2014
koordiniert er zudem als wissenschaftlicher Mitarbeiter der
Humboldt-Universität zu Berlin das aus Mitteln der Exzellenz-Initiative
geförderte Projekt ``Hidden Kosmos: Reconstructing A. v.
Humboldt's''Kosmos-Lectures"
(http://www.culture.hu-berlin.de/hidden-kosmos). Parallel dazu arbeitet
er an einer Promotion über die bislang größtenteils unveröffentlichten
Nachschriften der ``Kosmos-Vorlesungen'' Alexander von Humboldts.
